	\documentclass[twoside, 12pt]{book} %openany if I don't want blank even pages before chapter starts
\usepackage{estilo-ejercicios}
\setcounter{section}{0}
\usepackage{subfiles}
\usepackage{empheq}
\usepackage{adjustbox}
\usepackage[UKenglish]{datetime}
%\usepackage{fancyhdr}
\usepackage{setspace}
%\usepackage[
%    type={CC},
%    modifier={by-nc-sa},
%    version={3.0},
%]{doclicense}
%\doublespacing
\usepackage{geometry}
 \geometry{
a4paper,
left = 40mm,
right = 40mm 
 }
 

\newcommand*\widefbox[1]{\fbox{\hspace{2em}#1\hspace{2em}}}

\renewcommand{\baselinestretch}{1,3}

\newdateformat{monthyear}{\monthname[\THEMONTH] \THEYEAR} %https://mirror.apps.cam.ac.uk/pub/tex-archive/obsolete/macros/latex/contrib/datetime/datetime.pdf

%Below to introduce ¡ in mathmode https://tex.stackexchange.com/questions/471464/inverted-exclamation-mark-in-mathmode
\DeclareMathSymbol{\mathinvertedexclamationmark}{\mathclose}{operators}{'074}
\DeclareMathSymbol{\mathexclamationmark}{\mathclose}{operators}{'041}

\makeatletter
\newcommand{\raisedmathinvertedexclamationmark}{%
  \mathclose{\mathpalette\raised@mathinvertedexclamationmark\relax}%
}
\newcommand{\raised@mathinvertedexclamationmark}[2]{%
  \raisebox{\depth}{$\m@th#1\mathinvertedexclamationmark$}%
}
\begingroup\lccode`~=`! \lowercase{\endgroup
  \def~}{\@ifnextchar`{\raisedmathinvertedexclamationmark\@gobble}{\mathexclamationmark}}
\mathcode`!="8000
\makeatother


%--------------------------------------------------------
\begin{document}
\title{\bf{The Derived Deligne Conjecture}\\ \vspace{1.5cm}}
%\title{Derived $A_\infty$-structures on operads}



\author{Doctoral thesis by\\ Javier Aguilar Martín\\ [\baselineskip]\\ Supervised by\\ Constanze Roitzheim\\ \vspace{1.5cm}} %baselineskip https://latex.org/forum/viewtopic.php?t=9496



\date{\emph{A thesis submitted to fulfil the requirements for the degree of Doctor of Philosophy in Mathematics}\\ [2\baselineskip] School of Mathematics, Statistics and Actuarial Sciences\\ University of Kent\\ [\baselineskip] \monthyear\today}


\maketitle
%\doclicenseThis
 %https://osl.ugr.es/CTAN/macros/latex/contrib/doclicense/doclicense.pdf
 %more options
%\pagebreak
%TO ASK: INTRO (THEIR PAPERSS DON'T STATE THE THEOREMS AND I DO MENTION THEM, WHY I NEED TO DO THAT AND WHAT NUMBERING SHOULD I USE?) 
%TO COMMENT: NOTATION CHANGES (WOULD IT BE BETTER TO USE DJ FOR TOTAL DEGREE EVEN IF THERE ARE DIFFERNTIALS AROUND? MAYBE KEEEP N IN THE BACKGROUND BECAUSE THERE ARE DIFFERENTIALS), ANOTHER POSSIBILITY: USE J,L FOR ARITIES (LIKE FOR MIJ) AND THUS V,W FOR INDICES
\doublespacing
\sloppy

\pagestyle{empty}  % No headers or footers for the following pages

\null\vfill
% Now comes the "Funny Quote", written in italics
\textit{But mathematics is the sister, as well as the servant, of the arts and is touched by the same madness and genius.}

\begin{flushright}
Marston Morse quoted in S. Gudder's \emph{A Mathematical Journey}
\end{flushright}
%https://mathshistory.st-andrews.ac.uk/Biographies/Morse/quotations/#:~:text=Mathematics%20are%20the%20result%20of,pulls%20it%20down%20to%20earth.

\null\vfill
% Now comes the "Funny Quote", written in italics
\textit{The real voyage of discovery consists not in seeking out new landscapes but in having new eyes.}

\begin{flushright}
Marcel Proust in \emph{La Prisionnière}
\end{flushright}
\null\vfill
% Now comes the "Funny Quote", written in italics
\textit{Folk in those stories had lots of chances of turning back only they didn't.
Because they were holding on to something.}

\begin{flushright}
Samwise Gamgee in the film \emph{The Lord of The Rings: The Two Towers}
\end{flushright}


\vfill\vfill\vfill\vfill\vfill\vfill\null
\clearpage 

\chapter*{Abstract}


%SINCE I'M NOT IMPOSING O(0)=0 (I NEED B0 IN THE BRACES AND THE END OPERAD IS USUALLY NON 0 THERE), I PROBABLY NEED TO SPECIFY THAT THE ARITY 0 COMPONENT OF M IS 0 (UNLESS I CONSIDER UNITAL CASE SPEFICALLY), THIS CAN BE DONE BY USING SOME  NOTATION OF THE POSITIVE ARITY PART



%
%We study the brace structure on an operad of graded $R$-modules using a construction called \emph{operadic suspension} and use it to define $A_\infty$-algebra structures on certain operads. This construction provides us an operadic context from which $A_\infty$-algebras arise in a natural way. We focus on the relation between the brace structure on an operad and its operadic composition, with the endomorphism operad as our main example. This relation allows us to generalize the Lie bracket defined in \cite{RW}. Then we prove some results about $A_\infty$-algebras on operads that were originally claimed by Gerstenhaber and Voronov \cite{GV}. Next, we generalize these constructions to operads of bigraded $R$-modules, introducing a totalizaion functor, which allows to generalize the Lie bracket defined in \cite{LRW}. This construction and the use of some enriched categories allows us to obtain for derived $A_\infty$-algebras some results that are analogous to the ones we obtained for $A_\infty$-algebras.


We study the operad of derived $A_\infty$-algebras from a new point of view in order to find a derived version of the Deligne conjecture. We start by defining the brace structure on an operad of graded $R$-modules using operadic suspension, which we describe in depth for the first time as a functor, and use it to define $A_\infty$-algebra structures on certain operads, with the endomorphism operad as our main example. This construction provides us with an operadic context from which $A_\infty$-algebras arise in a natural way and allows us to generalize the Lie algebra structure on the Hochschild complex of an $A_\infty$-algebra. Next, we generalize these constructions to operads of bigraded $R$-modules, introducing a totalization functor. This allows us to generalize a Lie algebra structure on the total complex of a derived $A_\infty$-algebra. This construction and the use of some enriched categories allow us to obtain new versions of the Deligne conjecture.

\section*{Acknowledgements}

First and foremost I would like to thank my supervisor Constanze Roitzheim. Without her, this thesis would not have been possible. I wish every PhD student could have such a knowledgeable, supportive and caring supervisor. Many thanks as well to my external supervisor, Sarah  Whitehouse, who took the hard task of substituting Constanze for a while and was of immense help, not only as a supervisor, but also as a researcher. To Claire Carter, who has been able to help me in some of the most difficult situations, I can only say God bless your heart. 

I would also like to thank the University of Kent for making my university experience so enjoyable. In particular the Music Department, the Rutherford dining hall staff, the Mindfulness Society, the Salsa Society and the Tennis Society for helping me staying mentally and physically healthy during this tough years. In addition, I want to thank King's College London for the teaching opportunities that they offered me, this has contributed significantly to my development. Similarly, I am thankful to the staff from NTT Data Spain for giving me the opportunity to explore my potential career after the PhD and understanding my need to focus on my thesis. Among them I want to thank Diego in particular, who also happens to be one of my maths buddies together with Rafa.

I also owe a lot to people outside the university. Most of all my parents, I cannot thank you as much as you deserve. And of course the rest of my family. My friends who stayed there supporting me from the distance, in particular May, Inma and Diana, thank you so much. Thanks to someone who was a motivation while she was there, and made me focus on my work when she left. Thanks to anyone who I may have forgotten for any reason. 

And last but certainly not least, I would like to thank myself for having the courage to move to a different country to do a PhD and being able to cope with challenges such as the Brexit, the Covid-19 pandemic and other personal circumstances. Thank you for taking care of me and for putting the effort to complete these piece of work. I am proud of you.

\tableofcontents
%\pagestyle{headings}
%\pagenumbering{gobble}
%\pagestyle{fancy}
%\sloppy
\pagestyle{plain}
\chapter{Introduction}

There are a number of mathematical fields in which $A_\infty$-structures arise, ranging from topology to mathematical physics. To study these structures, different interpretations of $A_\infty$-algebras have been given. From the original definition in 1963 \cite{STASHEFF}, to alternative definitions in terms of tensor coalgebras \cite{keller}, \cite{penkava}, many approaches use the machinery of operads \cite{LRW}, \cite{lodayvallette} or certain Lie brackets \cite{RW} to obtain these objects. 

Another technique to describe $A_\infty$-structures comes from brace algebras \cite{GV},\cite{lada}, which often involves unwieldy sign calculations that are difficult to describe in a conceptual way.

Here we used an operadic approach to obtain these signs in a more conceptual and consistent way. As a consequence, we will generalize the Lie bracket used in \cite{RW} and will give a very simple interpretation of $A_\infty$-algebras. The difference between our operadic approach and others mentioned before is that ours uses much more elementary tools and can be use to talk about $A_\infty$-structures on any operad. We hope that this provides a useful way of thinking about $A_\infty$-structures. A first application of this simple formulation is the generalization of the Deligne conjecture. The classical Deligne conjecture states that the Hochschild complex of an associative algebra has the structure of a homotopy $G$-algebra \cite{GV}. This result has its roots in the theory of topological operads \cite{delignehistory}. Since $A_\infty$-generalize associative algebras, it is natural to ask what sort of algebraic structure arises on their Hochschild complex. Thanks to the tools we develop, we are able to answer this question.

Later in 2009, derived $A_\infty$-algebras were introduced by Savage \cite{sagave} as a bigraded generalization of $A_\infty$-algebras in order to bypass the projectivity requirements that are often imposed when working with classical $A_\infty$-algebras. We generalize the operadic description of classical $A_\infty$-algebras to the derived case by means of an operadic totalization inspired by the totalization functor described in \cite{whitehouse}. This way we obtain an operation similar to the star operation in \cite{LRW} and generalize the construction that has been done for $A_\infty$-algebras to more general derived $A_\infty$-algebras. This allows us to generalized the Deligne conjecture even further to obtain a \emph{derived} Deligne conjecture.

The text is organized as follows. In \Cref{Sec1} we recall some basic definitions and results, and establish some conventions for both the classical and the derived cases. In \Cref{Sec2} we define a device called \emph{operadic suspension} that will help us obtain the signs that we want and link this device to the classical operadic approach to $A_\infty$-algebras. We also take this construction to the level of the underlying collections of the operads to also obtain a nice description of $\infty$-morphisms of $A_\infty$-algebras. We then explore the functorial properties of operadic suspension, being monoidality (\Cref{monoidality}) the most remarkable of them. In \Cref{sectionbraces} we study the brace algebra induced by operadic suspension and obtain a relevant result, \Cref{bracesign}, which establishes a relation between the canonical brace structure on an operad and the one induced by its operadic suspension. We show that as a particular case of this result we obtain the Lie bracket from \cite{RW}.

Following the terminology of \cite{GV}, if $\OO$ is an operad with an $A_\infty$-multiplication $m\in\OO$, it is natural to ask whether there are linear maps $M_j:\OO^{\otimes j}\to \OO$ satisfying the $A_\infty$-algebra axioms. In \Cref{sect2} we use the aforementioned brace structure to define such linear maps on a shifted version of the operadic suspension. We then iterate this process in Section \ref{sect3} to define an $A_\infty$-structure on the Hochschild complex of an operad with $A_\infty$-multiplication. This iteration process was inspired by the work of Getzler in \cite{getzler}.

Next, we prove our first main result, \Cref{theorem}, which relates the $A_\infty$-structure on an operad with the one induced on its Hochschild complex. More precisely, we have the following.

\begin{manualtheorem}{A}
There is a morphism of $A_\infty$-algebras $\Phi:S\s\OO\to S\s\End_{S\s\OO}$.
\end{manualtheorem} 
This result was hinted at by Gerstenhaber and Voronov in \cite{GV}, but here we introduce a suitable context and prove it as \Cref{theorem}. We also draw a connection between our framework and the one from Gerstenhaber and Voronov. As a consequence of this theorem, if $A$ is an $A_\infty$-algebra and $\OO=\End_A$ its endomorphism operad, we obtain the following $A_\infty$-version of the Deligne conjecture in \Cref{ainftydeligne}. 

\begin{manualtheorem}{B}
The Hochschild complex $S\s\End_{S\s\OO}$ of an operad with an $A_\infty$-multiplication has a structure of $J$-algebra.
\end{manualtheorem} 

 In the above theorem, $J$-algebras play the role of homotopy $G$-algebras in the classical case \cite{GV}. After this, we move to the bigraded case. The goal here is showing that an operad $\OO$ with a derived $A_\infty$-multiplication $m\in\OO$ can be endowed with the structure of a derived $A_\infty$-algebra, just like in the classical case.  We start \Cref{deriveddef} recalling some definitions of derived $A_\infty$-algebras and filtered $A_\infty$-algebras. In \Cref{operadic}, we define the totalization functor for operads and then the bigraded version of operadic suspension. We combine these two constructions to define an operation that allows us to understand a derived $A_\infty$-multiplication as a Maurer-Cartan element. As a consequence we obtain the star operation that was introduced in \cite{LRW}, which also defines a Lie Bracket.  From this, we obtain in \Cref{sectionbibraces} a brace structure from which we can obtain a classical $A_\infty$-algebra on the graded operad $S\Tot(\s\OO)$. Finally, in \Cref{derivedstructure}, we prove our main results about derived $A_\infty$-algebras. The first one is \Cref{derivedmaps}, which shows that, under mild boundedness assumptions, the $A_\infty$-structure on totalization is equivalent to a derived $A_\infty$-algebra on $S\s\OO$. The statement can be summarised as follows.\pagebreak
 
 \begin{manualtheorem}{C}
	For any sufficiently bounded operad $\OO$ with a derived $A_\infty$-multiplication there are linear maps $M_{ij}:(S\s\OO)^{\otimes j}\to S\s\OO$ satisfying the derived $A_\infty$-algebra axioms.
 \end{manualtheorem}
 The next result is \Cref{bigradedtheorem}, which generalizes \Cref{theorem} to the derived setting. More precisely,
 \begin{manualtheorem}{D}
There is a morphism $\Phi:S\s\OO\to S\s\End_{S\s\OO}$ of derived $A_\infty$-algebras.
\end{manualtheorem}
As a consequence of this theorem we obtain a new version of the Deligne conjecture, \Cref{dainftydeligne}, this time in the setting of derived $A_\infty$-algebras. For this we also introduce a derived version of $J$-algebras.
\begin{manualtheorem}{E}
The Hochschild complex $S\s\End_{S\s\OO}$ of an operad with a derived $A_\infty$-multiplication has a structure of derived $J$-algebra.
\end{manualtheorem} 

We finish the thesis in \Cref{future} by outlining some open question that arise from our research. The first question is related to the boundedness assumptions that we need to make in order to obtain the derived Deligne conjecture. The other one would be the natural continuation of our research. In the classical case, the homotopy $G$-algebra structure on the Hochschild complex induced a Gerstenhaber algebra structure on cohomology \cite{GV}. We would like to know what structure there is on the Hochschild cohomology of a derived $A_\infty$-algebra.



%DO I NEED TO MENTION THE APPENDICES?


%STYLE, SHOULD I USE THEOREM STYLE FOR DEFINITIONS AND EVERYTHING ELSE AS WELL? IN THAT CASE HOW SHOULD I HIGHLIGH THE NAMES OF THE THINGS I DEFINE?
%
%I FEEL THAT THE BACKGROUND IS NOT WELL BALANCED, IN THE CLASSICAL CASE I INCLUDE THE DEF OF AINTY ALGEBRA AND I MENTION THE OPERAD ONLLY VERY BRIEFLY, WHILE FOR THE DERIVED CASE I HAVE A SEPARATE SECTION AFTERWARDS WITH MORE DETAIL, MAYBE ITS OKAY BECAUSE AINFTY SHOULD BE BETTER KNOWN


%\subfile{background}
%\subfile{SubSuspensionMultiplication}\\
\subfile{ThesisIntroduction}
\subfile{ThesisSuspensionMultiplication}
%\subfile{SubTotalization}
%\subfile{SubTotalizationx}
\subfile{ThesisTotalization}
\subfile{ThesisFuture}
\subfile{Thesis-appendices}


\addcontentsline{toc}{chapter}{Bibliography}
\bibliographystyle{alpha}
\bibliography{newbibliography}

\addcontentsline{toc}{chapter}{Glossary}
\chapter*{Glossary}
%https://www.baeldung.com/cs/latex-glossary
%\begin{tabular}{ll}
\begin{longtable}{ll}
$R$ & A ring of non-zero characteristic\\
%$A$ & A (bi)graded $R$-module\\
$\Hom_R(A,B)$ & The (bi)graded module of linear maps\\
$\deg(x)$ & Degree of an element $x$ in a graded module\\
$\Z$ & The group of integers\\
$S$ & Shift map for single-graded modules,\\
 & vertical shift for bigraded modules\\
%SYMMETRIC GROUP & I DON'T USE IT MUCH \\
%$\Sigma_n$ & Symmetric group on $n$ elements\\
%$sig_n$ & Sign representation of $\Sigma_n$ \\
$\OO$ & A linear (bi)graded operad\\
$\otimes$ & Tensor product of (bi)graded $R$-modules,\\
& also Hadamard product of operads\\
$\End_A$ & Endomorphism operad\\
$\End_B^A$ & Collection $\{\Hom_R(A^{\otimes n},B)\}_{n\geq 1}$\\
$\calA_\infty$ & Operad of $A_\infty$-algebras\\
$d\calA_\infty$ & Operad of derived $A_\infty$-algebras\\
$m=m_1+m_2+\cdots$ & $A_\infty$-structure maps on an $R$-module,\\
& or $A_\infty$-multiplication in an operad\\
$M=M_1+M_2+\cdots$ & $A_\infty$-structure maps on $S\s\OO$\\
$\overline{M} = \overline{M}_1+\overline{M}_2+\cdots$ & $A_\infty$-structure maps on $S\s\End_{S\s\OO}$\\
$m=\sum_{ij}m_{ij}$ & Derived $A_\infty$-structure maps,\\
& or derived $A_\infty$-multiplication in $\OO$\\
$M=\sum_{ij}M_{ij}$ & derived $A_\infty$-structure maps on $S\s\OO$\\
$\overline{M} = \sum_{ij}\overline{M}_{ij}$ & derived $A_\infty$-structure maps on $S\s\End_{S\s\OO}$\\
$\gamma$ & Operadic composition\\
& sometimes used as an exponent in signs\\
$\circ_i$ & Operadic insertion\\
$\circ$ & Plethysm of operads, \\
& also circle operation \\
& and composition of maps\\
%$\bar{\circ}$ & Plethysm of cooperads\\
$\Lambda$ & Operad structure on the shifts of $R$ \\
$\s\OO=\OO\otimes\Lambda$ & Operadic suspension of $\OO$\\
& also its underlying (bi)graded module\\
$|x|$ & Natural degree of $x$ in $\s\OO$ if single-graded\\
& also total degree of $x$ in a bigraded module\\
%$||x||=|x|+1$ & Degree of $x$ in $S\s\OO$ if $\OO$ is single-graded\\
$\mathrm{vdeg}(x)$ & Vertical degree of a bigraded element $x$\\
$\langle,\rangle$ & Dot product of bidegrees\\
$\tilde{\gamma}$ & Operadic composition on $\s\OO$\\
$\tilde{\circ}_i$ & Operadic insertion on $\s\OO$\\
$\tilde{\circ}$ & Circle operation on $\s\OO$\\
$[,]$ & Lie bracket\\
 & also internal hom\\
$b_n$ & Brace map on $\OO$ or $\s\OO$\\
$B_n$ & Brace map on $\End_{\s\OO}$\\
 
$\overline{B}_n$ & Brace map on $\s\End_{S\s\OO}$\\

$\overline{\sigma}$ & Isomorphism $\End_A\cong \s\End_{SA}$ \\

$\Phi: S\s\OO\to S\s\End_{S\s\OO}$ & Morphism of (derived) $A_\infty$-algebras \\

$\CC = (\CC,\otimes_\CC, 1_\CC)$ & A (closed) (symmetric) monoidal category\\

$\Hom_\CC(A,B)$ & Set of morphism from $A$ to $B$\\

$\CC^b$ & Category horizontally bounded on the right \\

$\mathrm{C}_R$ & Category of cochain complexes \\


$\mathrm{fC}_R$ & Category of filtered complexes  \\

$\underline{\Hom}(A,B)$ & Filtered hom complex\\

$\mathrm{bgMod}_R$ & Category of bigraded modules \\

$\mathrm{vbC}_R$ & Category of vertical bicomplexes \\ 



$\mathrm{tC}_R$ & Twisted complexes \\

$\uC$ & Enriched category\\

$\underline{\otimes}$ & Enriched tensor product\\

$\uEnd_A$ & Enriched endomorphism operad\\

$\underline{\mathpzc{bgMod}_R}$ & $\mathrm{bgMod}_R$-enriched category \\
& of bigraded modules \\

$\underline{t\mathcal{C}_R}$& $\mathrm{vbC}_R$-enriched category \\
& of twisted complexes\\


$\ufMod$ & $\mathrm{bgMod}_R$-enriched category \\
& of filtered modules \\

$\ufC$ & $\mathrm{vbC}_R$-enriched category \\
& of filtered complexes \\

%BOUNDED VERSIONS? & I ONLY USE THE NOTATION LIKE ONCE, MAYBE GENERIC $C^b$ \\

$\Tot$ & Totalization functor\\

$\mathfrak{Tot}$ & Enriched totalization functor\\

$\bar{\circ}_i$ & Operadic insertion on $\Tot(\OO)$\\

$\star_i$ & Operadic insertion on $\Tot(\s\OO)$\\

$\bar{\gamma}$ & Operadic composition on $\Tot(\OO)$\\

$\gamma^\star$ & Operadic composition on $\Tot(\s\OO)$\\

$b^\star_n$ & Brace map on $\Tot(\s\OO)$\\


$\mu=\mu_{A,B}$ & The map $\Tot(A)\otimes \Tot(B)\to\Tot(A\otimes B)$\\
%I AM OMITTING & AD HOC DEFINED SYMBOLS
%\end{tabular}
\end{longtable}

\end{document}