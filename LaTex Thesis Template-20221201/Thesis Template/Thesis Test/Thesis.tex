%% ----------------------------------------------------------------
%% Thesis.tex -- MAIN FILE (the one that you compile with LaTeX)
%% ---------------------------------------------------------------- 

% Set up the document
\documentclass[a4paper, 12pt, oneside]{Thesis}  % Use the "Thesis" style, based on the ECS Thesis style by Steve Gunn
\graphicspath{Figures/}  % Location of the graphics files (set up for graphics to be in PDF format)
% Include any extra LaTeX packages required

\usepackage{cleveref}
\usepackage[square, numbers, comma, sort&compress]{natbib}  % Use the "Natbib" style for the references in the Bibliography
\usepackage{datetime}
\newdateformat{monthyeardate}{%
  \monthname[\THEMONTH] \THEYEAR.}

\hypersetup{urlcolor=blue, colorlinks=true}  % Colours hyperlinks in blue, but this can be distracting if there are many links.

%% ----------------------------------------------------------------
\begin{document}
\frontmatter      % Begin Roman style (i, ii, iii, iv...) page numbering

% Set up the Title Page
\title  {Title}
\authors  {
{Javier Aguilar Martín}
            }
\addresses  {\groupname\\\deptname\\\univname}  % Do not change this here, instead these must be set in the "Thesis.cls" file, please look through it instead
\date       {\today}
\subject    {}
\keywords   {}

\maketitle
%% ----------------------------------------------------------------

\setstretch{1.5}  % It is better to have smaller font and larger line spacing than the other way round

% Define the page headers using the FancyHdr package and set up for one-sided printing
\fancyhead{}  % Clears all page headers and footers
\rhead{\thepage}  % Sets the right side header to show the page number
\lhead{}  % Clears the left side page header

\pagestyle{fancy}  % Finally, use the "fancy" page style to implement the FancyHdr headers

%% ----------------------------------------------------------------
% Declaration Page required for the Thesis, your institution may give you a different text to place here
%\Declaration{

%\addtocontents{toc}{\vspace{1em}}  % Add a gap in the Contents, for aesthetics

%Decleration

%\begin{itemize} 
%\item[\tiny{$\blacksquare$}] Item one
%\end{itemize}
 
 
%Signed:\\
%\rule[1em]{25em}{0.5pt}  % This prints a line for the signature
 
%Date:\\
%\rule[1em]{25em}{0.5pt}  % This prints a line to write the date
%}
%\clearpage  % Declaration ended, now start a new page

%% ----------------------------------------------------------------
% The "Funny Quote Page"
\pagestyle{empty}  % No headers or footers for the following pages

\null\vfill
% Now comes the "Funny Quote", written in italics
\textit{quote}

\begin{flushright}
name
\end{flushright}

\vfill\vfill\vfill\vfill\vfill\vfill\null
\clearpage  % Funny Quote page ended, start a new page
%% ----------------------------------------------------------------

% The Abstract Page
\addtotoc{Abstract}  % Add the "Abstract" page entry to the Contents
\abstract{
\addtocontents{toc}{\vspace{1em}}  % Add a gap in the Contents, for aesthetics
INCLUDE DELIGNE

We study the operad of derived $A_\infty$-algebras from a new point of view. We start by defining the brace structure on an operad of graded $R$-modules using operadic suspension, which we describe in depth for the first time as a functor, and use it to define $A_\infty$-algebra structures on certain operads, with the endomorphism operad as our main example. This construction provides us with an operadic context from which $A_\infty$-algebras arise in a natural way and allows us to generalize the Lie algebra structure on the Hochschild complex of an $A_\infty$-algebra. Next, we generalize these constructions to operads of bigraded $R$-modules, introducing a totalization functor. This allows us to generalize a Lie algebra structure on the total complex of a derived $A_\infty$-algebra. This construction and the use of some enriched categories allow us to obtain some new results about derived $A_\infty$-algebras that generalize those obtained for $A_\infty$-algebras.

}

\clearpage  % Abstract ended, start a new page
%%-------------------------------------------------------
%\addtotoc{Acknowledgements}  % Add the "Abstract" page entry to the Contents
\acknowledgements{
\addtocontents{toc}{\vspace{1em}}  % Add a gap in the Contents, for aesthetics

acknowledgements

\begin{flushright}
Javier Aguilar Martin, \monthyeardate\today
\end{flushright}

}

\clearpage  % Abstract ended, start a new page
\setstretch{1.5}  % Return the line spacing back to 1.3

%% ----------------------------------------------------------------
\mainmatter	  % Begin normal, numeric (1,2,3...) page numbering
\pagestyle{fancy}  % Return the page headers back to the "fancy" style

% Include the chapters of the thesis, as separate files
% Just uncomment the lines as you write the chapters

\tableofcontents

\doublespacing
\input{Chapters/Chapter1} 

%\input{Chapters/Chapter2} 

%\input{Chapters/Chapter3} 

%\input{Chapters/Chapter4} 

%\input{Chapters/Chapter5} 

%\input{Chapters/Chapter6} 

%\input{Chapters/Chapter7} 

%% ----------------------------------------------------------------
% Now begin the Appendices, including them as separate files

\addtocontents{toc}{\vspace{2em}} % Add a gap in the Contents, for aesthetics

\appendix % Cue to tell LaTeX that the following 'chapters' are Appendices

%\input{Appendices/AppendixA}	% Appendix Title

%\input{Appendices/AppendixB} % Appendix Title

%\input{Appendices/AppendixC} % Appendix Title

\addtocontents{toc}{\vspace{2em}}  % Add a gap in the Contents, for aesthetics
\backmatter

%% ----------------------------------------------------------------
\label{Bibliography}
\lhead{\emph{Bibliography}}  % Change the left side page header to "Bibliography"
\bibliographystyle{abbrv}  % Use the "unsrtnat" BibTeX style for formatting the Bibliography
\bibliography{Bibliography}  % The references (bibliography) information are stored in the file named "Bibliography.bib"

\end{document}  % The End
%% ----------------------------------------------------------------