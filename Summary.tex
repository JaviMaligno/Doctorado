	\documentclass[twoside]{article}
\usepackage{estilo-ejercicios}
\setcounter{section}{0}
\newtheorem{defin}{Definition}[section]
\newtheorem{lem}[defin]{Lemma}
\newtheorem{propo}[defin]{Proposition}
\newtheorem{thm}[defin]{Theorem}
\newtheorem{eje}[defin]{Example}
\newtheorem{obs}[defin]{Observación}
\renewcommand{\baselinestretch}{1,3}

\usepackage{empheq}
\newcommand*\widefbox[1]{\fbox{\hspace{2em}#1\hspace{2em}}}
%--------------------------------------------------------
\begin{document}

\title{Sign conventions}
\author{Javier Aguilar Martín}
\maketitle

\section{Things valid for any sign convention independent of any sign in the product}


All the equalities about signs are modulo 2.

Let $|x|$ denote the degree of $x$ that is gonna be used on the sign ($|x|=a(x)+q(x)-1$ so far) and we write $|x||y|$ to mean a product which is not necessarily the product of integers but maybe some kind of dot product as in R-W. Before, I studied the brace relation in the case of $m\{m\}\{x,y,z\}$ with the particular brace of the endomorphism operad. Let's now look at this without having any particular brace in mind. We have

$$0=m\{m\}\{x,y,z\}=m\{m\{x,y\},z\}+(-1)^{|x||m|}m\{x,m\{y,z\}\}$$

We want to define a product $x\cdot y=(-1)^{\varepsilon(x,y)}m\{x,y\}$ for which the brace relation above gives us associativity. I will also write $\varepsilon(xy)$ when there is no confusion. Associativity means

$$(-1)^{\varepsilon(x,y)+\varepsilon(xy,z)}m\{m\{x,y\},z\}=(-1)^{\varepsilon(x,yz)+\varepsilon(y,z)}m\{x,m\{y,z\}\}$$

By the brace relation this implies $\mod 2$ 

\begin{equation}\label{ass-sing}%\tag{5}
\varepsilon(x,y)+\varepsilon(xy,z)=\varepsilon(x,yz)+\varepsilon(y,z)+|x||m|+1
\end{equation}
%
%%Putting $y=0$ this means that $|x||m|=1\mod 2$. This can be achieve precisely with a dot product like that in R-W, i.e., for $|x|=a(x)+q(x)-1$ and $|y|=a(y)+q(y)-1$ we define $|x||y|=a(x)a(y)+q(x)q(y)+1$. In particular, $|x||m|=2a(x)+0q(x)+1=1$. 
%%
%%I NOT REALLY SURE HOW THIS FOLLOWS. IF $Y=0$ THEN THE EQUATION BECOMES $0=0$ AND WE CAN PUT WHATEVER SIGNS WE WANT. 
%
Being $|x||m|=1$, we can simply choose $\varepsilon(x,y)=0$ for all $x,y$ and the brace equation implies associativity. We will test this sign further a bit later. 







Let us define $d(x)=[m,x]=m\circ x-(-1)^{|x||m|}x\circ m$ and let's see what we need for this to be a diferential, i.e. $d^2=0$. 

\begin{align*}
&d(d(x))=d(m\circ x-(-1)^{|x||m|}x\circ m)=\\
&m\circ (m\circ x)-(-1)^{|x||m|}m\circ(x\circ m)-(-1)^{|m||m\circ x|}(m\circ x)\circ m +(-1)^{|x||m|+|m||m\circ x|}(x\circ m)\circ m
\end{align*}

Using the brace relation and $m\circ m=0$:

\begin{align*}
(m\circ x)\circ m= m\{x,m\}+m\circ(x \circ m)+(-1)^{|m||x|}m\{m,x\}\\
(x\circ m)\circ m=x\{m,m\}+x\circ (m\circ m)-x\{m,m\}=0
\end{align*}
In the last equation we have to assume $|m||m|= 1$, with both a usual product or a dot product this implies $|m|=1$ since squaring is the identity modulo 2. Then $d(d(x))$ rewrites as
 \[
 m\circ (m\circ x)-(-1)^{|x||m|}m\circ(x\circ m)-(-1)^{|m||m\circ x|}(m\{x,m\}+m\circ(x \circ m)+(-1)^{|m||x|}m\{m,x\})
 \]
 We need $m\circ(x\circ m)$ to cancel, so we impose $|m||x|= |m||m\circ x|+1$. This happens with my sign convention but not with a convention that implies $|m||x|=1$ for all $x$.
 
 Now we use
 
 $$0=m\{m\}\{x\}=m\{m,x\}+m\circ(m\circ x)+(-1)^{|m||x|}m\{x,m\}$$
 To get
 \[
-m\{m,x\}-(-1)^{|m||x|}m\{x,m\}-(-1)^{|m||m\circ x|}(m\{x,m\}+(-1)^{|m||x|}m\{m,x\}=0
 \]
 
 Again, we're assuming  $|m||x|\equiv |m||m\circ x|+1$. 
Before studying the associativity of the product, note that if we don't put any sign, we depend on $|m||x|=1$, which as we saw caused some problems. But in addition, we will see that the Leibniz rule can't be satisfied in the absence of any sign in front of $m$. First of all, it is natural to assume $|d(x)|=|x|+1$. In the Leibiz rule $d(xy)$ would yield a $-m\{m\{x\},y\}$ coming from the fact that $m\{m\{x,y\}\}=-m\{m\{x\},y\}-(-1)^{|m||x|}m\{x\{m\},y\}$ due to the brace relation. On the other hand, $d(x)y$ produces $m\{m\{x\},y\}$ with positive sign, so whatever sign we include in $xy=(-1)^{\varepsilon(xy)}m\{x,y\}$, it is necessary that $\varepsilon(xy)\equiv\varepsilon(d(x)y)-1$. There are a number of obvious ways to do this is by chosing $\varepsilon(xy)=|x|,|x|+1,|x|+|y|, |x|+|y|+1,\dots$ But using equation \ref{ass-sing} we can discard those like $|x|+|y|$ because we would find that $|m\{x,y\}|$ has to depend on $z$. 

%ABOUT THE WEIRD LEIBNIZ RULE: IN G-V PAGE 5 BEFORE COROLLARY 5 THERES AN EQUATION THAT BECOMES MY LEIBNIZ RULE WITH $X=M$. I MIGHT TRY TO PROVE IT WITH MY THINGS
%
%WITH $|X|-1$ THE LEIBNIZ RULE IS THE SAME, WRITE IT WITH $|X|$ BUT ALSO TRY TO DO IT WITH A MORE GENERAL SIGN TO SEE IF SOMETHING ELSE CAN BE TRIED

\paragraph{Summary of conclusions}

\begin{itemize}
\item Associativity sign equation: $$\varepsilon(x,y)+\varepsilon(xy,z)=\varepsilon(x,yz)+\varepsilon(y,z)+|x||m|+1$$

\item $|m||m|=1$, which under the usual product or a dot product implies $|m|=1$.
\item $|m||x|=|m||m\circ x|+1$.
\item If $|d(x)|=|x|+1$, $\varepsilon(d(x)y)=\varepsilon(xy)+1$.

\end{itemize}


\section{Case 1: $\varepsilon(xy)=|x|$}

%MAYBE IT IS BETTER TO CONSIDER $\varepsilon(xy)=|x|$ AND SEPARATE WHAT THE PRODUCT IS

So we will check $\varepsilon(xy)=|x|$ and $|x|+1$, being open to other alternatives.

 With both signs, from equation \ref{ass-sing} we conclude $|m\{x,y\}|=|m||x|+|y|+1$. Since we want this sign to particularise as the sign in R-W, I would expect $|m||x|=|x|$, which is a consequence of $|m|=1$, so $|m\{x,y\}|=|x|+|y|+1$, but I will use this as little as possible, since with different notions of $|x|$ and $|m||x|$ it could still coincide with the particular numbers in R-W.

So now let's move to the Leibniz rule. I'm going to check it first for $\varepsilon(xy)=|x|$ (it is analogous for $|x|-1$) and then I will try to do it for a general $\varepsilon(xy)$ satisfying the differential and associativity conditions.

\[
d(xy)=(-1)^{|x|}d(m\{x,y\})=(-1)^{|x|}(m\{m\{x,y\}\}-(-1)^{|m||m\{x,y\}|}m\{x,y\}\{m\})
\]
In the last term we use the breace relation to produce
\[
d(xy)=(-1)^{|x|}(m\{m\{x,y\}\}-(-1)^{|m||m\{x,y\}|}(m\{x,y\{m\}\}+(-1)^{|m||y|}m\{x\{m\},y\}))
\]

Also use the brace relation to rewrite $m\{m\{x,y\}\}$ and produce

\[
d(xy)=(-1)^{|x|}(-m\{m\{x\},y\}-(-1)^{|m||x|}m\{x,m\{y\}\}-(-1)^{|m||m\{x,y\}|}(m\{x,y\{m\}\}+(-1)^{|m||y|}m\{x\{m\},y\}))
\]

Now compute
\[
d(x)y=(-1)^{|x|-1}(m\{m\{x\},y\}-(-1)^{|m||x|}m\{x\{m\},y\})
\]
and
\[
xd(y)=(-1)^{|x|}(m\{x,m\{y\}\}-(-1)^{|m||y|}m\{x,y\{m\}\})
\]

Direct comparison of the terms using that $|m\{x,y\}|=|m||x|+|y|-1$ shows that we need $|m||m|=|m|=1$ to have $d(xy)=d(x)y+(-1)^{|x|-1}xd(y)$ (in particular we can't get rid of the minus sign there at least with any sign depending only on $x$, but this is a particular case of the derivation property of the bracket with respect to the product according to G-V page 5).

\paragraph{Summary of conclusions}
I think I didn't rely on $|x||y|$ to be any particular kind of product, so these conclusions should be valid for any sign on the product
\begin{itemize}
\item  $|m\{x,y\}|=|m||x|+|y|+1$.
\item $|m||m|=|m|=1$.
\item The Leibniz rule must be of the form: 
$$d(xy)=d(x)y+(-1)^{|x|-1}xd(y)$$
\item I implicitly used that the product of degrees was associative. With the usual product and the dot product that is true, and I would expect that to be true for any sign convention.
\end{itemize}

\section{Case 2: general $\varepsilon(xy)$}

There's not a lot to change if we use instead some other sign $\varepsilon(xy)$, so let's compare terms. 

\[
d(xy)=(-1)^{\varepsilon(xy)}(-m\{m\{x\},y\}-(-1)^{|m||x|}m\{x,m\{y\}\}-(-1)^{|m||m\{x,y\}|}(m\{x,y\{m\}\}+(-1)^{|m||y|}m\{x\{m\},y\}))
\]

Now compute
\[
d(x)y=(-1)^{\varepsilon(d(x)y)}(m\{m\{x\},y\}-(-1)^{|m||x|}m\{x\{m\},y\})
\]
and
\[
xd(y)=(-1)^{\varepsilon(xd(y))}(m\{x,m\{y\}\}-(-1)^{|m||y|}m\{x,y\{m\}\})
\]

Since $\varepsilon(d(x)y)=\varepsilon(xy)-1$, $m\{m\{x\},y\}$ has the same sign on both sides. For $m\{x\{m\},y\}$ we need

\[
\varepsilon(xy)+|m||x|=|m||y|+|m||m\{x,y\}|-1+\varepsilon(xy)
\] 

%I WILL NEED $|M|(|M||X|)=|X|$, AND WHAT I KNOW IS THAT $|M||M|=1$ SO IT WOULD BE HELPFUL TO BE ABLE TO DEFINE THIS PRODUCT IN AN ASSOCIATIVE WAY
%
%(i+j)((k+l)(m+n))=(i+j)(km+ln)=[(ikm+jln) or ikm+iln+jkm+jln]
%
%((i+j)(k+l))(m+n)=(ik+jl)(m+n)=[(ikm+jln) or ikm +ikn +jlm+jln]

%It can be defined in the first way so I can assume it is associative, but I have to say it explicitely


As we can see $\varepsilon(xy)$ cancels so we get a condition on the degrees, namely, $|m||m\{x,y\}|=|m||x|+|m||y|+1$. Taking into account that $|m||m|=1$, assuming distributiviy of the product, this rewrites nicely as 

$$|m\{x,y\}|=|m|+|x|+|y|.$$

Now for $m\{x,m\{y\}\}$ we have on one side $\varepsilon(xy)+|m||x|-1$ and on the other side $\varepsilon(xd(y))$. Finally for $m\{x,y\{m\}\}$ we have on the one side $\varepsilon(xy)+|m||m\{x,y\}|+1=\varepsilon(xy)+|m||x|+|m||y|$, and on the other side $\varepsilon(xd(y))+|m||y|+1$. One can see that both differences between the signs of each term are equal, and they're indeed the sign in the Leibniz rule. More precisely, $\varepsilon(xd(y))+\varepsilon(xy)+|m||x|+1$. Since we only want this sign to depend on $x$ and $m$, we need either $\varepsilon(xd(y))=\varepsilon(xy)$ or $\varepsilon(xd(y))=\varepsilon(xy)+1$. In the first case we get a similar shift in the sign as in the case $\varepsilon(xy)=|x|$, and in the second case we get the usual Leibniz rule. 


%so the difference is always $|m||x|-1$ (indeed $|x|$ cause $|m|=1$, whatever sign we use. Similarly, for $m\{x,y\{m\}\}$ we have $\varepsilon(xy)+|m||m\{x,y\}|-1$ on the one side and $\varepsilon(xy)+|m||y|-1$. Now the difference is $|m|(|x|+1)$, but using $|m|=1$ we have the same difference as before.

\paragraph{Summary of conclusions}

\begin{itemize}
\item Condition on the degrees: $$|m||m\{x,y\}|=|m||x|+|m||y|+1,$$
which rewrites as 
$$|m\{x,y\}|=|m|+|x|+|y|$$
under the assumption of distributy, which is true for the usual product and for the dot product, and I think we should use some product that satisfies this.

\item Condition to get the Leibniz rule:  $\varepsilon(xd(y))=\varepsilon(xy)$ to get $|m||x|+1$ in the sign, or $\varepsilon(xd(y))=\varepsilon(xy)+1$ to get $|m||x|$ in the sign.

\end{itemize}

\begin{remark}[On the odd Leibniz rule]
The Leibniz rule on G-V (and originally in Gerstenhaber algebras) is an ``odd'' Leibniz rule in the sense that the sign is not the one expected if we only think of permutations of the variables. More precisely, the equation is

$$[x,yz]=[x,y]z+(-1)^{|x|(|y|+1)}y[x,z].$$
IT MUST BE THE FOLLOWING, NOT THE PREVIOUS BUT THE PREVIOUS CAN BE OBTAIN IF I THINK THE VERTICAL LINES AS HAVING DEGREE 1

$$[x,yz]=[x,y]z+(-1)^{(|x|+1)|y|}y[x,z].$$

But the reason is just that the bracket has odd degree and $y$ is swapped by both $x$ and the bracket. It can be even more clear if we write $l(a,b)=[a,b]$ and $\mu(a,b)=ab$, and this will also allows to draw some conclusions. Thus, the lhs of the equation becomes $l(x,\mu(y,z))$ and on the rhs we have two terms: $\mu(l(x,y),z)$ and $\mu(y,l(x,z))$. Let us omit parentheses and commas to focus on the permutation of symbols. The first term is given by the permutation

\[
\begin{tikzcd}
l\arrow[dr] & x\arrow[dr] & \mu\arrow[dll] & y & z\\
\mu & l & x & y & z
\end{tikzcd}
\]

So the sign should be given by $|\mu|(|x|+|l|)$. Since $\mu$ is of even degree, this sign vanishes (for operads, we have to take $\deg+arity-1$ but for maps being applied to elements only $\deg+arity$.

The second term is given by the permutation
\[
\begin{tikzcd}
l \arrow[drr]& x\arrow[drr] & \mu\arrow[dll] & y\arrow[dll] & z\\
\mu & y & l & x & z
\end{tikzcd}
\]

So the sign is given by $(|l|+|x|)(|\mu|+|y|)\equiv (|x|+1)|y|$. 

If we applied this to $x=m$ (the element of the operad of total degree $2$ such that $m\circ m=0$) we should get

$$[m,yz]=[m,y]z+(-1)^{|y|}y[m,z]$$

which rewrites as

$$d(yz)=d(y)z+(-1)^{|y|}yd(z)$$

BUT I DID GET THE OTHER SIGN SO I HAVE TO REVISIT THAT. AN ALTERNATIVE COULD BE DEFININ $D=[,M]$. THIS WOULD INCLUDE THE SIGN IN FRONT OF THE FACTOR $M \circ X$, WHICH WOULD BE THE SAME AS IN THE ASSOCIATIVE PRODUCT DEFINED BY THE BRACE



-----------------------------------------------------------

AN INTERLUDE ABOUT OPERADIC SUSPENSION THAT I SHOULD WRITE ON THE OPERADIC SUSPENSION PART

Assuming $M_j\in \End_{\Sigma\mathfrak{s}\OO}$, I proved that I was able to define it so that $\deg(M)=2-j$ (and obviously the arity is $j$). So if I have to apply the Koszul rule here, the degree used is just $2-j$. If I get to define $M_j\in\mathfrak{s}\End_{\Sigma\mathfrak{s}\OO}$, then $M_j$ is actually $M_j\otimes e_J$ where $e_J=e_1\land\dots\land e_j$ has degree $j-1$. So 

$$M_j\otimes e_J(x_1,\dots, x_j)=(-1)^{(j-1)(|x_1|+\cdots+|x_j|)}M_j(x_1,\dots, x_j)\otimes e_J$$
being $|x|$ the total degree (the natural degree on $\Sigma\mathfrak{s}\OO$, recall that $M_j$ wa defined via composition on this odd operad). So passing by the $M_j$ component would yield a sign depending on its internal degree, i.e. $2-j$.

For instance, if I get to define $M_2$ such that $$0=M_2\tilde{\circ}M_2=M_2\tilde{\circ}_2 M_2+M_2\tilde{\circ}_1 M_2$$ in the suspension, evaluating at $(x,y,z)$ gives us on the first summand

$$(M_2\circ_2M_2)(x,y,z)=(M_2(1,M_2(1,1))\otimes (e_1\land e_2\land e_3))(x,y,z)=(-1)^{(|x|+|y|+|z|)(3-1)}M_2(x,M_2(y,z))$$

and on the second summand
$$(M_2\circ_1M_2)(x,y,z)=-(M_2(M_2(1,1),1)\otimes (e_1\land e_2\land e_3))(x,y,z)=-(-1)^{(|x|+|y|+|z|)(3-1)}M_2(x,M_2(y,z))$$

Adding the two equals zero so we get the associativity condition $M_2(x,M_2(y,z))=M_2(M_2(x,y),z)$. Note that here $x$ is beeing permuted with $M_2$ but no extra signs appears, which is equivalent to apply the Koszul rule with the internal degree of $M_2$, which is $0$, and is in fact what we have done in the evaluation.

\end{remark}
\subsection{Idea if my sign convention does not work}

The following dot product in the endomorphism case satisfies some of the properties needed above, but I would still have to check it if necessary: $((a(x)-1)+q(x))((a(y)-1)+q(y))=(a(x)-1)(a(y)-1)+q(x)q(y)$. A reason to do this is that the $-1$ comes from a shift map as well as the arity, while the internal degree is independent of that. Thus, it make sense to put arity $-1$ together and separate the internal degree.

%\section{Generalization to $A_\infty$-algebras}
%I'll use the same notation as in the Derived Deligne Conjecture.
%
%\subsection{Alternative simpler structure}
%I propose as an alternative to the $A_\infty$-structre in the Derived Deligne Conjecture (and also G-V), a simpler family of maps: $M_n(x_1,\dots, x_n)=m_n\{x_1,\dots, x_n\}=b_n(m_n;x_1,\dots, x_n)$. Which is just the last term of the sum taking place in the references.
%
%The proof of Lemma 3.2 in Derived Deligne Conjecture is correct (at least up to signs), so I'm going to prove it just for the simplification (up to signs, as everything in this subsection).
%
%\begin{lemma} The above choice turns $\mathcal{O}$ into an $A_\infty$-algebra.
%\end{lemma}
%IN THIS PROOF AND ITS EQUIVALENT IN THE GENERAL CASE WE SHOULD SEPARATE WHEN $M_1$ IS SOME OF THE MAPS BECAUSE IN THAT CASE IT IS DEFINED DIFFERENTLY
%\begin{proof}
%We need to show that $M\circ M=0$. 
%
%\[
%\underset{0\leq i\leq k}{\sum_{p,q,k=p+q-1}}M_p(x_1,\dots, x_i, M_q(x_{i+1},\dots, x_{i+q}),x_{i+q+1},\dots, x_k)=
%\]
%\[
%\underset{0\leq i\leq k}{\sum_{p,q,k=p+q-1}}b_p(m_p; x_1,\dots, x_i, b_q(m_q;x_{i+1},\dots, x_{i+q}),x_{i+q+1},\dots, x_k)=
%\]
%\[
%\sum_{k,p,q}b_k(b_1(m_p;m_q);x_1,\dots, x_k)=0
%\]
%because $\sum_{p,q}b_1(m_p;m_q)=0$ since $m\circ m=0$.
%
%IN THIS SIMPLER CASE I THINK IT CAN BE PROVED MORE DIRECTLY SINCE $M\circ M$ iff $\sum M_i\circ M_j=0$ iff $\sum m_i\circ m_j=0$. 
%\end{proof}
%
%The following theorem is not fully proved in the general case. This simpler case could serve as a base case, so let's try to prove it.
%
%\begin{theorem}
%The morphism $\Phi:\mathcal{O}=\prod_n\mathcal{O}(n)\to C^*(\mathcal{O},\mathcal{O})$ defined by $\Phi(x)=\sum b_n(x;-)$ is a (strict) morphism of $A_\infty$-algebras. 
%\end{theorem}
%
%\begin{proof}
%Let $n\geq 2$. Write $j_n$ for the arity %(degree? the definition of brace is in graded things, not things with arity, but in the hochschild complex the degree is given by the arity) 
%of $x_n$. 
%
%\[
%\Phi(M_n(x_1,\dots, x_n))=\Phi(b_n(m_n;x_1,\dots, x_n))=\sum_i b_i(b_n(m_n;x_1,\dots, x_n);-)=
%\]
%\[
%\sum_{i=i_1+\cdots+i_n}b_n(m_n;b_{i_1}(x_1;-),\dots, b_{i_n}(x_n;-))
%\]
%where $i_p\leq j_p$. Note that since the arity of $m_n$ is $n$, there are no terms inserted between the $x_p$'s.
%
%\[
%\overline{M}_n(\Phi(x_1),\dots, \Phi(x_n))=\sum_{i_1+\cdots+i_n}\overline{M}_n(b_{i_1}(x_1;-),\dots, b_{i_n}(x_n;-))=\sum_{i_1+\cdots+i_n}b_n(M_n;b_{i_1}(x_1;-),\dots, b_{i_n}(x_n;-))
%\] 
%
%Since the arity of $M_n$ is $n$, at least up to sign, this equals
%$$\sum_{i_1+\cdots+i_n}b_n(m_n;b_{i_1}(x_1;-),\dots, b_{i_n}(x_n;-))$$
%
%%For $n=1$ $M_1$ and $\overline{M}$ are defined slightly differently, but we can use the argument above on each summand and that proves $\Phi(M_1)=\overline{M}_1(\Phi)$. 
%
%%I'LL ASK CONSTANZE TO CHECK THIS. I THINK I MUST (AND CAN) ASSUME THAT THE BRACE IS OF THE KIND OF BRACE IN THE ENDOMORPHISM OPERAD. BUT IF A SIGN COMES ALONG TO DISTINGUISH BETWEEN EVALUATION AND BRACE, THE THEOREM MIGHT NOT BE TRUE ANYMORE. GENERALISING IT TO THE USUAL STRUCTURE WILL BE PROBLEMATIC TOO.
%\end{proof}
%
%\textbf{Comments:} It looks like this proof is highly dependent on the sign that appears in the brace for the endomorphism algebra. In particular, the difference between $M_n(x_1,...,x_n)$ and $b_n(M_n;x_1,\dots, x_n)$, which is just a sign, but it looks like in the proof should be exactly 1. 
%
%To generalise this to $M_n(x_1,\dots, x_n)=\sum b_n(m_i;x_1,\dots, x_n)$ we will probably have to write each summand down, because $b_n(M_s;x_1,\dots, x_n)=\sum \pm M_s(1,\dots, 1,x_1,1,\dots, 1,x_n,1\dots, 1)$, and we should check that each term appears on the other side of the equation.
%
%\subsection{Things with signs}
%IN LEMMA 3.2 DON'T FORGET THE SIGN OF THE KOSZUL RULE WHEN EVALUATING, THIS PROBABLY CANCELS THE SIGNS IN THE BRACE RELATION
%
%I'm going to add the signs to the proof of Lemman 2.3 in the original setting (for the simplified version it is analogous).
%
%\begin{lemma}
%\end{lemma}
%\begin{proof}
%In the first line we omit any signs because the maps $M_s$ are not homogeneous.
%\[
%\underset{0\leq i\leq k}{\sum_{p,q,k=p+q-1}}M_p(x_1,\dots, x_i, M_q(x_{i+1},\dots, x_{i+q}),x_{i+q+1},\dots, x_k)=
%\]
%\[
%\underset{0\leq i\leq k, s\geq p,t\geq q}{\sum_{p,q,k=p+q-1}}(-1)^{|m_t|\sum_{j=1}^i|x_j|}b_p(m_s;x_1,\dots, x_i, b_q(m_t;x_{i+1},\dots, x_{i+q}),x_{i+q+1},\dots, x_k)
%\]
%The sign above is precisely the koszul sign in the brace relation so this does indeed equal to 
%
%\[
%\sum_{k,s,t}b_k(b_1(m_s;m_t);x_1,\dots, x_k)=0.
%\]
%\end{proof}
%
%IN THE PROOF OF THEOREM 2.7 I GUESS THE SIGN USED FOR THE PRODUCT AND THE DIFFERENTIAL IN THE HOCHSCHILD COMPLEX IS THE ONE COMING FROM THE ENDOMORPHISM CASE, BUT I'LL TRY TO DO IT WITHOUT ASSUMING ANY PARTICULAR VALUE, PROBABLY I SHOULD ALSO CHECK WHAT SIGNS I NEED IN THE SAME WAY I DID WITH M IN ORDER TO GET A DGA IN C(A,A)
\end{document}
