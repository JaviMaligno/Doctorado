\documentclass{article}
\usepackage{amsmath,accents}%
\usepackage{amsfonts}%
\usepackage{amssymb}%
\usepackage{graphicx}
\usepackage{mathrsfs}
\usepackage[utf8]{inputenc}
\usepackage{amsfonts}
\usepackage{amssymb}
\usepackage{graphicx}
\usepackage{mathrsfs}
\usepackage{setspace}  
\usepackage{amsthm}
\usepackage{nccmath}
\usepackage[UKenglish]{babel}
\usepackage{multirow}
\usepackage{enumerate}


\renewcommand{\baselinestretch}{1,4}
\setlength{\oddsidemargin}{0.5in}
\setlength{\evensidemargin}{0.5in}
\setlength{\textwidth}{5.4in}
\setlength{\topmargin}{-0.25in}
\setlength{\headheight}{0.5in}
\setlength{\headsep}{0.6in}
\setlength{\textheight}{8in}
\setlength{\footskip}{0.75in}

\theoremstyle{plain}
\theoremstyle{definition}
\newtheorem{exercise}{Exercise}
\setcounter{exercise}{0}



%--------------------------------------------------------
\begin{document}

\title{Clustering with ToMATo - Exercises }
\author{Javier Aguilar Martín}
\date{}
\maketitle
\begin{exercise}
Give an example of a Morse function on a surface with at least one critical point and sketch its basins of attraction.
\end{exercise}

\begin{exercise}
Consider the following persistence diagram. Divide it into two regions corresponding to the division between signal and noise. Determine the number of clusters corresponding to your division.
\begin{figure}[h!]
\centering
\includegraphics[scale=0.4]{diagramex}
\end{figure}
\end{exercise}

\begin{exercise}
Consider the following graph with estimated density values on the vertices. Apply the ToMATo algorithm choosing a suitable value of the merging parameter $\tau$. 
\begin{figure}[h!]
\centering
\includegraphics[scale=0.6]{graphexercise}
\end{figure}
\end{exercise}
%
%\begin{exercise}[Bonus] Implement the ToMATo algorithm in a programming language of your choice and test it against some dataset.
%\end{exercise}

\end{document}