\documentclass{beamer}
\usepackage[utf8]{inputenc}
\usetheme{Copenhagen}
%\usepackage[spanish]{babel}
\usepackage{multirow}
%\usepackage{estilo-apuntes}
\usepackage{braids}
\usepackage[]{graphicx}
\usepackage{rotating}
\usepackage{pgf,tikz}
\usepackage{pgfplots}
\usepackage{tikz-cd}
%\usepackage{empheq}
%\usepackage[dvipsnames]{xcolor}
\usepackage{xcolor}

\usetikzlibrary{arrows}
\usetikzlibrary{cd}
\usetikzlibrary{babel}
\pgfplotsset{compat=1.13}
\usetikzlibrary{decorations.shapes}
\pgfkeyssetvalue{/tikz/braid height}{1cm} %no parece hacer nada
\pgfkeyssetvalue{/tikz/braid width}{1cm}
\pgfkeyssetvalue{/tikz/braid start}{(0,0)}
\pgfkeyssetvalue{/tikz/braid colour}{black}

\theoremstyle{definition}

\newtheorem{teorema}{Theorem}
\newtheorem{defi}{Definition}
\newtheorem{prop}[teorema]{Proposition}

\newcommand{\Z}{\mathbb{Z}}
\newcommand{\Q}{\mathbb{Q}}
\newcommand{\R}{\mathbb{R}}
\newcommand{\CC}{\mathcal{C}}
\newcommand{\D}{\mathbb{D}}
\providecommand{\gene}[1]{\langle{#1}\rangle}

\DeclareMathOperator{\argmax}{argmax}


\addtobeamertemplate{navigation symbols}{}{%
    \usebeamerfont{footline}%
    \usebeamercolor[fg]{footline}%
    \hspace{1em}%
    %\insertframenumber/\inserttotalframenumber
}
\setbeamercolor{footline}{fg=black}
\setbeamerfont{footline}{series=\bfseries}

\newcommand{\highlight}[1]{%
	\colorbox{red!50}{$\displaystyle#1$}}

\makeatletter
\newcommand*{\encircled}[1]{\relax\ifmmode\mathpalette\@encircled@math{#1}\else\@encircled{#1}\fi}
\newcommand*{\@encircled@math}[2]{\@encircled{$\m@th#1#2$}}
\newcommand*{\@encircled}[1]{%
	\tikz[baseline,anchor=base]{\node[draw,circle,outer sep=0pt,inner sep=.2ex] {#1};}}
\makeatother

\expandafter\def\expandafter\insertshorttitle\expandafter{%
  \insertshorttitle\hfill%
  \insertframenumber\,/\,\inserttotalframenumber}

%-----------------------------------------------------------

\title{Clustering with ToMATo}
\author{Javier Aguilar Mart\'in}
\institute{University of Kent}
\date{}
 
\begin{document}
\frame{\titlepage}




\setbeamercovered{highly dynamic}

\newcounter{saveenumi}
\newcommand{\seti}{\setcounter{saveenumi}{\value{enumi}}}
\newcommand{\conti}{\setcounter{enumi}{\value{saveenumi}}}

\makeatletter
\newcommand{\xRightarrow}[2][]{\ext@arrow 0359\Rightarrowfill@{#1}{#2}}
\makeatother

\resetcounteronoverlays{saveenumi}



\section{Clustering}



\begin{frame}
\frametitle{clustering}
\begin{itemize}
\item \textbf{Clustering} is the task of grouping a set of objects in such a way that objects in the same group (called a cluster) are more similar (in some sense) to each other than to those in other groups (clusters). %typically by distance %it is an ill posed problem in general

\begin{figure}
\includegraphics[scale=0.5]{cluster}
\end{figure}
\end{itemize}
\end{frame}

\subsection{Classical methods}
\begin{frame}
\begin{itemize}
\frametitle{}
\item \textbf{Density clustering}: assume that data points are drawn from
some unknown density function $f$. Thresholding the density at some fixed level $\alpha$, then treating the connected components
of the superlevel-set $F^{\alpha} = f^{-1}([\alpha,+\infty))$ as clusters and the rest of the data as noise. %density must be estimated, we will estudy how the superlevel-sets evolve with alpha
\item \textbf{Mode-seeking}: detecting local peaks of $f$ in order to use them as cluster centers and partition the data according to their
\emph{basins of attraction}. %essentialy the points that are attracted follwing the gradient to the local peak 
\end{itemize}
\end{frame}

\begin{frame}
\begin{figure}
\includegraphics[scale=0.5]{basin}
\caption{Basin of attraction}
\end{figure}
\end{frame}

\subsection{Topologicla persistence}
\begin{frame}
\begin{itemize}
\item \textbf{Topological persistence}: Study the topology of the superlevel-sets $F^{\alpha} = f^{-1}([\alpha,+\infty))$ as $\alpha$ ranges from $+\infty$ to $-\infty$ to estimate the \emph{prominence} of the density peaks. %alpha goes backwards  %prominence is the lifespan of connected components
\begin{figure}
\includegraphics[scale=0.6]{density}
\caption{(a) original density and its superlevel-sets; (b) a
piecewise-linear approximation $\tilde{f}$ of $f$; (c) superimposition of the PDs of $f$ (red) and $\tilde{f}$ (blue).} %components are created and die %prominent peaks are matched %the peaks tells us that the clusters are the basins of attraction %the density just means that there are more points concentrated near the pics
\end{figure}
\end{itemize}
\end{frame}

\begin{frame}
%to make more clear that the density is not the data
\begin{figure}
\includegraphics[scale=0.7]{ex}
\caption{(a) estimation of the underlying density function $f$ at the data points;
(b) initial clusters; (c) approximate PD showing 2 points far off the diagonal
corresponding to the 2 prominent peaks of $f$; (d) final result obtained after merging the clusters of
non-prominent peaks.}
\end{figure}
\end{frame}

\section{ToMATo}
\subsection{Continuous setting}
\begin{frame}[fragile]
\frametitle{Continuous setting}
\begin{itemize}
\item<1-> Let $X$ be a $n$-dimensional Riemannian Mannifold without boundary and $f:X\to\R$ a bounded above and proper Morse function with finitely many critical points.
\item<2-> Let $m$ be a critical point of $f$. The \textbf{ascending region} $A(m)$ is the set of points of $X$ that eventually reach $m$ moving along the gradient flow of $f$. For all $x\in A(m)$ we call $m=r(x)$ the \textbf{root} of $x$.

\item<3-> For $\alpha\in\R$ let $F^\alpha=f^{-1}([\alpha,+\infty))$ be a superlevel-set. The family $\{F^\alpha\}_{\alpha\in\R}$ is called the \textbf{superlevel-set filtration} of $f$. Denote $C(x,\alpha)\subseteq F^\alpha$ the path-connected component of $F^\alpha$ containing $x$. %consider alpha varying from +infty to -infty
\end{itemize}
\end{frame}

\begin{frame}
\begin{itemize}
\item<1-> Let $m_p$ be a peak of $f$ generating a component at time $p_x=f(m_p)$. Let $p_y\leq p_x$ be the time at which this component merges another component generated by a peak $m_q$. %The reason for the notation is the coordinates in persistence diagram
\item<2->We call $m_q=r(m_p)$ the \textbf{root} of $m_p$ and we say that $m_p$ is \textbf{$(p_x-p_y)$-prominent} or that the \textbf{prominence} of $m_p$ is $p_x-p_y\geq 0$.
\item<3-> Let $\tau\geq 0$. Iterate the root map $m_q\to r(m_q)$ until a peak of prominence $\tau$ is reached. Call this iterated root $r^*_\tau$.
\end{itemize}
\end{frame}
\begin{frame}
\frametitle{Basins of attraction}
\begin{itemize}
\item<1-> The union of the ascending regions of the peaks mapped to $m_p$ through $r^*_\tau$ is referred to as the \textbf{basin of attraction}.
\item<2-> $\forall m_p$ such that $p_x - py \geq\tau$
\[B_\tau(m_p)=\bigcup_{r^*_\tau(m_q)=m_p} A(m_q)\]
\item<3->[]The basins of attraction form a partition of the union of all ascending
regions. These basins are our target clusters.
\end{itemize}

\end{frame}

\subsection{Discrete setting}
\begin{frame}
\frametitle{Discrete setting}
\begin{itemize}
\item<1-> \textbf{Input}: unweighted simple graph $G$, whose
vertex set represents the data points and whose edges connect the points that are similar. %that are close together, user-defined
\begin{itemize}
\item<1-> Each vertex $i$ of $G$ must be assigned a value $\tilde{f}(i)\geq 0$ corresponding to the estimated density at that point.
\end{itemize}
\item<1-> Merging parameter $\tau\geq 0$.
\item<2-> \textbf{Output}: Subset of clusters $e$ whose root $r(e)$ are the peaks of $\tilde{f}$ with prominence at least $\tau$ and such that $\tilde{f}(r(e))\geq \tau$. %last bit to discard noise from sparse data that never gets merged
\end{itemize}
\end{frame}

\begin{frame}
\frametitle{Algorithm}
\textbf{Mode-seeking}

\begin{itemize}
\item<2-> Iterate over the vertices of $G$ sorted by decreasing $\tilde{f}$-values.
\item<3-> Connect vertex $i$ to its neighbour with highest $\tilde{f}$-value if that value is higher than $\tilde{f}(i)$. Otherwise, $i$ is declared a peak of $\tilde{f}$. %it simulates the effect of the gradient of the underlying density function
\end{itemize}  
\begin{itemize}
\item<4->[] The resulting collection of
pseudo-gradient edges forms a spanning forest of the graph, and each tree in this
forest is analogue to the ascending region of a peak of in the continuous setting.
\end{itemize}
\end{frame}
\begin{frame}
\frametitle{Algorithm}
\textbf{Merging}:
\begin{itemize}
\item<2-> Let $U$ be union-find data structure where each entry corresponds to a union of trees of the spanning forest.
\item<3-> We call \emph{root} of an entry $e$, or $r(e)$, the vertex contained in $e$ whose $\tilde{f}$-value is highest.
\item<4-> Iterate again over the vertices of $G$ in the same order: 
\begin{itemize}
\item<5-> If $i$ is a local peak of $\tilde{f}$, i.e. $i$ is the root of some tree $T$, create a new entry $e$ in $U$ containing $T$ and set $r(e)=i$.%bein a local peak of tilde f in G is the same as beeing the root of some tree
\item<6-> Otherwise, let $e_i$ the entry of $U$ to which $i$ belongs. Compute the set of entries $E\subseteq U$ that contain the neighbours of $i$. For each $e\in E$, if $e\neq e_i$ and $\min\{\tilde{f}(e),\tilde{f}(e_i)\}<\tilde{f}(i)+\tau$, then merge $e$ and $e_i$ into $e\cup e_i$ in $U$ and set $r(e\cup e_i)=\argmax_{\{r(e),r(e_i)\}}\tilde{f}$. %this determines the prominence, because if a cluster is merged before reaching a vertex that is close in height to the root, the prominence is precisely that height difference because components are always bron at the peak and die when merged
\end{itemize}
\end{itemize}
\end{frame}

\begin{frame}
\frametitle{Algorithm}
ITEMIZE, PARAGRAPH ABOUT LIFESPANS
Upon termination, the merged clusters stored in the entries of the union-find data
structure $U$ form a partition of the vertex set of $G$, and their roots are the peaks of $\tilde{f}$
of prominence at least $\tau$ within the graph. 

The output of ToMATo is then the subset of this collection of clusters that is stored in those entries $e$ such that $\tilde{f}(r(e))\geq \tau$. 

The rest of the data points is treated as
background noise and discarded from the data set.
\end{frame}

\begin{frame}
COMPLEXITY
\end{frame}

\section{Parameter selection}
\begin{frame}
\frametitle{Parameter selection}
VIETORIS RIPS MAINLY BECAUSE THERE ARE THEORETICAL RESULTS (EXPLAINED LATER)
NOT MUCH ABOUT DENSITY ESTIMATION
PROMINENCE PARAMETER: RUN TWICE AND OBSERVE PD (IT IS MENTIONED EARLIER IN THE PAPER BUT BETTER SAY JUST HERE)
\end{frame}

\section{Theoretical guarantees}
\begin{frame}
\frametitle{Theoretical guarantees}
THE NON-TECHNICAL SECTION, IF THERE'S TIME GO MORE TECHNICAL AND USE THE NON-TECHNICAL AS EXPLANATION (MULTI-BIJECTIONS, INTERLEAVING DIAGRAMS -NO NEED TO DEFINE THEM, IT'S ON OTHER TOPICS, BUT INCLUDE A SLIDE JUST IN CASE)
THE 4 POINTS OF SECTION 7 (NEED TO INCLUDE ASSUMPITON ON F AND ON THE MANIFOLD), THEOREM 7.2 (THIS IS RELATED TO THE QUARTERS) MAYBE MENTION THE TEXT BEFORE 7.3
THEOREM 8.2 WITH FIG15 LEFT
EXPLAIN THE DIFFERENT QUARTERS ACCORDING TO ALPHA
USE DEF 9.1 FOR SEPARATED WITH 16 AND THM 9.2
THM 10.1 WITH EXPLANATION BELOW
\end{frame}

\section{Experimental results}
\begin{frame}
\frametitle{Experimental results}
THE RINGS FOR HIGH DIMENSIONAL STUFF AND IMAGE SEGMENTATION WITHOUT THE WEIRD BARS
\end{frame}



\end{document}
