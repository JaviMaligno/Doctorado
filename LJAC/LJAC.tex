\documentclass{beamer}
\usepackage[utf8]{inputenc}
\usetheme{Copenhagen}
%\usepackage[spanish]{babel}
\usepackage{multirow}
%\usepackage{estilo-apuntes}
\usepackage{braids}
\usepackage[]{graphicx}
\usepackage{rotating}
\usepackage{pgf,tikz}
\usepackage{pgfplots}
\usepackage{tikz-cd}
%\usepackage{empheq}
%\usepackage[dvipsnames]{xcolor}
\usepackage{xcolor}

\usetikzlibrary{arrows}
\usetikzlibrary{cd}
\usetikzlibrary{babel}
\pgfplotsset{compat=1.13}
\usetikzlibrary{decorations.shapes}
\pgfkeyssetvalue{/tikz/braid height}{1cm} %no parece hacer nada
\pgfkeyssetvalue{/tikz/braid width}{1cm}
\pgfkeyssetvalue{/tikz/braid start}{(0,0)}
\pgfkeyssetvalue{/tikz/braid colour}{black}

\theoremstyle{definition}

\newtheorem{teorema}{Theorem}
\newtheorem{defi}{Definition}
\newtheorem{prop}[teorema]{Proposition}

\newcommand{\Z}{\mathbb{Z}}
\newcommand{\Q}{\mathbb{Q}}
\newcommand{\C}{\mathbb{C}}
\newcommand{\CC}{\mathcal{C}}
\newcommand{\D}{\mathbb{D}}
\providecommand{\gene}[1]{\langle{#1}\rangle}

\DeclareMathOperator{\im}{im}


\addtobeamertemplate{navigation symbols}{}{%
    \usebeamerfont{footline}%
    \usebeamercolor[fg]{footline}%
    \hspace{1em}%
    %\insertframenumber/\inserttotalframenumber
}
\setbeamercolor{footline}{fg=black}
\setbeamerfont{footline}{series=\bfseries}

\newcommand{\highlight}[1]{%
	\colorbox{red!50}{$\displaystyle#1$}}

\makeatletter
\newcommand*{\encircled}[1]{\relax\ifmmode\mathpalette\@encircled@math{#1}\else\@encircled{#1}\fi}
\newcommand*{\@encircled@math}[2]{\@encircled{$\m@th#1#2$}}
\newcommand*{\@encircled}[1]{%
	\tikz[baseline,anchor=base]{\node[draw,circle,outer sep=0pt,inner sep=.2ex] {#1};}}
\makeatother


%-----------------------------------------------------------

\title{$A_\infty$-algebras and operads}
\author{Javier Aguilar Mart\'in}
\institute{University of Kent}
\date{Junior London Algebra Colloquium}
 
\begin{document}
\frame{\titlepage}
%\begin{frame}
%
%c¡
%\title[About Beamer] %optional
%{About the Beamer class in presentation making}
% 
%\subtitle{A short story}
% 
%\author[Arthur, Doe] % (optional, for multiple authors)
%{A.~B.~Arthur\inst{1} \and J.~Doe\inst{2}}
% 
%\institute[VFU] % (optional)
%{
%  \inst{1}%
%  Faculty of Physics\\
%  Very Famous University
%  \and
%  \inst{2}%
%  Faculty of Chemistry\\
%  Very Famous University
%}

% 
%\date[VLC 2013] % (optional)
%{Very Large Conference, April 2013}


%\end{frame}
\setbeamercovered{highly dynamic}

\newcounter{saveenumi}
\newcommand{\seti}{\setcounter{saveenumi}{\value{enumi}}}
\newcommand{\conti}{\setcounter{enumi}{\value{saveenumi}}}

\makeatletter
\newcommand{\xRightarrow}[2][]{\ext@arrow 0359\Rightarrowfill@{#1}{#2}}
\makeatother

\resetcounteronoverlays{saveenumi}
%\AtBeginSection[]{
%\begin{frame}
%\frametitle{Tabla de contenidos}
%\tableofcontents
%\end{frame}
%}




%\begin{frame}
%DEFINITION OF AINFTY ALGEBRAS AND HOW THEY GENERALLIZE ASSOCIATIVE ALGEBRAS
%
%MAIN EXAMPLE: CHAIN COMPLEX LOOP SPACES
%
%OPERADS: DEFINITION (SYMMETRI AND  NS), ALGEBRAS OVER AN OPERAD, EXAMPLES (END, ASSOCIATIVE, COMMUTATIVE)
%
%OPERADIC SUSPENSION, ANOTHER WAY TO DEFINE AINFINTY MULTIPLICATIONS IN A SIMPLIFIED WAY, LEAVING M1 IN THE UNDERLYING CATEGORY AND DEFINING THE DIFFERENTIAL WITH RESPECT TO M1 REDIFINES AINFTY AS MAURER-CARTAN ELEMENT
%
%IF THERE'S TIME CONNECTION WITH KOSZUL DUAL (LODAY VALLETTE BEFORE 10.1.11)
%
%HIGHER A-INFTY STRUCTURE
%\end{frame}


\section{Framework}
\begin{frame}
\frametitle{Starting point}

\begin{itemize}
\item<1-> \textbf{Symmetric monoidal category of graded $R$-modules} 
%as vector spaces but we may have a ring that is not a field

Let $R$ be a commutative ring of characteristic distinct of 2. 
\item[$\ast$]<2-> Objects: graded $R$-modules $V=\bigoplus_p V^p$ where $V^p$ is the degree $p$ component. (Example $\Q[x]$)
%direct sum can be taken over any abelian monoid (more structure depending on what you need), tipically the integers
\item[$\ast$]<3-> Morphisms: $\hom(V,W)=\bigoplus_p \hom(V,W)^p$. A linear map $f:V\to W$ is of degree $p$ if $f(V^n)\subseteq W^{n+p}$ for all $n$. (Example $x:\Q[x]\to \Q[x]$)
\item[$\ast$]<4-> Tensor product $(V\otimes W)^n=\bigoplus_{p+q=n} V^p\otimes_R W^q$. 
\item[$\ast$]<5-> Symmetry $\tau:V\otimes W\to W\otimes V$, $\tau(a\otimes b)=(-1)^{\deg(a)\deg(b)}b\otimes a$.
\end{itemize}
\end{frame}
\begin{frame}
\frametitle{Orientations}
\includegraphics[scale=0.4]{orientation}
\end{frame}

\begin{frame}[fragile]
\begin{itemize}
\item<1-> \textbf{Symmetric monoidal category of chain complexes}
\item[$\ast$]<2-> A chain complex $V$ is a graded $R$-module with a differential $d:V\to V$ of degree $1$ such that $d^2=0$.

\[\cdots\to V^0\xrightarrow{d_0}V^1\xrightarrow{d_1}V^2\to\cdots\]
%the map d is the same, but may act different on different ddegree components
\item[]<3-> Example:
\[0\to \Z\xrightarrow{\times 2}\Z\xrightarrow{q}\Z/2\Z\to 0\]
\item[$\ast$]<4-> A chain map is a linear map $f:V\to W$ such that $fd_V=d_Wf$. %usually they must be of degree 0, that's how the appear in topology, but one could let more degrees and would obtain different things, even getting a chain complex on the morphisms
\item[]<5-> Example:
\[
\begin{tikzcd}
\Z\ar[r,"\times 2"]\ar[d,"\times 2"] &\Z\ar[r,"q"]\ar[d,"\times 3"]&\Z/2\Z\ar[d,"0"]\\
\Z\ar[r,"\times 3"] &\Z\ar[r,"q"]&\Z/3\Z
\end{tikzcd}
\]
\end{itemize}
\end{frame}

\begin{frame}
\begin{itemize}
\item<1-> \textbf{Symmetric monoidal category of chain complexes}
\item[$\ast$]<2-> Same tensor product $(V\otimes W)^n=\bigoplus_{p+q=n} V^p\otimes_R W^q$ with differential $d_V\otimes 1+1\otimes d_W$. %because it needs to be of degree 1$ %1 in the tensor means identity %talk about why this differential, maybe mention leibniz rule

\item[$\ast$]<3-> Also same symmetry $\tau:V\otimes W\to W\otimes V$, $\tau(a\otimes b)=(-1)^{\deg(a)\deg(b)}b\otimes a$.
\end{itemize}
\end{frame}

\section{Homotopy}
\begin{frame}
\frametitle{Homotopy in Topology}
\begin{itemize}
\item<1-> Equivalence weaker than homeomorphism.
\item<2-> $\mathbb{R}^n$ is homotopy equivalent to a single point.
\item<3-> An open cylinder is homotopy equivalent to a circle ($S^1\times \mathbb{R}\simeq S^1\times \{*\}\cong S^1$).
\item<4-> ``Holes'' are preserved.
\end{itemize}
\end{frame}
\begin{frame}
\frametitle{Homotopy of chain complexes}
\begin{itemize}

%one reason to stop at degree 0 for chain maps is that in degree 1 we have the following
\item[]<1->Let $f,g:V\to W$ be maps of chain complexes.

\item[$\bullet$]<2-> A \textbf{homotopy} between $f,g$ is a map $h:V\to W$ of degree $-1$ $f-g=d_Wh+hd_V$ %there is a measure of how different they are, and if you know about homology this means that f and g are equal in homology which is setting d=0

\item[$\ast$]<3-> When such a homotopy exists, $f$ and $g$ are \textbf{homotopy equivalent} or \textbf{homotopic}.
\item[$\ast$]<4-> Homotopy equivalence is an equivalence relation.
%we may consider homotopic maps to be the same  and study properties that are preserved by homotopy or not (there are certain classificationss of spaces using homotopy)
\end{itemize}
\end{frame}

\section{$A_\infty$-algebras}

\begin{frame}
\frametitle{$A_\infty$-algebras}
\begin{defi}
An $A_\infty$-\textbf{algebra} $A$ is a graded $R$-module equipped with a family of ``multiplications'' $m_i:A^{\otimes i}\to A$ of degree $2-i$ satisfying the relation %

\[\sum_{r+s+t=n}(-1)^{rs+t}m_{r+1+t}(1^{\otimes r}\otimes m_s\otimes 1^{\otimes s})=0\] %we are composing every map with itself
\end{defi}
\end{frame}





\begin{frame}
\frametitle{Some particular cases}
\begin{itemize}
\item<1-> We always have $m_1m_1=0$, so in particular $A$ is a chain complex. %m1 of degree 2-1=1 %CAN BE DEFINED ON THE CATEGORY OF CHAIN COMPLEX %THE OTHER RELATIONS INVOLVING M1 ARE A CONSEQUENCE OF MORPHISMS OF CHAIN COMPLEXES
\item<2-> If $m_i=0$ for $i\neq 2$, the relation becomes $m_2(1\otimes m_2)=m_2(m_2\otimes 1)$, so $A$ is an associative algebra.
\item<3-> We also have the relation \[m_1m_2=m_2(m_1\otimes 1)+m_2(1\otimes m_1)\]%use this example to show the equivalence of possible definitions %DG %MONOID IN CHAIN COMPLEX ANALOGUE TO MONOID IN K-VECT
\item[]<4-> This is the Leibniz rule, and $A$ is a differential graded (dg) algebra.

\end{itemize}
\end{frame}


\begin{frame}
%until here it doesn't seem to be relevant that there are two equivalent ways of defining things, why would you introduce chain complexes and homotopies? here is the reason
\frametitle{$A_\infty$-algebras are homotopy associative algebras.}
%how do they generalize associative algebras
\begin{itemize}
\item<1-> For $n=3$ we have the relation
\begin{align*}
&m_2(m_2\otimes 1)-m_2(1\otimes m_2)=\\ %the failure of m_2 to be associative
&m_1m_3+m_3(m_1\otimes 1\otimes 1)+m_3(1\otimes m_1\otimes 1)+m_3(1\otimes 1\otimes m_1)
\end{align*}
\item[]<2-> $m_2$ is homotopy associative with homotopy given by $m_3$.  %recall that m1 is a differential so on homology this vanishes
\item<3-> The higher relations are a homotopy coherent extension of this fact. %m3 satisfies some relation up to homotopy given by m4 and so on
\end{itemize}
\end{frame}



\begin{frame}
\frametitle{Examples}
\begin{itemize}
\item<1-> Associative and  dg algebras. %trivial
\item<2-> Let $A$ be an associative algebra with a non-zero element $e$ such that $ae=ea=0$ for all $a\in A$. Then $A$ is an $A_\infty$-algebra with $m_2$ the product and $m_3=e$, $m_i=0$ for all $i\neq 0$.%a little bit less trivial, example from Keller in which e is an arrow from a quiver
\item<3-> Cellular chains of loop spaces. ($\Omega(X)=\{f:S^1\to X\mid f(1)=x_0\}$)%not trivial

\end{itemize}
\end{frame}

\begin{frame}
\frametitle{Homotopy associativity of loop spaces}
\begin{tikzpicture}[line cap=round,line join=round,>=triangle 45,x=1.0cm,y=1.0cm]
\clip(-4.175394430564892,-2.5911383046897085) rectangle (7.490400123879831,3.3976612960713135);
\draw(0.,2.) circle (1.cm);
\draw(0.,-1.) circle (1.cm);
\draw (-3.49428016933844,2.311720973491667) node[anchor=north west] {$(\gamma_1*\gamma_2)*\gamma_3$};
\draw (-3.496796110195476,-0.6773028846978557) node[anchor=north west] {$\gamma_1*(\gamma_2*\gamma_3)$};
\draw (-0.6057688157486263,1.1) node[anchor=north west] {$\gamma_3$};
\draw (0.2964512548334696,0.4) node[anchor=north west] {$\gamma_1$};
\draw (0.7431133461355,2.9675859207922457) node[anchor=north west] {$\gamma_1$};
\draw (-1.3,3.0105934583201526) node[anchor=north west] {$\gamma_2$};
\draw (-1.3,-1.4729423289641315) node[anchor=north west] {$\gamma_2$};
\draw (0.7489686163804383,-1.4729423289641315) node[anchor=north west] {$\gamma_3$};
\draw (2.,2.)-- (3.,1.);
\draw (3.,1.)-- (4.,2.);
\draw (3.,2.)-- (2.514225889472578,1.485774110527422);
\draw (2.,-1.)-- (3.,-2.);
\draw (3.,-2.)-- (4.,-1.);
\draw (3.,-1.)-- (3.481998691455241,-1.518001308544759);
\draw (1.7811495170502019,2.6342775049509677) node[anchor=north west] {$\gamma_1$};
\draw (2.8240823021019423,2.623525620568991) node[anchor=north west] {$\gamma_2$};
\draw (3.7487443589519387,2.5912699674230613) node[anchor=north west] {$\gamma_3$};
\draw (1.7811495170502019,-0.4085057751484382) node[anchor=north west] {$\gamma_1$};
\draw (2.8133304177199654,-0.4085057751484382) node[anchor=north west] {$\gamma_2$};
\draw (3.813255665243799,-0.4300095439123916) node[anchor=north west] {$\gamma_3$};
\draw [->] (1.,2.) -- (0.9776684310102762,2.2101533701987783);
\draw [->] (0.,3.) -- (-0.2228209821222102,2.9748593795651215);
\draw [->] (-1.,2.) -- (-0.9818938166599703,1.8105678675490438);
\draw [->] (1.,-1.) -- (0.9832067635335799,-0.8175049037869153);
\draw [->] (-1.,-1.) -- (-0.9827773938908082,-1.1847933820708718);
\draw [->] (0.,-2.) -- (0.1987984337916885,-1.9800403985152712);
\end{tikzpicture}
\end{frame}
\begin{frame}
\frametitle{Homotopy coherence}
\definecolor{dcrutc}{rgb}{0.39215686274509803,0.39215686274509803,0.39215686274509803}
\begin{tikzpicture}[line cap=round,line join=round,>=triangle 45,x=1.0cm,y=1.0cm]
\clip(-3.0601652892561972,-1.781126972201351) rectangle (5.063801652892563,3.6734184823441027);
\fill[color=dcrutc,fill=dcrutc,fill opacity=0.10000000149011612] (2.5,2.5) -- (0.5,2.5) -- (-0.1180339887498949,0.5978869674096935) -- (1.5,-0.5776835371752531) -- (3.118033988749895,0.5978869674096927) -- cycle;
\draw (2.5,2.5)-- (0.5,2.5);
\draw (0.5,2.5)-- (-0.1180339887498949,0.5978869674096935);
\draw (-0.1180339887498949,0.5978869674096935)-- (1.5,-0.5776835371752531);
\draw (1.5,-0.5776835371752531)-- (3.118033988749895,0.5978869674096927);
\draw (3.118033988749895,0.5978869674096927)-- (2.5,2.5);
\draw (-0.4298121712997737,3.113688955672426)-- (0.16372652141247268,2.6629000751314793);
\draw (0.16372652141247268,2.6629000751314793)-- (0.6445679939894825,3.196333583771599);
\draw (-0.24422955190381973,2.9727401308147394)-- (-0.1367993989481584,3.113688955672426);
\draw (-0.031385527253731726,2.8110864412070775)-- (0.17123966942148847,3.10617580766341);
\draw (-1.2938241923365879,0.8973102930127725)-- (-0.6627197595792627,0.4314951164537945);
\draw (-0.6627197595792627,0.4314951164537945)-- (-0.16685199098422132,0.9649286250939145);
\draw (-0.8706947249429613,0.5850004480317625)-- (-0.5049436513899312,0.9799549211119465);
\draw (-0.6957015377932548,0.7739659686300595)-- (-0.9106536438767833,0.9799549211119465);
\draw (0.869962434259956,-0.8908189331329814)-- (1.516093163035313,-1.3791735537190069);
\draw (1.516093163035313,-1.3791735537190069)-- (2.0645529676934644,-0.8532531930879026);
\draw (1.699788860636136,-1.203026994375752)-- (1.1930277986476345,-0.838226897069871);
\draw (1.4723143911047147,-1.0392758405399645)-- (1.6888955672426758,-0.838226897069871);
\draw (3.2516303531179576,0.8522314049586783)-- (3.8151164537941407,0.2737190082644637);
\draw (3.8151164537941407,0.2737190082644637)-- (4.416168294515403,0.8221788129226153);
\draw (4.236244854403876,0.6579986738208468)-- (4.055537190082646,0.8372051089406468);
\draw (4.037275969199595,0.47643956607194105)-- (3.634800901577762,0.86725770097671);
\draw (2.2373553719008274,3.1437415477084896)-- (2.7783020285499633,2.5802554470323065);
\draw (2.7783020285499633,2.5802554470323065)-- (3.37184072126221,3.1136889556724268);
\draw (2.4140676158297834,2.959666293615827)-- (2.590473328324569,3.136228399699474);
\draw (3.189075837510895,2.949431908250359)-- (3.0112096168294524,3.1212021036814424);
\draw [->] (0.8624492862509402,2.880781367392938) -- (1.9894214876033067,2.8657550713749065);
\draw [->] (3.168985725018784,2.5351765589782125) -- (3.612261457550715,1.1903230653643886);
\draw [->] (-0.12928625093914242,2.647873779113449) -- (-0.6101277235161522,1.3180465815176567);
\draw [->] (-0.29457550713748953,0.12345604808414862) -- (0.6370548459804668,-0.78563486100676);
\draw [->] (2.395131480090159,-0.755582268970697) -- (3.37184072126221,0.1535086401202117);
\end{tikzpicture}
\end{frame}
%\begin{frame}
%$A_\infty$-algebras with $m_1=0$ are called \textbf{minimal}. \pause
%\begin{theorem}[Kadeishvili]
%$H^*(A)$ is a minimal $A_\infty$-algebra quasi-isomorphic to $A$. 
%\end{theorem}
%%I'm over a field, quasi-isommorphic means it induces isomorphism on homology
%%H*(A) is a minimal model of A
%\end{frame}
\section{Operads}
\begin{frame}
\frametitle{Operads}
	\begin{itemize}
			\item<1-> An \textbf{operad} is a collection of objects (spaces, modules, etc)  $\mathcal{O}=\{\mathcal{O}(n)\}_{n\geq 0}$, whose elements are thought to be $n$-ary operations $X^n\to X$. %decir que pueden ser (topological spaces, vector spaces, other objects) siempre que los axiomas tengan sentido, es decir, comentar que esto se puede hacer en cualquier categoría monoidal simétrica diciendo por encima lo que es: producto, unidad y axiomas. Poner algo de todos modos en alguna diapositiva
			\item<2-> We represent $n$-ary operations as trees with the following shape
			\begin{tikzpicture}[line cap=round,line join=round,>=triangle 45,x=1.0cm,y=1.0cm]
			\clip(-2.13333333333334,-0.093333333333332) rectangle (12.006666666666668,3.5);
			\draw [line width=2.pt] (2.,0.)-- (2.,1.);
			\draw [line width=2.pt] (2.,1.)-- (0.3666666666666659,3.);
			\draw [line width=2.pt] (2.,1.)-- (1.,3.);
			\draw [line width=2.pt] (2.,1.)-- (1.7,3.);
			\draw [line width=2.pt] (2.,1.)-- (3.,3.);
			\draw (0.1,3.493333333333331) node[anchor=north west] {$1$};
			\draw (0.8,3.52) node[anchor=north west] {$2$};
			\draw (1.5,3.493333333333331) node[anchor=north west] {$3$};
			\draw (2.8,3.453333333333331) node[anchor=north west] {$n$};
			\draw (2.1,3.453333333333331) node[anchor=north west] {$\cdots$};
			\end{tikzpicture}
	\end{itemize}

%\only<2->{\begin{defi}
%		An \textbf{operad} $\CC$ consists of topological spaces $\CC(j)$ for $j\geq 0$, together with the following data:
%		
%		\begin{itemize}
%			\item Continuous maps $\gamma : \CC(k) \times \CC(j_1) \times \cdots \times \CC(j_k) \to \CC(\sum_s j_s)$ such that the
%			following associativity formula is satisfied for all $c\in \CC(k)$, $d_s \in \CC(j_s)$, and $e_t \in \CC(i_t)$:
%			
%			\[\gamma(
%			\gamma(c; d_1, \dots , d_k); e_1, \dots , e_j) = 
%			\gamma(c; f_1, \dots , f_k),
%			\]
%			where $$f_s = \gamma(d_s; e_{j_1+\cdots+j_{s-1}+1}, \dots , e_{j_1+\cdots+j_s} ).$$%, and $f_s = *$ if $j_s = 0$ 
%			These maps are usually called \emph{structure maps}.
%			
%			
%			%{1,2,...,j} se divide en bloques y se permuta ese conjunto 
%		\end{itemize}
%	\end{defi}}
\end{frame}

\begin{frame}
	\begin{itemize}
		\item There are \textbf{composition maps} $\gamma : \mathcal{O}(n) \otimes \mathcal{O}(j_1) \otimes \cdots \otimes \mathcal{O}(j_n) \to \mathcal{O}(j_1+\cdots+j_n)$
		
		\begin{tikzpicture}[line cap=round,line join=round,>=triangle 45,x=1.0cm,y=1.0cm]
		\clip(-0.7355555555555552,-0.3222222222222197) rectangle (9.486666666666668,4.);
		\draw [line width=1.2pt] (3.,0.)-- (3.,1.);
		\draw [line width=1.2pt] (3.,1.)-- (1.,2.);
		\draw [line width=1.2pt] (3.,1.)-- (5.,2.);
		\draw [line width=1.2pt] (3.,1.)-- (2.,2.);
		\draw [line width=1.2pt] (3.,1.)-- (4.,2.);
		\draw [line width=1.2pt] (1.,2.)-- (1.0066666666666673,2.593333333333333);
		\draw [line width=1.2pt] (1.0066666666666673,2.593333333333333)-- (0.5,3.5);
		\draw [line width=1.2pt] (1.0066666666666673,2.593333333333333)-- (1.,3.5);
		\draw [line width=1.2pt] (1.0066666666666673,2.593333333333333)-- (1.362222222222223,3.4822222222222172);
		\draw [line width=1.2pt] (2.,2.)-- (1.9933333333333343,2.6555555555555532);
		\draw [line width=1.2pt] (1.9933333333333343,2.6555555555555532)-- (1.6555555555555563,3.491111111111106);
		\draw [line width=1.2pt] (1.9933333333333343,2.6555555555555532)-- (2.,3.5);
		\draw [line width=1.2pt] (1.9933333333333343,2.6555555555555532)-- (2.3222222222222233,3.4733333333333287);
		\draw [line width=1.2pt] (5.,2.)-- (4.997777777777778,2.691111111111109);
		\draw [line width=1.2pt] (4.997777777777778,2.691111111111109)-- (4.633333333333335,3.491111111111106);
		\draw [line width=1.2pt] (4.997777777777778,2.691111111111109)-- (5.,3.5);
		\draw [line width=1.2pt] (4.997777777777778,2.691111111111109)-- (5.362222222222223,3.5);
		\draw [line width=1.2pt] (4.,2.)-- (4.0022222222222235,2.60222222222222);
		\draw [line width=1.2pt] (4.0022222222222235,2.60222222222222)-- (3.691111111111112,3.5);
		\draw [line width=1.2pt] (4.0022222222222235,2.60222222222222)-- (4.,3.5);
		\draw [line width=1.2pt] (4.0022222222222235,2.60222222222222)-- (4.2955555555555565,3.4644444444444398);
		\draw (2.4,0.9933333333333338) node[anchor=north west] {$f$};
		\draw (0.6,4) node[anchor=north west] {$g_1$};
		\draw (1.6,4) node[anchor=north west] {$g_2$};
		\draw (4.7,4) node[anchor=north west] {$g_n$};
		\draw (3.6222222222222234,4) node[anchor=north west] {$g_{n-1}$};
		\draw [line width=1.2pt,dash pattern=on 2pt off 2pt] (1.,2.) circle (0.2951626461026548cm);
		\draw [line width=1.2pt,dash pattern=on 2pt off 2pt] (2.,2.) circle (0.2953633460212008cm);
		\draw [line width=1.2pt,dash pattern=on 2pt off 2pt] (4.,2.) circle (0.27550178686879956cm);
		\draw [line width=1.2pt,dash pattern=on 2pt off 2pt] (5.,2.) circle (0.2752866071013681cm);
		\draw (2.5,2.22) node[anchor=north west] {$\cdots$};
		\draw (5.8066666666666675,2.5933333333333315) node[anchor=north west] {$=\gamma(f;g_1,\dots, g_n)$};
		\end{tikzpicture}
		
		%DIBUJO DE LA COMPOSICIÓN PEGANDO $n$ ARBOLITOS  $g_i$ A UNO $f$ Y PONIENDO $\gamma(f;g_1,\dots, g_n)$
	\end{itemize}
\end{frame}

\begin{frame}
	\begin{itemize}
		\item Composition is associative:
		 %DIBUJO DE COMPOSICIÓN EN DOS PASOS CON UNA IGUALDAD A CADA LADO SEÑALAR EL ORDEN DE ALGÚN MODO
		\begin{tikzpicture}[line cap=round,line join=round,>=triangle 45,x=1.0cm,y=1.0cm]
		\clip(0.7133333333333343,-0.1) rectangle (11.62,6);
		\draw [line width=1.2pt,] (3.,0.)-- (3.0066666666666677,1.2333333333333327);
		\draw [line width=1.2pt,] (3.0066666666666677,1.2333333333333327)-- (1.98,2.18);
		\draw [line width=1.2pt,] (3.0066666666666677,1.2333333333333327)-- (4.006666666666668,2.1533333333333315);
		\draw [line width=1.2pt,] (1.98,2.18)-- (2.,3.);
		\draw [line width=1.2pt,] (2.,3.)-- (1.353333333333334,3.94);
		\draw [line width=1.2pt,] (2.,3.)-- (2.6466666666666674,3.9133333333333296);
		\draw [line width=1.2pt,] (4.006666666666668,2.1533333333333315)-- (4.,3.);
		\draw [line width=1.2pt,] (4.,3.)-- (3.446666666666667,3.9933333333333296);
		\draw [line width=1.2pt,] (4.,3.)-- (4.74,3.9133333333333296);
		\draw [line width=1.2pt,] (1.353333333333334,3.94)-- (1.353333333333334,4.62);
		\draw [line width=1.2pt,] (1.353333333333334,4.62)-- (0.8066666666666672,5.46);
		\draw [line width=1.2pt,] (1.353333333333334,4.62)-- (1.3666666666666674,5.446666666666661);
		\draw [line width=1.2pt,] (1.353333333333334,4.62)-- (1.8733333333333342,5.473333333333328);
		\draw [line width=1.2pt,] (2.6466666666666674,3.9133333333333296)-- (2.66,4.686666666666662);
		\draw [line width=1.2pt,] (2.66,4.686666666666662)-- (2.3666666666666676,5.473333333333328);
		\draw [line width=1.2pt,] (2.66,4.686666666666662)-- (2.9266666666666676,5.526666666666661);
		\draw [line width=1.2pt,] (3.446666666666667,3.9933333333333296)-- (3.446666666666667,4.62);
		\draw [line width=1.2pt,] (3.446666666666667,4.62)-- (3.3266666666666675,5.526666666666661);
		\draw [line width=1.2pt,] (3.446666666666667,4.62)-- (3.8066666666666675,5.5);
		\draw [line width=1.2pt,] (4.74,3.9133333333333296)-- (4.78,4.566666666666662);
		\draw [line width=1.2pt,] (4.78,4.566666666666662)-- (4.366666666666667,5.486666666666661);
		\draw [line width=1.2pt,] (4.78,4.566666666666662)-- (4.82,5.5);
		\draw [line width=1.2pt,] (4.78,4.566666666666662)-- (5.326666666666668,5.486666666666661);
		\draw (2.4866666666666672,1.18) node[anchor=north west] {$f$};
		\draw (1.3,3.) node[anchor=north west] {$g_1$};
		\draw (3.3,3.) node[anchor=north west] {$g_2$};
		\draw (0.67,4.6) node[anchor=north west] {$h_1$};
		\draw (2,4.6) node[anchor=north west] {$h_2$};
		\draw (2.9,4.6) node[anchor=north west] {$h_3$};
		\draw (4.1,4.6) node[anchor=north west] {$h_4$};
		\draw (4.956666666666667,3.3) node[anchor=north west] {$=\gamma(\gamma(f;g_1,g_2),h_1,h_2,h_3,h_4)$};
		\draw (4.953333333333334,2.3) node[anchor=north west] {$=\gamma(f;\gamma(g_1;h_1,h_2), \gamma(g_2;h_3,h_4))$};
		\end{tikzpicture}
	\end{itemize}
\end{frame}

\begin{frame}
	\begin{itemize}
		\item<1-> Identity element: 
	%UN ÁRBOL AL QUE SE LE METEN PALITOS CON 1 Y OTRO METIÉNDOSE EN EL 1 Y AL FINAL IGUAL AL ARBOL EN CUESTION
		\begin{tikzpicture}[line cap=round,line join=round,>=triangle 45,x=0.5cm,y=0.5cm]
\clip(-2.513333333333339,-.699166666666667) rectangle (12.903333333333347,7.66666666664);
\draw [line width=1.2pt,] (2.,0.)-- (2.,2.);
\draw [line width=1.2pt,] (2.,2.)-- (0.,4.);
\draw [line width=1.2pt,] (2.,2.)-- (1.,4.);
\draw [line width=1.2pt,] (2.,2.)-- (3.,4.);
\draw [line width=1.2pt,] (2.,2.)-- (4.,4.);
\draw [line width=1.2pt,] (0.,4.)-- (0.,6.);
\draw [line width=1.2pt,] (1.,4.)-- (1.,6.);
\draw [line width=1.2pt,] (3.,4.)-- (3.,6.);
\draw [line width=1.2pt,] (4.,4.)-- (4.,6.);
\draw (5.215833333333336,3.6133333333333337) node[anchor=north west] {$=f=$};
\draw (0.491666666666664,2.2175) node[anchor=north west] {$f$};
\draw (-0.5,7.050833333333333) node[anchor=north west] {$1$};
\draw (0.5,7.050833333333333) node[anchor=north west] {$1$};
\draw (2.5,7.050833333333333) node[anchor=north west] {$1$};
\draw (3.5,7.050833333333333) node[anchor=north west] {$1$};
\draw [line width=1.2pt,] (9.,0.)-- (9.,2.);
\draw [line width=1.2pt,] (9.,2.)-- (9.,4.);
\draw [line width=1.2pt,] (9.,4.)-- (7.,6.);
\draw [line width=1.2pt,] (9.,4.)-- (8.,6.);
\draw [line width=1.2pt,] (9.,4.)-- (10.,6.);
\draw [line width=1.2pt,] (9.,4.)-- (11.,6.);
\draw (9.465833333333338,3.9883333333333337) node[anchor=north west] {$f$};
\draw (9.445,1.5716666666666674) node[anchor=north west] {$1$};
\begin{scriptsize}
\draw [fill=black] (0.,4.) circle (2.pt);
\draw [fill=black] (1.,4.) circle (2.pt);
\draw [fill=black] (3.,4.) circle (2.pt);
\draw [fill=black] (4.,4.) circle (2.pt);
\draw [fill=black] (9.,2.) circle (2.pt);
\end{scriptsize}
\end{tikzpicture} %añadir un palo no cambia la forma delárbol
		\item<2-> An  operad is said to be \textbf{symmetric} if there is a right action of the symmetric group thought as reordering the inputs which is coherent with composition. %reordenas los inputs de f y metes los g_i como si nada. Eso es lo mismo que dejar f quieta, cambiar el orden de los g_i y consecuentemente en la composición los argumentos de arriba entran en otro orden. La otra es similar pero haciendo actuar permutaciones sobre los g_i, y luego la forma en la que esos g_i entran en f se reordena consecuentemente
	\end{itemize}
\end{frame}

\begin{frame}[fragile]
	Associativity and the existence of identity allows to understand compositions in terms of insertion maps \begin{align*}
	\circ_i:\mathcal{O}(n)\otimes\mathcal{O}(m)\to \mathcal{O}(m+n-1)\\ f\circ_i g=\gamma(f;1,\dots, 1,\underbrace{g}_{i},1,\dots, 1)
\end{align*} \pause
	
	Composition of insertions is thought as grafting one tree at a time.
	

\begin{tikzpicture}[line cap=round,line join=round,>=triangle 45,x=1.0cm,y=1.0cm]
\clip(-1.81,-0.5) rectangle (14.23,3.17);
\draw (3.,0.)-- (3.,1.);
\draw (3.,1.)-- (1.4,1.51);
\draw (3.,1.)-- (2.,1.5);
\draw (3.,1.)-- (3.,1.5);
\draw (3.,1.5)-- (3.,2.5);
\draw (3.,1.)-- (4.,1.5);
\draw (3.,1.)-- (4.5,1.5);
\draw (3.,2.)-- (2.5,2.5);
\draw (3.,2.)-- (3.5,2.5);
\draw (2.6,1) node[anchor=north west] {$f$};
\draw (2.5,2.1) node[anchor=north west] {$g$};
\begin{scriptsize}
\draw [fill=black] (3.,1.5) circle (1.5pt);
\end{scriptsize}
\end{tikzpicture}

\end{frame}
\begin{frame}
	\begin{defi}
	 A map of (symmetric) operads $f:\mathcal{O}\to \mathcal{O}'$ is a collection of maps $\mathcal{O}(n)\to \mathcal{O}'(n)$ such that:
		\begin{itemize}
			\item<1->   $f\circ 1_\mathcal{O}=1_{\mathcal{O}'}$.
			\item<2->  $f\circ \gamma_\mathcal{O}=\gamma_{\mathcal{O}'}\circ (f\otimes\cdots\otimes f)$.
			\item<3->   $f(x\sigma)=f(x)\sigma$ for $x\in\mathcal{O}(n)$ and $\sigma\in\Sigma_n$.
		\end{itemize}
	\end{defi}
	
	
\end{frame}
%\begin{frame}
%	\frametitle{Symmetric monoidal  categories}
%	\begin{itemize}
%		\item<1-> A category where there is a notion of tensor product $\otimes $ of objects.
%		\item<2-> There exists an object $I$ such that $I\otimes A\cong A\cong A\otimes I$ for all object $A$.
%		\item<3-> The product is commutative: $A\otimes B\cong B\otimes A$.
%		\item<4-> The product is associative: $(A\otimes B)\otimes C\cong A\otimes (B\otimes C)$ for all objects.
%		\item<5-> Other coherence axioms.
%	\end{itemize}
%%	COMENTAR QUE ESTA DEFINICIÓN SE PUEDE HACER EN CUALQUIER CATEGORÍA MONOIDAL SIMÉTRICA, DICIENDO LOS COMPONENTES DE LA DEFINICIÓN Y QUIZÁ DESTACANDO ALGÚN AXIOMA
%	
%   %PONER EJEMPLOS
%\end{frame}
\subsection{Algebras over an operad}
\begin{frame}
	\frametitle{Endomorphism operad}
	
	\begin{defi}
		Let $V$ be a graded $R$-module. The \textbf{endomorphism operad} $End_V = \{ \xi_V(n) \}_{n\geq 0}$ of $V$ consists of
		\begin{itemize}
			\item<1-> $\xi_V(n)=\hom(V^{\otimes n},V)
			$ the graded $R$-module of linear maps $V^{\otimes n} \to V$.
			\item<2-> composition $\gamma(f; g_1, \dots, g_n)= f(g_1\otimes\dots\otimes g_n)$
			\item<3-> identity $\operatorname{Id}_V$
			\item<4->  symmetric group action $\gamma (f; g_1, \dots, g_n) \cdot \sigma = f (g_{\sigma^{-1}(1)} \otimes \dots \otimes g_{\sigma^{-1}(n)})$,  $\sigma \in \Sigma_n$ %in the graded setting there are signs
		\end{itemize}
		 %(notice this is analogous to the fact that each ''R''-module structure on an abelian group ''M'' amounts to a ring homomorphism <math>R \to \operatorname{End}(M)</math>.)
	\end{defi}
\end{frame}
\begin{frame}
\begin{itemize}
\item<1->
If $\mathcal{O}$ is another operad, each operad morphism $\mathcal{O} \to End_V$ is called an \textbf{algebra over} $\mathcal{O}$. 
\item<2->Equivalently, a $\mathcal{O}$-algebra is given by a sequence of maps $\mathcal{O}(n)\otimes V^{\otimes n}\to V$.
\item<3-> This is a realization of the operad as a space of operations.
\end{itemize}
\end{frame}

\begin{frame}
\frametitle{Some examples}

\begin{itemize}
\item<1-> \textbf{Non-symmetric associative operad}: $\mathcal{O}(2)=R m_2$ and $\mathcal{O}(n)=R\mu_n$ for $n> 2$, $\mathcal{O}(1)=R$, $\mathcal{O}(0)=0$ ($\mathcal{O}(0)=R$ if there is a unit). %in arity 1 we always have the identity map


\item<2-> \textbf{Symmetric associative operad}: $\mathcal{O}(2)=Rm_2\oplus Rm_2(12)$ and $\mathcal{O}(n)=\bigoplus_{\sigma\in\Sigma_n}R_\sigma$, $\mathcal{O}(1)=R$, $\mathcal{O}(0)=0$. 
\item<3-> \textbf{Commutative (symmetric) operad}: same modules as non-symmetric associative operad. %it is necessarily symmetric

%\item<1-> An operad $\mathcal{O}$ is said to be \textbf{generated} by a set $S$ of elements if every element of $\mathcal{O}$ is a linear combination of compositions of elements in $S$ and actions of the symmetric groups on them.
%\item<2-> Let $\mathcal{O}$ be generated by an operation $m\in\mathcal{O}(2)$ such that $m\circ_1 m=m\circ_2 m$. This implies $m(m,1)=m(1,m)$ so an algebra over $\mathcal{O}$ is an associative algebra.
%\item<3-> If we add the relation $m(12)=m$ this implies $m(x,y)=m(y,x)$ so we get a commutative algebra
\end{itemize}
%operads are very useful to describe many types of algebras
\end{frame}


\begin{frame}
\frametitle{What about $A_\infty$-algebras?}
\begin{itemize}
\item<1-> We can define an operad generated by $m_i$ with $m_i\in\mathcal{O}(i)$ for all $i\geq 1$ such that 
$$\sum_{r+s+t=n}(-1)^{rs+t}m_{r+1+t}\circ_{r+1}m_s=0$$ 

\item<2-> We would like to obtain the signs directly from operadic composition
\end{itemize}
\end{frame}
\subsection{Operadic suspension}
\begin{frame}
\frametitle{Operadic suspension}
\begin{itemize}
\item<1-> For a graded $R$-module $V=\bigoplus_{p} V^p$ there is a \textbf{suspension} operation $\Sigma V$ such that $(\Sigma V)^p=V^{p-1}$. It raises the degree by 1.
\item<2-> $\Sigma V = V\otimes R[1]$.
\item<3-> We define an analogue of suspension for operads.
\end{itemize}
\end{frame}
\begin{frame}
\frametitle{Operadic suspension}

\begin{itemize}
\item<1-> Let $\Lambda(n)$ be a graded $R$-module concentrated in degree $n-1$ and spanned by $e^n=e_1\land\cdots\land e_n$. %for those who know abour rep theory, thiss is the sign representation of the  symmetric group
\item<2-> Consider the sign action of the permutation group on $e$:
\[(i\ i+1)\cdot e^n=e_1\land\cdots\land e_{i+1}\land e_i\land\cdots\land e_n=-e^n\]
\item<3-> Define insertion maps $\circ_i:\Lambda(n)\otimes\Lambda(m)\to\Lambda(n+m-1)$ as
\[(e_1\land\cdots\land e_n)\otimes(e_1\land\cdots\land e_m)\mapsto  (-1)^{(n-i)(m-1)}e_1\land\cdots\land e_{n+m-1}\]
\item[]<4-> \[e^n\circ_i e^m= (-1)^{(n-i)(m-1)}e^{m+n-1}\]
\end{itemize}
\end{frame}

\begin{frame}
\begin{itemize}
\item<1-> $\Lambda=\{\Lambda(n)\}$ is an operad.
\item<2-> The tensor product of operads $(\mathcal{O}\otimes \mathcal{P})(n)=\mathcal{O}(n)\otimes \mathcal{P}(n)$ is an operad with diagonal permutation action and composition. %the action and composition is done on each component separately
\item<3-> The operad $\mathfrak{s}\mathcal{O}=\mathcal{O}\otimes\Lambda$ is called the \textbf{operadic suspension} of $\mathcal{O}$.

%\begin{frame}
%\frametitle{Relation between suspension and operadic suspension}
\item[]<4-> \begin{theorem}[Markl]
There is an isomorphism of operads
\[ \mathfrak{s}End_{\Sigma V}\cong End_V\]
\end{theorem}
%\end{frame}

\end{itemize}
\end{frame}

\begin{frame}
\begin{itemize}
\item<1-> We identify each $x\in\mathcal{O}(n)$ with $x\otimes e^n\in \mathfrak{s}\mathcal{O}(n)$.
\item<2-> Then if $x$ has degree $p$ in $\mathcal{O}$, it has degree $p+n-1$ in $\mathfrak{s}\mathcal{O}$.
\item<3-> Let $\tilde{\circ}_i$ denote the insertion map in $\mathfrak{s}\mathcal{O}$.
\item[]<4-> \begin{align*}
\mathfrak{s}\mathcal{O}(n)\otimes\mathfrak{s}\mathcal{O}(m)=(\mathcal{O}(n)\otimes\Lambda(n))\otimes (\mathcal{O}(m)\otimes\Lambda(m))\\
\cong (\mathcal{O}(n)\otimes \mathcal{O}(m))\otimes (\Lambda(n)\otimes \Lambda(m))\\
\xrightarrow{\circ_i\otimes\circ_i} \mathcal{O}(m+n-1)\otimes \Lambda(n+m-1)\\=\mathfrak{s}\mathcal{O}(n+m-1).
\end{align*}
%note the isomorphism
\end{itemize}
\end{frame}
\begin{frame}
\begin{itemize}
\item<1-> The isomorphism $\Lambda(n)\otimes \mathcal{O}(m)\cong \mathcal{O}(m)\otimes \Lambda(n)$ is given by $x\otimes y\mapsto (-1)^{(n-1)\deg(y)}y\otimes x$.
\item<2-> If we add the sign of the insertion on $\Lambda$ we get for $a\in\mathcal{O}(n)$ and $b\in\mathcal{O}(m)$
\[a\tilde{\circ}_ib=(-1)^{(n-1)\deg(b)+(n-i)(m-1)}a\circ_i b.\]
\item<3-> Applied to $A_\infty$-maps the above formula gives 
\[m_{r+1+t}\tilde{\circ}_{r+1}m_s=(-1)^{rs+t}m_{r+1+t}\circ_{r+1}m_s\]
\item[]<5-> The sign of the $A_\infty$ equation!
\end{itemize}
\end{frame}

\begin{frame}
\begin{itemize}
\item<1-> This simplifies the equation to
\[\sum_{r+s+t=n}m_{r+1+t}\tilde{\circ}_{r+1}m_s=0\] %but we can simplify it even more
\item<2-> Let $a\tilde{\circ}b=\sum_{i}a\tilde{\circ}_ib$ and let $m=m_1+m_2+\cdots$. The equation becomes just
\item[]<3-> \[m\tilde{\circ}m=0.\] %Maurer-Cartan element
\item<4-> In addition, $m_i$ becomes of degree $1$ in the operadic suspension for all $i$ $\Rightarrow$
\item[]<5-> We can define an $A_\infty$-multiplication as an element $m\in\mathfrak{s}\mathcal{O}$ of degree $1$ such that $m\tilde{\circ}m=0$ (Maurer-Cartan element). 
\end{itemize}
\end{frame}
%\begin{frame}
%\frametitle{Relation between suspension and operadic suspension}
%\begin{theorem}[Markl]
%There is an isomorphism of operads
%\[ \mathfrak{s}End_{\Sigma V}\cong End_V\]
%\end{theorem}
%\end{frame}
\subsection{Algebraic structures on operads}
\begin{frame}
\frametitle{The circle operation}
%the operation we have defined is not so arbitrary or ad hoc
\begin{itemize}
\item<1-> For any operad, we can define a circle operation $a\circ b=\sum_i a\circ_i b$ similar to $\tilde{\circ}$. %even if it's not a suspension, the important thing is that we have an operad structure
\item The commutator of the circle operation 
\[[a,b]=a\circ b-(-1)^{\deg(a)\deg(b)}b\circ a\]
is a Lie bracket.
\end{itemize}
\end{frame}

\begin{frame}
\frametitle{Graded Lie bracket}
A bilinear map of degree 0 $[,]:V\otimes V\to V$ is a \textbf{graded Lie bracket} if the following holds:
\begin{itemize}
\item<2-> Graded anticommutativity:
\[[a,b]=-(-1)^{\deg(a)\deg(b)}[b,a].\]
\item<3-> Graded Jacobi identity:
\begin{align*}
(-1)^{\deg(x)\deg(z)}[x,[y,z]] +(-1)^{\deg(y)\deg(x)}[y,[z,x]]\\ +(-1)^{\deg(z)\deg(y)}[z,[x,y]] = 0.
\end{align*}

\end{itemize}
\end{frame}

\begin{frame}
\begin{itemize}
\item<1-> Since $m\tilde{\circ}m=0$, the Jacobi identity implies that $[m,[m,]]=0$ for the bracket induced by $\tilde{\circ}$.
\item<2-> Since $m$ is of degree $1$, this imples that the map $[m,]:\mathfrak{s}\mathcal{O}\to\mathfrak{s}\mathcal{O}$ turns $\mathfrak{s}\mathcal{O}$ into a chain complex. %this would actually be the shift but it's just arityty one so it doesn't change anything
\item<3-> Indeed, it is possible to define an $A_\infty$-algebra structure on $\Sigma\mathfrak{s}\mathcal{O}$. 
\end{itemize}
\end{frame}
\subsubsection{Braces}
\begin{frame}
\frametitle{Braces}
\begin{itemize}
\item For $a\in\mathfrak{s}\mathcal{O}(n)$ and $b_1,\dots,b_j\in\mathfrak{s}\mathcal{O}$ ($j\leq n$), define maps called \textbf{braces}
\[a\{b_1,\dots,b_j\}_j=\sum \gamma(a;1,\dots, 1,b_1,1,\dots,1,b_j,1,\dots, 1)\]\pause %the sum runs over all order preserving insertions
\begin{tikzpicture}[line cap=round,line join=round,>=triangle 45,x=1.0cm,y=1.0cm]
		\clip(-2.8466666666666676,-1) rectangle (11.486666666666668,2.);
		\draw[line width=1.pt] (2.833333333333334,0.) -- (2.833333333333334,0.5666666666666675) -- (7.006666666666668,0.5666666666666675) -- (7.,0.) -- cycle;
		\draw[line width=1.pt] (3.193333333333334,1.) -- (4.,1.) -- (4.006666666666668,1.58) -- (3.193333333333334,1.58) -- cycle;
		\draw[line width=1.pt] (6.,1.) -- (6.38,1.) -- (6.38,1.58) -- (5.593333333333335,1.5933333333333344) -- (5.606666666666667,1.) -- cycle;
		\draw (-2.9,0.6) node[anchor=north west] {$a\{b_1,\dots, b_j\}_j=\displaystyle{\sum_{\text{all possible insertions}}}$};
		\draw [line width=1.pt] (2.833333333333334,0.)-- (2.833333333333334,0.5666666666666675);
		\draw [line width=1.pt] (2.833333333333334,0.5666666666666675)-- (7.006666666666668,0.5666666666666675);
		\draw [line width=1.pt] (7.006666666666668,0.5666666666666675)-- (7.,0.);
		\draw [line width=1.pt] (7.,0.)-- (2.833333333333334,0.);
		\draw [line width=1.pt] (4.92,0.)-- (4.92,-0.36666666666666614);
		\draw [line width=1.pt] (2.993333333333334,0.5666666666666675)-- (3.,2.);
		\draw [line width=1.pt] (3.553333333333334,0.5666666666666675)-- (3.553333333333334,1.);
		\draw [line width=1.pt] (4.273333333333334,0.5666666666666675)-- (4.273333333333334,1.9933333333333345);
		\draw [line width=1.pt] (4.54,0.5666666666666675)-- (4.54,1.9933333333333345);
		\draw [line width=1.pt] (5.993333333333334,0.5666666666666675)-- (6.,1.);
		\draw [line width=1.pt] (6.62,0.5666666666666675)-- (6.62,1.9933333333333345);
		\draw [line width=1.pt] (6.833333333333335,0.5666666666666675)-- (6.833333333333335,1.98);
		\draw [line width=1.pt] (3.193333333333334,1.)-- (4.,1.);
		\draw [line width=1.pt] (4.,1.)-- (4.006666666666668,1.58);
		\draw [line width=1.pt] (4.006666666666668,1.58)-- (3.193333333333334,1.58);
		\draw [line width=1.pt] (3.193333333333334,1.58)-- (3.193333333333334,1.);
		\draw [line width=1.pt] (6.,1.)-- (6.38,1.);
		\draw [line width=1.pt] (6.38,1.)-- (6.38,1.58);
		\draw [line width=1.pt] (6.38,1.58)-- (5.593333333333335,1.5933333333333344);
		\draw [line width=1.pt] (5.593333333333335,1.5933333333333344)-- (5.606666666666667,1.);
		\draw [line width=1.pt] (5.606666666666667,1.)-- (6.,1.);
		\draw [line width=1.pt] (3.5666666666666673,1.58)-- (3.58,2.);
		\draw [line width=1.pt] (6.007451656136321,1.586314378709553)-- (6.,2.);
		\draw [line width=1.pt] (5.4,0.5666666666666675)-- (5.4,2.);
		\draw (3.3,1.5666666666666678) node[anchor=north west] {$b_1$};
		\draw (5.65,1.58) node[anchor=north west] {$b_j$};
		\draw (4.7,0.5) node[anchor=north west] {$a$};
		\draw (4.59,1.5533333333333343) node[anchor=north west] {$\cdots$};
		\end{tikzpicture}
\pause

\item Observe that $a\{b\}_1=a\tilde{\circ} b$. 

\end{itemize}
\end{frame}

\begin{frame}
\frametitle{$A_\infty$-multiplications}
\begin{theorem}
Up to shifts, the following maps define an $A_\infty$-algebra structure on $\mathfrak{s}\mathcal{O}$\pause
\begin{align*}
&M_1(x)=[m,x]\\
&M_j(x_1,\dots, x_j)=m\{x_1,\dots, x_j\}_j
\end{align*}
\end{theorem}
\end{frame}
%\begin{frame}
%
%\end{frame}
%\begin{frame}
%%there is a relation between the original structurer and the induced one
%\begin{theorem}[Gerstenhaber-Voronov]
%The map $\Phi:\mathfrak{s}\mathcal{O}\to $ defined by \[\Phi(x)=\sum_{n\geq 0} x\{x_1,\dots, x_n\}\]
%is a map of $A_\infty$-algebras, i.e. it satisfies 
%\[\Phi(M_n)=M_n(\Phi^{\otimes n})\]
%for all $n$.
%
%\end{theorem}
%\end{frame}
\begin{frame}
	\begin{center}
	\Huge{Thank you very much!}
\end{center}
\end{frame}

%the sign of the convolution operad described by Loday-Vallete depends on the choice of isomorphism sigma so my suspension is still valid
\end{document}

