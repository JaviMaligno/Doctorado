	\documentclass[twoside]{article}
\usepackage{estilo-ejercicios}
\setcounter{section}{0}
%\newtheorem{defin}{Definition}[section]
%\newtheorem{lem}[defin]{Lemma}
%\newtheorem{propo}[defin]{Proposition}
%\newtheorem{thm}[defin]{Theorem}
%\newtheorem{eje}[defin]{Example}
\renewcommand{\baselinestretch}{1,3}

\usepackage{empheq}
\newcommand*\widefbox[1]{\fbox{\hspace{2em}#1\hspace{2em}}}

%Below to introduce ¡ in mathmode https://tex.stackexchange.com/questions/471464/inverted-exclamation-mark-in-mathmode
\DeclareMathSymbol{\mathinvertedexclamationmark}{\mathclose}{operators}{'074}
\DeclareMathSymbol{\mathexclamationmark}{\mathclose}{operators}{'041}

\makeatletter
\newcommand{\raisedmathinvertedexclamationmark}{%
  \mathclose{\mathpalette\raised@mathinvertedexclamationmark\relax}%
}
\newcommand{\raised@mathinvertedexclamationmark}[2]{%
  \raisebox{\depth}{$\m@th#1\mathinvertedexclamationmark$}%
}
\begingroup\lccode`~=`! \lowercase{\endgroup
  \def~}{\@ifnextchar`{\raisedmathinvertedexclamationmark\@gobble}{\mathexclamationmark}}
\mathcode`!="8000
\makeatother
%--------------------------------------------------------
\begin{document}

\title{Introductory chapter}
\author{Javier Aguilar Martín}
\maketitle

\section{Introduction}

MAYBE THIS IS THE PLACE FOR FIXING R COMMUTATIVE WITH UNIT AND CHAR NOT 2 (POSSIBLY 0 OR NOT A FACTORIAL (representation theory things that I doubt I would need to clarify))
\begin{itemize}


\item MAYBE BASIC REVIEW OF CATEGORY THEORY INCLUDING EQUIVALENCE OF CATEGORIES


\item SYMMETRIC (CLOSED) MONOIDAL CATEGORIES, EXAMPLES

\item LAX/STRICT MONOIDAL FUNCTORS

\item ENRICHED VERSION OF THE ABOVE THINGS, INCLUDING THE CONTENT IN SARAH'S PAPER


\item S-MODULES/COLLECTIONS IN THE NON SYMMETRRIC CASE, OPERADS (SYMMETRIC AND NS)- POSSIBLY SEVERAL DEFINITIONS, AT LEAST CLASSICAL, PARTIAL AND THEIR EQUIVALENCE AND MONOIDAL, EXAMPLES, (INFINITESIMAL SEE 10.2.4) MODULE OVER AN OPERAD, QUASI-FREE OPERAD, PRE-LIE ALGEBRA DEFINED BY INSERTIONS, FORGETFUL FUNCTOR FROM SYM TO NS? SHOULD WRITE THE AXIOMS FOR GRADED OPERADS SOMEWHERE (MORE GENERALLY OPERAD IN SYMMETRIC MONOIDAL CATEGORY LIKE WARD), AND I MIGHT ALSO WRITE THE PROOF THAT THIS IS AN OPERAD (BUT SHOULD FOLLOW FROM LAX MONOIDALITY)

\item DIFFERENT PRODUCTS (HADAMARD, TENSOR PRODUCT OF S-MODULES, PLETHYSM)

\item COOPERADS, INFINITESIMAL COMPOSITION

\item LAX MONOIDALS CARRIES OPERADS TO OPERADS (SAME WITH COOPERADS? PROBABLY BY ABSTRACT NONSENSE OF OPPOSITE CATEGORY)

\item KOSZUL DUALITY, CONVOLUTION OPERAD, TWISTED MORPHISM?

\item INFINITY OPERADS? 

\item OPERADIC COHOMOLOGY OR JUST INTRODUCE COHOMOLOGY AD HOC FOR THE CASES I NEED?

\item BAR-COBAR  CONSTRUCTION FOR ALGEBRAS AND OPERADS?
\end{itemize}

-IN A DIFFERENT CHAPTER PROBABLY-
\begin{itemize}



\item A INFINITY ALGEBRAS CLASSICAL DEFINITION, BAR INTERPRETATION?, TWISTING MORPHISMS (L-V 10.1) REVIEW OF SOME KNOWN RESULTS. A-INFTY OPERAD AN ALGEBRAS AS MORPHISMS FROM THIS OPERAD (I CAN CONNECT IT TO THE COHOMOLOGY OF THE ASSOCIAHEDRA IF I EXPLAIN THAT) STRICT AND INFINITY MORPHIMS ACCORDING TO THE VARIOUS INTERPRETATIONS. DIFFERENCE BETWEEN INCLUDING THE DIFFERENTIAL OR NOT. TOPOLOGICAL ORIGIN? 

\item FILTERED MODULES, BIGRADED MODULES, TWISTED COMPLEXES, ENRICHMENTS FOR THESE CATEGORIES AS DONE BY SARAH

\item DERIVED A INFINITY ALGEBRAS, SOMETHING  FROM SAGAVE, UNIQUENESS OF AINFTY STRUTURES, DERIVED AINFTY IN OPERADIC CONTEXT, AND DERIVED AINFTY AND THEIR HOMOTOPIES. DIFFERENCE BETWEEN UNDERLYING TWISTED COMPLEXES OR NOT, MAIN THEOREM OF SARAH'S PAPER

\item FIND A BAR INTERPRETATION? (SINCE THIS WOULD BE KIND OF NEW, MAYBE IN A DIFFERENT CHAPTER) THIS IS ACTUALLY ON OPERADIC CONTEXT AND POSSIBLY EVEN IN SAGAVE, SO JUST STUDY IT AND INCLUDE IT JUST LIKE THE CLASSICAL CASE?
\item REFERENCE TO GEOMETRIC INTERPRETATIONS? (THESIS BY SARAH'S STUDENT)


\end{itemize}

\begin{propo}
Let $\OO$ be an operad in a symmetric monoidal catgory $\CC$. If $F:\CC\to\DD$ is a lax monoidal functor, then $F(\OO)$ has the sttructure of an operad in $\DD$.
\end{propo}
\begin{proof}
Let $\varepsilon: 1_\DD\to F(1_\CC)$ and $\mu\coloneqq\mu_{A,B}: F(A)\otimes F(B)\to F(A\otimes B)$ be the structure maps of the lax monoidal functor $F$. 

Let us first define the structure maps for the operad $F(\OO)$ in terms of insertions. Let $e:1_\CC\to\OO(1)$ be the unit of $\OO$. We define the unit $e_F:1_\DD\to F(\OO(1))$ as the composite
\[1_\DD\xrightarrow{\varepsilon} F(1)\xrightarrow{F(e)} F(\OO(1)).\]
Let $\circ_i :\OO(n)\otimes\OO(m)\to\OO(n+m-1)$ be the insertion map on $\OO$. We define the insertion map $\circ_i^F:F(\OO(n))\otimes F(\OO(m))\to F(\OO(n+m-1))$ as the composite
\[F(\OO(n))\otimes F(\OO(m))\xrightarrow{\mu} F(\OO(n)\otimes\OO(m))\xrightarrow{F(\circ_i)} F(\OO(n+m-1)).\]

We show now that $F(\OO)$ satisfies the unit axioms with the above structure maps. We only show the unit axiom with respect to the right unitor, the axiom with respect to the left unitor is analogous.


 Let $\lambda_\CC$ and $\lambda_\DD$ be the right unitors of $\CC$ and $\DD$ respectively. Since $\OO$ is an opererad, we have that the following diagram commutes.
% \[
% \begin{tikzcd}
% \OO(n)\otimes 1\arrow[r, "id\otimes e"]\arrow[rr, bend right = 20, "\lambda_\CC"] & \OO(n)\otimes \OO(1)\arrow[r, "\circ_i"]& \OO(n) 
% \end{tikzcd}
% \]
 
  \[
 \begin{tikzcd}
 \OO(n)\otimes 1_\CC\arrow[r, "\lambda_\CC"]\arrow[d, "id\otimes e"] &  \OO(n) \\
 \OO(n)\otimes \OO(1)\arrow[ur, bend right = 10, "\circ_i"]&
 \end{tikzcd}
 \]
 
Applying $F$ and introducing $\mu$ we get the following commutative diagram.
\begin{equation}\label{unitaux}
\begin{tikzcd}
F(\OO(n))\otimes F(1_\CC)\arrow[r,"\mu"]\arrow[d, "id\otimes F(e)"] & F(\OO(n)\otimes 1_\CC)\arrow[r, "F(\lambda_\CC)"]\arrow[d, "F(id\otimes e)"]&
F(\OO(n))\\
F(\OO(n))\otimes F(\OO(1))\arrow[r,"\mu"] & F(\OO(n)\otimes \OO(1))\arrow[ur, bend right = 15, "F(\circ_i)"']
\end{tikzcd}
\end{equation}

We need to show that the following diagram commutes.

  \[
 \begin{tikzcd}
F( \OO(n))\otimes 1_\DD\arrow[r, "\lambda_\DD"]\arrow[d, "id\otimes e_F"] &  F(\OO(n)) \\
 F(\OO(n))\otimes F(\OO(1))\arrow[ur, bend right = 10, "\circ_i^F"]&
 \end{tikzcd}
 \]
 
% Let us develop the diagram using the definition of the corresponding maps.
% 
% \[
%\begin{tikzcd}
%F( \OO(n))\otimes 1_\DD\arrow[r, "\lambda_\DD"]\arrow[d, "id\otimes F(e)\circ \varepsilon"] &  F(\OO(n)) \\
% F(\OO(n))\otimes F(\OO(1))\arrow[u, "\mu"] & \\
% F(\OO(
%\end{tikzcd} 
 %\]
 By monoidality of $F$ we know that $\lambda_\DD$ satisfies the following commutative diagram.
 
 \[
 \begin{tikzcd}
 F(\OO(n))\otimes 1_\DD\arrow[d, "\lambda_\DD"]\arrow[r, "id\otimes \varepsilon"] & F(\OO(n))\otimes F(1)\arrow[d,"\mu"]\\
 F(\OO(n)) & F(\OO(n)\otimes 1_\CC)\arrow[l, "F(\lambda)"]
 \end{tikzcd}
 \]
 Or, in other words, $\lambda_\DD = F(\lambda)\circ \mu(id\otimes \varepsilon)$.  On the other hand by diagram (\ref{unitaux}) we have that $F(\lambda)\circ \mu = F(\circ_i)\circ \mu(id\otimes F(e))$, meaning that
 \[\lambda_\DD =  F(\circ_i)\circ \mu(id\otimes F(e))\circ \mu(id\otimes \varepsilon) = \circ_i^F(id\otimes  e_F)\]
 as we wanted to show.
\end{proof}
%\phantomsection
\bibliographystyle{ieeetr}
\bibliography{newbibliography}
\end{document}
