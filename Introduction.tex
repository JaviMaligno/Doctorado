	\documentclass[twoside]{article}
\usepackage{estilo-ejercicios}
\setcounter{section}{0}
%\newtheorem{defin}{Definition}[section]
%\newtheorem{lem}[defin]{Lemma}
%\newtheorem{propo}[defin]{Proposition}
%\newtheorem{thm}[defin]{Theorem}
%\newtheorem{eje}[defin]{Example}
\renewcommand{\baselinestretch}{1,3}

\usepackage{empheq}
\newcommand*\widefbox[1]{\fbox{\hspace{2em}#1\hspace{2em}}}

%Below to introduce ¡ in mathmode https://tex.stackexchange.com/questions/471464/inverted-exclamation-mark-in-mathmode
\DeclareMathSymbol{\mathinvertedexclamationmark}{\mathclose}{operators}{'074}
\DeclareMathSymbol{\mathexclamationmark}{\mathclose}{operators}{'041}

\makeatletter
\newcommand{\raisedmathinvertedexclamationmark}{%
  \mathclose{\mathpalette\raised@mathinvertedexclamationmark\relax}%
}
\newcommand{\raised@mathinvertedexclamationmark}[2]{%
  \raisebox{\depth}{$\m@th#1\mathinvertedexclamationmark$}%
}
\begingroup\lccode`~=`! \lowercase{\endgroup
  \def~}{\@ifnextchar`{\raisedmathinvertedexclamationmark\@gobble}{\mathexclamationmark}}
\mathcode`!="8000
\makeatother
%--------------------------------------------------------
\begin{document}

\title{Introductory chapter}
\author{Javier Aguilar Martín}
\maketitle

\section{Introduction}

MAYBE THIS IS THE PLACE FOR FIXING R COMMUTATIVE WITH UNIT AND CHAR NOT 2 (POSSIBLY 0 OR NOT A FACTORIAL (representation theory things that I doubt I would need to clarify))
\begin{itemize}
\item MAYBE SOMETHING ABOUT GRADED / BIGRADED MODULES?

\item MAYBE BASIC REVIEW OF CATEGORY THEORY INCLUDING EQUIVALENCE OF CATEGORIES

FIND AN ACTUAL REFERENCE, I'M USING \url{https://math.ucr.edu/home/baez/qg-fall2004/definitions.pdf} BECAUSE IT'S PRACTICAL

FILL WITH SOME TEXT INTRODUCING WHAT WE ARE GOING TO DISCUSS
\subsection{Categories, Functors, Natural transformations}
IS THIS TOO ELEMENTARY TO BE INCLUDED?

THE ITEMS LOOK WEIRD BECAUSE AT THE MOMENT THHEY ARE INSIDE ANOTHER ITEMIZE, BUT THEY  WON'T BE IN THE END
\begin{defin}
A \emph{category} $\CC$ consist of
\begin{itemize}
\item a collection $\ob(\CC)$ of \emph{objects},
CHECK NOTATION FOR HOM SET TO BE CONSISTENT
\item for any pair of objects $x,y\in\ob(\CC)$, a set $\hom(x,y)$ of \emph{morphisms} (If $f\in\hom(x,y)$, we write $f:x\to y $)
\end{itemize}
equipped with
CHECK NOTATION FOR IDENTITY MORPHISM
\begin{itemize}
\item an \emph{identity morphism} $1_x: x → x$ for any object $x$,
\item for any pair of morphisms $f: x → y$ and $g: y → z$, a morphism $fg: x → z$ called the \emph{composite} or \emph{composition}
of $f$ and $g$, sometimes written as $f\circ g$,
\end{itemize}
such that
\begin{itemize}
\item for any morphism $f: x → y$, the \emph{left and right unit laws} hold: $1_xf = f = f1_y$,
\item for any triple of morphisms $f: w → x$, $g: x → y$, $h: y → z$, the \emph{associative law} holds:
$(fg)h = f(gh)$.
\end{itemize}
\end{defin}
We usually write $x ∈ \CC$ as an abbreviation for $x ∈ \ob(\CC)$. 

\begin{defin}
An \emph{isomorphism} is a morphism $f: x → y$
with an \emph{inverse}, i.e. a morphism $g: y → x$ such that $fg = 1_x$ and $gf = 1_y$.
\end{defin}


EXAMPLES


\begin{defin}
Given categories $\CC$, $\DD$, a \emph{functor} $F: \CC → \DD$ consists of
\begin{itemize}
\item a function $F: \ob(\CC) → \ob(\DD)$,
\item for any pair of objects $x, y ∈ \ob(\CC)$, a function $F: \hom(x, y) → \hom(F(x), F(y))$
\end{itemize}
such that
\begin{itemize}
\item $F$ preserves identities: for any object $x ∈ \ob(\CC)$, $F(1_x) = 1_{F (x)}$,
\item $F$ preserves composition: for any pair of morphisms $f: x → y$, $g: y → z$ in $\CC$, $F(fg) =
F(f)F(g)$.
\end{itemize}
\end{defin}

IDENTITY FUNCTOR, COMPOSITION OF FUNCTORS (CHECK IDENTITY AND ASSOCIATIVE LAW OR SAY IT IS NOT HARD), EXAMPLES

\begin{defin}
Given functors $F, G: \CC → \DD$, a \emph{natural transformation} $α: F ⇒ G$ consists of a function $α$ mapping each object $x ∈ \CC$ to a morphism $α_x: F(x) → G(x)$ such that for any morphism $f: x → y$ in $\CC$, the following diagram commutes
\[
\begin{tikzcd}
F(x)\arrow[r, "F(f)"]\arrow[d, "\alpha_x"] & F(y)\arrow[d, "\alpha_y"]\\
G(x)\arrow[r, "G(f)"] & G(y)
\end{tikzcd}
\]
\end{defin}

IDENTITY AND COMPOSITION OF NATURAL TRANSFORMATION, IDENTITY AND ASSOCIATIVE LAW


\begin{defin}
Given functors $F, G: \CC → \DD$, a \emph{natural isomorphism} $α: F ⇒ G$ is a natural
transformation that has an \emph{inverse}, i.e. a natural transformation $β: G ⇒ F$ such that $αβ = 1_F$ and
$βα = 1_G$.
\end{defin}

\begin{lem}
A natural transformation $α: F ⇒ G$ is a natural isomorphism if and only if for every
object $x ∈ \CC$, the morphism $α_x$ is invertible.
\end{lem}
\begin{proof}
PROVE OR FIND REFERENCE
\end{proof}

\begin{defin}
A functor $F: \CC → \DD$ is an equivalence if it has a \emph{weak inverse}, that is, a functor
$G: \DD → \CC$ such that there exist natural isomorphisms $α: FG ⇒ 1_\CC$, $β: GF ⇒ 1_\DD$.
\end{defin}

\begin{lem}
EQUIVALENCE IN TERMS OF FULLY FAITHFUL (FOR THAT I WILL NEED THE DEFINITIONS  AS WELL)
\end{lem}


\item SYMMETRIC (CLOSED) MONOIDAL CATEGORIES, EXAMPLES
\subsection{Monoidal, Braided and Symmetric Monoidal Categories}

\item LAX/STRICT MONOIDAL FUNCTORS

\item ENRICHED VERSION OF THE ABOVE THINGS, INCLUDING THE CONTENT IN SARAH'S PAPER


\item S-MODULES/COLLECTIONS IN THE NON SYMMETRRIC CASE, OPERADS (SYMMETRIC AND NS)- POSSIBLY SEVERAL DEFINITIONS, AT LEAST CLASSICAL, PARTIAL AND THEIR EQUIVALENCE AND MONOIDAL, EXAMPLES, (INFINITESIMAL SEE 10.2.4) MODULE OVER AN OPERAD, QUASI-FREE OPERAD, PRE-LIE ALGEBRA DEFINED BY INSERTIONS, FORGETFUL FUNCTOR FROM SYM TO NS? SHOULD WRITE THE AXIOMS FOR GRADED OPERADS SOMEWHERE (MORE GENERALLY OPERAD IN SYMMETRIC MONOIDAL CATEGORY LIKE WARD), AND I MIGHT ALSO WRITE THE PROOF THAT THIS IS AN OPERAD (BUT SHOULD FOLLOW FROM LAX MONOIDALITY)

\item  ALGEBRAS OVER OPERADS, MORPHISMS BETWEEN ALGEBRAS

\item DIFFERENT PRODUCTS (HADAMARD, TENSOR PRODUCT OF S-MODULES, PLETHYSM)

\item COOPERADS, INFINITESIMAL COMPOSITION

\item LAX MONOIDALS CARRIES OPERADS TO OPERADS (SAME WITH COOPERADS? PROBABLY BY ABSTRACT NONSENSE OF OPPOSITE CATEGORY)

\item KOSZUL DUALITY, CONVOLUTION OPERAD, TWISTED MORPHISM?

\item INFINITY OPERADS? 

\item OPERADIC COHOMOLOGY OR JUST INTRODUCE COHOMOLOGY AD HOC FOR THE CASES I NEED?

\item BAR-COBAR  CONSTRUCTION FOR ALGEBRAS AND OPERADS?
\end{itemize}

-IN A DIFFERENT CHAPTER PROBABLY-
\begin{itemize}



\item A INFINITY ALGEBRAS CLASSICAL DEFINITION, BAR INTERPRETATION?, TWISTING MORPHISMS (L-V 10.1) REVIEW OF SOME KNOWN RESULTS. A-INFTY OPERAD AN ALGEBRAS AS MORPHISMS FROM THIS OPERAD (I CAN CONNECT IT TO THE COHOMOLOGY OF THE ASSOCIAHEDRA IF I EXPLAIN THAT) STRICT AND INFINITY MORPHIMS ACCORDING TO THE VARIOUS INTERPRETATIONS. DIFFERENCE BETWEEN INCLUDING THE DIFFERENTIAL OR NOT. TOPOLOGICAL ORIGIN? 

\item FILTERED MODULES, BIGRADED MODULES, TWISTED COMPLEXES, ENRICHMENTS FOR THESE CATEGORIES AS DONE BY SARAH

\item DERIVED A INFINITY ALGEBRAS, SOMETHING  FROM SAGAVE, UNIQUENESS OF AINFTY STRUTURES, DERIVED AINFTY IN OPERADIC CONTEXT, AND DERIVED AINFTY AND THEIR HOMOTOPIES. DIFFERENCE BETWEEN UNDERLYING TWISTED COMPLEXES OR NOT, MAIN THEOREM OF SARAH'S PAPER

\item FIND A BAR INTERPRETATION? (SINCE THIS WOULD BE KIND OF NEW, MAYBE IN A DIFFERENT CHAPTER) THIS IS ACTUALLY ON OPERADIC CONTEXT AND POSSIBLY EVEN IN SAGAVE, SO JUST STUDY IT AND INCLUDE IT JUST LIKE THE CLASSICAL CASE?
\item REFERENCE TO GEOMETRIC INTERPRETATIONS? (THESIS BY SARAH'S STUDENT)


\end{itemize}

\begin{propo}
Any symmetric lax monoidal functor $F:\CC\to\DD$ induces a functor $\underline{F}:\mathrm{Op}_\CC\to \mathrm{Op}_\DD$ between the categories of operads in $\CC$ and $\DD$, respectively.
\end{propo}

REFERENCE THE AXIOMS WHEN I WRITE THEM DOWN
\begin{proof}
Let $\OO$ be an operad in $\CC$ and let $F:\CC\to\DD$ be a symmetric lax monoidal functor. On objects, we define $\underline{F}(\OO)= F(\OO)$ and on morphisms we define $\underline{F}(f)=F(f)$ for $f:\OO\to\PP$. 

Let $\varepsilon: 1_\DD\to F(1_\CC)$ and $\mu\coloneqq\mu_{A,B}: F(A)\otimes F(B)\to F(A\otimes B)$ be the structure maps of the lax monoidal functor $F$. 

Let us first define the structure maps for the operad $F(\OO)$ in terms of insertions. Let $e:1_\CC\to\OO(1)$ be the unit of $\OO$. We define the unit $e_F:1_\DD\to F(\OO(1))$ as the composite
\[1_\DD\xrightarrow{\varepsilon} F(1)\xrightarrow{F(e)} F(\OO(1)).\]
Let $\circ_i :\OO(n)\otimes\OO(m)\to\OO(n+m-1)$ be the insertion map on $\OO$. We define the insertion map $\circ_i^F:F(\OO(n))\otimes F(\OO(m))\to F(\OO(n+m-1))$ as the composite
\[F(\OO(n))\otimes F(\OO(m))\xrightarrow{\mu} F(\OO(n)\otimes\OO(m))\xrightarrow{F(\circ_i)} F(\OO(n+m-1)).\]

We show now that $F(\OO)$ satisfies the unit axioms with the above structure maps. We only show the unit axiom with respect to the right unitor, the axiom with respect to the left unitor is analogous.


 Let $\lambda_\CC$ and $\lambda_\DD$ be the right unitors of $\CC$ and $\DD$ respectively. Since $\OO$ is an opererad, we have that the following diagram commutes.
% \[
% \begin{tikzcd}
% \OO(n)\otimes 1\arrow[r, "id\otimes e"]\arrow[rr, bend right = 20, "\lambda_\CC"] & \OO(n)\otimes \OO(1)\arrow[r, "\circ_i"]& \OO(n) 
% \end{tikzcd}
% \]
 
  \[
 \begin{tikzcd}
 \OO(n)\otimes 1_\CC\arrow[r, "\lambda_\CC"]\arrow[d, "id\otimes e"] &  \OO(n) \\
 \OO(n)\otimes \OO(1)\arrow[ur, bend right = 10, "\circ_i"]&
 \end{tikzcd}
 \]
 
Applying $F$ and introducing $\mu$ we get the following commutative diagram.
\begin{equation}\label{unitaux}
\begin{tikzcd}
F(\OO(n))\otimes F(1_\CC)\arrow[r,"\mu"]\arrow[d, "id\otimes F(e)"] & F(\OO(n)\otimes 1_\CC)\arrow[r, "F(\lambda_\CC)"]\arrow[d, "F(id\otimes e)"]&
F(\OO(n))\\
F(\OO(n))\otimes F(\OO(1))\arrow[r,"\mu"] & F(\OO(n)\otimes \OO(1))\arrow[ur, bend right = 15, "F(\circ_i)"']
\end{tikzcd}
\end{equation}

We need to show that the following diagram commutes.

  \[
 \begin{tikzcd}
F( \OO(n))\otimes 1_\DD\arrow[r, "\lambda_\DD"]\arrow[d, "id\otimes e_F"] &  F(\OO(n)) \\
 F(\OO(n))\otimes F(\OO(1))\arrow[ur, bend right = 10, "\circ_i^F"]&
 \end{tikzcd}
 \]
 
% Let us develop the diagram using the definition of the corresponding maps.
% 
% \[
%\begin{tikzcd}
%F( \OO(n))\otimes 1_\DD\arrow[r, "\lambda_\DD"]\arrow[d, "id\otimes F(e)\circ \varepsilon"] &  F(\OO(n)) \\
% F(\OO(n))\otimes F(\OO(1))\arrow[u, "\mu"] & \\
% F(\OO(
%\end{tikzcd} 
 %\]
 By monoidality of $F$ we know that $\lambda_\DD$ satisfies the following commutative diagram.
 
 \[
 \begin{tikzcd}
 F(\OO(n))\otimes 1_\DD\arrow[d, "\lambda_\DD"]\arrow[r, "id\otimes \varepsilon"] & F(\OO(n))\otimes F(1)\arrow[d,"\mu"]\\
 F(\OO(n)) & F(\OO(n)\otimes 1_\CC)\arrow[l, "F(\lambda)"]
 \end{tikzcd}
 \]
 Or, in other words, $\lambda_\DD = F(\lambda)\circ \mu(id\otimes \varepsilon)$.  On the other hand by diagram (\ref{unitaux}) we have that $F(\lambda)\circ \mu = (F(\circ_i)\circ \mu)(id\otimes F(e))$, meaning that
 \[\lambda_\DD =  F(\circ_i)\circ \mu\circ (id\otimes F(e))\circ (id\otimes \varepsilon) = \circ_i^F(id\otimes  e_F)\]
 as we wanted to show.
 
 Next we need to show that the associativity axioms of operads hold for $F(\OO)$. Let us first prove the one that does not involve the symmetry isomorphism. 
 
 Let $a_\CC$ and $a_\DD$ the associators for $\CC$ and $\DD$, respectively. For $i\leq j\leq i+m-1$ we have the following commutative diagram.
 
 \[
\begin{tikzcd}
(\OO(n)\otimes \OO(m))\otimes \OO(l) \arrow[r, "a_\CC"]\arrow[d, "\circ_i\otimes"] & \OO(n)\otimes (\OO(m)\otimes\OO(l))\arrow[d, "id\otimes\circ_{j-i+1}"]\\
\OO(n+m-1)\otimes\OO(l)\arrow[d, "\circ_j"] & \OO(n)\otimes\OO(m+l-1)\arrow[dl, bend left = 10, "\circ_i"]\\
\OO(n+m+l-2) & 
\end{tikzcd} 
 \]
 
 Applying $F$ we obtain the following commutative diagram.
  \begin{equation}\label{assaux}
\begin{tikzcd}
F((\OO(n)\otimes \OO(m))\otimes \OO(l)) \arrow[r, "F(a_\CC)"]\arrow[d, "F(\circ_i\otimes id)"] & F(\OO(n)\otimes (\OO(m)\otimes\OO(l)))\arrow[d, "F(id\otimes\circ_{j-i+1})"]\\
F(\OO(n+m-1)\otimes\OO(l))\arrow[d, "F(\circ_j)"] & F(\OO(n)\otimes\OO(m+l-1))\arrow[dl, bend left = 10, "F(\circ_i)"]\\
F(\OO(n+m+l-2)) & 
\end{tikzcd} 
   \end{equation}
   
   According to the definition of $\circ_i^F$, we need to show that the following diagram commutes. 
   
   \[   
\begin{tikzcd}
(F(\OO(n))\otimes F(\OO(m)))\otimes F(\OO(l)) \arrow[r, "a_\DD"] \arrow[d, "\mu\otimes id"] & F(\OO(n))\otimes (F(\OO(m))\otimes F(\OO(l)))\arrow[d, "id\otimes \mu"]\\
F(\OO(n)\otimes \OO(m))\otimes F(\OO(l))\arrow[d, red, "F(\circ_i)\otimes id"] & F(\OO(n))\otimes F(\OO(m)\otimes\OO(l))\arrow[d, red, "id\otimes F(\circ_{j+i-1})"]\\
F(\OO(n+m-1))\otimes F(\OO(l))\arrow[d, red, "\mu"] & F(\OO(n))\otimes F(\OO(m+l-1))\arrow[d, red, "\mu"]\\
F(\OO(n+m-1)\otimes\OO(l))\arrow[d, "F(\circ_j)"] & F(\OO(n)\otimes\OO(m+l-1))\arrow[dl, bend left = 10, "F(\circ_i)"]\\
F(\OO(n+m+l-2)) & 
\end{tikzcd}   
   \]
   
   By naturality of $\mu$ we have 
   \begin{equation}\label{naturality}
   \mu\circ (F(\circ_i)\otimes id)= F( \circ_i \otimes id)\circ \mu 
   \end{equation} and \[\mu\circ (id \otimes F(\circ_{j-i+1}))= F(id\otimes \circ_{j-i+1})\circ \mu.\]
    Therefore we can replace the red arrows accordingly. We can also subdivide the above diagram into two by using $F(a_\CC)$ as follows.
   
      \[   
\begin{tikzcd}
(F(\OO(n))\otimes F(\OO(m)))\otimes F(\OO(l)) \arrow[r, "a_\DD"] \arrow[d, "\mu\otimes id"] & F(\OO(n))\otimes (F(\OO(m))\otimes F(\OO(l)))\arrow[d, "id\otimes \mu"]\\
F(\OO(n)\otimes \OO(m))\otimes F(\OO(l))\arrow[d, red, "\mu"] & F(\OO(n))\otimes F(\OO(m)\otimes\OO(l))\arrow[d, red, "\mu"]\\
F((\OO(n)\otimes\OO(m))\otimes\OO(l))\arrow[d, red, "F(\circ_i\otimes id)"]\arrow[r, dashed, "F(a_\DD)"] & F(\OO(n))\otimes F(\OO(n)\otimes(\OO(m)\otimes\OO(l)))\arrow[d, red, "F(id\otimes \circ_{j-1+1})"]\\
F(\OO(n+m-1)\otimes\OO(l))\arrow[d, "F(\circ_j)"] & F(\OO(n)\otimes\OO(m+l-1))\arrow[dl, bend left = 10, "F(\circ_i)"]\\
F(\OO(n+m+l-2)) & 
\end{tikzcd}   
   \]
   Now, the top diagram commutes because it is the associativity axiom of lax monoidal functors. The bottom diagram is precisely diagram (\ref{assaux}), so it commutes and we get the desired associativity axiom.
   
   Finally, we need to show that the associativity axioms involving the symmetry isomorphism hold for $F(\OO)$. Since they are analogous to each other, we only prove the first one.
   
   Let $B_\CC\coloneqq B_\CC^{X,Y}:X\otimes Y\to Y\otimes X$ the symmetry isomorphism on $\CC$ and similarly denote by $B_\DD$ the symmetry isomorphism on $\DD$.
   
   We have the following associativity commutative diagram for $j<i$.
   \[
   \begin{tikzcd}
   (\OO(n)\otimes \OO(m))\otimes\OO(l)\arrow[r, "a_\CC"]\arrow[d, "\circ_i\otimes id"] & \OO(n)\otimes (\OO(m)\otimes \OO(l)) \arrow[r, "id\otimes B_\CC"] & \OO(n)\otimes (\OO(l)\otimes \OO(m))\arrow[d, "a^{-1}_\CC"]\\
   \OO(n+m-1)\otimes\OO(l)\arrow[d, "\circ_j"] & & (\OO(n)\otimes\OO(l))\otimes \OO(m)\arrow[d, "\circ_j\otimes id"]\\
   \OO(n+m+l-2) & & \OO(n+l-1)\otimes \OO(m)\arrow[ll, "\circ_i"]
   \end{tikzcd}
   \]
   
   Applying $F$ we get the following commutative diagram.
   \begin{equation}\label{assiaux}
      \begin{tikzcd}
   F((\OO(n)\otimes \OO(m))\otimes\OO(l))\arrow[r, "F(a_\CC)"]\arrow[d, "F(\circ_i\otimes id)"] & F(\OO(n)\otimes (\OO(m)\otimes \OO(l))) \arrow[r, "F(id\otimes B^\CC)"] & \OO(n)\otimes (\OO(l)\otimes \OO(m))\arrow[d, "F(a_\CC)^{-1}"]\\
  F(\OO(n+m-1)\otimes\OO(l))\arrow[d, "F(\circ_j)"] & & F((\OO(n)\otimes\OO(l))\otimes \OO(m))\arrow[d, "F(\circ_j\otimes id)"]\\
   F(\OO(n+m+l-2)) & &F(\OO(n+l-1)\otimes \OO(m))\arrow[ll, "F(\circ_i)"]
   \end{tikzcd}
   \end{equation}
   
   We need to show that the following diagram commutes.
   \[
\begin{tikzcd}[column sep = -3em]
& F(\OO(n))\otimes (F(\OO(m))\otimes F(\OO(l)))\arrow[dr, "id\otimes B_\DD"]& \\
(F(\OO(n))\otimes F(\OO(m)))\otimes F(\OO(l))\arrow[d, "\mu\otimes id"]\arrow[ur, "a_\DD"] &  & F(\OO(n))\otimes (F(\OO(l))\otimes F(\OO(m)))\arrow[d, "a_\DD^{-1}"]\\
F(\OO(n)\otimes\OO(n))\otimes F(\OO(l))\arrow[d, "F(\circ_i)\otimes id"] & & (F(\OO(n))\otimes F(\OO(l)))\otimes F(\OO(m))\arrow[d, "\mu\otimes id"]\\
F(\OO(n+m-1)\otimes F(\OO(l))\arrow[d, "\mu\otimes id"] & & F(\OO(n)\otimes \OO(l))\otimes F(\OO(m))\arrow[d, "F(\circ_j)\otimes id"]\\
F(\OO(m+n-1)\otimes\OO(l))\arrow[d, "F(\circ_j)"] & & F(\OO(n+l-1))\otimes F(\OO(m))\arrow[d, "\mu"]\\
F((\OO(n+m+l-2)) & & F(\OO(n+l-1)\otimes\OO(m))\arrow[ll, "F(\circ_i)"]
\end{tikzcd}   
   \]
   
   We use naturality of $\mu$ (\Cref{naturality}) as we have done before to rewrite some of the arrows. We also subdivide the diagram into two by factoring by $F(a_\CC)^{-1}\circ F(id\otimes B_\CC)\circ F(a_\CC)$.
   
   \[
\begin{tikzcd}[column sep = -3em]
& F(\OO(n))\otimes (F(\OO(m))\otimes F(\OO(l)))\arrow[dr, "id\otimes B_\DD"]& \\
(F(\OO(n))\otimes F(\OO(m)))\otimes F(\OO(l))\arrow[d, "\mu\otimes id"]\arrow[ur, "a_\DD"] &  & F(\OO(n))\otimes (F(\OO(l))\otimes F(\OO(m)))\arrow[d, "a_\DD^{-1}"]\\
F(\OO(n)\otimes\OO(n))\otimes F(\OO(l))\arrow[d, "\mu"] & & (F(\OO(n))\otimes F(\OO(l)))\otimes F(\OO(m))\arrow[d, "\mu\otimes id"]\\
F((\OO(n)\otimes\OO(m))\otimes \OO(l))\arrow[d, "F(\circ_i\otimes id)"]\arrow[rrd, dashed, "F(a_\CC)^{-1}\circ F(id\otimes B_\CC)\circ F(a_\CC)", sloped] & & F(\OO(n)\otimes \OO(l))\otimes F(\OO(m))\arrow[d, "\mu"]\\
F(\OO(m+n-1)\otimes\OO(l))\arrow[d, "F(\circ_j)"] & & F((\OO(n)\otimes\OO(m))\otimes \OO(m))\arrow[d, "F(\circ_j\otimes id)"]\\
F((\OO(n+m+l-2)) & & F(\OO(n+l-1)\otimes\OO(m))\arrow[ll, "F(\circ_i)"]
\end{tikzcd}   
   \]
   
   The bottom diagram commutes as it is precisely diagram (\ref{assiaux}). We decompose the top diagram as follows.
   
   \[
\begin{tikzcd}[column sep = -3em]
 & F(\OO(n))\otimes (F(\OO(m))\otimes F(\OO(l)))\arrow[dr, "id\otimes B_\DD"]\arrow[ddddl, dashed, bend left = 30, sloped, "\mu\circ (id\otimes\mu)"]& \\
(F(\OO(n))\otimes F(\OO(m)))\otimes F(\OO(l))\arrow[d, "\mu\otimes id"]\arrow[ur, "a_\DD"] &  & F(\OO(n))\otimes (F(\OO(l))\otimes F(\OO(m)))\arrow[d, "a_\DD^{-1}"]\arrow[ddddl, dashed, bend right = 35, sloped, "\mu\circ (id\otimes\mu)"]\\
F(\OO(n)\otimes\OO(n))\otimes F(\OO(l))\arrow[d, "\mu"] & & (F(\OO(n))\otimes F(\OO(l)))\otimes F(\OO(m))\arrow[d, "\mu\otimes id"]\\
F((\OO(n)\otimes\OO(m))\otimes \OO(l))\arrow[d, "F(a)"] & & F(\OO(n)\otimes \OO(l))\otimes F(\OO(m))\arrow[d, "\mu"]\\
F(\OO(n)\otimes (\OO(m)\otimes\OO(l)))\arrow[dr,"F(id\otimes B_\CC)"] & & F((\OO(n)\otimes\OO(l))\otimes\OO(m))\\
& F(\OO(n)\otimes (\OO(l)\otimes\OO(m)))\arrow[ur, "F(a_\CC)^{-1}"]&
\end{tikzcd}  
   \]
   
   The left and right subdiagrams commute because of the associativity axiom of lax monoidal functors. We decompose the central subdiagram further as
   \[
\begin{tikzcd}[column sep = 5em]
F(\OO(n))\otimes (F(\OO(m))\otimes F(\OO(l))\arrow[r, "id\otimes B_\DD"]\arrow[d, "id\otimes \mu"] & F(\OO(n))\otimes (F(\OO(l))\otimes F(\OO(m)))\arrow[d, "id\otimes \mu"]\\
F(\OO(n))\otimes F(\OO(m)\otimes\OO(l))\arrow[r, dashed,  "id\otimes F(B_\CC)"]\arrow[d, "\mu"] & F(\OO(n))\otimes F(\OO(l)\otimes\OO(m))\arrow[d, "\mu"]\\
F(\OO(n)\otimes (\OO(m)\otimes\OO(l)))\arrow[r, "F(id\otimes B_\CC)"] & F(\OO(n)\otimes (\OO(l)\otimes \OO(m)))
\end{tikzcd}   
   \]
   
   The top part commutes because $F$ is symmetric lax monoidal and the bottom part commutes by naturality of $\mu$. This proves that $F(\OO)$ is an operad in $\DD$. 
   
   Lastly, we are only left with the proof that $F(f)$ is a map of operads. Since $f$ is a map of operads, we have for all $n$ the following commutative diagram.
   \[
\begin{tikzcd}[column sep = 5em]
\OO(n)\otimes \OO(m)\arrow[r, "f_n\otimes f_m"]\arrow[d, "\circ_i^\OO"] & \PP(n)\otimes\PP(m)\arrow[d, "\circ_i^\PP"]\\
\OO(n+m-1)\arrow[r, "f_{n+m-1}"] & \PP(n+m-1)
\end{tikzcd}   
   \]
   After applying $F$ we get the following commutative diagram.
   \begin{equation}\label{mapaux}
   \begin{tikzcd}[column sep = 5em]
F(\OO(n)\otimes \OO(m))\arrow[r, "F(f_n\otimes f_m)"]\arrow[d, "F(\circ_i^\OO)"] & F(\PP(n)\otimes\PP(m))\arrow[d, "F(\circ_i^\PP)"]\\
F(\OO(n+m-1))\arrow[r, "F(f_{n+m-1})"] & F(\PP(n+m-1))
\end{tikzcd} 
   \end{equation}
   
   We need to show that the following diagram commutes.
   \[
   \begin{tikzcd}[column sep = 5em]
F(\OO(n))\otimes F(\OO(m))\arrow[r, "F(f_n)\otimes F(f_m)"]\arrow[d,"\mu"] & F(\PP(n))\otimes F(\PP(m))\arrow[d, "\mu"]\\
F(\OO(n)\otimes \OO(m))\arrow[r, dashed, "F(f_n\otimes f_m)"]\arrow[d, "F(\circ_i^\OO)"] & F(\PP(n)\otimes\PP(m))\arrow[d, "F(\circ_i^\PP)"]\\
F(\OO(n+m-1))\arrow[r, "F(f_{n+m-1})"] & F(\PP(n+m-1))
\end{tikzcd}    
   \]
   The top subdiagram commutes because $\mu$ is natural and the bottom part is precisely diagram  (\ref{mapaux}), which commutes. This finishes the proof.
\end{proof}
%\phantomsection
\bibliographystyle{ieeetr}
\bibliography{newbibliography}
\end{document}
