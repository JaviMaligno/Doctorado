	\documentclass[twoside]{article}
\usepackage{estilo-ejercicios}
\setcounter{section}{0}
%\newtheorem{defin}{Definition}[section]
%\newtheorem{lem}[defin]{Lemma}
%\newtheorem{propo}[defin]{Proposition}
%\newtheorem{thm}[defin]{Theorem}
%\newtheorem{eje}[defin]{Example}
\renewcommand{\baselinestretch}{1,3}

\usepackage{empheq}
\newcommand*\widefbox[1]{\fbox{\hspace{2em}#1\hspace{2em}}}

%Below to introduce ¡ in mathmode https://tex.stackexchange.com/questions/471464/inverted-exclamation-mark-in-mathmode
\DeclareMathSymbol{\mathinvertedexclamationmark}{\mathclose}{operators}{'074}
\DeclareMathSymbol{\mathexclamationmark}{\mathclose}{operators}{'041}

\makeatletter
\newcommand{\raisedmathinvertedexclamationmark}{%
  \mathclose{\mathpalette\raised@mathinvertedexclamationmark\relax}%
}
\newcommand{\raised@mathinvertedexclamationmark}[2]{%
  \raisebox{\depth}{$\m@th#1\mathinvertedexclamationmark$}%
}
\begingroup\lccode`~=`! \lowercase{\endgroup
  \def~}{\@ifnextchar`{\raisedmathinvertedexclamationmark\@gobble}{\mathexclamationmark}}
\mathcode`!="8000
\makeatother
%--------------------------------------------------------
\begin{document}

\title{Introductory chapter}
\author{Javier Aguilar Martín}
\maketitle

\section{Introduction}

MAYBE THIS IS THE PLACE FOR FIXING R COMMUTATIVE WITH UNIT AND CHAR NOT 2 (POSSIBLY 0 OR NOT A FACTORIAL (representation theory things that I doubt I would need to clarify))
%\begin{itemize}
%\item MAYBE SOMETHING ABOUT GRADED / BIGRADED MODULES?

%\item MAYBE BASIC REVIEW OF CATEGORY THEORY INCLUDING EQUIVALENCE OF CATEGORIES

%FIND AN ACTUAL REFERENCE, I'M USING \url{https://math.ucr.edu/home/baez/qg-fall2004/definitions.pdf} BECAUSE IT'S PRACTICAL

%FILL WITH SOME TEXT INTRODUCING WHAT WE ARE GOING TO DISCUSS
%\subsection{Categories, Functors, Natural transformations}
%IS THIS TOO ELEMENTARY TO BE INCLUDED?
%
%THE ITEMS LOOK WEIRD BECAUSE AT THE MOMENT THHEY ARE INSIDE ANOTHER ITEMIZE, BUT THEY  WON'T BE IN THE END
%\begin{defin}
%A \emph{category} $\CC$ consist of
%\begin{itemize}
%\item a collection $\ob(\CC)$ of \emph{objects},
%CHECK NOTATION FOR HOM SET TO BE CONSISTENT
%\item for any pair of objects $x,y\in\ob(\CC)$, a set $\hom(x,y)$ of \emph{morphisms} (If $f\in\hom(x,y)$, we write $f:x\to y $)
%\end{itemize}
%equipped with
%CHECK NOTATION FOR IDENTITY MORPHISM
%\begin{itemize}
%\item an \emph{identity morphism} $1_x: x → x$ for any object $x$,
%\item for any pair of morphisms $f: x → y$ and $g: y → z$, a morphism $fg: x → z$ called the \emph{composite} or \emph{composition}
%of $f$ and $g$, sometimes written as $f\circ g$,
%\end{itemize}
%such that
%\begin{itemize}
%\item for any morphism $f: x → y$, the \emph{left and right unit laws} hold: $1_xf = f = f1_y$,
%\item for any triple of morphisms $f: w → x$, $g: x → y$, $h: y → z$, the \emph{associative law} holds:
%$(fg)h = f(gh)$.
%\end{itemize}
%\end{defin}
%We usually write $x ∈ \CC$ as an abbreviation for $x ∈ \ob(\CC)$. 
%
%\begin{defin}
%An \emph{isomorphism} is a morphism $f: x → y$
%with an \emph{inverse}, i.e. a morphism $g: y → x$ such that $fg = 1_x$ and $gf = 1_y$.
%\end{defin}
%
%
%EXAMPLES
%
%
%\begin{defin}
%Given categories $\CC$, $\DD$, a \emph{functor} $F: \CC → \DD$ consists of
%\begin{itemize}
%\item a function $F: \ob(\CC) → \ob(\DD)$,
%\item for any pair of objects $x, y ∈ \ob(\CC)$, a function $F: \hom(x, y) → \hom(F(x), F(y))$
%\end{itemize}
%such that
%\begin{itemize}
%\item $F$ preserves identities: for any object $x ∈ \ob(\CC)$, $F(1_x) = 1_{F (x)}$,
%\item $F$ preserves composition: for any pair of morphisms $f: x → y$, $g: y → z$ in $\CC$, $F(fg) =
%F(f)F(g)$.
%\end{itemize}
%\end{defin}
%
%IDENTITY FUNCTOR, COMPOSITION OF FUNCTORS (CHECK IDENTITY AND ASSOCIATIVE LAW OR SAY IT IS NOT HARD), EXAMPLES
%
%\begin{defin}
%Given functors $F, G: \CC → \DD$, a \emph{natural transformation} $α: F ⇒ G$ consists of a function $α$ mapping each object $x ∈ \CC$ to a morphism $α_x: F(x) → G(x)$ such that for any morphism $f: x → y$ in $\CC$, the following diagram commutes
%\[
%\begin{tikzcd}
%F(x)\arrow[r, "F(f)"]\arrow[d, "\alpha_x"] & F(y)\arrow[d, "\alpha_y"]\\
%G(x)\arrow[r, "G(f)"] & G(y)
%\end{tikzcd}
%\]
%\end{defin}
%
%IDENTITY AND COMPOSITION OF NATURAL TRANSFORMATION, IDENTITY AND ASSOCIATIVE LAW
%
%
%\begin{defin}
%Given functors $F, G: \CC → \DD$, a \emph{natural isomorphism} $α: F ⇒ G$ is a natural
%transformation that has an \emph{inverse}, i.e. a natural transformation $β: G ⇒ F$ such that $αβ = 1_F$ and
%$βα = 1_G$.
%\end{defin}
%
%\begin{lem}
%A natural transformation $α: F ⇒ G$ is a natural isomorphism if and only if for every
%object $x ∈ \CC$, the morphism $α_x$ is invertible.
%\end{lem}
%\begin{proof}
%PROVE OR FIND REFERENCE
%\end{proof}
%
%\begin{defin}
%A functor $F: \CC → \DD$ is an equivalence if it has a \emph{weak inverse}, that is, a functor
%$G: \DD → \CC$ such that there exist natural isomorphisms $α: FG ⇒ 1_\CC$, $β: GF ⇒ 1_\DD$.
%\end{defin}
%
%\begin{lem}
%EQUIVALENCE IN TERMS OF FULLY FAITHFUL (FOR THAT I WILL NEED THE DEFINITIONS  AS WELL)
%\end{lem}
%
%
%\item SYMMETRIC (CLOSED) MONOIDAL CATEGORIES, EXAMPLES
\subsection{Symmetric Monoidal Categories}

We assume that the reader is familiar with the basic terminology of category theory. For an introduction to this topic we refer the reader to \cite{maclane}. Here we briefly recall the notion of symmetric monoidal categories and several versions of monoidal functors. The detailed definitions with all the precise diagrams can also be found in \cite{maclane} and in \cite{borceux}.

\begin{defin}
A \emph{symmetric monoidal category} is a category $\CC$ equipped with a functor 
\[\otimes:\CC\times\CC\to\CC\]
called \emph{tensor product}, an object
\[1\in\CC\]
called \emph{unit object}, a natural isomorphism
\[a_{A,B,C} : (A \otimes B) \otimes C \to A \otimes (B \otimes C)\]
for all objects $A,B,C\in\CC$ called \emph{associators}, a natural isomorphism
\[\lambda_A:1\otimes A\to A\]
for every $A\in\CC$ called \emph{left unitor}, a natural isomorphism
\[\rho_A:A\otimes 1\to A\]
for every $A\in\CC$ called \emph{right unitor}, and a natural isomorphism
\[\tau_{A,B}:A\otimes B \to B\otimes A\]
for all $A,B\in\CC$ called \emph{braiding} or \emph{symmetry isomorphism}. These morphisms satisfy natural unitality and associativity axioms.
\end{defin}

\begin{remark}
If we drop the symmetry isomorphism we get what is simply called a \emph{monoidal category}.
\end{remark}

\begin{defin}
Let $(\CC,\otimes_\CC, 1_\CC)$ and $(\DD, \otimes_\DD, 1_\DD)$ be symmetric monoidal categories. A \emph{(lax) monoidal functor} between them is a functor
\[F:\CC\to\DD\]
with a morphism
\[\varepsilon:1_\DD\to F(1_\CC)\]
and a natural transformation
\[\mu_{A,B}:F(A)\otimes_\DD  F(B)\to F(A\otimes_\CC B)\]
for all $A,B\in\CC$ satisfying natural unitality, associativity and symmetry axioms. A lax monoidal functor is called \emph{strong monoidal} if $\varepsilon$ and $\mu_{A,B}$ are isomorphisms for all $A,B\in\CC$.
\end{defin}


\begin{defin}
Suppose $(F,\mu,\varepsilon)$ and $(F, \nu, \epsilon)$ are monoidal functors between the symmetric monoidal categories $\CC$ and $\DD$. Then a natural transformation $\alpha:F\to G$ is \emph{monoidal} if the diagrams
\[
\begin{tikzcd}
F(A)\otimes_\DD F(B)\arrow[r, "\alpha_A\otimes_\DD\alpha_B"]\arrow[d,"\mu_{A,B}"] & G(A)\otimes_\DD G(B)\arrow[d, "\nu_{A,B}"]\\
F(A\otimes_\CC B)\arrow[r, "\alpha_{A\otimes_\CC B}"] & G(A\otimes_\DD B)
\end{tikzcd}
\]

and
\[
\begin{tikzcd}
1_\DD\arrow[d, "\varepsilon"]\arrow[dr, "\epsilon"]&\\
F(1)\arrow[r, "\alpha_1"] &  G(1)
\end{tikzcd}
\]
commute.
\end{defin}

\begin{defin}
If $\CC$ and $\DD$ are symmetric monoidal categories, a lax monoidal functor $F: \CC → \DD$ is an \emph{monoidal
equivalence} if there is a lax monoidal functor $G: \DD → \CC$ such that there exist monoidal natural isomorphisms $α: FG ⇒ id_\CC$, $β: GF ⇒ id_\DD$.
\end{defin}

\begin{defin}
A symmetric monoidal category $\CC$ is \emph{closed} if for every object $A\in\CC$ the tensor product functor $A\otimes -:\CC\to\CC$ has a right adjoint functor $[A,-]:\CC\to\CC$. In other words, for all $A,B,C\in \CC$ we have a natural bijection between the hom-sets
\[\Hom_\CC(A\otimes B, C)\cong \Hom_\CC(A, [B,C])\]
natural in all arguments. The object $[A,B]$ is called the \emph{internal hom}.
\end{defin}


\begin{defin}
Let $\VV$ a monoidal category. A $\VV$-\emph{category} $\CC$ ($\VV$-\emph{
enrihed catgory} or \emph{category enriched over} $\VV$) consists of
\begin{itemize}
\item a set $\mathrm{Ob}(\CC)$ of objects in $\CC$,
\item for each pair $(A,B)$ of objects in $\CC$ an object $\CC(A,B)\in\VV$ called the \emph{hom-object} or \emph{object of morphisms} from $A$ to $B$,
\item for every triple $(A,B,C)$ of objects in $\CC$ a morphism
\[\circ_{A,B,C}:\CC(B,C)\otimes\CC(A,B)\to \CC(A,C)\]
in $\VV$ called \emph{composition morphism},
\item and for each object $A$ in $\CC$ a morphism $u_A:1\to \CC(A,A)$ called the \emph{identity element}. 
\end{itemize}
All this data is subject to associativity and unitality constrains that can be seen in detail in \cite{borceux}.
\end{defin}

THE CONTENT OF SARAH'S PAPER COULD COME HERE (BUT IT IS SPECIFIC ENOUGH TO BE WRITTEN BEFORE THE STUFF IT IS MADE FOR)

\subsection{Operads}
A BIT MORE OF WHAT IS IN JOIN, LIKE COOPERADS (THIS SHOULD ACTUALLY BE ADDED THERE) AND SOME OF THE STUFF IN THE ITEM BELOW (CHECK WHAT IS NECESSARY) 

SAY SOMETHING TO INTRODUCE IT AND MENTION THE MAIN REFERENCES
We start defining the underlying object of an operad.

\begin{defin}\label{collections}
A \emph{collection} is a family $\OO=\{\OO(n)\}_{n\geq 0}$ of graded $R$-modules. We call the integer $n$ the \emph{arity}. When there is an action of the symmetric group $\Sigma_n$ on each $\OO(n)$ we say that the collection is an $\mathbb{S}-$module. A map of collections $f:\OO\to\mathcal{P}$ is a family of maps $f_n:\OO(n)\to\mathcal{P}(n)$. A map of collection is a map of $\mathbb{S}-$modules when it preserves the symmetric group action.
\end{defin}

\begin{defin}
The \emph{plethysm} or \emph{composite} $\OO\circ\PP$ of two collections $\OO$ and $\PP$ given by
\[(\OO\circ\PP)(n)=\bigoplus_{N\geq 0}\OO(N)\otimes \left(\bigoplus_{a_1+\cdots+a_k=n} \PP(a_1)\otimes\cdots\otimes \PP(a_k)\right).\]
\end{defin}
There is a definition for $\mathbb{S}$-modules that requires some tools from the representation theory of symmetric groups that we are not going to introduce here. The reader is referred to \cite{lodayvallette} for the details. 

\begin{defin}
The \emph{plethysm} or \emph{composite} $f\circ g$ of maps $f:\OO\to\OO'$ and $g:\PP\to\PP'$ is given by
\[(f\circ g)(x\otimes x_1\otimes\cdots\otimes x_k)=(-1)^{\varepsilon} f(x)\otimes g(x_1)\otimes\cdots\otimes g(x_k),\]
where $\varepsilon$ is the Koszul sign obtained from swapping each $g$ by the correspoding elements. 
\end{defin}

 %, TENSOR PRODUCT OF S-MODULES

It is known that the category of collections with plethysm is a monoidal category, where the unit is the collection $I(1)=R$ and $I(n)=0$ for $n\neq 1$. 
\begin{defin}
A \emph{(non-symmetric) operad} is a collection $\OO=\{\OO(n)\}$ where there is a distinguished \emph{identity} element $1\in\OO(1)$ and with \emph{insertion maps} 
\[\circ_i:\OO(n)\otimes \OO(m)\to \OO(m+n-1)\]
for each $1\leq i\leq n$ satisfying natural unitality and associativity axioms. HERE REFER TO WARD BECAUSE ITS THE ONE THAT I USE, NOT LODAY AND VALETTE (MAYBE MENTION THAT IT IS SLIGHTLY DIFFERENT FROM LV)

Insertion maps can be iterated to define \emph{composition maps} \[\gamma(x;x_1,\dots, x_n)=(\cdots(x\circ_1 x_1)\circ_2 x_2\cdots
)\circ_n x_n).\]
If $\OO$ is an $\mathbb{S}-$module and the insertion maps satisfy some additional invariance axioms regarding the symmetric group action, we say that $\OO$ is a \emph{symmetric operad}. See \cite{lodayvallette} for more details.

A \emph{map of operads} (resp. \emph{symmetric operads}) is a map of collections (resp. $\mathbb{S}-$modules) that is compatible with insertions.
\end{defin}

The following lemma is a well-known fact. See \cite[\S 5]{lodayvallette} for more details. 
\begin{lemma}\label{monoid}
An operad $\OO$ is equivalent to a monoid in the monoidal category of collections with plethysm, where the multiplication  map is given precisely by the composition $\gamma:\OO\circ\OO\to\OO$. 
\end{lemma}


\begin{defin} An operad $\OO$ is called \emph{reduced} if $\OO(0)=0$.\end{defin}

\begin{defin} Let $\OO$ and $\PP$ be operads. The \emph{Hadamard product} $\OO\otimes\PP$ is given on each arity component by $(\OO\otimes\PP)(n)=\OO(n)\otimes\PP(n)$. The structure maps are given by diagonal composition and diagonal symmetric group action in the case of symmetric operads. \end{defin}

\begin{defin}
The \emph{endomorphism operad} $\End_A$ of a graded $R$-module $A$ is given by the modules \[\End_A(n)=\hom(A^{\otimes n},A).\] Insertion maps are given by
\[f\circ_i g=f(1^{\otimes i-1}\otimes g\otimes 1^{\otimes n-i})\]
for $f\in\End_A(n)$ and $g\in\End_A(m)$. The identity element is given by the identity map and there is a symmetric group action given by permuting the inputs.
\end{defin}
\begin{defin}
An \emph{algebra over an operad} $\OO$, or $\OO$-\emph{algebra}, is a map of operads $\OO\to\End_A$ for some $R$-module $A$. By adjunction, this is equivalent to a collection of maps $\OO(n)\otimes A^{\otimes n}\to A$ for each $\geq 0$.  
\end{defin}

REFERENCE TO THIS IN THE DEFINITION OF MORPHISM OF AAINFTY ALGEBRAS (FOR THE INFTY MORPHISM REFFER KELLER OR  SOMETHING WHERE THE INTUITION IS GIVEN)
\begin{defin}
A morphism of $\OO$-algebras $A$ and $B$ is a map of operads $\End_A\to\End_B$ so that the diagram
\[
\begin{tikzcd}
\OO \arrow[r]\arrow[dr] & \End_A\arrow[d]\\
& \End_B
\end{tikzcd}
\]
commutes. By adjunction, this is equivalent to a map of $R$-modules $f:A\to B$ such that the diagram
\[
\begin{tikzcd}
\OO(n)\otimes A^{\otimes n} \arrow[r]\arrow[d, "id\otimes f^{\otimes n}"] & A\arrow[d, "f"]\\
 \OO(n)\otimes B^{\otimes n}\arrow[r] & B
\end{tikzcd}
\]
commutes for all $n$.
%changing id by other map I could have maps of algebras over different operads
\end{defin}


The $\mathcal{A}_\infty$-operad is the non-symmetric operad whose algebras are $A_\infty$-algebras. WILL BE DEFINED LATER, SO REFERENCE Therefore, it is generated by elements $\mu_i\in\mathcal{A}_\infty(i)$ satisfying the operadic version of the $A_\infty$-equation. More details about this operad can be found in \cite[Chapter 9]{lodayvallette}.

All the above definitions generalize with no substantial changes to the bigraded case as well.

\subsubsection{Cooperads}

The monoidal definition of operad (\Cref{monoid}) allows to define the dual notion of a cooperad.



\begin{defin}
Let $\OO$ be a collection. A \emph{cooperad} is a structure of comonoid on $\OO$ in the monoidal category $(\col, \bar{\circ},I)$, where \[(\PP\bar{\circ}\mathcal{Q})(n) \coloneqq
\bigoplus_r \bigoplus_{n=i_1+\cdots+i_r}(\PP(r)\otimes \QQ(i_1)\otimes\cdots\otimes \QQ(i_r)),\]
and  $I$ is the collection such that $I(0)=R$ and is trivial elsewhere.
See \cite[\S 5.7.1]{lodayvallette} for more details and the symmetric version.
\end{defin}


Note that this is not the exact dual notion of an operad. To define the exact dual of the notion of operad, one should instead consider
the monoidal product
\[(\PP\hat{\circ}\QQ)(n) =\prod_{r\geq 0}(\PP(r)\otimes \prod_{n=i_1+\cdots+i_r}(\QQ(i_1)\otimes\cdots\otimes \QQ(i_r)))\]
in the category of collections, where the sums are replaced by products. In that case, a cooperad is defined as a comonoid $\Delta:\OO\to\OO\hat{\circ}\OO$.
When $\OO(0) = 0$ (the operad is reduced), the right-hand side product is equal to a sum. In this case, we are back to the previous denition.


SAY SOMETHING ABOUT THE REFERENCE FOR THIS RESULT (ALSO LIVE IT FOR THE END OF THE SECTION BECAUSE IT'S VERY LONG) 
\begin{propo}
Any symmetric lax monoidal functor $F:\CC\to\DD$ induces a functor $\underline{F}:\mathrm{Op}_\CC\to \mathrm{Op}_\DD$ between the categories of operads in $\CC$ and $\DD$, respectively.
\end{propo}

REFERENCE THE AXIOMS WHEN I WRITE THEM DOWN
\begin{proof}
Let $\OO$ be an operad in $\CC$ and let $F:\CC\to\DD$ be a symmetric lax monoidal functor. On objects, we define $\underline{F}(\OO)= F(\OO)$ and on morphisms we define $\underline{F}(f)=F(f)$ for $f:\OO\to\PP$. 

Let $\varepsilon: 1_\DD\to F(1_\CC)$ and $\mu\coloneqq\mu_{A,B}: F(A)\otimes F(B)\to F(A\otimes B)$ be the structure maps of the lax monoidal functor $F$. 

Let us first define the structure maps for the operad $F(\OO)$ in terms of insertions. Let $e:1_\CC\to\OO(1)$ be the unit of $\OO$. We define the unit $e_F:1_\DD\to F(\OO(1))$ as the composite
\[1_\DD\xrightarrow{\varepsilon} F(1)\xrightarrow{F(e)} F(\OO(1)).\]
Let $\circ_i :\OO(n)\otimes\OO(m)\to\OO(n+m-1)$ be the insertion map on $\OO$. We define the insertion map $\circ_i^F:F(\OO(n))\otimes F(\OO(m))\to F(\OO(n+m-1))$ as the composite
\[F(\OO(n))\otimes F(\OO(m))\xrightarrow{\mu} F(\OO(n)\otimes\OO(m))\xrightarrow{F(\circ_i)} F(\OO(n+m-1)).\]

We show now that $F(\OO)$ satisfies the unit axioms with the above structure maps. We only show the unit axiom with respect to the right unitor, the axiom with respect to the left unitor is analogous.


 Let $\lambda_\CC$ and $\lambda_\DD$ be the right unitors of $\CC$ and $\DD$ respectively. Since $\OO$ is an operad, by the unit axiom we have that the following diagram commutes.
% \[
% \begin{tikzcd}
% \OO(n)\otimes 1\arrow[r, "id\otimes e"]\arrow[rr, bend right = 20, "\lambda_\CC"] & \OO(n)\otimes \OO(1)\arrow[r, "\circ_i"]& \OO(n) 
% \end{tikzcd}
% \]
 
  \[
 \begin{tikzcd}
 \OO(n)\otimes 1_\CC\arrow[r, "\lambda_\CC"]\arrow[d, "id\otimes e"] &  \OO(n) \\
 \OO(n)\otimes \OO(1)\arrow[ur, bend right = 10, "\circ_i"]&
 \end{tikzcd}
 \]
 
Applying $F$ and introducing $\mu$ we get the following commutative diagram.
\begin{equation}\label{unitaux}
\begin{tikzcd}
F(\OO(n))\otimes F(1_\CC)\arrow[r,"\mu"]\arrow[d, "id\otimes F(e)"] & F(\OO(n)\otimes 1_\CC)\arrow[r, "F(\lambda_\CC)"]\arrow[d, "F(id\otimes e)"]&
F(\OO(n))\\
F(\OO(n))\otimes F(\OO(1))\arrow[r,"\mu"] & F(\OO(n)\otimes \OO(1))\arrow[ur, bend right = 15, "F(\circ_i)"']
\end{tikzcd}
\end{equation}

We need to show that the following diagram commutes.

  \[
 \begin{tikzcd}
F( \OO(n))\otimes 1_\DD\arrow[r, "\lambda_\DD"]\arrow[d, "id\otimes e_F"] &  F(\OO(n)) \\
 F(\OO(n))\otimes F(\OO(1))\arrow[ur, bend right = 10, "\circ_i^F"]&
 \end{tikzcd}
 \]
 
% Let us develop the diagram using the definition of the corresponding maps.
% 
% \[
%\begin{tikzcd}
%F( \OO(n))\otimes 1_\DD\arrow[r, "\lambda_\DD"]\arrow[d, "id\otimes F(e)\circ \varepsilon"] &  F(\OO(n)) \\
% F(\OO(n))\otimes F(\OO(1))\arrow[u, "\mu"] & \\
% F(\OO(
%\end{tikzcd} 
 %\]
 By monoidality of $F$ we know that $\lambda_\DD$ satisfies the following commutative diagram.
 
 \[
 \begin{tikzcd}
 F(\OO(n))\otimes 1_\DD\arrow[d, "\lambda_\DD"]\arrow[r, "id\otimes \varepsilon"] & F(\OO(n))\otimes F(1)\arrow[d,"\mu"]\\
 F(\OO(n)) & F(\OO(n)\otimes 1_\CC)\arrow[l, "F(\lambda)"]
 \end{tikzcd}
 \]
 Or, in other words, $\lambda_\DD = F(\lambda)\circ \mu(id\otimes \varepsilon)$.  On the other hand by diagram (\ref{unitaux}) we have that $F(\lambda)\circ \mu = (F(\circ_i)\circ \mu)(id\otimes F(e))$, meaning that
 \[\lambda_\DD =  F(\circ_i)\circ \mu\circ (id\otimes F(e))\circ (id\otimes \varepsilon) = \circ_i^F(id\otimes  e_F)\]
 as we wanted to show.
 
 Next we need to show that the associativity axioms of operads hold for $F(\OO)$. Let us first prove the one that does not involve the symmetry isomorphism. 
 
 Let $a_\CC$ and $a_\DD$ the associators for $\CC$ and $\DD$, respectively. For $i\leq j\leq i+m-1$ we have the following commutative diagram from the associativity axioms of the operad $\OO$.
 
 \[
\begin{tikzcd}
(\OO(n)\otimes \OO(m))\otimes \OO(l) \arrow[r, "a_\CC"]\arrow[d, "\circ_i\otimes"] & \OO(n)\otimes (\OO(m)\otimes\OO(l))\arrow[d, "id\otimes\circ_{j-i+1}"]\\
\OO(n+m-1)\otimes\OO(l)\arrow[d, "\circ_j"] & \OO(n)\otimes\OO(m+l-1)\arrow[dl, bend left = 10, "\circ_i"]\\
\OO(n+m+l-2) & 
\end{tikzcd} 
 \]
 
 Applying $F$ we obtain the following commutative diagram.
  \begin{equation}\label{assaux}
\begin{tikzcd}
F((\OO(n)\otimes \OO(m))\otimes \OO(l)) \arrow[r, "F(a_\CC)"]\arrow[d, "F(\circ_i\otimes id)"] & F(\OO(n)\otimes (\OO(m)\otimes\OO(l)))\arrow[d, "F(id\otimes\circ_{j-i+1})"]\\
F(\OO(n+m-1)\otimes\OO(l))\arrow[d, "F(\circ_j)"] & F(\OO(n)\otimes\OO(m+l-1))\arrow[dl, bend left = 10, "F(\circ_i)"]\\
F(\OO(n+m+l-2)) & 
\end{tikzcd} 
   \end{equation}
   
   According to the definition of $\circ_i^F$, we need to show that the following diagram commutes. 
   
   \[   
\begin{tikzcd}
(F(\OO(n))\otimes F(\OO(m)))\otimes F(\OO(l)) \arrow[r, "a_\DD"] \arrow[d, "\mu\otimes id"] & F(\OO(n))\otimes (F(\OO(m))\otimes F(\OO(l)))\arrow[d, "id\otimes \mu"]\\
F(\OO(n)\otimes \OO(m))\otimes F(\OO(l))\arrow[d, red, "F(\circ_i)\otimes id"] & F(\OO(n))\otimes F(\OO(m)\otimes\OO(l))\arrow[d, red, "id\otimes F(\circ_{j+i-1})"]\\
F(\OO(n+m-1))\otimes F(\OO(l))\arrow[d, red, "\mu"] & F(\OO(n))\otimes F(\OO(m+l-1))\arrow[d, red, "\mu"]\\
F(\OO(n+m-1)\otimes\OO(l))\arrow[d, "F(\circ_j)"] & F(\OO(n)\otimes\OO(m+l-1))\arrow[dl, bend left = 10, "F(\circ_i)"]\\
F(\OO(n+m+l-2)) & 
\end{tikzcd}   
   \]
   
   By naturality of $\mu$ we have 
   \begin{equation}\label{naturality}
   \mu\circ (F(\circ_i)\otimes id)= F( \circ_i \otimes id)\circ \mu 
   \end{equation} and \[\mu\circ (id \otimes F(\circ_{j-i+1}))= F(id\otimes \circ_{j-i+1})\circ \mu.\]
    Therefore we can replace the red arrows accordingly. We can also subdivide the above diagram into two by using $F(a_\CC)$ as follows.
   
      \[   
\begin{tikzcd}
(F(\OO(n))\otimes F(\OO(m)))\otimes F(\OO(l)) \arrow[r, "a_\DD"] \arrow[d, "\mu\otimes id"] & F(\OO(n))\otimes (F(\OO(m))\otimes F(\OO(l)))\arrow[d, "id\otimes \mu"]\\
F(\OO(n)\otimes \OO(m))\otimes F(\OO(l))\arrow[d, red, "\mu"] & F(\OO(n))\otimes F(\OO(m)\otimes\OO(l))\arrow[d, red, "\mu"]\\
F((\OO(n)\otimes\OO(m))\otimes\OO(l))\arrow[d, red, "F(\circ_i\otimes id)"]\arrow[r, dashed, "F(a_\DD)"] & F(\OO(n))\otimes F(\OO(n)\otimes(\OO(m)\otimes\OO(l)))\arrow[d, red, "F(id\otimes \circ_{j-1+1})"]\\
F(\OO(n+m-1)\otimes\OO(l))\arrow[d, "F(\circ_j)"] & F(\OO(n)\otimes\OO(m+l-1))\arrow[dl, bend left = 10, "F(\circ_i)"]\\
F(\OO(n+m+l-2)) & 
\end{tikzcd}   
   \]
   Now, the top diagram commutes because it is the associativity axiom of lax monoidal functors. The bottom diagram is precisely diagram (\ref{assaux}), so it commutes and we get the desired associativity axiom.
   
   Finally, we need to show that the associativity axioms involving the symmetry isomorphism hold for $F(\OO)$. Since they are analogous to each other, we only prove the first one.
   
   Let $B_\CC\coloneqq B_\CC^{X,Y}:X\otimes Y\to Y\otimes X$ the symmetry isomorphism on $\CC$ and similarly denote by $B_\DD$ the symmetry isomorphism on $\DD$.
   
   We have the following associativity commutative diagram for $j<i$.
   \[
   \begin{tikzcd}
   (\OO(n)\otimes \OO(m))\otimes\OO(l)\arrow[r, "a_\CC"]\arrow[d, "\circ_i\otimes id"] & \OO(n)\otimes (\OO(m)\otimes \OO(l)) \arrow[r, "id\otimes B_\CC"] & \OO(n)\otimes (\OO(l)\otimes \OO(m))\arrow[d, "a^{-1}_\CC"]\\
   \OO(n+m-1)\otimes\OO(l)\arrow[d, "\circ_j"] & & (\OO(n)\otimes\OO(l))\otimes \OO(m)\arrow[d, "\circ_j\otimes id"]\\
   \OO(n+m+l-2) & & \OO(n+l-1)\otimes \OO(m)\arrow[ll, "\circ_i"]
   \end{tikzcd}
   \]
   
   Applying $F$ we get the following commutative diagram.
   \begin{equation}\label{assiaux}
      \begin{tikzcd}
   F((\OO(n)\otimes \OO(m))\otimes\OO(l))\arrow[r, "F(a_\CC)"]\arrow[d, "F(\circ_i\otimes id)"] & F(\OO(n)\otimes (\OO(m)\otimes \OO(l))) \arrow[r, "F(id\otimes B^\CC)"] & \OO(n)\otimes (\OO(l)\otimes \OO(m))\arrow[d, "F(a_\CC)^{-1}"]\\
  F(\OO(n+m-1)\otimes\OO(l))\arrow[d, "F(\circ_j)"] & & F((\OO(n)\otimes\OO(l))\otimes \OO(m))\arrow[d, "F(\circ_j\otimes id)"]\\
   F(\OO(n+m+l-2)) & &F(\OO(n+l-1)\otimes \OO(m))\arrow[ll, "F(\circ_i)"]
   \end{tikzcd}
   \end{equation}
   
   We need to show that the following diagram commutes.
   \[
\begin{tikzcd}[column sep = -3em]
& F(\OO(n))\otimes (F(\OO(m))\otimes F(\OO(l)))\arrow[dr, "id\otimes B_\DD"]& \\
(F(\OO(n))\otimes F(\OO(m)))\otimes F(\OO(l))\arrow[d, "\mu\otimes id"]\arrow[ur, "a_\DD"] &  & F(\OO(n))\otimes (F(\OO(l))\otimes F(\OO(m)))\arrow[d, "a_\DD^{-1}"]\\
F(\OO(n)\otimes\OO(n))\otimes F(\OO(l))\arrow[d, "F(\circ_i)\otimes id"] & & (F(\OO(n))\otimes F(\OO(l)))\otimes F(\OO(m))\arrow[d, "\mu\otimes id"]\\
F(\OO(n+m-1)\otimes F(\OO(l))\arrow[d, "\mu\otimes id"] & & F(\OO(n)\otimes \OO(l))\otimes F(\OO(m))\arrow[d, "F(\circ_j)\otimes id"]\\
F(\OO(m+n-1)\otimes\OO(l))\arrow[d, "F(\circ_j)"] & & F(\OO(n+l-1))\otimes F(\OO(m))\arrow[d, "\mu"]\\
F((\OO(n+m+l-2)) & & F(\OO(n+l-1)\otimes\OO(m))\arrow[ll, "F(\circ_i)"]
\end{tikzcd}   
   \]
   
   We use naturality of $\mu$ (\Cref{naturality}) as we have done before to rewrite some of the arrows. We also subdivide the diagram into two by factoring by $F(a_\CC)^{-1}\circ F(id\otimes B_\CC)\circ F(a_\CC)$.
   
   \[
\begin{tikzcd}[column sep = -3em]
& F(\OO(n))\otimes (F(\OO(m))\otimes F(\OO(l)))\arrow[dr, "id\otimes B_\DD"]& \\
(F(\OO(n))\otimes F(\OO(m)))\otimes F(\OO(l))\arrow[d, "\mu\otimes id"]\arrow[ur, "a_\DD"] &  & F(\OO(n))\otimes (F(\OO(l))\otimes F(\OO(m)))\arrow[d, "a_\DD^{-1}"]\\
F(\OO(n)\otimes\OO(n))\otimes F(\OO(l))\arrow[d, "\mu"] & & (F(\OO(n))\otimes F(\OO(l)))\otimes F(\OO(m))\arrow[d, "\mu\otimes id"]\\
F((\OO(n)\otimes\OO(m))\otimes \OO(l))\arrow[d, "F(\circ_i\otimes id)"]\arrow[rrd, dashed, "F(a_\CC)^{-1}\circ F(id\otimes B_\CC)\circ F(a_\CC)", sloped] & & F(\OO(n)\otimes \OO(l))\otimes F(\OO(m))\arrow[d, "\mu"]\\
F(\OO(m+n-1)\otimes\OO(l))\arrow[d, "F(\circ_j)"] & & F((\OO(n)\otimes\OO(m))\otimes \OO(m))\arrow[d, "F(\circ_j\otimes id)"]\\
F((\OO(n+m+l-2)) & & F(\OO(n+l-1)\otimes\OO(m))\arrow[ll, "F(\circ_i)"]
\end{tikzcd}   
   \]
   
   The bottom diagram commutes as it is precisely diagram (\ref{assiaux}). We decompose the top diagram as follows.
   
   \[
\begin{tikzcd}[column sep = -3em]
 & F(\OO(n))\otimes (F(\OO(m))\otimes F(\OO(l)))\arrow[dr, "id\otimes B_\DD"]\arrow[ddddl, dashed, bend left = 30, sloped, "\mu\circ (id\otimes\mu)"]& \\
(F(\OO(n))\otimes F(\OO(m)))\otimes F(\OO(l))\arrow[d, "\mu\otimes id"]\arrow[ur, "a_\DD"] &  & F(\OO(n))\otimes (F(\OO(l))\otimes F(\OO(m)))\arrow[d, "a_\DD^{-1}"]\arrow[ddddl, dashed, bend right = 35, sloped, "\mu\circ (id\otimes\mu)"]\\
F(\OO(n)\otimes\OO(n))\otimes F(\OO(l))\arrow[d, "\mu"] & & (F(\OO(n))\otimes F(\OO(l)))\otimes F(\OO(m))\arrow[d, "\mu\otimes id"]\\
F((\OO(n)\otimes\OO(m))\otimes \OO(l))\arrow[d, "F(a)"] & & F(\OO(n)\otimes \OO(l))\otimes F(\OO(m))\arrow[d, "\mu"]\\
F(\OO(n)\otimes (\OO(m)\otimes\OO(l)))\arrow[dr,"F(id\otimes B_\CC)"] & & F((\OO(n)\otimes\OO(l))\otimes\OO(m))\\
& F(\OO(n)\otimes (\OO(l)\otimes\OO(m)))\arrow[ur, "F(a_\CC)^{-1}"]&
\end{tikzcd}  
   \]
   
   The left and right subdiagrams commute because of the associativity axiom of lax monoidal functors. We decompose the central subdiagram further as
   \[
\begin{tikzcd}[column sep = 5em]
F(\OO(n))\otimes (F(\OO(m))\otimes F(\OO(l))\arrow[r, "id\otimes B_\DD"]\arrow[d, "id\otimes \mu"] & F(\OO(n))\otimes (F(\OO(l))\otimes F(\OO(m)))\arrow[d, "id\otimes \mu"]\\
F(\OO(n))\otimes F(\OO(m)\otimes\OO(l))\arrow[r, dashed,  "id\otimes F(B_\CC)"]\arrow[d, "\mu"] & F(\OO(n))\otimes F(\OO(l)\otimes\OO(m))\arrow[d, "\mu"]\\
F(\OO(n)\otimes (\OO(m)\otimes\OO(l)))\arrow[r, "F(id\otimes B_\CC)"] & F(\OO(n)\otimes (\OO(l)\otimes \OO(m)))
\end{tikzcd}   
   \]
   
   The top part commutes because $F$ is symmetric lax monoidal and the bottom part commutes by naturality of $\mu$. This proves that $F(\OO)$ is an operad in $\DD$. 
   
   Lastly, we are only left with the proof that $F(f)$ is a map of operads. Since $f$ is a map of operads, we have for all $n$ the following commutative diagram.
   \[
\begin{tikzcd}[column sep = 5em]
\OO(n)\otimes \OO(m)\arrow[r, "f_n\otimes f_m"]\arrow[d, "\circ_i^\OO"] & \PP(n)\otimes\PP(m)\arrow[d, "\circ_i^\PP"]\\
\OO(n+m-1)\arrow[r, "f_{n+m-1}"] & \PP(n+m-1)
\end{tikzcd}   
   \]
   After applying $F$ we get the following commutative diagram.
   \begin{equation}\label{mapaux}
   \begin{tikzcd}[column sep = 5em]
F(\OO(n)\otimes \OO(m))\arrow[r, "F(f_n\otimes f_m)"]\arrow[d, "F(\circ_i^\OO)"] & F(\PP(n)\otimes\PP(m))\arrow[d, "F(\circ_i^\PP)"]\\
F(\OO(n+m-1))\arrow[r, "F(f_{n+m-1})"] & F(\PP(n+m-1))
\end{tikzcd} 
   \end{equation}
   
   We need to show that the following diagram commutes.
   \[
   \begin{tikzcd}[column sep = 5em]
F(\OO(n))\otimes F(\OO(m))\arrow[r, "F(f_n)\otimes F(f_m)"]\arrow[d,"\mu"] & F(\PP(n))\otimes F(\PP(m))\arrow[d, "\mu"]\\
F(\OO(n)\otimes \OO(m))\arrow[r, dashed, "F(f_n\otimes f_m)"]\arrow[d, "F(\circ_i^\OO)"] & F(\PP(n)\otimes\PP(m))\arrow[d, "F(\circ_i^\PP)"]\\
F(\OO(n+m-1))\arrow[r, "F(f_{n+m-1})"] & F(\PP(n+m-1))
\end{tikzcd}    
   \]
   The top subdiagram commutes because $\mu$ is natural and the bottom part is precisely diagram  (\ref{mapaux}), which commutes. This finishes the proof.
\end{proof}
SAME WITH COOPERADS? PROBABLY BY ABSTRACT NONSENSE OF OPPOSITE CATEGORY

\begin{itemize}


\item S-MODULES/COLLECTIONS IN THE NON SYMMETRRIC CASE, OPERADS (SYMMETRIC AND NS)- POSSIBLY SEVERAL DEFINITIONS, AT LEAST CLASSICAL, PARTIAL AND THEIR EQUIVALENCE AND MONOIDAL, EXAMPLES, (INFINITESIMAL SEE 10.2.4) MODULE OVER AN OPERAD, QUASI-FREE OPERAD, PRE-LIE ALGEBRA DEFINED BY INSERTIONS, FORGETFUL FUNCTOR FROM SYM TO NS? SHOULD WRITE THE AXIOMS FOR GRADED OPERADS SOMEWHERE (MORE GENERALLY OPERAD IN SYMMETRIC MONOIDAL CATEGORY LIKE WARD), AND I MIGHT ALSO WRITE THE PROOF THAT THIS IS AN OPERAD (BUT SHOULD FOLLOW FROM LAX MONOIDALITY)




\item INFINITESIMAL COMPOSITION?


\item KOSZUL DUALITY, CONVOLUTION OPERAD, TWISTED MORPHISM?

\item INFINITY OPERADS? 

\item OPERADIC COHOMOLOGY OR JUST INTRODUCE COHOMOLOGY AD HOC FOR THE CASES I NEED? LIKE ADD THE DEFINITION OF HOCHSCHILD COHOMOLOGY WHEN DEFINING THE HOCHSCHILD COMPLEX

\item BAR-COBAR  CONSTRUCTION FOR ALGEBRAS AND OPERADS?
\end{itemize}

-IN A DIFFERENT CHAPTER PROBABLY-
\begin{itemize}



\item A INFINITY ALGEBRAS CLASSICAL DEFINITION, BAR INTERPRETATION?, TWISTING MORPHISMS (L-V 10.1) REVIEW OF SOME KNOWN RESULTS. A-INFTY OPERAD AN ALGEBRAS AS MORPHISMS FROM THIS OPERAD (I CAN CONNECT IT TO THE COHOMOLOGY OF THE ASSOCIAHEDRA IF I EXPLAIN THAT) STRICT AND INFINITY MORPHIMS ACCORDING TO THE VARIOUS INTERPRETATIONS. DIFFERENCE BETWEEN INCLUDING THE DIFFERENTIAL OR NOT. TOPOLOGICAL ORIGIN? 

\item FILTERED MODULES, BIGRADED MODULES, TWISTED COMPLEXES, ENRICHMENTS FOR THESE CATEGORIES AS DONE BY SARAH

\item DERIVED A INFINITY ALGEBRAS, SOMETHING  FROM SAGAVE, UNIQUENESS OF AINFTY STRUTURES, DERIVED AINFTY IN OPERADIC CONTEXT, AND DERIVED AINFTY AND THEIR HOMOTOPIES. DIFFERENCE BETWEEN UNDERLYING TWISTED COMPLEXES OR NOT, MAIN THEOREM OF SARAH'S PAPER

\item FIND A BAR INTERPRETATION? (SINCE THIS WOULD BE KIND OF NEW, MAYBE IN A DIFFERENT CHAPTER) THIS IS ACTUALLY ON OPERADIC CONTEXT AND POSSIBLY EVEN IN SAGAVE, SO JUST STUDY IT AND INCLUDE IT JUST LIKE THE CLASSICAL CASE?
\item REFERENCE TO GEOMETRIC INTERPRETATIONS? (THESIS BY SARAH'S STUDENT)


\end{itemize}

%\phantomsection
\bibliographystyle{ieeetr}
\bibliography{newbibliography}
\end{document}
