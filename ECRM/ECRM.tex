\documentclass{beamer}
\usepackage[utf8]{inputenc}
\usetheme{Copenhagen}
%\usepackage[spanish]{babel}
\usepackage{multirow}
%\usepackage{estilo-apuntes}
\usepackage{braids}
\usepackage[]{graphicx}
\usepackage{rotating}
\usepackage{pgf,tikz}
\usepackage{pgfplots}
\usepackage{tikz-cd}
\usepackage{mathtools}
%\usepackage{empheq}
%\usepackage[dvipsnames]{xcolor}
\usepackage{xcolor}

\usetikzlibrary{arrows}
\usetikzlibrary{cd}
\usetikzlibrary{babel}
\pgfplotsset{compat=1.13}
\usetikzlibrary{decorations.shapes}
\pgfkeyssetvalue{/tikz/braid height}{1cm} %no parece hacer nada
\pgfkeyssetvalue{/tikz/braid width}{1cm}
\pgfkeyssetvalue{/tikz/braid start}{(0,0)}
\pgfkeyssetvalue{/tikz/braid colour}{black}

\theoremstyle{definition}

\newtheorem{teorema}{Theorem}
\newtheorem{defi}{Definition}
\newtheorem{prop}[teorema]{Proposition}

\newcommand{\Z}{\mathbb{Z}}
\newcommand{\Q}{\mathbb{Q}}
\newcommand{\C}{\mathbb{C}}
\newcommand{\CC}{\mathcal{C}}
\newcommand{\D}{\mathbb{D}}
\providecommand{\gene}[1]{\langle{#1}\rangle}

\DeclareMathOperator{\im}{im}


\addtobeamertemplate{navigation symbols}{}{%
    \usebeamerfont{footline}%
    \usebeamercolor[fg]{footline}%
    \hspace{1em}%
    %\insertframenumber/\inserttotalframenumber
}
\setbeamercolor{footline}{fg=black}
\setbeamerfont{footline}{series=\bfseries}

\newcommand{\highlight}[1]{%
	\colorbox{red!50}{$\displaystyle#1$}}

\makeatletter
\newcommand*{\encircled}[1]{\relax\ifmmode\mathpalette\@encircled@math{#1}\else\@encircled{#1}\fi}
\newcommand*{\@encircled@math}[2]{\@encircled{$\m@th#1#2$}}
\newcommand*{\@encircled}[1]{%
	\tikz[baseline,anchor=base]{\node[draw,circle,outer sep=0pt,inner sep=.2ex] {#1};}}
\makeatother

\expandafter\def\expandafter\insertshorttitle\expandafter{%
  \insertshorttitle\hfill%
  \insertframenumber\,/\,\inserttotalframenumber}

%-----------------------------------------------------------

\title{What are derived $A_\infty$-structures}
\author{Javier Aguilar Mart\'in}
\institute{University of Kent}
\date{}
 
\begin{document}
\frame{\titlepage}
%\begin{frame}
%
%c¡
%\title[About Beamer] %optional
%{About the Beamer class in presentation making}
% 
%\subtitle{A short story}
% 
%\author[Arthur, Doe] % (optional, for multiple authors)
%{A.~B.~Arthur\inst{1} \and J.~Doe\inst{2}}
% 
%\institute[VFU] % (optional)
%{
%  \inst{1}%
%  Faculty of Physics\\
%  Very Famous University
%  \and
%  \inst{2}%
%  Faculty of Chemistry\\
%  Very Famous University
%}

% 
%\date[VLC 2013] % (optional)
%{Very Large Conference, April 2013}


%\end{frame}
\setbeamercovered{highly dynamic}

\newcounter{saveenumi}
\newcommand{\seti}{\setcounter{saveenumi}{\value{enumi}}}
\newcommand{\conti}{\setcounter{enumi}{\value{saveenumi}}}

\makeatletter
%\newcommand{\xRightarrow}[2][]{\ext@arrow 0359\Rightarrowfill@{#1}{#2}}
\makeatother

\resetcounteronoverlays{saveenumi}
%\AtBeginSection[]{
%\begin{frame}
%\frametitle{Tabla de contenidos}
%\tableofcontents
%\end{frame}
%}

\begin{frame}
INTRO: ASS TO AINFTY TO DAINFTY

ASSOCIATIVITY, NON ASSOCIATIVITY, SOMETHING IN BETWEEN WHERE WE CAN CONTROL THE LACK OF ASSOCIATIVITY

A PATH BETWEEN THE TWO WAYS OF MULTIPLYING 3 ELEMENTS (THIS IS AN  INTUITION FOR HOMOTOPY) TRANSLATE TO ALGEBRA IN TERMS OF BOUNDARY (I WILL HAVE TO CLEAR THAT THE BOUNDARY MAP IS THE BRACKET, NOT D1 ITSELF)

WE CALL M3 THE WHOLE SEGMENT (MULTIPLICATION) AND TAKE BOUNDARRY

PENTAGON (MUST BE FILLED TO HAVE A BOUNDARY) TO PATHS AND HOMOTOPY IN BETWEEN ORIENTATION FOR SIGNS

DIMENSION VS DEGREE

THEN ACTUAL DEFINITIONS (PROBABLY IN TERMS OF BOUNDARIES, AND THEN DEFINING BOUNDARIES IN TERMS OF D1), MAYBE SHOW BASIC CASES RECALLING THE PICTURES

UNIQUENESS THEOREMS, FIELD ASSUMPTION, DERIVED
\end{frame}
\section{Introduction}
\begin{frame}
\frametitle{Generalizations of associativity}

Associative algebras $\Rightarrow$ $A_\infty$-algebras $\Rightarrow$ Derived $A_\infty$-algebras

%start with the first, spend most of the time on the second, finish with the third
\end{frame}


\begin{frame}
\frametitle{Associativity}
\begin{itemize}
\item $(ab)c=a(bc)$ for all $a,b,c$.
\item[]<2-> Example: \[3\cdot (e\cdot \pi) = (3\cdot e)\cdot \pi = 27\]
\end{itemize}
\end{frame}

\begin{frame}
\frametitle{Non-associativity}
\begin{itemize}
\item<1-> Cross product of vectors in $\mathbb{R}^3$:
\item[]<2-> 
\[
 \begin{bmatrix}
1\\
0\\
0
\end{bmatrix}\times\left(\begin{bmatrix}
1\\
1\\
0
\end{bmatrix} \times \begin{bmatrix}
1\\
1\\
1
\end{bmatrix}\right) \neq \left(\begin{bmatrix}
1\\
0\\
0
\end{bmatrix}\times\begin{bmatrix}
1\\
1\\
0
\end{bmatrix}\right)\times\begin{bmatrix}
1\\
1\\
1
\end{bmatrix}
\]
\item[]<3-> Check as an exercise.
\end{itemize}


\end{frame}


\section{Intuition}
\begin{frame}
\frametitle{Measuring the lack of associativity}

IMAGINE PRODUCTS AS POINTS IN THE SPACE, START CALLING IT M2(X,Y)?

SEVERAL SLIDES: DISJOINT POINT, POINTS JOINED BY A PATH, CALL IT M3 (THE BOUNDARY OF M3 IS THE 2 POINTS), CHOOSE ORIETATION (THE BOUNDARY IS THE DIFFERENCE OF THE POINTS)

PROBABLY INCLUDE TREES, WHEN A NODE HAS N BRANCHES IT IS MN
\end{frame}

\begin{frame}
PENTAGON, DIRECTION OF PATHS, ORIENTATION OF PENTAGON, SIGNS

MENTION THAT IT CAN BE GENERALIZED TO HIGHER DIMENSIONS
\end{frame}

\section{Algebra interlude}
\begin{frame}
\frametitle{Dimension vs degree}

\begin{itemize}
\item<1-> \textbf{Geometrical dimension}: 0 (points), 1 (lines), 2 (planes), etc
\item<2-> \textbf{Algebraic degree}: If $A=\bigoplus_{i\in\mathbb{N}} A_i$, we say $A_i$ is the $i$-th degree component.
\begin{itemize}
\item<3-> Example: $R[x]=\bigoplus_i R[x]_i$ (polynomials of degree $i$).
\end{itemize}
\item<4-> A map $f:A\to B$ is of \textbf{degree} $i$ if $f(A_j)\subset B_{j+i}$ for all $j$.
\begin{itemize}
\item<5-> Example: $x:R[x]\to R[x]$ is of degree 1 (multiplication by $x$).
\end{itemize}
\end{itemize}
\end{frame}

\section{$A_\infty$-algebras}
\begin{frame}
\frametitle{$A_\infty$-algebras}
\begin{defi}
An $A_\infty$-\emph{algebra} $A$ is a vector space over a field $k$ equipped with a family of ``multiplications'' $m_n:A^{\otimes n}\to A$ of degree $i-2$ for $n\geq 2$ satisfying the relation

\[\partial(m_n)=\sum_{r+s+t=n}(-1)^{rs+t}m_{r+1+t}(1^{\otimes r}\otimes m_s\otimes 1^{\otimes s})\] %we are composing every map with itself
\end{defi}

I CAN SAY $A$ IS A CHAIN COMPLEX SO THAT DELTA CAN BE DEFINED, OR DEFINE THIS AS SATISFFYING THAT DLTA IS A DIFFERNTIAL AND AVOID M1 (OR  SIMPLY SAY THAT IT COMES WITH DELTA)
\end{frame}





\begin{frame}
\frametitle{Some particular cases}
MAYBE AVOID TENSOR PRODUCTS AND WRITE EVERYTHING AS COMPONENTS

USE BLACKBOARD TO BETTER DESCRIBE WITH PICTURES (OR INCLUDE THEM BEFORE)
\begin{itemize}
\item<1-> $\partial(m_2)=0$ (the boundary of a point is empty).
\item<2-> $\partial(m_3)=m_2(m_2\otimes 1)-m_2(1\otimes m_2)$.
\begin{itemize}
\item<3-> An associative algebra is an $A_\infty$-algebra with $\partial=0$ and $m_n=0$ for $n>2$.
\item<3-> If we only require $m_n=0$ for $n>2$ we say that $A$ is a \textbf{dg-algebra}. %this not just a subproduct, it  is very important
\end{itemize}
\item<4-> The pentagon relation \begin{align*}
\partial(m_4)=&m_2(m_3\otimes 1)-m_3(m_2\otimes 1\otimes 1)\\
&+m_3(1\otimes m_2\otimes 1)-m_3(1\otimes 1\otimes m_2)+m_2(1\otimes m_3).
\end{align*}
\end{itemize}
\end{frame}

\begin{frame}
\frametitle{Uniqueness result}
\begin{itemize}
\item An $A_\infty$-algebra is \textbf{minimal} if $\partial = 0$. 
\end{itemize}\pause
\begin{theorem}[Kadeishvili]
Any ``nice enough'' dg-algebra can be ``replaced'' by a minimal $A_\infty$-algebra in an ``essentially unique'' way.
\end{theorem}\pause
%nice enough = HH(A,A) vanishes on degree 2-n %Replaced  means equivalent  %essentially = up to quasi-iso
We would like to extend this result to a ground ring $k$ that is not necessarily a field.
\end{frame}

\section{Derived $A_\infty$-algebras}

\begin{frame}
\frametitle{Bigraded algebras}
\begin{itemize}
\item We add an extra degree $A=\bigoplus_{i,j} A_i^j$, where $A_i^j$ is the bidegree $(i,j)$ component, and generalize previous definitions.
\begin{itemize}
\item<2-> Example: $A=R[x,y]$, $x:A\to A$ is of bidegree $(1,0)$.
\end{itemize}
\end{itemize}
\end{frame}

\begin{frame}
\frametitle{Derived $A_\infty$-algebras}
\begin{defi}
A \textbf{derived $A_\infty$-algebra} $A$ is a bigraded $k$-module equipped with a family of ``multiplications'' $m_{un}:A^{\otimes n}\to A$ of bidegree $(u,n+u-2)$ for $n\geq 2$ and $u\geq 0$ satisfying the relation

\[\partial(m_{un})=\underset{j=r+1+t}{\sum_{i+p=u,r+s+t=n}}(-1)^{rq+t+pj}m_{ij}(1^{\otimes r}\otimes m_{pq}\otimes 1^{\otimes s})\] %we are composing every map with itself
\end{defi}
\end{frame}


\begin{frame}
\frametitle{Uniqueness result}
\begin{itemize}
\item If $m_{un}=0$ for all $u>0$ we recover classical $A_\infty$-algebras.
\item<2-> If we require $m_{un}=0$ for $u+n\geq 3$ we get a \textbf{bidg-algebra}. %again this is important
\item<3-> A derived $A_\infty$-algebra is \textbf{minimal} if $\partial=0$. 
\item[]<4->
\begin{theorem}[Sagave]
Any ``nice enough'' bidg-algebra can be ``replaced'' by a minimal derived $A_\infty$-algebra in an ``essentially unique'' way.
\end{theorem}
\end{itemize}
\end{frame}
%\begin{frame}
%\frametitle{$A_\infty$-algebras are homotopy associative algebras.}
%%how do they generalize associative algebras
%\begin{itemize}
%\item<1-> For $n=3$ we have the relation
%\begin{align*}
%&m_2(m_2\otimes 1)-m_2(1\otimes m_2)=\\ %the failure of m_2 to be associative
%&m_1m_3+m_3(m_1\otimes 1\otimes 1)+m_3(1\otimes m_1\otimes 1)+m_3(1\otimes 1\otimes m_1)
%\end{align*}
%\item[]<2-> $m_2$ is homotopy associative with homotopy given by $m_3$. %recall that m1 is a differential so on homology this vanishes
%\item<3-> The higher relations are a homotopy coherent extension of this fact. %m3 satisfies some relation up to homotopy given by m4 and so on
%\end{itemize}
%\end{frame}


\end{document}
