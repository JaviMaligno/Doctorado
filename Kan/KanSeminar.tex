\documentclass{beamer}
\usepackage[utf8]{inputenc}
\usetheme{Copenhagen}
%\usepackage[spanish]{babel}
\usepackage{multirow}
%\usepackage{estilo-apuntes}
\usepackage{braids}
\usepackage[]{graphicx}
\usepackage{rotating}
\usepackage{pgf,tikz}
\usepackage{pgfplots}
\usepackage{tikz-cd}
%\usepackage{empheq}
%\usepackage[dvipsnames]{xcolor}
\usepackage{xcolor}

\usetikzlibrary{arrows}
\usetikzlibrary{cd}
\usetikzlibrary{babel}
\pgfplotsset{compat=1.13}
\usetikzlibrary{decorations.shapes}
\pgfkeyssetvalue{/tikz/braid height}{1cm} %no parece hacer nada
\pgfkeyssetvalue{/tikz/braid width}{1cm}
\pgfkeyssetvalue{/tikz/braid start}{(0,0)}
\pgfkeyssetvalue{/tikz/braid colour}{black}

\theoremstyle{definition}

\newtheorem{teorema}{Theorem}
\newtheorem{defi}{Definition}
\newtheorem{prop}[teorema]{Proposition}

\newcommand{\Z}{\mathbb{Z}}
\newcommand{\Q}{\mathbb{Q}}
\newcommand{\C}{\mathbb{C}}
\newcommand{\CC}{\mathcal{C}}
\newcommand{\D}{\mathbb{D}}
\providecommand{\gene}[1]{\langle{#1}\rangle}

\DeclareMathOperator{\im}{im}


\addtobeamertemplate{navigation symbols}{}{%
    \usebeamerfont{footline}%
    \usebeamercolor[fg]{footline}%
    \hspace{1em}%
    %\insertframenumber/\inserttotalframenumber
}
\setbeamercolor{footline}{fg=black}
\setbeamerfont{footline}{series=\bfseries}

\newcommand{\highlight}[1]{%
	\colorbox{red!50}{$\displaystyle#1$}}

\makeatletter
\newcommand*{\encircled}[1]{\relax\ifmmode\mathpalette\@encircled@math{#1}\else\@encircled{#1}\fi}
\newcommand*{\@encircled@math}[2]{\@encircled{$\m@th#1#2$}}
\newcommand*{\@encircled}[1]{%
	\tikz[baseline,anchor=base]{\node[draw,circle,outer sep=0pt,inner sep=.2ex] {#1};}}
\makeatother

\expandafter\def\expandafter\insertshorttitle\expandafter{%
  \insertshorttitle\hfill%
  \insertframenumber\,/\,\inserttotalframenumber}

%-----------------------------------------------------------

\title{Homotopy Associativity of H-spaces}
\author{Javier Aguilar Mart\'in}
\institute{University of Kent}
\date{}
 
\begin{document}
\frame{\titlepage}
%\begin{frame}
%
%c¡
%\title[About Beamer] %optional
%{About the Beamer class in presentation making}
% 
%\subtitle{A short story}
% 
%\author[Arthur, Doe] % (optional, for multiple authors)
%{A.~B.~Arthur\inst{1} \and J.~Doe\inst{2}}
% 
%\institute[VFU] % (optional)
%{
%  \inst{1}%
%  Faculty of Physics\\
%  Very Famous University
%  \and
%  \inst{2}%
%  Faculty of Chemistry\\
%  Very Famous University
%}

% 
%\date[VLC 2013] % (optional)
%{Very Large Conference, April 2013}


%\end{frame}
\setbeamercovered{highly dynamic}

\newcounter{saveenumi}
\newcommand{\seti}{\setcounter{saveenumi}{\value{enumi}}}
\newcommand{\conti}{\setcounter{enumi}{\value{saveenumi}}}

\makeatletter
\newcommand{\xRightarrow}[2][]{\ext@arrow 0359\Rightarrowfill@{#1}{#2}}
\makeatother

\resetcounteronoverlays{saveenumi}
%\AtBeginSection[]{
%\begin{frame}
%\frametitle{Tabla de contenidos}
%\tableofcontents
%\end{frame}
%}

%\begin{frame}
%	%AÑADIR ESTAS URL AL FINAL POR SI ME DA TIEMPO ENSEÑAR ESTAS COSAS 
%	%\url{https://www.blockchain.com/btc/blocks}
%	%\url{https://coin.dance/blocks}
%	%\url{https://www.blockchain.com/btc/unconfirmed-transactions}
%	
%%	ESTE TENGO PRIMERO QUE MIRARLO PARA HACERLO YO
%%	\url{https://anders.com/blockchain/}
%
%\end{frame}




%\begin{frame}
%	\begin{itemize}
%		\item Operads
%		\begin{itemize}
%			\item action of an operad (algebra over an operad)
%			\item little disks operad
%			\item chain/homology operad			
%		\end{itemize}
%	\item Gerstenhaber algebras
%	\item Hochschild cohomology of an associative algebra
%	\end{itemize}
%\end{frame}

\section{Introduction}

\subsection{Background}

\begin{frame}[fragile]
\frametitle{H-spaces}
%THE HOMOTOPY TYPE IS OF COUNTABLE CW-COMPLEXES AND MAPS AND HOMOTOPIES PRESERVE BASE POINTS
\begin{defi}
A topological space $X$ is an \emph{H-space} if if there exists a multiplication map $m:X\times X\to X$ and an unit element $e\in X$ such that $m(x,e)=x=m(e,x)$ for all $x\in X$. %a multiplication and a unit
\end{defi}\pause	
H-spaces generalize topological groups.\pause
%the relevant property for homotopy theory is associativity
\begin{lemma}[Sugawara]
If $X$ and $X\times X$ are CW complexes and $X$ is a homotopy associative H-spaces, then $X$ has inverses.
\end{lemma}

\end{frame}

\begin{frame}[fragile]

%to make sure that we are all in the same page
The multiplication $m:X\times X\to X$ is said to be \emph{homotopy associative} if the following diagram commutes up to homotopy
\[
\begin{tikzcd}
X\times X\times X\arrow[r,"m\times 1"]\arrow[d,"1\times m"']& X\times X\arrow[d,"m"]\\
X\times X\arrow[r,"m"]& X
\end{tikzcd}
\]\pause

In othwer words, there is a homotopy $m(m\times 1)\simeq m(1\times m)$.
\end{frame}


%\begin{frame}
%\frametitle{Another example}
%Let $(X,*)$ a pointed topological space and $\Omega X$ the spaces of based loops $f:S^1\to X$.\pause
%
%We have a concatenation map $m:\Omega X\times \Omega X\to \Omega X$, where $m(f_1,f_2)=f_1*f_2$ is given by\pause
%
%\begin{tikzpicture}[line cap=round,line join=round,>=triangle 45,x=1.0cm,y=1.0cm]
%\clip(-5,-3.) rectangle (5.,2.3);
%\draw(0.,0.) circle (1.5cm);
%\draw [->] (1.5,0.) -- (1.475763388700826,0.26855617768030476);
%\draw [->] (-1.5,0.) -- (-1.4622984077406362,-0.33419061434935543);
%\draw (-0.3,2.118952883889729) node[anchor=north west] {$f_1$};
%\draw (-0.3,-1.5566913118092813) node[anchor=north west] {$f_2$};
%\end{tikzpicture}
%\end{frame}
%
%\begin{frame}
%\frametitle{Homotopy-associative product}
%\begin{tikzpicture}[line cap=round,line join=round,>=triangle 45,x=1.0cm,y=1.0cm]
%\clip(-4.175394430564892,-2.5911383046897085) rectangle (7.490400123879831,3.3976612960713135);
%\draw(0.,2.) circle (1.cm);
%\draw(0.,-1.) circle (1.cm);
%\draw (-3.49428016933844,2.311720973491667) node[anchor=north west] {$(f_1*f_2)*f_3$};
%\draw (-3.496796110195476,-0.6773028846978557) node[anchor=north west] {$f_1*(f_2*f_3)$};
%\draw (-0.6057688157486263,1.1) node[anchor=north west] {$f_3$};
%\draw (0.2964512548334696,0.4) node[anchor=north west] {$f_1$};
%\draw (0.7431133461355,2.9675859207922457) node[anchor=north west] {$f_1$};
%\draw (-1.3,3.0105934583201526) node[anchor=north west] {$f_2$};
%\draw (-1.189603612937578,-1.4729423289641315) node[anchor=north west] {$f_2$};
%\draw (0.7489686163804383,-1.4729423289641315) node[anchor=north west] {$f_3$};
%\draw (2.,2.)-- (3.,1.);
%\draw (3.,1.)-- (4.,2.);
%\draw (3.,2.)-- (2.514225889472578,1.485774110527422);
%\draw (2.,-1.)-- (3.,-2.);
%\draw (3.,-2.)-- (4.,-1.);
%\draw (3.,-1.)-- (3.481998691455241,-1.518001308544759);
%\draw (1.7811495170502019,2.6342775049509677) node[anchor=north west] {$f_1$};
%\draw (2.8240823021019423,2.623525620568991) node[anchor=north west] {$f_2$};
%\draw (3.7487443589519387,2.5912699674230613) node[anchor=north west] {$f_3$};
%\draw (1.7811495170502019,-0.4085057751484382) node[anchor=north west] {$f_1$};
%\draw (2.8133304177199654,-0.4085057751484382) node[anchor=north west] {$f_2$};
%\draw (3.813255665243799,-0.4300095439123916) node[anchor=north west] {$f_3$};
%\draw [->] (1.,2.) -- (0.9776684310102762,2.2101533701987783);
%\draw [->] (0.,3.) -- (-0.2228209821222102,2.9748593795651215);
%\draw [->] (-1.,2.) -- (-0.9818938166599703,1.8105678675490438);
%\draw [->] (1.,-1.) -- (0.9832067635335799,-0.8175049037869153);
%\draw [->] (-1.,-1.) -- (-0.9827773938908082,-1.1847933820708718);
%\draw [->] (0.,-2.) -- (0.1987984337916885,-1.9800403985152712);
%\end{tikzpicture}
%\end{frame}


\begin{frame}
\frametitle{Examples of H-spaces}
\begin{itemize}
\item<1-> All topological groups, in particular $S^0$, $S^1$ and $S^3$ (real, complex and quaternionic numbers).
\item<2-> The Cayley (octonionic) numbers $S^7$.
\item<3-> Any loop spaces is a homotopy associative H-space via concatenation of loops.
\end{itemize}
%I focus on the first two items, i give these examples because they are key in the generalization of the following construction
	
\end{frame}

\begin{frame}
\frametitle{Projective spaces of topological groups}
Milnor defines projective spaces for any topological group $G$ in the following way:\pause
\begin{itemize}
\item<2-> Define $E_i=G\star\cdots\star G$ to be the $i$-fold join of $G$ with itself.
\item<2-> Construct fibre bundles $p_i:E_i\to B_i$ as quotient maps by the natural action of $G$ on $E_i$.
\end{itemize}\pause

If $G=S^{d-1}$ for $d=1,2,4$ we get the standard fibrations $S^{id-1}\to P^{i-1}$ onto the projective space of dimenion $i-1$.\pause

This construction does not work for $S^7$ since only the fibrations $S^7\to *$ and $S^{15}\to S^8$ can be constructed.
\end{frame}

\section{$A_n$-structures}
\begin{frame}
NOW DEFINE AN STRUCTURES AS A GENERALIZATION THAT WILL SOLVE THIS PROBLEM
\end{frame}
%
%\begin{frame}
%	\begin{itemize}
%		\item<1-> Identity element: 
%	%UN ÁRBOL AL QUE SE LE METEN PALITOS CON 1 Y OTRO METIÉNDOSE EN EL 1 Y AL FINAL IGUAL AL ARBOL EN CUESTION
%		
%		\begin{tikzpicture}[line cap=round,line join=round,>=triangle 45,x=1.0cm,y=1.0cm]
%		\clip(-2.2333333333333334,-1.) rectangle (13.1,3.35);
%		\draw [line width=1.2pt] (2.,0.)-- (2.,1.);
%		\draw [line width=1.2pt] (2.,1.)-- (1.,2.);
%		\draw [line width=1.2pt] (2.,1.)-- (1.6866666666666674,2.0066666666666664);
%		\draw [line width=1.2pt] (2.,1.)-- (2.42,1.98);
%		\draw [line width=1.2pt] (2.,1.)-- (3.,2.);
%		\draw [line width=1.2pt] (2.,0.)-- (2.,-1.);
%		\draw [line width=1.2pt] (1.,2.)-- (1.,3.);
%		\draw [line width=1.2pt] (1.6866666666666674,2.0066666666666664)-- (1.6866666666666674,2.9933333333333327);
%		\draw [line width=1.2pt] (2.42,1.98)-- (2.433333333333334,2.98);
%		\draw [line width=1.2pt] (3.,2.)-- (3.,3.);
%		\draw (1.4066666666666674,1.06) node[anchor=north west] {$f$};
%		\draw (0.8,3.42) node[anchor=north west] {$1$};
%		\draw (1.45,3.42) node[anchor=north west] {$1$};
%		\draw (2.2,3.42) node[anchor=north west] {$1$};
%		\draw (2.8,3.42) node[anchor=north west] {$1$};
%		\draw (1.593333333333334,-0.23333333333333328) node[anchor=north west] {$1$};
%		\draw (3.22,1.0866666666666664) node[anchor=north west] {$=f$};
%		\begin{scriptsize}
%		\draw [fill=black] (2.,0.) circle (1.5pt);
%		\end{scriptsize}
%		\end{tikzpicture} %añadir un palo no cambia la forma delárbol
%		\item<2-> A right action of the symmetric group thought as reordering the inputs which is coherent with composition. %reordenas los inputs de f y metes los g_i como si nada. Eso es lo mismo que dejar f quieta, cambiar el orden de los g_i y consecuentemente en la composición los argumentos de arriba entran en otro orden. La otra es similar pero haciendo actuar permutaciones sobre los g_i, y luego la forma en la que esos g_i entran en f se reordena consecuentemente
%	\end{itemize}
%\end{frame}
%
%\begin{frame}
%	Associativity and the existence of unit allows to understand compositions in terms of insertions $$f\circ_i g=\gamma(f;1,\dots, 1,\underbrace{g}_{i},1,\dots, 1)$$ \pause
%	
%	Composition of insertions is thought as grafting one tree at a time.
%\end{frame}
%\begin{frame}
%	\begin{defi}
%	 A map of operads $f:\mathcal{C}\to \mathcal{C}'$ is a collection of maps $\mathcal{C}(n)\to \mathcal{C}'(n)$ such that:
%		\begin{itemize}
%			\item<1->   $f\circ 1_\mathcal{C}=1_{\mathcal{C}'}$.
%			\item<2->  $f\circ \gamma_\mathcal{C}=\gamma_{\mathcal{C}'}\circ (f\times\cdots\times f)$.
%			\item<3->   $f(x\sigma)=f(x)\sigma$ for $x\in\CC(n)$ and $\sigma\in\Sigma_n$.
%		\end{itemize}
%	\end{defi}
%	
%	
%\end{frame}
%\begin{frame}
%	\frametitle{Symmetric monoidal  categories}
%	\begin{itemize}
%		\item<1-> A category where there is a notion of tensor product $\otimes $ of objects.
%		\item<2-> There exists an object $I$ such that $I\otimes A\cong A\cong A\otimes I$ for all object $A$.
%		\item<3-> The product is commutative: $A\otimes B\cong B\otimes A$.
%		\item<4-> The product is associative: $(A\otimes B)\otimes C\cong A\otimes (B\otimes C)$ for all objects.
%		\item<5-> Other coherence axioms.
%	\end{itemize}
%%	COMENTAR QUE ESTA DEFINICIÓN SE PUEDE HACER EN CUALQUIER CATEGORÍA MONOIDAL SIMÉTRICA, DICIENDO LOS COMPONENTES DE LA DEFINICIÓN Y QUIZÁ DESTACANDO ALGÚN AXIOMA
%	
%   %PONER EJEMPLOS
%\end{frame}
%\subsection{Algebras over an operad}
%\begin{frame}
%	\frametitle{Endomorphism operad}
%	%SI LA CATEGORÍA ES LO BASTANTE BUENA (CLOSED) TENEMOS LO SIGUIENTE, POR COMODIDAD LO DEFINIMOS EN ESTA CATEGORÍA
%	\begin{defi}
%		Let $V$ be a finite-dimensional vector space over a field $k$. Then the \textbf{endomorphism operad} $\xi_V = \{ \xi_V(n) \}_{n\geq 0}$ of $V$ consists of
%		\begin{itemize}
%			\item<1-> $\xi_V(n)=\hom(V^{\otimes n},V)
%			$ the space of linear maps $V^{\otimes n} \to V$.
%			\item<2-> composition $\gamma(f; g_1, \dots, g_n)= f(g_1\otimes\dots\otimes g_n)$
%			\item<3-> identity $\operatorname{Id}_V$
%			\item<4->  symmetric group action $\gamma (f; g_1, \dots, g_n) \cdot \sigma = f (g_{\sigma^{-1}(1)} \otimes \dots \otimes g_{\sigma^{-1}(n)})$,  $\sigma \in \Sigma_n$
%		\end{itemize}\pause
%\only<5>{	If $\mathcal{C}$ is another operad, each operad morphism $\mathcal{C} \to \xi_V$ is called an \textbf{algebra over} $\mathcal{C}$. Equivalently, a $\mathcal{C}$-algebra is given by a sequence of maps $\CC(n)\otimes V^{\otimes n}\to V$.}
%		 %(notice this is analogous to the fact that each ''R''-module structure on an abelian group ''M'' amounts to a ring homomorphism <math>R \to \operatorname{End}(M)</math>.)
%	\end{defi}
%\end{frame}
%
%\subsection{Little disks operad}
%\begin{frame}
%	\frametitle{Little Disks Operad}
%
%\begin{defi}
%	Let $E_2(n)$ be the configuration space of $n$ numbered disks $B(x_i,r_i)$ of center $x_i\in D^2$ and radius $r_i\in (0,1]$ inside the standard unit disk $D^2$. 
%	
%	\begin{itemize}
%		\item<2-> We call each $B(x_i,r_i)$ \textbf{little disk}.
%		%\item<3-> $E_2(n)$ can be viewed as a subspace of $(D^2\times (0,1])^n$ whose points are of the form $((x_1,r_1),\dots, (x_n,r_n))$ satisfying certain restrictions. % namely, $r_i$ must be such that the disk $B(x_i,r_i)\subset D^2$ does not intersect any other disk $B(x_j,r_j)$ and fits inside $D^2$. 
%		\item<3-> By convention, $E_2(0)=*$.
%	\end{itemize}
%	 
%\end{defi}
%\end{frame}
%
%\begin{frame}
%	\definecolor{xdxdff}{rgb}{0.49019607843137253,0.49019607843137253,1.}
%	\begin{figure}[h!]
%	\resizebox{10cm}{4.7cm}{%
%		\begin{tikzpicture}[line cap=round,line join=round,>=triangle 45,x=1.0cm,y=1.0cm]
%		\clip(-4,-3.4) rectangle (9.5,3.4);
%		\draw [line width=2.pt,color=xdxdff,fill=xdxdff,fill opacity=0.10000000149011612] (2.,1.) circle (0.7823042886243178cm);
%		\draw [line width=2.pt,color=xdxdff,fill=xdxdff,fill opacity=0.10000000149011612] (3.,-1.) circle (1.100727032465361cm);
%		\draw [line width=2.pt,color=xdxdff,fill=xdxdff,fill opacity=0.10000000149011612] (4.48,1.46) circle (0.7496665925596522cm);
%		\draw [line width=2.pt] (3.,0.) circle (3.1622776601683795cm);
%		\draw [line width=2.pt] (2.,1.) circle (0.7823042886243178cm);
%		\draw [line width=2.pt] (4.48,1.46) circle (0.7496665925596522cm);
%		\draw [line width=2.pt] (3.,-1.) circle (1.100727032465361cm);
%		\draw (1.84,1.3) node[anchor=north west] {$1$};
%		\draw (4.32,1.7) node[anchor=north west] {$2$};
%		\draw (2.8,-0.8) node[anchor=north west] {$3$};
%		%\draw (-1.44,0.6) node[anchor=north west] {\LARGE{$c=$}};
%		\end{tikzpicture}
%	}
%	\caption{A point of $ E_2(3)$.}
%\end{figure}
%\end{frame}
%
%\begin{frame}[fragile]
%	\frametitle{Operad of little disks}
%	
%	 We define for all positive integers $p$ and $q$ and  each $1\leq i\leq p$ the insertion maps 
%$$	
%\begin{tikzcd}[row sep=5]
%E_2(p)\times E_2(q)\arrow[r, "\circ_i"] & E_2(p+q-1)\\
%(c_1,c_2)\arrow[r, mapsto, shorten <= 1em, shorten >= 1em] & c_1\circ_i c_2
%\end{tikzcd}
%$$
%
%\end{frame}
%
%\begin{frame}
%	\frametitle{Operad of little disks}
%	\begin{figure}
%		\begin{tikzpicture}[line cap=round,line join=round,>=triangle 45,x=1.0cm,y=1.0cm]
%	\clip(-5.2,-1.5) rectangle (6.92,1.5);
%	\draw [line width=2.pt] (0.,0.) circle (1.42cm);
%	\draw (-0.2,0.2) node[anchor=north west] {$1$};
%	\end{tikzpicture}
%	\caption{Identity element $1\in E_2(1)$.}
%\end{figure}
%\end{frame}
%
%
%\subsection{Chain operad and Homology operad}
%\begin{frame}
%	\frametitle{Chain operad and homology operad of little disks}
%	
%	\begin{itemize}
%	\item<1-> Very well studied homotopy groups: $\pi_i(E_2(n))=0$ for $i>2$ and $\pi_1(E_2(n))=PB_n$.
%		\item<2-> The collection of chain complexes $C_*(E_2)=\{C_*(E_2(n))\}_{n\geq 0}$ forms an operad thanks to the Eilenberg-Zilber map $\nabla: C_*(A)\otimes C_*(B)\to C_*(A\times B)$.
%		\item<3-> This induces an operad structure on the collection $H_*(E_2)=\{H_*(E_2(n))\}_{n\geq 0}$.
%		\item<4-> $H_*(E_2)$ is generated by two operations: $\mu\in H_0(E_2(2))$ and $l\in H_1(E_2(2)) $.
%	\end{itemize}
%\end{frame}
%
%\begin{frame}
%
%
%%\begin{frame}
%%	\begin{itemize}
%%		\item Operads \checkmark
%%		\begin{itemize}
%%			\item action of an operad (algebra over an operad) \checkmark
%%			\item little disks operad \checkmark
%%			\item chain/homology operad	\checkmark		
%%		\end{itemize}
%%		\item Gerstenhaber algebras
%%		\item Hochschild cohomology of an associative algebra
%%	\end{itemize}
%%\end{frame}
%
%
%
%\section{Gerstenhaber algebras}
%
%\begin{frame}
%	
%	
%\end{frame}
%
%
%
%\begin{frame}
%	\frametitle{Homology operad of little disks and Gerstenhaber algebras}
%	
%	\begin{teorema}
%		Gerstenhaber algebras and $H_*(E_2)$-algebras are the same thing.
%	\end{teorema}\pause 
%	The commutative product is given by 
%	\[
%	G^{\otimes 2}\cong k\otimes G^{\otimes 2}\cong H_0(E_2(2))\otimes G^{\otimes 2}\to G
%	\]
%	\pause
%	and the Lie bracket of degree by the
%	\[
%	G^{\otimes 2}\cong k\otimes G^{\otimes 2}\cong H_1(E_2(2))\otimes G^{\otimes 2}\to G
%	\]
%\end{frame}
%
%
%%
%%\begin{frame}
%%	\begin{itemize}
%%		\item Operads \checkmark
%%		\begin{itemize}
%%			\item action of an operad (algebra over an operad) \checkmark
%%			\item little disks operad \checkmark
%%			\item chain/homology operad	\checkmark		
%%		\end{itemize}
%%		\item Gerstenhaber algebras \checkmark
%%		\item Hochschild cohomology of an associative algebra
%%	\end{itemize}
%%\end{frame}
%
%\section{Hochschild cohomology of an associative algebra}
%\begin{frame}
%	\frametitle{Hochschild complex}
%	\begin{defi}
%	 We define a \textbf{Hochschild $m$-cochain} $f^m$ of $A$ to be a $k$-module homormorphism $f^m: A^{\otimes m}\to A$. The \textbf{Hochschild complex} is given by the modules $C^m(A;A)=\hom_k(A^{\otimes m}, A)$, where $C^0(A;A)=A$ and the differential
%	 \begin{align*}
%	 \delta_m f(a_1\otimes\cdots\otimes a_{m+1})&=a_1f(a_2\otimes\cdots\otimes a_{m+1})\\
%	  +\sum_{i=1}^m(-1)^if(a_1\otimes\cdots&\otimes a_{i-1}\otimes a_ia_{i+1}\otimes a_{i+1}\otimes\cdots a_{m+1})\\
%	 & +(-1)^{m+1}f(a_1\otimes\cdots\otimes a_m)a_{m+1},
%	 \end{align*}
%	 %extensión lineal de esto
%	 \end{defi}
%\end{frame}
%
%\begin{frame}
%	\frametitle{Hochschild cohomology and cup product}
%	\begin{defi}
%		The $m$-th \textbf{Hochschild cohomology} of $A$ is defined to be $H^m(A;A)=\ker\delta_m/\im\delta_{m-1}$.
%		\end{defi}\pause
%		
%		For $f\in C^n(A;A)$ and $g\in C^m(A;A)$ define
%		
%		 $(f\smile g)(a_1\otimes\cdots\otimes a_n\otimes b_1\otimes\cdots\otimes b_m)=f(a_1\otimes\cdots\otimes a_n)g(b_1\otimes\cdots\otimes b_m)$\pause
%	\begin{teorema}
%		$\{H^*(A,A),\smile\}$ is a graded commutative algebra, with grading given by dimension, i.e., if $\eta\in H^m(A,A)$ and $\xi\in H^n(A,A)$, then $\eta\smile \xi =(-1)^{mn}\xi\smile \eta$.
%	\end{teorema} 
%\end{frame}
%
%%\begin{frame}
%%	\frametitle{Cup product}
%%	\begin{defi}
%%	 For $f\in C^m(A;A)$ and $g\in C^n(A;A)$ define their \textbf{cup product} $f\smile g\in C^{n+m}(A;A)$ by
%%	\[
%%	f\smile g(a_1\otimes\cdots\otimes a_m\otimes b_1\otimes\cdots\otimes b_n)=f(a_1\otimes\cdots\otimes a_m)g(b_1\otimes\cdots\otimes b_n).
%%	\]
%%	\end{defi}\pause
%%The cup product satisfies the Leibniz rule
%%$$\delta (f^m\smile g^n)=\delta f^m\smile g^n+(-1)^m f^m\smile \delta g^n$$\pause
%%SI ME QUEDA LARGO QUITAR LAS DEFINICIONES %Con lo cual induce un producto en cohomología
%%
%%
%%\end{frame}
%
%\begin{frame}
%\frametitle{Lie bracket}
%First, for $f\in C^m(A;A)$ and $g\in C^n(A;A)$ set
%
%$$
%f\circ_i g=f(1\otimes\cdots \otimes 1\otimes \underbrace{g}_{i-th}\otimes 1\otimes \cdots\otimes 1)\in C^{n+m-1}(A;A)
%$$
%\pause 
%Now set
%\[
%f\circ g=\sum_{i=0}^m (-1)^{ni}f\circ_i g.
%\]\pause
%Then the bracket is defined by $$[f,g]=f\circ g-(-1)^{(n-1)(m-1)}g\circ f$$
%\end{frame}
%
%\begin{frame}
%	\frametitle{The Hochschild cohomology is a Gerstenhaber algebra}
%	
%	\begin{teorema}
%		Let $A$ be an algebra and $\xi$, $\eta$ and $\zeta$ elements of $H^m(A;A)$, $H^n(A;A)$ and $H^p(A;A)$, respectively. The Lie bracket $[,]$ is of degree $-1$ and such that
%		\[
%		[\eta, \xi\smile \zeta]=[\eta,\xi]\smile \zeta+(-1)^{m(n-1)}\eta\smile[\xi,\zeta].
%		\]
%	\end{teorema}\pause
%	\begin{teorema}
%	The Hochschild cohomology $H^*(A;A)$ is a Gerstenhaber algebra.
%\end{teorema} 
%\end{frame}
%
%%\begin{frame}
%%	\begin{itemize}
%%		\item Operads \checkmark
%%		\begin{itemize}
%%			\item action of an operad (algebra over an operad) \checkmark
%%			\item little disks operad \checkmark
%%			\item chain/homology operad	\checkmark		
%%		\end{itemize}
%%		\item Gerstenhaber algebras \checkmark
%%		\item Hochschild cohomology of an associative algebra \checkmark
%%	\end{itemize}
%%\end{frame}
%%\begin{frame}
%%	\frametitle{Lie Bracket}
%%	For $f\in C^m(A;A)$ and $g\in C^n(A;A)$ set
%%	\begin{gather*}
%%	f\circ_i g(a_0\otimes\cdots\otimes a_{i-1}\otimes b_0\otimes\cdots\otimes b_{n-1}\otimes a_{i+1}\otimes\cdots \otimes a_{m-1})\\
%%	=f(a_0\otimes \cdots a_{i-1}\otimes g(b_0\otimes\cdots\otimes b_{n-1})\otimes a_{i+1}\otimes\cdots\otimes a_{m-1})
%%	\end{gather*}
%%	for $i=0,\dots, m-1$. Note that $f\circ_i g\in C^{n+m-1}(A;A)$.\pause
%%	
%%	Now set
%%	\[
%%	f\circ g=\sum_{i=0}^m (-1)^{ni}f\circ_i g.
%%	\]
%%\end{frame}
%%\begin{frame}
%%	\frametitle{Lie Bracket}
%%	\begin{defi}
%%		The bracket is defined by $[f,g]=f\circ g-(-1)^{(n-1)(m-1)}g\circ f$.
%%	\end{defi}\pause 
%%The bracket also satisfies the corresponding Leibniz rule
%%\[\delta[f^m,g^n]=(-1)^{n-1}[\delta f^m,g^n]+[f^m,\delta g^n].\]\pause 
%%
%%\end{frame}
%
%
%%\begin{frame}
%%	\begin{block}{}
%%	\begin{itemize}
%%	\item An identity element $1 \in \CC(1)$ such that 
%%	$\gamma(1; d) = d$ for $d \in \CC(j)$ and 
%%	$\gamma(c; 1,\dots,1) = c$ for
%%	$c \in \CC(k)$.
%%	
%%	\item A right action of the symmetric group $\Sigma_j$ on $\CC(j)$ such that the following equivariance
%%	formulas are satisfied for all $c\in \CC(k)$, $d_s \in \CC(j_s)$, $\sigma\in\Sigma_k$, and $\tau_s\in\Sigma_{j_s}$:
%%	\[
%%	\gamma(c\sigma; d_1, \dots , d_k) = 
%%	\gamma(c; d_{\sigma^{-1}(1)}, \dots , d_{\sigma^{-1}(k)})\sigma(j_1, \dots , j_k)
%%	\]
%%	and 
%%	\[
%%	\gamma(c; d_1\tau_1, \dots , d_k\tau_k) = \gamma(c; d_1, \dots , d_k)(\tau_1\oplus\cdots\oplus\tau_k),
%%	\] 
%%	where $\sigma(j_1, \dots , j_k)$ denotes the
%%	permutation of $j$ letters which permutes the $k$ blocks of letters determined by the given
%%	partition of $j$ (a first block of $j_1$, a second one of $j_2$ letters and so on), and $\tau_1\oplus\cdots\oplus\tau_k$ denotes the image of $(\tau_1, \dots , \tau_k)$ under the evident inclusion of $\Sigma_{j_1} \times \cdots \times \Sigma_{j_k}$ in $\Sigma_j$.
%%\end{itemize}
%%\end{block}
%%
%%\end{frame}
%
%
%
%\section{The Conjecture}
%\begin{frame}
%	\frametitle{Deligne Conjecture}
%	\begin{block}{The conjecture}
%	The action of the homology operad of little disks on the Hochschild cohomology of an associative algebra inducing its Gerstenhaber algebra structure lifts to an action of the chain operad of little disks on the Hochschild complex.
%	\end{block}
%	
%\end{frame}
%
%\begin{frame}[fragile]
%	\begin{center}
%		
%	
%	\begin{tikzcd}[column sep=60, row sep=30]
%	H_*(E_2)\arrow[r,"\text{Gerstenhaber}"] & H^*(A;A)\\
%	C_*(E_2)\arrow[u, "q"]\arrow[r, dashed, "?"] & C^*(A;A)\arrow[u, "q"]
%	\end{tikzcd}
%\end{center}
%\end{frame}
%\begin{frame}[fragile]
%	\begin{center}
%		
%		
%		\begin{tikzcd}[column sep=60, row sep=30]
%		H_*(E_2)\arrow[r,"\text{Gerstenhaber}"] & H^*(A;A)\\
%		C_*(E_2)\simeq \mathcal{O}?\arrow[u, "q"]\arrow[r, dashed, "?"] & C^*(A;A)\arrow[u, "q"]
%		\end{tikzcd}\pause
%		
%		\begin{defi}
%			Two chain operads $\mathcal{O}$ and $\mathcal{O}'$ are \textbf{equivalent} if there exists a chain of quasi-isomorphisms
%			\[
%			\mathcal{O}= \mathcal{O}_1\to\mathcal{O}_2\leftarrow\mathcal{O}_3\to\cdots\leftarrow\mathcal{O}_{2k+1} = \mathcal{O}'
%			\]
%		\end{defi}
%		%equivalentes en el sentido de que tengan la misma homología
%	\end{center}
%\end{frame}
%
%
%\section{Proof of the conjecture}
%
%\begin{frame}[fragile]
%	\frametitle{Proof of the conjecture}
%%	$$
%%	\begin{tikzcd}
%%	C_*(E_2)\arrow[r, "e.h."] & C_*(|N(PaB)|) \arrow[r] & O_A & \arrow[l] H_*(E_2) & \arrow[l] G_\infty \arrow[r] & B
%%	\end{tikzcd}
%%	$$
%	\scalebox{1}{$C_*(E_2)\xrightarrow{h. e.} C_*(|N(PaB)|)\to C_*(O_A)\leftarrow H_*(E_2) \leftarrow G_\infty \to \widetilde{\mathcal{B}}\xrightarrow{\cong}\mathcal{B}$}\pause %el último no es quasi-iso, es solo map pero nos sirve
%	\vspace{1cm}
%	
%	
%	Finally $\mathcal{B}$ acts on $C^*(A;A)$ inducing on $H^*(A;A)$ the Gersternhaber algebra structure.
%\end{frame}
%
%\begin{frame}
%	\frametitle{Proof of the conjecture}
%	\scalebox{1}{$\highlight{C_*(E_2)\xrightarrow{h.e.} C_*(|N(PaB)|)\to C_*(O_A)\leftarrow H_*(E_2)} \leftarrow G_\infty \to \widetilde{\mathcal{B}}\xrightarrow{\cong}\mathcal{B}$}
%\vspace{1cm}
%
%Finally $\mathcal{B}$ acts on $C^*(A;A)$ inducing on $H^*(A;A)$ the Gersternhaber algebra structure.
%\end{frame}
%
%\begin{frame}
%	\frametitle{Proof of the conjecture}
%	\scalebox{1}{$\highlight{C_*(E_2)\xrightarrow{h. e.} C_*(|N(PaB)|)\to C_*(O_A)\leftarrow H_*(E_2)} \leftarrow G_\infty \to \widetilde{\mathcal{B}}\xrightarrow{\cong}\mathcal{B}$}
%	\vspace{1cm}
%	
%	Finally \underline{$\mathcal{B}$ acts on $C^*(A;A)$ inducing on $H^*(A;A)$ the} \underline{Gersternhaber algebra structure}.
%\end{frame}
%
%\subsection{Formality of chain operad of little disks}
%\begin{frame}
%	\frametitle{Formality}
%	\begin{itemize}
%		\item<1-> We prove $C_*(E_2)\simeq H_*(E_2)$.
%		\item<2-> We define an operad of categories $PaB$ such that $|N(PaB)|\simeq E_2$. 
%		\item<3-> We find another chain operad $C_*(O_A)$ and quasi-isomorphisms $C_*(|N(PaB)|)\to C_*(O_A)\leftarrow H_*(E_2)$. 
%	\end{itemize}%QPaB y PaB son equivalentes como categoráis no enriquecidas porque los morfismos están en biyección (infinito numerable) y los objetos son los mismos. Como al nervio no le afecta lo de aditividad pues yastá
%\end{frame}
%
% \begin{frame}
% 	\frametitle{Braided operads}
% 	\begin{itemize}
% 		\item<1-> A \textbf{braided operad} is the same as an operad but with a right action of the braid groups. 
% 		\item<2-> A $B_\infty$-operad is a topological braided operad whose spaces are contractible.
% 		\item<3-> If $X$ is a $B_\infty$-operad, then the operad $\{X(n)/PB_n\}$ is called a \textbf{little disks operad}.
% 	\end{itemize}
% 
% \only<4->{\begin{prop}
% 		Any two little disks operads are homotopy equivalent.
% \end{prop}}
% 
% \only<5->{\begin{prop}
% 		$E_2$ is a little disks operad. %of a $B_\infty$-operad, more precisely.%, of its universal covering. %decir que esto es de lo que más he hecho, aunque me faltó completar algunos detalles
% 		
% \end{prop}}
%\end{frame}
%
%
%%\begin{frame}[fragile]
%%	\begin{proof}[Idea of the proof]
%%		The idea is to lift the operad structure of $E_2$ to a braided operad structure on $\widetilde{E}_2$ using the uniqueness of the lifting 
%%		\[
%%		\begin{tikzcd}
%%		\widetilde{E}_2(k)\times\widetilde{E}_2(j_1)\times\cdots\widetilde{E}_2(j_k) \arrow[r, "\widetilde{\gamma}", dashed] \arrow[d, "p"] & \widetilde{E}_2(j_1+\cdots+j_k) \arrow[d, "p"] \arrow[d] \\
%%		E_2(k)\times E_2(j_1)\times\cdots\times E_2(j_1+\cdots+j_k) \arrow[r, "\gamma"]                                                                      & E_2(j_1+\cdots+j_k)                                     
%%		\end{tikzcd}
%%		\]
%%		up to a choice of base point. But instead of a base point, we use a contractible subspace $E_1$, which has a homotopy discrete fiber. This subspace $E_1$ is an embedding of the \textbf{little intervals} operad, which is analogue to the little disks operad. 
%%	\end{proof}
%%\end{frame}
%%
%%\begin{frame}
%%	
%%	\begin{figure}
%%	\definecolor{xdxdff}{rgb}{0.49019607843137253,0.49019607843137253,1.}
%%	\begin{tikzpicture}[line cap=round,line join=round,>=triangle 45,x=1.0cm,y=1.0cm]
%%	\clip(-3.5,-0.4666666666666) rectangle (11.293333333333335,4.5);
%%	\fill[line width=2.pt,color=xdxdff,fill=xdxdff,fill opacity=0.10000000149011612] (0.41333333333333344,4.) -- (0.4,0.) -- (0.7733333333333334,0.) -- (0.7866666666666668,4.) -- cycle;
%%	\fill[line width=2.pt,color=xdxdff,fill=xdxdff,fill opacity=0.10000000149011612] (1.2,4.) -- (1.2133333333333338,0.) -- (1.586666666666667,0.) -- (1.586666666666667,4.) -- cycle;
%%	\fill[line width=2.pt,color=xdxdff,fill=xdxdff,fill opacity=0.10000000149011612] (3.,4.) -- (3.,0.) -- (3.386666666666667,0.) -- (3.4,4.) -- cycle;
%%	\draw[line width=2.pt] (0.41333333333333344,4.) -- (0.4,0.) -- (0.7733333333333334,0.) -- (0.7866666666666668,4.) -- cycle;
%%	\draw[line width=2.pt] (1.2,4.) -- (1.2133333333333338,0.) -- (1.586666666666667,0.) -- (1.586666666666667,4.) -- cycle;
%%	\draw[line width=2.pt] (3.,4.) -- (3.,0.) -- (3.386666666666667,0.) -- (3.4,4.) -- cycle;
%%	\draw [line width=2.pt] (0.,0.)-- (0.,4.);
%%	\draw [line width=2.pt] (0.,4.)-- (4.,4.);
%%	\draw [line width=2.pt] (4.,4.)-- (4.,0.);
%%	\draw [line width=2.pt] (4.,0.)-- (0.,0.);
%%	\draw [line width=2.pt,color=xdxdff] (0.41333333333333344,4.)-- (0.4,0.);
%%	\draw [line width=2.pt,color=xdxdff] (0.4,0.)-- (0.7733333333333334,0.);
%%	\draw [line width=2.pt,color=xdxdff] (0.7733333333333334,0.)-- (0.7866666666666668,4.);
%%	\draw [line width=2.pt,color=xdxdff] (0.7866666666666668,4.)-- (0.41333333333333344,4.);
%%	\draw [line width=2.pt,color=xdxdff] (1.2,4.)-- (1.2133333333333338,0.);
%%	\draw [line width=2.pt,color=xdxdff] (1.2133333333333338,0.)-- (1.586666666666667,0.);
%%	\draw [line width=2.pt,color=xdxdff] (1.586666666666667,0.)-- (1.586666666666667,4.);
%%	\draw [line width=2.pt,color=xdxdff] (1.586666666666667,4.)-- (1.2,4.);
%%	\draw [line width=2.pt,color=xdxdff] (3.,4.)-- (3.,0.);
%%	\draw [line width=2.pt,color=xdxdff] (3.,0.)-- (3.386666666666667,0.);
%%	\draw [line width=2.pt,color=xdxdff] (3.386666666666667,0.)-- (3.4,4.);
%%	\draw [line width=2.pt,color=xdxdff] (3.4,4.)-- (3.,4.);
%%	\draw [line width=2.pt] (0.41333333333333344,4.)-- (0.4,0.);
%%	\draw [line width=2.pt] (0.4,0.)-- (0.7733333333333334,0.);
%%	\draw [line width=2.pt] (0.7733333333333334,0.)-- (0.7866666666666668,4.);
%%	\draw [line width=2.pt] (0.7866666666666668,4.)-- (0.41333333333333344,4.);
%%	\draw [line width=2.pt] (1.2,4.)-- (1.2133333333333338,0.);
%%	\draw [line width=2.pt] (1.2133333333333338,0.)-- (1.586666666666667,0.);
%%	\draw [line width=2.pt] (1.586666666666667,0.)-- (1.586666666666667,4.);
%%	\draw [line width=2.pt] (1.586666666666667,4.)-- (1.2,4.);
%%	\draw [line width=2.pt] (3.,4.)-- (3.,0.);
%%	\draw [line width=2.pt] (3.,0.)-- (3.386666666666667,0.);
%%	\draw [line width=2.pt] (3.386666666666667,0.)-- (3.4,4.);
%%	\draw [line width=2.pt] (3.4,4.)-- (3.,4.);
%%	\draw (0.35,2.38) node[anchor=north west] {$1$};
%%	\draw (1.15,2.38) node[anchor=north west] {$2$};
%%	\draw (2.95,2.38) node[anchor=north west] {$n$};
%%	\draw (1.9,2.353333333333338) node[anchor=north west] {$\cdots$};
%%	\end{tikzpicture}
%%	\caption{Embedding of a point of $E_1$ into $E_2$.}
%%\end{figure}
%%\end{frame}
%
%\begin{frame}
%	\frametitle{Category operad $PaB$}
%\begin{itemize}
%	\item<1-> The objects of the category $PaB(n)$ are parenthesized permutations of the set $\{1,\dots, n\}$: $x_1(x_3x_2)$.
%	\item<2-> The morphisms between two objects whose permutations are $\sigma_1$ and $\sigma_2$ are $p^{-1}(\sigma_2^{-1}\sigma_1)$, where $p:B_n\to \Sigma_n$. %decir quién es este morfismo
%\end{itemize}
%\only<3>{\begin{figure}
%		\includegraphics[scale=0.45]{Imagenes/left}
%\end{figure}}
% \end{frame}
%\begin{frame}
%	\frametitle{Operad PaB}
%	\textbf{Insertion on objects}: $x_1(x_2x_3)\circ_2(x_3x_2)x_1=x_1(((x_4x_3)x_2)x_5)$.
%	
%	\only<2->{\textbf{Insertion on morphisms}:
%	\begin{figure}
%		\includegraphics[scale=0.6]{Imagenes/insercion.png}
%	\end{figure}}
%
%\begin{itemize}
%	\item<3-> The action of $\Sigma_n$ is by permutation on objects and trivial on morphisms.
%	\item<4-> The operad is generated by $x_1x_2$ on objects. On morphisms, it is generated by the generator $x\in B_2$ and the identity braid $i\in B_3$ between $(x_1x_2)x_3$ and $x_1(x_2x_3)$.
%\end{itemize}
%	
%	
%\end{frame}
%
%
%
%\begin{frame} 
%\frametitle{$C_*(E_2)\simeq H_*(E_2)$}
%	\begin{itemize}
%	\item Define an operad of categories $PaB$ s. t. $|N(PaB)|\simeq E_2$. \checkmark
%	\item Find another chain operad $C_*(O_A)$ and quasi-isomorphisms $C_*(|N(PaB)|)\to C_*(O_A)\leftarrow H_*(E_2)$. 
%\end{itemize}%QPaB y PaB son equivalentes como categoráis no enriquecidas porque los morfismos están en biyección (infinito numerable) y los objetos son los mismos. Como al nervio no le afecta lo de aditividad pues yastá
%\end{frame}
%
%
%\subsection{Operad $\mathcal{B}$}
%
%\begin{frame}
%\frametitle{Tensor coalgebra}
%
%	For a finite dimensional graded $k$-vector space $V$ let $TV=\bigoplus_{n\geq 0} V^{\otimes n}$. \pause
%	
%	\begin{defi}
%		A \textbf{coalgebra} structure on $TV$ is given by a comultiplication $\Delta:TV\to TV\otimes TV$. 
%	\end{defi}\pause
%
%\begin{example}
%	The \textbf{cofree conilpotent} coalgebra $TV$ is given by the comultiplication
%	\[
%	\Delta(w_0\otimes\cdots\otimes w_n)=\sum_{p=0}^n(w_0\otimes\cdots \otimes w_p)\otimes (w_{p+1}\otimes\cdots\otimes w_n).
%	\]
%\end{example}
%
%\end{frame}
%
%\begin{frame}
%\begin{defi}
%	A \textbf{bialgebra} structure on $TV$ is given by both an algebra structure and a coalgebra structure, which satisfy some compatibility axioms.
%\end{defi}\pause
%
%\begin{defi}
%	A $\mathcal{B}$-algebra structure on $V$  is given by an structure of differential graded bialgebra on $TV[1]=\bigoplus_{n\geq 0} V[1]^{\otimes n}$ such that the coalgebra structure is the cofree conilpotent one.
%\end{defi}
%\end{frame}
%
%\begin{frame}
%	\frametitle{Operad $\mathcal{B}$}
%	A $\mathcal{B}$-algebra structure on $V$ is given by:
%	\begin{itemize}
%			\item<2-> A differential $D:TV[1]\to TV[1]$ \only<3->{$\xRightarrow{decomposes} V[1]^{\otimes n}\to V[1]^{\otimes r}$ of degree 1} \only<4->{$\xRightarrow{r=1} m_n:V[1]^{\otimes n}\to V[2]$} \only<5->{$\Rightarrow \boxed{m_n:V^{\otimes n}\to V[2-n]}$.}
%		\item<6-> A multiplication $M:TV[1]\otimes TV[1]\to TV[1]$ \only<7->{$\xRightarrow{decomposes} V[1]^{\otimes p}\otimes V[1]^{\otimes q}\to V[1]^{\otimes r}$ of degree 0} \only<8->{$\xRightarrow{r=1} m_{pq}:V^{\otimes p}[1]\otimes V^{\otimes q}[1]\to V[1]$} \only<9->{$\Rightarrow	\boxed{m_{pq}:V^{\otimes p}\otimes V^{\otimes q}\to V[1-p-q]}$.}
%	
%		
%	\end{itemize}
%\end{frame}
%
%\begin{frame}
%	\frametitle{Operad $\mathcal{B}$}
%	\begin{defi}
%		The operad $\mathcal{B}$ is defined to be the operad of graded vector spaces generated by operations $m_n\in\mathcal{B}(n)_{2-n}$ and $m_{pq}\in\mathcal{B}(p+q)_{1-p-q}$ subject to some relations. 
%	\end{defi}
%\end{frame}
%	
%	\subsection{Action on $\mathcal{B}$ on $C^*(A;A)$}
%\begin{frame}
%	\frametitle{Brace algebra on $C^*(A;A)$}
%	
%\begin{itemize}
%	\item<1-> Recall that an element $f\in C^n(A;A)$ is a map $f:A^{\otimes n}\to A$, so we represent it as
%	 
%	\begin{tikzpicture}[line cap=round,line join=round,>=triangle 45,x=1.0cm,y=1.0cm]
%	\clip(-2.8466666666666676,-1) rectangle (11.486666666666668,3);
%	\draw[line width=1.pt] (0.,0.) -- (0.,1.38) -- (2.406666666666667,1.38) -- (2.406666666666667,0.) -- cycle;
%	\draw [line width=1.pt] (0.,0.)-- (0.,1.38);
%	\draw [line width=1.pt] (0.,1.38)-- (2.406666666666667,1.38);
%	\draw [line width=1.pt] (2.406666666666667,1.38)-- (2.406666666666667,0.);
%	\draw [line width=1.pt] (2.406666666666667,0.)-- (0.,0.);
%	\draw [line width=.pt] (1.2033333333333336,0.)-- (1.2033333333333336,-0.606666666666666);
%	\draw [line width=1.pt] (1.2033333333333336,1.38)-- (1.2033333333333336,2.3933333333333344);
%	\draw [line width=1.pt] (0.7666666666666668,2.3933333333333344)-- (0.7666666666666668,1.38);
%	\draw [line width=1.pt] (0.3933333333333334,2.3933333333333344)-- (0.38,1.38);
%	\draw [line width=1.pt] (1.606666666666667,2.3933333333333344)-- (1.606666666666667,1.38);
%	\draw [line width=1.pt] (1.9933333333333336,2.3933333333333344)-- (1.9933333333333336,1.38);
%	\draw (1.,0.9933333333333343) node[anchor=north west] {$f$};
%	\end{tikzpicture}
%\end{itemize}
%	
%\end{frame}
%
%
%\begin{frame}
%	\frametitle{Brace algebra on $C^*(A;A)$}
%	
%	\begin{itemize}
%		\item Let $f,g_1,\dots, g_n\in C^*(A;A)$. The the \textbf{brace} $f\{g_1,\dots, g_n\}$ is given by
%		
%		\begin{tikzpicture}[line cap=round,line join=round,>=triangle 45,x=1.0cm,y=1.0cm]
%		\clip(-2.8466666666666676,-1) rectangle (11.486666666666668,3);
%		\draw[line width=1.pt] (2.833333333333334,0.) -- (2.833333333333334,0.5666666666666675) -- (7.006666666666668,0.5666666666666675) -- (7.,0.) -- cycle;
%		\draw[line width=1.pt] (3.193333333333334,1.) -- (4.,1.) -- (4.006666666666668,1.58) -- (3.193333333333334,1.58) -- cycle;
%		\draw[line width=1.pt] (6.,1.) -- (6.38,1.) -- (6.38,1.58) -- (5.593333333333335,1.5933333333333344) -- (5.606666666666667,1.) -- cycle;
%		\draw (-2.9,0.6) node[anchor=north west] {$f\{g_1,\dots, g_n\}=\displaystyle{\sum_{\text{all possible insertions}}}$};
%		\draw [line width=1.pt] (2.833333333333334,0.)-- (2.833333333333334,0.5666666666666675);
%		\draw [line width=1.pt] (2.833333333333334,0.5666666666666675)-- (7.006666666666668,0.5666666666666675);
%		\draw [line width=1.pt] (7.006666666666668,0.5666666666666675)-- (7.,0.);
%		\draw [line width=1.pt] (7.,0.)-- (2.833333333333334,0.);
%		\draw [line width=1.pt] (4.92,0.)-- (4.92,-0.36666666666666614);
%		\draw [line width=1.pt] (2.993333333333334,0.5666666666666675)-- (3.,2.);
%		\draw [line width=1.pt] (3.553333333333334,0.5666666666666675)-- (3.553333333333334,1.);
%		\draw [line width=1.pt] (4.273333333333334,0.5666666666666675)-- (4.273333333333334,1.9933333333333345);
%		\draw [line width=1.pt] (4.54,0.5666666666666675)-- (4.54,1.9933333333333345);
%		\draw [line width=1.pt] (5.993333333333334,0.5666666666666675)-- (6.,1.);
%		\draw [line width=1.pt] (6.62,0.5666666666666675)-- (6.62,1.9933333333333345);
%		\draw [line width=1.pt] (6.833333333333335,0.5666666666666675)-- (6.833333333333335,1.98);
%		\draw [line width=1.pt] (3.193333333333334,1.)-- (4.,1.);
%		\draw [line width=1.pt] (4.,1.)-- (4.006666666666668,1.58);
%		\draw [line width=1.pt] (4.006666666666668,1.58)-- (3.193333333333334,1.58);
%		\draw [line width=1.pt] (3.193333333333334,1.58)-- (3.193333333333334,1.);
%		\draw [line width=1.pt] (6.,1.)-- (6.38,1.);
%		\draw [line width=1.pt] (6.38,1.)-- (6.38,1.58);
%		\draw [line width=1.pt] (6.38,1.58)-- (5.593333333333335,1.5933333333333344);
%		\draw [line width=1.pt] (5.593333333333335,1.5933333333333344)-- (5.606666666666667,1.);
%		\draw [line width=1.pt] (5.606666666666667,1.)-- (6.,1.);
%		\draw [line width=1.pt] (3.5666666666666673,1.58)-- (3.58,2.);
%		\draw [line width=1.pt] (6.007451656136321,1.586314378709553)-- (6.,2.);
%		\draw [line width=1.pt] (5.4,0.5666666666666675)-- (5.4,2.);
%		\draw (3.3,1.5666666666666678) node[anchor=north west] {$g_1$};
%		\draw (5.65,1.58) node[anchor=north west] {$g_n$};
%		\draw (4.7,0.5666666666666675) node[anchor=north west] {$f$};
%		\draw (4.59,1.5533333333333343) node[anchor=north west] {$\cdots$};
%		\end{tikzpicture}
%		\item<2-> $f\{\}=f$.
%	\end{itemize}
%\end{frame}
%
%\begin{frame}
%%	\begin{itemize}
%%		\item Notation: $f\{\}=f$ and $[f,g]=f\{g\}-(-1)^{(|f|-1)(|g|-1)}g\{g\}$.
%%	\end{itemize}
%\frametitle{Action of $\mathcal{B}$ on $C^*(A;A)$}
%We have to define the action of the operations $m_n$ and $m_{pq}$ on $C^*(A;A)$:
%
%\begin{itemize}
%	\item<2-> $m_0=0$.
%	\item<3->  $m_1$ is the differential in $C^*(A;A)$.
%	\item<4-> $m_2$ is the cup product.
%	\item<5-> $m_n=0$ for all $n>2$. 
%	\item<6-> $m_{0,1}=m_{1,0}=Id$.
%	\item<7-> $m_{0,q}=m_{q,0}=0$.
%	\item<8-> $m_{1q}(f\otimes g_1\otimes\cdots\otimes g_q)=f\{g_1,\dots, g_q\}$.
%	\item<9-> $m_{pq}=0$ for $p>1$.
%\end{itemize}
%
%\end{frame}
%
%\begin{frame}
%	\begin{teorema}
%		The action of $\mathcal{B}$ on $C^*(A;A)$ is well defined and induces the Gerstenhaber algebra structure on $H^*(A;A)$. 
%	\end{teorema}\pause
%
%\textcolor{red}{The conjecture is finally proven!}
%\end{frame}
%
%\begin{frame}
%	\begin{center}
%	\Huge{Thank you very much!}
%\end{center}
%\end{frame}
%%\begin{frame}
%%	Cada vez que explique una cosa ponerle un check a lo de antes \url{https://tex.stackexchange.com/questions/132783/how-to-write-checkmark-in-latex} (quizá en operads ponerlo al final al grande y ya)
%%\end{frame}
%
%%\begin{frame}
%%	Definición de operad y explicación dibujitos (más de los que he hecho en el trabajo)
%%	
%%	Operad de endomorfismos y álgebra sobre un operad
%%	
%%	Definición de operad en symmetric monoidal categories para que tenga sentido
%%	
%%	Comentar gracias al EZ map se hereda la operadición y de ahí a homología
%%	
%%	Las operaciones de la homología con los dos dibujitos (en el trabajo solo he metido uno)
%%	
%%	Gerstenhaber algebra definición del tirón (en la presentación comentar los criterios de derivación)
%%	
%%	Describir el complejo de cadenas sin detallar mucho en que es un complejo de cadenas e ir a su homología con sus propiedades de álgebra de Gerstenhaber (esto ya me relaciona con el último punto)
%%	
%%	Recuperar el homology operad y describir la acción sobre un álgebra asociativa
%%\end{frame}
%
%%\begin{frame}
%%	Retomar la conjetura de Deligne para recordarla y ver que está todo, seguido de un diagrama con las acciones y la que se pregunta si existe poniéndola dashed y con una interrogación de label
%%	
%%	Esquema de la prueba (destacar de algún modo las partes en las que me centro)
%%	
%%	Pensar qué meto de cada parte
%%\end{frame}
%
%%\begin{frame}
%%
%%\end{frame}

\end{document}
