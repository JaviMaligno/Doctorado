\documentclass{beamer}
\usepackage[utf8]{inputenc}
\usetheme{Copenhagen}
%\usepackage[spanish]{babel}
\usepackage{multirow}
%\usepackage{estilo-apuntes}
\usepackage{braids}
\usepackage[]{graphicx}
\usepackage{rotating}
\usepackage{pgf,tikz}
\usepackage{pgfplots}
\usepackage{tikz-cd}
%\usepackage{empheq}
%\usepackage[dvipsnames]{xcolor}
\usepackage{xcolor}

\usetikzlibrary{arrows}
\usetikzlibrary{cd}
\usetikzlibrary{babel}
\pgfplotsset{compat=1.13}
\usetikzlibrary{decorations.shapes}
\pgfkeyssetvalue{/tikz/braid height}{1cm} %no parece hacer nada
\pgfkeyssetvalue{/tikz/braid width}{1cm}
\pgfkeyssetvalue{/tikz/braid start}{(0,0)}
\pgfkeyssetvalue{/tikz/braid colour}{black}

\theoremstyle{definition}

\newtheorem{teorema}{Theorem}
\newtheorem{defi}{Definition}
\newtheorem{prop}[teorema]{Proposition}

\newcommand{\Z}{\mathbb{Z}}
\newcommand{\Q}{\mathbb{Q}}
\newcommand{\C}{\mathbb{C}}
\newcommand{\CC}{\mathcal{C}}
\newcommand{\D}{\mathbb{D}}
\providecommand{\gene}[1]{\langle{#1}\rangle}

\DeclareMathOperator{\im}{im}


\addtobeamertemplate{navigation symbols}{}{%
    \usebeamerfont{footline}%
    \usebeamercolor[fg]{footline}%
    \hspace{1em}%
    %\insertframenumber/\inserttotalframenumber
}
\setbeamercolor{footline}{fg=black}
\setbeamerfont{footline}{series=\bfseries}

\newcommand{\highlight}[1]{%
	\colorbox{red!50}{$\displaystyle#1$}}

\makeatletter
\newcommand*{\encircled}[1]{\relax\ifmmode\mathpalette\@encircled@math{#1}\else\@encircled{#1}\fi}
\newcommand*{\@encircled@math}[2]{\@encircled{$\m@th#1#2$}}
\newcommand*{\@encircled}[1]{%
	\tikz[baseline,anchor=base]{\node[draw,circle,outer sep=0pt,inner sep=.2ex] {#1};}}
\makeatother

\expandafter\def\expandafter\insertshorttitle\expandafter{%
  \insertshorttitle\hfill%
  \insertframenumber\,/\,\inserttotalframenumber}

%-----------------------------------------------------------

\title{Homotopy Associativity of H-spaces}
\author{Javier Aguilar Mart\'in}
\institute{University of Kent}
\date{}
 
\begin{document}
\frame{\titlepage}
%\begin{frame}
%
%c¡
%\title[About Beamer] %optional
%{About the Beamer class in presentation making}
% 
%\subtitle{A short story}
% 
%\author[Arthur, Doe] % (optional, for multiple authors)
%{A.~B.~Arthur\inst{1} \and J.~Doe\inst{2}}
% 
%\institute[VFU] % (optional)
%{
%  \inst{1}%
%  Faculty of Physics\\
%  Very Famous University
%  \and
%  \inst{2}%
%  Faculty of Chemistry\\
%  Very Famous University
%}

% 
%\date[VLC 2013] % (optional)
%{Very Large Conference, April 2013}


%\end{frame}
\setbeamercovered{highly dynamic}

\newcounter{saveenumi}
\newcommand{\seti}{\setcounter{saveenumi}{\value{enumi}}}
\newcommand{\conti}{\setcounter{enumi}{\value{saveenumi}}}

\makeatletter
\newcommand{\xRightarrow}[2][]{\ext@arrow 0359\Rightarrowfill@{#1}{#2}}
\makeatother

\resetcounteronoverlays{saveenumi}
%\AtBeginSection[]{
%\begin{frame}
%\frametitle{Tabla de contenidos}
%\tableofcontents
%\end{frame}
%}

%\begin{frame}
%	%AÑADIR ESTAS URL AL FINAL POR SI ME DA TIEMPO ENSEÑAR ESTAS COSAS 
%	%\url{https://www.blockchain.com/btc/blocks}
%	%\url{https://coin.dance/blocks}
%	%\url{https://www.blockchain.com/btc/unconfirmed-transactions}
%	
%%	ESTE TENGO PRIMERO QUE MIRARLO PARA HACERLO YO
%%	\url{https://anders.com/blockchain/}
%
%\end{frame}




%\begin{frame}
%	\begin{itemize}
%		\item Operads
%		\begin{itemize}
%			\item action of an operad (algebra over an operad)
%			\item little disks operad
%			\item chain/homology operad			
%		\end{itemize}
%	\item Gerstenhaber algebras
%	\item Hochschild cohomology of an associative algebra
%	\end{itemize}
%\end{frame}

\section{Introduction}

\subsection{Background}

\begin{frame}[fragile]
\frametitle{H-spaces}
%THE HOMOTOPY TYPE IS OF COUNTABLE CW-COMPLEXES AND MAPS AND HOMOTOPIES PRESERVE BASE POINTS
\begin{defi}
A topological space $X$ is an \emph{H-space} if if there exists a multiplication map $m:X\times X\to X$ and an unit element $e\in X$ such that $m(x,e)=x=m(e,x)$ for all $x\in X$. %a multiplication and a unit
\end{defi}\pause	
H-spaces generalize topological groups.\pause
%the relevant property for homotopy theory is associativity
\begin{lemma}[Sugawara]
If $X$ and $X\times X$ are CW complexes and $X$ is a homotopy associative H-spaces, then $X$ has inverses.
\end{lemma}

\end{frame}

\begin{frame}[fragile]

%to make sure that we are all in the same page
The multiplication $m:X\times X\to X$ is said to be \emph{homotopy associative} if the following diagram commutes up to homotopy
\[
\begin{tikzcd}
X\times X\times X\arrow[r,"m\times 1"]\arrow[d,"1\times m"']& X\times X\arrow[d,"m"]\\
X\times X\arrow[r,"m"]& X
\end{tikzcd}
\]\pause

In othwer words, there is a homotopy $m(m\times 1)\simeq m(1\times m)$.
\end{frame}




\begin{frame}
\frametitle{Examples of H-spaces}
\begin{itemize}
\item<1-> All topological groups, in particular $S^0$, $S^1$ and $S^3$ (real, complex and quaternionic numbers).
\item<2-> The Cayley (octonionic) numbers $S^7$.
\item<3-> Any loop space is a homotopy associative H-space via concatenation of loops. %say something about this
\end{itemize}
%I focus on the first two items, i give these examples because they are key in the generalization of the following construction
	
\end{frame}

\begin{frame}
\frametitle{Projective spaces of topological groups}
Milnor defines projective spaces for any topological group $G$ in the following way:\pause
\begin{itemize}
\item<2-> Define $E_i=G\star\cdots\star G$ to be the $i$-fold join of $G$ with itself. 
\item<2-> Construct fibre bundles $p_i:E_i\to B_i$ as quotient maps by the natural action of $G$ on $E_i$.
\end{itemize}\pause
\begin{defi}
The $n$-fold join $X_1\star\cdots\star X_n$ is the space of formal linear combinations $t_1x_1+\cdots+x_nt_n$ with $0\leq t_i\leq 1$ and $t_1+\cdots+t_n=1$. This can be seen as a quotient of the space $X_1\times\cdots\times X_n\times\Delta^{n-1}$, where $\Delta^{n-1}$ is the $(n-1)$-dimensional simplex.
\end{defi}
%It can be checked that this is well defined (X\star Y\star Z=(X\star Y)\star Z are homeomorphic)
%this can also be seen as the (n-1)-dimensional segments between the spaces. For instance X\star Y is the space of segments from X to Y

\end{frame}

\begin{frame}
\begin{block}{Remark}
If $G=S^{d-1}$ for $d=1,2,4$ we get the standard fibrations $S^{id-1}\to P^{i-1}$ onto the projective space of dimension $i-1$.
\end{block}\pause
\begin{block}{Remark}
This construction does not work for $S^7$ since only the fibrations $S^7\to *$ and $S^{15}\to S^8$ can be constructed.
\end{block}\pause


Sugawara and Dold and Lashof generalize the construction to homotopy associative spaces loosening the fibration property.
%from this Stasheff define the following structure
\end{frame}
\subsection{$A_n$-structures}
\begin{frame}[fragile]
\begin{defi}
An $A_n$-\emph{structure} on a space $X$ consists of an $n$-tuple of maps
\[
\begin{tikzcd}
X \arrow[r, equals] & E_1\arrow[r, hook]\arrow[d, "p_1"] & E_2\arrow[r, hook]\arrow[d, "p_2"] &\cdots\arrow[r, hook]& E_n\arrow[d, "p_n"] \\
*\arrow[r, equals] & B_1\arrow[r, hook] & 
B_2\arrow[r, hook]&\cdots\arrow[r, hook]& B_n
\end{tikzcd}
\]
such that $p_{i*}:\pi_q(E_i,X)\to \pi_q(B_i)$ is an isomorphism for all $q$, together with a contracting homotopy $h:CE_{n-1}\to E_n$ such that $h(CE_{i-1})\subset E_i$. %the last condition on the cone is technical, coming from the construction of Dold-Lashof. The indices are different because it is talking about the subspaces. 
%these maps are called quasi-isomorphisms
\end{defi}
\end{frame}

\begin{frame}
\begin{prop}
%not fibration but close
Given $X\subset E$ and a map $p:(E,X)\to (B,*)$ such that $p_*:\pi_q(X,E)\to \pi_q(B)$ is an isomorphism for all $q$, there exists an equivalent fibration $F\to W_p\xrightarrow{\bar{p}} B$ such that $F$ has the homotopy type of $X$. 
\end{prop}\pause
We say that $p$ is equivalent to $\bar{p}$ if there is a map $i:E\to W_p$ such that $\bar{p}\circ i=p$.
 
\end{frame}
\begin{frame}
\frametitle{Proof of the proposition}
\begin{itemize}
\item<1-> Standard fact: every map is equivalent to a fibration, so that we have $\bar{p}\circ i=p$ %this can be constructed explicitely https://en.wikipedia.org/wiki/Fibration
\item<2-> Since $p(X)=*$, $X$ is mapped into $F=p^{-1}(*)$.
\item<3-> From the induced map of exact sequences of the pairs $(E,X)$ and $(W_p,F)$ we conclude that $F$ has the homotopy type of $X$ (recall that all are CW-complexes). 
\end{itemize}

\end{frame}

\begin{frame}
\[
\begin{array}{ccc}
\text{Topological groups} & \rightarrow & \text{Milnor construction}\\
? & \rightarrow  & A_n-structures\\
& &
\end{array}
\]
\pause
An $A_n$-structure induces maps $M_i:I^{i-2}\times X^i\to X$ for $i<n$ in such a way that an $A_2$-space is an H-space and a $A_3$-space is a homotopy associative H-space. %the arrows are actually two-sided, it is possible to study topological groups using the classifying bundle, which is essentially Milnor construction, and the other arrow is given by theorem 5
\end{frame}
\section{Associahedra}
\begin{frame}
\frametitle{Associahedra}
\begin{itemize}
\item<1->We need to define cell complexes $K_i$ homeomorphic to $I^{i-2}$ called \emph{Stasheff associahedra}. %nowadays, not in the article
\item<2-> Consider a word $x_1\cdots x_i$ and all meaningful ways to insert a set of parentheses.
\item<3-> To each such insertion except for $(x_1\cdots x_i)$ corresponds a cell of the boundary $L_i$ of $K_i$.
\item<4-> If the parentheses enclose $x_k$ to $x_{k+s-1}$ we regard this cell as the homeomorphic image of $K_r\times K_s$ ($r+s=i+1$) under a map $\delta_k(r,s)$.
\item<5-> To such cells intersect only on their boundaries and the ``edges'' so formed correspond to intersecting two sets of parentheses in the word. %this provides boundary conditions that I'm not going to write explicitly
 \end{itemize}
\end{frame}



\begin{frame}
\frametitle{Associahedra}
This is enough to construct $K_i$ by induction:
\begin{itemize}
\item[1]<1-> Start with $K_2=*$.
\item[2]<2-> Given $K_2$ through $K_{i-1}$, construct $L_i$ fitting copies of $K_r\times K_s$ as indicated above.
\item[3]<3-> Take $K_i=CL_i$.
\end{itemize}
\end{frame}

\begin{frame}
\frametitle{Associahedra}
\[
\begin{tikzpicture}[line cap=round,line join=round,>=triangle 45,x=1.0cm,y=1.0cm]
\clip(-1.1761983471074378,-2.3030578512396707) rectangle (7.443801652892559,4.266942148760331);
\draw (2.5,2.)-- (5.,2.);
\draw (0.016528925618,2.96942148760331) node[anchor=north west] {$K_2$};
\draw (0.538016528925617,1.9169421487603309) node[anchor=north west] {$x_1x_2$};
\draw (2.153801652892561,1.986942148760331) node[anchor=north west] {$(x_1x_2)x_3$};
\draw (4.583801652892561,1.946942148760331) node[anchor=north west] {$x_1(x_2x_3)$};
\draw (2.0338016528925613,2.69421487603312) node[anchor=north west] {$\delta_1(2,2)$};
\draw (4.64380165289256,2.6942148760331) node[anchor=north west] {$\delta_2(2,2)$};
\draw (3.6738016528925606,3) node[anchor=north west] {$K_3$};
\begin{scriptsize}
\draw [fill=black] (1.,2.) circle (2.5pt);
\draw [fill=black] (2.5,2.) circle (2.5pt);
\draw [fill=black] (5.,2.) circle (2.5pt);
\end{scriptsize}
\end{tikzpicture}
\]
\end{frame}

\begin{frame}[fragile]
\frametitle{Associahedra}
$K_4$
\[
\definecolor{ffqqqq}{rgb}{1.,0.,0.}
\definecolor{uuuuuu}{rgb}{0.26666666666666666,0.26666666666666666,0.26666666666666666}
\begin{tikzpicture}[line cap=round,line join=round,>=triangle 45,x=1.0cm,y=1.0cm]
\clip(-4.770495867768595,-0.8872727272727274) rectangle (5.,5.085454545454545);
\fill [fill=black,fill opacity=0.10000000](-1.5,0.) -- (1.5,0.) -- (2.4270509831248424,2.85316954888546) -- (0.,4.616525305762879) -- (-2.427050983124842,2.853169548885461) -- cycle;
\draw (-1.5,0.)-- (1.5,0.);
\draw (1.5,0.)-- (2.4270509831248424,2.85316954888546);
\draw (2.4270509831248424,2.85316954888546)-- (0.,4.616525305762879);
\draw (0.,4.616525305762879)-- (-2.427050983124842,2.853169548885461);
\draw (-2.427050983124842,2.853169548885461)-- (-1.5,0.);
\draw [color=ffqqqq](-0.9,5.272727272727) node[anchor=north west] {$x_1(x_2(x_3x_4))$};
\draw [color=ffqqqq](2.5,3.2) node[anchor=north west] {$x_1((x_2x_3)x_4)$};
\draw [color=ffqqqq](1.638595041322314,0.08545454545454524) node[anchor=north west] {$(x_1(x_2x_3))x_4$};
\draw [color=ffqqqq](-3.6432231404958677,0.08545454545454524) node[anchor=north west] {$((x_1x_2)x_3)x_4$};
\draw [color=ffqqqq](-4.77685950413223,3.2) node[anchor=north west] {$(x_1x_2)(x_3x_4)$};
\draw (1.2749586776859503,4.094545454545454) node[anchor=north west] {$x_1(x_2x_3x_4)$};
\draw (2.0840495867768594,1.5490909090909089) node[anchor=north west] {$x_1(x_2x_3)x_4$};
\draw (-0.5,-0.014545454545454752) node[anchor=north west] {$(x_1x_2x_3)x_4$};
\draw (-4,1.558181818181818) node[anchor=north west] {$(x_1x_2)x_3x_4$};
\draw (-2.8,4.5) node[anchor=north west] {$x_1x_2(x_3x_4)$};
\draw (0.3,3.6309090909090904) node[anchor=north west] {$\delta_2(2,3)$};
\draw (0.6,1.7309090909090905) node[anchor=north west] {$\delta_2(3,2)$};
\draw (-0.3886776859504132,0.5490909090909089) node[anchor=north west] {$\delta_1(2,3)$};
\draw (-1.950413,1.9127272727272724) node[anchor=north west] {$\delta_1(3,2)$};
\draw (-1.413222,3.7272727272727) node[anchor=north west] {$\delta_3(3,2)$};
\begin{scriptsize}
\draw [fill=ffqqqq] (-1.5,0.) circle (2.5pt);
\draw [fill=ffqqqq] (1.5,0.) circle (2.5pt);
\draw [fill=ffqqqq] (2.4270509831248424,2.85316954888546) circle (2.5pt);
\draw [fill=ffqqqq] (0.,4.616525305762879) circle (2.5pt);
\draw [fill=ffqqqq] (-2.427050983124842,2.853169548885461) circle (2.5pt);
\end{scriptsize}
\end{tikzpicture}
\]

\end{frame}
\begin{frame}
\frametitle{Associahedra}
$K_5$
\[
\begin{tikzpicture}[line cap=round,line join=round,>=triangle 45,x=1.0cm,y=1.0cm]
\clip(-3.63,-1.8) rectangle (4.,3);
\draw(-0.5,0.) -- (0.,0.5) -- (-0.5,1.) -- (-1.,0.5) -- cycle;
\draw (-0.5,0.)-- (0.,0.5);
\draw (0.,0.5)-- (-0.5,1.);
\draw (-0.5,1.)-- (-1.,0.5);
\draw (-1.,0.5)-- (-0.5,0.);
\draw (-0.5,1.)-- (-0.76,2.87);
\draw (-0.76,2.87)-- (-2.13,1.59);
\draw (-2.13,1.59)-- (-1.98,0.82);
\draw (-2.13,1.59)-- (-2.5,1.);
\draw (-2.5,1.)-- (-2.31,0.29);
\draw (-2.31,0.29)-- (-1.98,0.82);
\draw (-1.98,0.82)-- (-1.,0.5);
\draw (0.,0.5)-- (0.93,0.89);
\draw (0.93,0.89)-- (0.98,1.68);
\draw (1.37,1.06)-- (0.98,1.68);
\draw (-0.76,2.87)-- (0.98,1.68);
\draw (-2.31,0.29)-- (-0.62,-1.3);
\draw (-0.5,0.)-- (-0.62,-1.3);
\draw (-0.62,-1.3)-- (1.22,0.35);
\draw (0.93,0.89)-- (1.22,0.35);
\draw (1.22,0.35)-- (1.37,1.06);
\draw [dash pattern=on 2pt off 2pt] (-2.5,1.)-- (1.37,1.06);
\end{tikzpicture}
\]
The squares correspond to $\delta_k(3,3)$ and the pentagons to $\delta_k(4,2)$ or $\delta_k(2,4)$.
\end{frame}

\begin{frame}
\begin{prop}
$K_i$ is homeomorphic to $I^{i-2}$. Degenerancy maps $s_j:K_i\to K_{i-1}$ for $1\leq j\leq i$ can be defined so that 
\begin{itemize}
\item $s_js_k=s_ks_{j+1}$ for $k\leq j$
\item boundary conditions and compatibility with the homeomorphisms $\delta_k(r,s)$.
% \begin{align*}
%s_j\delta_k(r,s)=\delta_{k-1}(r-1,s)(s_j\times 1)\\
%s_j\delta_k(r,s)
%\end{align*}

%thisi conditions are important because make the main theorem consistent but are too many and not very self-explanatory
\end{itemize}
%the homeomorphism with I^{i-2} is not hard to see. the other part is like saying that K_i can be seen as an alternative combinatorial model to simplices and cubes %compatibility conditions ensure that the conditions in the main theorem are consistent
\end{prop}\pause

This proposition is proved by explicitly constructing the complexes $K_i$ as subsets of $I^{i-2}$.
\end{frame}
\subsection{$K_i$ as a subset of $I^{i-2}$}
\begin{frame}
\frametitle{$K_i$ as a subset of $I^{i-2}$}
\begin{itemize}
\item<1-> Let $K_i$ be the subset of $I^{i-2}$ consisting of points $(t_1,\dots, t_{i-2})$ such that $2^jt_1\cdots t_j\geq 1$ for $1\leq j\leq i-2$.
\item<2-> The boundary $L_i$ of $K_i$ consists of the points such that for some $j$ either $2^jt_1\cdots t_j= 1$ or $t_j=1$ %t_j=1 is the boundary of the cube
\end{itemize}
\end{frame}
\begin{frame}
\begin{figure}
\includegraphics[scale=0.25]{k234}
\end{figure}
\end{frame}
\begin{frame}
\begin{figure}
\includegraphics[scale=0.15]{k5}
\end{figure}
\end{frame}
\begin{frame}
\begin{itemize}
\item<1-> The degenerancy maps $s_j:K_{i+1}\to K_i$ can be defined on $L_{i+1}$ using the boundary and compatibility conditions
\item<2->The map $s_j$ is defined on all $K_{i+1}$ by ``taking the cone'':
\item[1]<3-> Represent $K_{i+1}$ as pairs $(t,\tau)$ with $\tau\in L_{i+1}$ and similarly with $K_i$.
\item[2]<4-> If $s_j(\tau)=(s,\tau')$ with $\tau'\in L_i$, define $s_j(t,\tau)=(st,\tau')$. %K_i was actually defined as a cone of L_i 
%This satisfies the conditions
\end{itemize}
\end{frame}
\section{$A_n$-forms}
\begin{frame}
\frametitle{$A_n$-forms}
\begin{theorem}
A space $X$ admits an $A_n$-structure if and only if there exist maps $M_i:K_i\times X^i\to X$ for $2\leq i\leq n$ such that:
\begin{itemize}
\item[(1)] $M_2(*,e,x)=M_2(*,x,e)=x$ for $x\in X$. %existence of unit, K_2=*
\item[(2)] for $\rho\in K_r$, $\sigma\in K_s$ ($r+s=i+1$):
\begin{align*}
&M_i(\delta_k(r,s)(\rho,\sigma),x_1,\dots, x_i)=\\
&M_r(\rho,x_1,\dots, x_{k-1},M_s(\sigma,x_k,\dots, x_{k+s-1}),\dots, x_i)
\end{align*} %this is actually a boundary condition
\item[(3)] for $\tau\in K_i$ and $i>2$:
\begin{align*}
&M_i(\tau,x_1,\dots, x_{j-1},e,x_{j+1},\dots, x_i)=\\
&M_{i-1}(s_j(\tau),x_1,\dots, x_{j-1},x_{j+1},\dots, x_i)
\end{align*}%this condition is consistent with the previous one thanks to the compatibility of the degenerancy maps
\end{itemize}
\end{theorem}
\end{frame}

\begin{frame}
We call such a set of maps an $A_n$-\emph{form} and the pair $(X,\{M_i\})$ an $A_n$-\emph{space}.\pause
\begin{block}{Remark}
An $A_2$-space is just an H-space and $M_2(*,x,y)$ is often written as $xy$.
\end{block}\pause
Let's unravel condition (2).\pause
\begin{itemize}
\item[(2)] for $\rho\in K_r$, $\sigma\in K_s$ ($r+s=i+1$):
\begin{align*}
&M_i(\delta_k(r,s)(\rho,\sigma),x_1,\dots, x_i)=\\
&M_r(\rho,x_1,\dots, x_{k-1},M_s(\sigma,x_k,\dots, x_{k+s-1}),\dots, x_i)
\end{align*} %this is actually a boundary condition
\end{itemize}
\end{frame}
\begin{frame}
For $s=2$ the equation becomes
\[
M_i(\delta_k(i-1,2)(\rho,*),x_1,\dots, x_i)=M_{i-1}(\rho,x_1,\dots,x_kx_{k+1},\dots, x_i)
\]
\begin{itemize}
\item<2-> For $i=3$, $K_3\cong I$ so $M_3:I\times X^3\to X$ is a homotopy between $M_2(M_2\times 1)$ and $M_2(1\times M_2)$, that is, between $(xy)z$ and $x(yz)$.
\item[]<3-> Thus, $M_2$ is a homotopy associative multiplication.
\end{itemize}

\end{frame}

\begin{frame}
\begin{itemize}
\item For $i=4$, consider the five ways of associating a product of four factors. Since $M_2$ is homotopy associative these five products are related by a string of homotopies 

\[
\begin{tikzpicture}[line cap=round,line join=round,>=triangle 45,x=1.0cm,y=1.0cm]
\clip(-1.5,0.5) rectangle (5.2,4.6);
\draw(1.,1.) -- (3.,1.) -- (3.618033988749895,2.9021130325903064) -- (2.,4.077683537175253) -- (0.3819660112501053,2.9021130325903073) -- cycle;
\draw (1.,1.)-- (3.,1.);
\draw (3.,1.)-- (3.618033988749895,2.9021130325903064);
\draw (3.618033988749895,2.9021130325903064)-- (2.,4.077683537175253);
\draw (2.,4.077683537175253)-- (0.3819660112501053,2.9021130325903073);
\draw (0.3819660112501053,2.9021130325903073)-- (1.,1.);
\draw (1.3,4.7) node[anchor=north west] {$x(y(zt))$};
\draw (3.6,3.25) node[anchor=north west] {$x((yz)t)$};
\draw (3,1.1) node[anchor=north west] {$(x(yz))t$};
\draw (-0.5,1.1) node[anchor=north west] {((xy)z)t};
\draw (-1.15,3.25) node[anchor=north west] {(xy)(zt)};
\draw (2.8,3.8) node[anchor=north west] {$\simeq$};
\draw (3.3333333333336,2.1) node[anchor=north west] {$\simeq$};
\draw (1.8,0.8933333333333304) node[anchor=north west] {$\simeq$};
\draw (0.15,2.1) node[anchor=north west] {$\simeq$};
\draw (0.6,3.8) node[anchor=north west] {$\simeq$};
\begin{scriptsize}
\draw [fill=black] (1.,1.) circle (2.5pt);
\draw [fill=black] (3.,1.) circle (2.5pt);
\draw [fill=black] (3.618033988749895,2.9021130325903064) circle (2.5pt);
\draw [fill=black] (2.,4.077683537175253) circle (2.5pt);
\draw [fill=black] (0.3819660112501053,2.9021130325903073) circle (2.5pt);
\end{scriptsize}
\end{tikzpicture}
\]

\item[]<2->This defines a map $S^1\times X^4\to X$. The map $M_4$ can be regarded as an extension of this map to $I^2\times X^4$. %say something about homotopy coherence
\end{itemize}

\end{frame}




\begin{frame}
	Condition (3) in the definition of $A_n$-form is useful but offers no restriction:
	\begin{lemma}
	Suppose $\{M_i, i<n\}$ is an $A_{n-1}$-form and that $M'_n:K_n\times X^n\to X$ satisfies (2), then there exists $M_n:K_n\times X^n\to X$ satisfying (2) and (3).  
	\end{lemma}	
\end{frame}



\begin{frame}
\begin{block}{Examples}
\begin{itemize}
\item<1->Any associative H-spaces admits an $A_n$-form for every $n$ by defining $M_i(\tau,x_1,\dots,x_i)=x_1x_2\cdots x_i$. We call this a \emph{trivial} $A_n$-form.
\item<2-> Any loop space admits $A_n$-forms for all $n$.
\end{itemize}
\end{block}

\end{frame}


\section{Derived $A_n$-forms}
%this is one direction of the theorem: A_n-structure implies A_n-form
\subsection{Proof of the main theorem (i)}
\begin{frame}
\frametitle{Proof of the theorem (if)}
We construct $A_n$-structures $p_i:\mathcal{E}_i\to \mathcal{B}_i$ in which $\mathcal{E}_i$ has the homotopy type of $X\star\cdots\star X$ but using $K_{i+1}\times X^i$ instead of $\Delta^{i-1}\times X^i$.
\begin{itemize}
\item<2-> Let $X^{[i]}$ the subspace of $X^i$ of points in which at least one coordinate equals $e$.
\item<3-> Let $R=L_{i+1}\times X^i\cup K_{i+1}\times X\times X^{[i-1]}$ as a subspace of $K_{i+1}\times X^i$.
\item<4-> Let $S=L_{i+1}\times X^{i-1}\cup K_{i+1}\times X^{[i-1]}$ as a subspace of $K_{i+1}\times X^{i-1}$.
\end{itemize}
\end{frame}

\begin{frame}
\begin{block}{Construction}
\begin{itemize}
\item Let $\mathcal{E}_1=X$. Define $\mathcal{E}_i$ for $1<i\leq n$ by means of relative homeomorphisms %homeomorphism between the complements
\[
(K_{i+1}\times X^i,R)\xrightarrow{\alpha_i} (\mathcal{E}_i,\mathcal{E}_{i-1})
\]
where $\alpha_i$ is defined on $R$ by %it is needed that \alpha_i(R)\subseteq \mathcal{E}_{i-1}
\begin{align*}
\alpha_i(\delta_k(r,s)(\rho,\tau),x_1,\dots,x_i)&=\\
\alpha_{r-1}(\rho,x_1,\dots,&M_s(\sigma,x_k,\dots, x_{k+s-1}),\dots, x_i)
\end{align*}
\begin{align*}
\alpha_i(\tau,x_1,\dots,x_{j-1},e,x_{j+1},\dots, x_i)&=\\
\alpha_{i-1}(s_j(\tau),x_1,\dots,& x_{j-1},x_{j+1},\dots, x_i)
\end{align*}
\item<2->$\mathcal{E}_i=K_{i+1}\times X^i\cup_{\alpha_i}\mathcal{E}_{i-1}$.%this is not really mentioned in the article but I think it's the way
\end{itemize}
\end{block}
\end{frame}

\begin{frame}
\begin{block}{Remark}
Recall that $\mathcal{E}_1=X$ and $K_3\times X^2\cong I\times X^2$.
Then \[\alpha_2:(I\times X^2, X^2\times\{0,1\}\cup I\times X\times e)\to (\mathcal{E}_2,\mathcal{E}_1)\] identifies $0\times X^2$ with $X$ by the map $(0,x,y)\mapsto x$; $1\times X^2$ with $X$ by the map $(0,x,y)\mapsto xy$; and $I\times X\times e$ with $X$ in the obvious way. %I don't have an explicit description of E_2 but it is homeomorphic to the specified quotient

\end{block}

\end{frame}
\begin{frame}
\[
\begin{tikzpicture}[line cap=round,line join=round,>=triangle 45,x=1.0cm,y=1.0cm]
\clip(-3.88,-1.62) rectangle (6.74,4.95);
\fill[fill=black,fill opacity=0.10000000149011612] (0.,0.) -- (2.,0.) -- (2.5,0.5) -- (0.5,0.5) -- cycle;
\fill[fill=black,fill opacity=0.10000000149011612] (0.,0.) -- (0.,2.5) -- (0.5,3.) -- (0.5,0.5) -- cycle;
\fill[fill=black,fill opacity=0.10000000149011612] (0.,2.5) -- (2.,2.5) -- (2.5,3.) -- (0.5,3.) -- cycle;
\draw (0.,0.)-- (2.,0.);
\draw (2.,0.)-- (2.5,0.5);
\draw (2.5,0.5)-- (0.5,0.5);
\draw (0.5,0.5)-- (0.,0.);
\draw (0.,0.)-- (0.,2.5);
\draw (0.,2.5)-- (0.5,3.);
\draw (0.5,3.)-- (0.5,0.5);
\draw (0.5,0.5)-- (0.,0.);
\draw (0.,2.5)-- (2.,2.5);
\draw (2.,2.5)-- (2.5,3.);
\draw (2.5,3.)-- (0.5,3.);
\draw (0.5,3.)-- (0.,2.5);
\draw (2.,2.5)-- (2.,0.);
\draw (2.5,3.)-- (2.5,0.5);
\draw (0.8,3.8) node[anchor=north west] {$1\times X^2$};
\draw (0.5,-0.13) node[anchor=north west] {$0\times X^2$};
\draw (-1.9,1.71) node[anchor=north west] {$I\times X\times e$};
\draw (1.1,3) node[anchor=north west] {$xy$};
\draw (1.,0.5) node[anchor=north west] {$x$};
\draw (0.,1.59) node[anchor=north west] {$x$};
\end{tikzpicture}
\]
\end{frame}

\begin{frame}
\begin{block}{Construction}
\begin{itemize}
\item Let $\mathcal{B}_1=*$. Define $\mathcal{B}_i$ for $1<i\leq n$ by means of relative homeomorphisms %homeomorphism between the quotients
\[
(K_{i+1}\times X^{i-1},S)\xrightarrow{\beta_i} (\mathcal{B}_i,\mathcal{B}_{i-1})
\]
where $\beta_i$ is defined on $S$ by the same equations but replacing $\alpha$ with $\beta$ and omitting $x_1$.
\item<2-> $\mathcal{B}_i=K_{i+1}\times X^{i-1}\cup_{\beta_i}\mathcal{B}_{i-1}$.
\end{itemize}
\end{block}
\end{frame}

\begin{frame}
\begin{block}{Remark}
$\mathcal{B}_1=*$ and $\mathcal{B}_2=SX$ (reduced suspension). %this time there is an explicit description
The identifications work similarly as the previous case, but in this case the subspace $S\subset K_3\times X$ is identified with a point.
\end{block}
\[
\definecolor{ffqqqq}{rgb}{1.,0.,0.}
\begin{tikzpicture}[line cap=round,line join=round,>=triangle 45,x=1.0cm,y=1.0cm]
\clip(-2.19,-1) rectangle (4,2.5);
\draw(0.,0.) -- (0.,2.) -- (1.5,2.) -- (1.5,0.) -- cycle;


\draw [color=ffqqqq, line width=1pt] (0.,2.)-- (1.5,2.);
\draw [color=ffqqqq, line width=1pt] (0.,0.)-- (1.5,0.);
\draw [color=ffqqqq, line width=1pt] (0.,2.)-- (0.,0.);
\draw (0.2,2.62) node[anchor=north west] {$1\times X$};
\draw (0.2,-0.03) node[anchor=north west] {$0\times X$};
\draw (-1.1,1.29) node[anchor=north west] {$I\times e$};
\end{tikzpicture}
\]
%identifying top and bottom gives suspension, identifying another segment is reduced suspension. In this case the quotient is by a point so you get the space automatically
\end{frame}

\begin{frame}[fragile]
%now the maps p_i
\begin{block}{Construction}
\begin{itemize}
\item<1-> Let $\rho_i:K_{i+1}\times X^i\to K_{i+1}\times X^{i-1}$ defined by $\rho_i(\tau,x_1,\dots,x_i)=(\tau,x_2,\dots,x_i)$ %omitting x_1
\item<2-> $\rho$ induces a map $\mathfrak{p}_i:\mathcal{E}_i\to \mathcal{B}_i$.
\item<3-> We see that $\beta_i$ can be defined on $S$ as $\beta_i(\tau,x_2,\dots,x_i)=\mathfrak{p}_{i-1}\alpha_i(\tau,x_1,\dots,x_i)$, so $\mathfrak{p}_i$ is well defined by induction. %by what I said of just changing the letter and omitting x_1 %this ensures that R is mapped to S 
\item[]<4->
\[
\begin{tikzcd}
(K_{i+1}\times X^i,R)\arrow[r,"\alpha_i"]\arrow[d,"\rho_i"'] &(\mathcal{E}_i,\mathcal{E}_{i-1})\arrow[d,"\mathfrak{p}_{i-1}"]\arrow[d,"\mathfrak{p}_i"']\\
(K_{i+1}\times X^{i-1},S)\arrow[r, "\beta_i"] & (\mathcal{B}_i,\mathcal{B}_{i-1})
\end{tikzcd}
\]

\end{itemize}
\end{block}
\end{frame}

\begin{frame}
\begin{theorem}
If $X$ is arc-connected, then $\mathfrak{p}_{i*}:\pi_q(\mathcal{E}_i,X)\to \pi_q(\mathcal{B}_i)$ is an isomorphism for all $q$. %this is what we needed to get an A_n-structure
\end{theorem}\pause
%the following show us something about the topology of these spaces

\begin{theorem}
If $X$ is arc-connected, then $(\mathcal{E}_n, \dots, \mathcal{E}_1)$ has the homotopy type of $(X\star\cdots\star X,\dots, X)$. %it is stated like this because not only the spaces have the homotopy type but the homotopies are relative/the relative homotopy groups are isomorphic
\end{theorem}\pause

\begin{theorem}
$(\mathcal{B}_{i+1},\mathcal{B}_i)$ has the homotopy type of $(C\mathcal{E}_i\cup_{p_i}\mathcal{B}_i,\mathcal{B}_i)$. %pushout/attaching space with the cone
\end{theorem}\pause
%The theorems are important, but I think the construction itself is more important
This concludes the proof of the \emph{if} statement of the theorem.
\end{frame}
\begin{frame}
\frametitle{Projective spaces}
\begin{defi}
The $X$-projective $i$-space $XP(i)$ ($i\leq n$) associated with an $A_n$-space is the base space $\mathcal{B}_{i+1}$ of the derived $A_n$-structure.
\end{defi}\pause
\begin{block}{Remark}
$\mathcal{B}_{i+1}$ can be defined even when $\mathfrak{p}_{i+1}$ cannot, since it has the homotopy type of $C\mathcal{E}_i\cup_{\mathfrak{p}_i}\mathcal{B}_i$.
\end{block}\pause
\begin{block}{Remark-Proposition}%it is not obvious but it is not proved in the paper %it is a consequence of this construction being Milnor construction in particular
For $d=1,2,4$, the space $S^{d-1}P(i)$ is respectively the real, complex and quaternionic projective space of $i$ dimensions and the fibration can be identified with the previous construction using the associative multiplication of $S^{d-1}$.
\end{block}
\end{frame}

\subsection{Proof of the main theorem (ii)}
\begin{frame}
\frametitle{Proof of the theorem (only if)}
\begin{itemize}
\item<1-> It is sufficient to construct an $A_n$-form $\{M_i\}$ on some space $F$ of the same homotpy type of $X$ for we can then define an $A_n$-form $\{N_i\}$ by 
\[K_i\times X^i\xrightarrow{1\times j^i}K_i\times F^i\xrightarrow{M_i}F\xrightarrow{s}X\]
where $j:X\to F$ and $s:F\to X$ are homotopy inverses.
\item<2-> Denote by $E_i\xrightarrow{p_i} B_i$ the maps of the $A_n$-structure from the hypothesis. By the above line and a previous lemma, we may assume that $X\to E_i\xrightarrow{p_i} B_i$ is a fibration.
\end{itemize}

\end{frame}

\begin{frame}[fragile]
\frametitle{Proof of the theorem (only if)}
\begin{itemize}
\item<1-> Assume by induction that $M_j$ is defined, $\mathfrak{p}_j:\mathcal{E}_j\to\mathcal{B}_j$ are is constructed for $j<i$ and we have commutative diagrams
\[
\begin{tikzcd}
\mathcal{E}_j\arrow[r,"d_j"]\arrow[d,"\mathfrak{p}_j"]& E_j\arrow[d,"p_j"]\\
\mathcal{B}_j\arrow[r,"b_j"]&B_j
\end{tikzcd}
\]
such that $d_j|\mathcal{E}_{j-1}=d_{j-1}$ and $b_j|\mathcal{B}_{j-1}=b_{j-1}$. %they are contained in each other, see definition of A_n-structure
\item<2-> Base case $j=1$:
\[
\begin{tikzcd}
 X\arrow[r,equals]\arrow[d]& X\arrow[d]\\
* \arrow[r,equals]        & *
\end{tikzcd}
\]
\end{itemize}
\end{frame}
\begin{frame}
\frametitle{Proof of the theorem (only if)}
\begin{itemize}


 \item<1-> Recall $R=L_{i+1}\times X^i\cup K_{i+1}\times X\times X^{[i-1]}$ and let $J=int(\delta_1(2,i)(K_2\times K_i))$. 
 \item<2-> Note that $\alpha_i$ is defined without using $M_i$ on $R$ minus $J\times(X^i-X^{[i]})$.
 \item<3-> Extend from this subset to $\gamma:(L_{i+1}-J)\times X^i\cup K_{i+1}\times X^{[i]}\to C\mathcal{E}_{i-1}$.
 \item<4-> Let $k:CE_{i-1}\to E_i$ the contracting homotopy of the $A_n$-structure %pay attention to the notation

 \end{itemize}
\end{frame}
\begin{frame}
\frametitle{Proof of the theorem (only if)}
\begin{itemize}
 \item<1-> Define $f:K_{i+1}\times X^{i-1}\to B_i$ by \[f(\tau,x_2,\dots,x_i)=k\circ Cd_{i-1}\circ\gamma(\tau,e,x_2,\dots,x_i)\] so that $f\circ\rho_i=p_i\circ k\circ Cd_{i-1}\circ\gamma$ over the whole domain of $\gamma$. %rho consited of omitting the first coordinate
 \item<2-> Then $f$ induces an extension $b_i:\mathcal{B}_i\to B_i$ of $b_{j-1}$. %I have no idea why

\end{itemize}
\end{frame}
\begin{frame}[fragile]
\frametitle{Proof of the theorem (only if)}
\begin{itemize}
 \item<1-> Since $L_{i+1}-J$ is a deformation retract of $K_{i+1}$, $ k\circ Cd_{i-1}\circ\gamma$ can be extended to a map $d:K_{i+1}\times X^i\to E_i$ covering $f\circ\rho_i$. %the deformation retract is just pushing J into the interior of K_i+1 and then making it hollow
 \item<2-> This map satisfies $d(\delta_1(2,i)(K_2\times K_i)\times X^i)\subset X$ so $M_i$ is defined by \[M_i(\tau,x_1,\dots,x_i)=d(\delta_1(2,i)(*,\tau),x_1,\dots, x_i).\]
 \item<3-> Finally, $\mathcal{E}_i$ can now be constructed and $d$ induces maps $d_i:\mathcal{E}_i\to E_i$ such that the following diagram commutes.
 \[
\begin{tikzcd}
\mathcal{E}_j\arrow[r,"d_j"]\arrow[d,"\mathfrak{p}_j"]& E_j\arrow[d,"p_j"]\\
\mathcal{B}_j\arrow[r,"b_j"]&B_j
\end{tikzcd}
\]
\end{itemize}
\end{frame}

\begin{frame}
Some corollaries that were previously known theorems.\pause
\begin{corollary}
A space $X$ admits an associative multiplication if and only if there is a map $p:(X\star X, X)\to (SX,*)$ such that $p_*:\pi_q(X\star X, X)\to \pi_q(SX)$ is an isomorphism for all $q$. %this is precisely the first case of the construction that was exemplified
\end{corollary}\pause
\begin{corollary}
A space $X$ admits a homotopy associative multiplication if and only if there is a map $\bar{p}:(X\star X\star X,X\star X,X)\to (C(X\star X)\cup_p SX, SX,*)$ where $p=\bar{p}|X\star X$ and both $p_*:\pi_q(X\star X, X)\to \pi_q(SX)$ and $\bar{p}_*:\pi_q(X\star X\star X, X)\to \pi_q(C(X\star X)\cup_p SX)$ are isomorphisms for all $q$.
\end{corollary}
\end{frame}

\subsection{Existence of $A_p$-spaces}
\begin{frame}[fragile]
%now we see that there are examples (apart from the loop spaces)
\begin{theorem}
For each prime $p$, there exist spaces which admit $A_{p-1}$-structures but not $A_p$-structures.
\end{theorem}\pause
\begin{block}{Example}
For $p=3$, $S^7$ admits an $A_2$-structures (H-space) but not an $A_3$-structure.
\end{block}
\end{frame}
\begin{frame}
\frametitle{Proof (existence of $A_{p-1}$-structures)}
\begin{itemize}
\item<1-> Let $\Z_{(p)}$ be the group of fractions which, in lowest terms, have denominator prime to $p$. %Q^p in the original, but it is a localization
\item<2-> Let $X$ be the Moore space with one nonvanishing homology group $\Z_{(p)}$ in dimension $2n-1$. %For some n I guess
%\item<3-> $Q^p$ has no $p$-torsion and is divisible by any other prime. %divide the fraction by q, then the denominator is q times the previous denominator, which is also in the group
\item<3-> $X$ only has nontrivial cohomology groups in dimension $2n-1$.
%\item<3-> $H_*(X;\Z/(q))=0$ for $q\neq p$ and $H_i(X;\Z/(p))=0$ except for $i=0$ or $2n-1$, being $H_{2n-1}(X;\Z/(p))=\Z/(p)$.
%\item<4-> Deduce that $\pi_i(X)$ is isomorphic to the $p$-primary component of $\pi_i(S^{2n-1})$. %a p-primary group is such that the order of every element is a power of p (not necessarily the same power for all)
\end{itemize}
\end{frame}
\begin{frame}
\frametitle{Proof (existence of $A_{p-1}$-structures)}
\begin{itemize}
\item<1-> Embed $X$ in $Z=\Omega^2S^2X$. Since $Z$ is a loop space, it admits $A_n$-structures $\{N_i\}$ for all $n$.
\item<2-> Let $M'_i:K_i\times X^i\to Z$ the restrictions and let us try to deform it to an $A_{p-1}$-structure $\{M_i\}$ on $X$.
\item<3-> Assume by induction that $M_j$ lies on $X$ for $j<i$. Let $T=L_i\times X^i\cup K_i\times X^{[i]}$. $M'_i$ is defined on $T$ in terms of $M_j$, so $M_i(T)\subseteq X$. %because of the induction M_{i-1} is defined on L_i and there are compatibility conditions to be an A_n-form
\item<4-> The obstruction to deforming $M_i$ rel $T$ appears as classes in $H^q(K_i\times X^i,T;\pi_q(Z,X))\cong H^{q+2-i}(X^i;\pi_q(Z,X))$.
\item<5-> $\pi_q(Z,X)=0$ for $q<2pn-2$.

\end{itemize}
\end{frame}
\begin{frame}
\frametitle{Proof (existence of $A_{p-1}$-structures)}
\begin{itemize}
\item<1-> Since $X$ only has nontrivial cohomology groups in dimension $2n-1$, $H^q(X^i;G)=0$ for any $G$ if $q\neq i(2n-1)$. %using kunneth formula
\item<2-> The obstructions lie in trivial groups unless $i(2n-1)+2-i\geq 2pn-2\Leftrightarrow i\geq p$.
\item<3-> Therefore there is no obstruction to obtaining $A_{p-1}$-forms on $X$.
\end{itemize}
\end{frame}
\begin{frame}
\frametitle{Proof (non-existence of $A_{p}$-structures)}
\begin{itemize}
\item<1-> Suppose that $X$ admits an $A_p$-structure and let $XP(i)=C\mathcal{E}_i\cup_{\mathfrak{p}_i}\mathcal{B}_i$.
\item<2-> $H^*(XP(i);\Z/(p))$ would be a polynomial algebra with a generator $u\in H^{2n}(XP(i);\Z/(p))$ with $u^i\neq 0$.
\item<3-> Choosing $n$ prime to $p(p-1)$, the Adem relations on the Steenrod operations $\mod p$ would imply that $u^p$ must vanish. %i\geq p
\item<4-> The contradiction implies that $X$ cannot admit an $A_{p}$-structures.
\end{itemize}
\end{frame}


\end{document}
