	\documentclass[twoside]{article}
\usepackage{estilo-ejercicios}
\setcounter{section}{0}
%\newtheorem{defin}{Definition}[section]
%\newtheorem{lem}[defin]{Lemma}
%\newtheorem{propo}[defin]{Proposition}
%\newtheorem{thm}[defin]{Theorem}
%\newtheorem{eje}[defin]{Example}
\renewcommand{\baselinestretch}{1,3}

%\usepackage{calligra}
%\usepackage[T1]{fontenc}
\usepackage{empheq}
\newcommand*\widefbox[1]{\fbox{\hspace{2em}#1\hspace{2em}}}
%--------------------------------------------------------
\begin{document}

\title{Derived $A_\infty$-structures on operads}
\author{Javier Aguilar Martín}
\maketitle


\section{Introduction}
We use an operadic totalization inspired by WHITEHOUSE to obtain an operation similar to the star operation in RLW and generalize the construction based on operadic suspension that has been done for $A_\infty$-algebras to the more general derived $A_\infty$-algebra.

We start by collecting some preliminary definitions and results from WHITEHOUSE that we adapt to our conventions. We then define the totalization functor for operads and then the bigraded version of operadic suspension. We combine these two constructions to define an operation that allows us to understand a derived $A_\infty$-multiplication as a Maurer-Cartan element. From this, we obtain a brace structure from which we can obtain a classical $A_\infty$-algebra. We use THEOREM FROM WHITEHOUSE to show that this structure is equivalent to a derived $A_\infty$-algebra on the suspended bigraded operad.

\section{Brackground and conventions}
Fix a commutative ring with unit $R$ of characteristic distinct of $2$. All tensor products taken over $R$. COPY SECTION 2.2 OF DAINFTY AND THEIR HOMOTOPIES AS BACKGROUND, REFERENCES TO WHITEHOUSE. INCLUDING DEFINITION OF TWISTED COMPLEX

REDEFINE THINGS ACCORDING TO CONVENTIONS

Let $\CC$ be a category and let $A$, $B$ be arbitrary
objects in $\CC$. We denote by $\Hom_\CC(A,B)$ the set of morphisms from $A$ to $B$ in $\CC$. If $(\CC,⊗, 1)$ is
symmetric monoidal closed, then we denote its internal hom-object by $[A,B] ∈ \CC$.

We collect some preliminary definitions. Most of them come from WHITEHOUSE SECTION but we adapt them here to our conventions.

\subsection{Filtered Modules and complexes}
INCLUSION IS REVERSED
\begin{defin}
A \emph{filtered $R$-module} $(A, F)$ is given by a family of $R$-modules $\{F_pA\}_{p∈\Z}$ indexed by
the integers such that $F_{p}A ⊆ F_{p-1}A$ for all $p ∈ \Z$ and $A = ∪F_pA$. A morphism of filtered modules is a morphism $f : A → B$ of $R$-modules which is compatible with filtrations: 
\[f(F_pA) ⊂ F_pB \text{ for all }p ∈ \Z.\]
\end{defin}

\begin{defin}\label{filteredcomplex}
A \emph{filtered complex} $(K, d, F)$ is a complex $(K, d) ∈ \mathrm{C}_R$ together with a filtration $F$ of each $R$-module $K^n$ such that $d(F_pK^n) ⊂ F_pK^{n+1}$ for all $p, n ∈ \Z$. Its morphisms are given by
morphisms of complexes $f : K → L$ compatible with filtrations: \[f(F_pK) ⊂ F_pL\text{ for all }p ∈ \Z.\]
\end{defin}

We denote by $\fmod$ and $\fc$ the categories of filtered modules and filtered complexes of $R$-modules, respectively.

$-1$ CHANGED TO 1 (IT MUST BE LIKE THIS BECAUSE I THE MAPS PRESERVING FILTRATION WILL BE LEVEL 0 AND I WANT THE WHOLE AINFTY OPERAD TO BE MAPPED TO THAT, SO R MUST BE IN LEVEL 0 AS WELL)
\begin{defin}\label{filteredtensor}
The tensor product of two filtered $R$-modules $(A, F)$ and $(B, F)$ is a filtered $R$-module,
with
 \[F_p(A ⊗ B) :=\sum_{i+j=p}\Ima(F_iA ⊗ F_jB → A ⊗ B).\]
This makes the category of filtered $R$-modules into a symmetric monoidal category, where the unit is given by $R$ with the trivial filtration $0 = F_{1}R ⊂ F_0R = R$.
\end{defin}

\subsection{Bigraded modules, vertical bicomplexes, twisted complexes and sign conventions}

TRY TO INCLUDE ONLY THE NECESSARY

SPECIFY THAT BIGRADED MAP RAISES BOTH DEGREES
\begin{defin}
We consider $(\Z,\Z)$-bigraded
$R$-modules $A = \{A^j_i\}$, where elements of $A^j_i$ are said to have bidegree $(i, j)$. We sometimes refer to $i$
as the \emph{horizontal} degree and $j$ the \emph{vertical degree}. The \emph{total degree} of an element $a ∈ A^j_i$ is $|a| = i+j$.
\end{defin}
\begin{defin}
A \emph{morphism of bidegree $(p, q)$} maps $A^j_i$ to $A^{j+q}_{i+p}$. The tensor product of two bigraded $R$-modules $A$
and $B$ is the bigraded $R$-module $A ⊗ B$ given by
\[(A ⊗ B)^j_i \coloneqq\bigoplus_{p,q}A^q_p ⊗ B^{j−q}_{i−p} .\]
\end{defin}

We denote by $\bgmod$ the category whose objects are bigraded $R$-modules and whose morphisms
are morphisms of bigraded $R$-modules of bidegree $(0, 0)$. It is symmetric monoidal with the above
tensor product.

We introduce the following scalar product notation for bidegrees: for $x$, $y$ of bidegree $(x_1, x_2)$, $(y_1, y_2)$
respectively, we let $\langle x, y\rangle = x_1y_1 + x-2y_2$.

The symmetry isomorphism
\[τ_{A⊗B} : A ⊗ B → B ⊗ A\]
is given by
\[a ⊗ b \mapsto (−1)^{\langle a,b\rangle}b ⊗ a.\]
We follow the Koszul sign rule: if $f : A → B$ and $g : C → D$ are bigraded morphisms, then the
morphism $f ⊗ g : A ⊗ C → B ⊗ D$ is defined by
\[(f ⊗ g)(a ⊗ c) \coloneqq (−1)^{\langle g,a\rangle}f(a) ⊗ g(c).\]

\begin{defin}
A \emph{vertical bicomplex} is a bigraded $R$-module $A$ equipped with a vertical differential $d^A : A → A$ of bidegree $(0, 1)$. A morphism of vertical bicomplexes is a morphism of bigraded modules
of bidegree $(0, 0)$ commuting with the vertical differential.
\end{defin}

We denote by $\vbc$ the category of vertical bicomplexes. The tensor product of two vertical bicomplexes $A$ and $B$ is given by endowing the tensor product of underlying bigraded modules with
vertical differential \[d^{A⊗B} := d^A ⊗ 1 + 1 ⊗ d^B : (A ⊗ B)^v_u → (A ⊗ B)^{v+1}_u .\] This makes $\vbc$ into a
symmetric monoidal category.

The symmetric monoidal categories $(\mathrm{C}_R,⊗,R)$, $(\bgmod,⊗,R)$ and $(\vbc,⊗,R)$ are related by embeddings $\mathrm{C}_R\to\vbc$ and $\bgmod \to\vbc$ which are monoidal and full.



\begin{defin}\label{delta1}
Let $A,B$ be bigraded modules. We define $[A,B]^∗_∗$
to be the bigraded module of morphisms of bigraded modules $A → B$. Furthermore, if $A,B$ are vertical bicomplexes, and $f ∈
[A,B]^v_u$, we define
\[δ(f) := d_Bf − (−1)^vfd_A.\]
\end{defin}

\begin{lem}
If $A$, $B$ are vertical bicomplexes, then $([A,B]^∗_∗
, δ)$ is a vertical bicomplex.
\end{lem}
\begin{proof}
Direct computation shows $\delta^2=0$.
\end{proof}
END OF PAGE 5, I DEFINE THE SHIFT LATER, SO MAYBE IT IS NOT NECESSARY, BUT THINK ABOUT SHIFT OF MAPS (IS IT NECESSARY  TO ADD THAT SIGN?) AND COMPARE WITH THE GRADED CASE

I CHANGED HORIZONTAL DEGREE SIGNS
\begin{defin}\label{twistedcomplex} A \emph{twisted complex} $(A, d_m)$ is a bigraded $R$-module $A = \{A^j_i \}$ together with a family
of morphisms $\{d_m : A → A\}_{m≥0}$ of bidegree $(m,1−m )$ such that for all $m ≥ 0$,
CHECK CONVENTION IN CASE IT IS EASIER TO TAKE $(-1)^j$
\[\sum_{i+j=m}(−1)^id_id_j = 0.\]

\end{defin}

\begin{defin}\label{twistedmorphisms}
A morphism of twisted complexes $f : (A, d^A_m) → (B, d^B_m)$ is given by a family of morphisms of $R$-modules $\{f_m : A → B\}_{m≥0}$ of bidegree $(m,−m)$ such that for all $m ≥ 0$,
\[\sum_{i+j=m}d^B_if_j =\sum_{i+j=m}(−1)^if_id^A_j.\]
The composition of morphisms is given by $(g \circ f)_m :=\sum_{i+j=m} g_if_j$.

A morphism $f = \{f_m\}_{m≥0}$ is
said to be \emph{strict} if $f_i = 0$ for all $i > 0$. The \emph{identity} morphism $1_A : A → A$ is the strict morphism
given by $(1_A)_0(x) = x.$ A morphism $f = \{f_i\}$ is an isomorphism if and only if $f_0$ is an isomorphism of
bigraded $R$-modules. Indeed, an inverse of $f$ is obtained from an inverse of $f_0$ by solving a triangular system.
\end{defin}
Denote by $\tc$ the category of twisted complexes.The following construction endows $\tc$ with a symmetric monoidal structure. See LEMMA 3.3 for a proof.
\begin{lem}\label{tensortwisted}
The category $(\tc,⊗,R)$ is symmetric monoidal, where the monoidal structure is given
by the bifunctor
\[⊗ : \tc × \tc → \tc\]
which on objects is given by $((A, d^A_m), (B, d^B_m)) → (A ⊗ B, d^A_m ⊗ 1 + 1 ⊗ d^B_m)$ and on morphisms is
given by $(f, g) → f ⊗ g$, where $(f ⊗ g)_m :=\sum_{i+j=m} f_i ⊗ g_j$. In particular, by the Koszul sign rule we
have that \[(f_i ⊗g_j)(a⊗b) = (−1)^{\langle g_j ,a\rangle}f_i(a)⊗g_j(b).\] The symmetry isomorphism is given by the strict
morphism of twisted complexes
\[τ_{A⊗B} : A ⊗ B → B ⊗ A\]
defined by
\[a ⊗ b\mapsto (−1)^{\langle a,b,\rangle}b ⊗ a.\]
\end{lem}

The internal hom on bigraded modules can be extended to twisted complexes via the following lemma whose proof is in LEMMA 3.4.
\begin{lem}\label{di} Let $A,B$ be twisted complexes. For $f ∈ [A,B]^v_u$, setting
\[(d_if) := (−1)^{i(u+v)}d^B_if − (−1)^vfd^A_i,\]
for $i ≥ 0$, endows $[A,B]^∗_∗$ with the structure of a twisted complex.
\end{lem}

UNDERLINED CATEGORIES?
\subsection{Totalization}
DEFINITION WITH FILTRATION, DIFFERENTIAL, MONOIDALITY


CHOOSE NOTATION, T OR TOT

\begin{defin}
The \emph{totalization} $\Tot(A)$ of a bigraded $R$-module $A = \{A^j_i \}$ the graded $R$-module is given by
\[\Tot(A)^n \coloneqq
\bigoplus_{i<0}A^{n-i}_i ⊕\prod_{i\geq 0}A^{n-i}_i .\]
The \emph{column filtration} of $\Tot(A)$ is the filtration given by \[F_p\Tot(A)^n \coloneqq\prod_{i\geq p} A^{n-i}_i .\]
\end{defin}

Given a twisted complex $(A, dm_)$, define a map $d : \Tot(A) → \Tot(A)$ of degree 1 by letting
\[d(a)_j \coloneqq \sum_{m≥0}(−1)^{mn}d_m(a_{j-m}),\]
for $a = (a_i)_{i∈\Z} ∈ \Tot(A)^n$,
where $a_i ∈ A^{n-i}_i$ denotes the $i$-th component of $a$, and $d(a)_j$ denotes the $j$-th component of $d(a)$. Note
that, for a given $j ∈ \Z$ there is a sufficiently large $m ≥ 0$ such that $a_{j-m′} = 0$ for all $m′ ≥ m$. Hence
$d(a)_j$ is given by a finite sum. Also, for negative $j$ sufficiently large, one has $a_{j-m} = 0$ for all $m ≥ 0$, which
implies $d(a)_j = 0$.

Given a morphism $f : (A, d_m) → (B, d_m)$ of twisted complexes, let $\Tot(f) : \Tot(A) → \Tot(B)$ be
the map of degree 0 defined by
\[(\Tot(f)(a))_j \coloneqq \sum_{m≥0}(−1)^{mn}f_m(a_{j-m}),\]
 for $a = (a_i)_{i∈\Z} ∈ \Tot(A)^n$.
 
\begin{thm}
The assignments $(A, d_m) \mapsto (\Tot(A), d, F)$, where $F$ is the column filtration of $\Tot(A)$,
and $f \mapsto \Tot(f)$ define a functor $\Tot : \tc \to \fc$ which is an isomorphism of categories when restricted to its image.
\end{thm}
\begin{proof}
See THEOREM 3.8.
\end{proof}
MAYBE IT IS USEFUL TO INCLUDE THE DEFINITIONN OF THE INVERSE WHICH COMES  RIGHT AFTER THE THEOREM (CHECKING CONVENTIONS, CONSIDER DEFINING SPLIT OR JUST SAYING THE COMPLEX IS OF THE FORM TOT(A))

We will consider the following bounded categories since the totalizaton functor has better properties when restricted to them. 
I THINK I WON'T NEED TO SPECIFY THE SPLIT CATEGORIES
\begin{defin}
We let $\tc^b$, $\vbc^b$, $\bgmod^b$ be the full subcategories of \emph{horizontally bounded on the right} graded twisted
complexes, vertical bicomplexes and bigraded modules respectively. This means that if $A=\{A^j_i\}$ is an object on any of this categories, then there exists $i$ such that $A^j_{i'}=0$ for $i'>i$.

We let $\fmod^b$, $\fc^b$ be the full subcategories of bounded filtered modules, respectively complexes, i.e.
the full subcategories of objects $(K, F)$ such that there exists some $p$ with the property that $F_{p'}K^n = 0$ for all $p> p'$. We refer to all of these as the bounded subcategories of $\tc$, $\vbc$, $\bgmod$, $\fmod$ and $\fc$   respectively.
\end{defin}

\begin{propo}
The functors $\Tot : \bgmod → \fmod$ and $\Tot : \tc → \fc$ are lax symmetric
monoidal, with structure maps
\[\epsilon : R → \Tot(R)\text{ and }\mu=μ_{A,B} : \Tot(A) ⊗ \Tot(B) → \Tot(A ⊗ B),\]
given by $\epsilon = 1_R$ and for $a = (a_i)_i ∈ \Tot(A)^{n_1}$ and  $b=(b_j)_j ∈ \Tot(B)^{n_2}$,
\[μ(a ⊗ b)_k \coloneqq
\sum_{k_1+k_2=k}(−1)^{k_1n_2}a_{k_1} ⊗ b_{k_2} .\]
When restricted to the bounded case, $\Tot : \bgmod^b
 → \fmod^b$ and $\Tot : \tc^b → \fc^b$ are
strong symmetric monoidal functors.
\end{propo}
\begin{proof}
See PROPOSITION 3.11.
\end{proof}

In the bounded case, the inverse
\[\mu^{-1}:\Tot(A\otimes B)\to \Tot(A)\otimes \Tot(B)\]
is computed explicitly as follows.
Let  $c\in\Tot(A\otimes B)^n$. By definition, we have
\[\Tot(A\otimes B)^n=\bigoplus_k (A\otimes B)^{n-k}_k=\bigoplus_k\bigoplus_{n_1+n_2=n}\bigoplus_{k_1+k_2=k}A_{k_1}^{n_1-k_1}\otimes B_{k_2}^{n_2-k_2}\]
And thus $c=(c_k)_k$ may be written as a finite sum $c=\sum_k c_k$, where 
\[c_k=\sum_{n_1+n_2=n}\sum_{k_1+k_2=k}a_{k_1}^{n_1-k_1}\otimes b_{k_2}^{n_2-k_2}.\]
Here we introduced superscripts to indicate the vertical degree, which unlike in the definition of $\mu$, it is not solely determined by the horizontal degree, since the total degree also varies. Distributibity allows us to rewrite $c$ as
\[c=\sum_k \sum_{n_1+n_2=n}\sum_{k_1+k_2=k}a_{k_1}\otimes b_{k_2}=\sum_{n_1+n_2=n}\sum_{k_1}\sum_{k_2}(a_{k_1}\otimes b_{k_2})=\sum_{n_1+n_2=n}\left(\sum_{k_1}a_{k_1}\right)\otimes\left(\sum_{k_2}b_{k_2}\right).\]
Therefore, $\mu^{-1}$ can be defined as
\[\mu^{-1}(c)=\sum_{n_1+n_2=n}\left(\sum_{k_1}(-1)^{k_1n_2}a_{k_1}\right)\otimes\left(\sum_{k_2}b_{k_2}\right).\]

Inductively, one can deduce that the natural transformation 
\[\mu^{-1}:\Tot(A_{(1)}\otimes\cdots\otimes A_{(m)})\to \Tot(A_{(1)})\otimes\cdots\otimes \Tot(A_{(m)})\]
is given on pure tensors (for notational convenience) as
\[\mu^{-1}(a_{(1)}\otimes\cdots\otimes a_{(m)})=(-1)^{\sum_{j=2}^m n_j\sum_{i=1}^{j-1}k_i}a_{(1)}\otimes\cdots\otimes a_{(m)},\]
where $a_{(l)}\in (A_{(m)})_{k_l}^{n_l-k_l}$.
\subsection{Enriched categories and enriched totalization}
NOT SURE IF I'M GOING TO NEED SO MUCH GENERAL BACKGROUND (CERTAINLY DO NOT INCLUDE PROOFS) I WOULD LIKE TO INCLUDE THE PROOF OF THE INVERSE INDUCING THE INVERSE TRANSFORMATION

MAYBE INCLUDE HERE THE ENRICHED CATEGORIES THAT I AM GOING TO NEED AND ALSO ENRICHED TOTALIZATION (I ONLY NEED THE INVERSE)

SAY SOMETHING ABOUT WHERE THE DEFINITIONS COME FROM (WHITEHOUSE), AND ADJUST TO CONVENTIONS, CHECK IF I NEED SOME RESULTS

START WITH SOME SENTENCES ABOUT DEFINING A BUNCH OF USEFUL ENRICHED CATEGORIES

\begin{defin}\label{weirdenrichment}
Let $A,B,C$ be bigraded modules. We denote by $\underline{\mathpzc{bgMod}_R}(A,B)$ the bigraded module given by
\[\underline{\mathpzc{bgMod}_R}(A,B)^v_u :=\prod_{j≥0}[A,B]^{v−j}_{u+j}\]
where $[A,B]$ is the inner hom-object of bigraded modules. More precisely, $g ∈ \underline{\mathpzc{bgMod}_R}(A,B)^v_u$ is given
by $g := (g_0, g_1, g_2, \dots )$, where $g_j : A → B$ is a map of bigraded modules of bidegree $(u + j, v − j)$.

Moreover, we define a composition morphism
\[c : \underline{\mathpzc{bgMod}_R}(B,C) ⊗ \underline{\mathpzc{bgMod}_R}(A,B) → \underline{\mathpzc{bgMod}_R}(A,C)\]
by
\[c(f, g)_m :=\sum_{i+j=m}(−1)^{i|g|}f_ig_j .\]
\end{defin}

\begin{defin}\label{delta2}
Let $(A, d^A_i), (B, d^B_i)$ be twisted complexes, $f ∈ \underline{\mathpzc{bgMod}_R}(A,B)^v_u$ and consider $d^A :=(d^A_i)_i ∈ \underline{\mathpzc{bgMod}_R}(A,A)^1_0$
and $d^B := (d^B_i)_i ∈ \underline{\mathpzc{bgMod}_R}(B,B)^1_0$. We define
\[δ(f) := c(d^B, f) − (−1)^{\langle f,d^A\rangle}c(f, d^A) ∈ \underline{\mathpzc{bgMod}_R}(A,B)^{v+1}_u\]
where $\langle f, d^A\rangle$ is the scalar product for the bidegrees and $c$ is the composition morphism described in \Cref{weirdenrichment} More precisely,
\[(δ(f))_m :=\sum_{i+j=m}(−1)^{i|f|}d^B_if_j − (−1)^{v+i}f_id^A_j.\]
\end{defin}

The following lemma justifies the above definition. For a proof see LEMMA 4.18

\begin{lem}
The following equations hold
\begin{align*}
c(d^A, d^A) = 0,
δ^2 = 0,
δ(c(f, g)) = c(δ(f), g) + (−1)^v c(f, δ(g)),
\end{align*}
where the bidegree of $f$ is $(u, v)$. Furthermore, $f ∈ \ubgMod(A,B)$ is a map of twisted complexes if and
only if $δ(f) = 0$. In particular, $f$ is a morphism in $\tc$ if and only if the bidegree of $f$ is $(0, 0)$ and
$δ(f) = 0$. Moreover, for $f$, $g$ morphisms in $\tc$, we have that $c(f, g) = f\circ g$, where the latter denotes
composition in $\tc$.
\end{lem}

\begin{defin}
For $A,B$ twisted complexes, we define $\underline{t\mathcal{C}_R}(A,B)$ to be the vertical bicomplex
$\underline{t\mathcal{C}_R}(A,B) := (\underline{\mathpzc{bgMod}_R}(A,B), δ)$.
\end{defin}

\begin{defin}\label{ubgMod}
We denote by $\ubgMod$ the $\bgmod$-enriched category of bigraded modules given
by the following data.

\begin{enumerate}[(1)]
\item The objects of $\ubgMod$ are bigraded modules.
\item For $A,B$ bigraded modules the hom-object is the bigraded module $\ubgMod(A,B)$.
\item The composition morphism $c : \ubgMod(B,C) ⊗ \ubgMod(A,B) → \ubgMod(A,C)$ is given by \Cref{weirdenrichment}.
\item The unit morphism $R → \ubgMod(A,A)$ is given by the morphism of bigraded modules that
sends $1 ∈ R$ to $1_A : A → A$, the strict morphism given by the identity of $A$.
\end{enumerate}
\end{defin}

\begin{defin}\label{utC}
The $\vbc$-enriched category of twisted complexes $\utC$ is the enriched category given by the following data.
\begin{enumerate}[(1)]
\item The objects of $\utC$ are twisted complexes.
\item For $A,B$ twisted complexes the hom-object is the vertical bicomplex $\utC(A,B)$.
\item The composition morphism $c : \utC(B,C)⊗\utC(A,B) → \utC(A,C)$ is given by \Cref{weirdenrichment}.
\item The unit morphism $R → \utC(A,A)$ is given by the morphism of vertical bicomplexes sending
$1 ∈ R$ to $1_A : A → A$, the strict morphism of twisted complexes given by the identity of $A$.
\end{enumerate}
\end{defin}




The next tensor corresponds to $\underline{\otimes}$ in the categorical setting. SO MAYBE CHANGE NOTATION TTO MAKE IT CONSISTENT?


\begin{lem}\label{tensorenriched}
The monoidal structure of $\utC$ is given by the following map of vertical bicomplexes.
\[\widehat{⊗}: \utC(A,B) ⊗ \utC(A′,B′) → \utC(A ⊗ A′,B ⊗ B′)\]
\[(f, g) → (f\widehat{⊗}g)_m :=\sum_{i+j=m}(−1)^{ij}f_i ⊗ g_j\]
The monoidal structure of $\ubgMod$ is given by the restriction of this map.
\end{lem}
\begin{proof}
See LEMMA 4.27
\end{proof}



\begin{defin}\label{ufMod}
The $\bgmod$-enriched category of filtered modules $\ufMod$ is the enriched category given by the following data.
I HAVE TO KEEP J+U SO THAT THE DEGREES MATCH IN ENRICH TOT (ALTERNATIVELY KEEP V-U BUT I DON'T LIKE THAT)
\begin{enumerate}[(1)]
\item The objects of $\ufMod$ are filtered modules.
\item For filtered modules $(K, F)$ and $(L, F)$, the bigraded module $\ufMod(K,L)$ is given by
\[\ufMod(K,L)^v_u :=\{f : K → L\mid f(F_qK^m) ⊂ F_{q+u}L^{m+u+v}, ∀m, q ∈ \Z\}.\]
\item The composition morphism is given by $c(f, g) = (−1)^{u|g|}fg$, where $f$ has bidegree $(u, v)$.
\item The unit morphism is given by the map $R → \ufMod(K,K)$ given by $1 → 1_K$.
\end{enumerate}
\end{defin}


\begin{defin}\label{fmoddifferential}
Let $(K, d^K, F)$ and $(L, d^L, F)$ be filtered complexes. We define $\ufC(K,L)$ to be the
vertical bicomplex whose underlying bigraded module is $\ufMod(K,L)$ with vertical differential
\[δ(f) := c(d^L, f) − (−1)^{\langle f,d^K\rangle}c(f, d^K) = d^Lf − (−1)^{v+u}fd^K = d^Lf − (−1)^{|f|}fd^K\]
for $f ∈ \ufMod(K,L)^v_u$, where $c$ is the composition map from \Cref{ufMod}.
\end{defin}


\begin{defin}\label{ufC}
The $\vbc$-enriched category of filtered complexes $\ufC$ is the enriched category given
by the following data.
\begin{enumerate}[(1)]
\item The objects of $\ufC$ are filtered complexes.
\item For $K,L$ filtered complexes the hom-object is the vertical bicomplex $\ufC(K,L)$.
\item The composition morphism is given as in $\ufMod$ in \Cref{ufMod}. 
\item The unit morphism is given by the map $R → \ufC(K,K)$ given by $1 → 1_K$.
We denote by $\usfC$ the full subcategory of $\ufC$ whose objects are split filtered complexes.

\end{enumerate}
\end{defin}

The enriched monoidal structure is given by the following lemma.
\begin{defin}\label{tensorenriched2}
The monoidal structure of fMod R is given by the following map of vertical bicomplexes.
\[\widehat{⊗}: \ufC(K,L) ⊗ \ufC(K′,L′) → \ufC(K ⊗ K′,L ⊗ L′),\]
\[(f, g) → f\widehat{⊗}g := (−1)^{u|g|}f ⊗ g\]
where $f$ has bidegree $(u, v)$.
\end{defin}
\begin{proof}
See LEMMA 4.36
\end{proof}
LEMMA 4.35 I WILL NEED THE ADJUNCTION (AND MAYBE THE OTHER  ADJUNCTION AS WELL)

We now define an enriched version of the totalization functor. CHECK TOT, OR JUST OMIT IT BECAUSE I DON'T NEED IT
\begin{defin}\label{enrichedtot}
Let $A,B$ be bigraded modules and $f ∈ \ubgMod (A,B)^v_u$ we define

\[\Tot(f) ∈ \ufMod(\Tot(A),\Tot(B))^v_u\]
to be given on any $a ∈ \Tot(A)^n$ by
\[(\Tot(f)(a)))_{j+u} :=
\sum_{m≥0}(−1)^{(m+u)n}f_m(a_{j-m}) ∈ B^{n-j+v}_{j+u} ⊂ \Tot(B)^{n+u+v}.\]
Let $K = \Tot(A)$, $L = \Tot(B)$ and $g ∈ \ufMod(K,L)^v_u$ we define
\[f := \Tot^{−1}(g) ∈ \ubgMod(A,B)^v_u\]
to be $f := (f_0, f_1,\dots)$ where $f_i$ is given on each $A^{m+j}_j$ by the composite
\begin{align*}
f_i : A^{m-j}_j \hookrightarrow\prod_{k\geq j}A^{m-k}_k = F_j(\Tot(A)^m)\xrightarrow{g}&F_{j+u}(\Tot(B)^{m+u+v})\\
&=\prod_{l\geq j+u}B^{m+u+v-l}_l\xrightarrow{×(−1)^{(i+u)m}} B^{m-j+v−i}_{j+u+i} ,
\end{align*}
where the last map is a projection and multiplication with the indicated sign.

I HAVE TO KEEP J+U SO THAT THE DEGREES MATCH  (ALTERNATIVELY KEEP V-U BUT I DON'T LIKE THAT)
\end{defin} 

\begin{thm}
Let $A$, $B$ be twisted complexes. The assignments $\mathfrak{Tot}(A) := \Tot(A)$ and
\begin{align*}
\mathfrak{Tot}_{A,B} : \utC(A,B)& → \ufC(Tot(A),Tot(B))\\
f &→ \Tot(f)
\end{align*}
define a $\vbc$-enriched functor $\mathfrak{Tot} : \utC → \ufC$ which restricts to an isomorphism onto its image. Furthermore, this functor restricts to a $\bgmod$-enriched functor \[\mathfrak{Tot} : \ubgMod → \ufMod\]
 which also restricts to an isomorphism onto its image.
\end{thm}
\begin{proof}
THEOREM 4.39
\end{proof}

\begin{propo}
The enriched functors
\[\mathfrak{Tot} : \ubgMod  → \ufMod ,\hspace{1cm} \mathfrak{Tot} : \utC → \ufC\]
are lax symmetric monoidal in the enriched sense and when restricted to the bounded case they are strong symmetric monoidal in the enriched sense.
\end{propo}
\begin{proof}
PROP 4.40
\end{proof}

We now define an enriched endomorphism operad. I THINK I ONLY NEED THIS DEFINITION AND NOT THE PREVIOUS OR THE NEXT LEMMA
\begin{defin}
Let $\underline{\mathscr{C}}$ be a monoidal $\mathscr{V}$-enriched category and $A$ an object of $\uC$. We define $\uEnd_A$
to be the collection in $\mathscr{V}$ given by
\[\uEnd_A(n) \coloneqq \uC (A^{⊗n},A) \text{ for }n ≥ 1.\]
\end{defin}
AT SOME POINT DO SOME MENTION TO THE DIFFERENT CASES THAT I AM GOING TO USE BEECAUSE THE NOTATION IS GOING TO BE THE SSAME

PROP 4.46 I WANT TO GIVE A MORE DETAILED PROOF, BUT MAYBE THE MISSING DETAILS SHOULD BE GIVEN  IN THE GENERAL SETTING

LEMMMA 4.54 MAYBE WITH THE INVERSE

ONLY MENTION 4.47 WHEN YOU REACH THE PROOF OF THE LONG ISO

REMARK 4.48 TO MOTIVATE THE ALTERNATIVE VERSION 

PROP 4.55 REPHRASED
\section{Operadic totalization}


%I THINK I NEED  (N,Z)-BRIGRADED MODULES TO MAKE SURE THAT HORIZONTAL DEGREE IS NON-NEGATIVE WHEN DEFINING M AS AN ELEMENT OF TSO

%We are going to apply the totalization with compact support functor of WHITEHOUSE 4.13 to operads and we are going to simply call it \emph{totalization (functor)} and denote it as $T$. WHITEHOUSE 3.11 the functor $T$ with domain the category of bigraded modules and bidegree $(0,0)$ morphisms is lax monoidal, so applying LAX MONOIDAL PRESERVES OPERADS we can conclude that it takes operads of bigraded modules to operads of graded modules.  

We are going to apply the totalization  functor of CITE PREVIOUS SECTION to operads and we are going to simply call it \emph{totalization (functor)} and denote it as $T$. Tthe functor $T$ with domain the category $\bgmod$ (TWISTED COMPLEXES?) and bidegree $(0,0)$ morphisms (TWISTED COMPLEX MAPS?) is lax monoidal REFERENCE TO THEOREM, so applying LAX MONOIDAL PRESERVES OPERADS we can conclude that it takes operads of bigraded modules to operads of graded modules. 

Therefore, let $\OO$ be a bigraded operad, i.e. an operad in te category  of bigraded $R$-modules. We define $T\OO$ as the operad of graded $R$-modules for which \[T\OO(n)^d=\bigoplus_{i<0}\OO(n)^{d-i}_i\oplus\prod_{i\geq 0} \OO(n)^{d-i}_i\] is the image of $\OO(n)$ under the totalization functor and the insertion maps are given by the composition  %THE SECOND IF I WANT TO ORDER THEM BY HORIZONTAL DEGREE AND WRITE SUMS LIKE WHITEHOUSE and comes equipped with insertion maps \[a\bar{\circ}_rb=(-1)^{l(i+k)} a\circ_r b\]
\[T\OO(n)\otimes T\OO(m)\xrightarrow{\mu} T(\OO(n)\otimes \OO(m)) \xrightarrow{T(\circ_r)} T\OO(n+m-1),\]
that is explicitly 
\[(a\bar{\circ}_rb)_k=\sum_{k_1+k_2=k} (-1)^{k_1n_2} a_{k_1}\circ_r b_{k_2}\]

for $a=(a_i)_i\in T\OO(n)^{n_1}$ and $b=(b_j)_j\in T\OO(m)^{n_2}$.

IT WILL BE MORE CLEAR IF I WRITE DOWN THE DEFINITION OF THE FUNCTOR AND THE PROPOSITTION WHERE $\mu$ IS DEFINED

%MAYBE I SHOULD WRITE THINGS LIKE SUMS BUT I THINK IT MAKES SENSE TO WRITE IT LIKE THIS BECAUSE I KNOW THIS COMES FROM A BIGRADED MODULE
%where $a\in\OO(n)^k_i$, $b\in\OO(m)^j_l$ and $\circ_r$ is the insertion map in $\OO$.

%It can be checked that this is indeed an operad of graded vector spaces I 

SHOULD WRITE THE AXIOMS FOR GRADED OPERADS SOMEWHERE (MORE GENERALLY OPERAD IN SYMMETRIC MONOIDAL CATEGORY LIKE WARD), AND I MIGHT ALSO WRITE THE PROOF OF THIS FACT

HEURISTICS FOR THIS SIGN AND ALTERNATIVE, REFERRING TO WHITEHOUSE (THIS  LAST PART POSSIBLE LATER WITH DERIVED A INFTY)

%MY TOTALIZATION IS WHAT SARAH CALLS TOTALIZATION WITH COMPACT SUPPORT, WHICH IS STRICT MONOIDAL (CHECK THIS BECAUSE LEMMA 4.5 IS AN EXTERNAL PRODUCT)

COMPOSITION OF ARBITRARY BIGRADING IS PRESERVED BY TOT SINCE ALL SIGNS INVOLVED ARE HORIZONTAL DEGREE SO IT IS ANALOGUE TO WHITEHOUSE (WRITE DOWN THE CALCULATIONS IF NEEDED)

\section{Vertical suspension and totalization}
On an bigraded operad we can use operadic suspension on the vertical degree with analogue results to those of the graded case MAYBE SPECIFY SOME OF THEM

%Everything should be valid for R-modules (char not 2, as in fields). The sign representation would have to be a free R-module of rank 1

 %for a commutative (at least with 1\neq 0) ring the rank is well defined, in general it is not

Let $sig_n$ be the sign representation of the symmetric group on $n$ symbols concentrated in bidegree $(0,0)$. This is a free $R$-module of rank one that comes with a natural action of the symmetric group $S_n$ that multiplies each element by the sign of each given permutation. I MIGHT LEAVE OUT THE SYMMETRIC GROUP  ACTION UNLESS I FIND OUT HOW TO MODIFY IT IN TOTALIZATION, I SHOULD THINK ABOUT IT

We define $\Lambda(n)=S^{n-1}sign_n$, where  $S$ is a vertical shift of degree so that $\Lambda(n)$ is concentrated on bidegree  $(0,n-1)$.
The operad structure on the bigraded $\Lambda=\{\Lambda(n)\}_{n\geq 0}$ is the same as in the graded case, namely

\[
\begin{tikzcd}
\Lambda(n)\otimes\Lambda(m) \arrow[r, "\circ_{r+1}"] & \Lambda(n+m-1)\\
(e_1\land\cdots\land e_n)\otimes(e_1\land\cdots\land e_m)\arrow[r, mapsto] & (-1)^{(n-r-1)(m-1)}e_1\land\cdots\land e_{n+m-1}.
\end{tikzcd}
\]



In a similar way we can define $\Lambda^-(n)=S^{1-n}sig_n$, with the same insertion maps.
%The sign might arise naturally from the permutation action. If I have the wedge of n wedge the wedge of m-1 (because the final result must be n+m-1 in total), I would permute the last m-1 until the reach the i-th position via transpositions, each transpotision produces a minus sign. Or simply considering the lat m as a single element of degree m-1 being permuted in the wedge
\begin{definition}
Let $\mathcal{O}$ be a bigraded linear operad, i.e. an operad on the category of bigraded $R$-modules. The \emph{vertical operadic suspension} $\mathfrak{s}\OO$ of $\mathcal{O}$ is given arity-wise by the Hadamard product of the operads $\OO$ and $\Lambda$, in other words, $\mathfrak{s}\OO(n)=(\mathcal{O}\otimes\Lambda)(n)=\mathcal{O}(n)\otimes\Lambda(n)$ with diagonal composition and symmetric group action. Similarly, we define the \emph{vertical operadic desuspension} $\mathfrak{s}^{-1}\OO(n)=\mathcal{O}(n)\otimes\Lambda^-(n)$. %POSSIBLY EXCLUDE SYMMETRIC GROUP ACTION, ALTHOUGH IT MAKES SENSE ON ITS OWN
\end{definition}

CONSIDER CHANGING THE NOTATION FOR BIDEGREE TO BE CONSISTENT

We may identify the elements of $\mathcal{O}$ with the elements the elements of $\mathfrak{s}\OO$. For $a\in\OO(n)$ of bidegree $(k,i)$, its ``natural'' bidegree in $\s\OO$ is $(k,i+n-1)$. To distinguish both degrees we call $(k,i)$ the \emph{internal bidegree} of $a$, since this is the degree that $a$ inherits from the grading of $\OO$. If we write $\circ_{r+1}$ for the operadic insertion on $\OO$ and $\tilde{\circ}_{r+1}$ for the operadic insertion on $\mathfrak{s}\OO$, we may find a relation between the two insertion maps in the following way. Let $a\in\OO(n)^i_k$ and $b\in\OO(m)^j_l$, and let us compute $a\tilde{\circ}_{r+1} b$.

\begin{align*}
\mathfrak{s}\OO(n)\otimes\mathfrak{s}\OO(m)&=(\OO(n)\otimes\Lambda(n))\otimes (\OO(m)\otimes\Lambda(m))\cong (\OO(n)\otimes \OO(m))\otimes (\Lambda(n)\otimes \Lambda(m))\\
&\xrightarrow{\circ_{r+1}\otimes\circ_{r+1}} \OO(m+n-1)\otimes \Lambda(n+m-1)=\mathfrak{s}\OO(n+m-1).
\end{align*}

The symmetric monoidal structure produces the sign $(-1)^{(n-1)j}$ in the isomorphism $\Lambda(n)\otimes \OO(m)\cong\OO(m)\otimes\Lambda(n)$, and the operadic structure of $\Lambda$ produces the sign $(-1)^{(n-1)(m-1)+r(m-1)}$, so 

\begin{equation}\label{sign}
a\tilde{\circ}_{r+1}b=(-1)^{(n-1)j+(n-1)(m-1)+r(m-1)}a\circ_{r+1} b.
\end{equation}
As can be seen, this is the same sign as the graded operadic suspension but with vertical degree.

We of course have the following theorem with similar proof to the graded case, where all the suspensions are vertical.
\begin{thm}\label{markl}
Given a bigraded $R$-module $A$, there is an isomorphism of operads $\End_{ A}\cong \mathfrak{s}\End_{SA}$, where $\End_A$ is the endomorphism operad of $A$.\qed
\end{thm}

 

Now we are going to combine vertical operadic suspension and totalization. More precisely, the totalized vertical suspension a bigraded operad $\OO$ is the graded operad $T\s\OO$. 

%TsOO(n)^{n1} SOO(n)^{n1-k1}_k1  OO(n)^{n1-k1-n+1}_k1

This operad has an insertion map explicitly given by
\[(a\star_{r+1} b)_k=\sum_{k_1+k_2=k}(-1)^{(n-1)(n_2-k_2-n+1)+(n-1)(m-1)+r(m-1)+k_1n_2}a_{k_1}\circ_{r+1}b_{k_2}\]
for $a=(a_i)_i\in T\s\OO(n)^{n_1}$ and $b=(b_j)_j\in T\s\OO(m)^{n_2}$. As usual, denote \[a\star b=\sum_{r=0}^{m-1}a\star_{r+1}b.\]

This star operation is precisely the star operation from RWL, i.e the convolution operation on $\Hom((dAs)^!, \End_A)$ (see the reference for details). 

DESCRIBE DERIVED AINFTY MORPHISMS AND COMPOSITION WITH THIS (WILL HAVE TO TALK ABOUT TOTALIZING COLLECTIONS)

Before continuing, let us show a Lemma that allows us to work only with the single graded operadic suspension if needed.
\begin{lem}
For a bigraded operad $\OO$ we have an isomorphism $T\s\OO\cong \s T\OO$, where the suspension on the left hand side is the bigraded version and on the right hand side is the single graded version. 
\end{lem}
\begin{proof}
For an element $(a_k)_k\in T\s\OO(n)$ the isomorphism is given by
\begin{align*}
f:T\s\OO&\cong \s T\OO\\
(a_k)_k&\mapsto ((-1)^{kn}a_k)_k
\end{align*}
Cleary this map is bijective so we just need to check that it commutes with insertions. COMPLETE, MAYBE RESTRICT HOMOGENEOUS BIDEGREE IF NECESSARY
\end{proof}

\begin{remark}
There exist other possible ways of totalizing operads by varying the natural transformation $\mu$. For instance, we can choose the totalization functor $T'$ which is the same as $T$ but  with $\mu$ defined in such a way that the insertion on $T'\OO$ is defined by \[(a\hat{\circ}b)_k=\sum_{k_1+k_2=k}(-1)^{k_2n_1}a_{k_1}\circ b_{k_2}.\] 

This is also a valid approach for our purposes and there is simply a sign difference, but we have chosen our convention to be consistent with other conventions and constructions (SIGN FOR INFINITY MORPHISMS AND CONVOLUTION OPERAD). However, let us mention some differences and relations between $T$ and $T'$ . %There are of course many other possible conventions MAYBE NOT MENTION THE EXACT DIFFRENES, JUST STOP HERE BECAUSE THERE ARE MORE CONVENTIONS AND I DON'T THINK I WILL TREAT ALL OF THEM IN DETAIL, BUT THE DIFFERENCE BETWEEN IDENTITY AND ISOMORPHISM MIGHT BE WORTH MENTION

First of all, $T$ and $T'$ are isomorphic. The isomorphism $f:T\OO\cong T'\OO$ given by $f((a_k))=((-1)^{k(n_1-k)}a_k)$. MAYBE ALL OF THEM ARE ISOMORPHIC, THINK OF THIS USING JUST THAT A TOTALIZATION IS A FUNCTOR THAT SENDS A BIGRADED OPERAD TO A GRADED OPERAD, MAYBE ALSO USE THAT THE UNDERLYING MODULE IS ALWAYS THE SAME SO THAT THE ONLY MODIFICATION IS THE INSERTION

It can be readily verified that $T'\s\OO=\s T'\OO$. With the original totalization we have a non identity isomorphism $f_n:T\s\OO(n)\cong\s T\OO(n)$ given by $f((a_k))=((-1)^{kn}a_k)$. We will use this isomorphism later.

POSSIBLY FILL IN SOME DETAILS OF THESE ISOMORPHISMS

AT THE END  OF THE PROOF OF WHITEHOUSE 4.47 WHERE AINFTY ON TWISTED COMPLEX IS EQUIVALENT TO DERIVED-AINFTY THIS ISOMORPHISM IS  USED SINCE APLIED TO MIJ  THE EXPONENT IS PRECISELY IJ

ALSO IN THAT THEOREM THE DIFERENTIALS OF THE TWISTED COMPLEX CORRESPOND PRECISELY TO MI1 SIMILAR TO MY MI1 BEING LIKE DIFFERENTIALS (I SHOULD CHECK IF THEY SATISFY THE TWISTED COMPLEX RELATION)


DEF 4.51 FILTERED (SO MAYBE CHANGE TITLE TO FILTERED AINFTY AND DAINFTY)
\end{remark}
\section{Derived $A_\infty$-algebras}
DEFINE THE OPERAD,BOTH FREE WITH A DIFFERENTIAL AND ALSO WITH ALL THE DIFFERENTIALS INSIDE
  \begin{definition}
  Using the notation in \cite{RW}, a \emph{derived $A_\infty$-algebra} on a $(\N,\Z)$-bigraded $R$-module $A$ consist of a family of $R$-linear maps 
\[m_{ij}:A^{\otimes j}\to A\]
of bidegree $(i,2-(i+j))$ for each $j\geq 1$, $i\geq 0$, satisfying the equation
\[\underset{j=r+1+t}{\sum_{u=i+p, v=j+q-1}}(-1)^{rq+t+pj}m_{ij}(1^{\otimes r}\otimes m_{pq}\otimes 1^{\otimes t})=0\]
for all $u\geq 0$ and $v\geq 1$. 


Similarly, a \emph{derived $A_\infty$ multiplication} $m$ on a bigraded operad $\OO$ is an element $m=\sum_{ij}m_{ij}$ where $m_{ij}\in\OO(j)^{2-i-j}_i$ for each $j\geq 1$ such that 
\[\underset{j=r+1+t}{\sum_{u=i+p, v=j+q-1}}(-1)^{rq+t+pj}m_{ij}\circ_{r+1} m_{pq}=0.\]
\end{definition}
\begin{remark}
One can readily check that, on any derived $A_\infty$-algebra $A$, the maps $d_i=m_{i1}$ MAYBE MULTIPLIED BY $(-1)^i$ CHECK CONVENTIONS define a twisted complex structure. This leads to the possibility of defining a derived $A_\infty$-algebra as a twisted complex with some extra structure. SEE  REMARK 4.48 FROM WHITEHOUSE
\end{remark}
From the definition of $T\s\OO$, such a multiplication is equivalent to an element $m\in T\s\OO$ of degree 1 concentrated in positive arity satisfying $m\star m=0$. From this, it follows from our previous results that $m$ determines an $A_\infty$-algebra structre on $ST\s\OO\cong S\s T\OO$ %IT WAS EQUALITY WITH THE ALTERNATIVE OPERATION 
that can be iterated to define an $A_\infty$-structure on $S\s\End_{S\s T\OO}$, which is related to the structure on $S\s T\OO$ by a map of $A_\infty$-algebras  $\Phi:S\s T\OO
\to  S\s\End_{S\s T\OO}$.  MAYBE WRITE THIS MORE EXPLICITLY 

The goal now is showing that this $A_\infty$-structure on $ST\s\OO$ is equivalent to a derived $A_\infty$-structure on $S\s \OO$. 
%HERE THE STAR OPERATION, I NEED TO DEFINE DERIVED AINFTY MULTIPLICATION ON AN OPERAD, THEN STAR OPERATION, THEN ELEMENT OF TOTALIZED VERTICAL SUSPENSION


\section{Braces}\label{sectionbraces}
First recall the definition of a brace algebra.

\begin{defi}\label{braces}
A brace algebra on a graded module $A$ consists of a family of maps \[b_n:A^{\otimes 1+n}\to A\] called \emph{braces}, that we evaluate on $(x,x_1,\dots, x_n)$ as $b_n(x;x_1,\dots, x_n)$, satisfying the \emph{brace relation}


\begin{align*}
b_m(b_n(x;x_1,\dots, x_n);y_1,\dots,y_m)=&\\
\sum_{i_1,\dots, i_n, j_1\dots, j_n}(-1)^{\varepsilon}&b_l(x; y_1,\dots, y_{i_1},b_{j_1}(x_1;y_{i_1+1},\dots, y_{i_1+j_1}),\dots, b_{j_n}(x_n;y_{i_n+1},\dots, y_{i_n+j_n}),\dots,y_m)
\end{align*}
where $l=n+\sum_{p=1}^n i_p$ and $\varepsilon=\sum_{p=1}^n|x_p|\sum_{q=i}^{i_p}|y_q|,$ i.e. the sign is picked up by the $x_i$'s passing by the $y_i$'s in the shuffle.



\end{defi}

\begin{remark}
Some authors might use the notation $b_{1+n}$ instead of $b_n$, but the first element is usually going to have a different role than the others. A shorter notation for $b_n(x;x_1,\dots,x_n)$ found in the literature is $x\{x_1,\dots, x_n\}$. Also note that we have used the notation $|x_p|$ for the degree of $x_p$ in $A$. 
\end{remark}

We can define braces on $T\s\OO$ via operadic composition. More precisely, we define the maps 
$$b^T_n:T\mathfrak{s}\OO(N)\otimes T\mathfrak{s}\OO(a_1)\otimes\cdots\otimes T\mathfrak{s}\OO(a_n)\to T\mathfrak{s}\OO(N-\sum a_i)$$
using the operadic composition $\gamma$ on $T\mathfrak{s}\OO$ as

\[b^T_n(f;g_1,\dots,g_n)=\sum\gamma(f;1,\dots,1,g_1,1,\dots,1,g_n,1,\dots,1),\]

where the sum runs over all possible ordering preserving insertions. The brace $b^T_n(f;g_1,\dots,g_n)$ vanishes whenever $n>N$ and $b^T_0(f)=f$. We use the notation $b^T_n$ to distinguish this brace map from the brace $b_n$ that can be naturally defined on the bigraded operad $\s\OO$.

The operadic composition can be described in terms of insertions in the obvious way, namely 

$$\gamma(f;h_1,\dots, h_N)=(\cdots(f\star_1 h_1)\star_{1+a(h_1)}h_2\cdots)\star_{1+\sum a(h_p)}h_N,$$

where $a(h_p)$ is the arity of $h_p$ (in this case $h_p$ is either $1$ or some $g_i$). If we want to express this composition in terms of the composition in $\OO$ we just have to find out the factor sign iterating the same argument as in the graded case. In fact, there is a sign factor identical to the graded case replacing internal degree by vertical (internal) degree that comes from operadic suspension. Therefore we only need to compute the sign factor corresponding to totalization. Given $f_{p_0}\in \OO(N)^{q_0}_{p_0}$ and  $g_i\in\OO(a_i)^{q_i}_{p_i}$ for $1\leq i\leq n$ such that $p_0+p_1+\cdots+p_n=k$, the factor sign we are looking for is determined by the exponent

%\[\varepsilon_k=p_1(N+q_0+p_0-1)+p_2(N+a_1+q_0+q_1+p_0+p_1-2)+\cdots=\sum_{i=1}^np_i(\sum_{j=0}^{i-1}(a_j+q_j+p_j)+N-i).\]

\[\varepsilon
=p_0(a_1+q_1+p_1-1)+(p_0+p_1)(a_2+q_2+p_2-1)+\cdots=\sum_{i=1}^n(a_i+q_i+a_i-1)\sum_{j=0}^{i-1}p_j\]\[=\sum_{0\leq j<i\leq n}p_j(a_i+q_i+p_i-1).\]
This is obtained by iteration of the last factor sign in EQUATION WITH THE SIGN ABOVE which is precisely the sign determined by totalization. Therefore, if $(-1)^\eta$ is the sign produced by operadic suspension and $(-1)^{\varepsilon_k}$ the sign produced by totalization, the factor sign that distinguishes the brace in $T\s\OO$ from the usual operadic composition in $\OO$ is $(-1)^{\eta+\varepsilon_k}$ THE K IS BECAUSE I'M PLANNING TO REWRITE THINGS AS SUMS IN TOTALIZATION, I MIGHT ABOUT THIS, I WILL HAVE TO CHANGE THE NOTATION FOR THE MAPS BEING COMPOSED AND THEIR DEGREES

 USING BRACES TO DEFINE A DERIVED AINFTY ALGEBRA ON THE OPERAD (I WILL PROBABLY  WRITE THE COMPUTATION OF DEGREES DIFFERENTLY LATER BECAUSEE I AM IDENTIFYING THE BRACE AND ELEMENTS WITH THEIR SHIFTS HERE TO MAKE IT EASIER TO COMPPUTE)

Let $m_{il}$ a component of the derived $A_\infty$-multiplication $m$ and $x_1,\dots, x_j\in\OO$ with $x_k$ of bidegree $(i_k,l_k)$, and let us compute the bidegree of $b_j(m_{il};x_1,\dots, x_j)$ in $S\s\OO$. By definition, the horizontal degree on $S\s\OO$ is the same as in $\OO$, so it is $i+\sum_{k=1}^ji_k$, meaning that the horizontal degree of the map $b_j(m_{il};-):S\s\OO^{\otimes j}\to S\s\OO$ is precisely $i$. Now let us compute the vertical degree. By definition it is given by the internal vertical degree plus the arity. Therefore, let us compute

\[
a(b_j(m_{il};x_1,\dots, x_j))+\deg(b_j(m_{il};x_1,\dots, x_j))
\]
where

\[
a(b_j(m_{il};x_1,\dots, x_j))=l-j+\sum a(x_k)
\]
and 
\[\deg(b_j(m_{il};x_1,\dots, x_j))=2-l-i+\sum \deg(x_k)\]
so the sum is $2-i-j+\sum (a(x_k)+\deg(x_k))$, so that the vertical degree of the map $b_j(m_{il};-)$ on the shift is $2-i-j$. This means, that we have the following candidates for a derived $A_\infty$-structture
THESE ARE CANDIDATES BEFORE SHIFTING  AND I ALSO NEED TO TOTALIZE

\[M_{ij}(x_1,\dots, x_j)=\sum_l b_j(m_{il};x_1,\dots, x_j)\]
\[M_{i1}(x)= \sum_l (b_1(m_{il};x)-(-1)^{\langle x,m_{il}\rangle}b_1(x;m_{il}))\]
to be a derived structure on that shift. THE SSIGN SHOULD BE THE SCALAR PRODUCT OF THE BIDEGREES $\langle x,m_{il}\rangle=x_hi+x_v(1-i)$ (degrees in $\s\OO$)

CHECK THAT THEY SATISFYI THE EQUATION UP TO SIGN AND WORRY LATER ABOUT THE SIGNS

DO THE MI1 DEFINE A TWISTED COMPLEX? THIS IS A NECESSARY CONDITION TO SATISSFY THE DAINFTY EQUATION THAT MIGHT BE EASIER TO SHOW AND IIT WOULD MAKE EASIER TO CONNECT EVERYTHING HERE WITH SARAH


\section{Using the theorem (better name will come)}

VERY SKETCHY AT THE MOMENT

We are going to use the following theorem from WHITEHOUSE (CITE) to show thate there is a derived $A_\infty$-structure on $A=S\s\OO$. Note that $TSB=STB$ for any bigraded module $B$, where $SB$ is the vertical suspension and $STB$ is the suspension as graded modules.

THEOREM, REWRITE FOR CONVENTIONS

There are several important assumptions to make in order to use the theorem. First of all, we need $A$ to be horizontally bounded, meaning that there exists some integer $i$ such that $A_k^n=0$ for all $k>i$. In our case, this means that for each $j>0$ we can only have finitely many non-zero $dA_\infty$ maps $m_{ij}$. This situation happens in practice in all known examples of derived $A_\infty$-algebras so far MURO, RW, SARAH

We also need to provide $A$ with a twisted complex structure. The reason for this is that the theorem uses the definition of derived $A_\infty$-algebras on an underlying twisted complex WRITE DOWN THE TWO EQUIVALENT DEFINITIONS AND CITE HERE. We are going to do this by hand, but later we will prove another version of this theorem that works for bigraded modules and we will see that the induced twisted complex structure is the same that we are going to give. 


PROOF OF TWISTED COMPLEX
\section{I need to rewrite this but I'll leave it  here for now}
I have the following candidates to induce derived $A_\infty$-structure on $S\s\OO$ after shifting (similar to what I did for $A_\infty$-algebras). 
\[M_{ij}(x_1,\dots, x_j)=\sum_l b_j(m_{il};x_1,\dots, x_j)\]
\[M_{i1}(x)= \sum_l (b_1(m_{il};x)-(-1)^{\langle x,m_{il}\rangle}b_1(x;m_{il}))\]

To apply the theorem by Sarah I need the bigraded module $S\s\OO$ to be a twisted complex. Since all the arities involved are 1 and the shift is only vertical, while the signs appearing in the twisted complex equation are related only to the horizontal degree, this is identical to a twisted complex structure on $\s\OO$. The twisted complex structure should be given by the maps $\{M_{i1}\}_{i\geq 0}$. I am going to try to show that this the case. First I will try without signs, to see that at least it is conceptually possible, and  then I would work out the signs.

Up to sign, the maps  $\{M_{i1}\}_{i\geq 0}$ must satisfy the equation

\[\sum_{i+j=m} M_{i1}\circ M_{ji}=0,\]
where $\circ$ is composition of maps. I may or may not omit the sum to avoid writing too much, as it just means that the composition on every degree must vanish.

Therefre, up to signs I compute 

\begin{align*}
\sum_{i+j=m}M_{i1}(M_{j1}(x))=\sum_{i+j=m}M_{i1}\left(\sum_l b_1(m_{jl};x)+b_1(x;m_{jl})\right)=\\
\sum_{i+j=m}\sum_{l,k}\left(b_1(m_{ik}; b_1(m_{jl};x))+b_1(m_{ik};b_1(x;m_{jl}))+b_1(b_1(m_{jl};x);m_{ik})+b_1(b_1(x;m_{jl});m_{ik})\right)
\end{align*}

%AT FIRST SIGHT IT DOESN'T LOOK POSSIBLE TO CANCEL THE LAST BRACE BECAUSE IT IS THE ONLY ONE WITH X AT THE BEGINNING, BUT THAT SHOULD HAVE BEEN THE SAME FOR THE CLASSICAL CASE, SO I SHOULD REVIEW THAT ONE
%
%ON THE CLASSICAL CASE IT WAS MUCH EASIER BECAUSE AFTER BRACE RELATION B(X;M,M) APPEARS TWICE WITH OPPOSITE SIGN, SO IT CANCELS. HERE IT IS NOT SO OBVIOUS BECAUSE THE SIGN IS NOT JUST $-1$ SO MAYBE IT CANCELS WITH OTHER SUMMANDS (RECALL THAT I AM OMITTING ONE SUM)

Applying the brace relation we obtain

\begin{align*}
\sum_{i+j=m}\sum_{l,k}(b_1(m_{ik}; b_1(m_{jl};x))+b_1(m_{ik};b_1(x;m_{jl}))+\\
 b_2(m_{jl};x,m_{ik})+b_1(m_{jl};b_1(x;m_{ik}))+b_2(m_{jl};m_{ik},x)+\\
b_2(x;m_{jl},m_{ik})+b_1(x;b_1(m_{jl};m_{ik}))+b_2(x;m_{ik},m_{jl}))
\end{align*}

In the sum all terms of the form $b_1(x;b_1(m_{jl};m_{ik}))$ that can be seen in the last line should add up to vanish provided that $m$ is a $dA_\infty$-multiplication (meaning that up to sign $b_1(m;m)=0$). A sign of the form $(-1)^i$ (or maybe $(-1)^j$) should be in in front of each of these terms. Since $i$ and $j$ are interchangable (i.e. for each pair $(i,j)$ there is the pair $(j,i)$), the terms $b_2(x;m_{jl},m_{ik})+b_2(x;m_{ik},m_{jl}))$ in the last line should cancel as well (for this, the other appearance should come with opposite sign). Here it is relevant as well, that the sum covers all the values of $k$ and $l$, so that the pair $(j,i)$ comes with $l$ and $k$ interchanged as well. 

Then $b_1(m_{ik};b_1(x;m_{jl}))$ in te first line should cancel with $b_1(m_{jl};b_1(x;m_{ik}))$ on the second line (but from a different summand, the one where $i$ and $j$ are interchanged). Finnaly, the reamining terms $b_1(m_{ik}; b_1(m_{jl};x))+b_2(m_{jl};x,m_{ik})+b_2(m_{jl};m_{ik},x)$ seem to be adding up to $b_1(b_1(m;m);x)$, but again some signs are going to be needed. That would cancel everything, so at least up to sign this makes sense.


\subsection{Signed version}

Now let us add signs. We now compute 
\[\sum_{i+j=m} (-1)^iM_{i1}\circ M_{ji}\]
recalling that for the usual sign convention of twisted complex from a $dA_\infty$-algebra we need to define $d_i=(-1)^im_{i1}$, so that the sign in tthe equation is $(-1)^j$ instead of $(-1)^i$. This being said, let us compute 
\begin{align*}
\sum_{i+j=m}(-1)^iM_{i1}(M_{j1}(x))=\sum_{i+j=m}(-1)^iM_{i1}\left(\sum_l b_1(m_{jl};x)-(-1)^{\langle x|m_{jl}\rangle}b_1(x;m_{jl})\right)=\\
\sum_{i+j=m}(-1)^i\sum_{l,k}\left(b_1(m_{ik}; b_1(m_{jl};x))-(-1)^{\langle x|m_{jl}\rangle}b_1(m_{ik};b_1(x;m_{jl}))+\right.\\
\left. -(-1)^{\langle b_1(m_{jl};x)|m_{ik}\rangle}b_1(b_1(m_{jl};x);m_{ik})+(-1)^{\langle b_1(m_{jl};x)|m_{ik}\rangle+\langle x|m_{jl}\rangle}b_1(b_1(x;m_{jl});m_{ik})\right)
\end{align*}
Observe that $\langle b_1(m_{jl};x)|m_{ik}\rangle=\langle m_{ij}|m_{ik}\rangle+\langle x|m_{ik}\rangle$ in the usual bigraded sign convention (also in the total graded convention). I am not going to explicitly compute these signs yet to see what properties we need from them.

Applying the brace relation we obtain

\begin{align*}
\sum_{i+j=m}\sum_{l,k}((-1)^ib_1(m_{ik}; b_1(m_{jl};x))-(-1)^{i+\langle x|m_{jl}\rangle}b_1(m_{ik};b_1(x;m_{jl}))+\\
 -(-1)^{i+\langle b_1(m_{jl};x)|m_{ik}\rangle}(b_2(m_{jl};x,m_{ik})+(-1)^{\langle x|m_{ik}\rangle}b_2(m_{jl};m_{ik},x))\\
 -(-1)^{i+\langle b_1(m_{jl};x)|m_{ik}\rangle}b_1(m_{jl};b_1(x;m_{ik}))\\
+(-1)^{i+\langle b_1(m_{jl};x)|m_{ik}\rangle+\langle x|m_{jl}\rangle}(b_2(x;m_{jl},m_{ik})+(-1)^{\langle m_{ik}|m_{jl}\rangle}b_2(x;m_{ik},m_{jl}))\\
+(-1)^{i+\langle b_1(m_{jl};x)|m_{ik}\rangle+\langle x|m_{jl}\rangle}b_1(x;b_1(m_{jl};m_{ik})))
\end{align*}

Recall that $m$ being a $dA_\infty$-multiplication means that $\sum_{i+j=m}\sum_{k,l}(-1)^ib_1(m_{ik};m_{ij})=0$. Notice that the summand corresponding to each value of $i+j$ must vanish because it corresponds to a given horizontal degree. Let us check now the cancellations with the signs. First, let us check that the terms 
\[(-1)^{i+\langle b_1(m_{jl};x)|m_{ik}\rangle+\langle x|m_{jl}\rangle}b_1(x;b_1(m_{jl};m_{ik})))\]
can be added up to vanish. For that, we compute the sign \[\langle b_1(m_{jl};x)|m_{ik}\rangle+\langle x|m_{jl}\rangle=\langle m_{jl}|m_{ik}\rangle+\langle x|m_{ik}\rangle+\langle x|m_{jl}\rangle\]
Recall that the braces are defined on the operadic suspension, so that the bidegree of $m_{ik}$ is $(i,1-i)$. Therefore, writing the bidegree of $x$ as $(x_h,x_v)$, the above equals 
\[ji+(1-i)(1-j)+x_hi+x_v(1-i)+x_hj+x_v(1-j)\equiv 1+i+j + (i+j)x_h+(i+j)x_v\mod 2=\]
\[(i+j)(1+x_h+x_v)=1+m(1+x)\]
Since this sign is constant for all terms $b_1(m_{ik};m_{ij})$ that share the same horizontal degree $i+j=m$, we can rewrite
\[(-1)^{i+\langle b_1(m_{jl};x)|m_{ik}\rangle+\langle x|m_{jl}\rangle}b_1(x;b_1(m_{jl};m_{ik})))=-(-1)^{m(1+x)}b_1(x;(-1)^ib_1(m_{ik};m_{jl})),\]
which vanishes when we consider the whole sum
%Multiplying 0 by something is 0, some sum vanishe and you multiply it by a constant sign, it still vanishes
\[\sum_{i+j=m}\sum_{k,l}-(-1)^{m(1+x)}b_1(x;(-1)^ib_1(m_{ik};m_{jl}))=0.\]
Therefore, the equation after the brace relation reduces to
\begin{align*}
\sum_{i+j=m}\sum_{l,k}((-1)^ib_1(m_{ik}; b_1(m_{jl};x))-(-1)^{i+\langle x|m_{jl}\rangle}b_1(m_{ik};b_1(x;m_{jl}))+\\
 -(-1)^{i+\langle b_1(m_{jl};x)|m_{ik}\rangle}(b_2(m_{jl};x,m_{ik})+(-1)^{\langle x|m_{ik}\rangle}b_2(m_{jl};m_{ik},x))\\
 -(-1)^{i+\langle b_1(m_{jl};x)|m_{ik}\rangle}b_1(m_{jl};b_1(x;m_{ik}))\\
+(-1)^{i+\langle b_1(m_{jl};x)|m_{ik}\rangle+\langle x|m_{jl}\rangle}(b_2(x;m_{jl},m_{ik})+(-1)^{\langle m_{ik}|m_{jl}\rangle}b_2(x;m_{ik},m_{jl}))
\end{align*}
Let us focus on the last line. For each pair $(i,j)$ we should have cancellation with the pair $(j,i)$, which adds the same elements, but with different signs. We also need to consider the pairs $(k,l)$ and $(l,k)$ to get a cancellation. Let us compare the signs. For the pair $((i,j),(k,l))$ we have precisely the last line of the above equation
\[(-1)^{i+\langle b_1(m_{jl};x)|m_{ik}\rangle+\langle x|m_{jl}\rangle}(b_2(x;m_{jl},m_{ik})+(-1)^{\langle m_{ik}|m_{jl}\rangle}b_2(x;m_{ik},m_{jl}))\]

For the pair $((j,i),(l,k))$ we have
\[(-1)^{j+\langle b_1(m_{ik};x)|m_{jl}\rangle+\langle x|m_{ik}\rangle}(b_2(x;m_{ik},m_{jl})+(-1)^{\langle m_{jl}|m_{ik}\rangle}b_2(x;m_{jl},m_{ik}))\]
 Comparing the sign of $b_2(x;m_{jl},m_{ik})$ we find that for $((i,j),(k,l))$ we have

\[-(-1)^{i+(i+j)(1+x)}b_2(x;m_{jl},m_{ik})=-(-1)^{j+(i+j)x}b_2(x;m_{jl},m_{ik})\]
and for the pair $((j,i),(l,k))$ we have
\[(-1)^{j+(i+j)x}b_2(x;m_{jl},m_{ik})\]
As we see, we get opposite signs and thus cancellation. For $b_2(x;m_{ik},m_{jl})$ it is completely analogous. Thus, we have reduced our main equation to 
\begin{align*}
\sum_{i+j=m}\sum_{l,k}((-1)^ib_1(m_{ik}; b_1(m_{jl};x))-(-1)^{i+\langle x|m_{jl}\rangle}b_1(m_{ik};b_1(x;m_{jl}))+\\
 -(-1)^{i+\langle b_1(m_{jl};x)|m_{ik}\rangle}(b_2(m_{jl};x,m_{ik})+(-1)^{\langle x|m_{ik}\rangle}b_2(m_{jl};m_{ik},x))\\
 -(-1)^{i+\langle b_1(m_{jl};x)|m_{ik}\rangle}b_1(m_{jl};b_1(x;m_{ik}))
\end{align*}
In a similar fashion to the previous calculation, we are going to try to cancel $b_1(m_{ik};b_1(x;m_{jl}))$ in the first line with $b_1(m_{jl};b_1(x;m_{ik})$ in the last line by considering switched pairs. For the pair $((i,j),(k,l))$, the term in the first line is 
\[-(-1)^{i+\langle x|m_{jl}\rangle}b_1(m_{ik};b_1(x;m_{jl}))\]
and for the pair $((j,i),(l,k))$ the term in the last line is
\[-(-1)^{j+\langle b_1(m_{ik};x)|m_{jl}\rangle}b_1(m_{ik};b_1(x;m_{jl}))=(-1)^{1+j+\langle m_{ik}|m_{jl}\rangle+\langle x|m_{jl}\rangle}b_1(m_{ik};b_1(x;m_{jl}))=\]
\[(-1)^{i+\langle x|m_{jl}\rangle}b_1(m_{ik};b_1(x;m_{jl}))\]
which has precisely the opposite sign to the other pair, and thus cancels. This reduces the main equation to just 
\begin{align*}
\sum_{i+j=m}\sum_{l,k}((-1)^ib_1(m_{ik}; b_1(m_{jl};x))
 -(-1)^{i+\langle b_1(m_{jl};x)|m_{ik}\rangle}(b_2(m_{jl};x,m_{ik})+(-1)^{\langle x|m_{ik}\rangle}b_2(m_{jl};m_{ik},x))
\end{align*}
We want this therms to add up to something of the form $b_1(b_1(m;m);x)$. Notice that for that we need to switch some pairs. For simplity, we switch the pair of the first term and rewrite the sum after simplifying some signs as
\begin{align*}
\sum_{i+j=m}\sum_{l,k}((-1)^jb_1(m_{jl}; b_1(m_{ik};x))
 -(-1)^{i+\langle b_1(m_{jl};x)|m_{ik}\rangle}b_2(m_{jl};x,m_{ik})-(-1)^{i+\langle m_{jl}| m_{ik}\rangle}b_2(m_{jl};m_{ik},x))
\end{align*}
Simplifying further the signs we get
\begin{align*}
\sum_{i+j=m}\sum_{l,k}((-1)^jb_1(m_{jl}; b_1(m_{ik};x))
 +(-1)^{j+\langle x|m_{ik}\rangle}b_2(m_{jl};x,m_{ik})+(-1)^{j}b_2(m_{jl};m_{ik},x)).
\end{align*}
By the brace relation this equals
\[\sum_{i+j=m}\sum_{l,k}(-1)^jb_1(b_1(m_{j,l};m_{ik});x)=0\]
%For each position of insertion of x we have a 0 map applied to x, so the above sum is indeed equal to 0

\section{A section just to separate the previous thing from the rest}
LEMMA THAT THE INVERSE OF MU INDUCES THE INVERSE ENRICHED FUNCTOR?

PROOF OF DAINFTY (CHAIN OF ISOS IN REVERESE):
1- FIRST ISO DOES NOTHING
2- SPLIT INTO FINDING ENRICHED MU-1 AND THEN ENRICHED TOT-1
3- THE PROJECTION GIVES THE BIGRADED BRACE, SIGNS INCLUDED
4- FINAL SIGN


ALTERNATIVE VERSION

SHOW THAT THE  RESULTING TWISTED COMPLEX IS THE SAME
%\appendix
%\renewcommand{\appendixname}{Appendix:}
\begin{appendices}
\appendix
\gdef\thesection{Appendix \Alph{section}}
\section{Some proofs and details}




\end{appendices}
%\phantomsection
\bibliographystyle{ieeetr}
\bibliography{newbibliography}
\end{document}
