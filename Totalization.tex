	\documentclass[twoside]{article}
\usepackage{estilo-ejercicios}
\setcounter{section}{0}
%\newtheorem{defin}{Definition}[section]
%\newtheorem{lem}[defin]{Lemma}
%\newtheorem{propo}[defin]{Proposition}
%\newtheorem{thm}[defin]{Theorem}
%\newtheorem{eje}[defin]{Example}
%\renewcommand{\baselinestretch}{1,3}

%\usepackage{calligra}
%\usepackage[T1]{fontenc}
\usepackage{empheq}
\newcommand*\widefbox[1]{\fbox{\hspace{2em}#1\hspace{2em}}}

%--------------------------------------------------------
\begin{document}

\title{Derived $A_\infty$-structures on operads}
\author{Javier Aguilar Martín}
\maketitle


\section{Introduction}
STYLE, SHOULD I USE THEOREM STYLE FOR DEFINITIONS AND EVERYTHING ELSE AS WELL? IN THAT CASE HOW SHOULD I HIGHLIGH THE NAMES OF THE THINGS I DEFINE?

We use an operadic totalization inspired by the totalization functor described in \cite{whitehouse} to obtain an operation similar to the star operation in \cite{LRW} and generalize the construction based on operadic suspension that has been done for $A_\infty$-algebras to the more general derived $A_\infty$-algebra.

We start \Cref{background} by collecting some preliminary definitions and results from \cite{whitehouse} that we adapt to our conventions. In \Cref{operadic}, we then define the totalization functor for operads and then the bigraded version of operadic suspension. We combine these two constructions to define an operation that allows us to understand a derived $A_\infty$-multiplication as a Maurer-Cartan element. From this, we obtain in \Cref{sectionbraces} a brace structure from which we can obtain a classical $A_\infty$-algebra. Finally, in \Cref{derivedstructure}, we use \cite[Proposition 4.55]{whitehouse} to show that this structure is equivalent to a derived $A_\infty$-algebra on the suspended bigraded operad.

\section{Brackground and conventions}\label{background}
Let us start by  fixing some notation and conventions. Fix a commutative ring with unit $R$ of characteristic distinct of $2$. All tensor products taken over $R$. %COPY SECTION 2.2 OF DAINFTY AND THEIR HOMOTOPIES AS BACKGROUND, REFERENCES TO WHITEHOUSE. INCLUDING DEFINITION OF TWISTED COMPLEX

%REDEFINE THINGS ACCORDING TO CONVENTIONS

Let $\CC$ be a category and let $A$, $B$ be arbitrary
objects in $\CC$. We denote by $\Hom_\CC(A,B)$ the set of morphisms from $A$ to $B$ in $\CC$. If $(\CC,⊗, 1)$ is
symmetric monoidal closed, then we denote its internal hom-object by $[A,B] ∈ \CC$.

We collect some preliminary definitions. Most of them come from \cite[\S 2]{whitehouse} but we adapt them here to our conventions.

\subsection{Filtered Modules and complexes}
%INCLUSION IS REVERSED
We collect some definitions about filtered modules and filtered complexes.

\begin{defin}
A \emph{filtered $R$-module} $(A, F)$ is given by a family of $R$-modules $\{F_pA\}_{p∈\Z}$ indexed by
the integers such that $F_{p}A ⊆ F_{p-1}A$ for all $p ∈ \Z$ and $A = ∪F_pA$. A morphism of filtered modules is a morphism $f : A → B$ of $R$-modules which is compatible with filtrations: 
\[f(F_pA) ⊂ F_pB \text{ for all }p ∈ \Z.\]
\end{defin}

\begin{defin}\label{filteredcomplex}
A \emph{filtered complex} $(K, d, F)$ is a complex $(K, d) ∈ \mathrm{C}_R$ together with a filtration $F$ of each $R$-module $K^n$ such that $d(F_pK^n) ⊂ F_pK^{n+1}$ for all $p, n ∈ \Z$. Its morphisms are given by
morphisms of complexes $f : K → L$ compatible with filtrations: \[f(F_pK) ⊂ F_pL\text{ for all }p ∈ \Z.\]
\end{defin}

We denote by $\fmod$ and $\fc$ the categories of filtered modules and filtered complexes of $R$-modules, respectively.

%$-1$ CHANGED TO 1 (IT MUST BE LIKE THIS BECAUSE I THE MAPS PRESERVING FILTRATION WILL BE LEVEL 0 AND I WANT THE WHOLE AINFTY OPERAD TO BE MAPPED TO THAT, SO R MUST BE IN LEVEL 0 AS WELL)
\begin{defin}\label{filteredtensor}
The tensor product of two filtered $R$-modules $(A, F)$ and $(B, F)$ is a filtered $R$-module,
with
 \[F_p(A ⊗ B) :=\sum_{i+j=p}\Ima(F_iA ⊗ F_jB → A ⊗ B).\]
This makes the category of filtered $R$-modules into a symmetric monoidal category, where the unit is given by $R$ with the trivial filtration $0 = F_{1}R ⊂ F_0R = R$.
\end{defin}


\begin{defin}\label{filterend}
Let $K$ and $L$ be filtered complexes. We define $\underline{\Hom}(K,L)$ to be the filtered complex whose underlying chain complex is $\Hom_{\mathrm{C}_R}(K,L)$ and the filtration $F$ given by 
\[F_p\underline{\Hom}(K,L)=\{f:K\to L\mid f(F_qK)\subset F_{q+p}L\text{ for all }q ∈ \Z.\}\]
In particular, $\Hom_{\fmod}(K,L)=F_0\underline{\Hom}(K,L)$.
\end{defin}

\subsection{Bigraded modules, vertical bicomplexes, twisted complexes and sign conventions}


We collect some definitions of basic categories that we need to use and stablish some conventions.
%TRY TO INCLUDE ONLY THE NECESSARY

\begin{defin}
We consider $(\Z,\Z)$-bigraded
$R$-modules $A = \{A^j_i\}$, where elements of $A^j_i$ are said to have bidegree $(i, j)$. We sometimes refer to $i$
as the \emph{horizontal} degree and $j$ the \emph{vertical degree}. The \emph{total degree} of an element $a ∈ A^j_i$ is $|a| = i+j$.
\end{defin}
\begin{defin}
A \emph{morphism of bidegree $(p, q)$} maps $A^j_i$ to $A^{j+q}_{i+p}$. The tensor product of two bigraded $R$-modules $A$
and $B$ is the bigraded $R$-module $A ⊗ B$ given by
\[(A ⊗ B)^j_i \coloneqq\bigoplus_{p,q}A^q_p ⊗ B^{j−q}_{i−p} .\]
\end{defin}
We denote by $\bgmod$ the category whose objects are bigraded $R$-modules and whose morphisms
are morphisms of bigraded $R$-modules of bidegree $(0, 0)$. It is symmetric monoidal with the above
tensor product.

We introduce the following scalar product notation for bidegrees: for $x$, $y$ of bidegree $(x_1, x_2)$, $(y_1, y_2)$
respectively, we let $\langle x, y\rangle = x_1y_1 + x_2y_2$.

The symmetry isomorphism
\[τ_{A⊗B} : A ⊗ B → B ⊗ A\]
is given by
\[a ⊗ b \mapsto (−1)^{\langle a,b\rangle}b ⊗ a.\]
We follow the Koszul sign rule: if $f : A → B$ and $g : C → D$ are bigraded morphisms, then the
morphism $f ⊗ g : A ⊗ C → B ⊗ D$ is defined by
\[(f ⊗ g)(a ⊗ c) \coloneqq (−1)^{\langle g,a\rangle}f(a) ⊗ g(c).\]

\begin{defin}
A \emph{vertical bicomplex} is a bigraded $R$-module $A$ equipped with a vertical differential $d^A : A → A$ of bidegree $(0, 1)$. A morphism of vertical bicomplexes is a morphism of bigraded modules
of bidegree $(0, 0)$ commuting with the vertical differential.
\end{defin}

We denote by $\vbc$ the category of vertical bicomplexes. The tensor product of two vertical bicomplexes $A$ and $B$ is given by endowing the tensor product of underlying bigraded modules with
vertical differential \[d^{A⊗B} := d^A ⊗ 1 + 1 ⊗ d^B : (A ⊗ B)^v_u → (A ⊗ B)^{v+1}_u .\] This makes $\vbc$ into a
symmetric monoidal category.

The symmetric monoidal categories $(\mathrm{C}_R,⊗,R)$, $(\bgmod,⊗,R)$ and $(\vbc,⊗,R)$ are related by embeddings $\mathrm{C}_R\to\vbc$ and $\bgmod \to\vbc$ which are monoidal and full.



\begin{defin}\label{delta1}
Let $A,B$ be bigraded modules. We define $[A,B]^∗_∗$
to be the bigraded module of morphisms of bigraded modules $A → B$. Furthermore, if $A,B$ are vertical bicomplexes, and $f ∈
[A,B]^v_u$, we define
\[δ(f) := d_Bf − (−1)^vfd_A.\]
\end{defin}

\begin{lem}
If $A$, $B$ are vertical bicomplexes, then $([A,B]^∗_∗
, δ)$ is a vertical bicomplex.
\end{lem}
\begin{proof}
Direct computation shows $\delta^2=0$.
\end{proof}
%END OF PAGE 5, I DEFINE THE SHIFT LATER, SO MAYBE IT IS NOT NECESSARY, BUT THINK ABOUT SHIFT OF MAPS (IS IT NECESSARY  TO ADD THAT SIGN?) AND COMPARE WITH THE GRADED CASE

%I CHANGED HORIZONTAL DEGREE SIGNS
\begin{defin}\label{twistedcomplex} A \emph{twisted complex} $(A, d_m)$ is a bigraded $R$-module $A = \{A^j_i \}$ together with a family
of morphisms $\{d_m : A → A\}_{m≥0}$ of bidegree $(m,1−m )$ such that for all $m ≥ 0$,
%CHECK CONVENTION IN CASE IT IS EASIER TO TAKE $(-1)^j$ 

%FROM DAINFTY ONE GETS J, BUT MI1 SATISFIES I, SO IN ANY CASE I WOULD HAVE TO DO SOME CHANGE OF VARIABLES
\[\sum_{i+j=m}(−1)^id_id_j = 0.\]

\end{defin}

\begin{defin}\label{twistedmorphisms}
A morphism of twisted complexes $f : (A, d^A_m) → (B, d^B_m)$ is given by a family of morphisms of $R$-modules $\{f_m : A → B\}_{m≥0}$ of bidegree $(m,−m)$ such that for all $m ≥ 0$,
\[\sum_{i+j=m}d^B_if_j =\sum_{i+j=m}(−1)^if_id^A_j.\]
The composition of morphisms is given by $(g \circ f)_m :=\sum_{i+j=m} g_if_j$.

A morphism $f = \{f_m\}_{m≥0}$ is
said to be \emph{strict} if $f_i = 0$ for all $i > 0$. The \emph{identity} morphism $1_A : A → A$ is the strict morphism
given by $(1_A)_0(x) = x.$ A morphism $f = \{f_i\}$ is an isomorphism if and only if $f_0$ is an isomorphism of
bigraded $R$-modules. Indeed, an inverse of $f$ is obtained from an inverse of $f_0$ by solving a triangular system.
\end{defin}
Denote by $\tc$ the category of twisted complexes. The following construction endows $\tc$ with a symmetric monoidal structure. See \cite[Lemma 3.3]{whitehouse} for a proof.
\begin{lem}\label{tensortwisted}
The category $(\tc,⊗,R)$ is symmetric monoidal, where the monoidal structure is given
by the bifunctor
\[⊗ : \tc × \tc → \tc\]
which on objects is given by $((A, d^A_m), (B, d^B_m)) → (A ⊗ B, d^A_m ⊗ 1 + 1 ⊗ d^B_m)$ and on morphisms is
given by $(f, g) → f ⊗ g$, where $(f ⊗ g)_m :=\sum_{i+j=m} f_i ⊗ g_j$. In particular, by the Koszul sign rule we
have that \[(f_i ⊗g_j)(a⊗b) = (−1)^{\langle g_j ,a\rangle}f_i(a)⊗g_j(b).\] The symmetry isomorphism is given by the strict
morphism of twisted complexes
\[τ_{A⊗B} : A ⊗ B → B ⊗ A\]
defined by
\[a ⊗ b\mapsto (−1)^{\langle a,b\rangle}b ⊗ a.\]
\end{lem}

The internal hom on bigraded modules can be extended to twisted complexes via the following lemma whose proof is in \cite[Lemma 3.4]{whitehouse}.
\begin{lem}\label{di} Let $A,B$ be twisted complexes. For $f ∈ [A,B]^v_u$, setting
\[(d_if) := (−1)^{i(u+v)}d^B_if − (−1)^vfd^A_i,\]
for $i ≥ 0$, endows $[A,B]^∗_∗$ with the structure of a twisted complex.
\end{lem}

%UNDERLINED CATEGORIES?
\subsection{Totalization}\label{total}
%DEFINITION WITH FILTRATION, DIFFERENTIAL, MONOIDALITY
Here we recall the definition of the totalization functor from \cite{whitehouse} and some of the structure that it comes with.

%CHOOSE NOTATION, T OR TOT

\begin{defin}
The \emph{totalization} $\Tot(A)$ of a bigraded $R$-module $A = \{A^j_i \}$ the graded $R$-module is given by
\[\Tot(A)^n \coloneqq
\bigoplus_{i<0}A^{n-i}_i ⊕\prod_{i\geq 0}A^{n-i}_i .\]
The \emph{column filtration} of $\Tot(A)$ is the filtration given by \[F_p\Tot(A)^n \coloneqq\prod_{i\geq p} A^{n-i}_i .\]
\end{defin}

Given a twisted complex $(A, d_m)$, define a map $d : \Tot(A) → \Tot(A)$ of degree $1$ by letting
\[d(a)_j \coloneqq \sum_{m≥0}(−1)^{mn}d_m(a_{j-m}),\]
for $a = (a_i)_{i∈\Z} ∈ \Tot(A)^n$,
where $a_i ∈ A^{n-i}_i$ denotes the $i$-th component of $a$, and $d(a)_j$ denotes the $j$-th component of $d(a)$. Note
that, for a given $j ∈ \Z$ there is a sufficiently large $m ≥ 0$ such that $a_{j-m′} = 0$ for all $m′ ≥ m$. Hence
$d(a)_j$ is given by a finite sum. Also, for negative $j$ sufficiently large, one has $a_{j-m} = 0$ for all $m ≥ 0$, which
implies $d(a)_j = 0$.

Given a morphism $f : (A, d_m) → (B, d_m)$ of twisted complexes, let $\Tot(f) : \Tot(A) → \Tot(B)$ be
the map of degree 0 defined by
\[(\Tot(f)(a))_j \coloneqq \sum_{m≥0}(−1)^{mn}f_m(a_{j-m}),\]
 for $a = (a_i)_{i∈\Z} ∈ \Tot(A)^n$.
 
\begin{thm}
The assignments $(A, d_m) \mapsto (\Tot(A), d, F)$, where $F$ is the column filtration of $\Tot(A)$,
and $f \mapsto \Tot(f)$ define a functor $\Tot : \tc \to \fc$ which is an isomorphism of categories when restricted to its image.
\end{thm}
\begin{proof}
See \cite[Theorem 3.8]{whitehouse}.
\end{proof}
For a filtered complex of the form $(\Tot(A),d,F)$, where $A = \{A^j_i \}$ is a bigraded $R$-module, we can recover the twisted complex structure on  $A$ as follows. For all $m ≥ 0$, let
$d_m : A → A$ be the morphism of bidegree $(m,1-m)$ defined by 
\[d_m(a) = (−1)^{nm}d(a)_{i+m},\] 
where $a ∈ A^{n-i}_i$ and $d(a)_k$ denotes the $k$-th component of $d(a)$, which lies in $A^{n+1-k}_k$.

%Let us denote the image of $\Tot$ by $\sfc$ and define the inverse $\Tot^{-1}:\sfc\to\tc$ as follows. Let $(\Tot(A),d,F)\in\sfc$, where $A = \{A^j_i \}$ is a bigraded $R$-module. For all $m ≥ 0$, let
%$d_m : A → A$ be the morphism of bidegree $(m,1-m)$ defined by 
%\[d_m(a) = (−1)^{nm}d(a)_{i−m},\] 
%CHECK
%where $a ∈ A^{n-i}_i$ and $d(a)_k$ denotes the $k$-th component of $d(a)$, which lies in $A^{n+1-k}_k$. Since $d$ is compatible
%with the filtration $F$, we have $d_i = 0$ for $i < 0$ CHECK. Then $(A, d_m)$ is a twisted complex and its filtered
%total complex is $(\Tot(A), d, F)$. Lastly, let $f : (\Tot(A), d, F) → (\Tot(B), d, F)$ be a morphism of split
%filtered complexes. For all $m ≥ 0$, let $f_m : A → B$ be the morphism of bidegree $(m,−m)$ defined by
%\[f_m(a) = (−1)^{nm}f(a)_{i−m},\] SAME AS D where $a ∈ A^{n-i}_i$ and $f(a)_k$ denotes the $k$-th component of $f(a)$, which lies
%in $B^{n-k}_k$. Since $f$ is compatible with the filtration $F$, we have that $f_i = 0$ for $i < 0$  SAME AS D. Then the family
%$\{f_m\}_{m≥0}$ is a morphism of twisted complexes whose total morphism is $f$.

We will consider the following bounded categories since the totalization functor has better properties when restricted to them. 
%I THINK I WON'T NEED TO SPECIFY THE SPLIT CATEGORIES
\begin{defin}
We let $\tc^b$, $\vbc^b$, $\bgmod^b$ be the full subcategories of \emph{horizontally bounded on the right} graded twisted
complexes, vertical bicomplexes and bigraded modules respectively. This means that if $A=\{A^j_i\}$ is an object of any of this categories, then there exists $i$ such that $A^j_{i'}=0$ for $i'>i$.

We let $\fmod^b$, $\fc^b$ be the full subcategories of bounded filtered modules, respectively complexes, i.e.
the full subcategories of objects $(K, F)$ such that there exists some $p$ with the property that $F_{p'}K^n = 0$ for all $p> p'$. We refer to all of these as the bounded subcategories of $\tc$, $\vbc$, $\bgmod$, $\fmod$ and $\fc$   respectively.
\end{defin}

\begin{propo}\label{monoidal}
The functors $\Tot : \bgmod → \fmod$ and $\Tot : \tc → \fc$ are lax symmetric
monoidal, with structure maps
\[\epsilon : R → \Tot(R)\text{ and }\mu=μ_{A,B} : \Tot(A) ⊗ \Tot(B) → \Tot(A ⊗ B),\]
given by $\epsilon = 1_R$ and for $a = (a_i)_i ∈ \Tot(A)^{n_1}$ and  $b=(b_j)_j ∈ \Tot(B)^{n_2}$,
\[μ(a ⊗ b)_k \coloneqq
\sum_{k_1+k_2=k}(−1)^{k_1n_2}a_{k_1} ⊗ b_{k_2} .\]
When restricted to the bounded case, $\Tot : \bgmod^b
 → \fmod^b$ and $\Tot : \tc^b → \fc^b$ are
strong symmetric monoidal functors.
\end{propo}
\begin{proof}
See \cite[Proposition 3.11]{whitehouse}.
\end{proof}

\begin{remark}\label{heuristic}
There is a certain heuristic to obtain the sign appearing in the definition of $\mu$ in \Cref{monoidal}. In the bounded case, we can write \[\Tot(A)=\bigoplus_i A_i^{n-i}\]
and direct sum conmutes with tensor products. Therefore, we have
\[\Tot(A)\otimes\Tot(B)=(\bigoplus A_i^{n-i})\otimes \Tot(B)\cong \bigoplus_i  (A_i^{n-i}\otimes \Tot(B)).\]

In the isomorphism we can interpret that each $A_i^{n-i}$ passes by $\Tot(B)$. Since $\Tot(B)$ is total graded, we can think of this degree as being the horizontal degree, while having 0 vertical degree. Thus, using the Koszul sign rule we would get precisely the sign from \Cref{monoidal}. This explanation is just an intuition, and opens the door for other possible sign choices: what if we decide to distribute $\Tot(A)$ over $\bigoplus_i B_i^{n-i}$ instead, or if we consider the total degree as the vertical degree? These alternatives lead to valid definitions of $\mu$, and we will explore the consequences of some of them in \Cref{othermu}.
\end{remark}

\begin{lem}\label{mui}
In the conditions of \Cref{monoidal} for the bounded case, the inverse
\[\mu^{-1}:\Tot(A_{(1)}\otimes\cdots\otimes A_{(m)})\to \Tot(A_{(1)})\otimes\cdots\otimes \Tot(A_{(m)})\]
is given on pure tensors (for notational convenience) as
\begin{equation}\label{mu}
\mu^{-1}(a_{(1)}\otimes\cdots\otimes a_{(m)})=(-1)^{\sum_{j=2}^m n_j\sum_{i=1}^{j-1}k_i}a_{(1)}\otimes\cdots\otimes a_{(m)},
\end{equation}
where $a_{(l)}\in (A_{(m)})_{k_l}^{n_l-k_l}$.
\end{lem}
\begin{proof}
For the case $m=2$,
\[\mu^{-1}:\Tot(A\otimes B)\to \Tot(A)\otimes \Tot(B)\]
is computed explicitly as follows.
Let  $c\in\Tot(A\otimes B)^n$. By definition, we have
\[\Tot(A\otimes B)^n=\bigoplus_k (A\otimes B)^{n-k}_k=\bigoplus_k\bigoplus_{n_1+n_2=n}\bigoplus_{k_1+k_2=k}A_{k_1}^{n_1-k_1}\otimes B_{k_2}^{n_2-k_2}\]
And thus $c=(c_k)_k$ may be written as a finite sum $c=\sum_k c_k$, where 
\[c_k=\sum_{n_1+n_2=n}\sum_{k_1+k_2=k}a_{k_1}^{n_1-k_1}\otimes b_{k_2}^{n_2-k_2}.\]
Here we introduced superscripts to indicate the vertical degree, which unlike in the definition of $\mu$, it is not solely determined by the horizontal degree, since the total degree also varies. Distributibity allows us to rewrite $c$ as
\[c=\sum_k \sum_{n_1+n_2=n}\sum_{k_1+k_2=k}a_{k_1}\otimes b_{k_2}=\sum_{n_1+n_2=n}\sum_{k_1}\sum_{k_2}(a_{k_1}\otimes b_{k_2})=\sum_{n_1+n_2=n}\left(\sum_{k_1}a_{k_1}\right)\otimes\left(\sum_{k_2}b_{k_2}\right).\]
Therefore, $\mu^{-1}$ can be defined as
\[\mu^{-1}(c)=\sum_{n_1+n_2=n}\left(\sum_{k_1}(-1)^{k_1n_2}a_{k_1}\right)\otimes\left(\sum_{k_2}b_{k_2}\right).\]

The general case follows inductively.
\end{proof}

\subsection{Enriched categories and enriched totalization}
\subsubsection{Monoidal categories over  a base}
%NOT SURE IF I'M GOING TO NEED SO MUCH GENERAL BACKGROUND (CERTAINLY DO NOT INCLUDE PROOFS) I WOULD LIKE TO INCLUDE THE PROOF OF THE INVERSE INDUCING THE INVERSE TRANSFORMATION


%I MAY NOT  NEED MUCH (BUT I DO MENTION ENRICHED CATEGORIES, I COULD TRY TO ONLY SHOW RESULTS FOR PARTICULAR CATEGORIES) 

%WHAT I REALLY NEED IS THEE DEFINITION OF ENRICHED MU AND I CAN DEFINE IT EXPLICITELY IN THE LEMMA THAT IT IS USED (OR WITH AN EXTRA LEMMA IN WHICH I ALSO SHOW THE INVERSE) USING LEMMA 4.35

We collect some notions and results about enriched categories from \cite{riehl} and \cite[\S 4.2]{whitehouse} that we will need as a categorical setting for our results on derived $A_\infty$-algebras.

\begin{defin}
Let $(\VV ,⊗, 1)$ be a symmetric monoidal category and let $(\CC,⊗, 1)$ be a monoidal category. We say that $\CC$ is a monoidal category over $\VV$ if we have an external tensor product $∗ :\VV × \CC → \CC$ such that we have natural isomorphisms:
\begin{enumerate}[$\bullet$]
\item  $1 ∗ X \cong X$ for all $X ∈ \CC$,
\item $(C ⊗ D) ∗ X \cong C ∗ (D ∗ X)$ for all $C,D ∈ \VV$ and $X ∈ \CC$,
\item $C ∗ (X ⊗ Y ) \cong (C ∗ X) ⊗ Y \cong X ⊗ (C ∗ Y )$ for all $C ∈ \VV$ and $X, Y ∈ \CC$.
\end{enumerate}
\end{defin}
\begin{remark}\label{underline}
We will also assume that there is a bifunctor $\uC(−,−) : \CC^{op} × \CC → \VV$ such that we have natural
bijections
\[\Hom_\CC(C ∗ X, Y ) \cong \Hom_\VV (C,\uC(X, Y )).\]
Under this assumption we get a $\VV$-enriched category $\uC$ with the same objects as $\CC$ and with hom-objects given by $\uC (−,−)$. The unit
morphism $u_A : 1 → \uC (A,A)$ corresponds to the identity map in $\CC$ under the adjunction and the
composition morphism is given by the adjoint of the composite
\[(\uC (B,C) ⊗ \uC (A,B)) ∗ A
\cong \uC (B,C) ∗ (\uC (A,B) ∗ A)
\xrightarrow{id∗ev_{AB}}
\uC (B,C) ∗ B
\xrightarrow{ev_{BC}} C,\]
where $ev_{AB}$ is the adjoint of the identity $\uC (A,B) → \uC (A,B)$. Furthermore, $\uC$ is a monoidal $\VV$-enriched category, namely we have an
enriched functor
\[\underline{⊗} : \uC × \uC → \uC\]
where $\uC × \uC$ is the enriched category with objects $\mathrm{Ob}(\CC ) × \mathrm{Ob}(\CC )$ and hom-objects
\[\uC × \uC ((X, Y ), (W,Z)) \coloneqq \uC (X,W) ⊗ \uC (Y,Z).\]
In particular we get maps in $\VV$
\[\uC (X,W) ⊗ \uC (Y,Z) → \uC (X ⊗ Y,W ⊗ Z),\]
given by the adjoint of the composite
\[(\uC (X,W) ⊗ \uC (Y,Z)) ∗ (X ⊗ Y )\cong (\uC (X,W) ∗ X) ⊗ (\uC (Y,Z) ∗ Y )
\xrightarrow{ev_{XW}⊗ev_{Y Z}} W ⊗ Z\]
\end{remark}

\begin{defin}
Let $\CC$ and $\DD$ be monoidal categories over $\VV$. A \emph{lax functor over $\VV$} consists of a functor $F : \CC → \DD$ together with a natural transformation \[ν_F : − ∗_\DD F(−) ⇒ F(− ∗_\CC −)\]
which is associative and unital with respect to the monoidal structures over $\VV$ of $\CC$ and $\DD$. (See \cite[Proposition 10.1.5]{riehl} for explicit diagrams stating the coherence axioms.) If $ν_F$ is a natural isomorphism
we say $F$ is a \emph{functor over $\VV$}.
Let $F,G : \CC → \DD$ be lax functors over $\VV$. A \emph{natural transformation over $\VV$} is a natural transformation
$μ : F ⇒ G$ such that for any $C ∈ \VV$ and for any $X ∈ \CC$ we have
\[ν_G \circ (1 ∗_\DD μ_X) = μ_C∗_\CC X \circ ν_F .\]
A \emph{(lax) monoidal functor over $\VV$} is a triple $(F, \epsilon, μ)$, where $F : \CC → \DD$ is a lax functor over $\VV$,
$\epsilon : 1_\DD → F(1_\CC)$ is a morphism in $\DD$ and
\[μ : F(−) ⊗ F(−) ⇒ F(− ⊗ −)\]
is a natural transformation over $\VV$ satisfying the standard unit and associativity conditions. If $ν_F$
and $μ$ are natural isomorphisms then we say that $F$ is \emph{monoidal over $\VV$}. 
\end{defin}

\begin{propo}\label{enrichedtrans}
Let $F,G : \CC → \DD$ be lax functors over $\VV$. Then $F$ and $G$ extend to $\VV$-enriched
functors
\[\underline{F},\underline{G} : \uC → \uD\]
where $\uC$ and $\uD$ denote the $\VV$-enriched categories corresponding to $\CC$ and $\DD$ as described in \Cref{underline}. Moreover, any natural transformation $μ : F ⇒ G$ over $\VV$ also extends to a $\VV$-enriched natural
transformation
\[\underline{μ} : \underline{F} ⇒ \underline{G}.\]
In particular, if $F$ is (lax) monoidal over $\VV$, then $F$ is (lax) monoidal in the enriched sense.
\end{propo}
\begin{proof}
See \cite[Proposition 4.11]{whitehous}.
\end{proof}
\begin{lem}
Let $F,G:\CC\to\DD$ lax functors over $\VV$ and let $\mu : F\Rightarrow G$ a natural transformation over $\VV$. For every $X\in\CC$ and $Y\in\DD$ there is a map \[\uD(GX,Y)\to\uD(FX,Y)\] that is an isomorphism if $\mu$ is an isomorphism.
\end{lem}
\begin{proof}
By \Cref{enrichedtrans} there is a $\VV$-enriched natural transformation 
\[\underline{\mu}:\underline{F}\to\underline{G}\]
whose component \[\underline{\mu}_X:1\to\uD(FX,GX)\] is defined to be the adjoint of $\mu_X:FX\to GX$. The map $\uD(GX,Y)\to\uD(FX,Y)$ is defined as the composite

\begin{equation}\label{enrichedmap}
\uD(GX,Y)\cong\uD(GX,Y)\otimes 1\xrightarrow{1\otimes\umu_X}\uD(GX,Y)\otimes\uD(FX,GX)\xrightarrow{c}\uD(FX,Y)
\end{equation}
where $c$ is the composition map in the enriched setting. 

When $\mu$ is an isomorphism we may analogously define the following map

\[\uD(FX,Y)\cong\uD(FX,Y)\otimes 1\xrightarrow{1\otimes\umui_X}\uD(FX,Y)\otimes\uD(GX,FX)\xrightarrow{c}\uD(GX,Y).\]

We show that this map is the inverse of the map in \Cref{enrichedmap}.

\[
\adjustbox{scale=0.76,center}{%
\begin{tikzcd}[column sep = 0pt, row sep = 20pt]
{\uD(GX,Y) } \arrow[r, "\cong"] \arrow[rd, "(5)", phantom, bend left = 7]                 & {\uD(GX,Y)\otimes 1} \arrow[r, "1\otimes\umu_X"] \arrow[d, "1\otimes\alpha_X"] \arrow[rd, "(4)", phantom] & {\uD(GX,Y)\otimes\uD(FX,GX)} \arrow[r, "c"] \arrow[d, "\cong"]                                                                     & {\uD(FX,Y)} \arrow[ddd, "\cong"]                                               \\
                                                                                     & {\uD(GX,Y)\otimes\uD(GX,GX)} \arrow[lu, "c"] \arrow[lu]                                                  & {\uD(GX,Y)\otimes\uD(FX,GX)\otimes 1} \arrow[ld, "1\otimes 1\otimes \umui_X"] \arrow[rdd, "c\otimes 1"] \arrow[ru, "(1)"', phantom] &                                                                                \\
                                                                                     & {\uD(GX,Y)\otimes\uD(FX,GX)\otimes \uD(GX,FX)} \arrow[u, "1\otimes c"] \arrow[ld, "c\otimes 1"]          &                                                                                                                                    &                                                                                \\
{\uD(FX,Y)\otimes\uD(GX,FX)} \arrow[uuu, "c"] \arrow[ruu, "(3)"', phantom, bend left] & {}                                                                                                       &                                                                                                                                    & {\uD(FX,Y)\otimes 1} \arrow[lll, "1\otimes\umui_X"] \arrow[llu, "(2)"', phantom]
\end{tikzcd}
}
\]

%
%\[
%\adjustbox{scale=1,center}{%
%\begin{tikzcd}[column sep = 1pt]
%{\uD(GX,Y) } \arrow[r, "\cong"] \arrow[rd, "(5)", phantom, bend left = 7]                 & {\uD(GX,Y)\otimes 1} \arrow[r, "1\otimes\umu_X"] \arrow[d, "1\otimes\alpha_X"] \arrow[rd, "(4)", phantom] & {\uD(GX,Y)\otimes\uD(FX,GX)} \arrow[r, "c"] \arrow[d, "\cong"]                                                                     & {\uD(FX,Y)} \arrow[ddd, "\cong"]                                               \\
%                                                                                     & {\uD(GX,Y)\otimes\uD(GX,GX)} \arrow[lu, "c"] \arrow[lu]                                                  & {\uD(GX,Y)\otimes\uD(FX,GX)\otimes 1} \arrow[ld, "1\otimes 1\otimes \umui_X"] \arrow[rdd, "c\otimes 1"] \arrow[ru, "(1)"', phantom] &                                                                                \\
%                                                                                     & {\uD(GX,Y)\otimes\uD(FX,GX)\otimes \uD(GX,FX)} \arrow[u, "1\otimes c"] \arrow[ld, "c\otimes 1"]          &                                                                                                                                    &                                                                                \\
%{\uD(FX,Y)\otimes\uD(GX,FX)} \arrow[uuu, "c"] \arrow[ruu, "(3)"', phantom, bend left] & {}                                                                                                       &                                                                                                                                    & {\uD(FX,Y)\otimes 1} \arrow[lll, "1\otimes\umui_X"] \arrow[llu, "(2)"', phantom]
%\end{tikzcd}
%}
%\]

In the diagram, $\alpha_X$ is adjoint to $1_{GX}:GX\to GX$. Diagrams (1) and (2) clearly commute. Diagram (3) commutes by associativity of $c$. Diagram (4) commutes because $\umui_X$ and $\umu_X$ are adjoint to mutual inverses, so their composition results in the adjoint of the identity. Finally, diagram (5) commutes because we are composing with an identity map. 
\end{proof}

\begin{lem}\label{4.15}
The category $\fc$ is monoidal over $\vbc$. By restriction, $\fmod$ is monoidal over $\bgmod$.
\end{lem}
\begin{proof}
See \cite[Lemma 4.15]{whitehouse} for the proof and more details.
\end{proof}

\subsubsection{Enriched categories and totalization}

We define here some useful enriched categories and results from \cite[\S 4.3 and 4.4]{whitehouse}. Some of them had to be modified to adjust them to our conventions. 
\begin{defin}\label{weirdenrichment}
Let $A,B,C$ be bigraded modules. We denote by $\underline{\mathpzc{bgMod}_R}(A,B)$ the bigraded module given by
\[\underline{\mathpzc{bgMod}_R}(A,B)^v_u :=\prod_{j≥0}[A,B]^{v−j}_{u+j}\]
where $[A,B]$ is the inner hom-object of bigraded modules. More precisely, $g ∈ \underline{\mathpzc{bgMod}_R}(A,B)^v_u$ is given
by $g := (g_0, g_1, g_2, \dots )$, where $g_j : A → B$ is a map of bigraded modules of bidegree $(u + j, v − j)$.

Moreover, we define a composition morphism
\[c : \underline{\mathpzc{bgMod}_R}(B,C) ⊗ \underline{\mathpzc{bgMod}_R}(A,B) → \underline{\mathpzc{bgMod}_R}(A,C)\]
by
\[c(f, g)_m :=\sum_{i+j=m}(−1)^{i|g|}f_ig_j .\]
\end{defin}

\begin{defin}\label{delta2}
Let $(A, d^A_i), (B, d^B_i)$ be twisted complexes, $f ∈ \underline{\mathpzc{bgMod}_R}(A,B)^v_u$ and consider $d^A :=(d^A_i)_i ∈ \underline{\mathpzc{bgMod}_R}(A,A)^1_0$
and $d^B := (d^B_i)_i ∈ \underline{\mathpzc{bgMod}_R}(B,B)^1_0$. We define
\[δ(f) := c(d^B, f) − (−1)^{\langle f,d^A\rangle}c(f, d^A) ∈ \underline{\mathpzc{bgMod}_R}(A,B)^{v+1}_u\]
where $\langle f, d^A\rangle$ is the scalar product for the bidegrees and $c$ is the composition morphism described in \Cref{weirdenrichment} More precisely,
\[(δ(f))_m :=\sum_{i+j=m}(−1)^{i|f|}d^B_if_j − (−1)^{v+i}f_id^A_j.\]
\end{defin}

The following lemma justifies the above definition. For a proof see \cite[Lemma 4.18]{whitehouse}.

\begin{lem}
The following equations hold
\begin{align*}
&c(d^A, d^A) = 0,\\
&δ^2 = 0,\\
&δ(c(f, g)) = c(δ(f), g) + (−1)^v c(f, δ(g)),
\end{align*}
where the bidegree of $f$ is $(u, v)$. Furthermore, $f ∈ \ubgMod(A,B)$ is a map of twisted complexes if and
only if $δ(f) = 0$. In particular, $f$ is a morphism in $\tc$ if and only if the bidegree of $f$ is $(0, 0)$ and
$δ(f) = 0$. Moreover, for $f$, $g$ morphisms in $\tc$, we have that $c(f, g) = f\circ g$, where the latter denotes
composition in $\tc$.
\end{lem}

\begin{defin}
For $A,B$ twisted complexes, we define $\underline{t\mathcal{C}_R}(A,B)$ to be the vertical bicomplex
$\underline{t\mathcal{C}_R}(A,B) := (\underline{\mathpzc{bgMod}_R}(A,B), δ)$.
\end{defin}

\begin{defin}\label{ubgMod}
We denote by $\ubgMod$ the $\bgmod$-enriched category of bigraded modules given
by the following data.

\begin{enumerate}[(1)]
\item The objects of $\ubgMod$ are bigraded modules.
\item For $A,B$ bigraded modules the hom-object is the bigraded module $\ubgMod(A,B)$.
\item The composition morphism $c : \ubgMod(B,C) ⊗ \ubgMod(A,B) → \ubgMod(A,C)$ is given by \Cref{weirdenrichment}.
\item The unit morphism $R → \ubgMod(A,A)$ is given by the morphism of bigraded modules that
sends $1 ∈ R$ to $1_A : A → A$, the strict morphism given by the identity of $A$.
\end{enumerate}
\end{defin}

\begin{defin}\label{utC}
The $\vbc$-enriched category of twisted complexes $\utC$ is the enriched category given by the following data.
\begin{enumerate}[(1)]
\item The objects of $\utC$ are twisted complexes.
\item For $A,B$ twisted complexes the hom-object is the vertical bicomplex $\utC(A,B)$.
\item The composition morphism $c : \utC(B,C)⊗\utC(A,B) → \utC(A,C)$ is given by \Cref{weirdenrichment}.
\item The unit morphism $R → \utC(A,A)$ is given by the morphism of vertical bicomplexes sending
$1 ∈ R$ to $1_A : A → A$, the strict morphism of twisted complexes given by the identity of $A$.
\end{enumerate}
\end{defin}




The next tensor corresponds to $\underline{\otimes}$ in the categorical setting of \Cref{underline}.


\begin{lem}\label{tensorenriched}
The monoidal structure of $\utC$ is given by the following map of vertical bicomplexes.
\[\underline{⊗}: \utC(A,B) ⊗ \utC(A′,B′) → \utC(A ⊗ A′,B ⊗ B′)\]
\[(f, g) → (f\underline{⊗}g)_m :=\sum_{i+j=m}(−1)^{ij}f_i ⊗ g_j\]
The monoidal structure of $\ubgMod$ is given by the restriction of this map.
\end{lem}
\begin{proof}
See \cite[Lemma 4.27]{whitehouse}
\end{proof}



\begin{defin}\label{ufMod}
The $\bgmod$-enriched category of filtered modules $\ufMod$ is the enriched category given by the following data.
%I HAVE TO KEEP J+U SO THAT THE DEGREES MATCH IN ENRICH TOT (ALTERNATIVELY KEEP V-U BUT I DON'T LIKE THAT)
\begin{enumerate}[(1)]
\item The objects of $\ufMod$ are filtered modules.
\item For filtered modules $(K, F)$ and $(L, F)$, the bigraded module $\ufMod(K,L)$ is given by
\[\ufMod(K,L)^v_u :=\{f : K → L\mid f(F_qK^m) ⊂ F_{q+u}L^{m+u+v}, ∀m, q ∈ \Z\}.\]
\item The composition morphism is given by $c(f, g) = (−1)^{u|g|}fg$, where $f$ has bidegree $(u, v)$.
\item The unit morphism is given by the map $R → \ufMod(K,K)$ given by $1 → 1_K$.
\end{enumerate}
\end{defin}


\begin{defin}\label{fmoddifferential}
Let $(K, d^K, F)$ and $(L, d^L, F)$ be filtered complexes. We define $\ufC(K,L)$ to be the
vertical bicomplex whose underlying bigraded module is $\ufMod(K,L)$ with vertical differential
\[δ(f) := c(d^L, f) − (−1)^{\langle f,d^K\rangle}c(f, d^K) = d^Lf − (−1)^{v+u}fd^K = d^Lf − (−1)^{|f|}fd^K\]
for $f ∈ \ufMod(K,L)^v_u$, where $c$ is the composition map from \Cref{ufMod}.
\end{defin}


\begin{defin}\label{ufC}
The $\vbc$-enriched category of filtered complexes $\ufC$ is the enriched category given
by the following data.
\begin{enumerate}[(1)]
\item The objects of $\ufC$ are filtered complexes.
\item For $K,L$ filtered complexes the hom-object is the vertical bicomplex $\ufC(K,L)$.
\item The composition morphism is given as in $\ufMod$ in \Cref{ufMod}. 
\item The unit morphism is given by the map $R → \ufC(K,K)$ given by $1 → 1_K$.
We denote by $\usfC$ the full subcategory of $\ufC$ whose objects are split filtered complexes.

\end{enumerate}
\end{defin}

The enriched monoidal structure is given by the following lemma.
\begin{defin}\label{tensorenriched2}
The monoidal structure of $\ufC$ is given by the following map of vertical bicomplexes.
\[\underline{⊗}: \ufC(K,L) ⊗ \ufC(K′,L′) → \ufC(K ⊗ K′,L ⊗ L′),\]
\[(f, g) → f\underline{⊗}g := (−1)^{u|g|}f ⊗ g\]
where $f$ has bidegree $(u, v)$.
\end{defin}
\begin{proof}
See \cite[Lemma 4.36]{whitehouse}.
\end{proof}


\begin{lem}\label{adjunction}
Let $A$ be a vertical bicomplex that is horizontally bounded on the right and let $K$ and $L$ be filtered complexes. There is a natural bijection
\[\Hom_{\fc}(\Tot(A)\otimes K,L)\cong \Hom_{\vbc}(A,\ufC(K,L))\]
given by
\[f\mapsto \tilde{f}: a\mapsto (k\mapsto f(a\otimes k)).\]
\end{lem}
\begin{proof}
The proof is included in the proof of \cite[Lemma 4.35]{whitehouse}.
\end{proof}
We now define an enriched version of the totalization functor. %CHECK TOT, OR JUST OMIT IT BECAUSE I DON'T NEED IT
\begin{defin}\label{enrichedtot}
Let $A,B$ be bigraded modules and $f ∈ \ubgMod (A,B)^v_u$ we define

\[\Tot(f) ∈ \ufMod(\Tot(A),\Tot(B))^v_u\]
to be given on any $a ∈ \Tot(A)^n$ by
\[(\Tot(f)(a)))_{j+u} :=
\sum_{m≥0}(−1)^{(m+u)n}f_m(a_{j-m}) ∈ B^{n-j+v}_{j+u} ⊂ \Tot(B)^{n+u+v}.\]
Let $K = \Tot(A)$, $L = \Tot(B)$ and $g ∈ \ufMod(K,L)^v_u$ we define
\[f := \Tot^{−1}(g) ∈ \ubgMod(A,B)^v_u\]
to be $f := (f_0, f_1,\dots)$ where $f_i$ is given on each $A^{m+j}_j$ by the composite
\begin{align*}
f_i : A^{m-j}_j \hookrightarrow\prod_{k\geq j}A^{m-k}_k = F_j(\Tot(A)^m)\xrightarrow{g}&F_{j+u}(\Tot(B)^{m+u+v})\\
&=\prod_{l\geq j+u}B^{m+u+v-l}_l\xrightarrow{×(−1)^{(i+u)m}} B^{m-j+v−i}_{j+u+i} ,
\end{align*}
where the last map is a projection and multiplication with the indicated sign.

%I HAVE TO KEEP J+U SO THAT THE DEGREES MATCH  (ALTERNATIVELY KEEP V-U BUT I DON'T LIKE THAT)
\end{defin} 

\begin{thm}\label{4.39}
Let $A$, $B$ be twisted complexes. The assignments $\mathfrak{Tot}(A) := \Tot(A)$ and
\begin{align*}
\mathfrak{Tot}_{A,B} : \utC(A,B)& → \ufC(\Tot(A),\Tot(B))\\
f &→ \Tot(f)
\end{align*}
define a $\vbc$-enriched functor $\mathfrak{Tot} : \utC → \ufC$ which restricts to an isomorphism onto its image. Furthermore, this functor restricts to a $\bgmod$-enriched functor \[\mathfrak{Tot} : \ubgMod → \ufMod\]
 which also restricts to an isomorphism onto its image.
\end{thm}
\begin{proof}
See \cite[Theorem 4.39]{whitehouse}.
\end{proof}

\begin{propo}\label{4.40}
The enriched functors
\[\mathfrak{Tot} : \ubgMod  → \ufMod ,\hspace{1cm} \mathfrak{Tot} : \utC → \ufC\]
are lax symmetric monoidal in the enriched sense and when restricted to the bounded case they are strong symmetric monoidal in the enriched sense.
\end{propo}
\begin{proof}
See \cite[Proposition 4.40]{whitehouse}.
\end{proof}

We now define an enriched endomorphism operad. %I THINK I ONLY NEED THIS DEFINITION AND NOT THE PREVIOUS OR THE NEXT LEMMA
\begin{defin}
Let $\underline{\mathscr{C}}$ be a monoidal $\mathscr{V}$-enriched category and $A$ an object of $\uC$. We define $\uEnd_A$
to be the collection in $\mathscr{V}$ given by
\[\uEnd_A(n) \coloneqq \uC (A^{⊗n},A) \text{ for }n ≥ 1.\]
\end{defin}
%AT SOME POINT DO SOME MENTION TO THE DIFFERENT CASES THAT I AM GOING TO USE BEECAUSE THE NOTATION IS GOING TO BE THE SSAME

\begin{propo}\label{morphism}
Let $\CC$ and $\DD$ be monoidal categories over $\VV$ . Let
$F : \CC → \DD$ be a lax monoidal functor over $\VV$ . Then for any $X ∈ \CC$ there is an operad morphism
\[\uEnd_X→\uEnd_{F(X)}.\]

\end{propo}
\begin{proof}
The proof is in \cite[Proposition 4.46]{whitehouse}. 
\end{proof}

\begin{lem}\label{inverse}
Let $A$ be a twisted complex. Consider $\uEnd_A(n)=\utC(A^{\otimes n},A)$ and $\uEnd_{\Tot(A)}(n)=\ufC(\Tot(A)^{\otimes n},\Tot(A))$. There is a morphism of operads
\[\uEnd_A →\uEnd_{\Tot(A)},\]
which is an isomorphism of operads if $A$ is bounded. The same holds true if $A$ is just a bigraded module. In that case, we use the enriched operads $\uEnd_A(n)=\ubgMod(A^{\otimes n},A)$ and $\uEnd_{\Tot(A)}(n)=\ufMod(\Tot(A)^{\otimes n},\Tot(A))$.
\end{lem}
\begin{proof}
The proof of in the case of a $A$ beig a twisted complex can be found in \cite[Lemma 4.54]{whitehouse}. For the bigraded module case, we are going to do it analogously. First, by \Cref{4.39} we know that $\mathfrak{Tot}:\ubgMod\to\ufMod$ is $\bgmod$-enriched. In fact, by \Cref{4.40} it is lax monoidal in the enriched sense. In addition, both $\bgmod$ and $\fmod$ are monoidal over $\bgmod$. In the case of $\bgmod$ it is in the obvious way and for $\fmod$ is given by \Cref{4.15}. With all of this we may apply \Cref{morphism} to $\mathfrak{Tot}:\ubgMod\to\ufMod$ to obtain the desired map
\[\uEnd_A →\uEnd_{\Tot(A)}.\]
 The fact that it is an isomorphism in the bounded is analogous to the twisted complex case. 
\end{proof}

We are going to construct the inverse in the bounded case explicitly followig \Cref{enrichedmap} (the construction for the direct map is analogue but here we just need the inverse). We do it for a twisted complex $A$, but it is done similarly for a bigraded module.

\begin{lem}\label{composition}
In the conditions of \Cref{inverse} for the bounded case, the inverse is given by the map
\begin{align*}
\uEnd_{\Tot(A)}&\to\uEnd_A\\
f & \mapsto \Tot^{-1}(f\circ \mu^{-1}).
\end{align*}
\end{lem}
\begin{proof}
The inverse is given by the composite
\[\uEnd_{\Tot(A)}(n)=\ufC(\Tot(A)^{\otimes n},\Tot(A))\to \ufC(\Tot(A^{\otimes n}),\Tot(A))\to\utC(A^{\otimes n},A)=\uEnd_A(n) \]

The second map is given by $\mathfrak{Tot}^{-1}$ defined in \Cref{enrichedtot}. To describe the first map, let $R$ be concentrated in bidegree $(0,0)$ with trivial vertical differential. Then the first map is given by the following composite
\begin{align*}
\ufC(\Tot(A)^{\otimes n},\Tot(A))\cong R\otimes\ufC(\Tot(A)^{\otimes n},\Tot(A))\\
\xrightarrow{\underline{\mu}^{-1}\otimes 1}\ufC(\Tot(A^{\otimes n}),\Tot(A)^{\otimes n})\otimes\ufC(\Tot(A)^{\otimes n},\Tot(A))\\
\xrightarrow{c}\ufC(\Tot(A^{\otimes n}),\Tot(A)), 
\end{align*}
where $c$ is the composition in $\ufC$, which is defined in \Cref{ufMod}. The map $\underline{\mu}^{-1}$ is the adjoint of $\mu^{-1}$ under the bijection from \Cref{adjunction}. Explicitly,
\begin{align*}
\underline{\mu}^{-1}:R &\to \ufC(\Tot(A^{\otimes n}),\Tot(A)^{\otimes n})\\
1 &\mapsto (a\mapsto \mu^{-1}(a)).
\end{align*}
Putting all together we get the map 
\begin{align*}
\uEnd_{\Tot(A)}&\to\uEnd_A\\
f & \mapsto \Tot^{-1}(c(f, \mu^{-1})).
\end{align*}
Since the total degree of $\mu^{-1}$ is 0, composition reduces to $c(f,\mu^{-1})=f\circ \mu^{-1}$ and we get the desired map.
\end{proof}





%ONLY MENTION 4.47 WHEN YOU REACH THE PROOF OF THE LONG ISO



\section{Operadic totalization and vertical operadic suspension}\label{operadic}
\subsection{Operadic totalization}
%I THINK I NEED  (N,Z)-BRIGRADED MODULES TO MAKE SURE THAT HORIZONTAL DEGREE IS NON-NEGATIVE WHEN DEFINING M AS AN ELEMENT OF TSO

%We are going to apply the totalization with compact support functor of WHITEHOUSE 4.13 to operads and we are going to simply call it \emph{totalization (functor)} and denote it as $T$. WHITEHOUSE 3.11 the functor $T$ with domain the category of bigraded modules and bidegree $(0,0)$ morphisms is lax monoidal, so applying LAX MONOIDAL PRESERVES OPERADS we can conclude that it takes operads of bigraded modules to operads of graded modules.  

We are going to apply the totalization  functor defined in \Cref{total} to operads. By \Cref{monoidal} and the fact that the image of an operad under a lax monoidal functor is also an operad, this will define a functor from operads in brigraded modules (resp. twisted complexes) to operads in graded modules (resp. chain complexes). %and we are going to simply call it \emph{totalization (functor)} and denote it as $\Tot$. Tthe functor $\Tot$ with domain the category $\bgmod$ (TWISTED COMPLEXES?) and bidegree $(0,0)$ morphisms (TWISTED COMPLEX MAPS?) is lax monoidal REFERENCE TO THEOREM, so applying LAX MONOIDAL PRESERVES OPERADS we can conclude that it takes operads of bigraded modules to operads of graded modules. 

Therefore, let $\OO$ be either a bigraded operad, i.e. an operad in te category of bigraded $R$-modules or an operad in twisted complexes. We define $\Tot(\OO)$ as the operad of graded $R$-modules (or chain complexes) for which \[\Tot(\OO(n))^d=\bigoplus_{i<0}\OO(n)^{d-i}_i\oplus\prod_{i\geq 0} \OO(n)^{d-i}_i\] is the image of $\OO(n)$ under the totalization functor and the insertion maps are given by the composition  %THE SECOND IF I WANT TO ORDER THEM BY HORIZONTAL DEGREE AND WRITE SUMS LIKE WHITEHOUSE and comes equipped with insertion maps \[a\bar{\circ}_rb=(-1)^{l(i+k)} a\circ_r b\]
\begin{equation}\label{insertion}
\Tot(\OO(n))\otimes \Tot(\OO(m))\xrightarrow{\mu} \Tot(\OO(n)\otimes \OO(m)) \xrightarrow{\Tot(\circ_r)} \Tot(\OO(n+m-1)),
\end{equation}
that is explicitly 
\[(a\bar{\circ}_rb)_k=\sum_{k_1+k_2=k} (-1)^{k_1n_2} a_{k_1}\circ_r b_{k_2}\]

for $a=(a_i)_i\in \Tot(\OO(n))^{n_1}$ and $b=(b_j)_j\in \Tot(\OO(m))^{n_2}$.

More generally, operadic composition $\bar{\gamma}$ is defined by the composite
\begin{equation*}
\Tot(\OO(N))\otimes \Tot(\OO(a_1))\otimes\cdots\otimes \Tot(\OO(a_N))\xrightarrow{\mu} \Tot(\OO(N)\otimes \OO(a_1)\otimes\cdots\otimes \OO(a_N)) \xrightarrow{\Tot(\gamma)} \Tot(\OO(\sum a_i)),
\end{equation*}

This map can be computed explicitly by iteration of the insertion $\bar{\circ}$, giving the following.  %For simplicity, we abuse of notation by omitting sums

\begin{lem}\label{totcomp}
The operadic composition $\bar{\gamma}$ on $\Tot(\OO)$ is given by
\begin{equation*}%\label{totcomp}
\bar{\gamma}(x;x^1,\dots, x^N)_k=\sum_{k_0+k_1+\cdots+k_N=k}(-1)^{\varepsilon}\gamma(x_{k_0};x^1_{k_1},\dots, x^N_{k_N})
\end{equation*}
for $x=(x_k)_k\in\Tot(\OO(N))^{n_0}$ and $x^i=(x^i_k)_k\in\Tot(\OO(a_i))^{n_i}$, where 
\begin{equation}
\varepsilon=\sum_{j=1}^m n_j\sum_{i=0}^{j-1}k_i
\end{equation}
and $\gamma$ is the operadic composition on $\OO$.
\end{lem}
Notice that the sign is precisely the same appearing in \Cref{mu}.
%MAYBE I SHOULD WRITE THINGS LIKE SUMS BUT I THINK IT MAKES SENSE TO WRITE IT LIKE THIS BECAUSE I KNOW THIS COMES FROM A BIGRADED MODULE
%where $a\in\OO(n)^k_i$, $b\in\OO(m)^j_l$ and $\circ_r$ is the insertion map in $\OO$.

%It can be checked that this is indeed an operad of graded vector spaces I 


%COMPOSITION OF ARBITRARY BIGRADING IS PRESERVED BY TOT SINCE ALL SIGNS INVOLVED ARE HORIZONTAL DEGREE SO IT IS ANALOGUE TO WHITEHOUSE (WRITE DOWN THE CALCULATIONS IF NEEDED)

\subsection{Vertical operadic suspension}
On an bigraded operad we can use operadic suspension on the vertical degree with analogue results to those of the graded case. %MAYBE SPECIFY SOME OF THEM

%Everything should be valid for R-modules (char not 2, as in fields). The sign representation would have to be a free R-module of rank 1

 %for a commutative (at least with 1\neq 0) ring the rank is well defined, in general it is not

%Therefore, in analogy to the single graded cases, let $sig_n$ be the sign representation of the symmetric group on $n$ symbols concentrated in bidegree $(0,0)$. This is a free $R$-module of rank one that comes with a natural action of the symmetric group $S_n$ that multiplies each element by the sign of each given permutation. %I MIGHT LEAVE OUT THE SYMMETRIC GROUP  ACTION UNLESS I FIND OUT HOW TO MODIFY IT IN TOTALIZATION, I SHOULD THINK ABOUT IT

%We define $\Lambda(n)=S^{n-1}sign_n$, where  $S$ is a vertical shift of degree so that $\Lambda(n)$ is concentrated on bidegree  $(0,n-1)$.
We define $\Lambda(n)=S^{n-1}R$, where  $S$ is a vertical shift of degree so that $\Lambda(n)$ is the underlying ring $R$ concentrated in bidegree  $(0,n-1)$. As in the single graded case, we express the basis element of $\Lambda(n)$ as $e^n=e_1\land\cdots\land e_n$.

The operad structure on the bigraded $\Lambda=\{\Lambda(n)\}_{n\geq 0}$ is the same as in the graded case, namely

\[
\begin{tikzcd}
\Lambda(n)\otimes\Lambda(m) \arrow[r, "\circ_{r+1}"] & \Lambda(n+m-1)\\
(e_1\land\cdots\land e_n)\otimes(e_1\land\cdots\land e_m)\arrow[r, mapsto] & (-1)^{(n-r-1)(m-1)}e_1\land\cdots\land e_{n+m-1}.
\end{tikzcd}
\]



%In a similar way we can define $\Lambda^-(n)=S^{1-n}sig_n$, with the same insertion maps.
In a similar way we can define $\Lambda^-(n)=S^{1-n}R$, with the same insertion maps.
%The sign might arise naturally from the permutation action. If I have the wedge of n wedge the wedge of m-1 (because the final result must be n+m-1 in total), I would permute the last m-1 until the reach the i-th position via transpositions, each transpotision produces a minus sign. Or simply considering the lat m as a single element of degree m-1 being permuted in the wedge
\begin{definition}
Let $\mathcal{O}$ be a bigraded linear operad, i.e. an operad on the category of bigraded $R$-modules. The \emph{vertical operadic suspension} $\mathfrak{s}\OO$ of $\mathcal{O}$ is given arity-wise by the Hadamard product of the operads $\OO$ and $\Lambda$, in other words, $\mathfrak{s}\OO(n)=(\mathcal{O}\otimes\Lambda)(n)=\mathcal{O}(n)\otimes\Lambda(n)$ with diagonal composition and symmetric group action. Similarly, we define the \emph{vertical operadic desuspension} $\mathfrak{s}^{-1}\OO(n)=\mathcal{O}(n)\otimes\Lambda^-(n)$. %POSSIBLY EXCLUDE SYMMETRIC GROUP ACTION, ALTHOUGH IT MAKES SENSE ON ITS OWN
\end{definition}

%CONSIDER CHANGING THE NOTATION FOR BIDEGREE TO BE CONSISTENT (I THINK THE BEST OPTION IS USING D FOR TOTAL DEGREES, IS A LOT TO CHANGE BUT OTHERWISE I WILL HAVE TO USE FUNNY NOTATION FOR ARITIES), SHOULD I USE TOTAL-HORZNTAL FOR VERTICAL DEGREE HERE? IT FEELS INNECESSARY EVEN THOUGH IT WOULD BE CONSISTENT

We may identify the elements of $\mathcal{O}$ with the elements the elements of $\mathfrak{s}\OO$. For $a\in\OO(n)$ of bidegree $(k,d-k)$, its ``natural'' bidegree in $\s\OO$ is $(k,d+n-k-1)$. To distinguish both degrees we call $(k,d-k)$ the \emph{internal bidegree} of $a$, since this is the degree that $a$ inherits from the grading of $\OO$. If we write $\circ_{r+1}$ for the operadic insertion on $\OO$ and $\tilde{\circ}_{r+1}$ for the operadic insertion on $\mathfrak{s}\OO$, we may find a relation between the two insertion maps in the following way. Let $a\in\OO(n)^{d-k}_k$ and $b\in\OO(m)^{q-l}_l$, and let us compute $a\tilde{\circ}_{r+1} b$.

\begin{align*}
\mathfrak{s}\OO(n)\otimes\mathfrak{s}\OO(m)&=(\OO(n)\otimes\Lambda(n))\otimes (\OO(m)\otimes\Lambda(m))\cong (\OO(n)\otimes \OO(m))\otimes (\Lambda(n)\otimes \Lambda(m))\\
&\xrightarrow{\circ_{r+1}\otimes\circ_{r+1}} \OO(m+n-1)\otimes \Lambda(n+m-1)=\mathfrak{s}\OO(n+m-1).
\end{align*}

The symmetric monoidal structure produces the sign $(-1)^{(n-1)(q-l)}$ in the isomorphism $\Lambda(n)\otimes \OO(m)\cong\OO(m)\otimes\Lambda(n)$, and the operadic structure of $\Lambda$ produces the sign $$(-1)^{(n-1)(m-1)+r(m-1)},$$ so 

\begin{equation}\label{sign}
a\tilde{\circ}_{r+1}b=(-1)^{(n-1)(q-l)+(n-1)(m-1)+r(m-1)}a\circ_{r+1} b.
\end{equation}
As can be seen, this is the same sign as the single graded operadic suspension but with vertical degree. In particular, this operation leads to the Lie bracket from \cite{RW}, which implies that $m=\sum_{i,j}m_{ij}$ is a derived $A_\infty$-multiplication if and only if
\begin{equation}\label{sharp}
\sum_{i+j=u}\sum_{l,k}(-1)^jb_1(b_1(m_{jl};m_{ik});x)=0
\end{equation}
for the brace induced by $\tilde{\circ}$ (see \Cref{braces} to recall the definition of a brace).

We of course have the following theorem with similar proof to the graded case, where all the suspensions are vertical.
\begin{thm}\label{markl}
Given a bigraded $R$-module $A$, there is an isomorphism of operads $\End_{ A}\cong \mathfrak{s}\End_{SA}$, where $\End_A$ is the endomorphism operad of $A$.\qed
\end{thm}

\subsection{Vertical suspension and totalization} 

Now we are going to combine vertical operadic suspension and totalization. More precisely, the totalized vertical suspension a bigraded operad $\OO$ is the graded operad $\Tot(\s\OO)$. 

%TsOO(n)^{n1} SOO(n)^{n1-k1}_k1  OO(n)^{n1-k1-n+1}_k1

This operad has an insertion map explicitly given by
\begin{equation}\label{star}
(a\star_{r+1} b)_k=\sum_{k_1+k_2=k}(-1)^{(n-1)(n_2-k_2-m+1)+(n-1)(m-1)+r(m-1)+k_1n_2}a_{k_1}\circ_{r+1}b_{k_2}
\end{equation}
for $a=(a_i)_i\in \Tot(\s\OO(n))^{n_1}$ and $b=(b_j)_j\in \Tot(\s\OO(m))^{n_2}$. As usual, denote \[a\star b=\sum_{r=0}^{m-1}a\star_{r+1}b.\]

This star operation is precisely the star operation from \cite[\S 5.1]{LRW}, i.e the convolution operation on $\Hom((dAs)^{ \rotatebox[origin=c]{180}{¡}}, \End_A)$ (see \cite{LRW} for details). 

%(WILL HAVE TO TALK ABOUT TOTALIZING COLLECTIONS)

Before continuing, let us show a lemma that allows us to work only with the single graded operadic suspension if needed.
\begin{propo}\label{extrasign}
For a bigraded operad $\OO$ we have an isomorphism $\Tot(\s\OO)\cong \s \Tot(\OO)$, where the suspension on the left hand side is the bigraded version and on the right hand side is the single graded version. 
\end{propo}
\begin{proof}
 Note that we may identify each element $a=(a_k\otimes e^n )_k\in\Tot(\s\OO(n))$ with the element $a=(a_k)_k\otimes e^n\in\s\Tot(\OO(n))$. Thus, for an element $(a_k)_k\in \Tot(\s\OO(n))$ the isomorphism is given by
\begin{align*}
f:\Tot(\s\OO(n))&\cong \s \Tot(\OO(n))\\
(a_k)_k&\mapsto ((-1)^{kn}a_k)_k
\end{align*}
Cleary this map is bijective so we just need to check that it commutes with insertions. Recall from \Cref{star} that the insertion on $\Tot(\s\OO)$ is given by
\begin{equation*}
(a\star_{r+1} b)_k=\sum_{k_1+k_2=k}(-1)^{(n-1)(n_2-k_2-n+1)+(n-1)(m-1)+r(m-1)+k_1n_2}a_{k_1}\circ_{r+1}b_{k_2}
\end{equation*}
for $a=(a_i)_i\in \Tot(\s\OO(n))^{n_1}$ and $b=(b_j)_j\in \Tot(\s\OO(m))^{n_2}$. Similarly, we may compute the insertion on $\s\Tot(\OO)$ by combining the sign produced first by $\Tot$ and then by $\s$. This results in  the following insertion map 
\begin{equation*}
(a\star_{r+1}' b)_k=\sum_{k_1+k_2=k}(-1)^{(n-1)(n_2-n+1)+(n-1)(m-1)+r(m-1)+k_1(n_2-m+1)}a_{k_1}\circ_{r+1}b_{k_2}
\end{equation*}
for $a=(a_i)_i\in \s\Tot(\OO(n))^{n_1}$ and $b=(b_j)_j\in \s\Tot(\OO(m))^{n_2}$. Now let us show that $f(a\star b)=f(a)\star f(b)$. We have
\begin{align*}
f((a\star_{r+1} b))_k&=\sum_{k_1+k_2=k}(-1)^{k(n+m-1)+(n-1)(n_2-k_2-n+1)+(n-1)(m-1)+r(m-1)+k_1n_2}a_{k_1}\circ_{r+1}b_{k_2}\\
&=\sum_{k_1+k_2=k}(-1)^{(n-1)(n_2-n+1)+(n-1)(m-1)+r(m-1)+k_1(n_2-m+1)}f(a_{k_1})\circ_{r+1}f(b_{k_2})\\
&=(f(a)\star f(b))_k
\end{align*}
%I skipped one calculation but it is just that
as desired.
\end{proof}


\begin{remark}\label{othermu}


As we mentioned in \Cref{heuristic}, there exist other possible ways of totalizing by varying the natural transformation $\mu$. For instance, we can choose the totalization functor $\Tot'$ which is the same as $\Tot$ but with a natural transformation $\mu'$ defined in such a way that the insertion on $\Tot'(\OO)$ is defined by \[(a\hat{\circ}b)_k=\sum_{k_1+k_2=k}(-1)^{k_2n_1}a_{k_1}\circ b_{k_2}.\] 

This is also a valid approach for our purposes and there is simply a sign difference, but we have chosen our convention to be consistent with other conventions, like the derived $A_\infty$-equation. However, let us mention some differences and relations between $\Tot$ and $\Tot'$ . %There are of course many other possible conventions MAYBE NOT MENTION THE EXACT DIFFRENES, JUST STOP HERE BECAUSE THERE ARE MORE CONVENTIONS AND I DON'T THINK I WILL TREAT ALL OF THEM IN DETAIL, BUT THE DIFFERENCE BETWEEN IDENTITY AND ISOMORPHISM MIGHT BE WORTH MENTION

First of all, $\Tot$ and $\Tot'$ are isomorphic. The isomorphism $f:\Tot(\OO)\cong \Tot'\OO$ given by $f((a_k))=((-1)^{k(n_1-k)}a_k)$. %MAYBE ALL OF THEM ARE ISOMORPHIC, THINK OF THIS USING JUST THAT A TOTALIZATION IS A LAX SYMMETRIC MONOIDAL FUNCTOR FROM BIGRADED MODULES TO GRADED MODULES %, MAYBE ALSO USE THAT THE UNDERLYING MODULE IS ALWAYS THE SAME SO THAT THE ONLY MODIFICATION IS THE INSERTION

It can also be verified that $\Tot'(\s\OO)=\s \Tot'(\OO)$. With the original totalization we have a non identity isomorphism given by \Cref{extrasign}. Similar relations can be found among the other alternatives mentioned in \Cref{heuristic}. %We will use this isomorphism later.

%POSSIBLY FILL IN SOME DETAILS OF THESE ISOMORPHISMS


\end{remark}
%AT THE END  OF THE PROOF OF WHITEHOUSE 4.47 WHERE AINFTY ON TWISTED COMPLEX IS EQUIVALENT TO DERIVED-AINFTY THIS ISOMORPHISM IS  USED SINCE APLIED TO MIJ  THE EXPONENT IS PRECISELY IJ

%ALSO IN THAT THEOREM THE DIFERENTIALS OF THE TWISTED COMPLEX CORRESPOND PRECISELY TO MI1 SIMILAR TO MY MI1 BEING LIKE DIFFERENTIALS (I SHOULD CHECK IF THEY SATISFY THE TWISTED COMPLEX RELATION)




\section{Derived $A_\infty$-algebras and filtered $A_\infty$-algebras}

In this section we recall some definitions about derived $A_\infty$-algebras and present some ways of interpreting them in terms of operads and collections.

%DEFINE THE OPERAD,BOTH FREE WITH A DIFFERENTIAL AND ALSO WITH ALL THE DIFFERENTIALS INSIDE
  \begin{defin}
  Using the notation in \cite{RW}, a \emph{derived $A_\infty$-algebra} on a $(\Z,\Z)$-bigraded $R$-module $A$ consist of a family of $R$-linear maps 
\[m_{ij}:A^{\otimes j}\to A\]
of bidegree $(i,2-(i+j))$ for each $j\geq 1$, $i\geq 0$, satisfying the equation
\[\underset{j=r+1+t}{\sum_{u=i+p, v=j+q-1}}(-1)^{rq+t+pj}m_{ij}(1^{\otimes r}\otimes m_{pq}\otimes 1^{\otimes t})=0\]
for all $u\geq 0$ and $v\geq 1$. 
\end{defin}

According to the above definition, there are two equivalent ways of defining the operad of derived $A_\infty$-algebras $d\calA_\infty$ depending on the underlying category. We give the two of them here as we are going to use both.

\begin{defin}
The operad $d\calA_\infty$ in $\bgmod$ is the operad generated by $\{m_{ij}\}_{i\geq 0,j\geq 1}$ subject to the derived $A_\infty$-relation

\[\underset{j=r+1+t}{\sum_{u=i+p, v=j+q-1}}(-1)^{rq+t+pj}\gamma(m_{ij};1^{ r}, m_{pq}, 1^{t})=0\]
for all $u\geq 0$ and $v\geq 1$. 

The operad $d\calA_\infty$ in $\vbc$ is the quasi-free operad generated by $\{m_{ij}\}_{(i,j)\neq (0,1)}$ with vertical differential given by
\[\partial_\infty(m_{uv})=-\underset{j=r+1+t, (i,j)\neq (0,1)\neq (p,q)}{\sum_{u=i+p, v=j+q-1}}(-1)^{rq+t+pj}\gamma(m_{ij};1^{ r}, m_{pq}, 1^{t}).\]
\end{defin}
%NOT EXACTLY THE SAME AS LWR, BUT THE ONE THAT SARAH USES

\begin{defin}
Let $A$ and $B$ be derived $A_\infty$-algebras with respective structure maps $m^A$ and $m^B$. An $\infty$-morphism of derived $A_\infty$ algebras $f:A\to B$ is a family of maps $f_{st}:A^{\otimes t}\to B$ of bidegree $(s,1-s-t)$ satifying
\begin{equation}\label{dinftymaps}
\underset{j=r+1+t}{\sum_{u=i+p, v=j+q-1}}(-1)^{rq+t+pj}f_{ij}(1^{\otimes r}\otimes m_{pq}^A\otimes 1^{\otimes s})=\underset{v=q_1+\cdots +q_j}{\sum_{u=i+p_1+\cdots +p_j}}(-1)^{\epsilon} m^B_{ij}(f_{p_1 q_1}\otimes\cdots\otimes f_{p_j q_j})
\end{equation}
for all $u\geq 0$ and $v\geq 1$, where
\[\epsilon = u + \sum_{1\leq w < l \leq j} q_w(1-p_l-q_l)  + \sum_{w=1}^j p_w(j-w).\]
%I am confindent that this is the same as in RW, it is a matter of grouping differently (taking in to account how many times things are added up) and sometimes change w by j-w. But maybe I should write it down.
\end{defin}
\begin{ex}\
\begin{enumerate}
\item One can check that, on any derived $A_\infty$-algebra $A$, the maps $d_i=(-1)^{i}m_{i1}$ define a twisted complex structure. This leads to the possibility of defining a derived $A_\infty$-algebra as a twisted complex with some extra structure (see \Cref{equivalent}).
\item An $A_\infty$-algebra is the same as a derived $A_\infty$-algebra such that $m_{ij}=0$ for all $i>0$.
\end{enumerate}
\end{ex}

For bigraded modules $A$ and $B$ we may define the collection $\End^A_B=\{\Hom(A^{\otimes n}, B)\}_{n\geq 1}$ of bigraded modules,  in analogy with the single graded case. Recall that this collection has a left module structure over $\End_B$
\[\End_B\circ \End^A_B\to \End^A_B\]
given by composition of maps. Similarly, given a bigraded module $C$ we can define composition maps
\[\End^B_C\circ \End^A_B\to \End^A_C.\]
The collection $\End^A_B$ also has an infinitesimal right module structure over $\End_A$
\[\End^A_B\circ_{(1)}\End_A\to \End^A_B\]
given by insertion of maps.

Similarly to the single graded case, we may describe derived $\infty$-morphisms in terms of the above operations.

\begin{lem}\label{dinfinitymorphism}
A derived $\infty$-morphism of $A_\infty$-algebras $A\to B$ with respective structure maps $m^A$ and $m^B$ is equivalent to an element $f\in\Tot(\s\End^A_B)$ of degree 0 concentrated in positive arity such that \[\rho(f\circ_{(1)}m^A)=\lambda(m^B\circ f),\] 

where \[\lambda:\Tot(\mathfrak{s}\End_B)\circ \Tot(\mathfrak{s}\End^A_B)\to \Tot(\mathfrak{s}\End^A_B)\] is induced by the left module struncture on $\End^A_B$ and \[\rho:\Tot(\mathfrak{s}\End_B)\circ_{(1)}\Tot(\mathfrak{s}\End^A_B)\to \Tot(\mathfrak{s}\End^A_B)\] is induced by the right infinitesimal module structure on $\End^A_B$. 

In addition, the composition of $\infty$-morphisms is given by the natural composition \[\Tot(\s\End^B_C)\circ \Tot(\s\End^A_B)\to \Tot(\s\End^A_C).\]
\end{lem}
\begin{proof}
Since $f_j=(f_{ij})_i\in\Tot(\s\End^A_B(j))$ is of degree $0$, we have that that $f_{ij}$ is of bidegree $(i,1-i-j)$. Thus, the equation

\[\rho(f\circ_{(1)}m^A)=\lambda(m^B\circ f),\] 

coincides with the one defining $\infty$-morphisms of $A_\infty$-algebras (\Cref{dinftymaps}) up to sign. The signs that appear in the above equation are obtained in a similar way to that on the brace $b_j^T$ (see \Cref{bracetot}). Thus, it is enough to plug in the sign provided by \Cref{bracetot} the corresponding degrees and arities to obtain the desired result. The composition of derived $\infty$-morphisms follows similarly.
\end{proof}

In the case that $f:A\to A$ is an $\infty$-endomorphism, the above definition can be written in terms of operadic composition as $f\star m=\gamma^\star(m\circ f)$, where $\gamma^*$ is the composition map with derived from the $\star$ operation (see \Cref{gammastar}). Here, $\circ$ is the plethysm of maps of collections, not to be confused with composition of maps. 


\begin{defin}
A \emph{derived $A_\infty$-multiplication} on a bigraded operad $\OO$ is a map of operads $d\calA_\infty\to\OO$.
\end{defin}

\begin{lem}\label{mstar}
A derived $A_\infty$-multiplication on a bigraded operad $\OO$ is equivalent to an element $m\in\Tot(\s\OO)$ of degree 1 concentrated in positive arity such that $m\star m = 0$. 
%$m=\sum_{ij}m_{ij}$ where $m_{ij}\in\OO(j)^{2-i-j}_i$ for each $j\geq 1$ such that 
%\[\underset{j=r+1+t}{\sum_{u=i+p, v=j+q-1}}(-1)^{rq+t+pj}m_{ij}\circ_{r+1} %m_{pq}=0.\]
\end{lem}
\begin{proof}
A derived $A_\infty$-multiplication on $\OO$ is by definition a map 
\[f:d\calA_\infty\to\OO.\]
Since $\calA_\infty$ is generated by elements $\mu_{ij}$ of bidegree $(i,2-i-j)$, such a map is determined by the elements $m_{ij}=f(\mu_{ij})\in\OO^{2-i-j}_i(j)$. Consider $m_j = (m_{ij})_i\in\Tot(\s\OO(j))$. We have that $\deg(m_j)=1$ for all $j$. Therefore, let $m=m_1+m_2+\cdots\in\Tot(\s\OO)$. We may check that $m\star m=0$. For that we just need to check \Cref{star}. On arity $n$, this amounts to compute 
\[(m\star m)_k = \sum_{r=0}^{n-1}\underset{j+q=n-1}{\sum_{i+p=k}}(-1)^{rp+j-r-1+ pj}m_{ij}\circ_{r+1}m_{pq}=0.\]
The above expression vanishes precisely because the elements $m_{ij}$ satisfy the derived $A_\infty$-equation.

Conversely, let $m\in\Tot(\s\OO)$ of degree 1, is concentrated in positive arity and satisfying $m\star m=0$. We can split $m$ into its arity and horizontal degree components as $m=\sum_{i,j}m_{ij}$. As we have seen, the fact that $m\star m=0$ is equivalent to the elements $m_{ij}$ satisfying the derived $A_\infty$-equation, and therefore, a map $f:d\calA_\infty\to\OO$ is determined by $f(\mu_{ij})=m_{ij}$, which are of bidegree $(i,2-i-j)$. 
\end{proof}


From \Cref{mstar} and our previous results PREVIOUS ARTICLE, it follows that $m$ determines an $A_\infty$-algebra structre on $S\Tot(\s\OO)\cong S\s \Tot(\OO)$ %IT WAS EQUALITY WITH THE ALTERNATIVE OPERATION 
that can be iterated to define an $A_\infty$-structure on $S\s\End_{S\s \Tot(\OO)}$.%THIS PART ONLY IF I CAN DO THE SAME IN THE DERIVED CASE, which is related to the structure on $S\s \Tot(\OO)$ by a map of $A_\infty$-algebras  $\Phi:S\s \Tot(\OO)\to  S\s\End_{S\s \Tot(\OO)}$.  MAYBE WRITE THIS MORE EXPLICITLY 

The goal now is showing that this $A_\infty$-structure on $S\Tot(\s\OO)$ is equivalent to a derived $A_\infty$-structure on $S\s \OO$ and compute the structure maps explicitly. For this, we need a filtered version of $A_\infty$-algebras.
%HERE THE STAR OPERATION, I NEED TO DEFINE DERIVED AINFTY MULTIPLICATION ON AN OPERAD, THEN STAR OPERATION, THEN ELEMENT OF TOTALIZED VERTICAL SUSPENSION

\begin{defin}
A \emph{filtered $A_∞$-algebra} is an $A_∞$-algebra $(A,m_i)$ together with a filtration $\{F_pA^i\}_{p∈\Z}$
on each $R$-module $A^i$ such that for all $i ≥ 1$ and all $p_1,\dots , p_i ∈ \Z$ and $n_1,\dots , n_i ≥ 0$,
\[m_i(F_{p_1}A^{n_1} ⊗ \cdots ⊗ F_{p_i}A^{n_i} ) ⊆ F_{p_1+\cdots
+p_i}A^{n_1+\cdots+n_i+2−i}.\]
\end{defin}


\begin{remark}\label{filterversion}
Consider $\calA_∞$ as an operad in filtered complexes with the trivial filtration and let $K$
be a filtered complex. There is a one-to-one correspondence between filtered $A_∞$-algebra structures on $K$ and
morphisms of operads in filtered complexes $\calA_∞ → \underline{\End}_K$ (recall $\underline{\Hom}$ from \Cref{filterend}). To see this, notice that if one forgets the
filtrations such a map of operads gives an $A_∞$-algebra structure on $K$. The fact that this is a map of operads
in filtered complexes implies that all the $m_i$'s respect the filtrations. 

Since the image of $\calA_∞$ lies in $\End_K=F_0\underline{\End}_K$, if we regard $\calA_∞$ as an operad in chain complexes, then we get a one-to-one correspondence between filtered $A_∞$-algebra structures on $K$ and
morphisms of operads in chain complexes $\calA_∞ → \End_K$.
\end{remark}

\begin{defin}
A \emph{morphism of filtered $A_∞$-algebras} from $(A,m_i, F)$ to $(B,m_i, F)$ is a morphism
$f : (A,m_i) → (B,m_i)$ of $A_∞$-algebras such that each map $f_j : A^{⊗j} → A$ is compatible with filtrations:
\[f_j(F_{p_1}A^{n_1} ⊗ \cdots ⊗ F_{p_j}A^{n_j} ) ⊆ F_{p_1+\cdots +p_j}B^{n_1+\cdots +n_j+1−j} ,\]
for all $j ≥ 1$, $p_1,\dots p_j ∈ \Z$ and $n_1,\dots , n_j ≥ 0$.
\end{defin}

\section{Braces}\label{sectionbraces}
We are going to define a brace structure using totalization. First recall the definition of a brace algebra and introduce the definition for the bigraded case.

\begin{defin}\label{braces}
A brace algebra on a (bi)graded module $A$ consists of a family of maps \[b_n:A^{\otimes 1+n}\to A\] called \emph{braces}, that we evaluate on $(x,x_1,\dots, x_n)$ as $b_n(x;x_1,\dots, x_n)$, satisfying the \emph{brace relation}


\begin{align*}
b_m(b_n(x;x_1,\dots, x_n);y_1,\dots,y_m)=&\\
\sum_{i_1,\dots, i_n, j_1\dots, j_n}(-1)^{\varepsilon}&b_l(x; y_1,\dots, y_{i_1},b_{j_1}(x_1;y_{i_1+1},\dots, y_{i_1+j_1}),\dots, b_{j_n}(x_n;y_{i_n+1},\dots, y_{i_n+j_n}),\dots,y_m)
\end{align*}
where $l=n+\sum_{p=1}^n i_p$ and in the single graded case $\varepsilon=\sum_{p=1}^n|x_p|\sum_{q=i}^{i_p}|y_q|,$ i.e. the sign is picked up by the $x_i$'s passing by the $y_i$'s in the shuffle. In the bigraded case, $\varepsilon=\sum_{p=1}^n\sum_{q=i}^{i_p}\langle x_p,y_q\rangle$.



\end{defin}

%\begin{remark}
%Some authors might use the notation $b_{1+n}$ instead of $b_n$, but the first element is usually going to have a different role than the others. A shorter notation for $b_n(x;x_1,\dots,x_n)$ found in the literature is $x\{x_1,\dots, x_n\}$. Also note that we have used the notation $|x_p|$ for the degree of $x_p$ in $A$. 
%\end{remark}

We can define braces on $\Tot(\s\OO)$ via operadic composition. More precisely, we define the maps 
\[b^T_n:\Tot(\mathfrak{s}\OO(N))\otimes \Tot(\mathfrak{s}\OO(a_1))\otimes\cdots\otimes \Tot(\mathfrak{s}\OO(a_n))\to \Tot(\mathfrak{s}\OO(N-\sum a_i))\]
using the operadic composition $\gamma^\star$ on $\Tot(\mathfrak{s}\OO)$ as

\[b^T_n(f;g_1,\dots,g_n)=\sum\gamma^\star(f;1,\dots,1,g_1,1,\dots,1,g_n,1,\dots,1),\]

where the sum runs over all possible ordering preserving insertions. The brace $b^T_n(f;g_1,\dots,g_n)$ vanishes whenever $n>N$ and $b^T_0(f)=f$. We use the notation $b^T_n$ to distinguish this brace map from the brace $b_n$ that can be naturally defined on the bigraded operad $\s\OO$.

The operadic composition can be described in terms of insertions in the obvious way, namely 

\begin{equation}\label{gammastar}
\gamma^\star(f;h_1,\dots, h_N)=(\cdots(f\star_1 h_1)\star_{1+a(h_1)}h_2\cdots)\star_{1+\sum a(h_p)}h_N,
\end{equation}

where $a(h_p)$ is the arity of $h_p$ (in this case $h_p$ is either $1$ or some $g_i$). If we want to express this composition in terms of the composition in $\OO$ we just have to find out the factor sign applying the same strategy as in the single-graded case. In fact, there is a sign factor that comes from vertical operadic suspension identical to the graded case replacing internal degree by vertical (internal) degree. This is the sign that determines the brace $b_j$ on $\s\OO$. Explicitly, it is given by the following lemma, whose proof is identical to the single graded case.


 
 \begin{lem}\label{bigradedsign}
For $x\in \s\OO(N)$ and $x_i\in\s\OO(a_i)$ of internal vertical degree $q_i$ ($1\leq i\leq n$), we have
\[b_n(x;x_1,\dots,x_n)=\sum_{N-n=h_0+\cdots+h_n} (-1)^\eta \gamma
(x\otimes 1^{\otimes h_0}\otimes x_1\otimes \cdots\otimes x_n\otimes1^{\otimes h_n}),\]
where 
\[\eta=\sum_{0\leq j<l\leq n}h_jq_l+\sum_{1\leq j<l\leq n}a_jq_l+\sum_{j=1}^n (a_j+q_j-1)(n-j)+\sum_{1\leq j\leq l\leq n} (a_j+q_j-1)h_l.\]
\end{lem}

The other sign factor is produced by totalization. This was computed in \Cref{totcomp}. Combining both factors we obtain the following.

\begin{lem}
We have 
\begin{equation}\label{bracetot}
b_j^T(x;x^1,\dots, x^N)_k=\underset{h_0+h_1+\cdots+h_N=j-N}{\sum_{k_0+k_1+\cdots+k_N=k}}(-1)^{\eta+\sum_{j=1}^m n_j\sum_{i=0}^{j-1}k_i}\gamma(x_{k_0};1^{h_0},x^1_{k_1},1^{h_1},\dots, x^N_{k_N},1^{h_N})
\end{equation}
for $x=(x_k)_k\in\Tot(\s\OO(N))^{n_0}$ and $x^i=(x^i_k)_k\in\Tot(\s\OO(a_i))^{n_i}$, where $\eta$ is defined in \Cref{bigradedsign}. 
\end{lem}
%Therefore we only need to compute the sign factor corresponding to totalization. Given $f_{p_0}\in \OO(N)^{q_0}_{p_0}$ and  $g_i\in\OO(a_i)^{q_i}_{p_i}$ for $1\leq i\leq n$, %such that $p_0+p_1+\cdots+p_n=k$, 
%the factor sign we are looking for is determined by the exponent
%
%%\[\varepsilon_k=p_1(N+q_0+p_0-1)+p_2(N+a_1+q_0+q_1+p_0+p_1-2)+\cdots=\sum_{i=1}^np_i(\sum_{j=0}^{i-1}(a_j+q_j+p_j)+N-i).\]
%
%\[\varepsilon
%=p_0(a_1+q_1+p_1-1)+(p_0+p_1)(a_2+q_2+p_2-1)+\cdots=\sum_{i=1}^n(a_i+q_i+a_i-1)\sum_{j=0}^{i-1}p_j\]
%\begin{equation}\label{epsilon}
%=\sum_{0\leq j<i\leq n}p_j(a_i+q_i+p_i-1).
%\end{equation}
%This is obtained by iteration of the last factor sign in \Cref{star} which is precisely the sign determined by totalization. Note that $a_i+q_i+p_i-1$ is precisely the total degree of $g_i$ in $\Tot(\s\OO)$. Therefore, if $(-1)^\eta$ is the sign produced by operadic suspension and $(-1)^{\varepsilon}$ the sign produced by totalization, the factor sign that distinguishes the brace on $\Tot(\s\OO)$ from the usual operadic composition on $\OO$ is $(-1)^{\eta+\varepsilon}$ %THE K IS BECAUSE I'M PLANNING TO REWRITE THINGS AS SUMS IN TOTALIZATION, I MIGHT ABOUT THIS, I WILL HAVE TO CHANGE THE NOTATION FOR THE MAPS BEING COMPOSED AND THEIR DEGREES


% USING BRACES TO DEFINE A DERIVED AINFTY ALGEBRA ON THE OPERAD (I WILL PROBABLY  WRITE THE COMPUTATION OF DEGREES DIFFERENTLY LATER BECAUSEE I AM IDENTIFYING THE BRACE AND ELEMENTS WITH THEIR SHIFTS HERE TO MAKE IT EASIER TO COMPPUTE)
%
%Let $m_{il}$ a component of the derived $A_\infty$-multiplication $m$ and $x_1,\dots, x_j\in\OO$ with $x_k$ of bidegree $(i_k,l_k)$, and let us compute the bidegree of $b_j(m_{il};x_1,\dots, x_j)$ in $S\s\OO$. By definition, the horizontal degree on $S\s\OO$ is the same as in $\OO$, so it is $i+\sum_{k=1}^ji_k$, meaning that the horizontal degree of the map $b_j(m_{il};-):(S\s\OO)^{\otimes j}\to S\s\OO$ is precisely $i$. Now let us compute the vertical degree. By definition it is given by the internal vertical degree plus the arity. Therefore, let us compute
%
%\[
%a(b_j(m_{il};x_1,\dots, x_j))+\deg(b_j(m_{il};x_1,\dots, x_j))
%\]
%where
%
%\[
%a(b_j(m_{il};x_1,\dots, x_j))=l-j+\sum a(x_k)
%\]
%and 
%\[\deg(b_j(m_{il};x_1,\dots, x_j))=2-l-i+\sum \deg(x_k)\]
%so the sum is $2-i-j+\sum (a(x_k)+\deg(x_k))$, so that the vertical degree of the map $b_j(m_{il};-)$ on the shift is $2-i-j$. This means, that we have the following candidates for a derived $A_\infty$-structture
%THESE ARE CANDIDATES BEFORE SHIFTING  AND I ALSO NEED TO TOTALIZE
%
%\[M_{ij}(x_1,\dots, x_j)=\sum_l b_j(m_{il};x_1,\dots, x_j)\]
%\[M_{i1}(x)= \sum_l (b_1(m_{il};x)-(-1)^{\langle x,m_{il}\rangle}b_1(x;m_{il}))\]
%to be a derived structure on that shift. THE SSIGN SHOULD BE THE SCALAR PRODUCT OF THE BIDEGREES $\langle x,m_{il}\rangle=x_hi+x_v(1-i)$ (degrees in $\s\OO$)
%
%CHECK THAT THEY SATISFYI THE EQUATION UP TO SIGN AND WORRY LATER ABOUT THE SIGNS
%
%DO THE MI1 DEFINE A TWISTED COMPLEX? THIS IS A NECESSARY CONDITION TO SATISSFY THE DAINFTY EQUATION THAT MIGHT BE EASIER TO SHOW AND IIT WOULD MAKE EASIER TO CONNECT EVERYTHING HERE WITH 



\section{Derived $A_\infty$-structure on an operad}\label{derivedstructure}


We are going to use the following theorem from \cite{whitehouse} to show that there is a derived $A_\infty$-structure on $A=S\s\OO$. Note that $\Tot(SB)=S\Tot(B)$ for any bigraded module $B$, where $SB$ is the vertical suspension of $B$ and $S\Tot(B)$ is the suspension of $\Tot(B)$ as graded modules.

\begin{thm}\label{whitehouse}
Let $(A, d^A) ∈ \tc^b$ be an twisted complex horizontaly bounded on the right and $A$ its underlying
chain complex. We have natural bijections %this means that A has d0 as a differential and End_A has [d0,-]
\begin{align*}
\Hom_{\mathrm{vbOp},d^A}(d\calA_∞,\End_A) &\cong
\Hom_{\mathrm{vbOp}}(\calA_∞, \uEnd_A)\\
&\cong \Hom_{\mathrm{vbOp}}(\calA_∞, \uEnd_{\Tot(A)})\\
&\cong \Hom_{\mathrm{fCOp}}(\calA_∞,\underline{\End}_{\Tot(A)}),
\end{align*}
where $\vbOp$ and $\fCOp$ denote the categories of operads in $\vbc$ and $\fc$ respectively, and $\Hom_{\vbOp,d^A}$
denotes the subset of morphisms which send $μ_{i1}$ to $d^A_i$. We view $\mathcal{A}_∞$ as an operad in $\vbc$ sitting in
horizontal degree zero or as an operad in filtered complexes with trivial filtration.
\end{thm}
\begin{proof}
See \cite[Poposition 4.55]{whitehouse}.
\end{proof}

\begin{remark}
According to \Cref{filterversion}, the last isomorphism can be replaced by 
\[\Hom_{\mathrm{vbOp}}(\calA_∞, \uEnd_{\Tot(A)})\cong \Hom_{\mathrm{COp}}(\calA_∞,\End_{\Tot(A)}),\]
where $\mathrm{COp}$ is the category of operads in chain complexes. 
\end{remark}
There are several important assumptions to make in order to use the theorem. First of all, we need $A$ to be horizontally bounded on the right, meaning that there exists some integer $i$ such that $A_k^n=0$ for all $k>i$. In our case $A=S\s\OO$ for $\OO$ an operad with a derived $A_\infty$-multiplication, so being horizontally bounded on the right implies that for each $j>0$ we can only have finitely many non-zero $dA_\infty$ components $m_{ij}$. This situation happens in practice in all known examples of derived $A_\infty$-algebras so far (some of them are in \cite[Remark 6.5]{muro}, \cite{RW}, and \cite[\S 5]{women}). Under this assumption we may replace all direct products by direct sums.

We also need to provide $A$ with a twisted complex structure. The reason for this is that the theorem uses the definition of derived $A_\infty$-algebras on an underlying twisted complex (See \Cref{equivalent}). We do this explicitly in \Cref{twistedoperad}, but it also follows from \Cref{mi1}. We also provide another version of this theorem that works for bigraded modules, \Cref{alternative}. 

With these assumption, by \Cref{whitehouse} we can guarantee the existence of a derived $A_\infty$-algebra structure on $A$ provided that $\Tot(A)$ has an $A_\infty$-algebra structure.




\begin{thm}\label{derivedmaps}
Let $A=S\s\OO$ where $\OO$ is an operad horizontally bounded on the right with a derived $A_\infty$-multiplication $m=\sum_{ij}m_{ij}\in\OO$. Let $x_1\otimes\cdots\otimes x_j\in (A^{\otimes j})^{n-k}_k$ and let $x_v = Sy_v$ for $v=1,\dots, j$ and $y_v$ be of bidegree $(k_v,n_v-k_v)$. The following maps $M_{ij}$ for $j\geq 2$ determine a derived $A_\infty$-algebra structure on $A$.
%\[M_{ij}(x_1,\dots,x_j)= (-1)^{\sum_{v=2}^j n_v\sum_{w=1}^{v-1}k_w+\sum_{v=1}^j(j-v)(n_v-k_v)+\sum_{1\leq v<w\leq j}k_v(a_w+n_w-1)}\sum_lSb_j(m_{il};y_1,\dots, y_j) \]

%THINGS CANCEL, INCLUDE CALCULATION (THE SIGN FROM MU CANCELS THE SIGN FROM EPSILON BECAUSE IT ALSO COMES FROM MU, JUST LEAVING THE PART CORRESPONDING TO M-IL). SINCE THE ELEMNT IS FROM SO ITS VERTICAL DEGREE INCLUDES ARITY -1
\[M_{ij}(x_1,\dots,x_j)= (-1)^{\sum_{v=1}^j(j-v)(n_v-k_v)}\sum_lSb_j(m_{il};y_1,\dots, y_j). \]
\end{thm}
Note that we abuse of notation and identify $x_1\otimes\cdots\otimes x_j$ with an element of $\Tot(A^{\otimes j})$ with only one non-zero component. For a general element, extend linearly.

\begin{proof}
Since $m$ is a derived $A_\infty$-multiplication $\OO$, we have that $m\star m=0$ when we view $m$ as an element of $\Tot(\s\OO)$. By PREVIOUS ARTICLE this defines an $A_\infty$-algebra structure on $S\Tot(\s\OO)$ given by maps
 %This isomorphism introduces some signs that will cancel with the sigs introduced by the isomorphism FIRST ISO OF THE CHAIN so we will omit them. The $A_\infty$-structure is therefore given by maps THINK WHERE I HAVE TO PUT SUSPENSION
\[M_j:(S\Tot(\s\OO))^{\otimes j}\to S\Tot(\s\OO)\]
induced by shifting brace maps
\[b_j^T(m;-):(\Tot(\s\OO))^{\otimes j}\to \Tot(\s\OO).\]
 The graded module $S\Tot(\s\OO)$ is endowed with the structure of a filtered complex with differential $M_1$ and filtration induced by the column filtration on $\Tot(\s\OO)$. Note that $b^T_j(m;-)$ preserves the column filtration since every component $b^T_j(m_{ij};-)$ has positive horizontal degree. % since the shift is only vertical. MORE DETAIL ON THIS? (LIKE  EXPLIIT FILTRATION AND MAYBE CHECK THAT THE DIFFERENTIAL PRESERVES FILTRATION WHICH FOLLOWS FROM THE HORIZONTAL DEGREES OF THE MIJ BEING POSITTIVE)
 
Since $S\Tot(\s\OO)\cong \Tot(S\s\OO)$, we can view $M_j$ as the image of a morphism of operads of filtered complexes $f:\mathcal{A}_\infty\to \End_{\Tot(S\s\OO)}$, in such a way that $M_j=f(\mu_j)$ for $\mu_j\in\mathcal{A}_\infty(j)$. 

We now work our way backwards through the proof of \Cref{whitehouse}. The isomorphism 
\[\Hom_{\mathrm{vbOp}}(\calA_∞, \uEnd_{\Tot(A)})\cong \Hom_{\mathrm{COp}}(\calA_∞,\End_{\Tot(A)})\]
does not modify the map $M_j$ at all, but allows us to see it as a element of $\uEnd_{\Tot(A)}$ of bidegree $(0,2-j)$. 



The isomorphism 
\[\Hom_{\mathrm{vbOp}}(\calA_∞, \uEnd_A)\cong \Hom_{\mathrm{vbOp}}(\calA_∞, \uEnd_{\Tot(A)})\] 
in the direction we are following is the result of applying $\Hom_{\vbOp}(\calA_\infty,-)$ to the map described after \Cref{inverse}. Under this isomorphism, $f$ is sent to the map \[\mu_j\mapsto \mathfrak{Tot}^{-1}\circ c(M_j,\mu^{-1})=\mathfrak{Tot}^{-1}\circ M_j\circ \mu^{-1},\] where $c$ is the composition in $\ufC$. The functor $\mathfrak{Tot}^{-1}$ decomposes $M_j$ into a sum of maps $M_j=\sum_i \widetilde{M}_{ij}$, where each $\widetilde{M}_{ij}$ is of bidegree $(i,2-j-i)$. More explicitly, let $A=S\s\OO$ and let $x_1\otimes\cdots\otimes x_j\in (A^{\otimes j})^{n-k}_k$. We abuse of notation and identify $x_1\otimes\cdots\otimes x_j$ with an element of $\Tot(A^{\otimes j})$ with only one non-zero component. For a general element, extend linearly. Then we have

%I AM ABUSING NOTATION, THE ELEMENTS SHOULD BE IN TOT(A x ... x A) SO I AM REFERRING TO STRINGS WITH ONLY ONE NON-ZERO COMPONENT
\begin{align}\label{totsign}
\mathfrak{Tot}^{-1}(M_j( \mu^{-1}(x_1\otimes\cdots\otimes x_j)))&=\mathfrak{Tot}^{-1}(Sb_j^T(m;(S^{-1})^{\otimes j}(\mu^{-1}(x_1\otimes\cdots\otimes x_j))))\nonumber\\
&=\sum_i(-1)^{in}\sum_l Sb_j^T(m_{il};(S^{-1})^{\otimes j}(\mu^{-1}(x_1\otimes\cdots\otimes x_j)))\nonumber\\
&=\sum_i(-1)^{in}\sum_l(-1)^{\varepsilon} Sb_j(m_{il};(S^{-1})^{\otimes j}(\mu^{-1}(x_1\otimes\cdots\otimes x_j)))\nonumber\\
&=\sum_i\sum_l(-1)^{in+\varepsilon} Sb_j(m_{il};(S^{-1})^{\otimes j}(\mu^{-1}(x_1\otimes\cdots\otimes x_j)))
\end{align}
so that \[\widetilde{M}_{ij}(x_1,\dots,x_j)=\sum_l(-1)^{in+\varepsilon} Sb_j(m_{il};(S^{-1})^{\otimes j}(\mu^{-1}(x_1\otimes\cdots\otimes x_j))),\] where $b_j$ is the brace on $\s\OO$ and $\varepsilon$ is given in \Cref{totcomp}. 


According to the isomorphism 
\begin{equation}\label{firstiso}
\Hom_{\mathrm{vbOp},d^A}(d\calA_∞,\End_A)\cong
\Hom_{\mathrm{vbOp}}(\calA_∞, \uEnd_A),
\end{equation}
 the maps $M_{ij}=(-1)^{ij}\widetilde{M}_{ij}$ define an $A_\infty$-structure on $S\s\OO$. Therefore we now just have to work out the signs. Notice that $n_v$ is the total degree of $y_v$ as an element of $\s\OO$ and recall that $n$ is the total degree of $x_1\otimes\cdots\otimes x_j\in A^{\otimes j}$. Therefore, $\varepsilon$ can be written as
\[\varepsilon= i(n-j)+\sum_{1\leq v<w\leq j}k_vn_w.\]
The sign produced by $\mu^{-1}$, as we saw in \Cref{mui}, is precisely determined by the exponent 
\[\sum_{w=2}^jn_w\sum_{v=1}^{w-1}k_v=\sum_{1\leq v<w\leq j}k_vn_w,\]so this sign cancels the right hand summand of $\varepsilon$. This cancellation was expected since this sign comes from $\mu^{-1}$ and operadic composition is defined using $\mu$ (\Cref{insertion}). %In fact, both signs come from $\mu$, so the cancellation was expected. 
Finally, the sign $(-1)^{i(n-j)}$ left from $\varepsilon$ cancels with $(-1)^{in}$ in \Cref{totsign} and $(-1)^{ij}$ from the isomorphism (\ref{firstiso}). This means that we only need to consider signs produced by vertical shifts. This calculation has been done previously in PREVIOUS ARTICLE and as we claimed the result is 
\[M_{ij}(x_1,\dots,x_j)= (-1)^{\sum_{v=1}^j(j-v)(n_v-k_v)}\sum_lSb_j(m_{il};y_1,\dots, y_j). \]

\end{proof}

\begin{remark}\label{equivalent}
Note that as in the case of $A_∞$-algebras in $\mathrm{C}_R$  
we have two equivalent descriptions of $A_∞$-algebras in $\tc$.
%MAKE SURE  I HAVE STATED THAT FIRST EQUIVALENCE SOMEWHERE (IN THE OTHER ARTICLE FOR INSTANCE)
\begin{enumerate}[(1)]
\item A twisted complex $(A, d^A)$ together with a morphism $\calA_∞ → \uEnd_A$ of operads in $\vbc$, which is determined by a family of elements $M_i ∈ \utC(A^{⊗i},A)^{2−i}_0$ for $i ≥ 2$ for which the $A_\infty$-relations holds for $i\geq 2$, where the composition is the one prescribed by the composition morphisms of $\utC$.
\item A bigraded module $A$ together with a family of elements $M_i ∈ \ubgMod(A^{⊗i},A)^{2−i}_0$ for $i ≥ 1$ for
which all the $A_\infty$ relations hold REFERENCE, where the composition is the one prescribed by the composition
morphisms of $\ubgMod$.
\end{enumerate}
Since the composition morphism
in $\ubgMod$ is induced from the one in $\utC$ by forgetting the differential, these two presentations
are equivalent.
\end{remark}

This equivalence allows us to formulate the following alternative version of \Cref{whitehouse}.
\begin{corollary}\label{alternative}
Given a bigraded module $A$ horizontally bounded on the right we have isomorphhisms
\begin{align*}
\Hom_{\mathrm{bgOp}}(d\calA_∞,\End_A) &\cong
\Hom_{\mathrm{bgOp}}(\calA_∞, \uEnd_A)\\
&\cong \Hom_{\mathrm{bgOp}}(\calA_∞, \uEnd_{\Tot(A)})\\
&\cong \Hom_{\mathrm{fOp}}(\calA_∞,\underline{\End}_{\Tot(A)}),
\end{align*}
where $\mathrm{bgOp}$ is the category of operads of bigraded modules and $\mathrm{fOp}$ is the category of operads of filtered modules. %REFERENCE TO WHAT DEFINITIONS OF THE OPERADS I AM USING
\end{corollary}
\begin{proof}
Let us look at the first isomorphism

\[\Hom_{\mathrm{bgOp}}(\calA_∞, \uEnd_A)\cong \Hom_{\mathrm{bgOp}}(d\calA_∞,\End_A).\]

Let $f:\calA_\infty\to\uEnd_A$ be a map of operads in $\mathrm{bgOp}$. This is equivalent to maps in $\mathrm{bgOp}$
\[\calA_\infty(v)\to\uEnd_A(v)\]
for each $v\geq 1$, which are determined by elements $M_v\coloneqq f(\mu_v)\in\uEnd_A(v)$ for $v\geq 1$ of bidegree $(0,2-v)$ satisfying the $A_\infty$-equation with respect to the composition in $\ubgMod$. Moreover, $M_v\coloneqq (\tilde{m}_{0v},\tilde{m}_{1v},\dots)$ where $\tilde{m}_{uv}\coloneqq (M_v)_u:A^{\otimes n}\to A$ is a map of bidegree $(u,2-v-u)$. Since the composition in $\ubgMod$ is the same as in $\utC$, the computation of the $A_\infty$-equation becomes analogous to the computation done in \cite[Prop 4.47]{whitehouse}, showing that the maps $m_{ij}=(-1)^i\tilde{m}_{ij}$ for $i\geq 0$ and $j\geq 0$ define a derived $A_\infty$-algebra structure on $A$.

The second isomorphism
\[\Hom_{\mathrm{bgOp}}(\calA_∞, \uEnd_A)\cong \Hom_{\mathrm{bgOp}}(\calA_∞, \uEnd_{\Tot(A)})\]
follows from the bigraded module case of \Cref{inverse}. Finally, the isomorphism
\[\Hom_{\mathrm{bgOp}}(\calA_∞, \uEnd_{\Tot(A)})\cong \Hom_{\mathrm{fOp}}(\calA_∞,\underline{\End}_{\Tot(A)})\]
is analogous to the last isomorphism of \Cref{whitehouse}, replacing the quasi-free relation by the full $A_\infty$-algebra relations. 
\end{proof}

\begin{corollary}\label{mi1}
Let $\OO$ be a bigraded operad with a derived $A_\infty$-multiplication and let \[M_{i1}:S\s\OO\to S\s\OO\] be the arity 1 derived $A_\infty$-algebra maps induced by \Cref{alternative} from \[M_1:\Tot(S\s\OO)\to \Tot(S\s\OO).\]
Then \[M_{i1}(x)= \sum_l (Sb_1(m_{il};S^{-1}x)-(-1)^{\langle x,m_{il}\rangle}Sb_1(S^{-1}x;m_{il})),\]
where $x\in (S\s\OO)^{n-k}_k$ and $\langle x,m_{il}\rangle=ik+(1-i)(n-1-k)$.
\end{corollary}
\begin{proof}
Notice that the proof of \Cref{alternative} is essentially the same as the proof \Cref{whitehouse}. This means that the proof of this result is an arity 1 restriction of the proof of \Cref{derivedmaps}. Then we apply \Cref{totsign} to the case $j=1$. Recall that for $x\in (S\s\OO)^{n-k}_{k}$
\[M_1(x)=b_1^T(m;S^{-1}x)-(-1)^{n-1}b_1^T(S^{-1}x;m).\]
 In this case, there is no $\mu$ involved. Therefore, introducing here also the final extra sign $(-1)^i$ from the proof of \Cref{derivedmaps}, we get from \Cref{totsign} that
\[\widetilde{M}_{i1}(x)=(-1)^i\sum_l((-1)^{in+i(n-1)} Sb_1(m_{il};S^{-1}x)-(-1)^{n-1+in+k}Sb_1(S^{-1}x;m_{il})),\] where $b_1$ is the brace on $\s\OO$. Simplifying signs we obtain
\[\widetilde{M}_{i1}(x)=\sum_l Sb_1(m_{il};S^{-1}x)-(-1)^{\langle  m_{il},x\rangle}Sb_1(m_{il};S^{-1}x))=M_{i1}(x),\]

where $\langle  m_{il},x\rangle=ik+(1-i)(n-1-k)$, as claimed.
\end{proof}



\begin{appendices}
\appendix
\gdef\thesection{Appendix \Alph{section}}
\section{Twisted complex on an operad}\label{twistedoperad}
In this section we provide a description of the twisted complex structure on an operad $\OO$ with a derived $A_\infty$-multiplication. More precisely, we show by hand that the maps found in \Cref{mi1} define a twisted complex structure on $S\s\OO$.

\begin{lem}\label{twistedmaps}
Let $\OO$ be an operad with a derived $A_\infty$-multiplication $m\in\s\OO$. Then $S\s\OO$ becomes a twisted complex with structure maps
\[M_{i1}(x)= \sum_l (Sb_1(m_{il};S^{-1}x)-(-1)^{\langle x,m_{il}\rangle}Sb_1(S^{-1}x;m_{il})),\]
where $x\in (S\s\OO)^{n-k}_k$ and $\langle x,m_{il}\rangle=ik+(1-i)(n-1-k)$.
\end{lem}
\begin{proof}


Througout the proof we omit the shift maps. Let us first check the twisted complex equation up to signs, to give a conceptual proof before introducing the signs. Up to sign, the maps  $\{M_{i1}\}_{i\geq 0}$ must satisfy the equation

\[\sum_{i+j=u} M_{i1}\circ M_{j1}=0,\]
for all $u$, where $\circ$ is composition of maps. %I may or may not omit the sum to avoid writing too much, as it just means that the composition on every degree must vanish.

Therefore, up to signs we have to compute compute 

\begin{align*}
&\sum_{i+j=u}M_{i1}(M_{j1}(x))=\sum_{i+j=u}M_{i1}\left(\sum_l b_1(m_{jl};x)+b_1(x;m_{jl})\right)=\\
&\sum_{i+j=u}\sum_{l,k}\left(b_1(m_{ik}; b_1(m_{jl};x))+b_1(m_{ik};b_1(x;m_{jl}))+b_1(b_1(m_{jl};x);m_{ik})+b_1(b_1(x;m_{jl});m_{ik})\right).
\end{align*}

%AT FIRST SIGHT IT DOESN'T LOOK POSSIBLE TO CANCEL THE LAST BRACE BECAUSE IT IS THE ONLY ONE WITH X AT THE BEGINNING, BUT THAT SHOULD HAVE BEEN THE SAME FOR THE CLASSICAL CASE, SO I SHOULD REVIEW THAT ONE
%
%ON THE CLASSICAL CASE IT WAS MUCH EASIER BECAUSE AFTER BRACE RELATION B(X;M,M) APPEARS TWICE WITH OPPOSITE SIGN, SO IT CANCELS. HERE IT IS NOT SO OBVIOUS BECAUSE THE SIGN IS NOT JUST $-1$ SO MAYBE IT CANCELS WITH OTHER SUMMANDS (RECALL THAT I AM OMITTING ONE SUM)

Applying the brace relation we obtain

\begin{align*}
\sum_{i+j=u}\sum_{l,k}(b_1(m_{ik}; b_1(m_{jl};x))+b_1(m_{ik};b_1(x;m_{jl}))+\\
 b_2(m_{jl};x,m_{ik})+b_1(m_{jl};b_1(x;m_{ik}))+b_2(m_{jl};m_{ik},x)+\\
b_2(x;m_{jl},m_{ik})+b_1(x;b_1(m_{jl};m_{ik}))+b_2(x;m_{ik},m_{jl})).
\end{align*}

In the sum, all terms of the form $b_1(x;b_1(m_{jl};m_{ik}))$ that can be seen in the last line should add up to vanish provided that $m$ is a $dA_\infty$-multiplication (meaning that up to sign $b_1(m;m)=0$) %A sign of the form $(-1)^i$ %(or maybe $(-1)^j$, depending on the convention) 
 Since $i$ and $j$ are interchangable (i.e. for each pair $(i,j)$ there is the pair $(j,i)$), the terms $b_2(x;m_{jl},m_{ik})+b_2(x;m_{ik},m_{jl})$ in the last line should cancel as well (for this, we should have the pair $(j,i)$ with the opposite sign). Here it is also relevant that the sum runs through all possible values of $k$ and $l$, so that the pair $(j,i)$ appears with $l$ and $k$ interchanged as well. So far the entire last line vanishes up to sign.

Then $b_1(m_{ik};b_1(x;m_{jl}))$ on the first line should cancel with $b_1(m_{jl};b_1(x;m_{ik}))$ on the second line (but from a different summand, the one where $i$ and $j$ are interchanged). Finnaly, the remaining terms $b_1(m_{ik}; b_1(m_{jl};x))+b_2(m_{jl};x,m_{ik})+b_2(m_{jl};m_{ik},x)$ add up to $b_1(b_1(m;m);x)$ up to sign. That would cancel everything.



Let us now add signs. We now compute for all $u$
\[\sum_{i+j=u} (-1)^iM_{i1}\circ M_{j1}.\]
%recalling that for the usual sign convention of twisted complex from a $dA_\infty$-algebra we need to define $d_i=(-1)^im_{i1}$, so that the sign in the equation is $(-1)^j$ instead of $(-1)^i$. This being said, let us compute 
For $x\in\s\OO$, by definition, we have
\begin{align*}
\sum_{i+j=u}(-1)^iM_{i1}(M_{j1}(x))=\sum_{i+j=u}(-1)^iM_{i1}\left(\sum_l b_1(m_{jl};x)-(-1)^{\langle x|m_{jl}\rangle}b_1(x;m_{jl})\right)=\\
\sum_{i+j=u}(-1)^i\sum_{l,k}\left(b_1(m_{ik}; b_1(m_{jl};x))-(-1)^{\langle x|m_{jl}\rangle}b_1(m_{ik};b_1(x;m_{jl}))+\right.\\
\left. -(-1)^{\langle b_1(m_{jl};x)|m_{ik}\rangle}b_1(b_1(m_{jl};x);m_{ik})+(-1)^{\langle b_1(m_{jl};x)|m_{ik}\rangle+\langle x|m_{jl}\rangle}b_1(b_1(x;m_{jl});m_{ik})\right).
\end{align*}
Observe that $\langle b_1(m_{jl};x)|m_{ik}\rangle=\langle m_{ij}|m_{ik}\rangle+\langle x|m_{ik}\rangle$. %in the usual bigraded sign convention (also in the total graded convention). I am not going to explicitly compute these signs yet to see what properties we need from them.

Applying the brace relation we obtain

\begin{align}
\sum_{i+j=u}\sum_{l,k}((-1)^ib_1(m_{ik}; b_1(m_{jl};x))-(-1)^{i+\langle x|m_{jl}\rangle}b_1(m_{ik};b_1(x;m_{jl}))+\nonumber\\
 -(-1)^{i+\langle b_1(m_{jl};x)|m_{ik}\rangle}(b_2(m_{jl};x,m_{ik})+(-1)^{\langle x|m_{ik}\rangle}b_2(m_{jl};m_{ik},x))\nonumber\\
 -(-1)^{i+\langle b_1(m_{jl};x)|m_{ik}\rangle}b_1(m_{jl};b_1(x;m_{ik}))\label{twistedequation}\\
+(-1)^{i+\langle b_1(m_{jl};x)|m_{ik}\rangle+\langle x|m_{jl}\rangle}(b_2(x;m_{jl},m_{ik})+(-1)^{\langle m_{ik}|m_{jl}\rangle}b_2(x;m_{ik},m_{jl}))\nonumber\\
+(-1)^{i+\langle b_1(m_{jl};x)|m_{ik}\rangle+\langle x|m_{jl}\rangle}b_1(x;b_1(m_{jl};m_{ik}))).\nonumber
\end{align}

Recall that $m$ being a $dA_\infty$-multiplication means that $\sum_{i+j=u}\sum_{k,l}(-1)^ib_1(m_{jl};m_{ik})=0$. %Notice that the summand corresponding to each value of $i+j$ must vanish because it corresponds to a given horizontal degree. 
Let us check now the cancellations with the signs. First, let us check that the terms 
\[(-1)^{i+\langle b_1(m_{jl};x)|m_{ik}\rangle+\langle x|m_{jl}\rangle}b_1(x;b_1(m_{jl};m_{ik})))\]
can be added up to vanish. For that, we compute the sign \[\langle b_1(m_{jl};x)|m_{ik}\rangle+\langle x|m_{jl}\rangle=\langle m_{jl}|m_{ik}\rangle+\langle x|m_{ik}\rangle+\langle x|m_{jl}\rangle.\]
Recall that the braces are defined on the operadic suspension, so that the bidegree of $m_{ik}$ is $(i,1-i)$. Therefore, writing the bidegree of $x$ as $(k,n-k)$, so that the total degree is $|x|=n$, the above equals 
\[ji+(1-i)(1-j)+ki+(n-k)(1-i)+kj+(n-k)(1-j)\equiv 1+i+j + (i+j)k+(i+j)(n-k)\mod 2=\]
\[1+(i+j)(1+n)=1+u(1+|x|).\]
Since this sign is constant for all terms $b_1(m_{ik};m_{ij})$ that share the same horizontal degree $i+j=u$, we can rewrite
\[(-1)^{i+\langle b_1(m_{jl};x)|m_{ik}\rangle+\langle x|m_{jl}\rangle}b_1(x;b_1(m_{jl};m_{ik})))=-(-1)^{u(1+|x|)}b_1(x;(-1)^ib_1(m_{ik};m_{jl})).\]
Hence, 
%Multiplying 0 by something is 0, some sum vanishes and you multiply it by a constant sign, it still vanishes
\[\sum_{i+j=u}\sum_{k,l}-(-1)^{u(1+|x|)}b_1(x;(-1)^ib_1(m_{ik};m_{jl}))=0.\]
Therefore, the expression (\ref{twistedequation}) after the brace relation reduces to
\begin{align}
\sum_{i+j=u}\sum_{l,k}((-1)^ib_1(m_{ik}; b_1(m_{jl};x))-(-1)^{i+\langle x|m_{jl}\rangle}b_1(m_{ik};b_1(x;m_{jl}))+\nonumber\\
 -(-1)^{i+\langle b_1(m_{jl};x)|m_{ik}\rangle}(b_2(m_{jl};x,m_{ik})+(-1)^{\langle x|m_{ik}\rangle}b_2(m_{jl};m_{ik},x))\label{twistedequation2}\\
 -(-1)^{i+\langle b_1(m_{jl};x)|m_{ik}\rangle}b_1(m_{jl};b_1(x;m_{ik}))\nonumber\\
+(-1)^{i+\langle b_1(m_{jl};x)|m_{ik}\rangle+\langle x|m_{jl}\rangle}(b_2(x;m_{jl},m_{ik})+(-1)^{\langle m_{ik}|m_{jl}\rangle}b_2(x;m_{ik},m_{jl})).\nonumber
\end{align}
Let us focus on the last line. For each pair $(i,j)$ we should have cancellation with the pair $(j,i)$, which adds the same elements, but with different signs. We also need to consider the pairs $(k,l)$ and $(l,k)$ to get a cancellation. Let us compare the signs. For the pair $((i,j),(k,l))$ we have precisely the last line of the above equation
\[(-1)^{i+\langle b_1(m_{jl};x)|m_{ik}\rangle+\langle x|m_{jl}\rangle}(b_2(x;m_{jl},m_{ik})+(-1)^{\langle m_{ik}|m_{jl}\rangle}b_2(x;m_{ik},m_{jl}))\]

For the pair $((j,i),(l,k))$ we have
\[(-1)^{j+\langle b_1(m_{ik};x)|m_{jl}\rangle+\langle x|m_{ik}\rangle}(b_2(x;m_{ik},m_{jl})+(-1)^{\langle m_{jl}|m_{ik}\rangle}b_2(x;m_{jl},m_{ik})).\]
 Comparing the sign of $b_2(x;m_{jl},m_{ik})$ we find that for $((i,j),(k,l))$ we have

\[-(-1)^{i+(i+j)(1+|x|)}b_2(x;m_{jl},m_{ik})=-(-1)^{j+u|x|}b_2(x;m_{jl},m_{ik})\]
and for the pair $((j,i),(l,k))$ we have
\[(-1)^{j+u|x|}b_2(x;m_{jl},m_{ik}).\]
As we see, we get opposite signs and thus cancellation. For $b_2(x;m_{ik},m_{jl})$ it is completely analogous. Thus, we have reduced expression (\ref{twistedequation2}) to
\begin{align}
\sum_{i+j=u}\sum_{l,k}((-1)^ib_1(m_{ik}; b_1(m_{jl};x))-(-1)^{i+\langle x|m_{jl}\rangle}b_1(m_{ik};b_1(x;m_{jl}))+\nonumber\\
 -(-1)^{i+\langle b_1(m_{jl};x)|m_{ik}\rangle}(b_2(m_{jl};x,m_{ik})+(-1)^{\langle x|m_{ik}\rangle}b_2(m_{jl};m_{ik},x))\label{twistedequation3}\\
 -(-1)^{i+\langle b_1(m_{jl};x)|m_{ik}\rangle}b_1(m_{jl};b_1(x;m_{ik})).\nonumber
\end{align}
In a similar fashion to the previous calculation, we are going to cancel $b_1(m_{ik};b_1(x;m_{jl}))$ in the first line with $b_1(m_{jl};b_1(x;m_{ik})$ in the last line by considering switched pairs. For the pair $((i,j),(k,l))$, the term in the first line is 
\[-(-1)^{i+\langle x|m_{jl}\rangle}b_1(m_{ik};b_1(x;m_{jl}))\]
and for the pair $((j,i),(l,k))$ the term in the last line is
\[-(-1)^{j+\langle b_1(m_{ik};x)|m_{jl}\rangle}b_1(m_{ik};b_1(x;m_{jl}))=(-1)^{1+j+\langle m_{ik}|m_{jl}\rangle+\langle x|m_{jl}\rangle}b_1(m_{ik};b_1(x;m_{jl}))=\]
\[(-1)^{i+\langle x|m_{jl}\rangle}b_1(m_{ik};b_1(x;m_{jl})),\]
which has precisely the opposite sign to the other pair, and thus cancels. This reduces expression (\ref{twistedequation3}) to 
\begin{align}
\sum_{i+j=u}\sum_{l,k}((-1)^ib_1(m_{ik}; b_1(m_{jl};x))&\label{twistedequation4}\\
 -(-1)^{i+\langle b_1(m_{jl};x)|m_{ik}\rangle}(b_2(m_{jl};x,m_{ik})&+(-1)^{i+\langle m_{jl}|m_{ik}\rangle}b_2(m_{jl};m_{ik},x)).\nonumber
\end{align}
We want this terms to add up to something of the form $b_1(b_1(m;m);x)$. Notice that for this we need to switch some pairs. For simplity, we switch the pair of the first term and rewrite the sum as
\begin{align}
\sum_{i+j=u}\sum_{l,k}((-1)^jb_1(m_{jl}; b_1(m_{ik};x))&\\
 -(-1)^{i+\langle b_1(m_{jl};x)|m_{ik}\rangle}b_2(m_{jl};x,m_{ik})&+(-1)^{i+\langle m_{jl}| m_{ik}\rangle}b_2(m_{jl};m_{ik},x)).\nonumber
\end{align}
Simplifying the signs we get
\begin{align*}
\sum_{i+j=u}\sum_{l,k}((-1)^jb_1(m_{jl}; b_1(m_{ik};x))
 +(-1)^{j+\langle x|m_{ik}\rangle}b_2(m_{jl};x,m_{ik})+(-1)^{j}b_2(m_{jl};m_{ik},x)).
\end{align*}
By the brace relation and \Cref{sharp} this equals
\[\sum_{i+j=u}\sum_{l,k}(-1)^jb_1(b_1(m_{jl};m_{ik});x)=0.\]
%For each position of insertion of x we have a 0 map applied to x, so the above sum is indeed equal to 0
\end{proof}


Recall that if $\OO$ is an operad with a derived $A_\infty$-multiplication, there is an $A_\infty$-algebra structure on $B=\Tot(S\s\OO)$ given by maps $M_j:B^{\otimes j}\to B$ induced by braces. We show next that the twisted complex structure induced from the map $M_1$ via \Cref{alternative} is the same as the one we have given above.

\begin{corollary}
Let $\OO$ be a bigraded operad with a derived $A_\infty$-multiplication and let \[\widetilde{M}_{i1}:S\s\OO\to S\s\OO\] be the arity 1 derived $A_\infty$-algebra maps induced by \Cref{alternative} from \[M_1:\Tot(S\s\OO)\to \Tot(S\s\OO).\]
Then $\widetilde{M}_{i1}=M_{i1}$, where $M_{i1}$ is defined in \Cref{twistedmaps}.
\end{corollary}
\begin{proof}
Notice that the proof of \Cref{alternative} is essentially the same as the proof \Cref{whitehouse}. This means that the proof of this result is an arity 1 restriction of the proof of \Cref{derivedmaps}. Then we apply \Cref{totsign} to the case $j=1$. Recall that for $x\in (S\s\OO)^{n-k}_{k}$
\[M_1(x)=b_1^T(m;S^{-1}x)-(-1)^{n-1}b_1^T(S^{-1}x;m).\]
 In this case, there is no $\mu$ involved. Therefore, introducing here also the final extra sign $(-1)^i$ from the proof of \Cref{derivedmaps}, we get from \Cref{totsign} that
\[\widetilde{M}_{i1}(x)=(-1)^i\sum_l((-1)^{in+i(n-1)} Sb_1(m_{il};S^{-1}x)-(-1)^{n-1+in+k}Sb_1(S^{-1}x;m_{il})),\] where $b_1$ is the brace on $\s\OO$. Simplifying signs we obtain
\[\widetilde{M}_{i1}(x)=\sum_l Sb_1(m_{il};S^{-1}x)-(-1)^{\langle  m_{il},x\rangle}Sb_1(m_{il};S^{-1}x))=M_{i1}(x),\]

where $\langle  m_{il},x\rangle=ik+(1-i)(n-1-k)$, as claimed.
\end{proof}
%\appendix
%\renewcommand{\appendixname}{Appendix:}




%\section{Some proofs and details}




\end{appendices}
%\phantomsection
\bibliographystyle{ieeetr}
\bibliography{newbibliography}
\end{document}
