\documentclass[join.tex]{subfiles}
%\usepackage{estilo-ejercicios}%
%\setcounter{section}{0}
%\newtheorem{defin}{Definition}[section]
%\newtheorem{lem}[defin]{Lemma}
%\newtheorem{propo}[defin]{Proposition}
%\newtheorem{thm}[defin]{Theorem}
%\newtheorem{eje}[defin]{Example}


%\usepackage{calligra}
%\usepackage[T1]{fontenc}
%\usepackage{empheq}
%\newcommand*\widefbox[1]{\fbox{\hspace{2em}#1\hspace{2em}}}
%\usepackage{adjustbox}
%--------------------------------------------------------
\begin{document}





\section{Future research}

We finish by outlining some question that remain open after our research that would be interesting to investigate in the future. These questions arise naturally from the work done with (derived) $A_\infty$-algebras and from the classical results by Gerstenhaber and Voronov \cite{GV}.

\subsection{Deligne Conjecture}
Let us review the structure on the Hochschild complex of an associative operad in order to understand the question that we will be asking about the (derived) $A_\infty$-case.

Let $\OO$ be an operad with an associative multiplication $m$, i.e. an $A_\infty$-multiplication $m$ (\cref{ainftymult}) such that $m=m_2$. In this case, as a consequence of \cref{ainftystructure} or by \cite{GV}[Proposition 2], we have a dg-algebra structure on $S\s\OO$ given by the differential
\[d(Sx) =  Sb_1(m; x) -(-1)^{|x|}Sb_1(x; m)\]
and the multiplication
\[m(Sx,Sy) = Sb_2(m;x,y).\]

In particular, if $\OO = \End_{SA}$ is the endomorphism operad of an associative algebra $A$ (such as $\s\PP$ for an operad $\PP$ with associative multiplication), these maps provide a dg-algebra structure on the Hochschild complex of $A$, denoted by $C^*(A)$. But this is not all the structure that we get. Since any operad is a brace algebra, we have an interaction between the dg-algebra and the brace structure. More precisely, $\OO$ has a structure of \emph{homotopy $G$-algebra} (see Definition 2 and Theorem 3 of \cite{GV}). MAYBE WRITE DOWN THE DEFINITION IN MY TERMS, SINCE IT IS A CONSEQUENCE OF BRACE RELATION
 This result is what we call the \emph{Deligne Conjecture}, which has its roots in the theory of topological operads SOME REFERENCE TO DELIGNE CONJECTURE

In the $A_\infty$-case, i.e. when $m$ is an $A_\infty$-multiplication in $\OO$, we get an $A_\infty$-algebra structure on $\OO$ REFERENCE TO RESULT, Having an infinite number of structure maps, the interaction of these with the braces becomes combinatorially huge. MAYBE GIVE IT A TRY For the lack of a better name, let the algebraic structure generated by the $A_\infty$-maps and braces be called a \emph{$G_{A_\infty}$-algebra}. The problem that is left open at this point is finding a closed description in terms of algebraic relations between $A_\infty$-maps and braces of \emph{$G_{A_\infty}$-algebra}. Similarly, we would like to extend this description to \emph{$G_{dA_\infty}$-algebras}, i.e. the algebras generated by derived $A_\infty$-maps (REFERENCE) and bigraded braces (REFERENCE). 


\subsection{Hochschild Cohomology}

Given the algebraic structure described above on the Hochschild complex of an associative algebra, we can then take cohomology with respect to $d$ to compute the Hochschild cohomology of $A$, denoted by $HH^*(A)$. It is known that $m$ and the bracket 
\[[x,y]=Sb_1(x; y) -(-1)^{|x||y|}Sb_1(y; x)\]
induces a structure of a Gerstenhaber algebra on $HH^*(A)$ \cite{GV}[Corollary 5]. The proof relies on some identities that can be deduced from the definition of homotopy $G$-algebra, such has graded homotopy commutativity. 

If we try to replicate this argument in the derived $A_\infty$, as we mentioned before, we encounter an explosion in the number of relations and maps involved in the description of $G_{A_\infty}$-algebras. MAYBE GIVE IT A TRY Therefore, the resulting structure has not been feasible to manipulate and it is not very clear what kind of algebraic structure it induces on cohomology. The derived case if of course even more difficult to handle.

Therefore, the challenges for future research are finding a neat way of describing $G_{A_\infty}$-algebras and $G_{dA_\infty}$-algebras, and the structure it induces on cohomology.


DETAILS (ONLY IN THESIS, NOT IN PAPER)
\url{https://math.stackexchange.com/questions/3887857/show-that-this-map-is-well-defined-on-cohomology/3887939#3887939}%☺
I MAY ALSO NEED TO WRITE THE DERIVATION OF (11) FROM (9)
\url{https://arxiv.org/pdf/hep-th/9409063.pdf}


%\subsection{Connecting morphisms}
%%About how could $\Phi$ connected morphisms, it's not even clear what notion of morphism to use because if you have $O\to P$ and it is not clear how to get $P^n\to P$ from $O^n\to O$.
%
%
%Another question that arises from our constructions is the following. For an operad $\OO$ with a (derived) $A_\infty$-multiplication, we have stablished a map of (derived) $A_\infty$-algebras
%\[S\s\OO\xrightarrow{\Phi}S\s\End_{S\s\OO}\]
%induced by the brace structure. If we have another operad $P$, we would have a notion of morphism $\OO\to\PP$ that allows us to build the following commutative diagram
%
%\[
%\begin{tikzcd}
%S\s\OO\arrow[r, "\Phi"]\arrow[d, "f"] & S\s\End_{S\s\OO}\arrow[d]\\
%S\s\PP\arrow[r, "\Phi"] & S\s\End_{S\s\PP}
%\end{tikzcd}
%\]
%
%If $f$ is, say, a morphism of operads, this problems boils down to the problem of finding a map $\End_\OO\to\End_\PP$, which is not possible in general unless $f$ is an isomorphism. We would like to find a notion of morphism $\OO\to\PP$ that allows for such commutative diagrams more generally. 
%
%ASKED ABOUT THIS \url{https://math.stackexchange.com/questions/4615572/map-between-endomorphism-operads}

\end{document}
