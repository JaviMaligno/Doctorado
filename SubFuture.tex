\documentclass[join.tex]{subfiles}
%\usepackage{estilo-ejercicios}%
%\setcounter{section}{0}
%\newtheorem{defin}{Definition}[section]
%\newtheorem{lem}[defin]{Lemma}
%\newtheorem{propo}[defin]{Proposition}
%\newtheorem{thm}[defin]{Theorem}
%\newtheorem{eje}[defin]{Example}


%\usepackage{calligra}
%\usepackage[T1]{fontenc}
%\usepackage{empheq}
%\newcommand*\widefbox[1]{\fbox{\hspace{2em}#1\hspace{2em}}}
%\usepackage{adjustbox}
%--------------------------------------------------------
\begin{document}





\section{Future research}

We finish by outlining some question that remain open after our research that would be interesting to investigate in the future. These questions arise naturally from the work done with (derived) $A_\infty$-algebras and from the classical results by Gerstenhaber and Voronov \cite{GV}.

\subsection{Hochschild Cohomology}
Let us review the Hochschild cohomology of an associative operad in order to understand the question that we will be asking about the (derived) $A_\infty$-case.

Let $\OO$ be an operad with an associative multiplication $m$, i.e. an $A_\infty$-multiplication $m$ REFERENCE) such that $m=m_2$. In this case, REFERENCE TO A-INFTY RESULT we have a dg-algebra structure on $S\s\OO$ given by the differential
\[d(Sx) =  Sb_1(m; x) -(-1)^{|x|}Sb_1(x; m)\]
and the multiplication
\[m(Sx,Sy) = Sb_2(m;x,y).\]

In particular, if $OO = \End_{SA}$ is the endomorphism operad of an associative algebra $A$ (such as $\s\PP$ for an operad $\PP$ with associative multiplication), these maps provide a dg-algebra structure on the Hochschild complex of $A$. We can then take cohomology with respect to $d$ to compute the Hochschild cohomology of $A$, denoted by $HH*(A)$. It is known that $m$ and the bracket 
\[[x,y]=Sb_1(x; y) -(-1)^{|x||y|}Sb_1(y; x)\]
induce a structure of a Gerstenhaber algebra on $HH*(A)$ \cite{GV}[Corollary 5].


DETAILS (ONLY IN THESIS, NOT IN PAPER)
\url{https://math.stackexchange.com/questions/3887857/show-that-this-map-is-well-defined-on-cohomology/3887939#3887939}%☺
I MAY ALSO NEED TO WRITE THE DERIVATION OF (11) FROM (9)
\url{https://arxiv.org/pdf/hep-th/9409063.pdf}

PROBLEMS IN THE A-INFTY CASE
\subsection{Morphism}
About how could $\Phi$ connected morphisms, it's not even clear what notion of morphism to use because if you have $O\to P$ and it is not clear how to get $P^n\to P$ from $O^n\to O$.
\end{document}
