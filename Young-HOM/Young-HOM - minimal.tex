\documentclass{beamer}
\usepackage[utf8]{inputenc}
\usetheme{Copenhagen}
%\usepackage[spanish]{babel}
\usepackage{multirow}
%\usepackage{estilo-apuntes}
\usepackage{braids}
\usepackage[]{graphicx}
\usepackage{rotating}
\usepackage{pgf,tikz}
\usepackage{pgfplots}
\usepackage{tikz-cd}
\usepackage{oplotsymbl} %filled pentagon go brrrr
%\usepackage{empheq}
%\usepackage[dvipsnames]{xcolor}
\usepackage{xcolor}

\usetikzlibrary{arrows}
\usetikzlibrary{cd}
\usetikzlibrary{babel}
\pgfplotsset{compat=1.13}
\usetikzlibrary{decorations.shapes}
%\pgfkeyssetvalue{/tikz/braid height}{1cm} %no parece hacer nada
%\pgfkeyssetvalue{/tikz/braid width}{1cm}
%\pgfkeyssetvalue{/tikz/braid start}{(0,0)}
%\pgfkeyssetvalue{/tikz/braid colour}{black}

\theoremstyle{definition}

\newtheorem{teorema}{Theorem}
\newtheorem{defi}{Definition}
\newtheorem{prop}[teorema]{Proposition}

\newcommand{\Z}{\mathbb{Z}}
\newcommand{\Q}{\mathbb{Q}}
\newcommand{\C}{\mathbb{C}}
\newcommand{\CC}{\mathcal{C}}
\newcommand{\D}{\mathbb{D}}
\providecommand{\gene}[1]{\langle{#1}\rangle}

\DeclareMathOperator{\im}{im}


\addtobeamertemplate{navigation symbols}{}{%
    \usebeamerfont{footline}%
    \usebeamercolor[fg]{footline}%
    \hspace{1em}%
    %\insertframenumber/\inserttotalframenumber
}
\setbeamercolor{footline}{fg=black}
\setbeamerfont{footline}{series=\bfseries}

\newcommand{\highlight}[1]{%
	\colorbox{red!50}{$\displaystyle#1$}}

\makeatletter
\newcommand*{\encircled}[1]{\relax\ifmmode\mathpalette\@encircled@math{#1}\else\@encircled{#1}\fi}
\newcommand*{\@encircled@math}[2]{\@encircled{$\m@th#1#2$}}
\newcommand*{\@encircled}[1]{%
	\tikz[baseline,anchor=base]{\node[draw,circle,outer sep=0pt,inner sep=.2ex] {#1};}}
\makeatother


%-----------------------------------------------------------

\title{}
\author{Javier Aguilar Mart\'in}
\institute{University of Kent}
\date{}
 
\begin{document}
\frame{\titlepage}


\begin{frame}
OPTION 1 (ONLY IF TIME PERMITS):
DELIGNE CONJECTURE: OVERVIEW OF TOPOLOGICAL VERSION, EMPHASIS ON THE PART WHERE ALGEBRAIC VERSIONS ARISE, AINFTY AND DERIVED AINFTY VERSIONS

OPTION 2:
INTRODUCTION TO AINFTY AND DAINFTY ALGEBRAS THROUGH OPERADS (KINDA EXTENDED ECRM)

START WITH COMMON THINGS: (OPERADS), ASSOCIAHEDRA, AINFTY ALGEBRAS, (BRACE ALGEBRAS), DERIVED AINFTY, MINIMALITY RESULTS


\end{frame}

\begin{frame}
PLAN OF THE TALK
\end{frame}




\section{Homotopy associativity}
\begin{frame}
%an example of A\infty-space
\frametitle{Loop spaces}


Let $(Y,*)$ a pointed topological space and $X = \Omega Y$ the spaces of based loops, i.e. maps $f:S^1\to Y$ such that $f(1,0)=*$.\pause %base point s a preferred point, like an origin

We have a concatenation map $m:X\times  X\to  X$, where $m(x,y)=x*y$ is given by\pause

\begin{tikzpicture}[line cap=round,line join=round,>=triangle 45,x=1.0cm,y=1.0cm]
\clip(-5,-3.) rectangle (5.,2.3);
\draw(0.,0.) circle (1.5cm);
\draw [->] (1.5,0.) -- (1.475763388700826,0.26855617768030476);
\draw [->] (-1.5,0.) -- (-1.4622984077406362,-0.33419061434935543);
\draw (-0.3,2.118952883889729) node[anchor=north west] {$x$};
\draw (-0.3,-1.5566913118092813) node[anchor=north west] {$y$};
\end{tikzpicture}
\end{frame}

\begin{frame}
\frametitle{Homotopy-associative product}
\begin{tikzpicture}[line cap=round,line join=round,>=triangle 45,x=1.0cm,y=1.0cm]
\clip(-4.175394430564892,-2.5911383046897085) rectangle (7.490400123879831,3.3976612960713135);
\draw(0.,2.) circle (1.cm);
\draw(0.,-1.) circle (1.cm);
\draw (-3.49428016933844,2.311720973491667) node[anchor=north west] {$(x*y)*z$};
\draw (-3.496796110195476,-0.6773028846978557) node[anchor=north west] {$x*(y*z)$};
\draw (-0.6057688157486263,1.1) node[anchor=north west] {$z$};
\draw (0.2964512548334696,0.4) node[anchor=north west] {$x$};
\draw (0.7431133461355,2.9675859207922457) node[anchor=north west] {$x$};
\draw (-1.3,3.0105934583201526) node[anchor=north west] {$y$};
\draw (-1.189603612937578,-1.4729423289641315) node[anchor=north west] {$y$};
\draw (0.7489686163804383,-1.4729423289641315) node[anchor=north west] {$z$};
\draw (2.,2.)-- (3.,1.);
\draw (3.,1.)-- (4.,2.);
\draw (3.,2.)-- (2.514225889472578,1.485774110527422);
\draw (2.,-1.)-- (3.,-2.);
\draw (3.,-2.)-- (4.,-1.);
\draw (3.,-1.)-- (3.481998691455241,-1.518001308544759);
\draw (1.7811495170502019,2.6342775049509677) node[anchor=north west] {$x$};
\draw (2.8240823021019423,2.623525620568991) node[anchor=north west] {$y$};
\draw (3.7487443589519387,2.5912699674230613) node[anchor=north west] {$z$};
\draw (1.7811495170502019,-0.4085057751484382) node[anchor=north west] {$x$};
\draw (2.8133304177199654,-0.4085057751484382) node[anchor=north west] {$y$};
\draw (3.813255665243799,-0.4300095439123916) node[anchor=north west] {$z$};
\draw [->] (1.,2.) -- (0.9776684310102762,2.2101533701987783);
\draw [->] (0.,3.) -- (-0.2228209821222102,2.9748593795651215);
\draw [->] (-1.,2.) -- (-0.9818938166599703,1.8105678675490438);
\draw [->] (1.,-1.) -- (0.9832067635335799,-0.8175049037869153);
\draw [->] (-1.,-1.) -- (-0.9827773938908082,-1.1847933820708718);
\draw [->] (0.,-2.) -- (0.1987984337916885,-1.9800403985152712);
\end{tikzpicture}
\end{frame}

\begin{frame}
The product $m$ is homotopy associative if $m(m\times 1)\simeq m(1\times m)$\pause 


\[\Downarrow\] 

\begin{center}
There is a map $M_3:[0,1]\times X^3\to  X$ such that 
\end{center}

\[M_3(0,x,y,z)=(xy)z \text{ and }M_3(1,x,y,z)=x(yz)\]
\end{frame}

\begin{frame}[fragile]
\frametitle{Homotopy coherence}
Product of 4 elements
\[
\begin{tikzpicture}[line cap=round,line join=round,>=triangle 45,x=1.0cm,y=1.0cm]
\clip(-1.5,0.5) rectangle (5.2,4.6);
\draw(1.,1.) -- (3.,1.) -- (3.618033988749895,2.9021130325903064) -- (2.,4.077683537175253) -- (0.3819660112501053,2.9021130325903073) -- cycle;
\draw (1.,1.)-- (3.,1.);
\draw (3.,1.)-- (3.618033988749895,2.9021130325903064);
\draw (3.618033988749895,2.9021130325903064)-- (2.,4.077683537175253);
\draw (2.,4.077683537175253)-- (0.3819660112501053,2.9021130325903073);
\draw (0.3819660112501053,2.9021130325903073)-- (1.,1.);
\draw (1.3,4.7) node[anchor=north west] {$x(y(zt))$};
\draw (3.6,3.25) node[anchor=north west] {$x((yz)t)$};
\draw (3,1.1) node[anchor=north west] {$(x(yz))t$};
\draw (-0.5,1.1) node[anchor=north west] {((xy)z)t};
\draw (-1.15,3.25) node[anchor=north west] {(xy)(zt)};
\draw (2.8,3.8) node[anchor=north west] {$\simeq$};
\draw (3.3333333333336,2.1) node[anchor=north west] {$\simeq$};
\draw (1.8,0.8933333333333304) node[anchor=north west] {$\simeq$};
\draw (0.15,2.1) node[anchor=north west] {$\simeq$};
\draw (0.6,3.8) node[anchor=north west] {$\simeq$};
\begin{scriptsize}
\draw [fill=black] (1.,1.) circle (2.5pt);
\draw [fill=black] (3.,1.) circle (2.5pt);
\draw [fill=black] (3.618033988749895,2.9021130325903064) circle (2.5pt);
\draw [fill=black] (2.,4.077683537175253) circle (2.5pt);
\draw [fill=black] (0.3819660112501053,2.9021130325903073) circle (2.5pt);
\end{scriptsize}
\end{tikzpicture}
\]

If we can fill the pengaton with a homotopy $M_4=\pentagofill\times X^4\to X$ we say that the product is homotopy coherent. %there is a concatenation of homotopies and it makes sense to talk about homotopies between them, joining the points with paths
\end{frame}
\section{Stasheff Associahedra}

\begin{frame}
\frametitle{Associahedra}
Multiplying 5 elements
\[
\begin{tikzpicture}[line cap=round,line join=round,>=triangle 45,x=1.0cm,y=1.0cm]
\clip(-3.63,-1.8) rectangle (4.,3);
\draw(-0.5,0.) -- (0.,0.5) -- (-0.5,1.) -- (-1.,0.5) -- cycle;
\draw (-0.5,0.)-- (0.,0.5);
\draw (0.,0.5)-- (-0.5,1.);
\draw (-0.5,1.)-- (-1.,0.5);
\draw (-1.,0.5)-- (-0.5,0.);
\draw (-0.5,1.)-- (-0.76,2.87);
\draw (-0.76,2.87)-- (-2.13,1.59);
\draw (-2.13,1.59)-- (-1.98,0.82);
\draw (-2.13,1.59)-- (-2.5,1.);
\draw (-2.5,1.)-- (-2.31,0.29);
\draw (-2.31,0.29)-- (-1.98,0.82);
\draw (-1.98,0.82)-- (-1.,0.5);
\draw (0.,0.5)-- (0.93,0.89);
\draw (0.93,0.89)-- (0.98,1.68);
\draw (1.37,1.06)-- (0.98,1.68);
\draw (-0.76,2.87)-- (0.98,1.68);
\draw (-2.31,0.29)-- (-0.62,-1.3);
\draw (-0.5,0.)-- (-0.62,-1.3);
\draw (-0.62,-1.3)-- (1.22,0.35);
\draw (0.93,0.89)-- (1.22,0.35);
\draw (1.22,0.35)-- (1.37,1.06);
\draw [dash pattern=on 2pt off 2pt] (-2.5,1.)-- (1.37,1.06);
\end{tikzpicture}
\]

\end{frame}
\section{$A_\infty$-spaces}
\begin{frame}
\begin{itemize}
\item<1-> We get spaces $K_2=*$, $K_3=[0,1]$, $K_4=\pentagofill$, $K_5, \dots$ %a point cause there is only one way to multiply two elements, [0,1] parametrizes the homotopy, k5 is the previous slide
\item<2-> And maps $M_n:K_n\times X^n\to X$ satisfying certain relations. %homotopy relation similar to what we explain with the polygons
\item<3-> For instance, $M_3:[0,1]\times X^3\to X$ defines a homotopy between $M_2(M_2\times 1)$ and $M_2(1\times M_2)$. 
\item<4-> $M_4:K_4\times X^4\to X$ allows us to fill the pentagon, on the boundary it is equal to $M_3$. %and so on
\item<5->[]\begin{defi}
If $M_n$ exists for all $n\geq 2$ we say that $X$ is an $A_\infty$-\textbf{space}.
\end{defi}
\end{itemize}
\end{frame}


%\begin{frame}
%\begin{prop}
%Let $K_0=\emptyset$ and $K_1=\{*\}$. Then the collection $\{K_n\}_{n\geq 0}$ is an operad.
%\end{prop}\pause

%\begin{exampleblock}{Proof}
%We need to define insertion maps $\circ_i:K_r\times K_s\to K_{r+s-1}$. For this, we define a bijection between $K_n$ and (planar rooted) trees of $n$ leaves.
%\end{exampleblock}
%%another option is using that the faces of K_{r+s-1} are products of K_r\times K_s, so on thee faces it is defined by this identification and on the innterior just take the cone
%\end{frame}

%\begin{frame}
%\begin{tikzpicture}[line cap=round,line join=round,>=triangle 45,x=1.0cm,y=1.0cm]
%\clip(-4.4,-3.7033333333333225) rectangle (7.44,6);
%\draw(3.5,0.) -- (5.5,0.) -- (6.118033988749895,1.9021130325903064) -- (4.5,3.077683537175253) -- (2.881966011250105,1.9021130325903073) -- cycle;
%\draw (4.5,1.)-- (4.5,1.5);
%\draw (4.5,1.5)-- (3.748888888888889,2.0111111111111115);
%\draw (4.5,1.5)-- (4.2377777777777785,2.002222222222223);
%\draw (4.5,1.5)-- (4.735555555555557,2.002222222222223);
%\draw (4.5,1.5)-- (5.171111111111112,1.9933333333333338);
%\draw (4.5,3.5)-- (4.5,4.);
%\draw (4.5,4.)-- (3.5,4.5);
%\draw (3.841262580054896,4.329368709972552)-- (4.,4.5);
%\draw (4.173977257874788,4.1630113710626055)-- (4.5,4.5);
%\draw (4.5,4.)-- (5.,4.5);
%\draw [line width=1.6pt] (3.5,0.)-- (5.5,0.);
%\draw [line width=1.6pt] (5.5,0.)-- (6.118033988749895,1.9021130325903064);
%\draw [line width=1.6pt] (6.118033988749895,1.9021130325903064)-- (4.5,3.077683537175253);
%\draw [line width=1.6pt] (4.5,3.077683537175253)-- (2.881966011250105,1.9021130325903073);
%\draw [line width=1.6pt] (2.881966011250105,1.9021130325903073)-- (3.5,0.);
%\draw (2.5,1.5)-- (2.5,2.);
%\draw (2.5,2.)-- (1.7044444444444466,2.464444444444445);
%\draw (2.5,2.)-- (3.242222222222227,2.4822222222222226);
%\draw (2.1032829629629672,2.2316029629629615)-- (2.291111111111114,2.4911111111111115);
%\draw (2.8780385185185238,2.2456118518518537)-- (2.691111111111115,2.4911111111111115);
%\draw (6.5,1.5)-- (6.5,2.);
%\draw (6.5,2.)-- (5.802222222222222,2.5088888888888894);
%\draw (6.5,2.)-- (7.26,2.5);
%\draw (4.,-0.5)-- (3.,-1.);
%\draw (3.5022222222222226,-0.7488888888888887)-- (3.402222222222227,-0.4866666666666656);
%\draw (3.196444444444449,-0.9017777777777756)-- (3.,-0.5);
%\draw (3.,-1.)-- (2.5,-0.5);
%\draw (3.,-1.)-- (3.,-1.5);
%\draw (6.214595643598731,2.2081452153372316)-- (6.5,2.5);
%\draw (6.5,2.5)-- (6.148888888888899,2.802222222222223);
%\draw (6.5,2.5)-- (6.824444444444455,2.802222222222223);
%\draw (5.135555555555563,-0.5044444444444434)-- (6.,-1.);
%\draw (6.,-1.)-- (6.806666666666677,-0.5044444444444434);
%\draw (6.40225777777779,-0.752882962962958)-- (6.264444444444454,-0.49555555555555447);
%\draw (6.264444444444454,-0.49555555555555447)-- (5.917777777777787,-0.29111111111111004);
%\draw (6.264444444444454,-0.49555555555555447)-- (6.5933333333333435,-0.26444444444444337);
%\draw (6.,-1.)-- (6.,-1.5);
%\draw (-1.9766666666666668,4.443888888888889)-- (-1.972222222222222,5.013888888888889);
%\draw (-1.972222222222222,5.013888888888889)-- (-2.2961111111111117,5.477222222222244);
%\draw (-1.972222222222222,5.013888888888889)-- (-1.573888888888889,5.546666666666689);
%\draw [line width=1.6pt] (-4.,0.)-- (0.,0.);
%\draw (-4.008888888888889,0.485)-- (-4.004444444444445,1.005);
%\draw (-4.004444444444445,1.005)-- (-4.504444444444446,1.518888888888903);
%\draw (-4.284449471464683,1.2927829444374739)-- (-4.004444444444446,1.505);
%\draw (-4.004444444444445,1.005)-- (-3.56,1.4077777777777916);
%\draw (-0.4627777777777772,1.505)-- (0.,1.);
%\draw (0.,1.)-- (-0.004444444444443562,0.4077777777777945);
%\draw (0.,1.)-- (0.5233333333333345,1.4772222222222409);
%\draw (0.20730493434689184,1.1890392129554)-- (-0.004444444444443562,1.4772222222222409);
%\draw (-1.9533333333333338,0.34888888888888897)-- (-1.9488888888888891,0.9911111111111113);
%\draw (-1.9488888888888891,0.9911111111111113)-- (-2.31,1.435555555555571);
%\draw (-1.9488888888888891,0.9911111111111113)-- (-1.935,1.4772222222222378);
%\draw (-1.9488888888888891,0.9911111111111113)-- (-1.5044444444444445,1.4494444444444599);
%\draw (-2.4488888888888893,3.824444444444468) node[anchor=north west] {$K_2$};
%\draw (-2.4211111111111117,-0.2866666666666504) node[anchor=north west] {$K_3$};
%\draw (4.037222222222225,-0.87) node[anchor=north west] {$K_4$};
%\begin{scriptsize}
%\draw [fill=black] (3.5,0.) circle (3.0pt);
%\draw [fill=black] (5.5,0.) circle (3.0pt);
%\draw [fill=black] (6.118033988749895,1.9021130325903064) circle (3.0pt);
%\draw [fill=black] (4.5,3.077683537175253) circle (3.0pt);
%\draw [fill=black] (2.881966011250105,1.9021130325903073) circle (3.0pt);
%\draw [fill=black] (-2.,4.) circle (3.0pt);
%\draw [fill=black] (-4.,0.) circle (3.0pt);
%\draw [fill=black] (0.,0.) circle (3.0pt);
%\end{scriptsize}
%\end{tikzpicture}
%\end{frame}
%\begin{frame}
%\begin{corollary}
%$A_\infty$-spaces are algebras over the operad $K=\{K_n\}$.
%\end{corollary}
%\end{frame}



\section{$A_\infty$-algebras}
\begin{frame}
\frametitle{Back to $A_\infty$-spaces}
\begin{itemize}
\item<1-> There is a map $C_*(X)\otimes C_*(Y)\to C_*(X\times Y)$ satisfying naturality and associativity axioms (Eilenberg-Zilber map).
\item<2-> The maps $M_n:K_n\times X^n\to X$ induce maps $C_*(K_n)\otimes C_*(X)^{\otimes n}\to C_*(K_n\times  X^n)\to C_*(X)$. %esentially the induced map on chains but with extra step


%\item<3-> The collection $C_*(K)=\{C_*(K_n)\}$ is an operad of chain complexes with insertion maps induced from those of $\{K_n\}$.
\item<4-> The relations that satisfy the map $M_n$ induce the relations of what we call an $A_\infty$-algebra.
\end{itemize}
\end{frame}

\begin{frame}
%\begin{itemize}
%\item<1->$C_*(X)$ becomes a $C_*(K)$-algebra.
%
%
%\item<2-> The relations that satisfy the map $M_n$ induce the relations of what we call an $A_\infty$-algebra.
%\end{itemize}
\end{frame}
\begin{frame}
\frametitle{$A_\infty$-algebras}
\begin{defi}
An $A_\infty$-\textbf{algebra} $A$ is a graded vector space equipped with a family of ``multiplications'' $m_n:A^{\otimes n}\to A$ of degree $n-2$ satisfying the relation %MAYBE CHANGE CHAINS TO COCHAINS TO KEEP THE DEGREE 2-N, I WILL HAVE TO USE OPERADIC DESUSPENSION IN THIS CASE

\[\sum_{r+s+t\geq 1}(-1)^{rs+t}m_{r+1+t}(1^{\otimes r}\otimes m_s\otimes 1^{\otimes s})=0\] %we are composing every map with itself
\end{defi}
\end{frame}





\begin{frame}
\frametitle{Some particular cases}
\begin{itemize}
\item<1-> We always have $m_1m_1=0$, so in particular $A$ is a chain complex.%CAN BE DEFINED ON THE CATEGORY OF CHAIN COMPLEX
\item<2-> If $m_i=0$ for $i\neq 2$, the relation becomes $m_2(1\otimes m_2)=m_2(m_2\otimes 1)$, so $A$ is an associative algebra.
\item<3->  We also have the relation \[m_1m_2=m_2(m_1\otimes 1)+m_2(1\otimes m_1)\]%DG %MONOID IN CHAIN COMPLEX ANALOGUE TO MONOID IN K-VECT
\item[]<4-> This is the Leibniz rule, and $A$ is a differential graded (dg) algebra.
\end{itemize}
\end{frame}


\begin{frame}
\frametitle{$A_\infty$-algebras are homotopy associative algebras.}
%how do they generalize associative algebras
\begin{itemize}
\item<1-> For $r+s+t=3$ we have the relation
\begin{align*}
&m_2(m_2\otimes 1)-m_2(1\otimes m_2)=\\ %the failure of m_2 to be associative
&m_1m_3+m_3(m_1\otimes 1\otimes 1)+m_3(1\otimes m_1\otimes 1)+m_3(1\otimes 1\otimes m_1)
\end{align*}
\item[]<2-> $m_2$ is homotopy associative with homotopy given by $m_3$. %recall that m1 is a differential so on homology this vanishes
\item<3-> The higher relations are a homotopy coherent extension of this fact. %m3 satisfies some relation up to homotopy given by m4 and so on
\end{itemize}
\end{frame}


\begin{frame}
\frametitle{Morphisms of $A_\infty$-algebras}
\begin{defi}
An \emph{$\infty$-morphism} of $A_\infty$-algebras $A\to B$ is a family of maps \[f_n:A^{\otimes n}\to B\] of degree $1-n$ satisfying for all $n\geq 1$ the equation
\[\sum_{r+s+t=n} (-1)^{rs+t}f_{r+1+t}(1^{\otimes r} \otimes m^A_s\otimes 1^{\otimes t})=\sum_{i_1+\cdots+i_k=n} (-1)^s m^B_k(f_{i_1}\otimes\cdots\otimes f_{i_k}),\]
where

\[s=\sum_{\alpha<\beta}i_\alpha(1-i_\beta).\]
%The composition of $\infty$-morphisms $f:A\to B$ and  $g:B\to C$ is given by 
%
%\[(gf)_n=\sum_r\sum_{i_1+\cdots+i_r=n}(-1)^s g_r(f_{i_1}\otimes\cdots
%\otimes f_{i_r}).\]
\end{defi}

\end{frame}
\begin{frame}
\begin{itemize}
\item<1-> We have $f_1m_1 = m_1f_1$, i.e. $f_1$ is a morphism of complexes.
\item<2-> We have
\[
f_1m_2 = m_2 (f_1\otimes f_1) + m_1f_2 + f_2 (m_1\otimes 1 + 1\otimes m_1),\]
which means that $f_1$ commutes with the multiplication $m_2$ up to a homotopy
given by $f_2$.
\end{itemize}
\end{frame}

\section{Minimal models}

%\begin{frame}
%\frametitle{Operad of $A_\infty$-algebras}
%%FORMULATE THIS DIFFERENTLY SINCE I HAVE ALREADY OBTAINED THE OPERAD OF AINFTY ALGEBRAS THIS SHOULD BE JUST WRITING IT DOWN EXPLICITLY AND THEN ENCODE IT WITH OPERADIC SUSPENSION (PROBABLY JUST TELL THE RESULT OF INSERTIONS AND DEGREE BECAUSE THERE IS NO TIME TO EXPLAIN IT ALL)
%\begin{itemize}
%\item<1-> The operad $C_*(K)$ is generated by $m_n$ with $m_n\in C_*(K_n)$ for $n\geq 2$ and $m_1=\partial$ such that 
%\[\sum_{r+s+t\geq 1}(-1)^{rs+t}m_{r+1+t}\circ_{r+1}m_s=0\]
%
%\item<2-> We would like to obtain the signs directly from operadic composition. PROBABLY NOT GOING TO INCLUDE IT
%\end{itemize}
%\end{frame}


\begin{frame}
\frametitle{Uniqueness result}
\begin{itemize}
\item An $A_\infty$-algebra is \textbf{minimal} if $\partial = 0$. 
\end{itemize}\pause
\begin{theorem}[Kadeishvili]
\begin{itemize}
\item If $A$ is a dga over a field, its homology $H^*(A)$ is a minimal $A_\infty$-algebra with multiplication $m_2$ induced by the multiplication on $A$.
\item There is a morphism of $A_\infty$-algebras $f:H^*(A)\to A$ such that $f_1$ is a quasi-isomorphism.
\item Under certain homological conditions, any other dga $A'$ with $H^*(A')\cong H^*(A)$ is quasi-isomorphic to $A$. 
\end{itemize}
\end{theorem}\pause
The $A_\infty$-algebra $H^*(A)$ is called the \emph{minimal model} of $A$.
%nice enough = HH(A,A) vanishes on degree 2-n %Replaced  means equivalent  %essentially = up to quasi-iso
We would like to extend this result to a ground ring $k$ that is not necessarily a field.
\end{frame}

\section{Derived $A_\infty$-algebras}

\begin{frame}
\frametitle{Derived $A_\infty$-algebras}
\begin{defi}
  A \emph{derived $A_\infty$-algebra} on a $(\Z,\Z)$-bigraded $R$-module $A$ consist of a family of $R$-linear maps 
\[m_{ij}:A^{\otimes j}\to A\]
of bidegree $(i,2-(i+j))$ for each $j\geq 1$, $i\geq 0$, satisfying the equation
\begin{equation}
\underset{j=r+1+t}{\sum_{u=i+p, v=j+q-1}}(-1)^{rq+t+pj}m_{ij}(1^{\otimes r}\otimes m_{pq}\otimes 1^{\otimes t})=0
\end{equation}
for all $u\geq 0$ and $v\geq 1$. 
\end{defi}
\end{frame}

\begin{frame}
\frametitle{Particular cases}
\begin{itemize}
\item<1-> A $dA_\infty$-algebra where $m_{ij}=0$ for all $i>0$ is an $A_\infty$-algebra.
\item<2-> A $dA_\infty$-algebra with $m_{ij}=0$ except for for $m_{01}$ and $m_{11}$ is a \emph{bicomplex}: 
\[m_{10}m_{01}=0,\ m_{11}m_{11}=0,\ m_{01}m_{11}=m_{11}m_{01}\]
\item<3-> A \emph{bidga} is a monoid in the category of bicomplexes, equivalently, a $dA_\infty$-algebra with $m_{ij}=0$ for $i+j\geq 3$.
\end{itemize}
\end{frame}
\end{document}