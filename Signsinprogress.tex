	\documentclass[twoside]{article}
\usepackage{estilo-ejercicios}
\setcounter{section}{0}
\newtheorem{defin}{Definition}[section]
\newtheorem{lem}[defin]{Lemma}
\newtheorem{propo}[defin]{Proposition}
\newtheorem{thm}[defin]{Theorem}
\newtheorem{eje}[defin]{Example}
\newtheorem{obs}[defin]{Observación}
\renewcommand{\baselinestretch}{1,3}
%--------------------------------------------------------
\begin{document}

\title{Fucking signs}
\author{Javier Aguilar Martín}
\maketitle

%\varprojlim
%\varinjlim
\section{Possible MSE question}

I'm reading [this paper](https://arxiv.org/pdf/hep-th/9409063.pdf) (page 2) and I wanted to check an identity that the author says it's immediate. 

**Definitions**

I'm going to stick to operad notation but I'm interested in the standard example where $O(n)=Hom(V^{\otimes n},V)$ and for $f\in O(n)$ we write $|f|=n-1$. In this context, 

$\gamma(f;1,\dots,1,g_1,1,\dots, 1, g_m,1,\dots, 1)=f(1^{\otimes k_0}\otimes g_1\otimes 1^{\otimes k_1}\otimes\cdots\otimes g_m\otimes 1^{\otimes k_m})$

and we define the brace $f\{g_1,\dots,g_m\}$ as the following sum over all ordering preserving insertions of the $g_i$'s

$\sum (-1)^\epsilon \gamma(f;1,\dots,1,g_1,1,\dots, 1, g_m,1,\dots, 1)$

where $\epsilon=\sum_{p=1}^m|g_p|i_p$, $i_p$ being the number of $1$'s in front of $g_p$.

**Problem**

So I want to check that equation (2) in the paper holds. My problem is only on the signs, so I basically want to know how to simplify an expession of the form

$\sum (-1)^\epsilon\sum (-1)^\delta$

to get only one sum with a sign, say $\mu$, depending on $\delta$ and $\epsilon$. In this case, looking at the lhs of equation (2), we have $x\{x_1,\dots, x_m\}\{y_1,\dots, y_n\}$, so I would compute $x\{x_1,\dots, x_m\}$ first, which by definition is

$\sum (-1)^\epsilon \gamma(x;1,\dots,1,x_1,1,\dots, 1, x_m,1,\dots, 1)$

Then $x\{x_1,\dots, x_m\}\{y_1,\dots, y_n\}$ is, applying linearity, 

$\sum (-1)^\epsilon \sum(-1)^\delta\gamma(\gamma(x;1,\dots,1,x_1,1,\dots, 1, x_m,1,\dots, 1); 1,\dots, 1,y_1,1,\dots,1,y_n,1,\dots,1)$

where $\delta$ is defined analogous to $\epsilon$ but in this case we're counting the number of $1$'s in front on the $y_i$'s (after the inner $\gamma$, so the $1$'s inside that $\gamma$ don't count).

This composition can be rewritten using the associativity property of operads (see for instance *The geometry of iterated loops spaces* by P.May), so that it becomes something of the form

$\gamma(x;\gamma(\cdots),\dots, \gamma(\dots))$

This is really hard to write more clearly, since some $y_i$'s would be in place of some $1$'s (something like $\gamma(1;y_i)=y_i$ and others would be inside some $x_j$ (for example $\gamma(x_j;1,\dots, y_{i_1}, 1,\dots, 1,y_{i_j},1\dots,1)$)

IT IS POSSIBLE THAT I END UP GETTING THE BRACES INSIDE THANKS TO THE SUM OUTSIDE, SINCE INSIDE WHAT I HAVE IS OPERADIC COMPOSITION, MAYBE THAT WAY I CAN GET RID OF SOME OF THE SIGNS

TO BE ABLE TO WRITE THINGS PROPERLY TRY USING A NOTATION FOR EVERY INPUT INCLUDING THE ONES, SINCE THEIR ARITY MINUS ONE IS ZERO THAT'S NOT GONNA AFFECT THE FINAL RESULT. FOR THIS YOU SHOULD PROBABLY REWRITE THE ORIGINALS SIGNS TOO.

IT WOULDN'T BE SURPRISING THAT I HAD TO TAKE INTO ACCOUNT THE ARITY OF X SINCE DEPENDING ON IT THE NUMBER OF GAPS VARIES, IN PARTICULAR, THE NUMBER OF ONES IN FRONT OF SOME YI INSIDE XJ DEPENDS ON HOW MANY ARE OUTSIDE, WHICH DEPENDS ON THE ARITY OF X. MAYBE THAT'S WHERE THE DEGREE OF F IS GOING TO COME

CONSIDER ALSO THE POSSIBILITY OF DOING EVERYTHING WITHOUT SIGNS (THE ENDOMORPHISM OPERAD USUALLY DOESN'T HAVE SIGNS) TO SEE IF YOU GET EVERYTHING WITHOY SIGNS, SO THAT YOU COULD GET THE SIGNS BY SHIFTING. FOR THIS, ASK WHAT WAS THE BRACKET ``WITHOUT SIGNS" AND HOW I WOULD GET A SIGN RELATED TO F WHEN IT IS NOT SWAPPED
\end{document}
