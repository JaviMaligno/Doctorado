	\documentclass[twoside]{article}
\usepackage{estilo-ejercicios}
\setcounter{section}{0}
\newtheorem{defin}{Definition}[section]
\newtheorem{lem}[defin]{Lemma}
\newtheorem{propo}[defin]{Proposition}
\newtheorem{thm}[defin]{Theorem}
\newtheorem{eje}[defin]{Example}
\newtheorem{obs}[defin]{Observación}
\renewcommand{\baselinestretch}{1,3}
%--------------------------------------------------------
\begin{document}

\title{Fucking signs}
\author{Javier Aguilar Martín}
\maketitle

%\varprojlim
%\varinjlim

\section{Signs to try}
\begin{itemize}
\item No signs: this does not satisfy the defining equation of brace algebra since no sign at all would appear
\item With the notation of the draft of the derived Deligne conjecture, and $N=k_0+\cdots+k_n+n$, $g_0=f$:
\[
\varepsilon=\sum_{j<i}(a_i-1)(k_j+a_j-1)+\sum_{1\leq i}(N-i)q_i+\sum_{i\leq j<i}q_ia_j
\]
In the first subindex there are extra constrains $0\leq j<i\leq N$ but bah, $i$ can't always reach $N$, the same for the other indexes. With this sign we get that $f\{g\}$ is the first half of $[f,g]$ according to R-W.

\item Try to come up with a sign that makes sense using the Koszul Rule. Problem with this: the degree of $f$ shouldn't appear only applying koszul, but it is in the bracket. Maybe after applying shifts or something that takes into account the arity and degree of $f$, as the $N$ in the formula above.
\end{itemize}
\section{Possible MSE question}

I'm reading [this paper](https://arxiv.org/pdf/hep-th/9409063.pdf) (page 2) and I wanted to check an identity that the author says it's immediate. 

**Definitions**

I'm going to stick to operad notation but I'm interested in the standard example where $O(n)=Hom(V^{\otimes n},V)$ and for $f\in O(n)$ we write $|f|=n-1$. In this context, 

$\gamma(f;1,\dots,1,g_1,1,\dots, 1, g_m,1,\dots, 1)=f(1^{\otimes k_0}\otimes g_1\otimes 1^{\otimes k_1}\otimes\cdots\otimes g_m\otimes 1^{\otimes k_m})$

and we define the brace $f\{g_1,\dots,g_m\}$ as the following sum over all ordering preserving insertions of the $g_i$'s

$\sum (-1)^\epsilon \gamma(f;1,\dots,1,g_1,1,\dots, 1, g_m,1,\dots, 1)$

where $\epsilon=\sum_{p=1}^m|g_p|i_p$, $i_p$ being the number of $1$'s in front of $g_p$.

**Problem**

So I want to check that equation (2) in the paper holds. My problem is only on the signs, so I basically want to know how to simplify an expession of the form

$\sum (-1)^\epsilon\sum (-1)^\delta$

to get only one sum with a sign, say $\mu$, depending on $\delta$ and $\epsilon$. In this case, looking at the lhs of equation (2), we have $x\{x_1,\dots, x_m\}\{y_1,\dots, y_n\}$, so I would compute $x\{x_1,\dots, x_m\}$ first, which by definition is

$\sum (-1)^\epsilon \gamma(x;1,\dots,1,x_1,1,\dots, 1, x_m,1,\dots, 1)$

Then $x\{x_1,\dots, x_m\}\{y_1,\dots, y_n\}$ is, applying linearity, 

$\sum (-1)^\epsilon \sum(-1)^\delta\gamma(\gamma(x;1,\dots,1,x_1,1,\dots, 1, x_m,1,\dots, 1); 1,\dots, 1,y_1,1,\dots,1,y_n,1,\dots,1)$

where $\delta$ is defined analogous to $\epsilon$ but in this case we're counting the number of $1$'s in front on the $y_i$'s (after the inner $\gamma$, so the $1$'s inside that $\gamma$ don't count).

This composition can be rewritten using the associativity property of operads (see for instance *The geometry of iterated loops spaces* by P.May), so that it becomes something of the form

$\gamma(x;\gamma(\cdots),\dots, \gamma(\dots))$

This is really hard to write more clearly, since some $y_i$'s would be in place of some $1$'s (something like $\gamma(1;y_i)=y_i$ and others would be inside some $x_j$ (for example $\gamma(x_l;1,\dots, y_{i_l+1}, 1,\dots, 1,y_{j_l},1\dots,1)$, using the notation of G-V).

So, say we have an operadic composition like that, which is afected by $\delta$ for each insertion of the $y$'s. Notice that before that $\gamma$ there is a number of $1$'s and som $y_{i_l}$. I want to introduce some signs so that it becomes a brace. I'm going to change the notation so that $i_p$ (the number of $1$'s in front of $x_p$) becomes $u^x_p$ and for the $1$'s in front of $y_q$ I'll use $u^y_q$. This way, I have to introduce in front of the operadic composition the sign prescribed by the number of $1$'s in front of each $y$ but inside the composition. So let $i_l+1\leq r\leq j_l$. For $y_r$ we count $u^y_{i_l+1}-u^y_{i_l}-(u^x_l-u^y_{i_l})=u^y_{i_l+1}-u^x_l$ (yes, I could've come up with the final difference without doint the previous steps because I am just substracting the ones in fron of the $x$). So the sign in front of this particular composition (there should be one analogue for each composition inside the big composition and after that the sign for the big composition to be a brace)
$$\sum_{i_l+1\leq r\leq j_l}|y_r|(u^y_r-u^x_l)$$

If I want to substract this sign from $\delta$ I'd better do it by counting the ones left rather than to the sum. For every $y_r$ inside $x_l$ the amount of $1$'s that are left in $\delta$ are those in front of $x_l$, i.e., $u^x_l$. So in $\delta$, for those $y_r$ inside $x_l$ we replace $u^y_r$ by $u^x_l$. Now let's see what happens after doing this for every $x_l$. We have the following sum 

$$\delta_1=\sum_{l=1}^m\sum_{i_l+1\leq r\leq j_l}|y_r|u^x_l$$

But we still have the $y$'s that are outside every $x_l$. Using the notation of G-V, where $j_0=0$ and $i_{m+1}=n$, we have also the sum

$$\delta_2=\sum_{l=1}^{m+1}\sum_{j_{l-1}+1\leq r\leq i_l}|y_r|u^y_r$$

So let us rename $\delta=\delta_1+\delta_2$.

Now for the external brace I need the $1$'s in front of the $y$'s that are outside, but only the $1$'s outside the $x_l$. I would also need the ones in fron of the $x_l$ which are in $\epsilon$, but at this point I'm not sure how to introduce it. I think the sums can be interchanged since, as can be seen in the operadic composition, the position of the $y$'s are independent of the position of the $x$'s. But that was at the beginning, now I've done things that depend on how the $y$'s are inserted in the $x$'s. So maybe I should do that in the opposite order, since the sign of the external brace requires the whole $\epsilon$ and only part of $\delta$, which might be introduced later by distributivity I guess.

IT IS POSSIBLE THAT I END UP GETTING THE BRACES INSIDE THANKS TO THE SUM OUTSIDE, SINCE INSIDE WHAT I HAVE IS OPERADIC COMPOSITION, MAYBE THAT WAY I CAN GET RID OF SOME OF THE SIGNS. I ALSO NEED TO TAKE SOMES SIGNS OUTSIDE TO GET A BIG BRACE OUTSIDE.

TO COUNT THE ONES INSIDE XI IN FRONT OF YJ (ALREADY ASSUMING I'M COUNTING ONE YJ THAT IS INSIDE) I HAVE TO SUBSTRACT TO THE NUMBER OF ARGUMENTS (NOT ONLY ONES) THE NUMBER OF YS IN FRONT OF YJ TOGETHER WITH THE ONES

TO BE ABLE TO WRITE THINGS PROPERLY TRY USING A NOTATION FOR EVERY INPUT INCLUDING THE ONES, SINCE THEIR ARITY MINUS ONE IS ZERO THAT'S NOT GONNA AFFECT THE FINAL RESULT. FOR THIS YOU SHOULD PROBABLY REWRITE THE ORIGINALS SIGNS TOO.

IT WOULDN'T BE SURPRISING THAT I HAD TO TAKE INTO ACCOUNT THE ARITY OF X SINCE DEPENDING ON IT THE NUMBER OF GAPS VARIES, IN PARTICULAR, THE NUMBER OF ONES IN FRONT OF SOME YI INSIDE XJ DEPENDS ON HOW MANY ARE OUTSIDE, WHICH DEPENDS ON THE ARITY OF X. MAYBE THAT'S WHERE THE DEGREE OF F IS GOING TO COME

CONSIDER ALSO THE POSSIBILITY OF DOING EVERYTHING WITHOUT SIGNS (THE ENDOMORPHISM OPERAD USUALLY DOESN'T HAVE SIGNS) TO SEE IF YOU GET EVERYTHING WITHOUT SIGNS, SO THAT YOU COULD GET THE SIGNS BY SHIFTING. FOR THIS, ASK WHAT WAS THE BRACKET ``WITHOUT SIGNS" AND HOW I WOULD GET A SIGN RELATED TO F WHEN IT IS NOT SWAPPED

\section{Sign with consistent explanation}

Let's use the same strategy as R-W for the signs of the bracket $[f,g]$. Let $V$ be a graded vector space and $f\in C^{N,i}(V,V)=\hom(C^{\otimes N},V)^i$. Let $S(V)$ the graded vector space with $S(V)^v=V^{v+1}$, and so the suspension map $S:V\to S(V)$ given by the identity map has internal degree $-1$. Define $\sigma(f)$ as the map making the following diagram commutative
\[
\begin{tikzcd}
S(V)^{\otimes N}\arrow[r, "\sigma(f)"]\arrow[d, "(S^{-1})^{\otimes N}"'] & S(V)\\
V^{\otimes N}\arrow[r,"f"] & V\arrow[u, "S"']
\end{tikzcd}
\]

Explicitly, $\sigma(f)=S\circ f\circ (S^{-1})^{\otimes N}\in C^{N,i+N-1}(V,V)$. 

In R-W there is a sign in front of $f$ but it seems to be irrelevant for this purpose. 

Notice that, by the Koszul sign rule $(S^{-1})^{\otimes N}\circ S^{\otimes N}=(-1)^{\sum_{j=1}^N j}1=(-1)^{\frac{N(N-1)}{2}}1=(-1)^{\binom{N}{2}}1$, so $(S^{-1})^{\otimes N}= (-1)^{\binom{N}{2}}(S^{\otimes N})^{-1}$. For this reason, given $F\in C^{m,j}(S(V),S(V))$, we have
\[
\sigma^{-1}(F)=(-1)^{\binom{m}{2}}S^{-1}\circ F\circ S^{\otimes m}\in C^{m,j-m+1}(V,V).
\]

For $g_j\in C^{a_j,q_j}(V,V)$, let $$f[g_1,\dots, g_n]=\sum_{k_0+\cdots+k_n=N-n}f(1^{\otimes k_0}\otimes g_1\otimes 1^{\otimes k_1}\otimes\cdots\otimes g_n\otimes 1^{\otimes k_n})\in C^{N-n+\sum a_j, i+\sum q_j}(V,V).$$

We define $f\{g_1,\dots, g_n\}=\sigma^{-1}(\sigma(f)[\sigma(g_1),\dots, \sigma(g_n)])\in C^{N-n+\sum a_j, i+N-1+\sum (q_j+a_j-1)}(V,V).$

This is the same as $f[g_1,dots, g_n]$ up to sign, so let us compute the sign.
\[
\sigma^{-1}(\sigma(f)[\sigma(g_1),\dots, \sigma(g_n)])=(-1)^{\binom{N-n+\sum a_j}{2}}S^{-1}\circ (\sigma(f)(1^{\otimes k_0}\otimes g_1\otimes 1^{\otimes k_1}\otimes\cdots\otimes g_n\otimes 1^{\otimes k_n}))\circ S^{\otimes N-n+\sum a_j}
\]
\[
=(-1)^{\binom{N-n+\sum a_j}{2}}S^{-1}\circ S\circ f\circ (S^{-1})^{\otimes N}\circ (1^{\otimes k_0}\otimes (S\circ g_1\circ (S^{-1})^{\otimes a_1})\otimes 1^{\otimes k_1}\otimes\cdots\otimes (S\circ g_n\circ (S^{-1})^{\otimes a_n})\otimes 1^{\otimes k_n}))\circ S^{\otimes N-n+\sum a_j}
\]
\[
=(-1)^{\binom{N-n+\sum a_j}{2}}f\circ (S^{-1})^{k_0}\otimes  S^{-1}\otimes\cdots \otimes  S^{-1}\otimes  (S^{-1})^{k_n})\circ(1^{\otimes k_0}\otimes (S\circ g_1\circ (S^{-1})^{\otimes a_1})\otimes\cdots\otimes (S\circ g_n\circ (S^{-1})^{\otimes a_n})\otimes 1^{\otimes k_n}))\circ S^{\otimes N-n+\sum a_j}
\]



Now we move each $1^{\otimes k_{j-1}}\otimes (S\circ g_j\circ (S^{-1})^{a_j}$ to apply $(S^{-1})^{k_{j-1}}\otimes S^{-1}$ to it. Doing this to all of them produces a sign


\section{With inverse of the tensor to avoid some binomial coefficients}
Now define $\sigma(f)=S\circ f\circ (S^{\otimes N})^{-1}$. Now the inverse is $\sigma^{-1}(F)=S^{-1}\circ F\circ S^{\otimes m}$.
\end{document}
