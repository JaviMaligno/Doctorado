	%\documentclass[twoside]{book}
	\documentclass[Thesis.tex]{subfiles}
%\usepackage{estilo-ejercicios}
%\setcounter{section}{0}
%\usepackage{adjustbox}
%\newtheorem{defin}{Definition}[section]
%\newtheorem{lem}[defin]{Lemma}
%\newtheorem{propo}[defin]{Proposition}
%\newtheorem{thm}[defin]{Theorem}
%\newtheorem{eje}[defin]{Example}
%\renewcommand{\baselinestretch}{1,3}
%
%\usepackage{empheq}
%\newcommand*\widefbox[1]{\fbox{\hspace{2em}#1\hspace{2em}}}
%
%%Below to introduce ¡ in mathmode https://tex.stackexchange.com/questions/471464/inverted-exclamation-mark-in-mathmode
%\DeclareMathSymbol{\mathinvertedexclamationmark}{\mathclose}{operators}{'074}
%\DeclareMathSymbol{\mathexclamationmark}{\mathclose}{operators}{'041}
%
%\makeatletter
%\newcommand{\raisedmathinvertedexclamationmark}{%
%  \mathclose{\mathpalette\raised@mathinvertedexclamationmark\relax}%
%}
%\newcommand{\raised@mathinvertedexclamationmark}[2]{%
%  \raisebox{\depth}{$\m@th#1\mathinvertedexclamationmark$}%
%}
%\begingroup\lccode`~=`! \lowercase{\endgroup
%  \def~}{\@ifnextchar`{\raisedmathinvertedexclamationmark\@gobble}{\mathexclamationmark}}
%\mathcode`!="8000
%\makeatother

%--------------------------------------------------------
\begin{document}

%\title{Introductory chapter}
%\author{Javier Aguilar Martín}
%\maketitle
\doublespacing
\sloppy
\chapter{Background and conventions}\label{Sec1}

In this initial chapter we establish the necessary background and conventions for the rest of the thesis. We start \Cref{back1} with some category theory background and results, including notions of enriched categories that will play an essential role connecting derived $A_\infty$-algebras with classical $A_\infty$-algebras. We recall the motivation  for the study of $A_\infty$-algebras as well as some definitions and well-known results in \Cref{back2}. In \Cref{back3} we recall the main definitions regarding operads, since that is the framework in which we will work with derived $A_\infty$-algebras. At last, in \Cref{categories} we list several categories that we will use in our study and introduce the totalization functor, which is essential to encode derived $A_\infty$-algebras.

%MAYBE THIS IS THE PLACE FOR FIXING R COMMUTATIVE WITH UNIT AND CHAR NOT 2 (POSSIBLY 0 OR NOT A FACTORIAL (representation theory things that I doubt I would need to clarify))
%\begin{itemize}
%\item MAYBE SOMETHING ABOUT GRADED / BIGRADED MODULES?

%\item MAYBE BASIC REVIEW OF CATEGORY THEORY INCLUDING EQUIVALENCE OF CATEGORIES

%FIND AN ACTUAL REFERENCE, I'M USING \url{https://math.ucr.edu/home/baez/qg-fall2004/definitions.pdf} BECAUSE IT'S PRACTICAL

%FILL WITH SOME TEXT INTRODUCING WHAT WE ARE GOING TO DISCUSS
%\subsection{Categories, Functors, Natural transformations}
%IS THIS TOO ELEMENTARY TO BE INCLUDED?
%
%THE ITEMS LOOK WEIRD BECAUSE AT THE MOMENT THHEY ARE INSIDE ANOTHER ITEMIZE, BUT THEY  WON'T BE IN THE END
%\begin{defin}
%A \emph{category} $\CC$ consist of
%\begin{itemize}
%\item a collection $\ob(\CC)$ of \emph{objects},
%CHECK NOTATION FOR HOM SET TO BE CONSISTENT
%\item for any pair of objects $x,y\in\ob(\CC)$, a set $\hom(x,y)$ of \emph{morphisms} (If $f\in\hom(x,y)$, we write $f:x\to y $)
%\end{itemize}
%equipped with
%CHECK NOTATION FOR IDENTITY MORPHISM
%\begin{itemize}
%\item an \emph{identity morphism} $1_x: x → x$ for any object $x$,
%\item for any pair of morphisms $f: x → y$ and $g: y → z$, a morphism $fg: x → z$ called the \emph{composite} or \emph{composition}
%of $f$ and $g$, sometimes written as $f\circ g$,
%\end{itemize}
%such that
%\begin{itemize}
%\item for any morphism $f: x → y$, the \emph{left and right unit laws} hold: $1_xf = f = f1_y$,
%\item for any triple of morphisms $f: w → x$, $g: x → y$, $h: y → z$, the \emph{associative law} holds:
%$(fg)h = f(gh)$.
%\end{itemize}
%\end{defin}
%We usually write $x ∈ \CC$ as an abbreviation for $x ∈ \ob(\CC)$. 
%
%\begin{defin}
%An \emph{isomorphism} is a morphism $f: x → y$
%with an \emph{inverse}, i.e. a morphism $g: y → x$ such that $fg = 1_x$ and $gf = 1_y$.
%\end{defin}
%
%
%EXAMPLES
%
%
%\begin{defin}
%Given categories $\CC$, $\DD$, a \emph{functor} $F: \CC → \DD$ consists of
%\begin{itemize}
%\item a function $F: \ob(\CC) → \ob(\DD)$,
%\item for any pair of objects $x, y ∈ \ob(\CC)$, a function $F: \hom(x, y) → \hom(F(x), F(y))$
%\end{itemize}
%such that
%\begin{itemize}
%\item $F$ preserves identities: for any object $x ∈ \ob(\CC)$, $F(1_x) = 1_{F (x)}$,
%\item $F$ preserves composition: for any pair of morphisms $f: x → y$, $g: y → z$ in $\CC$, $F(fg) =
%F(f)F(g)$.
%\end{itemize}
%\end{defin}
%
%IDENTITY FUNCTOR, COMPOSITION OF FUNCTORS (CHECK IDENTITY AND ASSOCIATIVE LAW OR SAY IT IS NOT HARD), EXAMPLES
%
%\begin{defin}
%Given functors $F, G: \CC → \DD$, a \emph{natural transformation} $α: F ⇒ G$ consists of a function $α$ mapping each object $x ∈ \CC$ to a morphism $α_x: F(x) → G(x)$ such that for any morphism $f: x → y$ in $\CC$, the following diagram commutes
%\[
%\begin{tikzcd}
%F(x)\arrow[r, "F(f)"]\arrow[d, "\alpha_x"] & F(y)\arrow[d, "\alpha_y"]\\
%G(x)\arrow[r, "G(f)"] & G(y)
%\end{tikzcd}
%\]
%\end{defin}
%
%IDENTITY AND COMPOSITION OF NATURAL TRANSFORMATION, IDENTITY AND ASSOCIATIVE LAW
%
%
%\begin{defin}
%Given functors $F, G: \CC → \DD$, a \emph{natural isomorphism} $α: F ⇒ G$ is a natural
%transformation that has an \emph{inverse}, i.e. a natural transformation $β: G ⇒ F$ such that $αβ = 1_F$ and
%$βα = 1_G$.
%\end{defin}
%
%\begin{lem}
%A natural transformation $α: F ⇒ G$ is a natural isomorphism if and only if for every
%object $x ∈ \CC$, the morphism $α_x$ is invertible.
%\end{lem}
%\begin{proof}
%PROVE OR FIND REFERENCE
%\end{proof}
%
%\begin{defin}
%A functor $F: \CC → \DD$ is an equivalence if it has a \emph{weak inverse}, that is, a functor
%$G: \DD → \CC$ such that there exist natural isomorphisms $α: FG ⇒ 1_\CC$, $β: GF ⇒ 1_\DD$.
%\end{defin}
%
%\begin{lem}
%EQUIVALENCE IN TERMS OF FULLY FAITHFUL (FOR THAT I WILL NEED THE DEFINITIONS  AS WELL)
%\end{lem}
%
%
%\item SYMMETRIC (CLOSED) MONOIDAL CATEGORIES, EXAMPLES
\section{Symmetric monoidal categories and enrichments}\label{back1}

We assume that the reader is familiar with the basic terminology of category theory. For an introduction to this topic we refer the reader to \cite{maclane}. Here we briefly recall the notion of symmetric monoidal categories and several versions of monoidal functors. The detailed definitions with all the precise diagrams can also be found in \cite{maclane} and in \cite{borceux}.

\begin{defin}
A \emph{symmetric monoidal category} is a category $\CC$ equipped with a functor 
\[\otimes:\CC\times\CC\to\CC\]
called \emph{tensor product}, an object $1\in\CC$
called \emph{unit object}, natural isomorphisms called \emph{associators}
\[a_{A,B,C} : (A \otimes B) \otimes C \to A \otimes (B \otimes C)\]
for all objects $A,B,C\in\CC$, a natural isomorphism called \emph{left unitor}
\[\lambda_A:1\otimes A\to A\]
for every $A\in\CC$, a natural isomorphism  called \emph{right unitor}
\[\rho_A:A\otimes 1\to A\]
for every $A\in\CC$, and a natural isomorphism called \emph{braiding} or \emph{symmetry isomorphism}
\[\tau_{A,B}:A\otimes B \to B\otimes A\]
for all $A,B\in\CC$. These morphisms satisfy natural unitality and associativity axioms.
\end{defin}

\begin{remark}
If we drop the symmetry isomorphism we get what is simply called a \emph{monoidal category}.
\end{remark}

\begin{defin}
Let $(\CC,\otimes_\CC, 1_\CC)$ and $(\DD, \otimes_\DD, 1_\DD)$ be symmetric monoidal categories. A \emph{lax monoidal functor} is a functor $F:\CC\to\DD$ with a morphism $\varepsilon:1_\DD\to F(1_\CC)$
and a natural transformation
\[\mu_{A,B}:F(A)\otimes_\DD  F(B)\to F(A\otimes_\CC B)\]
for all $A,B\in\CC$ satisfying natural unitality, associativity and symmetry axioms. A lax monoidal functor is called \emph{strong monoidal} if $\varepsilon$ and $\mu_{A,B}$ are isomorphisms for all $A,B\in\CC$.
\end{defin}


\begin{defin}
Suppose $(F,\mu,\varepsilon)$ and $(F, \nu, \epsilon)$ are monoidal functors between the symmetric monoidal categories $\CC$ and $\DD$. Then a \emph{natural transformation} $\alpha:F\to G$ is \emph{monoidal} if the following diagrams commute.
\[
\begin{tikzcd}[column sep = 42pt]
F(A)\otimes_\DD F(B)\arrow[r, "\alpha_A\otimes_\DD\alpha_B"]\arrow[d,"\mu_{A,B}"] & G(A)\otimes_\DD G(B)\arrow[d, "\nu_{A,B}"] & 1_\DD\arrow[d, "\varepsilon"]\arrow[dr, "\epsilon"]& \\
F(A\otimes_\CC B)\arrow[r, "\alpha_{A\otimes_\CC B}"] & G(A\otimes_\CC B) & F(1)\arrow[r, "\alpha_1"] &  G(1)
\end{tikzcd}
\]
%commute.
%
%\[
%\begin{tikzcd}[column sep = 50pt]
%F(A)\otimes_\DD F(B)\arrow[r, "\alpha_A\otimes_\DD\alpha_B"]\arrow[d,"\mu_{A,B}"] & G(A)\otimes_\DD G(B)\arrow[d, "\nu_{A,B}"]\\
%F(A\otimes_\CC B)\arrow[r, "\alpha_{A\otimes_\CC B}"] & G(A\otimes_\CC B)
%\end{tikzcd}
%\]
%
%and
%\[
%\begin{tikzcd}
%1_\DD\arrow[d, "\varepsilon"]\arrow[dr, "\epsilon"]&\\
%F(1)\arrow[r, "\alpha_1"] &  G(1)
%\end{tikzcd}
%\]
%commute.
\end{defin}
\pagebreak
\begin{defin}
If $\CC$ and $\DD$ are symmetric monoidal categories, a lax monoidal functor $F: \CC → \DD$ is a \emph{monoidal equivalence} if there is a lax monoidal functor $G: \DD → \CC$ such that there exist monoidal natural isomorphisms $α: FG ⇒ id_\CC$, $β: GF ⇒ id_\DD$.
\end{defin}

\begin{defin}
A symmetric monoidal category $\CC$ is \emph{closed} if for every object $A\in\CC$ the tensor product functor $A\otimes -:\CC\to\CC$ has a right adjoint functor $[A,-]:\CC\to\CC$. In other words, for all $A,B,C\in \CC$ we have a natural bijection between the morphism sets
\[\Hom_\CC(A\otimes B, C)\cong \Hom_\CC(A, [B,C])\]
natural in all arguments. The object $[A,B]$ is called the \emph{internal hom}.
\end{defin}



%THE CONTENT OF SARAH'S PAPER COULD COME HERE (BUT IT IS SPECIFIC ENOUGH TO BE WRITTEN BEFORE THE STUFF IT IS MADE FOR)
\subsection{Monoidal categories over a base}
%NOT SURE IF I'M GOING TO NEED SO MUCH GENERAL BACKGROUND (CERTAINLY DO NOT INCLUDE PROOFS) I WOULD LIKE TO INCLUDE THE PROOF OF THE INVERSE INDUCING THE INVERSE TRANSFORMATION


%I MAY NOT  NEED MUCH (BUT I DO MENTION ENRICHED CATEGORIES, I COULD TRY TO ONLY SHOW RESULTS FOR PARTICULAR CATEGORIES) 

%WHAT I REALLY NEED IS THEE DEFINITION OF ENRICHED MU AND I CAN DEFINE IT EXPLICITELY IN THE LEMMA THAT IT IS USED (OR WITH AN EXTRA LEMMA IN WHICH I ALSO SHOW THE INVERSE) USING LEMMA 4.35

We collect some results about enriched categories from \cite{riehl} and \cite[\S 4.2]{whitehouse} that we will need as a categorical setting for our results on derived $A_\infty$-algebras. Here we combine the idea of enriched category with that of symmetric monoidal category.


\begin{defin}
Let $\VV$ a monoidal category. A $\VV$-\emph{category} $\CC$, also called $\VV$-\emph{
enriched category} or \emph{category enriched over} $\VV$, consists of
\begin{itemize}
\item a set $\mathrm{Ob}(\CC)$ of objects in $\CC$,
\item for each pair $(A,B)$ of objects in $\CC$ an object $\CC(A,B)\in\VV$ called the \emph{hom-object} or \emph{object of morphisms} from $A$ to $B$,
\item for every triple $(A,B,C)$ of objects in $\CC$ a morphism
\[\circ_{A,B,C}:\CC(B,C)\otimes\CC(A,B)\to \CC(A,C)\]
in $\VV$ called \emph{composition morphism},
\item and for each object $A$ in $\CC$ a morphism $u_A:1\to \CC(A,A)$ in $\VV$ called the \emph{identity element}. 
\end{itemize}
All this data is subject to associativity and unitality constrains that can be seen in detail in \cite{borceux}.
\end{defin}

\begin{defin}
Let $(\VV ,⊗, 1)$ be a symmetric monoidal category and let $(\CC,⊗, 1)$ be a monoidal category. We say that $\CC$ is a \emph{monoidal category over $\VV$} if there is an \emph{external (tensor) product} $∗ :\VV × \CC → \CC$ such that the following natural isomorphisms hold.%\pagebreak
\begin{itemize}
%\begin{enumerate}%[$\bullet$]
\item  $1 ∗ X \cong X$ for all $X ∈ \CC$,
\item $(C ⊗ D) ∗ X \cong C ∗ (D ∗ X)$ for all $C,D ∈ \VV$ and $X ∈ \CC$,
\item $C ∗ (X ⊗ Y ) \cong (C ∗ X) ⊗ Y \cong X ⊗ (C ∗ Y )$ for all $C ∈ \VV$ and $X, Y ∈ \CC$.
%\end{enumerate}
\end{itemize}
\end{defin}
\begin{remark}\label{underline}
Throughout the thesis we will also assume that there is a bifunctor $\uC(−,−) : \CC^{op} × \CC → \VV$ such that we have natural
bijections
\[\Hom_\CC(C ∗ X, Y ) \cong \Hom_\VV (C,\uC(X, Y )).\]
Under this assumption we get a $\VV$-enriched category $\uC$ with the same objects as $\CC$ and with hom-objects given by $\uC (−,−)$. The unit
morphism $u_X : 1 → \uC (X,X)$ corresponds to the identity map in $\CC$ under the adjunction, and the
composition morphism is given by the adjoint of the composite

%\[(\uC (B,C) ⊗ \uC (A,B)) ∗ A\cong \uC (B,C) ∗ (\uC (A,B) ∗ A)\xrightarrow{id∗ev_{AB}}\uC (B,C) ∗ B\xrightarrow{ev_{BC}} C,\]
\[
\begin{tikzcd}
(\uC (Z,Y) ⊗ \uC (X,Z)) ∗ X \arrow[r, phantom, "\isom"]& \uC (Z,Y) ∗ (\uC (X,Z) ∗ X) \arrow[d, "id∗ev_{XZ}"]&\\
 & \uC (Z,Y) ∗ Z \arrow[r, "ev_{ZY}"] & Y
\end{tikzcd}
\]%[ar symbol/.style = {draw=none,"#1" description,sloped}, isomorphic/.style = {ar symbol={\cong}}]%isomorphic, shorten <= -.5em

where $ev_{XZ}$ is the adjoint of the identity $\uC (X,Z) → \uC (X,Z)$. Furthermore, $\uC$ is a monoidal $\VV$-enriched category, namely we have an
enriched functor
\[\underline{⊗} : \uC × \uC → \uC\]
where $\uC × \uC$ is the enriched category whose objects are $\mathrm{Ob}(\CC ) × \mathrm{Ob}(\CC )$ and whose hom-objects are defined as
\[(\uC × \uC) ((X, Y ), (W,Z)) \coloneqq \uC (X,W) ⊗ \uC (Y,Z).\]
As a consequence, we get maps in $\VV$
\[\uC (X,W) ⊗ \uC (Y,Z) → \uC (X ⊗ Y,W ⊗ Z),\]
by considering the adjoint of the composite
%\[(\uC (X,W) ⊗ \uC (Y,Z)) ∗ (X ⊗ Y )\cong (\uC (X,W) ∗ X) ⊗ (\uC (Y,Z) ∗ Y )\xrightarrow{ev_{XW}⊗ev_{Y Z}} W ⊗ Z.\]
\[
\begin{tikzcd}[column sep = 1.2em]
(\uC (X,W) ⊗ \uC (Y,Z)) ∗ (X ⊗ Y )\arrow[r, phantom, "\isom"]& (\uC (X,W) ∗ X) ⊗ (\uC (Y,Z) ∗ Y ) \arrow[d, "ev_{XW}⊗ev_{Y Z}"]\\
 & W ⊗ Z
\end{tikzcd}
\]
\end{remark}
\pagebreak
\begin{defin}
Let $\CC$ and $\DD$ be monoidal categories over $\VV$. A \emph{lax functor over $\VV$} consists of a functor $F : \CC → \DD$ along with a natural transformation \[ν_F : − ∗_\DD F(−) ⇒ F(− ∗_\CC −)\]
which is associative and unital with respect to the monoidal structures over $\VV$ of $\CC$ and $\DD$. The coherence axioms for $ν_F$ are detailed in \cite[Proposition 10.1.5]{riehl}. If $ν_F$ is a natural isomorphism, then $F$ is said to be a \emph{functor over $\VV$}.
\end{defin}
\begin{defin}
Let $F,G : \CC → \DD$ be lax functors over $\VV$. A \emph{natural transformation over $\VV$} is a natural transformation
$μ : F ⇒ G$ such that for any $C ∈ \VV$ and for any $X ∈ \CC$ we have
\[ν_G \circ (1 ∗_\DD μ_X) = μ_{C∗_\CC X} \circ ν_F .\]
\end{defin}
\begin{defin}
A \emph{lax monoidal functor over $\VV$} is a triple $(F, \epsilon, μ)$, where $F : \CC → \DD$ is a lax functor over $\VV$,
$\epsilon : 1_\DD → F(1_\CC)$ is a morphism in $\DD$ and
\[μ : F(−) ⊗ F(−) ⇒ F(− ⊗ −)\]
is a natural transformation over $\VV$ satisfying the usual unit and associativity conditions. If $ν_F$
and $μ$ are natural isomorphisms then $F$ is said to be \emph{monoidal over $\VV$}. 
\end{defin}

Another notion of natural transformation in the enriched setting is given below, see \cite[Definition 3.5.8]{riehl} for the detailed diagrams.
\pagebreak
\begin{defin}
A \emph{$\VV$-enriched natural transformation} $\underline{\mu}:\underline{F}\Rightarrow\underline{G}$ between a pair of $\VV$-enriched functors $F,G:\CC\to\DD$ consists of a morphism $\underline{\mu}_X:1\to\DD(FX, GX)$ in $\VV$ for each $X\in \CC$ satisfying certain naturality conditions with respect to the external product and enriched composition.
\end{defin}

The following is \cite[Proposition 4.11]{whitehouse}.
%\pagebreak
\begin{propo}\label{enrichedtrans}
Let $F,G : \CC → \DD$ be lax functors over $\VV$. Then $F$ and $G$ extend to $\VV$-enriched
functors
\[\underline{F},\underline{G} : \uC → \uD\]
where $\uC$ and $\uD$ denote the $\VV$-enriched categories corresponding to $\CC$ and $\DD$ as described in \Cref{underline}. Moreover, any natural transformation $μ : F ⇒ G$ over $\VV$ also extends to a $\VV$-enriched natural
transformation
\[\underline{μ} : \underline{F} ⇒ \underline{G}.\]
In particular, if $F$ is lax monoidal over $\VV$,  then $\underline{F}$ is lax monoidal in the enriched sense, where the monoidal structure on $\uC × \uC$ is described in \Cref{underline}.
\end{propo}

\begin{lem}
Let $F,G:\CC\to\DD$ lax functors over $\VV$ and let $\mu : F\Rightarrow G$ a natural transformation over $\VV$. For every $X\in\CC$ and $Y\in\DD$ there is a map \[\uD(GX,Y)\to\uD(FX,Y)\] that is an isomorphism if $\mu$ is an isomorphism.
\end{lem}
We would like to thank Sarah Whitehouse for her contribution to the proof of this result.

\begin{proof}
By \Cref{enrichedtrans} $\mu$ extends to a $\VV$-enriched natural transformation 
\[\underline{\mu}:\underline{F}\Rightarrow\underline{G}\]
that at each object $X$ evaluates to \[\underline{\mu}_X:1\to\uD(FX,GX)\] defined to be the adjoint of $\mu_X:FX\to GX$. We define the map $\uD(GX,Y)\to\uD(FX,Y)$ as the composite

\begin{equation}\label{enrichedmap}
\begin{tikzcd}
\uD(GX,Y)\cong\uD(GX,Y)\otimes 1\arrow[r,"1\otimes\umu_X"]&\uD(GX,Y)\otimes\uD(FX,GX)\arrow[d, "c"]\\ 
& \uD(FX,Y)
\end{tikzcd}
\end{equation}
where $c$ is the composition map in the enriched setting. 

When $\mu$ is an isomorphism we may analogously define the following map

\begin{equation}\label{enrichedmapinverse}
\begin{tikzcd}
\uD(FX,Y)\cong\uD(FX,Y)\otimes 1\arrow[r,"1\otimes\umui_X"]&\uD(FX,Y)\otimes\uD(GX,FX)\arrow[d, "c"]\\ 
& \uD(GX,Y)
\end{tikzcd}
\end{equation}

We show that the above map is the inverse of the map (\ref{enrichedmap}). Consider the following diagram where the external arrows are the composition of (\ref{enrichedmap}) and (\ref{enrichedmapinverse}).

\begin{equation}
\adjustbox{scale=0.73,center}{%
\begin{tikzcd}[sep = tiny, column sep = 0pt, row sep = 20pt]\label{complicateddiagram}
{\uD(GX,Y) } \arrow[r, "\cong"] \arrow[rd, "(5)", phantom, bend left = 7]                 & {\uD(GX,Y)\otimes 1} \arrow[r, "1\otimes\umu_X"] \arrow[d, "1\otimes\alpha_X"] \arrow[rd, "(4)", phantom] & {\uD(GX,Y)\otimes\uD(FX,GX)} \arrow[r, "c"] \arrow[d, "\cong"]                                                                     & {\uD(FX,Y)} \arrow[lddd, "\cong", bend left]                                               \\
                                                                                     & {\uD(GX,Y)\otimes\uD(GX,GX)} \arrow[lu, "c", bend left = 5]                                                  & {\uD(GX,Y)\otimes\uD(FX,GX)\otimes 1} \arrow[ld, "1\otimes 1\otimes \umui_X"] \arrow[dd, "c\otimes 1"] \arrow[ru, "(1)"', phantom] &                                                                                \\
                                                                                     & {\uD(GX,Y)\otimes\uD(FX,GX)\otimes \uD(GX,FX)} \arrow[u, "1\otimes c"] \arrow[d, "c\otimes 1"]          &                                                                                                                                    &                                                                                \\
{}& {\uD(FX,Y)\otimes\uD(GX,FX)} \arrow[luuu, "c", bend left = 40] \arrow[uuul, near end,  "(3)"', phantom, bend left]                                                                                                         &                                                                                                                                    {\uD(FX,Y)\otimes 1} \arrow[l, "1\otimes\umui_X"] \arrow[lu, "(2)"', phantom] & {}
\end{tikzcd}
}
\end{equation}
% I believe this result is not true in full generality because the right unitor can be anything. In R-modules enriched over themselves, normally one would chose v -> vx1, but if you choose a unit u and instead do v -> uvx1=vxu, this is still an iso (forces the left unitor to be also like this for the triangle equality) and u is carried all the time

In the above diagram (\ref{complicateddiagram}), $\alpha_X$ is adjoint to $1_{GX}:GX\to GX$. Diagrams (1) and (2) clearly commute. Diagram (3) commutes by associativity of $c$. Diagram (4) commutes because $\umui_X$ and $\umu_X$ are adjoint to mutual inverses, so their composition results in the adjoint of the identity. Finally, diagram (5) commutes because we are composing with an isomorphism. In particular, diagram (5) is a decomposition of the identity map on $\uD(GX,Y)$. By commutativity, this means that the overall diagram composes to the identity, showing that the maps (\ref{enrichedmap}) and (\ref{enrichedmapinverse}) are mutually inverse.
\end{proof}


\section{$A_\infty$-algebras}\label{back2}


%Let us start by recalling some background definitions and results that we will need to study $A_\infty$-algebras, as well as stating some conventions.

%MAYBE OMIT ALL MENTION TO SYMMETRIC GROUPS BECAUSE I'M NOT REALLY GONNA USE THEM


%We assume that the reader is familiar with the basic definitions regarding $A_{∞}$-algebras and operads, but 


In the early sixties, J. Stasheff introduced $A_\infty$-spaces and $A_\infty$-algebras \cite{STASHEFFI}, \cite{STASHEFF} as a tool in the study of ``group-like'' spaces. We are going to motivate them by explaining their topological origin and later we will give precise definitions. We will also recall minimal models to further motivate the study of $A_\infty$-algebras and their limitations. A more detailed survey can be found in \cite{keller} and in \cite{lodayvallette}.

\subsection{Topological origin}\label{origin}

Let us consider the basic example. Let $(X, *)$ be a topological space with a base point $*$ and let $\Omega X$ denote the space of based loops in $X$: a point of 
$\Omega X$ is thus a continuous map $f : S^ 1 \to X$ taking the base point of the circle to the base point $*$. 

We have the composition map
\[
m_2 : \Omega X × \Omega X \to \Omega X
\]
sending a pair of loops $(f_1, f_2)$ to the loop $f_1 * f_2 = m_2(f_1, f_2)$ obtained by running through $f_1$ on the first half of the circle at twice the speed and through $f_2$ on the second half.

\[
\begin{tikzpicture}[line cap=round,line join=round,>=triangle 45,x=1.0cm,y=1.0cm, scale = 0.7]
\clip(-5,-3.) rectangle (5.,2.3);
\draw(0.,0.) circle (1.5cm);
\draw [-] (-1.6,0.) -- (-1.4,0);
\draw [-] (1.6,0.) -- (1.4,0);
\draw [->] (1.45,0.3) -- (1.4,0.55617768030476);
\draw [->] (-1.5,-0.2) -- (-1.45,-0.419061434935543);
\draw (-0.3,2.2883889729) node[anchor=north west] {$f_1$};
\draw (-0.3,-1.5566913118092813) node[anchor=north west] {$f_2$};
\draw (-3.6,0.311720973491667) node[anchor=north west] {$f_1*f_2$};
\end{tikzpicture}
\]


This composition is not associative: for three loops $f_1$, $f_2$, $f_3$, the composition $(f_1*f_2)*f_3$ runs through $f_1$ on the first quarter of the circle whereas the composition $f_1 * (f_2 * f_3)$ runs through $f_1$ on the first half of the circle. We symbolize the two possibilities by two binary trees with three leaves.

\[
\begin{tikzpicture}[line cap=round,line join=round,>=triangle 45,x=1.0cm,y=1.0cm]
\clip(-4.175394430564892,-2.5911383046897085) rectangle (7.490400123879831,3.3976612960713135);
\draw(0.,2.) circle (1.cm);
\draw [-] (-1.1,2.) -- (-0.9,2);
\draw [-] (1.1,2) -- (0.9,2);
\draw [-] (0,2.9) -- (0,3.1);
\draw(0.,-1.) circle (1.cm);
\draw [-] (-1.1,-1) -- (-0.9,-1);
\draw [-] (1.1,-1) -- (0.9,-1);
\draw [-] (0,-1.9) -- (0,-2.1);
\draw (-3.8016933844,2.311720973491667) node[anchor=north west] {$(f_1*f_2)*f_3$};
\draw (-3.796110195476,-0.6773028846978557) node[anchor=north west] {$f_1*(f_2*f_3)$};
\draw (-0.6057688157486263,1.1) node[anchor=north west] {$f_3$};
\draw (0.2964512548334696,0.5) node[anchor=north west] {$f_1$};
\draw (0.7431133461355,3.1) node[anchor=north west] {$f_1$};
\draw (-1.5,3.0105934583201526) node[anchor=north west] {$f_2$};
\draw (-1.5,-1.4729423289641315) node[anchor=north west] {$f_2$};
\draw (0.7489686163804383,-1.4729423289641315) node[anchor=north west] {$f_3$};
\draw (2.,2.)-- (3.,1.);
\draw (3.,1.)-- (4.,2.);
\draw (3.,2.)-- (2.514225889472578,1.485774110527422);
\draw (2.,-1.)-- (3.,-2.);
\draw (3.,-2.)-- (4.,-1.);
\draw (3.,-1.)-- (3.481998691455241,-1.518001308544759);
\draw (1.7811495170502019,2.6342775049509677) node[anchor=north west] {$f_1$};
\draw (2.8240823021019423,2.623525620568991) node[anchor=north west] {$f_2$};
\draw (3.7487443589519387,2.5912699674230613) node[anchor=north west] {$f_3$};
\draw (1.7811495170502019,-0.4085057751484382) node[anchor=north west] {$f_1$};
\draw (2.8133304177199654,-0.4085057751484382) node[anchor=north west] {$f_2$};
\draw (3.813255665243799,-0.4300095439123916) node[anchor=north west] {$f_3$};
\draw [->] (0.97,2.2) -- (0.94,2.33701987783);
\draw [->] (-0.15,2.98) -- (-0.3,2.95651215);
\draw [->] (-0.97,1.8) -- (-0.95,1.675490438);
\draw [->] (0.958,-0.75) -- (0.945,-0.67);
\draw [->] (-0.98,-1.19) -- (-0.93938908082,-1.33820708718);
\draw [->] (0.18,-1.98) -- (0.3,-1.96152712);
\end{tikzpicture}
\]

There is a homotopy
\[
m_3 : [0, 1]× \Omega X × \Omega X \to \Omega X
\]
joining the two possibilities of composing three loops by a reparametrization. When we want to compose four
loops, there are five possibilities corresponding to the five binary trees with four leaves. Using
$m_3$, we obtain two concatenations of homotopies linking the compositions $(f_1, f_2, f_3, f_4) \mapsto
((f_1 * f_2) * f_3) * f_4$ and $(f_1, f_2, f_3, f_4) \mapsto f_1 * (f_2 * (f_3 * f_4))$. The homotopy relation between the previous concatenations can be depicted with the following picture.

\definecolor{dcrutc}{rgb}{0.39215686274509803,0.39215686274509803,0.39215686274509803}
\[
\begin{tikzpicture}[line cap=round,line join=round,>=triangle 45,x=1.0cm,y=1.0cm, scale = 0.96]
\clip(-3.0601652892561972,-1.4) rectangle (5.063801652892563,3.1);
\fill[color=dcrutc,fill=dcrutc,fill opacity=0.10000000149011612] (2.5,2.5) -- (0.5,2.5) -- (-0.1180339887498949,0.5978869674096935) -- (1.5,-0.5776835371752531) -- (3.118033988749895,0.5978869674096927) -- cycle;
\draw (2.5,2.5)-- (0.5,2.5);
\draw (0.5,2.5)-- (-0.1180339887498949,0.5978869674096935);
\draw (-0.1180339887498949,0.5978869674096935)-- (1.5,-0.5776835371752531);
\draw (1.5,-0.5776835371752531)-- (3.118033988749895,0.5978869674096927);
\draw (3.118033988749895,0.5978869674096927)-- (2.5,2.5);
\draw (-0.4298121712997737,3.113688955672426)-- (0.16372652141247268,2.6629000751314793);
\draw (0.16372652141247268,2.6629000751314793)-- (0.6445679939894825,3.196333583771599);
\draw (-0.24422955190381973,2.9727401308147394)-- (-0.1367993989481584,3.113688955672426);
\draw (-0.031385527253731726,2.8110864412070775)-- (0.17123966942148847,3.10617580766341);
\draw (-1.2938241923365879,0.8973102930127725)-- (-0.6627197595792627,0.4314951164537945);
\draw (-0.6627197595792627,0.4314951164537945)-- (-0.16685199098422132,0.9649286250939145);
\draw (-0.8706947249429613,0.5850004480317625)-- (-0.5049436513899312,0.9799549211119465);
\draw (-0.6957015377932548,0.7739659686300595)-- (-0.9106536438767833,0.9799549211119465);
\draw (0.869962434259956,-0.8908189331329814)-- (1.516093163035313,-1.3791735537190069);
\draw (1.516093163035313,-1.3791735537190069)-- (2.0645529676934644,-0.8532531930879026);
\draw (1.699788860636136,-1.203026994375752)-- (1.1930277986476345,-0.838226897069871);
\draw (1.4723143911047147,-1.0392758405399645)-- (1.6888955672426758,-0.838226897069871);
\draw (3.2516303531179576,0.8522314049586783)-- (3.8151164537941407,0.2737190082644637);
\draw (3.8151164537941407,0.2737190082644637)-- (4.416168294515403,0.8221788129226153);
\draw (4.236244854403876,0.6579986738208468)-- (4.055537190082646,0.8372051089406468);
\draw (4.037275969199595,0.47643956607194105)-- (3.634800901577762,0.86725770097671);
\draw (2.2373553719008274,3.1437415477084896)-- (2.7783020285499633,2.5802554470323065);
\draw (2.7783020285499633,2.5802554470323065)-- (3.37184072126221,3.1136889556724268);
\draw (2.4140676158297834,2.959666293615827)-- (2.590473328324569,3.136228399699474);
\draw (3.189075837510895,2.949431908250359)-- (3.0112096168294524,3.1212021036814424);
\draw [->] (0.8624492862509402,2.880781367392938) -- (1.9894214876033067,2.8657550713749065);
\draw [->] (3.168985725018784,2.5351765589782125) -- (3.612261457550715,1.1903230653643886);
\draw [->] (-0.12928625093914242,2.647873779113449) -- (-0.6101277235161522,1.3180465815176567);
\draw [->] (-0.29457550713748953,0.12345604808414862) -- (0.6370548459804668,-0.78563486100676);
\draw [->] (2.395131480090159,-0.755582268970697) -- (3.37184072126221,0.1535086401202117);
\end{tikzpicture}
\]
%\definecolor{zzttqq}{rgb}{0.6,0.2,0.}
%\begin{tikzpicture}[line cap=round,line join=round,>=triangle 45,x=1.0cm,y=1.0cm]
%\clip(-2.4,-0.38685809569094237) rectangle (9.456589431026222,5.804555737079);
%\fill[color=zzttqq,fill=zzttqq,fill opacity=0.1] (2.,1.) -- (4.,1.) -- (4.618033988749895,2.9021130325903064) -- (3.,4.077683537175253) -- (1.3819660112501053,2.9021130325903073) -- cycle;
%\draw [color=zzttqq] (2.,1.)-- (4.,1.);
%\draw [color=zzttqq] (4.,1.)-- (4.618033988749895,2.9021130325903064);
%\draw [color=zzttqq] (4.618033988749895,2.9021130325903064)-- (3.,4.077683537175253);
%\draw [color=zzttqq] (3.,4.077683537175253)-- (1.3819660112501053,2.9021130325903073);
%\draw [color=zzttqq] (1.3819660112501053,2.9021130325903073)-- (2.,1.);
%\draw (2.4,4.6) node[anchor=north west] {\tiny{$(a(bc))d$}};
%\draw (4.631960418,3.3) node[anchor=north west] {\tiny{$((ab)c)d$}};
%\draw (0.3,3.3) node[anchor=north west] {\tiny{$(ab)(cd)$}};
%\draw (0.7,1.1) node[anchor=north west] {\tiny{$a(b(cd))$}};
%\draw (4.,1.1) node[anchor=north west] {\tiny{$a((bc)d)$}};
%\draw (3.,2.)-- (3.,2.5);
%\draw (3.,2.5)-- (2.5003101320434795,2.823227149181719);
%\draw (3.,2.5)-- (2.8259485989935587,2.836252687859722);
%\draw (3.,2.5)-- (3.1320487579266336,2.8297399185207204);
%\draw (3.,2.5)-- (3.4186106088427035,2.8167143798427174);
%\draw (2.7051847,1.8788257123772178) node[anchor=north west] {$m_4$};
%\draw (4.252245084234906,3.598246700522905)-- (4.258757853573909,3.9694745528459943);
%\draw (4.258757853573909,3.9694745528459943)-- (4.5778835511849865,4.177883171694044);
%\draw (4.258757853573909,3.9694745528459943)-- (4.0633747734038606,4.164857633016041);
%\draw (4.0633747734038606,4.164857633016041)-- (3.8289150771998033,4.314651327813078);
%\draw (4.0633747734038606,4.164857633016041)-- (4.050349234725857,4.360240713186088);
%\draw (4.0633747734038606,4.164857633016041)-- (4.245732314895905,4.314651327813078);
%\draw (5.00121355822009,1.357854047906367)-- (5.00121355822009,1.7355946695684576);
%\draw (5.00121355822009,1.7355946695684576)-- (4.708138937965018,1.9440032884165077);
%\draw (5.00121355822009,1.7355946695684576)-- (5.00121355822009,2.1003097525525454);
%\draw (5.00121355822009,2.1003097525525454)-- (4.851419863423053,2.315231140739597);
%\draw (5.00121355822009,2.1003097525525454)-- (5.1249561756611195,2.295692832722592);
%\draw (5.00121355822009,1.7355946695684576)-- (5.307313717153164,1.9505160577555094);
%\draw (3.0213316791636062,-0.2312616708100148)-- (3.014818909824605,0.19206833622508687);
%\draw (3.014818909824605,0.19206833622508687)-- (2.7152315202305317,0.459091879124151);
%\draw (3.014818909824605,0.19206833622508687)-- (3.3209190687576795,0.4460663404461479);
%\draw (3.3209190687576795,0.4460663404461479)-- (3.1125104499096286,0.7521664993792214);
%\draw (3.3209190687576795,0.4460663404461479)-- (3.327431838096681,0.7717048073962262);
%\draw (3.3209190687576795,0.4460663404461479)-- (3.5228149182667288,0.7391409607012183);
%\draw (1.712265042024287,3.4940423910988803)-- (1.7057522726852854,3.897834090116977);
%\draw (1.7057522726852854,3.897834090116977)-- (1.4061648830912123,4.132293786321034);
%\draw (1.4061648830912123,4.132293786321034)-- (1.1261158015141441,4.3407024051690835);
%\draw (1.4061648830912123,4.132293786321034)-- (1.4973436538372347,4.3732662518640915);
%\draw (1.7057522726852854,3.897834090116977)-- (1.7187778113632886,4.197421479711049);
%\draw (1.7057522726852854,3.897834090116977)-- (1.9923141236013553,4.145319324999036);
%\draw (1.0023731840731138,1.3057518931943544)-- (1.0088859534121155,1.6900052841954467);
%\draw (1.0088859534121155,1.6900052841954467)-- (0.6181197930720201,2.028669289823528);
%\draw (1.0088859534121155,1.6900052841954467)-- (1.,2.);
%\draw (1.0088859534121155,1.6900052841954467)-- (1.2824222656501822,2.002618212467522);
%\draw (1.2824222656501822,2.002618212467522)-- (1.1326285708531456,2.341282218095603);
%\draw (1.2824222656501822,2.002618212467522)-- (1.4387287297862204,2.315231140739597);
%\draw (3.,4.077683537175253)-- (4.618033988749895,2.9021130325903064);
%\draw (4.618033988749895,2.9021130325903064)-- (4.,1.);
%\draw (4.,1.)-- (2.,1.);
%\draw (2.,1.)-- (1.3819660112501053,2.9021130325903073);
%\draw (1.3819660112501053,2.9021130325903073)-- (3.,4.077683537175253);
%\draw [->] (3.,4.077683537175253) -- (3.895997820398156,3.4267030156558835);
%\draw [->] (4.618033988749895,2.9021130325903064) -- (4.263241770003597,1.8101748618369442);
%\draw [->] (4.,1.) -- (2.8910762923835747,1.);
%\draw [->] (3.,4.077683537175253) -- (2.1179720645146554,3.4368527311563817);
%\draw [->] (1.3819660112501053,2.9021130325903073) -- (1.7339911627499194,1.8186910191477048);
%\draw [shift={(3.0017933711466016,3.376812542996852)}] plot[domain=-4.688583954199619:0.02083032003621599,variable=\t]({1.*0.27361383422126806*cos(\t r)+0.*0.27361383422126806*sin(\t r)},{0.*0.27361383422126806*cos(\t r)+1.*0.27361383422126806*sin(\t r)});
%\draw [->] (3.,3.65) -- (3.15,3.65);
%
%\draw (3.,4.5)-- (2.9949604700273404,4.840669349740066);
%\draw (2.9949604700273404,4.840669349740066)-- (3.2670956760243546,4.986230041319864);
%\draw (2.9949604700273404,4.840669349740066)-- (2.75446889263463,5.043188572807612);
%\draw (2.75446889263463,5.043188572807612)-- (2.4190464294290073,5.188749264387409);
%\draw (2.75446889263463,5.043188572807612)-- (3.064576452956809,5.252036521596017);
%\draw (2.8942394348657974,5.1373197543102345)-- (2.75446889263463,5.308995053083764);
%
%\draw (5.728969981439207,2.8407920219480567)-- (5.722641255718346,3.1825432108745395);
%\draw (5.722641255718346,3.1825432108745395)-- (6.121350976132577,3.334432628175198);
%\draw (5.722641255718346,3.1825432108745395)-- (5.4315198725587495,3.353418805337781);
%\draw (5.4315198725587495,3.353418805337781)-- (5.140398489399153,3.5685954798470476);
%\draw (5.277773616415782,3.4670573424869304)-- (5.425191146837888,3.58758165700963);
%\draw (5.4315198725587495,3.353418805337781)-- (5.74795615860179,3.543280576963604);
%
%%\draw (5.108754860794849,0.31563045932460204)-- (5.115083586515709,0.676367825413667);
%%\draw (5.115083586515709,0.676367825413667)-- (4.842948380518695,0.8725583227603515);
%%\draw (5.115083586515709,0.676367825413667)-- (5.36190388962928,0.8662295970394907);
%%\draw (5.36190388962928,0.8662295970394907)-- (5.703655078555764,1.005461562898428);
%%\draw (5.36190388962928,0.8662295970394907)-- (5.115083586515709,1.10039244871134);
%%\draw (5.231917556986078,0.9895499639061185)-- (5.399876243954446,1.1320360773156437);
%\begin{scriptsize}
%\draw [fill=black] (2.,1.) circle (2.5pt);
%\draw [fill=black] (4.,1.) circle (2.5pt);
%\draw [fill=black] (4.618033988749895,2.9021130325903064) circle (2.5pt);
%\draw [fill=black] (3.,4.077683537175253) circle (2.5pt);
%\draw [fill=black] (1.3819660112501053,2.9021130325903073) circle (2.5pt);
%\end{scriptsize}
%\end{tikzpicture}

These two concatenations are homotopic. Denote a homotopy by
\[
m_4 : K_4 × (\Omega X)^4 \to X
\]
where $K_4$ denotes the pentagon bounded by the two paths depicted above. When we want to compose five loops, there are fourteen possibilities corresponding to the fourteen binary trees
with five leaves. Using $m_4$ and $m_3$, we obtain homotopies linking the compositions and faces linking the homotopies. The geometrical representation of these homotopies is the boundary of the polytope $K_5$ depicted below.
\[
\begin{tikzpicture}[line cap=round,line join=round,>=triangle 45,x=1.0cm,y=1.0cm]
\clip(-3.63,-1.3) rectangle (4.,2.9);
\draw(-0.5,0.) -- (0.,0.5) -- (-0.5,1.) -- (-1.,0.5) -- cycle;
\draw (-0.5,0.)-- (0.,0.5);
\draw (0.,0.5)-- (-0.5,1.);
\draw (-0.5,1.)-- (-1.,0.5);
\draw (-1.,0.5)-- (-0.5,0.);
\draw (-0.5,1.)-- (-0.76,2.87);
\draw (-0.76,2.87)-- (-2.13,1.59);
\draw (-2.13,1.59)-- (-1.98,0.82);
\draw (-2.13,1.59)-- (-2.5,1.);
\draw (-2.5,1.)-- (-2.31,0.29);
\draw (-2.31,0.29)-- (-1.98,0.82);
\draw (-1.98,0.82)-- (-1.,0.5);
\draw (0.,0.5)-- (0.93,0.89);
\draw (0.93,0.89)-- (0.98,1.68);
\draw (1.37,1.06)-- (0.98,1.68);
\draw (-0.76,2.87)-- (0.98,1.68);
\draw (-2.31,0.29)-- (-0.62,-1.3);
\draw (-0.5,0.)-- (-0.62,-1.3);
\draw (-0.62,-1.3)-- (1.22,0.35);
\draw (0.93,0.89)-- (1.22,0.35);
\draw (1.22,0.35)-- (1.37,1.06);
\draw [dash pattern=on 2pt off 2pt] (-2.5,1.)-- (1.37,1.06);
\end{tikzpicture}
\]

The pentagonal faces are copies of $K_4$. More generally, Stasheff \cite{STASHEFFI} defined polytopes $K_n$ of dimension $n−2$ for all $n \geq 2$, including $K_2 = *$ and $K_3 = [0, 1]$. He defined an \emph{$A_\infty$-space} to be a topological space
$Y$ endowed with maps $m_n : K_n × Y^n \to Y$, for $n \geq 2$, satisfying suitable compatibility
conditions and admitting a ``strict unit". The loop space $\Omega X$ is the prime example of
such a space $Y$. Conversely \cite{Adams}, a topological space that admits the structure of
an $A_\infty$-space and whose connected components form a group is homotopy equivalent
to a loop space. If $Y$ is an $A_\infty$-space, the singular chain complex of $Y$ is the paradigmatic example
of an $A_\infty$-algebra \cite{STASHEFF}, which will be formally introduced in the next section using cohomological grading.

\subsection{Definitions}

In this section we first establish notation and assumptions about graded modules and sign conventions. We then briefly recall some of the basic definitions regarding $A_\infty$-algebras.

Our base category is the category of $\Z$-graded $R$-modules and linear maps, where $R$ is a commutative ring with unit of characteristic distinct from 2. All tensor products are taken over $R$. We denote the $i$-th degree component of $A$ as $A^i$. If $x\in A^i$ we write $\deg(x)=i$ and we use cohomological grading. The symmetry isomorphism is given by the following Koszul sign convention.

\begin{align*}
\tau_{A,B}:A\otimes B&\to B\otimes A\\
x\otimes y &\mapsto (-1)^{\deg(x)\deg(y)}y\otimes x
\end{align*}

 A map $f:A\to B$ of degree $i$ satisfies $f(A^n)\subseteq B^{n+i}$ for all $n$. The $R$-modules $\Hom_R(A,B)$ are naturally graded by \[\Hom_R(A,B)^i=\prod_k\Hom_R(A^k,B^{k+i}).\]

As a consequence of the above sign convention, we also adopt the following Koszul sign convention: for $x\in A$, $y\in B$, $f\in\Hom_R(A,C)$ and $g\in\Hom_R(B,D)$,

\[(f\otimes g)(x\otimes y)=(-1)^{\deg(x)\deg(g)}f(x)\otimes g(y).\]

Recall that if $(A,\partial^A)$ and $(B,\partial^B)$ are (co)chain complexes, the module $\Hom_R(A,B)$ also becomes a (co)chain complex with differential
\[\partial(f) = \partial^B\circ f +(-1)^{\deg(f)}f\circ\partial^A.\]



%\pagebreak
With all the notations and conventions established we can now introduce $A_\infty$-algebras. 

\begin{defin}\label{ainftyalgebra}
An \emph{$A_\infty$-algebra} is a graded $R$-module $A$ together with a family of maps $m_n:A^{\otimes n}\to A$ of degree $2-n$ satisfying the equation

\begin{equation}\label{ainftyequation}
\sum_{r+s+t=n}(-1)^{rs+t}m_{r+t+1}(1^{\otimes r}\otimes m_s\otimes 1^{\otimes t})=0
\end{equation}
for all $n\geq 1$.
\end{defin}
%This is equivalent to define it on SA without signs,  but there's not need to point that out here and it also depends on the convention of S^n^{-1}
The above equation will sometimes be referred to as the $A_\infty$-\emph{equation}. The signs are related to the orientation given to the Stasheff polytopes, see \Cref{origin}. This seemingly obscure definition captures the idea of an algebra that is associative up to homotopy. To see this, let us have a look at the first few cases.

\begin{itemize}
\item We have $m_1m_1=0$, so $(A, m_1)$ is a cochain complex.
\item We have the Leibniz rule
\[
m_1 m_2 = m_2 (m_1 \otimes 1 + 1 \otimes m_1)
\]
as maps $A^{\otimes 2}\to A$. Here 1 denotes the identity map on $A$. So $m_1$ is
a graded derivation with respect to the multiplication $m_2$.
\item We have
\[
m_2(1\otimes m_2 − m_2\otimes 1)= m_1m_3 + m_3 (m_1\otimes 1\otimes 1 + 1\otimes m_1\otimes 1 + 1\otimes 1\otimes m_1)
\]
as maps $A^{\otimes 3}\to A$. Note that the left hand side is the associator for $m_2$ and
that the right hand side may be viewed as the boundary of $m_3$ in the morphism
complex $\Hom_R (A^{\otimes 3},A)$. This implies that $m_2$ is associative up to homotopy. 
\end{itemize}


 For more details about this the reader is referred to \cite{keller} and to \cite[\S 9.2]{lodayvallette}, where they use a different sign convention but the concepts are the same.

\begin{defin}\label{inftymorphism}
An \emph{$\infty$-morphism} of $A_\infty$-algebras $A\to B$ is a family of maps \[f_n:A^{\otimes n}\to B\] of degree $1-n$ satisfying for all $n\geq 1$ the equation
\[\sum_{r+s+t=n} (-1)^{rs+t}f_{r+1+t}(1^{\otimes r} \otimes m^A_s\otimes 1^{\otimes t})=\sum_{i_1+\cdots+i_k=n} (-1)^s m^B_k(f_{i_1}\otimes\cdots\otimes f_{i_k}),\]
where

\[s=\sum_{\alpha<\beta}i_\alpha(1-i_\beta).\]
The composition of $\infty$-morphisms $f:A\to B$ and  $g:B\to C$ is given by 

\[(gf)_n=\sum_r\sum_{i_1+\cdots+i_r=n}(-1)^s g_r(f_{i_1}\otimes\cdots
\otimes f_{i_r}).\]
\end{defin}

Similarly to the $A_\infty$-equation (\ref{ainftyequation}), the above definition captures the idea of a map that is a morphism of algebras up to homotopy. This can be observed in the first few cases.

\begin{itemize}
\item We have $f_1m_1 = m_1f_1$, i.e. $f_1$ is a morphism of complexes.
\item We have
\[
f_1m_2 = m_2 (f_1\otimes f_1) + m_1f_2 + f_2 (m_1\otimes 1 + 1\otimes m_1),\]
which means that $f_1$ commutes with the multiplication $m_2$ up to a homotopy
given by $f_2$.
\end{itemize}

Again, the reader can find more details in \cite{keller} and \cite[\S 9.2]{lodayvallette}. We also need another, perhaps more intuitive, notion of morphism of $A_\infty$-algebras.
 
\begin{defin}\label{inftyalgebramorphism}
A \emph{morphism} of $A_\infty$-algebras is a map of $R$-modules $f:A\to B$ such that
\[f(m^A_j)=m^B_j\circ f^{\otimes j}.\]
Equivalently, it is an $\infty$-morphism where $f_n=0$ for $n>1$.
\end{defin}

Notice that working with $\Z$-graded modules forces every morphism of $A_\infty$-algebras to be of degree 0.

We will discuss the topics of this section in the language of operads, which will be introduced in a later section. Before that, let us motivate the importance of $A_\infty$-algebras through the theory of minimal models.
\pagebreak
\subsection{Minimal models}\label{minimalmodels}
We now recall a definition and a theorem about minimal models of $A_\infty$-algebras. The theorem relates differential graded algebras to $A_\infty$-structures on their homology. This theorem is the main reason why $A_\infty$-algebras became a relevant subject of study.

\begin{defin}\label{minimal}
An $A_\infty$-algebra is called \emph{minimal} if $m_1 = 0$.
\end{defin}

Over a field, one can replace any $A_\infty$-algebra by a quasi-isomorphic minimal one, where by quasi-isomorphic we mean that there is a map that induces an isomorphism on cohomology with respect to $m_1$. This gives a very convenient way to describe a quasi-isomorphism class of an $A_\infty$-algebra. More precisely we have the following.

\begin{thm}[Kadeishvili]\label{minimaltheorem} Let $A$ be a differential graded algebra over a field $k$ of any characteristic, and let $H^*(A)$ be its cohomology module. Then $H^*(A)$ has an $A_\infty$-structure such that
\begin{itemize}
\item $m_1 = 0$ and the multiplication $m_2$ is induced by the multiplication on $A$,
\item there is an $\infty$-morphism of $A_\infty$-algebras $f : H^*(A) \to A$ such that $f_1$ is a quasi-isomorphism.
\end{itemize}
This $A_\infty$-algebra $H^*(A)$ is called the \emph{minimal model} of $A$.
\end{thm}

Using this result it is also possible to show that under some conditions any other dga $A'$ with $H^*(A')\cong H^*(A)$ is quasi-isomorphic to $A$. For more details see \cite{kade}.
\section{Operads}\label{back3}
%A BIT MORE OF WHAT IS IN JOIN, LIKE COOPERADS (THIS SHOULD ACTUALLY BE ADDED THERE) AND SOME OF THE STUFF IN THE ITEM BELOW (CHECK WHAT IS NECESSARY) 

In this section we recall the notion of operad, an object that is particularly useful to study algebraic structures given by multilinear maps. Operads will allow us to formulate definitions and results concerning derived $A_\infty$-algebras in a very convenient way. The main references for this section are \cite{lodayvallette} and \cite{ward}.

We will be working in the category of graded $R$-modules, but all the definitions and results in this section generalize with no substantial changes to any symmetric monoidal category like the ones we see in \Cref{categories}. 

\subsection{Definitions}
We first give the main definitions that we will be using throughout the thesis. We start defining the underlying object of an operad.

\begin{defin}\label{collections}
A \emph{collection} is a family $\OO=\{\OO(n)\}_{n\geq 0}$ of graded $R$-modules. We call the integer $n$ the \emph{arity}. When there is an action of the symmetric group $\Sigma_n$ on each $\OO(n)$ we say that the collection is an \emph{$\mathbb{S}-$module}. A \emph{map of collections} $f:\OO\to\mathcal{P}$ is a family of maps $f_n:\OO(n)\to\mathcal{P}(n)$. A map of collection is a \emph{map of $\mathbb{S}-$modules} when it preserves the symmetric group action.
\end{defin}

We will mostly focus on the non-symmetric case, but our general results about operads generalize to the symmetric case as well. On a collection we can define an operad by adding some extra structure as in the following definition.

\begin{defin}
A \emph{(non-symmetric) operad} is a collection $\OO=\{\OO(n)\}$ where there is a distinguished \emph{identity} element $1\in\OO(1)$ and with \emph{insertion maps} 
\[\circ_i:\OO(n)\otimes \OO(m)\to \OO(m+n-1)\]
for each $1\leq i\leq n$ satisfying natural unitality and associativity axioms, see \cite[\S 1.1.2]{ward}. 

Insertion maps can be iterated to define \emph{composition maps} \[\gamma(x;x_1,\dots, x_j)=(\cdots(x\circ_1 x_1)\circ_{1+m_1} x_2\cdots
)\circ_{1+ m_1+\cdots m_{j-1}} x_j,\]%\sum_{i=1}^{n-1}
where $x_i\in \OO(m_i)$. 
If $\OO$ is an $\mathbb{S}-$module and the insertion maps satisfy some additional invariance axioms regarding the symmetric group action, we say that $\OO$ is a \emph{symmetric operad}, see \cite{lodayvallette} for more details.

A \emph{map of operads} (resp. \emph{symmetric operads}) is a map of collections (resp. $\mathbb{S}-$modules) that is compatible with insertions.
\end{defin}

Collections also come with the following algebraic operation that will provide an alternative way of describing operads.
\begin{defin}
The \emph{plethysm} or \emph{composite} $\OO\circ\PP$ of two collections $\OO$ and $\PP$ given by
\[(\OO\circ\PP)(n)=\bigoplus_{N\geq 0}\OO(N)\otimes \left(\bigoplus_{a_1+\cdots+a_k=n} \PP(a_1)\otimes\cdots\otimes \PP(a_k)\right).\]
\end{defin}
There is a definition of plethysm for $\mathbb{S}$-modules that requires some tools from the representation theory of symmetric groups such as the induced representation that we are not going to introduce here. The reader is referred to \cite[\S 5.1]{lodayvallette} for the details. 

\begin{defin}
The \emph{plethysm} or \emph{composite} $f\circ g$ of maps $f:\OO\to\OO'$ and $g:\PP\to\PP'$ is given by
\[(f\circ g)(x_0\otimes x_1\otimes\cdots\otimes x_k)=(-1)^{\varepsilon} f(x_0)\otimes g(x_1)\otimes\cdots\otimes g(x_k),\]
where $\varepsilon = \deg(g)\sum_{i=0}^k\deg(x_i)(k-i)$ is the Koszul sign obtained from swapping each $g$ by the corresponding elements. 
\end{defin}

 %, TENSOR PRODUCT OF S-MODULES

It is known that the category of collections with plethysm is a monoidal category, where the unit is the collection $I(1)=R$ and $I(n)=0$ for $n\neq 1$. The following lemma is a well-known fact that describes operads in terms of this monoidal structure, see \cite[\S 5.2]{lodayvallette} for more details. 
\begin{lem}\label{monoid}
An operad $\OO$ is equivalent to a monoid in the monoidal category of collections with plethysm, where the multiplication map is given precisely by the composition $\gamma:\OO\circ\OO\to\OO$. 
\end{lem}


\begin{defin} An operad $\OO$ is called \emph{reduced} if $\OO(0)=0$.\end{defin}

\begin{defin}The \emph{Hadamard product} $\OO\otimes\PP$ of  two operads $\OO$ and $\PP$ is given on each arity component by $(\OO\otimes\PP)(n)=\OO(n)\otimes\PP(n)$. The structure maps are given by diagonal composition and diagonal symmetric group action in the case of symmetric operads. \end{defin}

The next definition is the most important and common example of an operad. It is very useful to keep it in mind when intuitively thinking about operads.

\begin{defin}
The \emph{endomorphism operad} $\End_A$ of a graded $R$-module $A$ is given by the modules \[\End_A(n)=\Hom_R(A^{\otimes n},A).\] Insertion maps are given by
\[f\circ_i g=f(1^{\otimes i-1}\otimes g\otimes 1^{\otimes n-i})\]
for $f\in\End_A(n)$ and $g\in\End_A(m)$. The identity element is given by the identity map and there is a symmetric group action given by permuting the inputs.
\end{defin}
The endomorphism operad also allows us to define algebras over any operad.
\begin{defin}
An \emph{algebra over an operad} $\OO$, or $\OO$-\emph{algebra}, is a map of operads $\OO\to\End_A$ for some $R$-module $A$. By the tensor-hom adjunction, this corresponds to a collection of maps $\OO(n)\otimes A^{\otimes n}\to A$ for each $n\geq 0$.  
\end{defin}

\begin{defin}\label{algebramorphism}
A \emph{morphism of $\OO$-algebras} $A$ and $B$ is a map of operads $\End_A\to\End_B$ so that the diagram
\[
\begin{tikzcd}
\OO \arrow[r]\arrow[dr] & \End_A\arrow[d]\\
& \End_B
\end{tikzcd}
\]
commutes. By adjunction, this is equivalent to a map $f:A\to B$ of $R$-modules such that the diagram
\[
\begin{tikzcd}
\OO(n)\otimes A^{\otimes n} \arrow[r]\arrow[d, "id\otimes f^{\otimes n}"] & A\arrow[d, "f"]\\
 \OO(n)\otimes B^{\otimes n}\arrow[r] & B
\end{tikzcd}
\]
commutes for all $n$.
%changing id by other map I could have maps of algebras over different operads
\end{defin}

\begin{defin}
The \emph{$\mathcal{A}_\infty$-operad} is the non-symmetric operad whose algebras are $A_\infty$-algebras, see \Cref{ainftyalgebra}. Therefore, it is generated by elements $\mu_i\in\mathcal{A}_\infty(i)$ satisfying the operadic version of the $A_\infty$-equation (\ref{ainftyequation}).

\begin{equation}\label{operadicainftyequation}
\sum_{r+s+t=n}(-1)^{rs+t}\mu_{r+t+1}\circ_{r+1}\mu_s=0.
\end{equation}
More details about this operad can be found in \cite[Chapter 9]{lodayvallette}.
\end{defin}

Notice that for the $\calA_\infty$-operad, a morphism of $\calA_\infty$-algebras, \Cref{algebramorphism}, is the same thing as a morphism of $A_\infty$-algebras, \Cref{inftyalgebramorphism}. We will also provide a new operadic interpretation of $\infty$-morphisms, \Cref{inftymorphism}, and relate it to an existing interpretation in \Cref{reinterpretation}. 

\begin{remark}\label{internal}
If one considers $m_1$ as an internal differential of the algebra $A$, \Cref{ainftyequation} reads for each $j$ as
\begin{align*}
&m_1(m_j) -(-1)^s \sum _{r+t+1 = j}m_j(1^{\otimes r}\otimes m_1\otimes 1^{\otimes t})\\
&=-\underset{s>1, r+t >0}{\sum_{r+s+t=j}}(-1)^{rs+t}m_{r+t+1}(1^{\otimes r}\otimes m_s\otimes 1^{\otimes t}).
\end{align*}

This leads to a definition of the operad $\mathcal{A}_\infty$ in the category of cochain complexes as the operad generated by $\mu_i\in\mathcal{A}_\infty(i)$ for $i>1$ and with differential given by
\[
\partial_\infty(\mu_j)= -{\sum_{r+s+t=j}}(-1)^{rs+t}\mu_{r+t+1}\circ_{r+1} \mu_s.
\]
Notice that this operad can now be described with no other relations than the differential. This is an example of what is called a \emph{quasi-free} operad in the literature, see \cite[\S 6.3.3]{lodayvallette}
\end{remark}



\subsection{Operads and monoidality}

The monoidal definition of operad from \Cref{monoid} allows to define the dual notion of a cooperad.



\begin{defin}
Let $\OO$ be a collection. A \emph{cooperad} is a structure of comonoid on $\OO$ in the monoidal category $(\col, \bar{\circ},I)$, where \[(\PP\bar{\circ}\mathcal{Q})(n) \coloneqq
\bigoplus_r \bigoplus_{n=i_1+\cdots+i_r}(\PP(r)\otimes \QQ(i_1)\otimes\cdots\otimes \QQ(i_r)),\]
and  $I$ is the collection such that $I(0)=R$ and is trivial elsewhere.
See \cite[\S 5.7.1]{lodayvallette} for more details and the symmetric version.
\end{defin}


Note that this is not the exact dual notion of an operad. To define the exact dual of the notion of operad, one should instead consider
the monoidal product
\[(\PP\hat{\circ}\QQ)(n) =\prod_{r\geq 0}(\PP(r)\otimes \prod_{n=i_1+\cdots+i_r}(\QQ(i_1)\otimes\cdots\otimes \QQ(i_r)))\]
in the category of collections, where the sums are replaced by products. In that case, a cooperad is defined as a comonoid $\Delta:\OO\to\OO\hat{\circ}\OO$.
When $\OO(0) = 0$ (the operad is reduced), the right-hand side product is equal to a sum. In this case, we are back to the previous definition.



The following result is stated and proved in \cite[Proposition 3.1.1(a)]{fresse}, but the proof omits many details, so we are writing down the full proof here.
\begin{propo}
Any symmetric lax monoidal functor $F:\CC\to\DD$ induces a functor $\underline{F}:\mathrm{Op}_\CC\to \mathrm{Op}_\DD$ between the categories of operads in $\CC$ and $\DD$, respectively. The result is also true for cooperads.
\end{propo}

%REFERENCE THE AXIOMS WHEN I WRITE THEM DOWN
\begin{proof}
We prove the result for operads, since for cooperads is analogous. Let $\OO$ be an operad in $\CC$ and let $F:\CC\to\DD$ be a symmetric lax monoidal functor. On objects, we define $\underline{F}(\OO)(n)= F(\OO(n))$ and on morphisms we define $\underline{F}(f)_n=F(f_n)$ for $f:\OO\to\PP$. 

Let $\varepsilon: 1_\DD\to F(1_\CC)$ and $\mu\coloneqq\mu_{A,B}: F(A)\otimes F(B)\to F(A\otimes B)$ be the structure maps of the lax monoidal functor $F$. 

Let us first define the structure maps for the operad $F(\OO)$ in terms of insertions. Let $e:1_\CC\to\OO(1)$ be the unit of $\OO$. We define the unit $e_F:1_\DD\to F(\OO(1))$ as the composite
\[1_\DD\xrightarrow{\varepsilon} F(1)\xrightarrow{F(e)} F(\OO(1)).\]
Let $\circ_i :\OO(n)\otimes\OO(m)\to\OO(n+m-1)$ be the insertion map on $\OO$. We define the insertion map $\circ_i^F:F(\OO(n))\otimes F(\OO(m))\to F(\OO(n+m-1))$ as the composite
\[F(\OO(n))\otimes F(\OO(m))\xrightarrow{\mu} F(\OO(n)\otimes\OO(m))\xrightarrow{F(\circ_i)} F(\OO(n+m-1)).\]

We show now that $F(\OO)$ satisfies the unit axioms with the above structure maps. We only show the unit axiom with respect to the right unitor, the axiom with respect to the left unitor being analogous.


 Let $\lambda_\CC$ and $\lambda_\DD$ be the right unitors of $\CC$ and $\DD$ respectively. Since $\OO$ is an operad, by the unit axiom we have that the following diagram commutes.
% \[
% \begin{tikzcd}
% \OO(n)\otimes 1\arrow[r, "id\otimes e"]\arrow[rr, bend right = 20, "\lambda_\CC"] & \OO(n)\otimes \OO(1)\arrow[r, "\circ_i"]& \OO(n) 
% \end{tikzcd}
% \]
 
  \[
 \begin{tikzcd}
 \OO(n)\otimes 1_\CC\arrow[r, "\lambda_\CC"]\arrow[d, "id\otimes e"] &  \OO(n) \\
 \OO(n)\otimes \OO(1)\arrow[ur, bend right = 10, "\circ_i"]&
 \end{tikzcd}
 \]
 
Applying $F$ and introducing $\mu$ we get the following commutative diagram.
\begin{equation}\label{unitaux}
\begin{tikzcd}
F(\OO(n))\otimes F(1_\CC)\arrow[r,"\mu"]\arrow[d, "id\otimes F(e)"] & F(\OO(n)\otimes 1_\CC)\arrow[r, "F(\lambda_\CC)"]\arrow[d, "F(id\otimes e)"]&
F(\OO(n))\\
F(\OO(n))\otimes F(\OO(1))\arrow[r,"\mu"] & F(\OO(n)\otimes \OO(1))\arrow[ur, bend right = 15, "F(\circ_i)"']
\end{tikzcd}
\end{equation}

We need to show that the following diagram commutes.

  \[
 \begin{tikzcd}
F( \OO(n))\otimes 1_\DD\arrow[r, "\lambda_\DD"]\arrow[d, "id\otimes e_F"] &  F(\OO(n)) \\
 F(\OO(n))\otimes F(\OO(1))\arrow[ur, bend right = 10, "\circ_i^F"]&
 \end{tikzcd}
 \]
 
% Let us develop the diagram using the definition of the corresponding maps.
% 
% \[
%\begin{tikzcd}
%F( \OO(n))\otimes 1_\DD\arrow[r, "\lambda_\DD"]\arrow[d, "id\otimes F(e)\circ \varepsilon"] &  F(\OO(n)) \\
% F(\OO(n))\otimes F(\OO(1))\arrow[u, "\mu"] & \\
% F(\OO(
%\end{tikzcd} 
 %\]
 By monoidality of $F$ we know that $\lambda_\DD$ satisfies the following commutative diagram.
 
 \[
 \begin{tikzcd}
 F(\OO(n))\otimes 1_\DD\arrow[d, "\lambda_\DD"]\arrow[r, "id\otimes \varepsilon"] & F(\OO(n))\otimes F(1)\arrow[d,"\mu"]\\
 F(\OO(n)) & F(\OO(n)\otimes 1_\CC)\arrow[l, "F(\lambda)"]
 \end{tikzcd}
 \]
 Or, in other words, $\lambda_\DD = F(\lambda)\circ \mu(id\otimes \varepsilon)$.  On the other hand, by diagram (\ref{unitaux}) we have that $F(\lambda)\circ \mu = (F(\circ_i)\circ \mu)(id\otimes F(e))$, meaning that
 \[\lambda_\DD =  F(\circ_i)\circ \mu\circ (id\otimes F(e))\circ (id\otimes \varepsilon) = \circ_i^F(id\otimes  e_F)\]
 as we wanted to show.
 
 Next we need to show that the associativity axioms of operads hold for $F(\OO)$, we refer the reader to \cite[\S 1.1.2]{ward} to recall them. Let us first prove the one that does not involve the symmetry isomorphism. 
 
 Let $a_\CC$ and $a_\DD$ the associators for $\CC$ and $\DD$, respectively. We have the following commutative diagram from the associativity axioms of the operad $\OO$ for $i\leq j\leq i+m-1$.
 
 \[
\begin{tikzcd}
(\OO(n)\otimes \OO(m))\otimes \OO(l) \arrow[r, "a_\CC"]\arrow[d, "\circ_i\otimes"] & \OO(n)\otimes (\OO(m)\otimes\OO(l))\arrow[d, "id\otimes\circ_{j-i+1}"]\\
\OO(n+m-1)\otimes\OO(l)\arrow[d, "\circ_j"] & \OO(n)\otimes\OO(m+l-1)\arrow[dl, bend left = 10, "\circ_i"]\\
\OO(n+m+l-2) & 
\end{tikzcd} 
 \]
 
 Applying $F$ we obtain the following commutative diagram.
 
  \begin{equation}\label{assaux}
\begin{tikzcd}
F((\OO(n)\otimes \OO(m))\otimes \OO(l)) \arrow[r, "F(a_\CC)"]\arrow[d, "F(\circ_i\otimes id)"] & F(\OO(n)\otimes (\OO(m)\otimes\OO(l)))\arrow[d, "F(id\otimes\circ_{j-i+1})"]\\
F(\OO(n+m-1)\otimes\OO(l))\arrow[d, "F(\circ_j)"] & F(\OO(n)\otimes\OO(m+l-1))\arrow[dl, bend left = 10, "F(\circ_i)"]\\
F(\OO(n+m+l-2)) & 
\end{tikzcd} 
   \end{equation}
   
   According to the definition of $\circ_i^F$, we need to show that the following diagram commutes. 
   
   \begin{equation}\label{red}  
   \adjustbox{scale=0.95,center}{%
\begin{tikzcd}[column sep = 1.5em]
(F(\OO(n))\otimes F(\OO(m)))\otimes F(\OO(l)) \arrow[r, "a_\DD"] \arrow[d, "\mu\otimes id"] & F(\OO(n))\otimes (F(\OO(m))\otimes F(\OO(l)))\arrow[d, "id\otimes \mu"]\\
F(\OO(n)\otimes \OO(m))\otimes F(\OO(l))\arrow[d, red, "F(\circ_i)\otimes id"] & F(\OO(n))\otimes F(\OO(m)\otimes\OO(l))\arrow[d, red, "id\otimes F(\circ_{j+i-1})"]\\
F(\OO(n+m-1))\otimes F(\OO(l))\arrow[d, red, "\mu"] & F(\OO(n))\otimes F(\OO(m+l-1))\arrow[d, red, "\mu"]\\
F(\OO(n+m-1)\otimes\OO(l))\arrow[d, "F(\circ_j)"] & F(\OO(n)\otimes\OO(m+l-1))\arrow[dl, bend left = 10, "F(\circ_i)"]\\
F(\OO(n+m+l-2)) & 
\end{tikzcd}   }
   \end{equation}
   
   By naturality of $\mu$ we have 
   \begin{equation}\label{naturality}
   \mu\circ (F(\circ_i)\otimes id)= F( \circ_i \otimes id)\circ \mu 
   \end{equation} and \[\mu\circ (id \otimes F(\circ_{j-i+1}))= F(id\otimes \circ_{j-i+1})\circ \mu.\]
    Therefore we can replace the above compositions in diagram (\ref{red}) accordingly. We can also subdivide the above diagram into two by using $F(a_\CC)$ as follows.
   
      \[ 
      \adjustbox{scale=0.95,center}{%  
\begin{tikzcd}[column sep = 1em]
(F(\OO(n))\otimes F(\OO(m)))\otimes F(\OO(l)) \arrow[r, "a_\DD"] \arrow[d, "\mu\otimes id"] & F(\OO(n))\otimes (F(\OO(m))\otimes F(\OO(l)))\arrow[d, "id\otimes \mu"]\\
F(\OO(n)\otimes \OO(m))\otimes F(\OO(l))\arrow[d, red, "\mu"] & F(\OO(n))\otimes F(\OO(m)\otimes\OO(l))\arrow[d, red, "\mu"]\\
F((\OO(n)\otimes\OO(m))\otimes\OO(l))\arrow[d, red, "F(\circ_i\otimes id)"]\arrow[r, dashed, "F(a_\CC)"] & F(\OO(n))\otimes F(\OO(n)\otimes(\OO(m)\otimes\OO(l)))\arrow[d, red, "F(id\otimes \circ_{j-1+1})"]\\
F(\OO(n+m-1)\otimes\OO(l))\arrow[d, "F(\circ_j)"] & F(\OO(n)\otimes\OO(m+l-1))\arrow[dl, bend left = 10, "F(\circ_i)"]\\
F(\OO(n+m+l-2)) & 
\end{tikzcd}   }
   \]
   Now, the top diagram commutes because it is the associativity axiom of lax monoidal functors. The bottom diagram is precisely diagram (\ref{assaux}), so it commutes and we get the desired associativity axiom.
   
   Finally, we need to show that the associativity axioms involving the symmetry isomorphism hold for $F(\OO)$. Since they are analogous to each other, we only prove the first one.
   
   Let $B_\CC\coloneqq B_\CC^{X,Y}:X\otimes Y\to Y\otimes X$ the symmetry isomorphism on $\CC$ and similarly denote by $B_\DD$ the symmetry isomorphism on $\DD$.
   
   We have the following associativity commutative diagram for $j<i$.
   \[
   \begin{tikzcd}
   (\OO(n)\otimes \OO(m))\otimes\OO(l)\arrow[r, "a_\CC"]\arrow[d, "\circ_i\otimes id"] & \OO(n)\otimes (\OO(m)\otimes \OO(l)) \arrow[d, "id\otimes B_\CC"] \\
   \OO(n+m-1)\otimes\OO(l)\arrow[d, "\circ_j"] & \OO(n)\otimes (\OO(l)\otimes \OO(m))\arrow[d, "a^{-1}_\CC"]\\
   \OO(n+m+l-2) & (\OO(n)\otimes\OO(l))\otimes \OO(m)\arrow[d, "\circ_j\otimes id"]\\
    & \OO(n+l-1)\otimes \OO(m)\arrow[ul, "\circ_i", bend left = 10]
   \end{tikzcd}
   \]
   
   Applying $F$ we get the following commutative diagram.
   \begin{equation}\label{assiaux}
      \begin{tikzcd}
   F((\OO(n)\otimes \OO(m))\otimes\OO(l))\arrow[r, "F(a_\CC)"]\arrow[d, "F(\circ_i\otimes id)"] & F(\OO(n)\otimes (\OO(m)\otimes \OO(l))) \arrow[d, "F(id\otimes B^\CC)"]\\
   F(\OO(n+m-1)\otimes\OO(l))\arrow[d, "F(\circ_j)"] & \OO(n)\otimes (\OO(l)\otimes \OO(m))\arrow[d, "F(a_\CC)^{-1}"]\\
   F(\OO(n+m+l-2)) &  F((\OO(n)\otimes\OO(l))\otimes \OO(m))\arrow[d, "F(\circ_j\otimes id)"]\\
     &F(\OO(n+l-1)\otimes \OO(m))\arrow[ul, "F(\circ_i)", bend left = 10]
   \end{tikzcd}
   \end{equation}
   
   We need to show that the following diagram commutes.
   \[
   \adjustbox{scale=0.95,center}{%
\begin{tikzcd}[column sep = 1.5em]
(F(\OO(n))\otimes F(\OO(m)))\otimes F(\OO(l))\arrow[d, "\mu\otimes id"]\arrow[r, "a_\DD"]& F(\OO(n))\otimes (F(\OO(m))\otimes F(\OO(l)))\arrow[d, "id\otimes B_\DD"] \\
 F(\OO(n)\otimes\OO(n))\otimes F(\OO(l))\arrow[d, "F(\circ_i)\otimes id"]  & F(\OO(n))\otimes (F(\OO(l))\otimes F(\OO(m)))\arrow[d, "a_\DD^{-1}"]\\
 F(\OO(n+m-1)\otimes F(\OO(l))\arrow[d, "\mu\otimes id"] & (F(\OO(n))\otimes F(\OO(l)))\otimes F(\OO(m))\arrow[d, "\mu\otimes id"]\\
 F(\OO(m+n-1)\otimes\OO(l))\arrow[d, "F(\circ_j)"] & F(\OO(n)\otimes \OO(l))\otimes F(\OO(m))\arrow[d, "F(\circ_j)\otimes id"]\\
 F((\OO(n+m+l-2)) & F(\OO(n+l-1))\otimes F(\OO(m))\arrow[d, "\mu"]\\
  & F(\OO(n+l-1)\otimes\OO(m))\arrow[ul, "F(\circ_i)", bend left = 10]
\end{tikzcd}   }
   \]
   
   We use naturality of $\mu$, i.e \Cref{naturality}, as we have done before to rewrite some of the arrows. We also subdivide the diagram into two by factoring by $F(a_\CC)^{-1}\circ F(id\otimes B_\CC)\circ F(a_\CC)$.
   
   \[
   \adjustbox{scale=0.95,center}{%
\begin{tikzcd}[column sep = 1.5em]
(F(\OO(n))\otimes F(\OO(m)))\otimes F(\OO(l))\arrow[d, "\mu\otimes id"']\arrow[r, "a_\DD"]& F(\OO(n))\otimes (F(\OO(m))\otimes F(\OO(l)))\arrow[d, "id\otimes B_\DD"] \\
 F(\OO(n)\otimes\OO(n))\otimes F(\OO(l))\arrow[d, "\mu"'] & F(\OO(n))\otimes (F(\OO(l))\otimes F(\OO(m)))\arrow[d, "a_\DD^{-1}"]\\
 F((\OO(n)\otimes\OO(m))\otimes \OO(l))\arrow[d, "F(\circ_i\otimes id)"']\arrow[rdd, dashed, "F(a_\CC)^{-1}\circ F(id\otimes B_\CC)\circ F(a_\CC)", sloped, bend right = 5] & (F(\OO(n))\otimes F(\OO(l)))\otimes F(\OO(m))\arrow[d, "\mu\otimes id"]\\
 F(\OO(m+n-1)\otimes\OO(l))\arrow[d, "F(\circ_j)"'] & F(\OO(n)\otimes \OO(l))\otimes F(\OO(m))\arrow[d, "\mu"]\\
 F((\OO(n+m+l-2)) & F((\OO(n)\otimes\OO(m))\otimes \OO(m))\arrow[d, "F(\circ_j\otimes id)"]\\
  & F(\OO(n+l-1)\otimes\OO(m))\arrow[ul, "F(\circ_i)", bend left = 10]
\end{tikzcd}   }
   \]
   
   The bottom diagram commutes as it is precisely diagram (\ref{assiaux}). We decompose the top diagram as follows.
%   
%   \[
%\begin{tikzcd}[column sep = -3em]
% & F(\OO(n))\otimes (F(\OO(m))\otimes F(\OO(l)))\arrow[dr, "id\otimes B_\DD"]\arrow[ddddl, dashed, bend left = 30, sloped, "\mu\circ (id\otimes\mu)"]& \\
%(F(\OO(n))\otimes F(\OO(m)))\otimes F(\OO(l))\arrow[d, "\mu\otimes id"]\arrow[ur, "a_\DD"] &  & F(\OO(n))\otimes (F(\OO(l))\otimes F(\OO(m)))\arrow[d, "a_\DD^{-1}"]\arrow[ddddl, dashed, bend right = 35, sloped, "\mu\circ (id\otimes\mu)"]\\
%F(\OO(n)\otimes\OO(n))\otimes F(\OO(l))\arrow[d, "\mu"] & & (F(\OO(n))\otimes F(\OO(l)))\otimes F(\OO(m))\arrow[d, "\mu\otimes id"]\\
%F((\OO(n)\otimes\OO(m))\otimes \OO(l))\arrow[d, "F(a)"] & & F(\OO(n)\otimes \OO(l))\otimes F(\OO(m))\arrow[d, "\mu"]\\
%F(\OO(n)\otimes (\OO(m)\otimes\OO(l)))\arrow[dr,"F(id\otimes B_\CC)"] & & F((\OO(n)\otimes\OO(l))\otimes\OO(m))\\
%& F(\OO(n)\otimes (\OO(l)\otimes\OO(m)))\arrow[ur, "F(a_\CC)^{-1}"]&
%\end{tikzcd}  
%   \]
   
   
      \[
      \adjustbox{scale=0.95,center}{%
\begin{tikzcd}[column sep = 2.5em]
 F(\OO(n))\otimes (F(\OO(m))\otimes F(\OO(l)))\arrow[r, "id\otimes B_\DD"] \arrow[dddd, dashed, bend left = 80, sloped, "\mu\circ (id\otimes\mu)"']& F(\OO(n))\otimes (F(\OO(l))\otimes F(\OO(m)))\arrow[d, "a_\DD^{-1}"]\arrow[dddd, dashed, bend right = 80, sloped, "\mu\circ (id\otimes\mu)"]\\
   (F(\OO(n))\otimes F(\OO(m)))\otimes F(\OO(l))\arrow[d, "\mu\otimes id"]\arrow[u, "a_\DD"]& (F(\OO(n))\otimes F(\OO(l)))\otimes F(\OO(m))\arrow[d, "\mu\otimes id"]\\
 F(\OO(n)\otimes\OO(n))\otimes F(\OO(l))\arrow[d, "\mu"] & F(\OO(n)\otimes \OO(l))\otimes F(\OO(m))\arrow[d, "\mu"]\\
 F((\OO(n)\otimes\OO(m))\otimes \OO(l))\arrow[d, "F(a)"] & F((\OO(n)\otimes\OO(l))\otimes\OO(m))\\
 F(\OO(n)\otimes (\OO(m)\otimes\OO(l)))\arrow[r,"F(id\otimes B_\CC)"] & F(\OO(n)\otimes (\OO(l)\otimes\OO(m)))\arrow[u, "F(a_\CC)^{-1}"]\\
\end{tikzcd}  }
   \]
   
   
   The left and right subdiagrams commute because of the associativity axiom of lax monoidal functors. We decompose the central subdiagram further as
   
   \[
   \adjustbox{scale=0.95,center}{%
\begin{tikzcd}[column sep = 3em]
F(\OO(n))\otimes (F(\OO(m))\otimes F(\OO(l))\arrow[r, "id\otimes B_\DD"]\arrow[d, "id\otimes \mu"] & F(\OO(n))\otimes (F(\OO(l))\otimes F(\OO(m)))\arrow[d, "id\otimes \mu"]\\
F(\OO(n))\otimes F(\OO(m)\otimes\OO(l))\arrow[r, dashed,  "id\otimes F(B_\CC)"]\arrow[d, "\mu"] & F(\OO(n))\otimes F(\OO(l)\otimes\OO(m))\arrow[d, "\mu"]\\
F(\OO(n)\otimes (\OO(m)\otimes\OO(l)))\arrow[r, "F(id\otimes B_\CC)"] & F(\OO(n)\otimes (\OO(l)\otimes \OO(m)))
\end{tikzcd}}   
   \]
   
   
   The top part commutes because $F$ is symmetric lax monoidal and the bottom part commutes by naturality of $\mu$. This proves that $F(\OO)$ is an operad in $\DD$. 
   
   Lastly, we are only left with the proof that $F(f)$ is a map of operads. Since $f$ is a map of operads, we have for all $n$ the following commutative diagram.
   \[
\begin{tikzcd}[column sep = 5em]
\OO(n)\otimes \OO(m)\arrow[r, "f_n\otimes f_m"]\arrow[d, "\circ_i^\OO"] & \PP(n)\otimes\PP(m)\arrow[d, "\circ_i^\PP"]\\
\OO(n+m-1)\arrow[r, "f_{n+m-1}"] & \PP(n+m-1)
\end{tikzcd}   
   \]
   After applying $F$ we get the following commutative diagram.
   \begin{equation}\label{mapaux}
   \begin{tikzcd}[column sep = 5em]
F(\OO(n)\otimes \OO(m))\arrow[r, "F(f_n\otimes f_m)"]\arrow[d, "F(\circ_i^\OO)"] & F(\PP(n)\otimes\PP(m))\arrow[d, "F(\circ_i^\PP)"]\\
F(\OO(n+m-1))\arrow[r, "F(f_{n+m-1})"] & F(\PP(n+m-1))
\end{tikzcd} 
   \end{equation}
   
   We need to show that the following diagram commutes.
   \[
   \begin{tikzcd}[column sep = 5em]
F(\OO(n))\otimes F(\OO(m))\arrow[r, "F(f_n)\otimes F(f_m)"]\arrow[d,"\mu"] & F(\PP(n))\otimes F(\PP(m))\arrow[d, "\mu"]\\
F(\OO(n)\otimes \OO(m))\arrow[r, dashed, "F(f_n\otimes f_m)"]\arrow[d, "F(\circ_i^\OO)"] & F(\PP(n)\otimes\PP(m))\arrow[d, "F(\circ_i^\PP)"]\\
F(\OO(n+m-1))\arrow[r, "F(f_{n+m-1})"] & F(\PP(n+m-1))
\end{tikzcd}    
   \]
   The top subdiagram commutes because $\mu$ is natural and the bottom part is precisely diagram  (\ref{mapaux}), which commutes. This finishes the proof.
\end{proof}
%SAME WITH COOPERADS? PROBABLY BY ABSTRACT NONSENSE OF OPPOSITE CATEGORY

%\begin{itemize}


%\item S-MODULES/COLLECTIONS IN THE NON SYMMETRRIC CASE, OPERADS (SYMMETRIC AND NS)- POSSIBLY SEVERAL DEFINITIONS, AT LEAST CLASSICAL, PARTIAL AND THEIR EQUIVALENCE AND MONOIDAL, EXAMPLES, (INFINITESIMAL SEE 10.2.4) MODULE OVER AN OPERAD, QUASI-FREE OPERAD, PRE-LIE ALGEBRA DEFINED BY INSERTIONS, FORGETFUL FUNCTOR FROM SYM TO NS? SHOULD WRITE THE AXIOMS FOR GRADED OPERADS SOMEWHERE (MORE GENERALLY OPERAD IN SYMMETRIC MONOIDAL CATEGORY LIKE WARD), AND I MIGHT ALSO WRITE THE PROOF THAT THIS IS AN OPERAD (BUT SHOULD FOLLOW FROM LAX MONOIDALITY)




%\item INFINITESIMAL COMPOSITION? THIS CAN PROBABLY BE LEFT WHERE IT IS BECAUSE IT IS INTRODUCED SOLELY FOR INFTY-MORPHISMS AND NEVER USED AGAIN


%\item KOSZUL DUALITY, CONVOLUTION OPERAD, TWISTED MORPHISM?

%\item INFINITY OPERADS? 

%\item OPERADIC COHOMOLOGY OR JUST INTRODUCE COHOMOLOGY AD HOC FOR THE CASES I NEED? LIKE ADD THE DEFINITION OF HOCHSCHILD COHOMOLOGY WHEN DEFINING THE HOCHSCHILD COMPLEX

%\item BAR-COBAR  CONSTRUCTION FOR ALGEBRAS AND OPERADS?
%\end{itemize}
\section{Base categories and totalization}\label{categories}
%MAYBE CHANGE THE TITLE HERE TO "SOME USEFUL  CATEGORIES AND CONVENTIONS" OR SMTH LIKE THAT

%Now  we move to the background and conventions that we need in order to study derived $A_\infty$-algebras. 
%Now we introduce some categories over which we will be working, specially in order to study derived $A_\infty$-algebras. 
%We collect some preliminary definitions. 

%Let us start by  fixing some notation and conventions. Fix a commutative ring with unit $R$ of characteristic distinct from $2$. All tensor products taken over $R$. %COPY SECTION 2.2 OF DAINFTY AND THEIR HOMOTOPIES AS BACKGROUND, REFERENCES TO WHITEHOUSE. INCLUDING DEFINITION OF TWISTED COMPLEX

%REDEFINE THINGS ACCORDING TO CONVENTIONS

Now introduce some categories and conventions that we need in order to study derived $A_\infty$-algebras. Many results of $A_\infty$-algebras need $R$ to be a field because of projectivity, see \Cref{minimalmodels}. Thus, it is necessary to build in projective resolutions. In particular we need an extra degree compatible with derived $A_\infty$-setting. In order to do that, we need a way to connect a single graded category with a bigraded category. This is usually done through totalization. But in order to properly translate $A_\infty$-algebras into totalized derived $A_\infty$-algebras we need to go through several suitably enriched categories that are defined in this section. Most of the definitions come from \cite[\S 2]{whitehouse} but we adapt them here to our conventions.

%Let $\CC$ be a category and let $A$, $B$ be arbitrary
%objects in $\CC$. We denote by $\Hom_\CC(A,B)$ the set of morphisms from $A$ to $B$ in $\CC$. If $(\CC,⊗, 1)$ is
%symmetric monoidal closed, then we denote its internal hom-object by $[A,B] ∈ \CC$.

\subsection{Filtered modules and complexes}
%INCLUSION IS REVERSED
First, we collect some definitions about filtered modules and filtered complexes. Filtrations will allow to add an extra degree to single-graded objects that will be necessary in order to connect them with bigraded objects.
\begin{defin}
A \emph{filtered $R$-module} $(A, F)$ is an $R$-module $A$ that can be decomposed as a union $A = \bigcup_{p\in\Z} F_pA$ of submodules indexed by the integers satisying $F_{p}A ⊆ F_{p-1}A$ for all $p ∈ \Z$. The assignment $F:p\mapsto F_pA$ is called a \emph{filtration}.

A \emph{morphism of filtered modules} is a morphism $f : A \to B$ of $R$-modules which is compatible with filtrations, namely 
\[f(F_pA) ⊂ F_pB \text{ for all }p ∈ \Z.\]
\end{defin}
%\begin{defin}
%A \emph{filtered $R$-module} $(A, F)$ is given by a family of $R$-modules $\{F_pA\}_{p∈\Z}$ indexed by
%the integers such that $F_{p}A ⊆ F_{p-1}A$ for all $p ∈ \Z$ and $A = \bigcup_p F_pA$. 
%
%A \emph{morphism of filtered modules} is a morphism $f : A \to B$ of $R$-modules which is compatible with filtrations, namely 
%\[f(F_pA) ⊂ F_pB \text{ for all }p ∈ \Z.\]
%\end{defin}
Note that some other sources may reverse inclusions in the above definition and consider $F_{p}A ⊆ F_{p+1}A$ instead. We denote by $\mathrm{C}_R$ the category of cochain complexes of $R$-modules.
\begin{defin}\label{filteredcomplex}
A \emph{filtered complex} $(K, d, F)$ is given by a cochain complex $(K, d) ∈ \mathrm{C}_R$ coupled with a filtration $F$ applied to each $R$-module $K^n$ satisfying the condition $d(F_pK^n) ⊂ F_pK^{n+1}$ for all integers $p$ and $n$. A \emph{morphism} of filtered complex is morphism of complexes $f : K → L$ that is compatible with filtrations, i.e., \[f(F_pK) ⊂ F_pL\text{ for all }p ∈ \Z.\]
\end{defin}

%\begin{defin}\label{filteredcomplex}
%A \emph{filtered complex} $(K, d, F)$ is a cochain complex $(K, d) ∈ \mathrm{C}_R$ together with a filtration $F$ of each $R$-module $K^n$ such that $d(F_pK^n) ⊂ F_pK^{n+1}$ for all $p, n ∈ \Z$. A \emph{morphism} of filtered complexes is given by a morphism of complexes $f : K → L$ compatible with filtrations, namely \[f(F_pK) ⊂ F_pL\text{ for all }p ∈ \Z.\]
%\end{defin}

We denote by $\fmod$ and $\fc$ the categories of filtered modules and filtered complexes of $R$-modules, respectively. 

%I reversed inclusion from the original sourcce. As a consequence I used F_1 instead of F_{-1}. I do it because the maps preserving filtratiion will be in level 0 and I want the whole A_\infty operad to be mapped to that level, so R must be in level 0 as well

\begin{defin}\label{filteredtensor}
The \emph{tensor product of two filtered $R$-modules} $(A, F)$ and $(B, F)$  is defined as the filtered $R$-module $A\otimes B$ with the filtration
 \[F_p(A ⊗ B) :=\sum_{i+j=p}\Ima(F_iA ⊗ F_jB → A ⊗ B).\] %image so that we have have inclusions
This turns the category of filtered $R$-modules into a symmetric monoidal category, where the unit is given by $R$ with the trivial filtration \[0 = F_{1}R ⊂ F_0R = R.\]
\end{defin}


\begin{defin}\label{filterend}
Let $K$ and $L$ be filtered complexes. We define $\underline{\Hom}(K,L)$ to be the filtered complex whose underlying cochain complex is $\Hom_{\mathrm{C}_R}(K,L)$ and the filtration $F$ given by 
\[F_p\underline{\Hom}(K,L)=\{f:K\to L\mid f(F_qK)\subset F_{q+p}L\text{ for all }q ∈ \Z\}.\]
In particular, $\Hom_{\fmod}(K,L)=F_0\underline{\Hom}(K,L)$.
\end{defin}

\subsection{Bigraded modules, vertical bicomplexes, twisted complexes and sign conventions}\label{bigradedbackground}



We collect some basic definitions of bigraded categories that we need to use, and we establish some conventions.


\begin{defin}
We work with $(\Z,\Z)$-bigraded
$R$-modules $A = \{A^j_i\}$, where $(i, j)$ is called the \emph{bidegree} of the elements of $A^j_i$. We may refer to $i$
as the \emph{horizontal degree} and $j$ as the \emph{vertical degree}. The \emph{total degree} of an element $x ∈ A^j_i$ is $i+j$ and is sometimes denoted by $|x|$.
\end{defin}
%\begin{defin}
%A \emph{morphism of bidegree $(p, q)$} maps $A^j_i$ to $A^{j+q}_{i+p}$. The tensor product of two bigraded $R$-modules $A$
%and $B$ is the bigraded $R$-module $A ⊗ B$ given by
%\[(A ⊗ B)^j_i \coloneqq\bigoplus_{p,q}A^q_p ⊗ B^{j−q}_{i−p} .\]
%\end{defin}
\begin{defin}
A \emph{morphism of bidegree $(p, q)$} is a morphism of modules that takes $A^j_i$ to $A^{j+q}_{i+p}$. The tensor product of two bigraded $R$-modules $A$ and $B$ is the bigraded $R$-module $A ⊗ B$ defined in terms of bidegree components as \[(A ⊗ B)^j_i \coloneqq \bigoplus_{p,q}A^q_p ⊗ B^{j−q}_{i−p}.\]
\end{defin}
We denote by $\bgmod$ the category whose objects are bigraded $R$-modules and whose morphisms
are morphisms of bigraded $R$-modules of bidegree $(0, 0)$. It is symmetric monoidal with the above
tensor product.

We introduce a scalar product for bidegrees, where for elements $x$ and $y$ with bidegrees $(x_1, x_2)$ and $(y_1, y_2)$, respectively, the notation $\langle x, y\rangle = x_1y_1 + x_2y_2$ is employed.

The symmetry isomorphism \[τ_{A⊗B} : A ⊗ B → B ⊗ A\] is defined as \[x ⊗ y \mapsto (-1)^{\langle x,y\rangle}y ⊗ x.\] Adhering to the Koszul sign rule, if $f : A → B$ and $g : C → D$ are bigraded morphisms, then the morphism $f ⊗ g : A ⊗ C → B ⊗ D$ is evaluates as \[(f ⊗ g)(x ⊗ z) \coloneqq (-1)^{\langle g,x\rangle}f(x) ⊗ g(z).\]

\begin{defin}
A \emph{vertical bicomplex} $(A,d)$ is a bigraded $R$-module $A$ together with a \emph{vertical differential} $d^A : A → A$ of bidegree $(0, 1)$. A \emph{morphism of vertical bicomplexes} is a morphism of bigraded modules
of bidegree $(0, 0)$ that commutes with the vertical differential.
\end{defin}

%We denote by $\vbc$ the category of vertical bicomplexes. The tensor product of two vertical bicomplexes $A$ and $B$ is given by endowing the tensor product of underlying bigraded modules with
%vertical differential \[d^{A⊗B} := d^A ⊗ 1 + 1 ⊗ d^B : (A ⊗ B)^v_u → (A ⊗ B)^{v+1}_u .\] This makes $\vbc$ into a
%symmetric monoidal category. The symmetric monoidal categories $(\mathrm{C}_R,⊗,R)$, $(\bgmod,⊗,R)$ and $(\vbc,⊗,R)$ are related by embeddings $\mathrm{C}_R\to\vbc$ and $\bgmod \to\vbc$ which are monoidal and full. 

We write $\vbc$ to denote the category of vertical bicomplexes. The tensor product of two vertical bicomplexes $A$ and $B$ is obtained by providing the tensor product of their underlying bigraded modules with a vertical differential \[d^{A⊗B} := d^A ⊗ 1 + 1 ⊗ d^B : (A ⊗ B)^v_u → (A ⊗ B)^{v+1}_u.\] This structure transforms $\vbc$ into a symmetric monoidal category. The symmetric monoidal categories $(\mathrm{C}_R,⊗,R)$, $(\bgmod,⊗,R)$, and $(\vbc,⊗,R)$ are interconnected through embeddings $\mathrm{C}_R\to\vbc$ and $\bgmod \to\vbc$, both being monoidal and full.

\begin{defin}\label{delta1}
Consider bigraded modules $A$ and $B$. The notation $[A,B]^∗_∗$ designates the bigraded module of morphisms from $A$ to $B$. Additionally, for vertical bicomplexes $A$ and $B$, and for $f ∈ [A,B]^v_u$, the operator $δ(f)$ is defined as \[δ(f) := d^Bf − (-1)^vfd^A.\]
\end{defin}

\begin{lem}
If $A$, $B$ are vertical bicomplexes, then $([A,B]^∗_∗
, δ)$ is a vertical bicomplex.
\end{lem}
\begin{proof}
Direct computation shows $\delta^2=0$.
\end{proof}


%the horizontal degree sign is changed to fit in the total degree convention
\begin{defin}\label{twistedcomplex} A \emph{twisted complex} $(A, d_m)$ is a bigraded $R$-module $A = \{A^j_i \}$ equipped with a family
of morphisms $\{d_m : A → A\}_{m≥0}$ of bidegree $(m,1−m )$ satisfying for all $m ≥ 0$ the equation
%some sources have (-1)^j, but they're equivalent and this is easier for me.
\[\sum_{i+j=m}(−1)^id_id_j = 0.\]

\end{defin}

\begin{defin}\label{twistedmorphisms}
Let $(A, d^A_m)$ and $(B, d^B_m)$ be twisted complexes. A \emph{morphism of twisted complexes} $f : (A, d^A_m) → (B, d^B_m)$ is given by a family of morphisms of $R$-modules $\{f_m : A → B\}_{m≥0}$ of bidegree $(m,−m)$ satisfying for all $m ≥ 0$ the compatibility condition
\[\sum_{i+j=m}d^B_if_j =\sum_{i+j=m}(−1)^if_id^A_j.\]
The composition of morphisms is given by $(g \circ f)_m :=\sum_{i+j=m} g_if_j$.

A morphism $f = \{f_m\}_{m≥0}$ is called \emph{strict} if $f_i = 0$ for all $i > 0$. The \emph{identity} morphism $1_A : A → A$ is the strict morphism
given by $(1_A)_0(x) = x.$ 
\end{defin}
It is not hard to see that a morphism $f = \{f_i\}$ is an isomorphism of twisted complexes if and only if $f_0$ is an isomorphism of bigraded $R$-modules. Note that if $f$ is an isomorphism, then an inverse of $f$ is obtained from an inverse of $f_0$ by solving a triangular system of linear equations.

Denote by $\tc$ the category of twisted complexes. The following construction endows $\tc$ with a symmetric monoidal structure, see \cite[Lemma 3.3]{whitehouse} for a proof.
%\begin{lem}\label{tensortwisted}
%The category $(\tc,⊗,R)$ is symmetric monoidal, where the monoidal structure is given
%by the bifunctor
%
%\[⊗ : \tc × \tc → \tc.\]
%
%On objects it is given by \[((A, d^A_m), (B, d^B_m)) → (A ⊗ B, d^A_m ⊗ 1 + 1 ⊗ d^B_m)\] and on morphisms it is
%given by $(f, g) → f ⊗ g$, where \[(f ⊗ g)_m :=\sum_{i+j=m} f_i ⊗ g_j.\] In particular, by the Koszul sign rule we
%have that \[(f_i ⊗g_j)(x⊗z) = (−1)^{\langle g_j ,x\rangle}f_i(x)⊗g_j(z).\] The symmetry isomorphism is given by the strict
%morphism of twisted complexes
%
%\begin{align*}
%τ_{A⊗B} \colon &A ⊗ B → B ⊗ A\\
%&x ⊗ y\mapsto (−1)^{\langle x,y\rangle}y ⊗ x.
%\end{align*}
%\end{lem}
\begin{lem}\label{tensortwisted}
The category $(\tc,⊗,R)$ is endowed with a symmetric monoidal structure, where the bifunctor \[⊗ : \tc × \tc → \tc\] operates on objects according to \[((A, d^A_m), (B, d^B_m)) → (A ⊗ B, d^A_m ⊗ 1 + 1 ⊗ d^B_m)\] and on morphisms as $(f, g) → f ⊗ g$, with \[(f ⊗ g)_m :=\sum_{i+j=m} f_i ⊗ g_j.\] 
 In particular, the Koszul sign rule implies
 \[(f_i ⊗g_j)(x⊗z) = (−1)^{\langle g_j ,x\rangle}f_i(x)⊗g_j(z).\] The symmetry isomorphism is given by the strict
morphism of twisted complexes

\begin{align*}
τ_{A⊗B} \colon &A ⊗ B → B ⊗ A\\
&x ⊗ y\mapsto (−1)^{\langle x,y\rangle}y ⊗ x.
\end{align*}
\end{lem}

The internal hom on bigraded modules can be extended to twisted complexes via the following lemma whose proof is in \cite[Lemma 3.4]{whitehouse}.
\begin{lem}\label{di} Consider twisted complexes $A$ and $B$. For $f ∈ [A,B]^v_u$, defining \[(d_if) := (-1)^{i(u+v)}d^B_if − (-1)^vfd^A_i\] for $i ≥ 0$ endows $[A,B]^∗_∗$ with the structure of a twisted complex.
\end{lem}
%I STILL DON'T  KNOW WHERE TO PUT WHAT COMES NEXT, EITHER HERE OR BEFORE OPERADIC TOTALIZATION

\subsection{Totalization}\label{total}
%DEFINITION WITH FILTRATION, DIFFERENTIAL, MONOIDALITY
Here we recall the definition of the totalization functor from \cite{whitehouse} and some of the structure that it comes with. This functor and its enriched versions are key to establish a correspondence between $A_\infty$-algebras and derived $A_\infty$-algebras.

%\pagebreak
\begin{defin}\label{totdef}
The \emph{totalization} of a bigraded $R$-module $A = \{A^j_i \}$ is the graded $R$-module $\Tot(A)$ determined by
\[\Tot(A)^n \coloneqq
\bigoplus_{i<0}A^{n-i}_i ⊕\prod_{i\geq 0}A^{n-i}_i .\]
The \emph{column filtration} of $\Tot(A)$ is the filtration defined as \[F_p\Tot(A)^n \coloneqq\prod_{i\geq p} A^{n-i}_i .\]
\end{defin}

For a twisted complex $(A, d_m)$, define a map $d : \Tot(A) → \Tot(A)$ of degree $1$ by setting \[d(x)_j \coloneqq \sum_{m≥0}(-1)^{mn}d_m(x_{j-m})\] for $x = (x_i)_{i∈\Z} ∈ \Tot(A)^n$. Here, $x_i ∈ A^{n-i}_i$ represents the $i$-th component of $x$, and $d(x)_j$ denotes the $j$-th component of $d(x)$. It is important to note that, for a given $j ∈ \Z$, there exists a sufficiently large $m ≥ 0$ such that $x_{j-m′} = 0$ for all $m′ ≥ m$. Consequently, $d(x)_j$ is expressed as a finite sum. Moreover, for sufficiently large negative $j$, one observes that $x_{j-m} = 0$ for all $m ≥ 0$, leading to $d(x)_j = 0$.

For a morphism $f : (A, d_m) → (B, d_m)$ of twisted complexes, define the \emph{totalization of $f$} be the map $\Tot(f) : \Tot(A) → \Tot(B)$ of degree 0 given by
\[(\Tot(f)(x))_j \coloneqq \sum_{m≥0}(−1)^{mn}f_m(x_{j-m})\]
 for $x = (x_i)_{i∈\Z} ∈ \Tot(A)^n$.
 
 The following is \cite[Theorem 3.8]{whitehouse}.
\begin{thm}
The assignments $(A, d_m) \mapsto (\Tot(A), d, F)$, where $F$ is the column filtration of $\Tot(A)$,
and $f \mapsto \Tot(f)$ define a functor $\Tot : \tc \to \fc$ which is an isomorphism when restricted to its image.
\end{thm}

For a filtered complex of the form $(\Tot(A),d,F)$ where $A = \{A^j_i \}$ is a bigraded $R$-module, we can recover the twisted complex structure on  $A$ as follows. For all $m ≥ 0$, let
$d_m : A → A$ be the morphism of bidegree $(m,1-m)$ defined by 
\[d_m(x) = (−1)^{nm}d(x)_{i+m},\] 
where $x ∈ A^{n-i}_i$ and $d(x)_k$ denotes the $k$-th component of $d(x)$. Note that $d(x)_k$ lies in $A^{n+1-k}_k$. We can also recover the bidegree of an element because in totalization they are sorted by horizontal degree.

We will consider the following bounded categories since the totalization functor has better monoidal properties when restricted to them. 

\begin{defin}
We let $\tc^b$, $\vbc^b$ and $\bgmod^b$ be the full subcategories of \emph{horizontally bounded on the right} graded twisted
complexes, vertical bicomplexes and bigraded modules respectively. This means that if $A=\{A^j_i\}$ is an object of any of these categories, then there exists $i$ such that $A^j_{i'}=0$ for $i'>i$.
We let $\fmod^b$ and $\fc^b$ be the full subcategories of bounded filtered modules, respectively complexes, i.e.
the full subcategories of objects $(K, F)$ such that there exists some $p$ with the property that $F_{p'}K^n = 0$ for all $p'>p$. We refer to all of these as the \emph{bounded subcategories} of $\tc$, $\vbc$, $\bgmod$, $\fmod$ and $\fc$ respectively.
\end{defin}

The following is \cite[Proposition 3.11]{whitehouse}.

\begin{propo}\label{monoidal}
The totalization functors $\Tot : \bgmod → \fmod$ and $\Tot : \tc → \fc$ are lax symmetric
monoidal with structure maps
\[\epsilon : R → \Tot(R)\text{ and }\mu=μ_{A,B} : \Tot(A) ⊗ \Tot(B) → \Tot(A ⊗ B)\]
given by $\epsilon = 1_R$. For $x = (x_i)_i ∈ \Tot(A)^{n_1}$ and  $y=(y_j)_j ∈ \Tot(B)^{n_2}$,
\begin{equation}\label{mu1}
μ(x ⊗ y)_k \coloneqq
\sum_{k_1+k_2=k}(−1)^{k_1n_2}x_{k_1} ⊗ y_{k_2} .
\end{equation}

When restricted to the bounded case, $\Tot : \bgmod^b
 → \fmod^b$ and $\Tot : \tc^b → \fc^b$ are
strong symmetric monoidal functors.
\end{propo}

Using totalization it is possible to show the following, which is \cite[Lemma 4.15]{whitehouse}.

\begin{lem}\label{4.15}
The category $\fc$ is monoidal over $\vbc$. By restriction, $\fmod$ is monoidal over $\bgmod$.
\end{lem}

\begin{remark}\label{heuristic}
There is a certain heuristic to obtain the sign appearing in the definition of $\mu$ in \Cref{monoidal}. In the bounded case, we can write \[\Tot(A)=\bigoplus_i A_i^{n-i}.\]
As direct sums commute with tensor products, we have
\[\Tot(A)\otimes\Tot(B)=(\bigoplus A_i^{n-i})\otimes \Tot(B)\cong \bigoplus_i  (A_i^{n-i}\otimes \Tot(B)).\]

In the isomorphism above we can interpret that each $A_i^{n-i}$ passes by $\Tot(B)$. Since $\Tot(B)$ uses total grading, we can think of this degree as being the horizontal degree, while having 0 vertical degree. Thus, using the Koszul sign rule we would get precisely the sign from \Cref{monoidal}. This explanation is just an intuition, and opens the door for other possible sign choices: what if we decide to distribute $\Tot(A)$ over $\bigoplus_i B_i^{n-i}$ instead, or if we consider the total degree as the vertical degree? These alternatives lead to other valid definitions of $\mu$, and we will explore the consequences of some of them in \Cref{othermu}.
\end{remark}

\begin{lem}\label{mui}
In the conditions of \Cref{monoidal} for the bounded case, the inverse
\[\mu^{-1}:\Tot(A_{(1)}\otimes\cdots\otimes A_{(m)})\to \Tot(A_{(1)})\otimes\cdots\otimes \Tot(A_{(m)})\]
is given on pure tensors (for notational convenience) as
\begin{equation}\label{mu}
\mu^{-1}(x_{(1)}\otimes\cdots\otimes x_{(m)})=(-1)^{\sum_{j=2}^m n_j\sum_{i=1}^{j-1}k_i}x_{(1)}\otimes\cdots\otimes x_{(m)},
\end{equation}
where $x_{(l)}\in (A_{(m)})_{k_l}^{n_l-k_l}$.
\end{lem}
\begin{proof}
For the case $m=2$,
\[\mu^{-1}:\Tot(A\otimes B)\to \Tot(A)\otimes \Tot(B)\]
is computed explicitly as follows.
Let  $c\in\Tot(A\otimes B)^n$. By definition, we have
\[\Tot(A\otimes B)^n=\bigoplus_k (A\otimes B)^{n-k}_k=\bigoplus_k\underset{n_1+n_2=n}{\bigoplus_{k_1+k_2=k}}A_{k_1}^{n_1-k_1}\otimes B_{k_2}^{n_2-k_2}.\]
And thus, $c=(c_k)_k$ may be written as a finite sum $c=\sum_k c_k$, where 
\[c_k=\underset{n_1+n_2=n}{\sum_{k_1+k_2=k}}x_{k_1}^{n_1-k_1}\otimes y_{k_2}^{n_2-k_2}.\]
Here, we introduced superscripts to indicate the vertical degree, which, unlike in the definition of $\mu$ (\Cref{mu1}), is not solely determined by the horizontal degree since the total degree also varies. However we are going to omit them in what follows for simplicity of notation. Distributivity allows us to rewrite $c$ as
\begin{align*}
c&=\sum_k \underset{n_1+n_2=n}{\sum_{k_1+k_2=k}}x_{k_1}\otimes y_{k_2}\\&=\sum_{n_1+n_2=n}\sum_{k_1}\sum_{k_2}(x_{k_1}\otimes y_{k_2})\\
&=\sum_{n_1+n_2=n}\left(\sum_{k_1}x_{k_1}\right)\otimes\left(\sum_{k_2}y_{k_2}\right).
\end{align*}
Therefore, $\mu^{-1}$ can be defined as
\[\mu^{-1}(c)=\sum_{n_1+n_2=n}\left(\sum_{k_1}(-1)^{k_1n_2}x_{k_1}\right)\otimes\left(\sum_{k_2}y_{k_2}\right).\]

The general case follows inductively.
\end{proof}
\pagebreak
\section{Enriched categories and totalization}



We define here some useful enriched categories and collect results from \cite[\S 4.3 and \S 4.4]{whitehouse}. Each of these categories will be a piece in \Cref{whitehouse}, which establishes a connection between $A_\infty$-algebras and derived $A_\infty$-algebras. Some of them have been modified according to our conventions. 

The following definition, extracted from \cite[Definition 3.32]{whitehouse}, provides an alternative enrichment to the category of bigraded modules. From this, similar enrichments can be provided to other categories built on top of this one.
\begin{defin}\label{weirdenrichment}
Let $A,B$ and $C$ be bigraded modules. We denote by $\underline{\mathpzc{bgMod}_R}(A,B)$ the bigraded module given by
\[\underline{\mathpzc{bgMod}_R}(A,B)^v_u :=\prod_{j≥0}[A,B]^{v−j}_{u+j}\]
where $[A,B]$ is the internal hom-object of bigraded modules. More precisely, $g ∈ \underline{\mathpzc{bgMod}_R}(A,B)^v_u$ is given
by $g := (g_0, g_1, g_2, \dots )$, where $g_j : A → B$ is a map of bigraded modules of bidegree $(u + j, v − j)$.

Furthermore, we can define a composition morphism
\[c : \underline{\mathpzc{bgMod}_R}(B,C) ⊗ \underline{\mathpzc{bgMod}_R}(A,B) → \underline{\mathpzc{bgMod}_R}(A,C)\]
by
\[c(f, g)_m :=\sum_{i+j=m}(−1)^{i|g|}f_ig_j .\]
\end{defin}

\begin{defin}\label{delta2}
Let $(A, d^A_i)$ and $(B, d^B_i)$ be twisted complexes and consider $f ∈ \underline{\mathpzc{bgMod}_R}(A,B)^v_u$. Consider also $d^A :=(d^A_i)_i ∈ \underline{\mathpzc{bgMod}_R}(A,A)^1_0$
and $d^B := (d^B_i)_i ∈ \underline{\mathpzc{bgMod}_R}(B,B)^1_0$. We define

\[δ(f) := c(d^B, f) − (−1)^{\langle f,d^A\rangle}c(f, d^A) ∈ \underline{\mathpzc{bgMod}_R}(A,B)^{v+1}_u.\]
%where $\langle f, d^A\rangle$ is the scalar product for the bidegrees and $c$ is the composition morphism described in \Cref{weirdenrichment} 
More explicitly,
\[(δ(f))_m :=\sum_{i+j=m}(−1)^{i|f|}d^B_if_j − (−1)^{v+i}f_id^A_j.\]
\end{defin}

The following lemma justifies the above definition. For a proof see \cite[Lemma 4.18]{whitehouse}.

\begin{lem}
The following equations hold.
\begin{align*}
&c(d^A, d^A) = 0\\
&δ^2 = 0\\
&δ(c(f, g)) = c(δ(f), g) + (−1)^v c(f, δ(g))
\end{align*}
where $v$ is the vertical degree of $f$. Furthermore, $f ∈ \ubgMod(A,B)$ is a map of twisted complexes if and
only if $δ(f) = 0$. In particular, $f$ is a morphism in $\tc$ if and only if the bidegree of $f$ is $(0, 0)$ and
$δ(f) = 0$. Moreover, given two morphisms $f$ and $g$ in $\tc$, we have that $c(f, g) = f\circ g$, where the latter is the
composition in $\tc$.
\end{lem}

\begin{defin}
Let $A$ and $B$ be twisted complexes. We define $\underline{t\mathcal{C}_R}(A,B)$ to be the vertical bicomplex
$\underline{t\mathcal{C}_R}(A,B) := (\underline{\mathpzc{bgMod}_R}(A,B), δ)$.
\end{defin}

\begin{defin}\label{ubgMod}
We denote by $\ubgMod$ the \emph{$\bgmod$-enriched category of bigraded modules} given
by the following data.

\begin{enumerate}[(1)]
\item The objects of $\ubgMod$ are bigraded modules.
\item For any bigraded modules $A$ and $B$ the hom-object is the bigraded module $\ubgMod(A,B)$.
\item The composition map \[c:\ubgMod(B,C) ⊗ \ubgMod(A,B) → \ubgMod(A,C)\] is defined in \Cref{weirdenrichment}.
\item The unit morphism $R → \ubgMod(A,A)$ is the morphism of bigraded modules that
sends $1 ∈ R$ to $1_A : A → A$, the strict morphism given by the identity of $A$.
\end{enumerate}
\end{defin}

\begin{defin}\label{utC}
We denote by $\utC$ the \emph{$\vbc$-enriched category of twisted complexes} given by the following data.
\begin{enumerate}[(1)]
\item The objects of $\utC$ are twisted complexes.
\item For any twisted complexes $A$ and $B$ the hom-object is the vertical bicomplex $\utC(A,B)$.
\item The composition map $c : \utC(B,C)⊗\utC(A,B) → \utC(A,C)$ is defined in \Cref{weirdenrichment}.
\item The unit morphism $R → \utC(A,A)$ is given by the morphism of vertical bicomplexes sending
$1 ∈ R$ to $1_A : A → A$, the strict morphism of twisted complexes given by the identity of $A$.
\end{enumerate}
\end{defin}


The next tensor corresponds to $\underline{\otimes}$ in the categorical setting of \Cref{underline}, see \cite[Lemma 4.27]{whitehouse}.


\begin{lem}\label{tensorenriched}
The monoidal structure of $\utC$ is given by the following map of vertical bicomplexes.
\[\underline{⊗}: \utC(A,B) ⊗ \utC(A′,B′) → \utC(A ⊗ A′,B ⊗ B′)\]
\[(f, g) → (f\underline{⊗}g)_m :=\sum_{i+j=m}(−1)^{ij}f_i ⊗ g_j\]
The monoidal structure of $\ubgMod$ is given by the restriction of this map.
\end{lem}




\begin{defin}\label{ufMod}
We denote by $\ufMod$ the \emph{$\bgmod$-enriched category of filtered modules} given by the following data.
%I keep j+u from the original source so that degrees match in enriched totalization. Alternatively one can keep v-u
\begin{enumerate}[(1)]
\item The objects of $\ufMod$ are filtered modules.
\item For any filtered modules $(K, F)$ and $(L, F)$, the bigraded module $\ufMod(K,L)$ is given by
\[\ufMod(K,L)^v_u :=\{f: K → L | f(F_qK^m) ⊂ F_{q+u}L^{m+u+v}, ∀m, q ∈ \Z\}.\]
%\end{enumerate}
%\begin{enumerate}
%\setcounter{enumi}{2}
\item The composition map is given by $c(f, g) = (−1)^{u|g|}fg$, where $u$ is the horizontal degree of $f$.
\item The unit morphism is given by the map $R → \ufMod(K,K)$ sending $1 → 1_K$.
\end{enumerate}
\end{defin}


\begin{defin}\label{fmoddifferential}
Let $(K, d^K, F)$ and $(L, d^L, F)$ be filtered complexes. We define $\ufC(K,L)$ to be the
vertical bicomplex whose underlying bigraded module is $\ufMod(K,L)$ and the vertical differential is
\[δ(f) := c(d^L, f) − (−1)^{\langle f,d^K\rangle}c(f, d^K) = d^Lf − (−1)^{|f|}fd^K\] %= d^Lf − (−1)^{v+u}fd^K 
for $f ∈ \ufMod(K,L)^v_u$, where $c$ is the composition map from \Cref{ufMod}.
\end{defin}


\begin{defin}\label{ufC}
The \emph{$\vbc$-enriched category of filtered complexes} $\ufC$ is the enriched category given
by the following data.
\begin{enumerate}[(1)]
\item The objects of $\ufC$ are filtered complexes.
\item For $K,L$ filtered complexes the hom-object is the vertical bicomplex $\ufC(K,L)$.
\item The composition map is given as in $\ufMod$ in \Cref{ufMod}. 
\item The unit morphism is given by the map $R → \ufC(K,K)$ sending $1 → 1_K$.
\end{enumerate}
We write $\usfC$ to denote the full subcategory of $\ufC$ whose objects are in the image of the totalization functor.
\end{defin}

The enriched monoidal structure is given as follows and can be found in \cite[Lemma 4.36]{whitehouse}.
\begin{defin}\label{tensorenriched2}
The monoidal structure of $\ufC$ is given by the following map of vertical bicomplexes.
\[\underline{⊗}: \ufC(K,L) ⊗ \ufC(K′,L′) → \ufC(K ⊗ K′,L ⊗ L′),\]
\[(f, g) \mapsto f\underline{⊗}g := (−1)^{u|g|}f ⊗ g\]
Here $u$ is the horizontal degree of $f$.
\end{defin}


The proof of the following lemma is included in the proof of \cite[Lemma 4.35]{whitehouse}.
\begin{lem}\label{adjunction}
Let $A$ be a vertical bicomplex that is horizontally bounded on the right and let $K$ and $L$ be filtered complexes. There is a natural bijection
\[\Hom_{\fc}(\Tot(A)\otimes K,L)\cong \Hom_{\vbc}(A,\ufC(K,L))\]
given by
$f\mapsto \tilde{f}: a\mapsto (k\mapsto f(a\otimes k))$.
\end{lem}

We now define an enriched version of the totalization functor. 
\begin{defin}\label{enrichedtot}
Let $A$ and $B$ be bigraded modules. We define

\[\Tot(f) ∈ \ufMod(\Tot(A),\Tot(B))^v_u\]
for $f ∈ \ubgMod (A,B)^v_u$ to be given on any $x ∈ \Tot(A)^n$ by
\[(\Tot(f)(x)))_{j+u} :=
\sum_{m≥0}(−1)^{(m+u)n}f_m(x_{j-m}) ∈ B^{n-j+v}_{j+u} ⊂ \Tot(B)^{n+u+v}.\]
We also define an inverse as follows. Let $K = \Tot(A)$, $L = \Tot(B)$ and $g ∈ \ufMod(K,L)^v_u$. We define
\[f := \Tot^{−1}(g) ∈ \ubgMod(A,B)^v_u\]
to be given by $f := (f_0, f_1,\dots)$ where $f_i$ is defined on each $A^{m+j}_j$ by the composite
\begin{align*}
f_i : A^{m-j}_j \hookrightarrow\prod_{k\geq j}A^{m-k}_k &= F_j(\Tot(A)^m)\xrightarrow{g}F_{j+u}(\Tot(B)^{m+u+v})\\
&=\prod_{l\geq j+u}B^{m+u+v-l}_l\xrightarrow{×(−1)^{(i+u)m}} B^{m-j+v−i}_{j+u+i} ,
\end{align*}
where the last map is a projection and multiplication by the indicated sign factor.
\end{defin} 

The following is \cite[Theorem 4.39]{whitehouse}.
\begin{thm}\label{4.39}
Let $A$ and $B$ be twisted complexes. The assignments $\mathfrak{Tot}(A) := \Tot(A)$ and
\begin{align*}
\mathfrak{Tot}_{A,B} : \utC(A,B)& → \ufC(\Tot(A),\Tot(B))\\
f &\mapsto \Tot(f)
\end{align*}
define a $\vbc$-enriched functor $\mathfrak{Tot} : \utC → \ufC$ which restricts to an isomorphism onto its image. Furthermore, this functor restricts to a $\bgmod$-enriched functor \[\mathfrak{Tot} : \ubgMod → \ufMod\] 
which also restricts to an isomorphism onto its image.
\end{thm}

We now present an enriched endomorphism operad. The precise operad structure is shown in \cite[Lemma 4.41]{whitehouse}. 
\begin{defin}
Let $\underline{\mathscr{C}}$ be a monoidal $\mathscr{V}$-enriched category and $A$ an object of $\uC$. We define $\uEnd_A$
to be the collection in $\mathscr{V}$ given by
\[\uEnd_A(n) \coloneqq \uC (A^{⊗n},A) \text{ for }n ≥ 1.\]
\end{defin}

The following contains Proposition 4.40, Lemma 4.43 and Proposition 4.46 from \cite{whitehouse}.%\pagebreak

\begin{propo}\label{S4}\
%The following hold.
\begin{itemize}
\item The enriched functors %\label{4.40}
\[\mathfrak{Tot} : \ubgMod  → \ufMod ,\hspace{1cm} \mathfrak{Tot} : \utC → \ufC\]
are lax symmetric monoidal in the enriched sense and when restricted to the bounded case they are strong symmetric monoidal in the enriched sense.
\item For $A\in\uC$, the collection $\uEnd_A$ defines an operad in $\VV$. %4.43

\item Let $\CC$ and $\DD$ be monoidal categories over $\VV$. Let %\label{morphism} 4.46
$F : \CC → \DD$ be a lax monoidal functor over $\VV$. Then for any $X ∈ \CC$ there is an operad morphism
\[\uEnd_X→\uEnd_{F(X)}.\]

\end{itemize}
\end{propo}



\begin{lem}\label{inverse}
Let $A$ be a twisted complex. Consider the operads $\uEnd_A(n)=\utC(A^{\otimes n},A)$ and $\uEnd_{\Tot(A)}(n)=\ufC(\Tot(A)^{\otimes n},\Tot(A))$. There is a morphism of operads
\[\uEnd_A →\uEnd_{\Tot(A)},\]
which is an isomorphism of operads if $A$ is bounded. If $A$ is just a bigraded module, the same holds for the operads $\uEnd_A(n)=\ubgMod(A^{\otimes n},A)$ and $\uEnd_{\Tot(A)}(n)=\ufMod(\Tot(A)^{\otimes n},\Tot(A))$.
\end{lem}
\begin{proof}
The proof of in the case of a $A$ being a twisted complex can be found in \cite[Lemma 4.54]{whitehouse}. For the bigraded module case, we are going to do it analogously. First, by \Cref{4.39} we know that the functor $\mathfrak{Tot}:\ubgMod\to\ufMod$ is $\bgmod$-enriched. In fact, by \Cref{S4} it is lax monoidal in the enriched sense. In addition, both $\bgmod$ and $\fmod$ are monoidal over $\bgmod$. In the case of $\bgmod$ it is in the obvious way and for $\fmod$ is given by \Cref{4.15}. With all of this we may apply \Cref{S4} to the totalization functor $\mathfrak{Tot}:\ubgMod\to\ufMod$ to obtain the desired map
\[\uEnd_A →\uEnd_{\Tot(A)}.\]
 The fact that it is an isomorphism in the bounded case is analogous to the twisted complex case. 
\end{proof}

We are going to construct the inverse in the bounded case explicitly from \Cref{enrichedmap}. The construction for the direct map is analogue but here we just need the inverse. We do it for a twisted complex $A$, but it is done similarly for a bigraded module.

\begin{lem}\label{composition}
In the conditions of \Cref{inverse} for the bounded case, the inverse is given by the map
\begin{align*}
\uEnd_{\Tot(A)}&\to\uEnd_A\\
f & \mapsto \Tot^{-1}(f\circ \mu^{-1}).
\end{align*}
\end{lem}

\begin{proof}
The inverse is given by the following composite.

\[
\begin{tikzcd}[column sep = 1em]
&\uEnd_{\Tot(A)}(n)=\ufC(\Tot(A)^{\otimes n},\Tot(A))\arrow[d]&\\
& \ufC(\Tot(A^{\otimes n}),\Tot(A))\arrow[r]&\utC(A^{\otimes n},A)=\uEnd_A(n)
\end{tikzcd}
 \]

The second map is given by $\mathfrak{Tot}^{-1}$, see \Cref{enrichedtot}. To describe the first map, let $R$ be concentrated in bidegree $(0,0)$ with trivial vertical differential. Then the first map is given by the following composite
\begin{align*}
\ufC(\Tot(A)^{\otimes n},\Tot(A))\cong R\otimes\ufC(\Tot(A)^{\otimes n},\Tot(A))\\
\xrightarrow{\underline{\mu}^{-1}\otimes 1}\ufC(\Tot(A^{\otimes n}),\Tot(A)^{\otimes n})\otimes\ufC(\Tot(A)^{\otimes n},\Tot(A))\\
\xrightarrow{c}\ufC(\Tot(A^{\otimes n}),\Tot(A)), 
\end{align*}
where $c$ is the composition in $\ufC$, see \Cref{ufMod}. The map $\underline{\mu}^{-1}$ is the adjoint of $\mu^{-1}$ under the bijection from \Cref{adjunction}. Explicitly,
\begin{align*}
\underline{\mu}^{-1}:R &\to \ufC(\Tot(A^{\otimes n}),\Tot(A)^{\otimes n})\\
1 &\mapsto (a\mapsto \mu^{-1}(a)).
\end{align*}
Putting all this together, we get the map 
\begin{align*}
\uEnd_{\Tot(A)}&\to\uEnd_A\\
f & \mapsto \Tot^{-1}(c(f, \mu^{-1})).
\end{align*}
Since the total degree of $\mu^{-1}$ is 0, the composition map reduces to $c(f,\mu^{-1})=f\circ \mu^{-1}$ and we get the desired map.
\end{proof}

%\begin{itemize}



%\item A INFINITY ALGEBRAS CLASSICAL DEFINITION, BAR INTERPRETATION?, TWISTING MORPHISMS (L-V 10.1) REVIEW OF SOME KNOWN RESULTS. A-INFTY OPERAD AN ALGEBRAS AS MORPHISMS FROM THIS OPERAD (I CAN CONNECT IT TO THE COHOMOLOGY OF THE ASSOCIAHEDRA IF I EXPLAIN THAT) STRICT AND INFINITY MORPHIMS ACCORDING TO THE VARIOUS INTERPRETATIONS. DIFFERENCE BETWEEN INCLUDING THE DIFFERENTIAL OR NOT. TOPOLOGICAL ORIGIN? 

%\item FILTERED MODULES, BIGRADED MODULES, TWISTED COMPLEXES, ENRICHMENTS FOR THESE CATEGORIES AS DONE BY SARAH (PROBABLY ONLY THE BASIC CATEGORIES, UNTIL SECOND SECTION SUBSECTION OF THE BACKGROUND)

%\item THIS MAY BE BETTER INTRODUCED ALREADY THE MAIN BODY? DERIVED A INFINITY ALGEBRAS, SOMETHING  FROM SAGAVE, UNIQUENESS OF AINFTY STRUTURES, DERIVED AINFTY IN OPERADIC CONTEXT, AND DERIVED AINFTY AND THEIR HOMOTOPIES. DIFFERENCE BETWEEN UNDERLYING TWISTED COMPLEXES OR NOT, MAIN THEOREM OF SARAH'S PAPER

%\item FIND A BAR INTERPRETATION? (SINCE THIS WOULD BE KIND OF NEW, MAYBE IN A DIFFERENT CHAPTER) THIS IS ACTUALLY ON OPERADIC CONTEXT AND POSSIBLY EVEN IN SAGAVE, SO JUST STUDY IT AND INCLUDE IT JUST LIKE THE CLASSICAL CASE?
%\item REFERENCE TO GEOMETRIC INTERPRETATIONS? (THESIS BY SARAH'S STUDENT)


%\end{itemize}

%\phantomsection
%\bibliographystyle{ieeetr}
%\bibliography{newbibliography}
\end{document}
