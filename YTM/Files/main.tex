%%%%%%%%%%%%%%%%%%%%%%%%%%%%%%%%%%%%%%%%%
% MUW Poster
% LaTeX Template
% Version 1.0 (31/08/2016)
% (Based on Version 1.0 (31/08/2015) of the Jacobs Portrait Poster
%
% License:
% CC BY-NC-SA 3.0 (http://creativecommons.org/licenses/by-nc-sa/3.0/)
%
% Created by:
% Nicolas Ballarini, CeMSIIS, Medical University of Vienna
% nicoballarini@gmail.com
% http://statistics.msi.meduniwien.ac.at/
%%%%%%%%%%%%%%%%%%%%%%%%%%%%%%%%%%%%%%%%%


\def\footer#1{\def\insertfooter{#1}}
%--------------------------------------------------------------------------------------
%	PACKAGES AND OTHER DOCUMENT CONFIGURATIONS
%--------------------------------------------------------------------------------------

\documentclass[final]{beamer}

\usepackage{pgf,tikz}
\usepackage{pgfplots}
\usepackage{tikz-cd}
%\usepackage{tcolorbox}
\usepackage{oplotsymbl} 


\usepackage[scale=1.150]{beamerposter} % Use the beamerposter package
\usetheme{MUWposter} % Use the MUWposter theme supplied with this template

% Include a logo of your project if desired
\logo{\pgfputat{\pgfxy(-15,107)}{\pgfbox[center,base]{\includegraphics[width=15cm]{YTMLogo.png}}}}  

\usepackage{multicol}
\usepackage{array}
%The following two are column definitions for the aknowledgements section
\newcolumntype{L}{>{\arraybackslash}m{22cm}}
\newcolumntype{S}{>{\arraybackslash}m{5cm}}
\usepackage{pgf}  
\usepackage{mathtools}
\usepackage{amsmath, amsthm, amssymb, amsfonts}
\usepackage{exscale}
\usepackage{xcolor}
\usepackage{ushort}
\usepackage{setspace}
\usepackage[square,numbers]{natbib}
\usepackage{url}
\bibliographystyle{abbrvnat}
\renewcommand{\vec}[1]{\ushort{#1}}
\renewcommand{\vec}[1]{\mathbf{#1}}
\definecolor{greenMUW}{RGB}{60,191,174}
\definecolor{blueMUW}{RGB}{17,29,79}
\definecolor{skinMUW}{RGB}{254,228,217}
\definecolor{hellblauMUW}{RGB}{95,180,229}

%-----------------------------------------------
%  START Set the colors
%  Uncomment to apply colors you want to use.
%-----------------------------------------------
%\colorlet{themecolor}{greenMUW}
%\usebackgroundtemplate{\includegraphics{MUW_green.pdf}}
%\colorlet{themecolor}{skinMUW}
%\colorlet{themecolor}{blueMUW}
%\usebackgroundtemplate{\includegraphics{MUW_skin.pdf}}

\colorlet{themecolor}{blueMUW}
\colorlet{themecolor}{hellblauMUW}
\usebackgroundtemplate{\includegraphics{ground.pdf}}
%-----------------------------------------------
%  END Set the colors
%-----------------------------------------------


%-----------------------------------------------
%  START Set the width of the columns
%-----------------------------------------------
\setlength{\paperwidth}{33.1in} % A0 width: 46.8in
\setlength{\paperheight}{46.8in} % A0 height: 33.1in
\newlength{\sepmargin}
\newlength{\sepwid}
\newlength{\onecolwid}
\newlength{\twocolwid}
\newlength{\threecolwid}

% The following measures are used for 2 columns
\setlength{\sepmargin}{0.055\paperwidth} % Separation width (white space) between columns
\setlength{\sepwid}{0.03\paperwidth} % Separation width (white space) between columns
\setlength{\onecolwid}{0.43\paperwidth} % Width of one column
\setlength{\twocolwid}{0.9\paperwidth} % Width of two columns

%-----------------------------------------------------------
% The following measures are used for 3 columns
%\setlength{\sepmargin}{0.06\paperwidth} % Separation width (white space) between columns
%\setlength{\sepwid}{0.02\paperwidth} % Separation width (white space) between columns
%\setlength{\onecolwid}{0.28\paperwidth} % Width of one column
%\setlength{\twocolwid}{0.58\paperwidth} % Width of two columns
%\setlength{\threecolwid}{0.88\paperwidth} % Width of three columns
%\setlength{\columnsep}{30pt}

%-----------------------------------------------
%  END Set the width of the columns
%-----------------------------------------------


%--------------------------------------------------------------------------------------
%	TITLE SECTION 
%--------------------------------------------------------------------------------------
\setbeamertemplate{title}[left]
\setbeamertemplate{frametitle}[default][left]
%\setmainfont{Georgia}

\title{$A_\infty$-structures on operads} % Poster title

\author{Javier Aguilar Martín} % Author(s)

\institute{University of Kent} % Institution(s)
%--------------------------------------------------------------------------------------



\begin{document}

  \addtobeamertemplate{block end}{}{\vspace*{1ex}} % White space under blocks
  \addtobeamertemplate{block alerted end}{}{\vspace*{0ex}} % White space under highlighted (alert) blocks
  \setlength{\belowcaptionskip}{2ex} % White space under figures
  \setlength\belowdisplayshortskip{1ex} % White space under equations
  
  
  \begin{frame}[t] % The whole poster is enclosed in one beamer frame

      \begin{columns}[t] % The whole poster consists of two major columns
	  
      \begin{column}{\sepmargin}\end{column}
      
	    \begin{column}{\onecolwid} % The first column


		  \begin{block}{Objective}
          %\begin{multicols}{2}
          Obtaining an elementary and simple description of $A_\infty$-algebras from an operadic point of view and using it to induce $A_\infty$-structures on operads.
          %\end{multicols}
          \end{block}
          
          \begin{block}{Operads}
          %\begin{multicols}{2}
         An \textbf{operad} is a collection of spaces  $\mathcal{O}=\{\mathcal{O}(n)\}_{n\geq 0}$, whose points are thought to be $n$-ary operations $X^n\to X$, usually represented as trees,
			\[\begin{tikzpicture}[line cap=round,line join=round,>=triangle 45,x=1.0cm,y=1.0cm]
			\clip(-2.13333333333334,-0.093333333333332) rectangle (12.006666666666668,3.5);
			\draw [line width=2.pt] (2.,0.)-- (2.,1.);
			\draw [line width=2.pt] (2.,1.)-- (0.3666666666666659,3.);
			\draw [line width=2.pt] (2.,1.)-- (1.,3.);
			\draw [line width=2.pt] (2.,1.)-- (1.7,3.);
			\draw [line width=2.pt] (2.,1.)-- (3.,3.);
			\end{tikzpicture}\]
			
		
          %\end{multicols}
          
          together with \textbf{insertion maps} $\circ_i: \mathcal{O}(n)\otimes\mathcal{O}(m)\to\mathcal{O}(n+m-1)$ satisfying natural associativity axioms
          
				\[\begin{tikzpicture}[line cap=round,line join=round,>=triangle 45,x=1.0cm,y=1.0cm]
\clip(-10.04,-0.86) rectangle (4.14,5.18);
\draw (2.,0.)-- (2.,1.);
\draw (2.,1.)-- (0.,2.);
\draw (2.,1.)-- (1.,2.);
\draw (2.,1.)-- (2.,2.);
\draw (2.,1.)-- (3.,2.);
\draw (2.,1.)-- (4.,2.);
\draw (2.,2.)-- (2.,3.);
\draw (2.,3.)-- (1.,4.);
\draw (2.,3.)-- (2.,4.);
\draw (2.,3.)-- (3.,4.);
\draw (-2,2.18) node[anchor=north west] {$=$};
\draw (-3.,1.)-- (-3.,2.);
\draw (-3.,2.)-- (-2.,3.);
\draw (-3.,2.)-- (-3.,3.);
\draw (-3.,2.)-- (-4.,3.);
\draw (-5.64,2.14) node[anchor=north west] {$\circ_3$};
\draw (-8.,1.)-- (-8.,2.);
\draw (-8.,2.)-- (-6.,3.);
\draw (-8.,2.)-- (-10.,3.);
\draw (-8.,2.)-- (-9.,3.);
\draw (-8.,2.)-- (-8.,3.);
\draw (-8.,2.)-- (-7.,3.);
\end{tikzpicture}\]

and a unit element for insertions $1\in\mathcal{O}(1)$.
% \begin{block}{\vspace*{1cm}}
          %\begin{multicols}{2}
    
    
    
    % A \textbf{map} of operads $f:\mathcal{O}\to\mathcal{P}$ is an arity-wise defined map $f_n:\mathcal{O}(n)\to\mathcal{P}(n)$ such that $f(1_\mathcal{O})=1_{\mathcal{P}}$ and $f(a\circ_i b)=f(a)\circ_i f(b)$.
          %\end{multicols}
       %   \end{block}
          
         \begin{alertblock}{}
   \textbf{Endomorphism operad.} For a space $V$, the operad $End_V$ is given by $End_V(n)=\hom(V^{\otimes n},V)$ and $f\circ_i g=f(1^{\otimes i-1}\otimes g\otimes 1^{\otimes n-i})$. When $V$ is a chain complex, $End_V(n)$ has a differential $\partial(f)=d_V\circ f-(-1)^{|f|}f\circ d_{V^{\otimes n}}$.
    \end{alertblock} 
    
    \begin{itemize}
        \item The \textbf{circle operation} $a\circ b=\sum_i a\circ_i b$ is a pre-Lie product.
    \end{itemize}

          \end{block}
          
          
         
          
          
          \begin{block}{Stasheff Associahedra}
          %\begin{multicols}{2}
          Let $M_2:X^2\to X$ be a binary multiplication. It is \textbf{homotopy-associative} when there is a homotopy $M_3:[0,1]\times X^3\to X$ such that $M_3|_{0\times X^3}=M_2(M_2\times 1)$ and $M_3|_{1\times X^3}=M_2(1\times M_2)$. The possible products of 4 elements give us homotopies as follows.
          
         \[ \begin{tikzpicture}[line cap=round,line join=round,>=triangle 45,x=1.0cm,y=1.0cm]
\clip(-4,-2) rectangle (12,8);
\draw(2.,0.) -- (6.,0.) -- (7.23606797749979,3.804226065180613) -- (4.,6.155367074350506) -- (0.7639320225002106,3.8042260651806146) -- cycle;
\draw (2.,0.)-- (6.,0.);
\draw (6.,0.)-- (7.23606797749979,3.804226065180613);
\draw (7.23606797749979,3.804226065180613)-- (4.,6.155367074350506);
\draw (4.,6.155367074350506)-- (0.7639320225002106,3.8042260651806146);
\draw (0.7639320225002106,3.8042260651806146)-- (2.,0.);
\draw (1.5,8) node[anchor=north west] {$x(y(zt))$};
\draw (-4,4.8) node[anchor=north west] {$(xy)(zt)$};
\draw (-2,0) node[anchor=north west] {$((xy)z)t$};
\draw (5.6,0) node[anchor=north west] {$(x(yz))t$};
\draw (7.3,4.8) node[anchor=north west] {$x((yz)t)$};
\draw (5.7,5.9) node[anchor=north west] {$\simeq$};
\draw (0.5,5.9) node[anchor=north west] {$\simeq$};
\draw (-0.3,2.2) node[anchor=north west] {$\simeq$};
\draw (3.2,-0.1) node[anchor=north west] {$\simeq$};
\draw (6.5,2.2) node[anchor=north west] {$\simeq$};
\begin{scriptsize}
\draw [fill=black] (2.,0.) circle (2.5pt);
\draw [fill=black] (6.,0.) circle (2.5pt);
\draw [fill=black] (7.23606797749979,3.804226065180613) circle (2.5pt);
\draw [fill=black] (4.,6.155367074350506) circle (2.5pt);
\draw [fill=black] (0.7639320225002106,3.8042260651806146) circle (2.5pt);
\end{scriptsize}
\end{tikzpicture}\]
      
      %If there is a map $M_4=\pentagofill\times X^4\to X$ extending the homotopies we say that $M_2$ is homotopy coherent. 
      
      The products of $n$ elements produce an $(n-2)$-dimensional polytope $K_n$ called \textbf{Stasheff associahedron}.  Higher homotopy coherence is achieved with maps $M_n:K_n\times X^n\to X$ satisfying boundary conditions.
      
      \begin{itemize}
          \item Since $K_n\times K_m$ is a cell of $K_{n+m-1}$, the collection $\{K_n\}$ is an operad and so is the collection of cellular chains $\{C_*(K_n)\}$.
          \item A space for which $M_n$ exists for all $n$ is called an $A_\infty$-\textbf{space}.
          \item The maps $M_n$ induce maps $m_n:C_*(K_n)\otimes C_*(X)^{\otimes n}\to C_*(X)$ satisfying some relations described below.
      \end{itemize}
          %\end{multicols}An \textbf{operad} is a collection of spaces  $\mathcal{O}=\{\mathcal{O}(n)\}_{n\geq 0}$, whose points are thought to be $n$-ary operations $X^n\to X$.
          \end{block}
          
          
          \begin{block}{$A_\infty$-algebras}
          %\begin{multicols}{2}
          An $A_\infty$-\textbf{algebra} $A$ is chain complex with maps $m_n:A^{\otimes n}\to A$ of degree $n-2$ satisfying for all $n\geq 2$ the relation

\[\sum_{r+s+t=n}(-1)^{rs+t}m_{r+1+t}\circ_{r+1} m_s=-\partial(m_n).\]

\begin{itemize}
    \item When $m_n=0$ for all $n\neq 2$, $A$ is an associative algebra.
    \item In general, $m_2$ is only associative up to homotopy given by $m_3$.
\end{itemize}


          
          

          %\end{multicols}
          \end{block}
          
            
          
         
          
         \end{column}
                  
                  
                  
         \begin{column}{\sepwid}  \end{column}
         
       
         
         
         
         \begin{column}{\onecolwid} %The second column
         \begin{block}{\vspace*{1.95cm}}
         \begin{itemize}
             \item A \textbf{morphism}  of $A_\infty$-algebras is a chain map $f:A\to B$ such that $f(m_n)=m_n(f^{\otimes n})$ for all $n$.
         \end{itemize}
           An operad of chain complexes $\mathcal{O}$ has an $A_\infty$-multiplication if there are elements $m_n\in\mathcal{O}(n)$ of degree $n-2$ satisfying the equation of $A_\infty$-algebras. 
%\[\sum_{r+s+t\geq 2}(-1)^{rs+t}m_{r+1+t}\circ_{r+1} m_s=.\]
%I NEED THE DIFFERENTIAL ON THE RHS BECAUSE I AM OMITTING M1 (SEE MARKL ON DESKTOP OR L-V), CALL THIS DIFFERENTIAL DELTA TO USE IT ONN THE MAURER-CARTAN EQUATION (ALSO SEE L-V FOR DG-OPERADS, IT'S PROBABLY GOING TO BE AN ABUSE OF NOTATION)
\begin{alertblock}{}
  The operad $\{C_*(K_n)\}$ is an operad with  $A_\infty$-multiplication.
  \vspace{0.5cm}
\end{alertblock}


         \end{block}
         
         
        
            \begin{block}{Operadic supension}
          %\begin{multicols}{2}
         \begin{itemize}
             \item Analogue to suspension of graded vector spaces $\Sigma V=V\otimes \mathbb{F}[-1]$.
             \item For an operad $\mathcal{O}$, its \textbf{operadic suspension} is the operad given by $\mathfrak{s}\mathcal{O}(n)=\mathcal{O}(n)\otimes\mathbb{F}[1-n]$ and insertion maps determined by 
              %\begin{alertblock}{}
           %\[\mathfrak{s}End_V=End_{\Sigma V}\]
            % \end{alertblock}
             \begin{alertblock}{}
               \hspace{13.5cm}$\mathfrak{s}End_{\Sigma V}=End_V$.
               \vspace{0.4cm}
             \end{alertblock}
            %\begin{tcolorbox}[colframe=white]
            % \[\mathfrak{s}End_V=End_{\Sigma V}\]
             %\end{tcolorbox}
         \end{itemize}
         \vspace*{-0.5cm}
          \begin{alertblock}{}
          \textbf{Proposition.} An $A_\infty$-multiplication on $\mathcal{O}$ is equivalent to an element $m\in\mathfrak{s}\mathcal{O}$ of degree $-1$ such that $\partial(m)+m\circ m=0.$ %WITH NO ARITY 0, AND THE DIFFERENTIAL IS ALSO MISSING
         \end{alertblock}
          %\end{multicols}
          \end{block}
          
          \begin{block}{Brace algebras}
          %\begin{multicols}{2}
          
A \textbf{brace algebra} is a space $V$ with brace maps $b_n:V^{\otimes (1+n)}\to V$ satisfying certain associativity axioms.

\begin{itemize}
    \item Any operad defines a brace algebra by setting \[b_n(x;x_1,\dots, x_n)=\sum_{i_1<\cdots<i_n} (\cdots (x\circ_{i_1}x_1)\cdots )\circ_{i_n}x_n.\]
\end{itemize}

 %\end{multicols}
          \end{block}
          
          \begin{block}{$A_\infty$-algebra structure on an operad}
          %\begin{multicols}{2}
          Given an operad $\mathcal{O}$ with multiplication $m\in\mathfrak{s}\mathcal{O}$, define $M_n$ to be the map that makes the following diagram commutative.
          \[
         \begin{tikzcd}[ampersand replacement=\&]
(\Sigma \mathfrak{s}\mathcal{O})^{\otimes n} \arrow[r, "M_n"]  \& \Sigma \mathfrak{s}\mathcal{O} \\
\mathfrak{s}\mathcal{O}^{\otimes n} \arrow[r, "b_n(m;-)"]  \arrow[u]         \& \mathfrak{s}\mathcal{O} \arrow[u]         
\end{tikzcd}
          \]
          
          \begin{alertblock}{}
            \textbf{Proposition.} The maps $M_n$ define an $A_\infty$-algebra struture on $\Sigma\mathfrak{s}\mathcal{O}$. As a consequence, they define an $A_\infty$-multiplication on the endomorphism operad of $\Sigma\mathfrak{s}\mathcal{O}$.
          \end{alertblock}
          %\end{multicols}
          This process can be iterated to define an $A_\infty$-algebra on $\Sigma\mathfrak{s}End_{\Sigma\mathfrak{s}\mathcal{O}}$ and we have  the following
          
           \begin{alertblock}{}
            \textbf{Theorem.} There is a morphism of $A_\infty$-algebras \[\Phi:\Sigma\mathfrak{s}\mathcal{O}\to \Sigma\mathfrak{s}End_{\Sigma\mathfrak{s}\mathcal{O}}\]
            induced by the map $\mathfrak{s}\mathcal{O}\to End_{\mathfrak{
            s}\mathcal{O}}$ given by 
            \[x\mapsto \sum_{n\geq 0}b_n(x;-).\]
          \end{alertblock}
          \end{block}
         %\begin{block}{\vspace*{2.7cm}}
          %\begin{multicols}{2}
      
          %\end{multicols}
          %\end{block}
          
          %\begin{block}{ }
		%		\begin{figure}
         %       	\vspace*{-1cm}
                    %\includegraphics[width=.9\linewidth]{plot.PNG}
				%\end{figure}
				%\begin{figure}
                    %\includegraphics[width=.9\linewidth]{Bild1.jpg}
				%\end{figure}
                
                %\begin{multicols}{2}
            %    \begin{figure}
             %   	\vspace*{-0.95cm}
                    %\includegraphics[width=.8\linewidth]{Bild2.jpg}
				%\end{figure}
            %    \begin{figure}
             %   	\vspace*{-0.95cm}
                    %\includegraphics[width=.8\linewidth]{Bild3.jpg}
				%\end{figure}
                %\end{multicols}
                
                %\begin{figure}
                    %\includegraphics[width=.9\linewidth]{Bild4.jpg}
				%\end{figure}
                
		%\end{block}
		\begin{block}{\large References}
			  \vspace*{-0.5cm}
              	\nocite{*} % Insert publications even if they are not cited in the poster
					{\footnotesize
                    	%\bibliographystyle{plainurl}
						\bibliography{bibliog.bib}}
				\end{block} 
      \end{column}
      
      \begin{column}{\sepmargin} \end{column}
    \end{columns} 
       
   %   \begin{columns}[t] % Split up the two columns wide column again
      
  %   \begin{column}{\sepmargin} \end{column}
 %       \begin{column}{\onecolwid} % The first column
			%\begin{block}{\large Acknowledgements}
                    %\begin{center}
						%\begin{tabular}{SL}
							%\includegraphics[width=\linewidth]{Flag_of_Europe.png}  &
			%				\footnotesize This project has received funding from the  grant agreement No 11111.
			%			\end{tabular}
			%		\end{center}
			%	\end{block}	
             %   \vspace*{-0.9cm}
			%	\begin{alertblock}{\large Contact Information}
            %\vspace*{-0.5cm}
					%\begin{footnotesize}
					%\begin{itemize}
						%\item \href{mailto:email@meduniwien.ac.at}{email@meduniwien.ac.at}
						%\item %\href{http://www.example.com/}{www.example.com} - %\href{www.meduniwien.ac.at/medstat}{www.meduniwien.ac.at/medstat}
				%\end{itemize}
				%	\end{footnotesize}	
					
			%	\end{alertblock}
%		    \end{column} % End of the first column
%			\begin{column}{\sepwid}\end{column} % Empty spacer column
%			\begin{column}{\onecolwid} % Begin a column 
             
%			\end{column} % End of the second column
            
%			\begin{column}{\sepmargin}\end{column} % Empty spacer column
            
%\end{columns} % End of all the columns in the poster


\end{frame} % End of the enclosing frame
	
\end{document}
