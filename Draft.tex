	\documentclass[twoside]{article}
\usepackage{estilo-ejercicios}
\setcounter{section}{0}

\DeclareMathAlphabet{\mathpzc}{OT1}{pzc}{m}{it}
\renewcommand{\baselinestretch}{1,3}

\usepackage{empheq}
\newcommand*\widefbox[1]{\fbox{\hspace{2em}#1\hspace{2em}}}


\crefname{defin}{Definition}{definition}
%--------------------------------------------------------
\begin{document}

\title{Table of enriched categories}
\author{Javier Aguilar Martín}
\maketitle

\section*{Purpose}
The idea of this was writing a draft containing all the necessary definitions and lemmas so that the main theorem can be understood. It should include the definition of the $dA_\infty$ and $A_\infty$ operad and the basic definitions and conventions of bigraded modules as well. It's a lot of work and was just to send it to James Shank for him to give some feedback, so maybe I'll do it later. I think many of the things that I would include here should go in a background chapter anyway, but for a paper I should write things down like I intend here.
\section{Definitions}

Create a custom ennviroment to use the numbering of Sarah's paper for better identification \url{https://tex.stackexchange.com/questions/53978/custom-theorem-numbering}

Definition 2.2
\begin{defin}[Filtered $R$-module and morphisms]
A \emph{filtered $R$-module} $(A, F)$ is given by a family of $R$-modules $\{F_pA\}_{p∈\Z}$ indexed by the integers such that $F_{p−1}A ⊆ F_pA$ for all $p ∈ \Z$ and $A = ∪F_pA$. A morphism of filtered modules is a
morphism $f : A → B$ of $R$-modules which is compatible with filtrations: \[f(F_pA) ⊂ F_pB\text{ for all }p ∈ \Z.\]
\end{defin}

Definitions 2.4 and 2.5
\begin{defin}[Filtered complexes and morphism]\label{filteredcomplex}
A \emph{filtered complex} $(K, d, F)$ is a complex $(K, d) ∈ \mathrm{C}_R$ together with a filtration $F$ of each $R$-module $K^n$ such that $d(F_pK^n) ⊂ F_pK^{n+1}$ for all $p, n ∈ \Z$. Its morphisms are given by
morphisms of complexes $f : K → L$ compatible with filtrations: \[f(F_pK) ⊂ F_pL\text{ for all }p ∈ \Z.\]
\end{defin}

Definition 2.3
\begin{defin}\label{filteredtensor}
The tensor product of two filtered $R$-modules $(A, F)$ and $(B, F)$ is a filtered $R$-module,
with
 \[F_p(A ⊗ B) :=\sum_{i+j=p}\Ima(F_iA ⊗ F_jB → A ⊗ B).\]
This makes the category of filtered $R$-modules into a symmetric monoidal category, where the unit is given by $R$ with the trivial filtration $0 = F_{−1}R ⊂ F_0R = R$.
\end{defin}

\begin{remark}
Taking the image of the map is necessary because tensor product is not left-exact, in particular, the product of monomorphisms is not necessarily a  monomorphism, so we need to consider the image to have inclusions of filtration levels.
\end{remark}

Definition 2.7
\begin{defin}[Vertical bicomplexes and morphisms]\label{vbC}
A \emph{vertical bicomplex} is a bigraded $R$-module $A$ equipped with a vertical differential $d_A : A → A$ of bidegree $(0, 1)$. A morphism of vertical bicomplexes is a morphism of bigraded modules
of bidegree $(0, 0)$ commuting with the vertical differential.

The tensor product of two vertical bicomplexes $A$ and $B$ is given by endowing the tensor product of underlying bigraded modules with
vertical differential \[d^{A⊗B} := d^A ⊗ 1 + 1 ⊗ d^B : (A ⊗ B)^v_u → (A ⊗ B)^{v+1}_u.\]
\end{defin}


Definition 2.8
\begin{defin}[Enrichment over complexes]\label{delta1}
Let $A,B$ be bigraded modules. We define $[A,B]^∗_∗$
to be the bigraded module of morphisms of bigraded modules $A → B$. Furthermore, if $A,B$ are vertical bicomplexes, and $f ∈
[A,B]^v_u$, we define
\[δ(f) := d_Bf − (−1)^vfd_A.\]
\end{defin}

Definition 3.1
\begin{defin}[Twisted complexes]\label{twistedcomplex} A \emph{twisted complex} $(A, d_m)$ is a bigraded $R$-module $A = \{A^j_i \}$ together with a family
of morphisms $\{d_m : A → A\}_{m≥0}$ of bidegree $(−m,−m + 1)$ such that for all $m ≥ 0$,
\[\sum_{i+j=m}(−1)^id_id_j = 0.\]
\end{defin}

Definition 3.2
\begin{defin}[Morphisms of twisted complexes]\label{twistedmorphisms}
A morphism of twisted complexes $f : (A, d^A_m) → (B, d^B_m)$ is given by a family of morphisms of $R$-modules $\{f_m : A → B\}_{m≥0}$ of bidegree $(−m,−m)$ such that for all $m ≥ 0$,
\[\sum_{i+j=m}d^B_if_j =\sum_{i+j=m}(−1)^if_id^A_j.\]
The composition of morphisms is given by $(g \circ f)_m :=\sum_{i+j=m} g_if_j$.
\end{defin}


Lemma 3.3
\begin{defin}\label{tensortwisted}
The category $(\tc,⊗,R)$ is symmetric monoidal, where the monoidal structure is given
by the bifunctor
\[⊗ : \tc × \tc → \tc\]
which on objects is given by $((A, d^A_m), (B, d^B_m)) → (A ⊗ B, d^A_m ⊗ 1 + 1 ⊗ d^B_m)$ and on morphisms is
given by $(f, g) → f ⊗ g$, where $(f ⊗ g)_m :=\sum_{i+j=m} f_i ⊗ g_j$.
\end{defin}

Lemma 3.4
\begin{defin}[Enrichment over twisted complexes]\label{di} Let $A,B$ be twisted complexes. For $f ∈ [A,B]^v_u$, setting
\[(d_if) := (−1)^{i(u+v)}d^B_if − (−1)^vfd^A_i,\]
for $i ≥ 0$, endows $[A,B]^∗_∗$ with the structure of a twisted complex.
\end{defin}

%I will probably use direct sum totalization, whih has a specific notation and name in the paper, but I think the results for this totalization applies to the direct sum version (totalization with compact support).

Definition 3.6
\begin{defin}[Totalization and column filtration]
The total graded $R$-module $\mathrm{Tot}(A)$ of a bigraded $R$-module $A =\{A^j_i \}$ is given by
\[\mathrm{Tot}(A)^n :=\prod_{i≤0}A^{n+i}_i ⊕\bigoplus_{i>0}A^{n+i}_i .\]
The \emph{column filtration} of $\mathrm{Tot}(A)$ is the filtration given by \[F_p\mathrm{Tot}(A)^n :=\prod_{i≤p} A^{n+i}_i\text{ for all }p, n ∈ \Z.\]
\end{defin}

I should include the differential and Tot(f) here

This one comes much later but I prefer it to be here. It is possible to omit the filtration (but would have to check what consequences it has).

Definition 4.13
\begin{defin}
The \emph{totalization with compact support} of a vertical bicomplex $A$ is the filtered complex given by
\[\mathrm{Tot}_c(A)^n :=\bigoplus_{i∈\Z}A^{n+i}_i\]
with the column filtration and with differential as for the totalization functor.

Given a morphism of vertical bicomplexes $f : A → B$ we get a morphism of filtered complexes $\mathrm{Tot}_c(f) : \mathrm{Tot}_c(A) → \mathrm{Tot}_c(B)$
constructed analogously to $\mathrm{Tot}(f)$.
\end{defin}
MAYBE I CAN GIVE sO A VERTICAL BICOMPLEX STRUCTURE USING THE 0 HORIZONTAL DEGREE PART OF THE DERIVED AINFTY STRUCTURE (ESSENTIALLY THE SAME AS IN LWR) TO APPLY COMPACT SUPPORT TOTALIZATION AND FORGET ABOUT BOUNDEDNESS, BUT THIS SEEMS LIKE CHEATING AND PROBABLY BOUNDEDNES WILL POP UP LATER. WHAT'S MORE,  SINCE O IS ESSENTIALLY THE OPERAD OF DAINFTY ALGEBRAS, UNDER THIS CONVEENTION I WOULD NOT BE ABLE TO USE ANY OF THE STRUCTURE MAPS BECAUSE ALL OF THEM ARE ELEMENTS OF NEGATIVE HORIZONTAL DEGREE


Definition 3.7
\begin{defin}[Split (filtered) complexes]\label{splitcomplex}
A filtered complex $(K, d, F)$ is said to be \emph{split} if $K = \mathrm{Tot}(A)$ is the total graded module of a bigraded $R$-module $A = \{A^j_i \}$ and $F$ is the column filtration of $\mathrm{Tot}(A)$.
\end{defin}

Definition 3.32
\begin{defin}[Weird enrichment]\label{weirdenrichment}
Let $A,B,C$ be bigraded modules. We denote by $\underline{\mathpzc{bgMod}_R}(A,B)$ the bigraded module given by
\[\underline{\mathpzc{bgMod}_R}(A,B)^v_u :=\prod_{j≥0}[A,B]^{v−j}_{u−j}\]
where $[A,B]$ is the inner hom-object of bigraded modules. More precisely, $g ∈ \underline{\mathpzc{bgMod}_R}(A,B)^v_u$ is given
by $g := (g_0, g_1, g_2, \dots )$, where $g_j : A → B$ is a map of bigraded modules of bidegree $(u − j, v − j)$.

Moreover, we define a composition morphism
\[c : \underline{\mathpzc{bgMod}_R}(B,C) ⊗ \underline{\mathpzc{bgMod}_R}(A,B) → \underline{\mathpzc{bgMod}_R}(A,C)\]
by
\[c(f, g)_m :=\sum_{i+j=m}(−1)^{i|g|}f_ig_j .\]
\end{defin}
Try to find a conceptual explanation of the sign in the above composition map.

Definition 4.17
\begin{defin}\label{delta2}
Let $(A, d^A_i), (B, d^B_i)$ be twisted complexes, $f ∈ \underline{\mathpzc{bgMod}_R}(A,B)^v_u$ and consider $d^A :=(d^A_i)_i ∈ \underline{\mathpzc{bgMod}_R}(A,A)^1_0$
and $d^B := (d^B_i)_i ∈ \underline{\mathpzc{bgMod}_R}(B,B)^1_0$. We define
\[δ(f) := c(d^B, f) − (−1)^{\langle f,d^A\rangle}c(f, d^A) ∈ \underline{\mathpzc{bgMod}_R}(A,B)^{v+1}_u\]
where $\langle f, d^A\rangle$ is the scalar product for the bidegrees and $c$ is the composition morphism described in Definition 3.32. More precisely,
\[(δ(f))_m :=\sum_{i+j=m}(−1)^{i|f|}d^B_if_j − (−1)^{v+i}f_id^A_j.\]
\end{defin}
Lemma 4.18? (in results obvsly)

Definition 4.19
\begin{defin}[Twisted complex weird enrichment]
For $A,B$ twisted complexes, we define $\underline{t\mathcal{C}_R}(A,B)$ to be the vertical bicomplex
$\underline{t\mathcal{C}_R}(A,B) := (\underline{\mathpzc{bgMod}_R}(A,B), δ)$.
\end{defin}

Definition 4.23
\begin{defin}\label{utC}
The $\vbc$-enriched category of twisted complexes $\utC$ is the enriched category given by the following data.
\begin{enumerate}[(1)]
\item The objects of $\utC$ are twisted complexes.
\item For $A,B$ twisted complexes the hom-object is the vertical bicomplex $\utC(A,B)$.
\item The composition morphism $c : \utC(B,C)⊗\utC(A,B) → \utC(A,C)$ is given by Definition 3.32.
\item The unit morphism $R → \utC(A,A)$ is given by the morphism of vertical bicomplexes sending
$1 ∈ R$ to $1_A : A → A$, the strict morphism of twisted complexes given by the identity of $A$.
\end{enumerate}
\end{defin}

Definition 4.24
\begin{defin}\label{ubgMod}
We denote by $\ubgMod$ the $\bgmod$-enriched category of bigraded modules given
by the following data.

\begin{enumerate}[(1)]
\item The objects of $\ubgMod$ are bigraded modules.
\item For $A,B$ bigraded modules the hom-object is the bigraded module $\ubgMod(A,B)$.
\item The composition morphism $c : \ubgMod(B,C) ⊗ \ubgMod(A,B) → \ubgMod(A,C)$ is given by
Definition 3.32.
\item The unit morphism $R → \ubgMod(A,A)$ is given by the morphism of bigraded modules that
sends $1 ∈ R$ to $1_A : A → A$, the strict morphism given by the identity of $A$.
\end{enumerate}
\end{defin}

This next tensor corresponds to $\underline{\otimes}$ in the categorical set up.

Lemma 4.27
\begin{defin}\label{tensorenriched}
The monoidal structure of $\utC$ is given by the following map of vertical bicomplexes.
\[\widehat{⊗}: \utC(A,B) ⊗ \utC(A′,B′) → \utC(A ⊗ A′,B ⊗ B′)\]
\[(f, g) → (f\widehat{⊗}g)_m :=\sum_{i+j=m}(−1)^{ij}f_i ⊗ g_j\]
The monoidal structure of $\ubgMod$ is given by the restriction of this map.
\end{defin}


 Find a concetual explanation for that tensor product or at least see why it is necessary to satisfy the axioms of monoidal category. Also check whether the restriction to bgMod gives the signs of the composition map on that bigraded module, because that is what Sarah said

Definition 4.28

\begin{defin}\label{ufMod}
The $\bgmod$-enriched category of filtered modules $\ufMod$ is the enriched category
given by the following data.
\begin{enumerate}[(1)]
\item The objects of $\ufMod$ are filtered modules.
\item For filtered modules $(K, F)$ and $(L, F)$, the bigraded module $\ufMod(K,L)$ is given by
\[\ufMod(K,L)^v_u :=\{f : K → L\mid f(F_qK^m) ⊂ F_{q+u}L^{m+v−u}, ∀m, q ∈ \Z\}.\]
\item The composition morphism is given by $c(f, g) = (−1)^{u|g|}fg$, where $f$ has bidegree $(u, v)$.
\item The unit morphism is given by the map $R → \ufMod(K,K)$ given by $1 → 1_K$.
\end{enumerate}
\end{defin}

Definition 4.31
\begin{defin}\label{fmoddifferential}
Let $(K, d^K, F)$ and $(L, d^L, F)$ be filtered complexes. We define $\ufC(K,L)$ to be the
vertical bicomplex whose underlying bigraded module is $\ufMod(K,L)$ with vertical differential
\[δ(f) := c(d^L, f) − (−1)^{\langle f,d^K\rangle}c(f, d^K) = d^Lf − (−1)^{v+u}fd^K = d^Lf − (−1)^{|f|}fd^K\]
for $f ∈ \ufMod(K,L)^v_u$. Composition from $\ufMod$.
\end{defin}

Definition 4.33
\begin{defin}\label{ufC}
The $\vbc$-enriched category of filtered complexes $\ufC$ is the enriched category given
by the following data.
\begin{enumerate}[(1)]
\item The objects of $\ufC$ are filtered complexes.
\item For $K,L$ filtered complexes the hom-object is the vertical bicomplex $\ufC(K,L)$.
\item The composition morphism is given as in $\ufMod$ in Definition 4.28 
\item The unit morphism is given by the map $R → \ufC(K,K)$ given by $1 → 1_K$.
We denote by $\usfC$ the full subcategory of $\ufC$ whose objects are split filtered complexes.

\end{enumerate}
\end{defin}

Lemma 4.36
\begin{defin}\label{tensorenriched2}
The monoidal structure of fMod R is given by the following map of vertical bicomplexes.
\[\widehat{⊗}: \ufC(K,L) ⊗ \ufC(K′,L′) → \ufC(K ⊗ K′,L ⊗ L′),\]
\[(f, g) → f\widehat{⊗}g := (−1)^{u|g|}f ⊗ g\]
where $f$ has bidegree $(u, v)$.
\end{defin}
 Again see how the signs do their job

Definition 4.38
\begin{defin}\label{enrichedtot}
Let $A,B$ be bigraded modules and $f ∈ \ubgMod (A,B)^v_u$ we define
\[\Tot(f) ∈ \ufMod(\Tot(A),\Tot(B))^v_u\]
to be given on any $a ∈ \Tot(A)^n$ by
\[(\Tot(f)(a)))_{j+u} :=
\sum_{m≥0}(−1)^{(m+u)n}f_m(a_{j+m}) ∈ B^{j+n+v}_{j+u} ⊂ \Tot(B)^{n+v−u}.\]
Let $K = \Tot(A)$, $L = \Tot(B)$ and $g ∈ \ufMod(K,L)^v_u$ we define
\[f := \Tot^{−1}(g) ∈ \ubgMod(A,B)^v_u\]
to be $f := (f_0, f_1,\dots)$ where $f_i$ is given on each $A^{m+j}_j$ by the composite
\begin{align*}
f_i : A^{m+j}_j \hookrightarrow\prod_{k≤j}A^{m+k}_k = F_j(\Tot(A)^m)\xrightarrow{g}&F_{j+u}(\Tot(B)^{m+v−u})\\
&=\prod_{l≤j+u}B^{m+v−u+l}_l\xrightarrow{×(−1)^{(i+u)m}} B^{m+j+v−i}_{j+u−i} ,
\end{align*}
where the last map is a projection and multiplication with the indicated sign.
\end{defin} 


Tot comes with column filtration, so recall that (It wouldn't hurt to filter my modules with the column filtration).  Tot being monoidal over a bigradaded category implies that the source and target categories must be enriched accordingly over the same category (see also Lemma  4.37)

Definition 4.42 for the weird End (Example 4.44 as well)

Definitionn 4.49 for Ainfty morphism of twisted complexes. Later, before 4.50 they say Ainfty algebra in twisted complex, I understand this means the Ainfty multiplications are morphisms in the category, but exactly which one (like, some enrichment, the same as in def 4.49?) IN THE ENRICHED CATEGORY SO A BETTER NOTATION WOULD USE THE CALIGRAPHIC UNDERLINED TC

Remark 4.52 to better understan filtrations in the last isomorphism. I would say $\End_K$ also has the trivial filtration, and then the fact that $m_i$ respect filtrations  comes from the fact that $\End_K$ is only made up of maps that respect the filtration, not from the fact that the operad is in filtered modules.  

\section{Results}
If the results need previous definitions or results that I haven't included, indicate it and add them (even if they're above it might be useful to refer to the definitions)


Theorem 3.8 with the previous definitions of Tot and d on morphisms

Theorem 3.11, remark that forgetting about the filtration the result is still true

Lemma 4.27, I should probably try to understand  the remark in the proof because the signs come either from the bijection (adjunction) or it was there before the bijection.

Lemma 4.37, maybe the filtratioon can be omitted as in Theorem 3.11. This one needs Proposition 3.11 but also some theory that I have skipped for now

Theorem 4.39, checking this proof might shed some light onto why we need filtrations to extend Tot to any bidegree. The proof makes perfect sense when one understand all the adjoint stuff. The filtrations go back to the definition of fMod itself, where I think the key is (in the end it is used to enrich it over bigraded modules, but why it is bigraded the way it is?) HOW DOES THE BIGRADING ON FMOD AFFECT THIS PROOF? I SHOULD CHECK MU AND TOT, BUT I THINK IT JUST MAKES THE NECESSARRY SIGNS BE LEFT IN THE END

Proposition 4.46 which uses the categorical machinery, I would like to know how the enriched functor things works for instance if we are working with twisted complexes (so that the weird End is as in Example 4.44). In the case of totalization, how can you enrich source and target to have the same enrichment whhen Tot always goes from bigraded to graded: $\ufC$ and $\utC$ both are enriched over $\vbc$, thanks to the filtration, since it gives $\ufMod$ (and thus $\ufC$) the extra grading

Propositioon 4.47. A very important one that I have checked before, but may check again in more detail after understanding monoidal structures. Should clear what $\mathrm{End}_A$ is in this theorem. Since it is in $\vbc$ it should come with some of the enrichment (the one giving it a vertical diferential), that should be more clear SINCE WE ARE IMPOSING MU GOES TO DA IT DOESNT MATTER

Theorem 4.50? May be useful

NEXT SECTION 4.6

Lemma 4.54

PROPOSITION 4.11 

Proposition 4.40

Proposition 4.55


Original version ($\End_K$ for $K$ filered includes all morphisms of complexes)

Let $K$ be a filtered complex and let $\End_K$ the operad in cochain complexes of the underlying complex $K$. We filter $\End_K(n)$ as follows
\[F_p\End_K(n)=\{f\in\End_K(n)\mid f(F_qK^{\otimes n})\subset F_{q+p}K\ \forall q\in\Z\}\]
Note that elements of $F_p\End_K(n)$ may have non-zero degree.
\begin{thm}
Let $(A,d^A)$ be a  $(\N,\Z)$-graded twisted complex (or at leas horizontally bounded below). Let $\End_A$ the endomorphism operad in vertical bicomplexes of the bigraded module $A$ with vertical differential given by $[d_0,-]$. We have natural bijections
\[\Hom_{vbOp,d^A}(dA_\infty,\End_A)\cong \Hom_{vbOp}(A_\infty,\underline{\End}_A)\cong  \Hom_{vbOp}(A_\infty,\underline{\End}_{\Tot(A)})\cong\Hom_{fCOp}(A_\infty,\End_{\Tot(A)})\]
where $vbOp$ and $fCOp$ denote the categories of operads in $\vbc$ and $\fc$ respectively, and $\Hom_{vbOp,d^A}$ denotes the subset of morphisms which send $μ_{i1}$ to $d^A_i$. We view $A_∞$ as an operad in $\vbc$ sitting in horizontal degree zero or as an operad in filtered complexes with trivial filtration.
\end{thm}

Version 2 ($\End_K$ includes only morphisms of filtered complexes)
 \begin{thm}
Let $(A,d^A)$ be a  $(\N,\Z)$-graded twisted complex (or at leas horizontally bounded below). Let $\End_A$ the endomorphism operad in vertical bicomplexes of the bigraded module $A$ with differential given by $[d_0,-]$. We have natural bijections
\[\Hom_{vbOp,d^A}(dA_\infty,\End_A)\cong \Hom_{vbOp}(A_\infty,\underline{\End}_A)\cong  \Hom_{vbOp}(A_\infty,\underline{\End}_{\Tot(A)})\cong\Hom_{COp}(A_\infty,\End_{\Tot(A)})\]

where $vbOp$ and $COp$ denote the categories of operads in vertical bicomplexes and cochain complexes respectively, and $\Hom_{vbOp,d^A}$ denotes the subset of morphisms which send $μ_{i1}$ to $d^A_i$. We view $A_∞$ as an operad in $\vbc$ sitting in
horizontal degree zero or as an operad in cochain complexes.
\end{thm}

Hopeful alternative
\begin{thm}
Let $A$ be a bigraded module
\[\Hom_{bgOp}(dA_\infty,\End_A)\cong \Hom_{bgOp}(A_\infty,\underline{\End}_A)\cong  \Hom_{bgOp}(A_\infty,\underline{\End}_{\Tot(A)})\cong\Hom_{gOp}(A_\infty,\End_{\Tot(A)})\]

where $bgOp$ and $gOp$ denote the categories of operads in bigraded modules and graded modules respectively. We view $A_∞$ as an operad in bigraded modules sitting in
horizontal degree zero or as an operad in graded modules complexes.
\end{thm}
This would have two version depending on the last $\End$, but I would choose the simpler one. This reformulation may need some redefinition of the underlined operads (but probably just omitting differentials). Of course the definitions of the operads are slightly different. I would need to be careful with the enrichment as well, especially for the isomorphism connecting $A$ and $\Tot(A)$, I should check that the necessary conditions hold.

\end{document}
