\documentclass[Thesis.tex]{subfiles}
%\usepackage{estilo-ejercicios}%
%\setcounter{section}{0}
%\newtheorem{defin}{Definition}[section]
%\newtheorem{lem}[defin]{Lemma}
%\newtheorem{propo}[defin]{Proposition}
%\newtheorem{thm}[defin]{Theorem}
%\newtheorem{eje}[defin]{Example}
\doublespacing
\sloppy

%\usepackage{calligra}
%\usepackage[T1]{fontenc}
%\usepackage{empheq}
%\newcommand*\widefbox[1]{\fbox{\hspace{2em}#1\hspace{2em}}}
%\usepackage{adjustbox}
%--------------------------------------------------------
\begin{document}





\chapter{Derived $A_\infty$-algebras on operads}\label{deriveddef}
A lot of the research on $A_\infty$-algebras relies on the existence and uniqueness of minimal models for dgas. This is guaranteed by the results of Kadeishvili \cite{kade} when the dgas and their homologies are assumed to be degree-wise projective. In practice, this is implied by assuming a ground field. However, there are important examples arising from homotopy theory where projectivity cannot be guaranteed. In 2008, Sagave introduced the notion of derived $A_\infty$-algebras, providing a framework for not necessarily projective modules over an arbitrary commutative ground ring \cite{sagave}.

In this chapter we recall some definitions about derived $A_\infty$-algebras and their motivation through minimal models. We also present some new ways of interpreting them in terms of operads and collections. We then recall the notion of filtered $A_\infty$-algebra, since it will play a role in obtaining derived $A_\infty$-algebras from $A_\infty$-algebras on totalization. We combine a bigraded operadic suspension with totalization to encode derived $A_\infty$-algebras. Using suitable brace structures we are able to define derived $A_\infty$-algebra structures on certain operads and in turn show theorem \Cref{bigradedtheorem}, which generalizes \Cref{theorem} to the derived setting. From this follows our major result, \Cref{dainftydeligne}, a derived version of the Deligne conjecture.


\section{Derived $A_\infty$-algebras}
In this section we introduce derived $A_\infty$-algebras. We first give some definitions and then motivate them by explaining how they generalize the theory of minimal models that we saw in \Cref{minimalmodels}. 

\subsection{Definitions}
Here we recall some definitions about derived $A_\infty$-algebras from \cite{sagave} and provide some operadic interpretations. We also refer to \Cref{bigradedbackground} to recall some definitions and sign conventions.

  \begin{defin}
  Using the notation in \cite{RW}, a \emph{derived $A_\infty$-algebra} on a $(\Z,\Z)$-bigraded $R$-module $A$ consist of a family of $R$-linear maps 
\[m_{ij}:A^{\otimes j}\to A\]
of bidegree $(i,2-(i+j))$ for each $j\geq 1$, $i\geq 0$, satisfying the equation

\begin{equation}\label{dainftyequation}
\underset{\mathclap{j=r+1+t}}{\underset{\mathclap{v=j+q-1}}{\sum_{u=i+p}}}(-1)^{rq+t+pj}m_{ij}(1^{\otimes r}\otimes m_{pq}\otimes 1^{\otimes t})=0
\end{equation}
for all $u\geq 0$ and $v\geq 1$. 
\end{defin}

According to the above definition, there are two equivalent ways of defining the operad of derived $A_\infty$-algebras $d\calA_\infty$ depending on the underlying category. One of them works on the category of bigraded modules $\bgmod$ and the other one is suitable for the category of vertical bicomplexes $\vbc$. This is similar to the alternative definition of the $\calA_\infty$-operad in \Cref{internal}. We give the two definitions here as we are going to use both.

\begin{defin}\label{dainftyoperad}
The operad $d\calA_\infty$ in $\bgmod$ is the operad generated by $\{m_{ij}\}_{i\geq 0,j\geq 1}$ subject to the derived $A_\infty$-relation

\[\underset{j=r+1+t}{\underset{v=j+q-1}{\sum_{u=i+p}}}(-1)^{rq+t+pj}m_{ij}\circ_{r+1} m_{pq}=0\]
%\[\underset{j=r+1+t}{\sum_{\mathclap{u=i+p, v=j+q-1}}}(-1)^{rq+t+pj}m_{ij}\circ_{r+1} m_{pq}=0\]
for all $u\geq 0$ and $v\geq 1$. 

The operad $d\calA_\infty$ in $\vbc$ is the dg operad generated by $\{m_{ij}\}_{(i,j)\neq (0,1)}$ with vertical differential given by
\[\partial_\infty(m_{uv})=-\underset{\mathclap{(i,j)\neq (0,1)\neq (p,q)}}{\underset{j=r+1+t}{\sum_{\mathclap{u=i+p, v=j+q-1}}}}(-1)^{rq+t+pj}m_{ij}\circ_{r+1} m_{pq}.\] %quasi-free -> dg operad whose underlying graded operad is free
\end{defin}

\pagebreak
\begin{defin}
Let $A$ and $B$ be derived $A_\infty$-algebras with respective structure maps $m^A$ and $m^B$. An \emph{$\infty$-morphism of derived $A_\infty$-algebras} $f:A\to B$ is a family of maps $f_{st}:A^{\otimes t}\to B$ of bidegree $(s,1-s-t)$ satisfying
%\begin{align}\label{dinftymaps}
%\underset{j=r+1+t}{\sum_{u=i+p, v=j+q-1}}(-1)^{rq+t+pj}f_{ij}(1^{\otimes r}\otimes m_{pq}^A\otimes 1^{\otimes s})=&\\
%\underset{v=q_1+\cdots +q_j}{\sum_{u=i+p_1+\cdots +p_j}}(-1)^{\epsilon} & m^B_{ij}(f_{p_1 q_1}\otimes\cdots\otimes f_{p_j q_j})\nonumber
%\end{align}
\begin{align}\label{dinftymaps}
\underset{j=r+1+t}{\underset{v=j+q-1}{\sum_{u=i+p}}}(-1)^{rq+t+pj}f_{ij}(1^{\otimes r}\otimes m_{pq}^A\otimes 1^{\otimes s})=&\\
\underset{v=q_1+\cdots +q_j}{\sum_{\mathclap{u=i+p_1+\cdots +p_j}}}(-1)^{\epsilon} & m^B_{ij}(f_{p_1 q_1}\otimes\cdots\otimes f_{p_j q_j})\nonumber
\end{align}
for all $u\geq 0$ and $v\geq 1$, where
\[\epsilon = u +\sum_{1\leq w < l \leq j} q_w(1-p_l-q_l)  + \sum_{w=1}^j p_w(j-w).\]
%I am confindent that this is the same as in RW, it is a matter of grouping differently (taking in to account how many times things are added up) and sometimes change w by j-w. But maybe I should write it down.
\end{defin}
\begin{ex}\
\begin{enumerate}
\item An $A_\infty$-algebra is the same as a derived $A_\infty$-algebra such that $m_{ij}$ vanishes for all $i>0$.
\item One can check that, on any derived $A_\infty$-algebra $A$, the maps $d_i=(-1)^{i}m_{i1}$ define a twisted complex structure. This leads to the possibility of defining a derived $A_\infty$-algebra as a twisted complex with some extra structure, see \Cref{equivalent}.

\end{enumerate}
\end{ex}


Analogously to \Cref{ainftymultiplication}, we have the following.

\begin{defin}\label{derivedmultiplication}
A \emph{derived $A_\infty$-multiplication} on a bigraded operad $\OO$ is a map of operads $d\calA_\infty\to\OO$.
\end{defin}

\subsection{Minimal models}\label{derivedminimalmodels}


We would like to motivate the introduction of derived $A_\infty$-algebras by stating the derived version of minimal models that we saw for $A_\infty$-algebras in \Cref{minimalmodels}. In order to do that, we need some previous definitions that we take from \cite{sagave}. We also refer to this paper for all the technical details.

\begin{defin}
A \emph{bidga} is a monoid in the category of bicomplexes. Equivalently, a bidga is a derived $A_\infty$-algebra with $m_{ij} = 0$ for $i + j \geq 3$.
\end{defin}

Recall that a quasi-isomorphism of $A_\infty$-algebras is a morphism of $A_\infty$-algebras that induces a quasi-isomorphism of complexes with respect to $m_1$. In the case of derived $A_\infty$-algebras, the role of the quasi-isomorphisms is played by the so called $E_2$-equivalences, see \cite{spectral} for more details about these equivalences.

\begin{remark}
The equations (\ref{dainftyequation}) defining a derived $A_\infty$-structure include $m_{01}m_{01} = 0$. For a derived $A_\infty$-algebra
$A$, let $H^*_{ver}(A)$ denote its homology with respect to the \emph{vertical differential} $m_{01}$.
The map $m_{01}$ is called the vertical differential because it raises the vertical degree.

Since the equations defining a derived $A_\infty$-algebra also include
\[
m_{21}m_{01} - m_{11}m_{11} + m_{01}m_{21} = 0,
\]
it follows that the map $m_{11}$ becomes a differential in horizontal direction on the bigraded
module $H^*_{ver}(A)$. Therefore, we can form $H^*_{hor}(H^*_{ver}(A)) = H^*(H^*_{ver}(A);m_{11})$.
\end{remark}

\begin{defin}
An $\infty$-morphism $f : A \to B$ of derived $A_\infty$-algebras
is called an \emph{$E_2$-equivalence} if $H^*_{hor}(H^*_{ver}(f_{01}))$
is an isomorphism of $R$-modules.
\end{defin}

We would like to extend some applications of $A_\infty$-algebras to differential graded
algebras that are not necessarily projective over the ground ring $R$ or whose homology
is not projective. The problem we encounter is that not all differential graded algebras
possess a minimal model as an $A_\infty$-algebra. However, Sagave showed that dgas have
reasonable minimal models in the world of derived $A_\infty$-algebras. 
For this, one has to apply a special projective resolution.

\begin{defin}
Let $A$ be a graded algebra. A \emph{termwise $R$-projective
resolution} of $A$ is a termwise $R$-projective
bidga $P$ with $m_{01} = 0$ together with an $E_2$-equivalence $P \to A$.
\end{defin}

The following definition is analogue to \Cref{minimal}

\begin{defin}
A derived $A_\infty$-algebra is called \emph{minimal} if $m_{01} = 0$.
\end{defin}

Finally, we can state the derived version of \Cref{minimaltheorem}.
\begin{thm}\cite[Theorem 1.1]{sagave}
Let $A$ be a dga over $R$. Then there is a degree-wise
$R$-projective derived $A_\infty$-algebra $E$ together with an $E_2$-equivalence $E \to A$ such that
\begin{itemize}
\item $E$ is minimal,
\item $E$ is well-defined up to $E_2$-equivalence,
\item together with the differential $m_{11}$ and the multiplication $m_{02}$, $E$ is a termwise $R$-projective
resolution of the graded algebra $H^*(A)$.
\end{itemize}
\end{thm}

\section{Filtered $A_\infty$-algebras}

We will make use of the filtration induced by the totalization functor in order to relate classical $A_\infty$-algebras to derived $A_\infty$-algebras. For this reason, we recall the notion of filtered $A_\infty$-algebras.


\begin{defin}
A \emph{filtered} $A_\infty$-algebra is an $A_\infty$-algebra $(A,m_i)$ together with a filtration $\{F_pA^i\}_{p∈\Z}$
on each $R$-module $A^i$ such that for all $i ≥ 1$ and all $p_1,\dots , p_i ∈ \Z$ and $n_1,\dots , n_i ≥ 0$,
\[m_i(F_{p_1}A^{n_1} ⊗ \cdots ⊗ F_{p_i}A^{n_i} ) ⊆ F_{p_1+\cdots
+p_i}A^{n_1+\cdots+n_i+2−i}.\]
\end{defin}


\begin{remark}\label{filterversion}
Consider $\calA_∞$ as an operad in filtered complexes with the trivial filtration and let $K$
be a filtered complex. There is a one-to-one correspondence between filtered $A_∞$-algebra structures on $K$ and
morphisms of operads in filtered complexes $\calA_\infty → \underline{\End}_K$ (recall $\underline{\Hom}$ from \Cref{filterend}). To see this, notice that if one forgets the
filtrations such a map of operads gives an $A_∞$-algebra structure on $K$. The fact that this is a map of operads
in filtered complexes implies that all the $m_i$'s respect the filtrations. 

Since the image of $\calA_\infty$ lies in $\End_K=F_0\underline{\End}_K$, if we regard $\calA_\infty$ as an operad in cochain complexes, then we get a one-to-one correspondence between filtered $A_\infty$-algebra structures on $K$ and
morphisms of operads in cochain complexes $\calA_∞ → \End_K$.
\end{remark}

We will not need to distinguish between morphisms and $\infty$-morphisms in the filtered case, so we give the following definition. 

\begin{defin}
A \emph{morphism of filtered $A_∞$-algebras} from $(A,m_i, F)$ to $(B,m_i, F)$ is an $\infty$-morphism
$f : (A,m_i) → (B,m_i)$ of $A_∞$-algebras such that each map $f_j : A^{⊗j} → A$ is compatible with filtrations, i.e.
\[f_j(F_{p_1}A^{n_1} ⊗ \cdots ⊗ F_{p_j}A^{n_j} ) ⊆ F_{p_1+\cdots +p_j}B^{n_1+\cdots +n_j+1−j} ,\]
for all $j ≥ 1$, $p_1,\dots p_j ∈ \Z$ and $n_1,\dots , n_j ≥ 0$.
\end{defin}

We will study the notions from this section from an operadic point of view. For this purpose we introduce some useful constructions in the next section.

\section{Operadic totalization and vertical operadic suspension}\label{operadic}
We extend the totalization functor from \Cref{total} to the category of bigraded operads. We then extend operadic suspension from \Cref{Sec2} to the bigraded setting. Combining these two devices we can use results of classical $A_\infty$-algebras to study derived $A_\infty$-algebras.
\subsection{Operadic totalization}

By \Cref{monoidal} and the fact that the image of an operad under a lax monoidal functor is also an operad \cite[Proposition 3.1.1(a)]{fresse}, the totalization functor defined in \Cref{total} will define a functor from operads in brigraded modules (resp. twisted complexes) to operads in graded modules (resp. cochain complexes).

Therefore, let $\OO$ be either a bigraded operad, i.e. an operad in the category of bigraded $R$-modules or an operad in twisted complexes. We define $\Tot(\OO)$ as the operad of graded $R$-modules (or cochain complexes) for which \[\Tot(\OO(n))^d=\bigoplus_{i<0}\OO(n)^{d-i}_i\oplus\prod_{i\geq 0} \OO(n)^{d-i}_i\] is the image of $\OO(n)$ under the totalization functor. The differential would be as described after \Cref{totdef}. The insertion maps $\bar{\circ}_r$ are given by the composition  

\begin{equation}\label{insertion}
\Tot(\OO(n))\otimes \Tot(\OO(m))\xrightarrow{\mu} \Tot(\OO(n)\otimes \OO(m)) \xrightarrow{\Tot(\circ_r)} \Tot(\OO(n+m-1)),
\end{equation}
that is explicitly 
\[(x\bar{\circ}_ry)_k\coloneqq (\Tot(\circ_r)\circ \mu(x,y))_k=\sum_{k_1+k_2=k} (-1)^{k_1d_2} x_{k_1}\circ_r y_{k_2}\]

for $x=(x_i)_i\in \Tot(\OO(n))^{d_1}$ and $y=(y_j)_j\in \Tot(\OO(m))^{d_2}$.

More generally, operadic composition $\bar{\gamma}$ is defined by the composite
\[
\begin{tikzcd}
\Tot(\OO(N))\otimes \Tot(\OO(a_1))\otimes\cdots\otimes \Tot(\OO(a_N))\arrow[d, "\mu"]\\
 \Tot(\OO(N)\otimes \OO(a_1)\otimes\cdots\otimes \OO(a_N))\arrow[r, "\Tot(\gamma)"]& \Tot\left(\OO\left(\sum a_i\right)\right),
\end{tikzcd}
\]

This map can be computed explicitly by iteration of the insertions $\bar{\circ}_r$, giving the following.  %For simplicity, we abuse of notation by omitting sums

\begin{lem}\label{totcomp}
The operadic composition $\bar{\gamma}$ on $\Tot(\OO)$ is given by
\begin{equation*}%\label{totcomp}
\bar{\gamma}(x;x^1,\dots, x^N)_k=\sum_{k_0+k_1+\cdots+k_N=k}(-1)^{\varepsilon}\gamma(x_{k_0};x^1_{k_1},\dots, x^N_{k_N})
\end{equation*}
for $x=(x_k)_k\in\Tot(\OO(N))^{d_0}$ and $x^i=(x^i_k)_k\in\Tot(\OO(a_i))^{d_i}$, where 
\begin{equation}
\varepsilon=\sum_{j=1}^m d_j\sum_{i=0}^{j-1}k_i
\end{equation}
and $\gamma$ is the operadic composition on $\OO$. \qed
\end{lem}
Notice that the sign is precisely the same appearing in \Cref{mu}.

%It can be checked that this is indeed an operad of graded vector spaces 

\subsection{Vertical operadic suspension}\label{vertical}
On a bigraded operad we can use operadic suspension on the vertical degree with analogue results to those of the graded case that we explored in \Cref{Sec2}.

We define $\Lambda(n)=S^{n-1}R$, where  $S$ is a vertical shift of degree so that $\Lambda(n)$ is the underlying ring $R$ concentrated in bidegree  $(0,n-1)$. As in the single-graded case, we express the basis element of $\Lambda(n)$ as $e^n=e_1\land\cdots\land e_n$. Similarly, $\Lambda^-(n)=S^{1-n}R$ is defined.

The operad structure on the bigraded $\Lambda=\{\Lambda(n)\}_{n\geq 0}$ is the same as in the graded case, namely

\[
\begin{tikzcd}[column sep = 1.2em]
\Lambda(n)\otimes\Lambda(m) \arrow[r, "\circ_{r+1}"] & \Lambda(n+m-1)\\
(e_1\land\cdots\land e_n)\otimes(e_1\land\cdots\land e_m)\arrow[r, mapsto] & (-1)^{(n-r-1)(m-1)}e_1\land\cdots\land e_{n+m-1}.
\end{tikzcd}
\]


\begin{defin}
Let $\mathcal{O}$ be a bigraded linear operad, i.e. an operad on the category of bigraded $R$-modules. The \emph{vertical operadic suspension} $\mathfrak{s}\OO$ of $\mathcal{O}$ is given arity-wise by the Hadamard product of the operads $\OO$ and $\Lambda$, in other words, $\mathfrak{s}\OO(n)=(\mathcal{O}\otimes\Lambda)(n)=\mathcal{O}(n)\otimes\Lambda(n)$ with diagonal composition. Similarly, we define the \emph{vertical operadic desuspension} $\mathfrak{s}^{-1}\OO(n)=\mathcal{O}(n)\otimes\Lambda^-(n)$.
\end{defin}


We may identify the elements of $\mathcal{O}$ with the elements of $\mathfrak{s}\OO$. 
\begin{defin}
For $x\in\OO(n)$ of bidegree $(k,d-k)$, its \emph{natural bidegree} in $\s\OO$ is the pair $(k,d+n-k-1)$. To distinguish both degrees we call $(k,d-k)$ the \emph{internal bidegree} of $x$, since this is the degree that $x$ inherits from the grading of $\OO$. 
\end{defin}

If we write $\circ_{r+1}$ for the operadic insertion on $\OO$ and $\tilde{\circ}_{r+1}$ for the operadic insertion on $\mathfrak{s}\OO$, we may find a relation between the two insertion maps in a completely analogous way to \Cref{tilde}.

\begin{lem}\label{bitildecirc}
For $x\in\OO(n)$ and $y\in\OO(m)^{q}_l$ we have

\begin{equation}\label{sign}
x\tilde{\circ}_{r+1}y=(-1)^{(n-1)q+(n-1)(m-1)+r(m-1)}x\circ_{r+1} y. 
\end{equation}
\qed
\end{lem}

%\begin{proof}
%Let $x\in\OO(n)$ and $y\in\OO(m)^{q}_l$, and let us compute $x\tilde{\circ}_{r+1} y$, which is
%
%\begin{align*}
%\mathfrak{s}\OO(n)\otimes\mathfrak{s}\OO(m)&=(\OO(n)\otimes\Lambda(n))\otimes (\OO(m)\otimes\Lambda(m))\\
%&\cong (\OO(n)\otimes \OO(m))\otimes (\Lambda(n)\otimes \Lambda(m))\\
%&\xrightarrow{\circ_{r+1}\otimes\circ_{r+1}} \OO(m+n-1)\otimes \Lambda(n+m-1)\\
%&=\mathfrak{s}\OO(n+m-1).
%\end{align*}
%
%The symmetric monoidal structure produces the sign $(-1)^{(n-1)q}$ in the isomorphism $\Lambda(n)\otimes \OO(m)\cong\OO(m)\otimes\Lambda(n)$, and the operadic structure of $\Lambda$ produces the sign \[(-1)^{(n-1)(m-1)+r(m-1)},\] so 
%
%\[
%x\tilde{\circ}_{r+1}y=(-1)^{(n-1)q+(n-1)(m-1)+r(m-1)}x\circ_{r+1} y.
%\]
%
%\end{proof}
As can be seen, this is the same sign as the single-graded operadic suspension but with vertical degree. We will see that this is also the case more generally for bigraded braces in \Cref{sectionbibraces}. As a consequence we have the following theorem with similar proof to the single-graded case, where all the suspensions are vertical.
\begin{thm}
Given a bigraded $R$-module $A$, there is an isomorphism of operads $\End_{ A}\cong \mathfrak{s}\End_{SA}$.\qed
\end{thm}
Another consequence of \Cref{bitildecirc} is that $\tilde{\circ}$ leads to the Lie bracket from \cite{RW}, which implies that $m=\sum_{i,j}m_{ij}$ is a derived $A_\infty$-multiplication if and only if for all $u\geq 0$
\begin{equation}\label{sharp}
\sum_{i+j=u}\sum_{l,k}(-1)^im_{jl}\tilde{\circ}m_{ik}=0.
\end{equation}
In \cite[Proposition 2.15]{RW} this equation is described in terms of a sharp operator $\sharp$. 

We also get the functorial properties that we studied for the single-graded case in \Cref{functorial} and \Cref{monoidality}.

\subsection{Vertical suspension and totalization} 

Now we are going to combine vertical operadic suspension and totalization. More precisely, the \emph{totalized vertical suspension} of a bigraded operad $\OO$ is the graded operad $\Tot(\s\OO)$. This operad has an insertion map explicitly given by
\begin{equation}\label{star}
(x\star_{r+1} y)_k=\sum_{k_1+k_2=k}(-1)^{(n-1)(d_2-k_2-m+1)+(n-1)(m-1)+r(m-1)+k_1d_2}x_{k_1}\circ_{r+1}y_{k_2}
\end{equation}
for $x=(x_i)_i\in \Tot(\s\OO(n))^{d_1}$ and $x=(x_j)_j\in \Tot(\s\OO(m))^{d_2}$. As usual, denote 
\begin{equation}
x\star y=\sum_{r=0}^{m-1}x\star_{r+1}y.
\end{equation}
This star operation is precisely the star operation from \cite[\S 5.1]{LRW}, i.e. the convolution operation on $\Hom((dAs)^{¡}, \End_A)$. In particular, we can recover the Lie bracket from in \cite{LRW}. We will do this in \Cref{biliebracket}.


Before continuing, let us show a lemma that allows us to work only with the single-graded operadic suspension if needed.
\begin{propo}\label{extrasign}
For a bigraded operad $\OO$ we have an isomorphism $\Tot(\s\OO)\cong \s \Tot(\OO)$, where the suspension on the left hand side is the bigraded version and on the right hand side is the single-graded version. 
\end{propo}

\begin{proof}
 We may identify each element $x=(x_k\otimes e^n )_k\in\Tot(\s\OO(n))$ with the element $x=(x_k)_k\otimes e^n\in\s\Tot(\OO(n))$. Thus, abusing of notation, for an element $(x_k)_k\in \Tot(\s\OO(n))$ the isomorphism is given by
\begin{align*}
f:\Tot(\s\OO(n))&\cong \s \Tot(\OO(n))\\
(x_k)_k&\mapsto ((-1)^{kn}x_k)_k
\end{align*}
Clearly, this map is bijective so we just need to check that it commutes with insertions. Recall from \Cref{star} that the insertion on $\Tot(\s\OO)$ is given by
\begin{equation*}
(x\star_{r+1} y)_k=\sum_{k_1+k_2=k}(-1)^{(n-1)(d_2-k_2-n+1)+(n-1)(m-1)+r(m-1)+k_1d_2}x_{k_1}\circ_{r+1}y_{k_2}
\end{equation*}
for $x=(x_i)_i\in \Tot(\s\OO(n))^{d_1}$ and $y=(y_j)_j\in \Tot(\s\OO(m))^{d_2}$. Similarly, we may compute the insertion on $\s\Tot(\OO)$ by combining the sign produced first by $\Tot$ and then by $\s$. This results in the following insertion map 
\begin{equation*}
(x\star_{r+1}' y)_k=\sum_{\mathclap{k_1+k_2=k}}(-1)^{(n-1)(d_2-n+1)+(n-1)(m-1)+r(m-1)+k_1(d_2-m+1)}x_{k_1}\circ_{r+1}y_{k_2}
\end{equation*}
for $x=(x_i)_i\in \s\Tot(\OO(n))^{d_1}$ and $y=(y_j)_j\in \s\Tot(\OO(m))^{d_2}$. Now let us show that $f(x\star y)=f(x)\star f(y)$. We do this by showing that all the insertions are equal on both sides. By definition, we have that $f((x\star_{r+1} y))_k$ equals the following.

\begin{align*}
&\sum_{k_1+k_2=k}(-1)^{k(n+m-1)+(n-1)(d_2-k_2-n+1)+(n-1)(m-1)+r(m-1)+k_1d_2}x_{k_1}\circ_{r+1}y_{k_2}\\
&=\sum_{\mathclap{k_1+k_2=k}}(-1)^{(n-1)(d_2-n+1)+(n-1)(m-1)+r(m-1)+k_1(d_2-m+1)}f(x_{k_1})\circ_{r+1}f(y_{k_2})\\
&=(f(x)\star_{r+1} f(y))_k
\end{align*}
%This is precisely $(f(x)\star f(y))_k$.
%I skipped one calculation but it is just that
as desired.
\end{proof}


\begin{remark}\label{othermu}
As we mentioned in \Cref{heuristic}, there exist other possible ways of totalizing by varying the natural transformation $\mu$. For instance, we can choose the totalization functor $\Tot'$ which is the same as $\Tot$ but with a natural transformation $\mu'$ defined in such a way that the insertion on $\Tot'(\OO)$ is defined by \[(x\hat{\circ}y)_k=\sum_{k_1+k_2=k}(-1)^{k_2n_1}x_{k_1}\circ y_{k_2}.\] 

This is also a valid approach for our purposes and there is simply a sign difference. But we have chosen our convention to be consistent with other conventions, such as the derived $A_\infty$-equation. However, it can be verified that $\Tot'(\s\OO)=\s \Tot'(\OO)$. With the original totalization we have a non identity isomorphism given by \Cref{extrasign}. Similar relations can be found among the other alternatives mentioned in \Cref{heuristic}. %We will use this isomorphism later.



\end{remark}


Using the operadic structure on $\Tot(\s\OO)$, we can describe derived $A_\infty$-multiplications in a new conceptual way.

\begin{lem}\label{mstar}
A derived $A_\infty$-multiplication on a bigraded operad $\OO$ is equivalent to an element $m\in\Tot(\s\OO)$ of degree 1 concentrated in positive arity such that $m\star m = 0$. 
%$m=\sum_{ij}m_{ij}$ where $m_{ij}\in\OO(j)^{2-i-j}_i$ for each $j\geq 1$ such that 
%\[\underset{j=r+1+t}{\sum_{u=i+p, v=j+q-1}}(-1)^{rq+t+pj}m_{ij}\circ_{r+1} %m_{pq}=0.\]
\end{lem}
\begin{proof}
A derived $A_\infty$-multiplication on $\OO$ is by \Cref{derivedmultiplication} a map 
\[f:d\calA_\infty\to\OO.\]
Since $\calA_\infty$ is generated by elements $\mu_{ij}$ of bidegree $(i,2-i-j)$, such a map is determined by the elements $m_{ij}=f(\mu_{ij})\in\OO^{2-i-j}_i(j)$. Consider the $A_\infty$-multiplication $m_j = (m_{ij})_i\in\Tot(\s\OO(j))$. We have that $\deg(m_j)=1$ for all $j$. Therefore, let $m=m_1+m_2+\cdots\in\Tot(\s\OO)$. We may check that $m\star m=0$. For that we just need to check \Cref{star}. On arity $n$, this amounts to computing
\[(m\star m)_k = \sum_{r=0}^{n-1}\underset{j+q=n-1}{\sum_{i+p=k}}(-1)^{rp+j-r-1+ pj}m_{ij}\circ_{r+1}m_{pq}=0.\]
The above expression vanishes precisely because the elements $m_{ij}$ satisfy the derived $A_\infty$-equation.

Conversely, let $m\in\Tot(\s\OO)$ of degree 1, is concentrated in positive arity and satisfying $m\star m=0$. We can split $m$ into its arity and horizontal degree components as $m=\sum_{i,j}m_{ij}$. As we have seen, the fact that $m\star m=0$ is equivalent to the elements $m_{ij}$ satisfying the derived $A_\infty$-equation, and therefore, a map $f:d\calA_\infty\to\OO$ is determined by the images $f(\mu_{ij})=m_{ij}$, which are of bidegree $(i,2-i-j)$. 
\end{proof}

\begin{remark}
Similarly to \Cref{dg}, one can use the definition of $d\calA_\infty$ as an operad of vertical bicomplexes from \Cref{dainftyoperad}. In that case, one obtains in an analogous way to \Cref{mstar}, that a derived $A_\infty$-multiplication on of vertical bicomplexes $\OO$ with vertical differential $\partial$ is equivalent to an element of degree 1 concentrated on arity at least 2 satisfying the equation $\partial(m)+m\star m =0$. However, we stick to operads of bigraded modules for the sake of consistency.
\end{remark}

From \Cref{mstar}, since now $m$ is an $A_\infty$-multiplication on a single-graded operad, we can proceed as in the proof of \Cref{ainftystructure} to show that $m$ determines an $A_\infty$-algebra structure on $S\Tot(\s\OO)\cong S\s \Tot(\OO)$. The goal now is showing that this $A_\infty$-structure on $S\Tot(\s\OO)$ is equivalent to a derived $A_\infty$-structure on $S\s \OO$ and compute the structure maps explicitly. We will do this in \Cref{derivedstructure}. 

Before that, let us explore the brace structures that appear from this new operadic constructions and use them to reinterpret derived $\infty$-morphisms and their composition.



%\pagebreak
\section{Bigraded braces and totalized braces}\label{sectionbibraces}

In this section we generalize the brace algebras we saw in \Cref{sectionbraces} to bigraded and totalized operads. This will allow to continue our generalization towards a derived version of the Deligne conjecture by following similar steps to the $A_\infty$-case. We also use this generalization to reinterpret derived $\infty$-morphisms in a similar way to \Cref{reinterpretation}.

\subsection{Braces}
We are going to define a brace structure on $\Tot(\s\OO)$ using totalization. First note that one can define bigraded braces just like in the single-graded case, only changing the sign $\varepsilon$ in \Cref{braces} to be $\varepsilon=\sum_{p=1}^n\sum_{q=i}^{i_p}\langle x_p,y_q\rangle$ according to the bigraded sign convention.

As one might expect, we can define bigraded brace maps $b_n$ on a bigraded operad $\OO$ and also on its operadic suspension $\s\OO$, obtaining similar signs as in the single-graded case, but with vertical internal degrees, see \Cref{bracesign}. 

We can also define braces on $\Tot(\s\OO)$ via operadic composition. In this case, these are usual single-graded braces. More precisely, we define the maps 
\[b^\star_n:\Tot(\mathfrak{s}\OO(N))\otimes \bigotimes_{j=1}^n\Tot(\mathfrak{s}\OO(a_j))\longrightarrow \Tot\left(\mathfrak{s}\OO\left(N-n+\sum a_i\right)\right)\]
using the operadic composition $\gamma^\star$ on $\Tot(\mathfrak{s}\OO)$ as

\[b^\star_n(x;x_1,\dots,x_n)=\sum\gamma^\star(x;1,\dots,1,x_1,1,\dots,1,x_n,1,\dots,1),\]

where the sum runs over all possible ordering preserving insertions. The brace map $b^\star_n(x;x_1,\dots,x_n)$ vanishes whenever $n>N$ and $b^\star_0(x)=x$. %We use the notation $b^\star_n$ to distinguish this brace map from the bigraded brace $b_n$ that can be naturally defined on the bigraded operad $\s\OO$.

Operadic composition can be described in terms of insertions in the obvious way, namely 

\begin{equation}\label{gammastar}
\gamma^\star(x;y_1,\dots,y_N)=(\cdots(x\star_1 y_1)\star_{1+a(y_1)}y_2\cdots)\star_{1+\sum a(y_p)}y_N,
\end{equation}

where $a(y_p)$ is the arity of $y_p$ (in the case of a brace $y_p$ is either $1$ or some $x_i$). If we want to express this composition in terms of the composition $\gamma$ in $\OO$ we just have to find out the sign factor applying the same strategy as in the single-graded case. In fact, as we said, there is a sign factor that comes from vertical operadic suspension that is identical to the graded case, but replacing internal degree by internal vertical degree. This is the sign that determines the brace $b_n$ on $\s\OO$. Explicitly, it is given by the following lemma, whose proof is identical to the single-graded case, see \Cref{bracesign}.
 
 \begin{lem}\label{bigradedsign}
For $x\in \s\OO(N)$ and $x_i\in\s\OO(a_i)$ of internal vertical degree $q_i$ ($1\leq i\leq n$), we have
\[b_n(x;x_1,\dots,x_n)=\sum_{N-n=h_0+\cdots+h_n} (-1)^\eta \gamma
(x\otimes 1^{\otimes h_0}\otimes x_1\otimes \cdots\otimes x_n\otimes1^{\otimes h_n}),\]
where 
\[\eta=\sum_{0\leq j<l\leq n}h_jq_l+\sum_{1\leq j<l\leq n}a_jq_l+\sum_{j=1}^n (a_j+q_j-1)(n-j)+\sum_{1\leq j\leq l\leq n} (a_j+q_j-1)h_l.\]
\end{lem}
%\pagebreak
%\begin{remark}
% In particular, this operation leads to the Lie bracket from \cite{RW}, which implies that $m=\sum_{i,j}m_{ij}$ is a derived $A_\infty$-multiplication if and only if
%\begin{equation}\label{sharp}
%\sum_{i+j=u}\sum_{l,k}(-1)^jb_1(b_1(m_{jl};m_{ik});x)=0
%\end{equation}
%for the brace induced by $\tilde{\circ}$. 
%\end{remark}
The other sign factor is produced by totalization. This was computed in \Cref{totcomp}. Combining both factors we obtain the following.

\begin{lem}
We have 
\begin{align}\label{bracetot}
b_j^\star(x;x^1,\dots, x^N)_k=&\\
\underset{\mathclap{h_0+h_1+\cdots+h_N=j-N}}{\sum_{k_0+k_1+\cdots+k_N=k}}&(-1)^{\eta+\sum_{j=1}^m d_j\sum_{i=0}^{j-1}k_i}\gamma(x_{k_0};1^{h_0},x^1_{k_1},1^{h_1},\dots, x^N_{k_N},1^{h_N})\nonumber
\end{align}
for $x=(x_k)_k\in\Tot(\s\OO(N))^{d_0}$ and $x^i=(x^i_k)_k\in\Tot(\s\OO(a_i))^{d_i}$, where $\eta$ is defined in \Cref{bigradedsign}. \qed
\end{lem}

\begin{corollary}\label{biliebracket}
 For $\OO = \End_A$, where $A$ is a bigraded module, the brace $b_1^\star(f;g)$ is the operation $f\star g$ defined in \cite{LRW}. As a consequence,
\[
[f,g]=b_1^\star(f;g)-(-1)^{NM}b_1^\star(g;f)
\]
for $f\in\Tot(\s\End_A)^N$ and $g\in\Tot(\s\End_A)^M$, is the same Lie bracket that was defined in \cite{LRW}. \qed
\end{corollary}

Notice that in \cite{LRW} the sign in the bracket is $(-1)^{(N+1)(M+1)}$, but this is because their total degree differs by one with respect to ours.

\subsection{Reinterpretation of derived $\infty$-morphisms}

Just like we did for graded modules in \Cref{reinterpretation}, for bigraded modules $A$ and $B$ we may define the collection $\End^A_B=\{\Hom_R(A^{\otimes n}, B)\}_{n\geq 1}$ of bigraded modules. Recall that this collection has a left module structure over $\End_B$
\[\End_B\circ \End^A_B\to \End^A_B\]
given by composition of maps. Similarly, given a bigraded module $C$, we can define composition maps
\[\End^B_C\circ \End^A_B\to \End^A_C.\]
The collection $\End^A_B$ also has an infinitesimal right module structure over $\End_A$
\[\End^A_B\circ_{(1)}\End_A\to \End^A_B\]
given by insertion of maps.

Similarly to the single-graded case, we may describe derived $\infty$-morphisms in terms of the above operations.

\begin{lem}\label{dinfinitymorphism}
A derived $\infty$-morphism of $A_\infty$-algebras $A\to B$ with respective structure maps $m^A$ and $m^B$ is equivalent to a degree 0 element $f\in\Tot(\s\End^A_B)$ concentrated in positive arity such that

 \[\rho(f\circ_{(1)}m^A)=\lambda(m^B\circ f),\] 

where \[\lambda:\Tot(\mathfrak{s}\End_B)\circ \Tot(\mathfrak{s}\End^A_B)\to \Tot(\mathfrak{s}\End^A_B)\] is induced by the left module structure on $\End^A_B$, and \[\rho:\Tot(\mathfrak{s}\End_B)\circ_{(1)}\Tot(\mathfrak{s}\End^A_B)\to \Tot(\mathfrak{s}\End^A_B)\] is induced by the right infinitesimal module structure on $\End^A_B$. 

In addition, the composition of $\infty$-morphisms is given by the natural composition \[\Tot(\s\End^B_C)\circ \Tot(\s\End^A_B)\to \Tot(\s\End^A_C).\]
\end{lem}
\begin{proof}
Since $f_j=(f_{ij})_i\in\Tot(\s\End^A_B(j))$ is of degree $0$, we have that that $f_{ij}$ is of bidegree $(i,1-i-j)$. Thus, the equation

\[\rho(f\circ_{(1)}m^A)=\lambda(m^B\circ f)\] 

coincides up to signs with with the \Cref{dinftymaps}, the equation defining derived $\infty$-morphisms of derived $A_\infty$-algebras. The signs that appear in the above equation are obtained in a similar way to that on the brace $b_j^\star$, see \Cref{bracetot}. Thus, it is enough to plug in the sign provided by \Cref{bracetot} from the corresponding degrees and arities to obtain the desired result. The composition of derived $\infty$-morphisms follows similarly.
\end{proof}

In the case that $f:A\to A$ is an $\infty$-endomorphism, the above definition can be written in terms of operadic composition as $f\star m=\gamma^\star(m\circ f)$, where $\gamma^\star$ is the composition map derived from the operation $\star$, see \Cref{gammastar}. Here, $\circ$ is the plethysm of maps of collections, not to be confused with composition of maps. 

\section{The derived $A_\infty$-structure on an operad}\label{derivedstructure}


In this section we show that, under some reasonable assumptions, an operad with a derived $A_\infty$-multiplication is a derived $A_\infty$-algebra and compute the structure maps explicitly. From this structure we obtain a derived version of the Deligne conjecture for the Hochschild complex of a derived $A_\infty$-algebra.

\subsection{Derived $A_\infty$-structures}
As in the single-graded case, we identify $\s\OO = \prod_n \s\OO(n)$. We follow a strategy inspired by the proof of the following theorem to show that there is a derived $A_\infty$-structure on $S\s\OO$. We refer the reader to \Cref{categories} to recall the definitions of the categories used. %Note that $\Tot(SB)=S\Tot(B)$ for any bigraded module $B$, where $SB$ is the vertical suspension of $B$ and $S\Tot(B)$ is the suspension of $\Tot(B)$ as graded modules. %We identify with the underlying bigraded module, We are going to follow, The proof can be found in

\begin{thm}(\cite[Proposition 4.55]{whitehouse})\label{whitehouse}
Let $(A, d^A) ∈ \tc^b$ be a twisted complex horizontally bounded on the right and $A$ its underlying
cochain complex. We have natural bijections %this means that A has d0 as a differential and End_A has [d0,-]

\begin{align*}
\Hom_{\mathrm{vbOp},d^A}(d\calA_∞,\End_A) &\cong
\Hom_{\mathrm{vbOp}}(\calA_∞, \uEnd_A)\\
&\cong \Hom_{\mathrm{vbOp}}(\calA_∞, \uEnd_{\Tot(A)})\\
&\cong \Hom_{\mathrm{fCOp}}(\calA_∞,\underline{\End}_{\Tot(A)}),
\end{align*}
where $\vbOp$ and $\fCOp$ denote the categories of operads in $\vbc$ and $\fc$ respectively, and $\Hom_{\vbOp,d^A}$
denotes the subset of morphisms which send $μ_{i1}$ to $d^A_i$. We view $\mathcal{A}_∞$ as an operad in $\vbc$ sitting in
horizontal degree zero or as an operad in filtered complexes with trivial filtration.
\end{thm}




\begin{remark}\label{boundednessremark}
According to \Cref{filterversion}, the last isomorphism can be replaced by 
\[\Hom_{\mathrm{vbOp}}(\calA_∞, \uEnd_{\Tot(A)})\cong \Hom_{\mathrm{COp}}(\calA_∞,\End_{\Tot(A)}),\]
where $\mathrm{COp}$ is the category of operads in cochain complexes. 
\end{remark}

There are several important assumptions to make in order to use the theorem. First of all, we need $A$ to be horizontally bounded on the right, meaning that there exists some integer $i$ such that $A_k^{d-k}=0$ for all $k>i$. In our case, $A=S\s\OO$ for $\OO$ an operad with a derived $A_\infty$-multiplication $m$, so being horizontally bounded on the right implies that, for each $j>0$, we can only have at most a finite number of non-zero components $m_{ij}$. This situation happens in practice in all known examples of derived $A_\infty$-algebras so far, some of them are in \cite[Remark 6.5]{muro}, \cite{RW}, and \cite[\S 5]{women}. Under this assumption we may replace all direct products by direct sums, implying thus extra monoidality properties.

We also need to provide $A$ with a twisted complex structure. The reason for this is that \Cref{whitehouse} uses the definition of derived $A_\infty$-algebras on an underlying twisted complex, see \Cref{equivalent}. We show explicitly the existence of a twisted complex structure on an operad with derived $A_\infty$-multiplication in \Cref{twistedoperad}, but it also follows from \Cref{mi1}. We also provide another version of this theorem that works for bigraded modules, \Cref{alternative}. 

With these assumption, by \Cref{whitehouse} we can guarantee the existence of a derived $A_\infty$-algebra structure on $A$ provided that $\Tot(A)$ has an $A_\infty$-algebra structure.


\begin{thm}\label{derivedmaps}
Let $A=S\s\OO$ where $\OO$ is an operad horizontally bounded on the right with a derived $A_\infty$-multiplication $m=\sum_{ij}m_{ij}\in\OO$. Let $x_1\otimes\cdots\otimes x_j\in (A^{\otimes j})^{d-k}_k$ and let $x_v = Sy_v$ for $v=1,\dots, j$ and $y_v$ be of bidegree $(k_v,d_v-k_v)$. The following maps $M_{ij}$ for $j\geq 2$ determine a derived $A_\infty$-algebra structure on $A$.


\[M_{ij}(x_1,\dots,x_j)= (-1)^{\sum_{v=1}^j(j-v)(d_v-k_v)}\sum_lSb_j(m_{il};y_1,\dots, y_j). \]
\end{thm}
Note that we abuse of notation and identify $x_1\otimes\cdots\otimes x_j$ with an element of $\Tot(A^{\otimes j})$ with only one non-zero component. For a general element, extend linearly.

\begin{proof}
Since $m$ is a derived $A_\infty$-multiplication $\OO$, we have that $m\star m=0$ when we view $m$ as an element of $\Tot(\s\OO)$. By \Cref{ainftystructure}, this defines an $A_\infty$-algebra structure on $S\Tot(\s\OO)$ given by maps
 %This isomorphism introduces some signs that will cancel with the signs introduced by the firts isomorphism of the chain so we will omit them. 
\[M_j:(S\Tot(\s\OO))^{\otimes j}\to S\Tot(\s\OO)\]
induced by shifting brace maps
\[b_j^\star(m;-):(\Tot(\s\OO))^{\otimes j}\to \Tot(\s\OO).\]
 The graded module $S\Tot(\s\OO)$ is endowed with the structure of a filtered complex with differential $M_1$ and filtration induced by the column filtration on $\Tot(\s\OO)$. Note that $b^\star_j(m;-)$ preserves the column filtration since every component $b^\star_j(m_{ij};-)$ has positive horizontal degree. % since the shift is only vertical.
 
Since $S\Tot(\s\OO)\cong \Tot(S\s\OO)$, we can view $M_j$ as the image of a morphism of operads of filtered complexes $f:\mathcal{A}_\infty\to \End_{\Tot(S\s\OO)}$ in such a way that $M_j=f(\mu_j)$ for $\mu_j\in\mathcal{A}_\infty(j)$. 

We now work our way backwards using the strategy also employed by the proof of \Cref{whitehouse}. The isomorphism 
\[\Hom_{\mathrm{vbOp}}(\calA_∞, \uEnd_{\Tot(A)})\cong \Hom_{\mathrm{COp}}(\calA_∞,\End_{\Tot(A)})\]

does not modify the map $M_j$ at all but allows us to see it as a element of $\uEnd_{\Tot(A)}$ of bidegree $(0,2-j)$. 

The isomorphism 
\[\Hom_{\mathrm{vbOp}}(\calA_∞, \uEnd_A)\cong \Hom_{\mathrm{vbOp}}(\calA_∞, \uEnd_{\Tot(A)})\] 
in the direction we are following is the result of applying $\Hom_{\vbOp}(\calA_\infty,-)$ to the map described in \Cref{composition}. Under this isomorphism, $f$ is sent to the map \[\mu_j\mapsto \mathfrak{Tot}^{-1}\circ c(M_j,\mu^{-1})=\mathfrak{Tot}^{-1}\circ M_j\circ \mu^{-1},\] where $c$ is the composition in $\ufC$. The functor $\mathfrak{Tot}^{-1}$ decomposes $M_j$ into a sum of maps $M_j=\sum_i \widetilde{M}_{ij}$, where each $\widetilde{M}_{ij}$ is of bidegree $(i,2-j-i)$. More explicitly, let $A=S\s\OO$ and let $x_1\otimes\cdots\otimes x_j\in (A^{\otimes j})^{d-k}_k$. We abuse of notation and identify $x_1\otimes\cdots\otimes x_j$ with an element of $\Tot(A^{\otimes j})$ with only one non-zero component. For a general element, extend linearly. Then we have


\begin{align}\label{totsign}
\mathfrak{Tot}^{-1}(M_j( \mu^{-1}(x_1\otimes\cdots\otimes x_j)))=& \nonumber\\ 
\mathfrak{Tot}^{-1}(Sb_j^\star(m;(S^{-1})^{\otimes j}(\mu^{-1}(x_1\otimes\cdots\otimes x_j))))=&\nonumber\\
\sum_i(-1)^{id}\sum_l Sb_j^\star(m_{il};(S^{-1})^{\otimes j}(\mu^{-1}(x_1\otimes\cdots\otimes x_j)))=&\nonumber\\
\sum_i(-1)^{id}\sum_l(-1)^{\varepsilon} Sb_j(m_{il};(S^{-1})^{\otimes j}(\mu^{-1}(x_1\otimes\cdots\otimes x_j)))=&\nonumber\\
\sum_i\sum_l(-1)^{id+\varepsilon} Sb_j(m_{il};(S^{-1})^{\otimes j}(\mu^{-1}(x_1\otimes\cdots\otimes x_j)))
\end{align}
so that \[\widetilde{M}_{ij}(x_1,\dots,x_j)=\sum_l(-1)^{id+\varepsilon} Sb_j(m_{il};(S^{-1})^{\otimes j}(\mu^{-1}(x_1\otimes\cdots\otimes x_j))),\] where $b_j$ is the brace on $\s\OO$ and $\varepsilon$ is given in \Cref{totcomp}. 


According to the isomorphism 
\begin{equation}\label{firstiso}
\Hom_{\mathrm{vbOp},d^A}(d\calA_∞,\End_A)\cong
\Hom_{\mathrm{vbOp}}(\calA_∞, \uEnd_A),
\end{equation}
 the maps $M_{ij}=(-1)^{ij}\widetilde{M}_{ij}$ define an $A_\infty$-structure on $S\s\OO$. Therefore we now just have to work out the signs. Notice that $d_v$ is the total degree of $y_v$ as an element of $\s\OO$ and recall that $d$ is the total degree of $x_1\otimes\cdots\otimes x_j\in A^{\otimes j}$. Therefore, $\varepsilon$ can be written as
\[\varepsilon= i(d-j)+\sum_{1\leq v<w\leq j}k_vd_w.\]
The sign produced by $\mu^{-1}$, as we saw in \Cref{mui}, is precisely determined by the exponent 
\[\sum_{w=2}^jd_w\sum_{v=1}^{w-1}k_v=\sum_{1\leq v<w\leq j}k_vd_w,\]so this sign cancels the right hand summand of $\varepsilon$. This cancellation was expected since this sign comes from $\mu^{-1}$, and operadic composition is defined using $\mu$, see \Cref{insertion}. %In fact, both signs come from $\mu$, so the cancellation was expected. 
Finally, the sign $(-1)^{i(d-j)}$ left from $\varepsilon$ cancels with $(-1)^{id}$ in \Cref{totsign} and $(-1)^{ij}$ from the isomorphism (\ref{firstiso}). This means that we only need to consider signs produced by vertical shifts. This calculation has been done previously in \Cref{explicit} and as we claimed the result is 
\[M_{ij}(x_1,\dots,x_j)= (-1)^{\sum_{v=1}^j(j-v)(d_v-k_v)}\sum_lSb_j(m_{il};y_1,\dots, y_j). \]

\end{proof}

\begin{remark}\label{equivalent}
As in the case of $A_∞$-algebras in $\mathrm{C}_R$, see \Cref{internal}, it can be seen that
we have two equivalent descriptions of $A_∞$-algebras in $\tc$, see \cite{whitehouse}.

\begin{enumerate}[(1)]
\item A twisted complex $(A, d^A)$ together with a morphism $\calA_∞ → \uEnd_A$ of operads in $\vbc$, which is determined by a family of elements $M_i ∈ \utC(A^{⊗i},A)^{2−i}_0$ for $i ≥ 2$ for which the $A_\infty$-relations holds for $i\geq 2$, see \Cref{ainftyequation}. The composition is the one prescribed by the composition morphisms of $\utC$.
\pagebreak
\item A bigraded module $A$ with elements $M_i ∈ \ubgMod(A^{⊗i},A)^{2−i}_0$ for $i ≥ 1$ for
which all the $A_\infty$-relations hold, see \Cref{ainftyequation}. The composition is prescribed by the composition
morphisms of $\ubgMod$.
\end{enumerate}
Since the composition morphism
in $\ubgMod$ is induced from the one in $\utC$ by forgetting the differential, these two presentations
are equivalent.
\end{remark}

This equivalence allows us to formulate the following alternative version of \Cref{whitehouse}.
\begin{corollary}\label{alternative}
Given a bigraded module $A$ horizontally bounded on the right we have isomorphisms
\begin{align*}
\Hom_{\mathrm{bgOp}}(d\calA_∞,\End_A) &\cong
\Hom_{\mathrm{bgOp}}(\calA_∞, \uEnd_A)\\
&\cong \Hom_{\mathrm{bgOp}}(\calA_∞, \uEnd_{\Tot(A)})\\
&\cong \Hom_{\mathrm{fOp}}(\calA_∞,\underline{\End}_{\Tot(A)}),
\end{align*}
where $\mathrm{bgOp}$ is the category of operads of bigraded modules and $\mathrm{fOp}$ is the category of operads of filtered modules. %REFERENCE TO WHAT DEFINITIONS OF THE OPERADS I AM USING
\end{corollary}
\begin{proof}
Let us look at the first isomorphism

\[\Hom_{\mathrm{bgOp}}(\calA_∞, \uEnd_A)\cong \Hom_{\mathrm{bgOp}}(d\calA_∞,\End_A).\]

Let $f:\calA_\infty\to\uEnd_A$ be a map of operads in $\mathrm{bgOp}$. This is equivalent to maps in $\mathrm{bgOp}$
\[\calA_\infty(j)\to\uEnd_A(j)\]
for each $j\geq 1$, which are determined by elements $M_j\coloneqq f(\mu_j)\in\uEnd_A(j)$ for $v\geq 1$ of bidegree $(0,2-j)$ satisfying the $A_\infty$-equation with respect to the composition in $\ubgMod$. Moreover, $M_j\coloneqq (\tilde{m}_{0j},\tilde{m}_{1j},\dots)$ where $\tilde{m}_{ij}\coloneqq (M_j)_i:A^{\otimes n}\to A$ is a map of bidegree $(i,2-i-j)$. Since the composition in $\ubgMod$ is the same as in $\utC$, the computation of the $A_\infty$-equation becomes analogous to the computation done in \cite[Prop 4.47]{whitehouse}, showing that the maps $m_{ij}=(-1)^i\tilde{m}_{ij}$ for $i\geq 0$ and $j\geq 0$ define a derived $A_\infty$-algebra structure on $A$.

The second isomorphism
\[\Hom_{\mathrm{bgOp}}(\calA_∞, \uEnd_A)\cong \Hom_{\mathrm{bgOp}}(\calA_∞, \uEnd_{\Tot(A)})\]
follows from the bigraded module case of \Cref{inverse}. Finally, the isomorphism
\[\Hom_{\mathrm{bgOp}}(\calA_∞, \uEnd_{\Tot(A)})\cong \Hom_{\mathrm{fOp}}(\calA_∞,\underline{\End}_{\Tot(A)})\]
is analogous to the last isomorphism of \Cref{whitehouse}, replacing the quasi-free relation by the full $A_\infty$-algebra relations. 
\end{proof}

According to \Cref{alternative}, if we have an $A_\infty$-algebra structure on $A = S\s\OO$, we can consider its arity 1 component $M_1\in\underline{\End}_{\Tot(A)}$ and split it into maps $M_{i1}\in \End_A$. Since these maps must satisfy the derived $A_\infty$-relations, they define a twisted complex structure on $A$. The next corollary describes the maps $M_{i1}$ explicitly.
\pagebreak
\begin{corollary}\label{mi1}
Let $\OO$ be a bigraded operad with a derived $A_\infty$-multiplication and let \[M_{i1}:S\s\OO\to S\s\OO\] be the arity 1 derived $A_\infty$-algebra maps induced by \Cref{alternative} from \[M_1:\Tot(S\s\OO)\to \Tot(S\s\OO).\]
Then \[M_{i1}(x)= \sum_l (Sb_1(m_{il};S^{-1}x)-(-1)^{\langle x,m_{il}\rangle}Sb_1(S^{-1}x;m_{il})),\]
where $x\in (S\s\OO)^{d-k}_k$ and $\langle x,m_{il}\rangle=ik+(1-i)(d-1-k)$.
\end{corollary}
\begin{proof}
Notice that the proof of \Cref{alternative} is essentially the same as the proof \Cref{whitehouse}. This means that the proof of this result is an arity 1 restriction of the proof of \Cref{derivedmaps}. Thus, we apply \Cref{totsign} to the case $j=1$. Recall that for $x\in (S\s\OO)^{d-k}_{k}$,

\[M_1(x)=b_1^\star(m;S^{-1}x)-(-1)^{n-1}b_1^\star(S^{-1}x;m).\]
 In this case, there is no $\mu$ involved. Therefore, introducing the final extra sign $(-1)^i$ from the proof of \Cref{derivedmaps}, we get from \Cref{totsign} that
\begin{align*}
\widetilde{M}_{i1}(x)=(-1)^i&\sum_l((-1)^{id+i(d-1)} Sb_1(m_{il};S^{-1}x)\\
-(-1)^i&\sum_l(-1)^{d-1+id+k}Sb_1(S^{-1}x;m_{il})),
\end{align*} where $b_1$ is the brace on $\s\OO$. Simplifying signs we obtain
\[\widetilde{M}_{i1}(x)=\sum_l Sb_1(m_{il};S^{-1}x)-(-1)^{\langle  m_{il},x\rangle}Sb_1(m_{il};S^{-1}x))=M_{i1}(x),\]

where $\langle  m_{il},x\rangle=ik+(1-i)(d-1-k)$, as claimed.
\end{proof}

\subsection{The derived Deligne conjecture}


Note that the maps given by \Cref{derivedmaps} and \Cref{mi1} formally look the same as their single-graded analogues in \Cref{explicit} but with an extra index that is fixed for each $M_{ij}$. This means that we can follow the same procedure as in \Cref{sect3} to define higher derived $A_\infty$-maps on the Hochschild complex of a derived $A_\infty$-algebra. More precisely, given an operad $\OO$ with a derived multiplication and $A=S\s\OO$, we will define a derived $A_\infty$-algebra structure on $S\s\End_A$. We will then connect the algebraic structure on $A$ with the structure on $S\s\End_A$ through braces. This connection will allow us to formulate and show a new, more general version of the Deligne conjecture that generalizes the one that we obtained in \Cref{ainftydeligne}. 

Let $\overline{B}_j$ be the bigraded brace map on $\s\End_{S\s\OO}$ and consider the maps

\begin{equation}\label{barbimaps}
\overline{M}'_{ij}:(\s\End_{S\s\OO})^{\otimes j}\to \s\End_{S\s\OO}
\end{equation}
defined as 

\begin{align*}
&\overline{M}'_{ij}(f_1,\dots,f_j)=\overline{B}_j(M_{i\bullet};f_1,\dots, f_j) & j>1,\\
&\overline{M}'_{i1}(f)=\overline{B}_1(M_{i\bullet};f)-(-1)^{ip+(1-i)q}\overline{B}_1(f;M_{i\bullet}),
\end{align*}
for $f$ of natural bidegree $(p,q)$, where $M_{i\bullet}=\sum_j M_{ij}$. We define 
\begin{align*}
\overline{M}_{ij}& :(S\s\End_{S\s\OO})^{\otimes j}\to S\s\End_{S\s\OO},\\
 \overline{M}_{ij}& \coloneqq \overline{\sigma}(M'_{ij})=S\circ M'_{ij}\circ (S^{\otimes n})^{-1}.
\end{align*}

As in the single-graded case we can define a map
\[\Phi:S\s\OO\to S\s\End_{S\s\OO}\]
as the map making the following diagram commute
\begin{equation}\label{derivedPhi}
\begin{tikzcd}
S\s\OO\arrow[rr, "\Phi"]\arrow[d] & & S\s\End_{S\s\OO}\\
\s\OO\arrow[r, "\Phi'"]& \End_{\s\OO}\arrow[r, "\cong"]& \s\End_{S\s\OO}\arrow[u]
\end{tikzcd}
\end{equation}
where 
\begin{align*}
\Phi'\colon&\s\OO \to \End_{\s\OO}\\
&x\mapsto \sum_{n\geq 0}b_n(x;-).
\end{align*}
The isomorphism $\End_{\s\OO}\cong\s\End_{S\s\OO}$ is given by $\overline{\sigma}$. 

In this setting we have the bigraded version of \Cref{theorem}. But before stating the theorem, for the sake of completeness let us state the definition of the Hochschild complex of a bigraded module.

\begin{defin}
We define the \emph{Hochschild cochain complex} of a bigraded module $A$ to be the bigraded module $S\s\End_A$. If $(A,d)$ is a vertical bicomplex, then the Hochschild complex has a vertical differential given by $\partial(f)=[d,f]=d\circ f-(-1)^{q}f\circ d$, where $f$ has natural vertical degree $q$ and $\circ$ is the plethysm corresponding to operadic insertions.
\end{defin}
In particular, $S\s\End_{S\s\OO}$ is the Hochschild cochain complex of $S\s\OO$. If $\OO$ has a derived $A_\infty$-multiplication, then the differential of the Hochschild complex $S\s\End_{S\s\OO}$ is given by $\overline{M}_{01}$ from \Cref{barbimaps}.

The following is the same as \Cref{theorem} but carrying the extra index $i$ and using the bigraded sign conventions.
\begin{thm}\label{bigradedtheorem}
The map $\Phi$ defined in diagram (\ref{derivedPhi}) above is a morphism of $d\mathcal{A}_\infty$-algebras, i.e. for all $i\geq 0$ and $j\geq 1$ the equation

\[\Phi(M_{ij})=\overline{M}_{ij}(\Phi^{\otimes j})\]
holds.\qed%, where the $M_{ij}$ is the $j$-th component of the $A_\infty$-algebra structure on $S\s\OO$ and $\overline{M}_j$ is the $j$-th component of the $A_\infty$-algebra structure on $S\s\End_{S\s\OO}$. 
\end{thm}

As a consequence of this theorem, we can obtain a derived version of the Deligne conjecture. In order to formulate this new Deligne conjecture, we need to introduce the notion of \emph{derived $J$-algebra}, as a derived version of $J$-algebras introduced in \Cref{Jalgebras}. To have a more succinct formulation we use the notation $\vdeg(x)$ for the vertical degree of $x$.

\begin{defin}\label{derivedJalgebras}
A \emph{derived $J$-algebra} $V$ is a derived $A_\infty$-algebra with structure maps $\{M_{ij}\}_{i\geq 0, j\geq 1}$ such that the shift is $S^{-1}V$ a brace algebra. Furthermore, the braces and the derived $A_\infty$-structure satisfy the following compatibility relations. Let $x, x_1,\dots, x_j, y_1,\dots, y_n\in S^{-1}V$. %BIDEGREES? USE Q FOR VERTICAL DEGREE FOR SHORT IF NEEDED 
For all $n,i\geq 0$ we demand 

%OBTAIN SIGNS

%LHS IS IDENTICAL BY BARS MEAN VERTICAL DEGREE, ALTERNATIVE NOTATION? ON RHS ONLY EVALUATION SIGN DIFFERS WHEN Y PASSES BY X (IN THE SCALAR %PRODUCT THE VERTICAL DEGREES ARE MODIFIED, THINK OF A NOTATION OF SOME SORT. %FOR J=1 THE SIGN CHANGES SLIGHTLY TOO BECAUSE THE BRACKET SIGN IS BIGRADED, CHECK IF IT IS THE SAME AS DOING THE SCALAR PRODUCT WITH X
\begin{align*}
(-1)^{\sum_{i=1}^n(n-v)\mathrm{vdeg}(y_v)}Sb_n(&S^{-1}M_{i1}(Sx);y_1,\dots, y_n)=\\
&\underset{\mathclap{1\leq i_1\leq n-k+1}}{\sum_{\mathclap{l+k-1=n}}}(-1)^{\varepsilon}M_{il}(Sy_1,\dots, Sb_{k}(x;y_{i_1},\dots),\dots, Sy_n)\\
-(-1)^{\langle x,M_{il}\rangle}\underset{\mathclap{1\leq i_1\leq n-k+1}}{\sum_{\mathclap{l+k-1=n}}}&(-1)^{\eta} Sb_k(x;y_1,\dots, S^{-1}M_{il}(Sy_{i_1},\dots,), \dots, y_n)
\end{align*}
where
%\[
%\varepsilon = \sum_{v=1}^{i_1-1}(|y_v|+1)(|x|-k+1)+\sum_{v=0}^{k-1}(|y_{v+i_1}|+1)(k-v+1)+(l-1)|x|+(i_1-1)k.
%\]
\begin{align*}
\varepsilon = \sum_{v=1}^{i_1-1}\langle Sy_v,S^{1-k}x\rangle&+\sum_{v=1}^{k}\vdeg(y_{i_1+v-1})(k-v)+(l-i_1)\vdeg(x).
\end{align*}
and
\begin{align*}
\eta=& \sum_{v=1}^{i_1-1}(k-v)\vdeg(y_v)+l\sum_{v=1}^{i_1-1}\vdeg(y_v)\\
&+\sum_{v=i_1}^{i_1+l-1}(k-i_1)\vdeg(y_v)+\sum_{v=i_1}^{n-l}(k-v)\vdeg(y_{v+l})
\end{align*}
%\[
%\eta=l(k-1)+\sum_{v=1}^{i_1-1}(n+1-v)(|y_v|+1)+\sum_{v=i_1}^{i_1+l-1}(k-i_1)(|y_v|+1)+\sum_{v=i_1+l}^n(n-v)|(|y_v|+1).
%\]

For $j>1$ we demand
\begin{align*}
&(-1)^{\sum_{i=1}^n(n-v)\mathrm{vdeg}(y_v)}Sb_n(S^{-1}M_{ij}(Sx_1,\dots, Sx_j);y_1,\dots, y_n)=\\
&\sum(-1)^{\varepsilon}M_{il}(Sy_1,\dots, Sb_{k_1}(x_1;y_{i_1},\dots),\dots, Sb_{k_j}(x_j;y_{i_j},\dots),\dots, Sy_n).
\end{align*}
The unindexed sum runs over all possible choices of non-negative integers that satisfy $l+k_1+\cdots+k_j-j=n$ and over all possible ordering preserving insertions. The right hand side sign is given by 

\begin{align*}
\varepsilon =&\underset{1\leq v\leq k_t}{\sum_{\mathclap{1\leq t\leq j}}} \vdeg(y_{i_t+v-1})(k_v-v)+ \sum_{\mathclap{1\leq i< l\leq j}}k_v\vdeg(x_l)+\underset{\mathclap{i_{t}\leq v< i_{t+1}}}{\sum_{\mathclap{0\leq t< l\leq j}}}\langle Sy_v,S^{1-k_l}x_l\rangle\\
&+\sum_{\mathclap{0\leq v<l\leq j}}(i_{v+1}-i_v-k_v)\vdeg(S^{1-k_l}x_l)+\sum_{\mathclap{1\leq v\leq l\leq j}} \vdeg(x_v)(i_{l+1}-i_l-k_l)
\end{align*}
All the above shifts are vertical and we are setting $i_0=0$, $i_{j+1}=n+1$. 
\end{defin}

\begin{corollary}[The derived Deligne conjecture]\label{dainftydeligne}
If $A$ is a derived $A_\infty$-algebra horizontally bounded on the right, then its Hochschild complex $S\s\End_A$ is a derived $J$-algebra.
\end{corollary}
\begin{proof}
The result follows from \Cref{bigradedtheorem} analogously to \Cref{ainftydeligne} using the explicit expressions and signs given by \Cref{derivedmaps}, \Cref{mi1} and \Cref{bigradedsign}.
\end{proof}



\end{document}
