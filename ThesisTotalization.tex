\documentclass[Thesis.tex]{subfiles}
%\usepackage{estilo-ejercicios}%
%\setcounter{section}{0}
%\newtheorem{defin}{Definition}[section]
%\newtheorem{lem}[defin]{Lemma}
%\newtheorem{propo}[defin]{Proposition}
%\newtheorem{thm}[defin]{Theorem}
%\newtheorem{eje}[defin]{Example}


%\usepackage{calligra}
%\usepackage[T1]{fontenc}
%\usepackage{empheq}
%\newcommand*\widefbox[1]{\fbox{\hspace{2em}#1\hspace{2em}}}
%\usepackage{adjustbox}
%--------------------------------------------------------
\begin{document}





\chapter{Derived $A_\infty$-algebras on operads}

In this section we recall some definitions about derived $A_\infty$-algebras and present some ways of interpreting them in terms of operads and collections. We also recall the notion of filtered $A_\infty$-algebra, since it will play a role in obtaining derived $A_\infty$-algebras from $A_\infty$-algebras on totalization. We combine a bigraded operadic suspension with totalization to encode derived $A_\infty$-algebras. Using suitable brace structures we are able to define derived $A_\infty$-algebra structures on certain operads and in turn show theorem \Cref{bigradedtheorem}, which generalizes \Cref{theorem} to the derived setting.


\section{Derived $A_\infty$-algebras}\label{deriveddef}

%DEFINE THE OPERAD,BOTH FREE WITH A DIFFERENTIAL AND ALSO WITH ALL THE DIFFERENTIALS INSIDE
  \begin{defin}
  Using the notation in \cite{RW}, a \emph{derived $A_\infty$-algebra} on a $(\Z,\Z)$-bigraded $R$-module $A$ consist of a family of $R$-linear maps 
\[m_{ij}:A^{\otimes j}\to A\]
of bidegree $(i,2-(i+j))$ for each $j\geq 1$, $i\geq 0$, satisfying the equation
\[\underset{j=r+1+t}{\sum_{u=i+p, v=j+q-1}}(-1)^{rq+t+pj}m_{ij}(1^{\otimes r}\otimes m_{pq}\otimes 1^{\otimes t})=0\]
for all $u\geq 0$ and $v\geq 1$. 
\end{defin}

According to the above definition, there are two equivalent ways of defining the operad of derived $A_\infty$-algebras $d\calA_\infty$ depending on the underlying category. We give the two of them here as we are going to use both.

\begin{defin}
The operad $d\calA_\infty$ in $\bgmod$ is the operad generated by $\{m_{ij}\}_{i\geq 0,j\geq 1}$ subject to the derived $A_\infty$-relation

\[\underset{j=r+1+t}{\sum_{u=i+p, v=j+q-1}}(-1)^{rq+t+pj}\gamma(m_{ij};1^{ r}, m_{pq}, 1^{t})=0\]
for all $u\geq 0$ and $v\geq 1$. 

The operad $d\calA_\infty$ in $\vbc$ is the quasi-free operad generated by $\{m_{ij}\}_{(i,j)\neq (0,1)}$ with vertical differential given by
\[\partial_\infty(m_{uv})=-\underset{j=r+1+t, (i,j)\neq (0,1)\neq (p,q)}{\sum_{u=i+p, v=j+q-1}}(-1)^{rq+t+pj}\gamma(m_{ij};1^{ r}, m_{pq}, 1^{t}).\]
\end{defin}
%NOT EXACTLY THE SAME AS LWR, BUT THE ONE THAT SARAH USES

\begin{defin}
Let $A$ and $B$ be derived $A_\infty$-algebras with respective structure maps $m^A$ and $m^B$. An \emph{$\infty$-morphism of derived $A_\infty-algebras$} $f:A\to B$ is a family of maps $f_{st}:A^{\otimes t}\to B$ of bidegree $(s,1-s-t)$ satifying
\begin{equation}\label{dinftymaps}
\underset{j=r+1+t}{\sum_{u=i+p, v=j+q-1}}(-1)^{rq+t+pj}f_{ij}(1^{\otimes r}\otimes m_{pq}^A\otimes 1^{\otimes s})=\underset{v=q_1+\cdots +q_j}{\sum_{u=i+p_1+\cdots +p_j}}(-1)^{\epsilon} m^B_{ij}(f_{p_1 q_1}\otimes\cdots\otimes f_{p_j q_j})
\end{equation}
for all $u\geq 0$ and $v\geq 1$, where
\[\epsilon = u + \sum_{1\leq w < l \leq j} q_w(1-p_l-q_l)  + \sum_{w=1}^j p_w(j-w).\]
%I am confindent that this is the same as in RW, it is a matter of grouping differently (taking in to account how many times things are added up) and sometimes change w by j-w. But maybe I should write it down.
\end{defin}
\begin{ex}\
\begin{enumerate}
\item One can check that, on any derived $A_\infty$-algebra $A$, the maps $d_i=(-1)^{i}m_{i1}$ define a twisted complex structure. This leads to the possibility of defining a derived $A_\infty$-algebra as a twisted complex with some extra structure (see \Cref{equivalent}).
\item An $A_\infty$-algebra is the same as a derived $A_\infty$-algebra such that $m_{ij}=0$ for all $i>0$.
\end{enumerate}
\end{ex}




\begin{defin}\label{derivedmultiplication}
A \emph{derived $A_\infty$-multiplication} on a bigraded operad $\OO$ is a map of operads $d\calA_\infty\to\OO$.
\end{defin}

\section{Filtered $A_\infty$-algebras}

We will make use of the filtration induced by the totalization functor in order to relate classical $A_\infty$-algebras to derived $A_\infty$-algebras. For this reason, we recall the notion of filtered $A_\infty$-algebras.


\begin{defin}
A \emph{filtered} $A_\infty$-algebra is an $A_\infty$-algebra $(A,m_i)$ together with a filtration $\{F_pA^i\}_{p∈\Z}$
on each $R$-module $A^i$ such that for all $i ≥ 1$ and all $p_1,\dots , p_i ∈ \Z$ and $n_1,\dots , n_i ≥ 0$,
\[m_i(F_{p_1}A^{n_1} ⊗ \cdots ⊗ F_{p_i}A^{n_i} ) ⊆ F_{p_1+\cdots
+p_i}A^{n_1+\cdots+n_i+2−i}.\]
\end{defin}


\begin{remark}\label{filterversion}
Consider $\calA_∞$ as an operad in filtered complexes with the trivial filtration and let $K$
be a filtered complex. There is a one-to-one correspondence between filtered $A_∞$-algebra structures on $K$ and
morphisms of operads in filtered complexes $\calA_\infty → \underline{\End}_K$ (recall $\underline{\Hom}$ from \Cref{filterend}). To see this, notice that if one forgets the
filtrations such a map of operads gives an $A_∞$-algebra structure on $K$. The fact that this is a map of operads
in filtered complexes implies that all the $m_i$'s respect the filtrations. 

Since the image of $\calA_\infty$ lies in $\End_K=F_0\underline{\End}_K$, if we regard $\calA_\infty$ as an operad in chain complexes, then we get a one-to-one correspondence between filtered $A_\infty$-algebra structures on $K$ and
morphisms of operads in chain complexes $\calA_∞ → \End_K$.
\end{remark}

\begin{defin}
A \emph{morphism of filtered $A_∞$-algebras} from $(A,m_i, F)$ to $(B,m_i, F)$ is a morphism
$f : (A,m_i) → (B,m_i)$ of $A_∞$-algebras such that each map $f_j : A^{⊗j} → A$ is compatible with filtrations:
\[f_j(F_{p_1}A^{n_1} ⊗ \cdots ⊗ F_{p_j}A^{n_j} ) ⊆ F_{p_1+\cdots +p_j}B^{n_1+\cdots +n_j+1−j} ,\]
for all $j ≥ 1$, $p_1,\dots p_j ∈ \Z$ and $n_1,\dots , n_j ≥ 0$.
\end{defin}

We will study the notions from this section from an operadic point of view. For that purpose we introduce some useful constructions in the next section.

\section{Operadic totalization and vertical operadic suspension}\label{operadic}
\subsection{Operadic totalization}
%I THINK I NEED  (N,Z)-BRIGRADED MODULES TO MAKE SURE THAT HORIZONTAL DEGREE IS NON-NEGATIVE WHEN DEFINING M AS AN ELEMENT OF TSO

%We are going to apply the totalization with compact support functor of WHITEHOUSE 4.13 to operads and we are going to simply call it \emph{totalization (functor)} and denote it as $T$. WHITEHOUSE 3.11 the functor $T$ with domain the category of bigraded modules and bidegree $(0,0)$ morphisms is lax monoidal, so applying LAX MONOIDAL PRESERVES OPERADS we can conclude that it takes operads of bigraded modules to operads of graded modules.  

We are going to apply the totalization  functor defined in \Cref{total} to operads. By \Cref{monoidal} and the fact that the image of an operad under a lax monoidal functor is also an operad \cite[Proposition 3.1.1(a)]{fresse}, this will define a functor from operads in brigraded modules (resp. twisted complexes) to operads in graded modules (resp. chain complexes). %and we are going to simply call it \emph{totalization (functor)} and denote it as $\Tot$. Tthe functor $\Tot$ with domain the category $\bgmod$ (TWISTED COMPLEXES?) and bidegree $(0,0)$ morphisms (TWISTED COMPLEX MAPS?) is lax monoidal REFERENCE TO THEOREM, so applying LAX MONOIDAL PRESERVES OPERADS we can conclude that it takes operads of bigraded modules to operads of graded modules. 

Therefore, let $\OO$ be either a bigraded operad, i.e. an operad in te category of bigraded $R$-modules or an operad in twisted complexes. We define $\Tot(\OO)$ as the operad of graded $R$-modules (or chain complexes) for which \[\Tot(\OO(n))^d=\bigoplus_{i<0}\OO(n)^{d-i}_i\oplus\prod_{i\geq 0} \OO(n)^{d-i}_i\] is the image of $\OO(n)$ under the totalization functor and the insertion maps are given by the composition  %THE SECOND IF I WANT TO ORDER THEM BY HORIZONTAL DEGREE AND WRITE SUMS LIKE WHITEHOUSE and comes equipped with insertion maps \[a\bar{\circ}_rb=(-1)^{l(i+k)} a\circ_r b\]
\begin{equation}\label{insertion}
\Tot(\OO(n))\otimes \Tot(\OO(m))\xrightarrow{\mu} \Tot(\OO(n)\otimes \OO(m)) \xrightarrow{\Tot(\circ_r)} \Tot(\OO(n+m-1)),
\end{equation}
that is explicitly 
\[(x\bar{\circ}_ry)_k=\sum_{k_1+k_2=k} (-1)^{k_1d_2} x_{k_1}\circ_r y_{k_2}\]

for $x=(x_i)_i\in \Tot(\OO(n))^{d_1}$ and $y=(y_j)_j\in \Tot(\OO(m))^{d_2}$.

More generally, operadic composition $\bar{\gamma}$ is defined by the composite
\begin{equation*}
\Tot(\OO(N))\otimes \Tot(\OO(a_1))\otimes\cdots\otimes \Tot(\OO(a_N))\xrightarrow{\mu} \Tot(\OO(N)\otimes \OO(a_1)\otimes\cdots\otimes \OO(a_N)) \xrightarrow{\Tot(\gamma)} \Tot(\OO(\sum a_i)),
\end{equation*}

This map can be computed explicitly by iteration of the insertion $\bar{\circ}$, giving the following.  %For simplicity, we abuse of notation by omitting sums

\begin{lem}\label{totcomp}
The operadic composition $\bar{\gamma}$ on $\Tot(\OO)$ is given by
\begin{equation*}%\label{totcomp}
\bar{\gamma}(x;x^1,\dots, x^N)_k=\sum_{k_0+k_1+\cdots+k_N=k}(-1)^{\varepsilon}\gamma(x_{k_0};x^1_{k_1},\dots, x^N_{k_N})
\end{equation*}
for $x=(x_k)_k\in\Tot(\OO(N))^{d_0}$ and $x^i=(x^i_k)_k\in\Tot(\OO(a_i))^{d_i}$, where 
\begin{equation}
\varepsilon=\sum_{j=1}^m d_j\sum_{i=0}^{j-1}k_i
\end{equation}
and $\gamma$ is the operadic composition on $\OO$.
\end{lem}
Notice that the sign is precisely the same appearing in \Cref{mu}.
%MAYBE I SHOULD WRITE THINGS LIKE SUMS BUT I THINK IT MAKES SENSE TO WRITE IT LIKE THIS BECAUSE I KNOW THIS COMES FROM A BIGRADED MODULE
%where $a\in\OO(n)^k_i$, $b\in\OO(m)^j_l$ and $\circ_r$ is the insertion map in $\OO$.

%It can be checked that this is indeed an operad of graded vector spaces I 


%COMPOSITION OF ARBITRARY BIGRADING IS PRESERVED BY TOT SINCE ALL SIGNS INVOLVED ARE HORIZONTAL DEGREE SO IT IS ANALOGUE TO WHITEHOUSE (WRITE DOWN THE CALCULATIONS IF NEEDED)

\subsection{Vertical operadic suspension}
On an bigraded operad we can use operadic suspension on the vertical degree with analogue results to those of the graded case. %MAYBE SPECIFY SOME OF THEM

%Everything should be valid for R-modules (char not 2, as in fields). The sign representation would have to be a free R-module of rank 1

 %for a commutative (at least with 1\neq 0) ring the rank is well defined, in general it is not

%Therefore, in analogy to the single graded cases, let $sig_n$ be the sign representation of the symmetric group on $n$ symbols concentrated in bidegree $(0,0)$. This is a free $R$-module of rank one that comes with a natural action of the symmetric group $S_n$ that multiplies each element by the sign of each given permutation. %I MIGHT LEAVE OUT THE SYMMETRIC GROUP  ACTION UNLESS I FIND OUT HOW TO MODIFY IT IN TOTALIZATION, I SHOULD THINK ABOUT IT

%We define $\Lambda(n)=S^{n-1}sign_n$, where  $S$ is a vertical shift of degree so that $\Lambda(n)$ is concentrated on bidegree  $(0,n-1)$.
We define $\Lambda(n)=S^{n-1}R$, where  $S$ is a vertical shift of degree so that $\Lambda(n)$ is the underlying ring $R$ concentrated in bidegree  $(0,n-1)$. As in the single graded case, we express the basis element of $\Lambda(n)$ as $e^n=e_1\land\cdots\land e_n$.

The operad structure on the bigraded $\Lambda=\{\Lambda(n)\}_{n\geq 0}$ is the same as in the graded case, namely

\[
\begin{tikzcd}
\Lambda(n)\otimes\Lambda(m) \arrow[r, "\circ_{r+1}"] & \Lambda(n+m-1)\\
(e_1\land\cdots\land e_n)\otimes(e_1\land\cdots\land e_m)\arrow[r, mapsto] & (-1)^{(n-r-1)(m-1)}e_1\land\cdots\land e_{n+m-1}.
\end{tikzcd}
\]



%In a similar way we can define $\Lambda^-(n)=S^{1-n}sig_n$, with the same insertion maps.
In a similar way we can define $\Lambda^-(n)=S^{1-n}R$, with the same insertion maps.
%The sign might arise naturally from the permutation action. If I have the wedge of n wedge the wedge of m-1 (because the final result must be n+m-1 in total), I would permute the last m-1 until the reach the i-th position via transpositions, each transpotision produces a minus sign. Or simply considering the lat m as a single element of degree m-1 being permuted in the wedge
\begin{defin}
Let $\mathcal{O}$ be a bigraded linear operad, i.e. an operad on the category of bigraded $R$-modules. The \emph{vertical operadic suspension} $\mathfrak{s}\OO$ of $\mathcal{O}$ is given arity-wise by the Hadamard product of the operads $\OO$ and $\Lambda$, in other words, $\mathfrak{s}\OO(n)=(\mathcal{O}\otimes\Lambda)(n)=\mathcal{O}(n)\otimes\Lambda(n)$ with diagonal composition.% and symmetric group action. 
Similarly, we define the \emph{vertical operadic desuspension} $\mathfrak{s}^{-1}\OO(n)=\mathcal{O}(n)\otimes\Lambda^-(n)$. %POSSIBLY EXCLUDE SYMMETRIC GROUP ACTION, ALTHOUGH IT MAKES SENSE ON ITS OWN
\end{defin}


We may identify the elements of $\mathcal{O}$ with the elements the elements of $\mathfrak{s}\OO$. 
\begin{defin}
For $x\in\OO(n)$ of bidegree $(k,d-k)$, its \emph{natural bidegree} in $\s\OO$ is $(k,d+n-k-1)$. To distinguish both degrees we call $(k,d-k)$ the \emph{internal bidegree} of $x$, since this is the degree that $x$ inherits from the grading of $\OO$. 
\end{defin}

If we write $\circ_{r+1}$ for the operadic insertion on $\OO$ and $\tilde{\circ}_{r+1}$ for the operadic insertion on $\mathfrak{s}\OO$, we may find a relation between the two insertion maps in the following way. 

\begin{lem}
For $x\in\OO(n)$ and $y\in\OO(m)^{q}_l$ we have

\begin{equation}\label{sign}
x\tilde{\circ}_{r+1}y=(-1)^{(n-1)q+(n-1)(m-1)+r(m-1)}x\circ_{r+1} y.
\end{equation}

\end{lem}

\begin{proof}
Let $x\in\OO(n)$ and $y\in\OO(m)^{q}_l$, and let us compute $x\tilde{\circ}_{r+1} y$.

\begin{align*}
\mathfrak{s}\OO(n)\otimes\mathfrak{s}\OO(m)&=(\OO(n)\otimes\Lambda(n))\otimes (\OO(m)\otimes\Lambda(m))\cong (\OO(n)\otimes \OO(m))\otimes (\Lambda(n)\otimes \Lambda(m))\\
&\xrightarrow{\circ_{r+1}\otimes\circ_{r+1}} \OO(m+n-1)\otimes \Lambda(n+m-1)=\mathfrak{s}\OO(n+m-1).
\end{align*}

The symmetric monoidal structure produces the sign $(-1)^{(n-1)q}$ in the isomorphism $\Lambda(n)\otimes \OO(m)\cong\OO(m)\otimes\Lambda(n)$, and the operadic structure of $\Lambda$ produces the sign \[(-1)^{(n-1)(m-1)+r(m-1)},\] so 

\[
x\tilde{\circ}_{r+1}y=(-1)^{(n-1)q+(n-1)(m-1)+r(m-1)}x\circ_{r+1} y.
\]

\end{proof}


\begin{remark}
As can be seen, this is the same sign as the single graded operadic suspension but with vertical degree. In particular, this operation leads to the Lie bracket from \cite{RW}, which implies that $m=\sum_{i,j}m_{ij}$ is a derived $A_\infty$-multiplication if and only if
\begin{equation}\label{sharp}
\sum_{i+j=u}\sum_{l,k}(-1)^jb_1(b_1(m_{jl};m_{ik});x)=0
\end{equation}
for the brace induced by $\tilde{\circ}$ (see \Cref{bibraces} to recall the definition of a brace).
\end{remark}

We of course have the following theorem with similar proof to the graded case, where all the suspensions are vertical.
\begin{thm}
Given a bigraded $R$-module $A$, there is an isomorphism of operads $\End_{ A}\cong \mathfrak{s}\End_{SA}$, where $\End_A$ is the endomorphism operad of $A$.\qed
\end{thm}

We also get the functorial properties that we studied for the single graded case in \Cref{functorial} and \Cref{monoidal}.

\subsection{Vertical suspension and totalization} 

Now we are going to combine vertical operadic suspension and totalization. More precisely, the totalized vertical suspension a bigraded operad $\OO$ is the graded operad $\Tot(\s\OO)$. 

%TsOO(n)^{n1} SOO(n)^{n1-k1}_k1  OO(n)^{n1-k1-n+1}_k1

This operad has an insertion map explicitly given by
\begin{equation}\label{star}
(x\star_{r+1} y)_k=\sum_{k_1+k_2=k}(-1)^{(n-1)(d_2-k_2-m+1)+(n-1)(m-1)+r(m-1)+k_1d_2}x_{k_1}\circ_{r+1}y_{k_2}
\end{equation}
for $x=(x_i)_i\in \Tot(\s\OO(n))^{d_1}$ and $x=(x_j)_j\in \Tot(\s\OO(m))^{d_2}$. As usual, denote \[x\star y=\sum_{r=0}^{m-1}x\star_{r+1}y.\]

This star operation is precisely the star operation from \cite[\S 5.1]{LRW}, i.e. the convolution operation on $\Hom((dAs)^{!}, \End_A)$ (see \cite{LRW} for details). In particular, we can recover the Lie bracket from in \cite{LRW}. We do this in \Cref{biliebracket}.

%THEIR BRACKET IS  EXACTLY THE ONE I WOULD DEFINEE, BECAUSE THEIR TOTAL DEGREE DOES NOT SUBTRACT 1 LIKE SUSPENSION DOES

%(WILL HAVE TO TALK ABOUT TOTALIZING COLLECTIONS)

Before continuing, let us show a lemma that allows us to work only with the single graded operadic suspension if needed.
\begin{propo}\label{extrasign}
For a bigraded operad $\OO$ we have an isomorphism $\Tot(\s\OO)\cong \s \Tot(\OO)$, where the suspension on the left hand side is the bigraded version and on the right hand side is the single graded version. 
\end{propo}
\begin{proof}
 Note that we may identify each element $x=(x_k\otimes e^n )_k\in\Tot(\s\OO(n))$ with the element $x=(x_k)_k\otimes e^n\in\s\Tot(\OO(n))$. Thus, for an element $(x_k)_k\in \Tot(\s\OO(n))$ the isomorphism is given by
\begin{align*}
f:\Tot(\s\OO(n))&\cong \s \Tot(\OO(n))\\
(x_k)_k&\mapsto ((-1)^{kn}x_k)_k
\end{align*}
Cleary this map is bijective so we just need to check that it commutes with insertions. Recall from \Cref{star} that the insertion on $\Tot(\s\OO)$ is given by
\begin{equation*}
(x\star_{r+1} y)_k=\sum_{k_1+k_2=k}(-1)^{(n-1)(d_2-k_2-n+1)+(n-1)(m-1)+r(m-1)+k_1d_2}x_{k_1}\circ_{r+1}y_{k_2}
\end{equation*}
for $x=(x_i)_i\in \Tot(\s\OO(n))^{d_1}$ and $y=(y_j)_j\in \Tot(\s\OO(m))^{d_2}$. Similarly, we may compute the insertion on $\s\Tot(\OO)$ by combining the sign produced first by $\Tot$ and then by $\s$. This results in  the following insertion map 
\begin{equation*}
(x\star_{r+1}' y)_k=\sum_{k_1+k_2=k}(-1)^{(n-1)(d_2-n+1)+(n-1)(m-1)+r(m-1)+k_1(d_2-m+1)}x_{k_1}\circ_{r+1}y_{k_2}
\end{equation*}
for $x=(x_i)_i\in \s\Tot(\OO(n))^{d_1}$ and $y=(y_j)_j\in \s\Tot(\OO(m))^{d_2}$. Now let us show that $f(x\star y)=f(x)\star f(y)$. We have
\begin{align*}
f((x\star_{r+1} y))_k&=\sum_{k_1+k_2=k}(-1)^{k(n+m-1)+(n-1)(d_2-k_2-n+1)+(n-1)(m-1)+r(m-1)+k_1d_2}x_{k_1}\circ_{r+1}y_{k_2}\\
&=\sum_{k_1+k_2=k}(-1)^{(n-1)(d_2-n+1)+(n-1)(m-1)+r(m-1)+k_1(d_2-m+1)}f(x_{k_1})\circ_{r+1}f(y_{k_2})\\
&=(f(x)\star f(y))_k
\end{align*}
%I skipped one calculation but it is just that
as desired.
\end{proof}


\begin{remark}\label{othermu}


As we mentioned in \Cref{heuristic}, there exist other possible ways of totalizing by varying the natural transformation $\mu$. For instance, we can choose the totalization functor $\Tot'$ which is the same as $\Tot$ but with a natural transformation $\mu'$ defined in such a way that the insertion on $\Tot'(\OO)$ is defined by \[(x\hat{\circ}y)_k=\sum_{k_1+k_2=k}(-1)^{k_2n_1}x_{k_1}\circ y_{k_2}.\] 

This is also a valid approach for our purposes and there is simply a sign difference, but we have chosen our convention to be consistent with other conventions, like the derived $A_\infty$-equation. However, let us mention some differences and relations between $\Tot$ and $\Tot'$. %There are of course many other possible conventions MAYBE NOT MENTION THE EXACT DIFFRENES, JUST STOP HERE BECAUSE THERE ARE MORE CONVENTIONS AND I DON'T THINK I WILL TREAT ALL OF THEM IN DETAIL, BUT THE DIFFERENCE BETWEEN IDENTITY AND ISOMORPHISM MIGHT BE WORTH MENTION
First of all, $\Tot$ and $\Tot'$ are isomorphic. The isomorphism $f:\Tot(\OO)\cong \Tot'\OO$ given by \[f((x_k))=((-1)^{k(d-k)}x_k).\]
for $x=(x_k)_k\in\Tot(\OO)^d$. %MAYBE ALL OF THEM ARE ISOMORPHIC, THINK OF THIS USING JUST THAT A TOTALIZATION IS A LAX SYMMETRIC MONOIDAL FUNCTOR FROM BIGRADED MODULES TO GRADED MODULES %, MAYBE ALSO USE THAT THE UNDERLYING MODULE IS ALWAYS THE SAME SO THAT THE ONLY MODIFICATION IS THE INSERTION

It can also be verified that $\Tot'(\s\OO)=\s \Tot'(\OO)$. With the original totalization we have a non identity isomorphism given by \Cref{extrasign}. Similar relations can be found among the other alternatives mentioned in \Cref{heuristic}. %We will use this isomorphism later.

%POSSIBLY FILL IN SOME DETAILS OF THESE ISOMORPHISMS


\end{remark}
%AT THE END  OF THE PROOF OF WHITEHOUSE 4.47 WHERE AINFTY ON TWISTED COMPLEX IS EQUIVALENT TO DERIVED-AINFTY THIS ISOMORPHISM IS  USED SINCE APLIED TO MIJ  THE EXPONENT IS PRECISELY IJ

%ALSO IN THAT THEOREM THE DIFERENTIALS OF THE TWISTED COMPLEX CORRESPOND PRECISELY TO MI1 SIMILAR TO MY MI1 BEING LIKE DIFFERENTIALS (I SHOULD CHECK IF THEY SATISFY THE TWISTED COMPLEX RELATION)

Using the operadic structure on $\Tot(\s\OO)$ we can describe derived $A_\infty$-multipilcations in a new conceptual way.

\begin{lem}\label{mstar}
A derived $A_\infty$-multiplication on a bigraded operad $\OO$ is equivalent to an element $m\in\Tot(\s\OO)$ of degree 1 concentrated in positive arity such that $m\star m = 0$. 
%$m=\sum_{ij}m_{ij}$ where $m_{ij}\in\OO(j)^{2-i-j}_i$ for each $j\geq 1$ such that 
%\[\underset{j=r+1+t}{\sum_{u=i+p, v=j+q-1}}(-1)^{rq+t+pj}m_{ij}\circ_{r+1} %m_{pq}=0.\]
\end{lem}
\begin{proof}
A derived $A_\infty$-multiplication on $\OO$ is by \Cref{derivedmultiplication} a map 
\[f:d\calA_\infty\to\OO.\]
Since $\calA_\infty$ is generated by elements $\mu_{ij}$ of bidegree $(i,2-i-j)$, such a map is determined by the elements $m_{ij}=f(\mu_{ij})\in\OO^{2-i-j}_i(j)$. Consider $m_j = (m_{ij})_i\in\Tot(\s\OO(j))$. We have that $\deg(m_j)=1$ for all $j$. Therefore, let $m=m_1+m_2+\cdots\in\Tot(\s\OO)$. We may check that $m\star m=0$. For that we just need to check \Cref{star}. On arity $n$, this amounts to compute 
\[(m\star m)_k = \sum_{r=0}^{n-1}\underset{j+q=n-1}{\sum_{i+p=k}}(-1)^{rp+j-r-1+ pj}m_{ij}\circ_{r+1}m_{pq}=0.\]
The above expression vanishes precisely because the elements $m_{ij}$ satisfy the derived $A_\infty$-equation.

Conversely, let $m\in\Tot(\s\OO)$ of degree 1, is concentrated in positive arity and satisfying $m\star m=0$. We can split $m$ into its arity and horizontal degree components as $m=\sum_{i,j}m_{ij}$. As we have seen, the fact that $m\star m=0$ is equivalent to the elements $m_{ij}$ satisfying the derived $A_\infty$-equation, and therefore, a map $f:d\calA_\infty\to\OO$ is determined by $f(\mu_{ij})=m_{ij}$, which are of bidegree $(i,2-i-j)$. 
\end{proof}


From \Cref{mstar}, we can proceed as in the proof of \Cref{ainftystructure}, to show that $m$ determines an $A_\infty$-algebra structure on $S\Tot(\s\OO)\cong S\s \Tot(\OO)$. %IT WAS EQUALITY WITH THE ALTERNATIVE OPERATION 
%that can be iterated to define an $A_\infty$-structure on $S\s\End_{S\s \Tot(\OO)}$.%THIS PART ONLY IF I CAN DO THE SAME IN THE DERIVED CASE, which is related to the structure on $S\s \Tot(\OO)$ by a map of $A_\infty$-algebras  $\Phi:S\s \Tot(\OO)\to  S\s\End_{S\s \Tot(\OO)}$.  MAYBE WRITE THIS MORE EXPLICITLY 

The goal now is showing that this $A_\infty$-structure on $S\Tot(\s\OO)$ is equivalent to a derived $A_\infty$-structure on $S\s \OO$ and compute the structure maps explicitly. We will do this in \Cref{derivedstructure}. 

Before that, let us use this new operadic structure to reinterpret derived $\infty
$-morphisms and their composition.
%HERE THE STAR OPERATION, I NEED TO DEFINE DERIVED AINFTY MULTIPLICATION ON AN OPERAD, THEN STAR OPERATION, THEN ELEMENT OF TOTALIZED VERTICAL SUSPENSION
\subsection{Reinterpretation of derived $\infty$-morphisms}

Just like we did for graded modules on \Cref{reinterpretation}, for bigraded modules $A$ and $B$ we may define the collection $\End^A_B=\{\Hom(A^{\otimes n}, B)\}_{n\geq 1}$ of bigraded modules,  in analogy with the single graded case. Recall that this collection has a left module structure over $\End_B$
\[\End_B\circ \End^A_B\to \End^A_B\]
given by composition of maps. Similarly, given a bigraded module $C$ we can define composition maps
\[\End^B_C\circ \End^A_B\to \End^A_C.\]
The collection $\End^A_B$ also has an infinitesimal right module structure over $\End_A$
\[\End^A_B\circ_{(1)}\End_A\to \End^A_B\]
given by insertion of maps.

Similarly to the single graded case, we may describe derived $\infty$-morphisms in terms of the above operations.

\begin{lem}\label{dinfinitymorphism}
A derived $\infty$-morphism of $A_\infty$-algebras $A\to B$ with respective structure maps $m^A$ and $m^B$ is equivalent to an element $f\in\Tot(\s\End^A_B)$ of degree 0 concentrated in positive arity such that \[\rho(f\circ_{(1)}m^A)=\lambda(m^B\circ f),\] 

where \[\lambda:\Tot(\mathfrak{s}\End_B)\circ \Tot(\mathfrak{s}\End^A_B)\to \Tot(\mathfrak{s}\End^A_B)\] is induced by the left module struncture on $\End^A_B$ and \[\rho:\Tot(\mathfrak{s}\End_B)\circ_{(1)}\Tot(\mathfrak{s}\End^A_B)\to \Tot(\mathfrak{s}\End^A_B)\] is induced by the right infinitesimal module structure on $\End^A_B$. 

In addition, the composition of $\infty$-morphisms is given by the natural composition \[\Tot(\s\End^B_C)\circ \Tot(\s\End^A_B)\to \Tot(\s\End^A_C).\]
\end{lem}
\begin{proof}
Since $f_j=(f_{ij})_i\in\Tot(\s\End^A_B(j))$ is of degree $0$, we have that that $f_{ij}$ is of bidegree $(i,1-i-j)$. Thus, the equation

\[\rho(f\circ_{(1)}m^A)=\lambda(m^B\circ f),\] 

coincides with the one defining derived $\infty$-morphisms of derived $A_\infty$-algebras (\Cref{dinftymaps}) up to sign. The signs that appear in the above equation are obtained in a similar way to that on the brace $b_j^T$ (see \Cref{bracetot}). Thus, it is enough to plug in the sign provided by \Cref{bracetot} the corresponding degrees and arities to obtain the desired result. The composition of derived $\infty$-morphisms follows similarly.
\end{proof}

In the case that $f:A\to A$ is an $\infty$-endomorphism, the above definition can be written in terms of operadic composition as $f\star m=\gamma^\star(m\circ f)$, where $\gamma^*$ is the composition map with derived from the $\star$ operation (see \Cref{gammastar}). Here, $\circ$ is the plethysm of maps of collections, not to be confused with composition of maps. 


\section{Bigraded braces and totalized braces}\label{sectionbibraces}
We are going to define a brace structure using totalization. First we introduce the definition of braces for the bigraded case, which is analogue to the single graded version we defined in \Cref{braces}.

\begin{defin}\label{bibraces}
A \emph{brace algebra} on a bigraded module $A$ consists of a family of maps \[b_n:A^{\otimes 1+n}\to A\] called \emph{braces}, that we evaluate on $(x,x_1,\dots, x_n)$ as $b_n(x;x_1,\dots, x_n)$, satisfying the \emph{brace relation}


\begin{align*}
b_m(b_n(x;x_1,\dots, x_n);y_1,\dots,y_m)=&\\
\sum_{i_1,\dots, i_n, j_1\dots, j_n}(-1)^{\varepsilon}&b_l(x; y_1,\dots, y_{i_1},b_{j_1}(x_1;y_{i_1+1},\dots, y_{i_1+j_1}),\dots, b_{j_n}(x_n;y_{i_n+1},\dots, y_{i_n+j_n}),\dots,y_m)
\end{align*}
where $l=n+\sum_{p=1}^n i_p$ and $\varepsilon=\sum_{p=1}^n\sum_{q=i}^{i_p}\langle x_p,y_q\rangle$.



\end{defin}

%\begin{remark}
%Some authors might use the notation $b_{1+n}$ instead of $b_n$, but the first element is usually going to have a different role than the others. A shorter notation for $b_n(x;x_1,\dots,x_n)$ found in the literature is $x\{x_1,\dots, x_n\}$. Also note that we have used the notation $|x_p|$ for the degree of $x_p$ in $A$. 
%\end{remark}

As one might expect, we can define bigraded brace maps $b_n$ on a bigraded operad $\OO$ and also on its operadic suspension $\s\OO$, obtaining similar signs as in the single graded case, but with vertical (internal) degrees (see \Cref{bracesign}). 

We can also define braces on $\Tot(\s\OO)$ via operadic composition. In this case, these are usual single graded braces. More precisely, we define the maps 
\[b^T_n:\Tot(\mathfrak{s}\OO(N))\otimes \Tot(\mathfrak{s}\OO(a_1))\otimes\cdots\otimes \Tot(\mathfrak{s}\OO(a_n))\to \Tot(\mathfrak{s}\OO(N-\sum a_i))\]
using the operadic composition $\gamma^\star$ on $\Tot(\mathfrak{s}\OO)$ as

\[b^T_n(x;x_1,\dots,x_n)=\sum\gamma^\star(x;1,\dots,1,x_1,1,\dots,1,x_n,1,\dots,1),\]

where the sum runs over all possible ordering preserving insertions. The brace $b^T_n(x;x_1,\dots,x_n)$ vanishes whenever $n>N$ and $b^T_0(x)=x$. We use the notation $b^T_n$ to distinguish this brace map from the bigraded brace $b_n$ that can be naturally defined on the bigraded operad $\s\OO$.

Operadic composition can be described in terms of insertions in the obvious way, namely 

\begin{equation}\label{gammastar}
\gamma^\star(x;y_1,\dots,y_N)=(\cdots(x\star_1 y_1)\star_{1+a(y_1)}y_2\cdots)\star_{1+\sum a(y_p)}y_N,
\end{equation}

where $a(y_p)$ is the arity of $y_p$ (in this case $y_p$ is either $1$ or some $x_i$). If we want to express this composition in terms of the composition in $\OO$ we just have to find out the factor sign applying the same strategy as in the single-graded case. In fact, as we said, there is a sign factor that comes from vertical operadic suspension that is identical to the graded case, but replacing internal degree by internal vertical degree. This is the sign that determines the brace $b_n$ on $\s\OO$. Explicitly, it is given by the following lemma, whose proof is identical to the single graded case.


 
 \begin{lem}\label{bigradedsign}
For $x\in \s\OO(N)$ and $x_i\in\s\OO(a_i)$ of internal vertical degree $q_i$ ($1\leq i\leq n$), we have
\[b_n(x;x_1,\dots,x_n)=\sum_{N-n=h_0+\cdots+h_n} (-1)^\eta \gamma
(x\otimes 1^{\otimes h_0}\otimes x_1\otimes \cdots\otimes x_n\otimes1^{\otimes h_n}),\]
where 
\[\eta=\sum_{0\leq j<l\leq n}h_jq_l+\sum_{1\leq j<l\leq n}a_jq_l+\sum_{j=1}^n (a_j+q_j-1)(n-j)+\sum_{1\leq j\leq l\leq n} (a_j+q_j-1)h_l.\]
\end{lem}

The other sign factor is produced by totalization. This was computed in \Cref{totcomp}. Combining both factors we obtain the following.

\begin{lem}
We have 
\begin{equation}\label{bracetot}
b_j^T(x;x^1,\dots, x^N)_k=\underset{h_0+h_1+\cdots+h_N=j-N}{\sum_{k_0+k_1+\cdots+k_N=k}}(-1)^{\eta+\sum_{j=1}^m d_j\sum_{i=0}^{j-1}k_i}\gamma(x_{k_0};1^{h_0},x^1_{k_1},1^{h_1},\dots, x^N_{k_N},1^{h_N})
\end{equation}
for $x=(x_k)_k\in\Tot(\s\OO(N))^{d_0}$ and $x^i=(x^i_k)_k\in\Tot(\s\OO(a_i))^{d_i}$, where $\eta$ is defined in \Cref{bigradedsign}. 
\end{lem}
%Therefore we only need to compute the sign factor corresponding to totalization. Given $f_{p_0}\in \OO(N)^{q_0}_{p_0}$ and  $g_i\in\OO(a_i)^{q_i}_{p_i}$ for $1\leq i\leq n$, %such that $p_0+p_1+\cdots+p_n=k$, 
%the factor sign we are looking for is determined by the exponent
%
%%\[\varepsilon_k=p_1(N+q_0+p_0-1)+p_2(N+a_1+q_0+q_1+p_0+p_1-2)+\cdots=\sum_{i=1}^np_i(\sum_{j=0}^{i-1}(a_j+q_j+p_j)+N-i).\]
%
%\[\varepsilon
%=p_0(a_1+q_1+p_1-1)+(p_0+p_1)(a_2+q_2+p_2-1)+\cdots=\sum_{i=1}^n(a_i+q_i+a_i-1)\sum_{j=0}^{i-1}p_j\]
%\begin{equation}\label{epsilon}
%=\sum_{0\leq j<i\leq n}p_j(a_i+q_i+p_i-1).
%\end{equation}
%This is obtained by iteration of the last factor sign in \Cref{star} which is precisely the sign determined by totalization. Note that $a_i+q_i+p_i-1$ is precisely the total degree of $g_i$ in $\Tot(\s\OO)$. Therefore, if $(-1)^\eta$ is the sign produced by operadic suspension and $(-1)^{\varepsilon}$ the sign produced by totalization, the factor sign that distinguishes the brace on $\Tot(\s\OO)$ from the usual operadic composition on $\OO$ is $(-1)^{\eta+\varepsilon}$ %THE K IS BECAUSE I'M PLANNING TO REWRITE THINGS AS SUMS IN TOTALIZATION, I MIGHT ABOUT THIS, I WILL HAVE TO CHANGE THE NOTATION FOR THE MAPS BEING COMPOSED AND THEIR DEGREES


% USING BRACES TO DEFINE A DERIVED AINFTY ALGEBRA ON THE OPERAD (I WILL PROBABLY  WRITE THE COMPUTATION OF DEGREES DIFFERENTLY LATER BECAUSEE I AM IDENTIFYING THE BRACE AND ELEMENTS WITH THEIR SHIFTS HERE TO MAKE IT EASIER TO COMPPUTE)
%
%Let $m_{il}$ a component of the derived $A_\infty$-multiplication $m$ and $x_1,\dots, x_j\in\OO$ with $x_k$ of bidegree $(i_k,l_k)$, and let us compute the bidegree of $b_j(m_{il};x_1,\dots, x_j)$ in $S\s\OO$. By definition, the horizontal degree on $S\s\OO$ is the same as in $\OO$, so it is $i+\sum_{k=1}^ji_k$, meaning that the horizontal degree of the map $b_j(m_{il};-):(S\s\OO)^{\otimes j}\to S\s\OO$ is precisely $i$. Now let us compute the vertical degree. By definition it is given by the internal vertical degree plus the arity. Therefore, let us compute
%
%\[
%a(b_j(m_{il};x_1,\dots, x_j))+\deg(b_j(m_{il};x_1,\dots, x_j))
%\]
%where
%
%\[
%a(b_j(m_{il};x_1,\dots, x_j))=l-j+\sum a(x_k)
%\]
%and 
%\[\deg(b_j(m_{il};x_1,\dots, x_j))=2-l-i+\sum \deg(x_k)\]
%so the sum is $2-i-j+\sum (a(x_k)+\deg(x_k))$, so that the vertical degree of the map $b_j(m_{il};-)$ on the shift is $2-i-j$. This means, that we have the following candidates for a derived $A_\infty$-structture
%THESE ARE CANDIDATES BEFORE SHIFTING  AND I ALSO NEED TO TOTALIZE
%
%\[M_{ij}(x_1,\dots, x_j)=\sum_l b_j(m_{il};x_1,\dots, x_j)\]
%\[M_{i1}(x)= \sum_l (b_1(m_{il};x)-(-1)^{\langle x,m_{il}\rangle}b_1(x;m_{il}))\]
%to be a derived structure on that shift. THE SSIGN SHOULD BE THE SCALAR PRODUCT OF THE BIDEGREES $\langle x,m_{il}\rangle=x_hi+x_v(1-i)$ (degrees in $\s\OO$)
%
%CHECK THAT THEY SATISFYI THE EQUATION UP TO SIGN AND WORRY LATER ABOUT THE SIGNS
%
%DO THE MI1 DEFINE A TWISTED COMPLEX? THIS IS A NECESSARY CONDITION TO SATISSFY THE DAINFTY EQUATION THAT MIGHT BE EASIER TO SHOW AND IIT WOULD MAKE EASIER TO CONNECT EVERYTHING HERE WITH 
\begin{corollary}\label{biliebracket}
 For $\OO = \End_A$, the endomorphism operad of a bigraded module, the brace $b_1^T(f;g)$ is the operation $f\star g$ defined in \cite{LRW} that induces a Lie bracket. More precisely,
\[
[f,g]=b_1(f;g)-(-1)^{NM}b_1(g;f)
\]
for $f\in\Tot(\s\End_A)^N$ and $g\in\Tot(\s\End_A)^M$, is the same bracket that was defined in \cite{LRW}. 
\end{corollary}

Notice that in \cite{LRW} the sign in the bracket is $(-1)^{(N+1)(M+1)}$, but this is because their total degree differs by 1 with respect to ours.

\section{Derived $A_\infty$-structure on an operad}\label{derivedstructure}


We are going to use the following theorem from \cite{whitehouse} to show that there is a derived $A_\infty$-structure on $A=S\s\OO$. Note that $\Tot(SB)=S\Tot(B)$ for any bigraded module $B$, where $SB$ is the vertical suspension of $B$ and $S\Tot(B)$ is the suspension of $\Tot(B)$ as graded modules.

\begin{thm}\label{whitehouse}
Let $(A, d^A) ∈ \tc^b$ be an twisted complex horizontally bounded on the right and $A$ its underlying
chain complex. We have natural bijections %this means that A has d0 as a differential and End_A has [d0,-]
\begin{align*}
\Hom_{\mathrm{vbOp},d^A}(d\calA_∞,\End_A) &\cong
\Hom_{\mathrm{vbOp}}(\calA_∞, \uEnd_A)\\
&\cong \Hom_{\mathrm{vbOp}}(\calA_∞, \uEnd_{\Tot(A)})\\
&\cong \Hom_{\mathrm{fCOp}}(\calA_∞,\underline{\End}_{\Tot(A)}),
\end{align*}
where $\vbOp$ and $\fCOp$ denote the categories of operads in $\vbc$ and $\fc$ respectively, and $\Hom_{\vbOp,d^A}$
denotes the subset of morphisms which send $μ_{i1}$ to $d^A_i$. We view $\mathcal{A}_∞$ as an operad in $\vbc$ sitting in
horizontal degree zero or as an operad in filtered complexes with trivial filtration.
\end{thm}
\begin{proof}
See \cite[Poposition 4.55]{whitehouse}.
\end{proof}

\begin{remark}
According to \Cref{filterversion}, the last isomorphism can be replaced by 
\[\Hom_{\mathrm{vbOp}}(\calA_∞, \uEnd_{\Tot(A)})\cong \Hom_{\mathrm{COp}}(\calA_∞,\End_{\Tot(A)}),\]
where $\mathrm{COp}$ is the category of operads in chain complexes. 
\end{remark}
There are several important assumptions to make in order to use the theorem. First of all, we need $A$ to be horizontally bounded on the right, meaning that there exists some integer $i$ such that $A_k^{d-k}=0$ for all $k>i$. In our case $A=S\s\OO$ for $\OO$ an operad with a derived $A_\infty$-multiplication, so being horizontally bounded on the right implies that for each $j>0$ we can only have finitely many non-zero $dA_\infty$ components $m_{ij}$. This situation happens in practice in all known examples of derived $A_\infty$-algebras so far (some of them are in \cite[Remark 6.5]{muro}, \cite{RW}, and \cite[\S 5]{women}). Under this assumption we may replace all direct products by direct sums.

We also need to provide $A$ with a twisted complex structure. The reason for this is that the theorem uses the definition of derived $A_\infty$-algebras on an underlying twisted complex (See \Cref{equivalent}). We do this explicitly in \Cref{twistedoperad}, but it also follows from \Cref{mi1}. We also provide another version of this theorem that works for bigraded modules, \Cref{alternative}. 

With these assumption, by \Cref{whitehouse} we can guarantee the existence of a derived $A_\infty$-algebra structure on $A$ provided that $\Tot(A)$ has an $A_\infty$-algebra structure.




\begin{thm}\label{derivedmaps}
Let $A=S\s\OO$ where $\OO$ is an operad horizontally bounded on the right with a derived $A_\infty$-multiplication $m=\sum_{ij}m_{ij}\in\OO$. Let $x_1\otimes\cdots\otimes x_j\in (A^{\otimes j})^{d-k}_k$ and let $x_v = Sy_v$ for $v=1,\dots, j$ and $y_v$ be of bidegree $(k_v,d_v-k_v)$. The following maps $M_{ij}$ for $j\geq 2$ determine a derived $A_\infty$-algebra structure on $A$.
%\[M_{ij}(x_1,\dots,x_j)= (-1)^{\sum_{v=2}^j n_v\sum_{w=1}^{v-1}k_w+\sum_{v=1}^j(j-v)(n_v-k_v)+\sum_{1\leq v<w\leq j}k_v(a_w+n_w-1)}\sum_lSb_j(m_{il};y_1,\dots, y_j) \]

%THINGS CANCEL, INCLUDE CALCULATION (THE SIGN FROM MU CANCELS THE SIGN FROM EPSILON BECAUSE IT ALSO COMES FROM MU, JUST LEAVING THE PART CORRESPONDING TO M-IL). SINCE THE ELEMNT IS FROM SO ITS VERTICAL DEGREE INCLUDES ARITY -1
\[M_{ij}(x_1,\dots,x_j)= (-1)^{\sum_{v=1}^j(j-v)(d_v-k_v)}\sum_lSb_j(m_{il};y_1,\dots, y_j). \]
\end{thm}
Note that we abuse of notation and identify $x_1\otimes\cdots\otimes x_j$ with an element of $\Tot(A^{\otimes j})$ with only one non-zero component. For a general element, extend linearly.

\begin{proof}
Since $m$ is a derived $A_\infty$-multiplication $\OO$, we have that $m\star m=0$ when we view $m$ as an element of $\Tot(\s\OO)$. By \Cref{ainftystructure} this defines an $A_\infty$-algebra structure on $S\Tot(\s\OO)$ given by maps
 %This isomorphism introduces some signs that will cancel with the sigs introduced by the isomorphism FIRST ISO OF THE CHAIN so we will omit them. The $A_\infty$-structure is therefore given by maps THINK WHERE I HAVE TO PUT SUSPENSION
\[M_j:(S\Tot(\s\OO))^{\otimes j}\to S\Tot(\s\OO)\]
induced by shifting brace maps
\[b_j^T(m;-):(\Tot(\s\OO))^{\otimes j}\to \Tot(\s\OO).\]
 The graded module $S\Tot(\s\OO)$ is endowed with the structure of a filtered complex with differential $M_1$ and filtration induced by the column filtration on $\Tot(\s\OO)$. Note that $b^T_j(m;-)$ preserves the column filtration since every component $b^T_j(m_{ij};-)$ has positive horizontal degree. % since the shift is only vertical. MORE DETAIL ON THIS? (LIKE  EXPLIIT FILTRATION AND MAYBE CHECK THAT THE DIFFERENTIAL PRESERVES FILTRATION WHICH FOLLOWS FROM THE HORIZONTAL DEGREES OF THE MIJ BEING POSITTIVE)
 
Since $S\Tot(\s\OO)\cong \Tot(S\s\OO)$, we can view $M_j$ as the image of a morphism of operads of filtered complexes $f:\mathcal{A}_\infty\to \End_{\Tot(S\s\OO)}$, in such a way that $M_j=f(\mu_j)$ for $\mu_j\in\mathcal{A}_\infty(j)$. 

We now work our way backwards through the proof of \Cref{whitehouse}. The isomorphism 
\[\Hom_{\mathrm{vbOp}}(\calA_∞, \uEnd_{\Tot(A)})\cong \Hom_{\mathrm{COp}}(\calA_∞,\End_{\Tot(A)})\]
does not modify the map $M_j$ at all, but allows us to see it as a element of $\uEnd_{\Tot(A)}$ of bidegree $(0,2-j)$. 



The isomorphism 
\[\Hom_{\mathrm{vbOp}}(\calA_∞, \uEnd_A)\cong \Hom_{\mathrm{vbOp}}(\calA_∞, \uEnd_{\Tot(A)})\] 
in the direction we are following is the result of applying $\Hom_{\vbOp}(\calA_\infty,-)$ to the map described after \Cref{inverse}. Under this isomorphism, $f$ is sent to the map \[\mu_j\mapsto \mathfrak{Tot}^{-1}\circ c(M_j,\mu^{-1})=\mathfrak{Tot}^{-1}\circ M_j\circ \mu^{-1},\] where $c$ is the composition in $\ufC$. The functor $\mathfrak{Tot}^{-1}$ decomposes $M_j$ into a sum of maps $M_j=\sum_i \widetilde{M}_{ij}$, where each $\widetilde{M}_{ij}$ is of bidegree $(i,2-j-i)$. More explicitly, let $A=S\s\OO$ and let $x_1\otimes\cdots\otimes x_j\in (A^{\otimes j})^{d-k}_k$. We abuse of notation and identify $x_1\otimes\cdots\otimes x_j$ with an element of $\Tot(A^{\otimes j})$ with only one non-zero component. For a general element, extend linearly. Then we have

%I AM ABUSING NOTATION, THE ELEMENTS SHOULD BE IN TOT(A x ... x A) SO I AM REFERRING TO STRINGS WITH ONLY ONE NON-ZERO COMPONENT
\begin{align}\label{totsign}
\mathfrak{Tot}^{-1}(M_j( \mu^{-1}(x_1\otimes\cdots\otimes x_j)))&=\mathfrak{Tot}^{-1}(Sb_j^T(m;(S^{-1})^{\otimes j}(\mu^{-1}(x_1\otimes\cdots\otimes x_j))))\nonumber\\
&=\sum_i(-1)^{id}\sum_l Sb_j^T(m_{il};(S^{-1})^{\otimes j}(\mu^{-1}(x_1\otimes\cdots\otimes x_j)))\nonumber\\
&=\sum_i(-1)^{id}\sum_l(-1)^{\varepsilon} Sb_j(m_{il};(S^{-1})^{\otimes j}(\mu^{-1}(x_1\otimes\cdots\otimes x_j)))\nonumber\\
&=\sum_i\sum_l(-1)^{id+\varepsilon} Sb_j(m_{il};(S^{-1})^{\otimes j}(\mu^{-1}(x_1\otimes\cdots\otimes x_j)))
\end{align}
so that \[\widetilde{M}_{ij}(x_1,\dots,x_j)=\sum_l(-1)^{id+\varepsilon} Sb_j(m_{il};(S^{-1})^{\otimes j}(\mu^{-1}(x_1\otimes\cdots\otimes x_j))),\] where $b_j$ is the brace on $\s\OO$ and $\varepsilon$ is given in \Cref{totcomp}. 


According to the isomorphism 
\begin{equation}\label{firstiso}
\Hom_{\mathrm{vbOp},d^A}(d\calA_∞,\End_A)\cong
\Hom_{\mathrm{vbOp}}(\calA_∞, \uEnd_A),
\end{equation}
 the maps $M_{ij}=(-1)^{ij}\widetilde{M}_{ij}$ define an $A_\infty$-structure on $S\s\OO$. Therefore we now just have to work out the signs. Notice that $d_v$ is the total degree of $y_v$ as an element of $\s\OO$ and recall that $d$ is the total degree of $x_1\otimes\cdots\otimes x_j\in A^{\otimes j}$. Therefore, $\varepsilon$ can be written as
\[\varepsilon= i(d-j)+\sum_{1\leq v<w\leq j}k_vd_w.\]
The sign produced by $\mu^{-1}$, as we saw in \Cref{mui}, is precisely determined by the exponent 
\[\sum_{w=2}^jd_w\sum_{v=1}^{w-1}k_v=\sum_{1\leq v<w\leq j}k_vd_w,\]so this sign cancels the right hand summand of $\varepsilon$. This cancellation was expected since this sign comes from $\mu^{-1}$ and operadic composition is defined using $\mu$ (\Cref{insertion}). %In fact, both signs come from $\mu$, so the cancellation was expected. 
Finally, the sign $(-1)^{i(d-j)}$ left from $\varepsilon$ cancels with $(-1)^{id}$ in \Cref{totsign} and $(-1)^{ij}$ from the isomorphism (\ref{firstiso}). This means that we only need to consider signs produced by vertical shifts. This calculation has been done previously in \Cref{explicit} and as we claimed the result is 
\[M_{ij}(x_1,\dots,x_j)= (-1)^{\sum_{v=1}^j(j-v)(d_v-k_v)}\sum_lSb_j(m_{il};y_1,\dots, y_j). \]

\end{proof}

\begin{remark}\label{equivalent}
Note that as in the case of $A_∞$-algebras in $\mathrm{C}_R$  
we have two equivalent descriptions of $A_∞$-algebras in $\tc$.
%MAKE SURE  I HAVE STATED THAT FIRST EQUIVALENCE SOMEWHERE (IN THE OTHER ARTICLE FOR INSTANCE)
\begin{enumerate}[(1)]
\item A twisted complex $(A, d^A)$ together with a morphism $\calA_∞ → \uEnd_A$ of operads in $\vbc$, which is determined by a family of elements $M_i ∈ \utC(A^{⊗i},A)^{2−i}_0$ for $i ≥ 2$ for which the $A_\infty$-relations (\Cref{ainftyequation}) holds for $i\geq 2$, where the composition is the one prescribed by the composition morphisms of $\utC$.
\item A bigraded module $A$ together with a family of elements $M_i ∈ \ubgMod(A^{⊗i},A)^{2−i}_0$ for $i ≥ 1$ for
which all the $A_\infty$ relations hold (\Cref{ainftyequation}), where the composition is the one prescribed by the composition
morphisms of $\ubgMod$.
\end{enumerate}
Since the composition morphism
in $\ubgMod$ is induced from the one in $\utC$ by forgetting the differential, these two presentations
are equivalent.
\end{remark}

This equivalence allows us to formulate the following alternative version of \Cref{whitehouse}.
\begin{corollary}\label{alternative}
Given a bigraded module $A$ horizontally bounded on the right we have isomorphisms
\begin{align*}
\Hom_{\mathrm{bgOp}}(d\calA_∞,\End_A) &\cong
\Hom_{\mathrm{bgOp}}(\calA_∞, \uEnd_A)\\
&\cong \Hom_{\mathrm{bgOp}}(\calA_∞, \uEnd_{\Tot(A)})\\
&\cong \Hom_{\mathrm{fOp}}(\calA_∞,\underline{\End}_{\Tot(A)}),
\end{align*}
where $\mathrm{bgOp}$ is the category of operads of bigraded modules and $\mathrm{fOp}$ is the category of operads of filtered modules. %REFERENCE TO WHAT DEFINITIONS OF THE OPERADS I AM USING
\end{corollary}
\begin{proof}
Let us look at the first isomorphism

\[\Hom_{\mathrm{bgOp}}(\calA_∞, \uEnd_A)\cong \Hom_{\mathrm{bgOp}}(d\calA_∞,\End_A).\]

Let $f:\calA_\infty\to\uEnd_A$ be a map of operads in $\mathrm{bgOp}$. This is equivalent to maps in $\mathrm{bgOp}$
\[\calA_\infty(j)\to\uEnd_A(j)\]
for each $j\geq 1$, which are determined by elements $M_j\coloneqq f(\mu_j)\in\uEnd_A(j)$ for $v\geq 1$ of bidegree $(0,2-j)$ satisfying the $A_\infty$-equation with respect to the composition in $\ubgMod$. Moreover, $M_j\coloneqq (\tilde{m}_{0j},\tilde{m}_{1j},\dots)$ where $\tilde{m}_{ij}\coloneqq (M_j)_i:A^{\otimes n}\to A$ is a map of bidegree $(i,2-i-j)$. Since the composition in $\ubgMod$ is the same as in $\utC$, the computation of the $A_\infty$-equation becomes analogous to the computation done in \cite[Prop 4.47]{whitehouse}, showing that the maps $m_{ij}=(-1)^i\tilde{m}_{ij}$ for $i\geq 0$ and $j\geq 0$ define a derived $A_\infty$-algebra structure on $A$.

The second isomorphism
\[\Hom_{\mathrm{bgOp}}(\calA_∞, \uEnd_A)\cong \Hom_{\mathrm{bgOp}}(\calA_∞, \uEnd_{\Tot(A)})\]
follows from the bigraded module case of \Cref{inverse}. Finally, the isomorphism
\[\Hom_{\mathrm{bgOp}}(\calA_∞, \uEnd_{\Tot(A)})\cong \Hom_{\mathrm{fOp}}(\calA_∞,\underline{\End}_{\Tot(A)})\]
is analogous to the last isomorphism of \Cref{whitehouse}, replacing the quasi-free relation by the full $A_\infty$-algebra relations. 
\end{proof}

\begin{corollary}\label{mi1}
Let $\OO$ be a bigraded operad with a derived $A_\infty$-multiplication and let \[M_{i1}:S\s\OO\to S\s\OO\] be the arity 1 derived $A_\infty$-algebra maps induced by \Cref{alternative} from \[M_1:\Tot(S\s\OO)\to \Tot(S\s\OO).\]
Then \[M_{i1}(x)= \sum_l (Sb_1(m_{il};S^{-1}x)-(-1)^{\langle x,m_{il}\rangle}Sb_1(S^{-1}x;m_{il})),\]
where $x\in (S\s\OO)^{d-k}_k$ and $\langle x,m_{il}\rangle=ik+(1-i)(d-1-k)$.
\end{corollary}
\begin{proof}
Notice that the proof of \Cref{alternative} is essentially the same as the proof \Cref{whitehouse}. This means that the proof of this result is an arity 1 restriction of the proof of \Cref{derivedmaps}. Then we apply \Cref{totsign} to the case $j=1$. Recall that for $x\in (S\s\OO)^{d-k}_{k}$
\[M_1(x)=b_1^T(m;S^{-1}x)-(-1)^{n-1}b_1^T(S^{-1}x;m).\]
 In this case, there is no $\mu$ involved. Therefore, introducing here also the final extra sign $(-1)^i$ from the proof of \Cref{derivedmaps}, we get from \Cref{totsign} that
\[\widetilde{M}_{i1}(x)=(-1)^i\sum_l((-1)^{id+i(d-1)} Sb_1(m_{il};S^{-1}x)-(-1)^{d-1+id+k}Sb_1(S^{-1}x;m_{il})),\] where $b_1$ is the brace on $\s\OO$. Simplifying signs we obtain
\[\widetilde{M}_{i1}(x)=\sum_l Sb_1(m_{il};S^{-1}x)-(-1)^{\langle  m_{il},x\rangle}Sb_1(m_{il};S^{-1}x))=M_{i1}(x),\]

where $\langle  m_{il},x\rangle=ik+(1-i)(d-1-k)$, as claimed.
\end{proof}

\subsection{Iterating the process}

Note that the maps given by \Cref{derivedmaps} and \Cref{mi1} formally look the same as their single graded analogues (see \Cref{explicit}) but with an extra index that is fixed for each $M_{ij}$. This means that we can follow the same procedure as in \Cref{sect3} to define higher derived $A_\infty$-maps. 

More precisely, let $\overline{B}_j$ be the bigraded brace map on $\s\End_{S\s\OO}$ and consider the maps 
\begin{equation}\label{barbimaps}
\overline{M}'_{ij}:(\s\End_{S\s\OO})^{\otimes j}\to \s\End_{S\s\OO}
\end{equation}
defined as 
\begin{align*}
&\overline{M}'_{ij}(f_1,\dots,f_j)=\overline{B}_j(M_{i\bullet};f_1,\dots, f_j) & j>1,\\
&\overline{M}'_{i1}(f)=\overline{B}_1(M_{i\bullet};f)-(-1)^{ip+(1-i)q}\overline{B}_1(f;M_{i\bullet}),
\end{align*}
for $f$ of natural bidegree $(p,q)$, where $M_{i\bullet}=\sum_j M_{ij}$. We define 
\[\overline{M}_{ij}:(S\s\End_{S\s\OO})^{\otimes j}\to S\s\End_{S\s\OO}\]
as \[\overline{M}_{ij} = \overline{\sigma}(M'_{ij})=S\circ M'_{ij}\circ (S^{\otimes n})^{-1}.\]

As in the single graded case we can define a map
\[\Phi:S\s\OO\to S\s\End_{S\s\OO}\]
as the map making the following diagram commute
\[
\begin{tikzcd}
S\s\OO\arrow[rr, "\Phi"]\arrow[d] & & S\s\End_{S\s\OO}\\
\s\OO\arrow[r, "\Phi'"]& \End_{\s\OO}\arrow[r, "\cong"]& \s\End_{S\s\OO}\arrow[u]
\end{tikzcd}
\]
where 
\begin{align*}
\Phi'\colon&\s\OO \to \End_{\s\OO}\\
&x\mapsto \sum_{n\geq 0}b_n(x;-)
\end{align*}
and the isomorphism $\End_{\s\OO}\cong\s\End_{S\s\OO}$ is given by $\overline{\sigma}$.

In this setting we have the bigraded version of \Cref{theorem}. But before stating the theorem, for the sake of completeness let us state the definition of the Hochschild complex of a bigraded module.
\begin{defin}
We define the \emph{Hochschild cochain complex} of a bigraded module $A$ to be the bigraded module $S\s\End_A$. If $(A,d)$ is a vertical bicomplex, then the Hochschild complex has a differential given by $\partial(f)=[d,f]=d\circ f-(-1)^{q}f\circ d$, where $f$ has natural bidigree $(p,q)$ and $\circ$ is the plethysm corresponding to operadic inertions.
\end{defin}
In particular, $S\s\End_{S\s\OO}$ is the Hochschild cochain complex of $S\s\OO$. If $\OO$ has a derived $A_\infty$-multiplication, then the differential of the Hochschild complex $S\s\End_{S\s\OO}$ is given by $\overline{M}_{01}$ from \Cref{barbimaps}.
%\phantomsection
\begin{thm}\label{bigradedtheorem}
The map $\Phi$ defined above is a morphism of derived $A_\infty$-algebras, i.e. for all $j\geq 1$ the equation

\[\Phi(M_{ij})=\overline{M}_{ij}(\Phi^{\otimes j})\]
holds.%, where the $M_{ij}$ is the $j$-th component of the $A_\infty$-algebra structure on $S\s\OO$ and $\overline{M}_j$ is the $j$-th component of the $A_\infty$-algebra structure on $S\s\End_{S\s\OO}$. 
\end{thm}
\begin{proof}
The proof is the same as that of \Cref{theorem} but carrying the extra index $i$ and using the bigraded sign conventions.
\end{proof}
\end{document}
