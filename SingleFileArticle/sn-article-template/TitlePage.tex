%Version 2.1 April 2023
% See section 11 of the User Manual for version history
%
%%%%%%%%%%%%%%%%%%%%%%%%%%%%%%%%%%%%%%%%%%%%%%%%%%%%%%%%%%%%%%%%%%%%%%
%%                                                                 %%
%% Please do not use \input{...} to include other tex files.       %%
%% Submit your LaTeX manuscript as one .tex document.              %%
%%                                                                 %%
%% All additional figures and files should be attached             %%
%% separately and not embedded in the \TeX\ document itself.       %%
%%                                                                 %%
%%%%%%%%%%%%%%%%%%%%%%%%%%%%%%%%%%%%%%%%%%%%%%%%%%%%%%%%%%%%%%%%%%%%%

%%\documentclass[referee,sn-basic]{sn-jnl}% referee option is meant for double line spacing

%%=======================================================%%
%% to print line numbers in the margin use lineno option %%
%%=======================================================%%

%%\documentclass[lineno,sn-basic]{sn-jnl}% Basic Springer Nature Reference Style/Chemistry Reference Style

%%======================================================%%
%% to compile with pdflatex/xelatex use pdflatex option %%
%%======================================================%%

%%\documentclass[pdflatex,sn-basic]{sn-jnl}% Basic Springer Nature Reference Style/Chemistry Reference Style


%%Note: the following reference styles support Namedate and Numbered referencing. By default the style follows the most common style. To switch between the options you can add or remove �Numbered� in the optional parenthesis. 
%%The option is available for: sn-basic.bst, sn-vancouver.bst, sn-chicago.bst, sn-mathphys.bst. %  
 
%%\documentclass[sn-nature]{sn-jnl}% Style for submissions to Nature Portfolio journals
%%\documentclass[sn-basic]{sn-jnl}% Basic Springer Nature Reference Style/Chemistry Reference Style
\documentclass[sn-mathphys,Numbered]{sn-jnl}% Math and Physical Sciences Reference Style
%%\documentclass[sn-aps]{sn-jnl}% American Physical Society (APS) Reference Style
%%\documentclass[sn-vancouver,Numbered]{sn-jnl}% Vancouver Reference Style
%%\documentclass[sn-apa]{sn-jnl}% APA Reference Style 
%%\documentclass[sn-chicago]{sn-jnl}% Chicago-based Humanities Reference Style
%%\documentclass[default]{sn-jnl}% Default
%%\documentclass[default,iicol]{sn-jnl}% Default with double column layout

%%%% Standard Packages
%%<additional latex packages if required can be included here>

\usepackage{graphicx}%
\usepackage{multirow}%
\usepackage{amsmath,amssymb,amsfonts}%
\usepackage{amsthm}%
\usepackage{mathrsfs}%
\usepackage[title]{appendix}%
\usepackage{xcolor}%
\usepackage{textcomp}%
\usepackage{manyfoot}%
\usepackage{booktabs}%
\usepackage{algorithm}%
\usepackage{algorithmicx}%
\usepackage{algpseudocode}%
\usepackage{listings}%
\usepackage{cleveref}
\usepackage{pgf,tikz}
\usetikzlibrary{arrows}
\usetikzlibrary{cd}
\usepackage{adjustbox}
\usepackage{empheq}
%%%%

\newtheorem{manualtheoreminner}{Theorem}
\newenvironment{manualtheorem}[1]{%
  \renewcommand\themanualtheoreminner{#1}%
  \manualtheoreminner
}{\endmanualtheoreminner}
\newtheorem{questioninner}{Question}
\newenvironment{question}[1]{%
  \renewcommand\thequestioninner{#1}%
  \questioninner
}{\endquestioninner}


%%%%
\newcommand{\Z}{\mathbb{Z}}
\newcommand{\OO}{{\mathcal O}}
\newcommand{\PP}{\mathcal{P}}
\newcommand{\QQ}{\mathcal{Q}}
\newcommand{\CC}{{\mathcal C}}
\newcommand{\DD}{{\mathcal D}}
\newcommand{\V}{\mathcal{V}}
\newcommand{\calA}{\mathcal{A}}
\newcommand{\uC}{\underline{\mathscr{C}}}
\newcommand{\uD}{\underline{\mathscr{D}}}
\newcommand{\VV}{\mathscr{V}}
\newcommand{\umu}{\underline{\mu}}
\newcommand{\umui}{\underline{\mu}^{-1}}
\newcommand{\s}{\mathfrak{s}}
\newcommand{\col}{\mathrm{Col}}
\newcommand{\ob}{\mathrm{Ob}}
\newcommand{\tc}{\mathrm{tC}_R}
\newcommand{\fmod}{\mathrm{fMod}_R}
\newcommand{\fc}{\mathrm{fC}_R}
\newcommand{\vbOp}{\mathrm{vbOp}}
\newcommand{\fCOp}{\mathrm{fCOp}}
\newcommand{\sfc}{\mathrm{sfC}_R}
\newcommand{\bgmod}{\mathrm{bgMod}_R}
\newcommand{\vbc}{\mathrm{vbC}_R}
\newcommand{\ucr}{\underline{\mathrm{C}_R}}
\newcommand{\ubgmod}{\underline{\mathrm{bgMod}_R}}
\newcommand{\uvbc}{\underline{\mathrm{vbC}_R}}
\newcommand{\utc}{\underline{\mathrm{tC}_R}}
\newcommand{\ubgMod}{\underline{\mathpzc{bgMod}_R}}
\newcommand{\ufMod}{\underline{\mathpzc{fMod}_R}}
\newcommand{\usfMod}{\underline{\mathpzc{sfMod}_R}}
\newcommand{\uEnd}{\underline{\mathpzc{End}}}
\newcommand{\utC}{\underline{t\mathcal{C}_R}}
\newcommand{\ufC}{\underline{\mathpzc{fC}_R}}
\newcommand{\usfC}{\underline{\mathpzc{sfC}_R}}
\newcommand{\Tot}{\mathrm{Tot}}
\newcommand{\vdeg}{\mathrm{vdeg}}

\newcommand{\antishriek}{\text{\raisebox{\depth}{\textexclamdown}}}
%%%%
\DeclareMathOperator{\End}{End}
\DeclareMathOperator{\Ima}{Im}
\DeclareMathOperator{\Hom}{Hom}

\DeclareMathAlphabet{\mathpzc}{OT1}{pzc}{m}{it}


%%%%%=============================================================================%%%%
%%%%  Remarks: This template is provided to aid authors with the preparation
%%%%  of original research articles intended for submission to journals published 
%%%%  by Springer Nature. The guidance has been prepared in partnership with 
%%%%  production teams to conform to Springer Nature technical requirements. 
%%%%  Editorial and presentation requirements differ among journal portfolios and 
%%%%  research disciplines. You may find sections in this template are irrelevant 
%%%%  to your work and are empowered to omit any such section if allowed by the 
%%%%  journal you intend to submit to. The submission guidelines and policies 
%%%%  of the journal take precedence. A detailed User Manual is available in the 
%%%%  template package for technical guidance.
%%%%%=============================================================================%%%%

%\jyear{2021}%

%% as per the requirement new theorem styles can be included as shown below
\theoremstyle{theorem}%
\newtheorem{thm}{Theorem}[section]
%%\newtheorem{theorem}{Theorem}[section]% meant for sectionwise numbers
%% optional argument [theorem] produces theorem numbering sequence instead of independent numbers for Proposition
\newtheorem{propo}[thm]{Proposition}% 
%%\newtheorem{proposition}{Proposition}% to get separate numbers for theorem and proposition etc.
\newtheorem{lem}[thm]{Lemma}% 
\newtheorem{corollary}[thm]{Corollary}%

\theoremstyle{remark}%
\newtheorem{ex}{Example}%
\newtheorem{remark}{Remark}%

\theoremstyle{definition}%
\newtheorem{defin}{Definition}%

\raggedbottom
%%\unnumbered% uncomment this for unnumbered level heads

\begin{document}

\title[Article Title]{The Derived Deligne Conjecture}

%%=============================================================%%
%% Prefix	-> \pfx{Dr}
%% GivenName	-> \fnm{Joergen W.}
%% Particle	-> \spfx{van der} -> surname prefix
%% FamilyName	-> \sur{Ploeg}
%% Suffix	-> \sfx{IV}
%% NatureName	-> \tanm{Poet Laureate} -> Title after name
%% Degrees	-> \dgr{MSc, PhD}
%% \author*[1,2]{\pfx{Dr} \fnm{Joergen W.} \spfx{van der} \sur{Ploeg} \sfx{IV} \tanm{Poet Laureate} 
%%                 \dgr{MSc, PhD}}\email{iauthor@gmail.com}
%%=============================================================%%

\author*{\fnm{Javier} \sur{Aguilar Mart\'in}}\email{ja683@kent.ac.uk, javiecija96@gmail.com}

%\author[2,3]{\fnm{Second} \sur{Author}}\email{iiauthor@gmail.com}
%\equalcont{These authors contributed equally to this work.}

%\author[1,2]{\fnm{Third} \sur{Author}}\email{iiiauthor@gmail.com}
%\equalcont{These authors contributed equally to this work.}

\affil{\orgdiv{School of Mathematics, Statistics and Actuarial Sciences}, \orgname{University of Kent}}

%\affil[2]{\orgdiv{Department}, \orgname{Organization}, \orgaddress{\street{Street}, \city{City}, \postcode{10587}, \state{State}, \country{Country}}}
%
%\affil[3]{\orgdiv{Department}, \orgname{Organization}, \orgaddress{\street{Street}, \city{City}, \postcode{610101}, \state{State}, \country{Country}}}

%%==================================%%
%% sample for unstructured abstract %%
%%==================================%%

\abstract{We study the operad of derived $A_\infty$-algebras from a new point of view in order to find a derived version of the Deligne conjecture. We start by defining the brace structure on an operad of graded $R$-modules using operadic suspension, which we describe in depth for the first time as a functor, and use it to define $A_\infty$-algebra structures on certain operads, with the endomorphism operad as our main example. This construction provides us with an operadic context from which $A_\infty$-algebras arise in a natural way and allows us to generalize the Lie algebra structure on the Hochschild complex of an $A_\infty$-algebra. Next, we generalize these constructions to operads of bigraded $R$-modules, introducing a totalization functor. This allows us to generalize a Lie algebra structure on the total complex of a derived $A_\infty$-algebra. This construction and the use of some enriched categories allow us to obtain new versions of the Deligne conjecture.}

%%================================%%
%% Sample for structured abstract %%
%%================================%%

% \abstract{\textbf{Purpose:} The abstract serves both as a general introduction to the topic and as a brief, non-technical summary of the main results and their implications. The abstract must not include subheadings (unless expressly permitted in the journal's Instructions to Authors), equations or citations. As a guide the abstract should not exceed 200 words. Most journals do not set a hard limit however authors are advised to check the author instructions for the journal they are submitting to.
% 
% \textbf{Methods:} The abstract serves both as a general introduction to the topic and as a brief, non-technical summary of the main results and their implications. The abstract must not include subheadings (unless expressly permitted in the journal's Instructions to Authors), equations or citations. As a guide the abstract should not exceed 200 words. Most journals do not set a hard limit however authors are advised to check the author instructions for the journal they are submitting to.
% 
% \textbf{Results:} The abstract serves both as a general introduction to the topic and as a brief, non-technical summary of the main results and their implications. The abstract must not include subheadings (unless expressly permitted in the journal's Instructions to Authors), equations or citations. As a guide the abstract should not exceed 200 words. Most journals do not set a hard limit however authors are advised to check the author instructions for the journal they are submitting to.
% 
% \textbf{Conclusion:} The abstract serves both as a general introduction to the topic and as a brief, non-technical summary of the main results and their implications. The abstract must not include subheadings (unless expressly permitted in the journal's Instructions to Authors), equations or citations. As a guide the abstract should not exceed 200 words. Most journals do not set a hard limit however authors are advised to check the author instructions for the journal they are submitting to.}

\keywords{Operads, Derived $A_\infty$-algebras, Enriched categories, Deligne conjecture}

%%\pacs[JEL Classification]{D8, H51}

\pacs[MSC Classification]{18M70, 18M60, 18N70}

\maketitle



\end{document}
