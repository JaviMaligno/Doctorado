	\documentclass[twoside]{article}
\usepackage{estilo-ejercicios}
\setcounter{section}{0}
\newtheorem{defin}{Definition}[section]
\newtheorem{lem}[defin]{Lemma}
\newtheorem{propo}[defin]{Proposition}
\newtheorem{thm}[defin]{Theorem}
\newtheorem{eje}[defin]{Example}
\newtheorem{obs}[defin]{Observación}
\renewcommand{\baselinestretch}{1,3}

\usepackage{empheq}
\newcommand*\widefbox[1]{\fbox{\hspace{2em}#1\hspace{2em}}}
%--------------------------------------------------------
\begin{document}

\title{Sign conventions}
\author{Javier Aguilar Martín}
\maketitle

\section{Operadic suspension}

Let $sig_n$ be the sign representation of the symmetric group on $n$ symbols concentrated in degree 0. It comes with a natural action of the symmetriic group $S_n$. Let $\Lambda(n)=\Sigma^{n-1}sig_n$. This space can be realized as the one-dimensional vector space spanned by the exterior power $e_1\land\cdots\land e_n$ of degree $n-1$. Let us define an operad structure on $\Lambda=\{\Lambda(n)\}_{n\geq 0}$ via the following insertion maps

\[
\begin{tikzcd}
\Lambda(n)\otimes\Lambda(m) \arrow[r, "\circ_i"] & \Lambda(n+m-1)\\
(e_1\land\cdots\land e_n)\otimes(e_1\land\cdots\land e_m)\arrow[r, mapsto] & (-1)^{(n-i)(m-1)}e_1\land\cdots\land e_{n+m-1}.
\end{tikzcd}
\]

The only difference from Ward's operad is that his sign in the above map comes from counting the $i$-th insertion backwards from $n$, and here we count in the usual way. 

In a similar way we can define $\Lambda^-(n)=\Sigma^{1-n}sig_n$, with the same composition maps.

\begin{definition}
Let $\mathcal{O}$ be a graded linear operad. The \emph{operadic suspension} $\mathfrak{s}\OO$ of $\mathcal{O}$ is given arity-wise by the Hadamard product of operads $\mathfrak{s}\OO(n)=(\mathcal{O}\otimes\Lambda)(n)=\mathcal{O}(n)\otimes\Lambda(n)$. Similarly, we define the \emph{operadic desuspension} $\mathfrak{s}^{-1}\OO(n)=\mathcal{O}(n)\otimes\Lambda^-(n)$.
\end{definition}

It can be checked that $\mathfrak{s}\mathfrak{s}^{-1}\OO\cong\OO$ using the automorphism $f_n:\OO(n)\to\OO(n)$ defined by $f_n(a)=(-1)^{\frac{n(n+1)}{2}+1}a$.

We may identify the elements of $\mathcal{O}$ with the elements the elements of $\mathfrak{s}\OO$. If we write $\circ_i$ for the operadic insertion on $\OO$ and $\tilde{\circ}_i$ for the operadic insertion on $\mathfrak{s}\OO$, we may find a relation between the two insertion maps in the following way. Let $a\in\OO(n)$ and $b\in\OO(m)$, and let us compute $a\tilde{\circ}_i b$.

\begin{align*}
\mathfrak{s}\OO(n)\otimes\mathfrak{s}\OO(m)&=(\OO(n)\otimes\Lambda(n))\otimes (\OO(m)\otimes\Lambda(m))\cong (\OO(n)\otimes \OO(m))\otimes (\Lambda(n)\otimes \Lambda(m))\\
&\xrightarrow{\circ_i\otimes\circ_i} \OO(m+n-1)\otimes \Lambda(n+m-1)=\mathfrak{s}\OO(n+m-1).
\end{align*}

The symmetric monoidal structure produces the sign $(-1)^{(n-1)\deg(b)}$ and the operadic structure of $\Lambda$ produces the sign $(-1)^{(n-i)(m-1)}$, so 

$$a\tilde{\circ}_ib=(-1)^{(n-1)\deg(b)+(n-i)(m-1)}a\circ_i b.$$

Now we have mod 2

$$(n-i)(m-1)=(n-1-i-1)(m-1)=(n-1)(m-1)+(i-1)(m-1)$$

so we conclude 

$$a\tilde{\circ}_ib=(-1)^{(n-1)(m-1)+(n-1)\deg(b)+(i-1)(m-1)}a\circ_i b.$$

This is exactly the sign from R-W and differs from Ward's (also Gerstenhaber's, which implies Keller's) sign by $(-1)^{(n-1)(m-1)}$. We can show these two alternative supensions to be isomorphic as operads. For that purpuse we need to find an automorphism $f$ of $\OO$ such that 

$$f(a\circ_i b)=(-1)^{(n-1)(m-1)}f(a)\circ_i f(b).$$

It can be checked that $f(a)=(-1)^{\frac{n(n+1)}{2}+1}a$ is such an automorphism.

\begin{theorem}[Operads in Algebra, Topology and Physics]
Given a graded vector space $V$, there is an isomorphism of operads $\End_{\Sigma V}\cong \mathfrak{s}^{-1}\End_V$, where $\End_V$ is the endomorphism operad of $v$.
\end{theorem}
The proof in the original reference is not very explicit, but in the case of our operadic suspension, the isomorphism is given by $$\sigma^{-1}:\End_{\Sigma A}\to\mathfrak{s}^{-1}\End_A,$$ where $\sigma^{-1}(F)=(-1)^{\binom{n}{2}}\Sigma^{-1}\circ F\circ \Sigma^{\otimes n}$ for $F\in \End_{\Sigma A}(n)$. %I CAN BE MORE EXPLICIT ABOUT THIS, MAYBE WRITE IT LATER, AT LEAST THE INSERTION, ASK ABOUT THE PERMUTATION ACTION 
Note that we are using the identification of elements of $\End_A$ with those in $\mathfrak{s}^{-1}\End_A$. The notation $\sigma^{-1}$ comes from R-W. In the case of Ward's suspension, the isomorphism is given by $(-1)^{\binom{n}{2}}\sigma^{-1}$ (the same map but without the sign).

In R-W the sign for the insertion maps was obtained by computing $\sigma^{-1}(\sigma(a)\circ_i\sigma(b))$. This can be interpreted as sending $a$ and $b$ from $\End_A$ to $\End_{\Sigma A}$ via $\sigma$ (but only as graded vector spaces, not as operads) and then applying the isomorphism $\sigma^{-1}$. In the end this is the same as simply sending $a$ and $b$ to their images in $\mathfrak{s}^{-1}\End_A$, which is what was done at the beginning to obtain the sign.
\subsection{Braces}
We can define the braces using the operadic suspension via operadic composition. More precisely, we define the maps 
$$b_n:\mathfrak{s}\OO(N)\otimes\mathfrak{s}\OO(a_1)\otimes\cdots\otimes\mathfrak{s}\OO(a_n)\to\mathfrak{s}\OO(N-\sum a_i)$$
using the operadic composition $\gamma$ in $\mathfrak{s}\OO$ as

$$b_n(f;g_1,\dots,g_n)=\sum\gamma(f;1,\dots,1,g_1,1,\dots,1,g_n,1\dots,1),$$

where the sum runs over all possible ordering preserving insertions. 

 This way we will get the same sign as mimicking R-W's strategy, with the advantage that now the brace relation follows immediatly from the associativity axiom of operadic composition. To obtain the signs that make this composition differ from the composition in $\OO$, we must first look at the operadic composition on $\Lambda$. In the case of the endomorphism operad, we are interested in compositions of the form $f(1^{\otimes k_0}\otimes g_1\otimes 1^{k_1}\otimes\cdots\otimes g_n\otimes 1^{k_n})$ where $N-n=k_0+\cdots+k_n$, $f$ has (operadic) arity $N$ and each $g_i$ has arity $a_i$ and internal degree $q_i$. Therefore, let us consider the operadic composition

\[
\begin{tikzcd}
\Lambda(N)\otimes\Lambda(1)^{k_0}\otimes\Lambda(a_1)\otimes\Lambda(1)^{\otimes k_1}\otimes\cdots\otimes\Lambda(a_n)\otimes\Lambda(1)^{k_n}\arrow[r] & \Lambda(N-n+\sum_{i=1}^na_i)
\end{tikzcd}
\]

The composition can be described in terms of insertions in the obvious way, so we just have to find out the sign iterating the same argument as in the $i$-th insertion. If we were using Ward's suspension we would have to be more careful since counting backwards make other signs appear. But in this case, each $\Lambda(a_i)$ produces a sign given by the exponent $$(a_i-1)(N-k_0+\cdots+k_{i-1}-i).$$ 

For this, recall that the degree of $\Lambda(n)$ is $n-1$. Now, since $N-n=k_0+\cdots+k_n$, we have that
$$(a_i-1)(N-k_0+\cdots+k_{i-1}-i)=(a_i-1)(n-i+\sum_{l=i}^nk_l).$$

Now we can compute the sign produced by a brace. For this, notice that the isomorphism $(\OO(1)\otimes \Lambda(1))^{\otimes k}\cong \OO(1)^{\otimes k}\otimes \Lambda(1)^{\otimes k}$ does not produce any signs because of degree reasons. Therefore, therefore the sign coming from the isomorphism

$$\OO(N)\otimes\Lambda(N)\otimes (\OO(1)\otimes \Lambda(1))^{\otimes k_0}\bigotimes_{i=1}^n(\OO(a_i)\otimes\Lambda(a_i)\otimes(\OO(1)\otimes\Lambda(1))^{\otimes k_i}$$
$$\cong \OO(N)\otimes(\bigotimes_{i=1}^n \OO(a_i))\otimes \Lambda(N)\otimes(\bigotimes_{i=1}^n \Lambda(a_i))$$
is determined by the exponent

$$(N-1)\sum_{i=1}^nq_i+\sum_{i=1}^n (a_i-1)\sum_{l>i}q_l.$$

This equals
$$(\sum_{j=0}^nk_j +n-1)\sum_{i=1}^nq_i+\sum_{i=1}^n (a_i-1)\sum_{l>i}q_l.$$

After doing the composition 
$$\OO(N)\otimes(\bigotimes_{i=1}^n \OO(a_i))\otimes \Lambda(N)\otimes(\bigotimes_{i=1}^n \Lambda(a_i))\longrightarrow \OO(N-n+\sum_{i=1}^na_i)\otimes \Lambda(N-n+\sum_{i=1}^na_i)$$

we can add the sign coming from the suspension, so all in all the sign for the braces is

$$(a_i-1)(n-i+\sum_{l=i}^nk_l)+(\sum_{j=0}^nk_j +n-1)\sum_{i=1}^nq_i+\sum_{i=1}^n (a_i-1)\sum_{l>i}q_l.$$

Recall that the sign we obtained doing the analogue process of R-W was 

$$\sigma=\sum_{0\leq j<l\leq n}k_jq_l+\sum_{1\leq j<l\leq n}a_jq_l+\sum_{j=1}^n (a_j+q_j-1)(n-j)+\sum_{1\leq j\leq l\leq n} (a_j+q_j-1)k_l.$$

It is not hard to check that the two expressions are equal (not only mod 2, they have in fact the same value as integers).


\subsection{What can we do with this?}
First of all, we get an easier way to obtain the signs. In addition, this explanation fits better in the context of operads and feels more natural.

In addition, since we have an isomorphism of operads $\End_{\Sigma A}\cong \mathfrak{s}^{-1}\End_A$, we will be able to translate the results of Getzler, who works in $\End_{\Sigma A}$, to our situation, which is $\mathfrak{s}^{-1}\End_A$. This could help us prove the lemmas that we had in mind. In principle, the desuspension has the same signs as the suspension, but if it's convenient we may use the backwards insertion from Ward to get the isomorphism $\mathfrak{s}\mathfrak{s}^{-1}\OO\cong\OO$ in a more straightforward way (recall that there was an non-identity automorphism of $\OO$ in this isomorphism).

\section{Appendix: some proofs}

\begin{lemma}
$\mathfrak{s}^{-1}\mathfrak{s}\OO\cong\OO\cong\mathfrak{s}\mathfrak{s}^{-1}\OO$.
\end{lemma}
\begin{proof}
We are only showing the first isomorphism since the other one is analogous. We only need to look at the isomorphism
\begin{align*}
(\mathcal{O}(n)\otimes\Sigma^{n-1}sig_n\otimes \Sigma^{1-n}sig_n)\otimes (\mathcal{O}(m)\otimes\Sigma^{m-1}sig_m\otimes \Sigma^{1-m}sig_m)\cong\\ (\mathcal{O}(m)\otimes \mathcal{O}(m))\otimes (\Sigma^{n-1}sig_n\otimes \Sigma^{m-1}sig_m)\otimes (\Sigma^{1-n}sig_n\otimes \Sigma^{1-m}sig_m).
\end{align*}
After insertions, the only sign that do not cancel is $(-1)^{(n-1)(m-1)}$. So we need to find an automorphism $f$ of $\OO$ such that, for $a\in\OO(n)$ and $b\in\OO(m)$,

$$f(a\circ_i b)=(-1)^{(n-1)(m-1)}f(a)\circ_i f(b).$$

It can be checked that $f(a)=(-1)^{\frac{n(n+1)}{2}+1}a$ is such an automorphism.
\end{proof}

\begin{lemma}
For any integers $n$ and $m$, the following equiality  holds mod 2:

$$\binom{n+m-1}{2}+\binom{n}{2}+\binom{m}{2}=(n-1)(m-1).$$
\end{lemma}
\begin{proof}
Let us compute 

$$\binom{n+m-1}{2}+\binom{n}{2}+\binom{m}{2}+(n-1)(m-1)\mod 2.$$

By definition, this equals

\begin{gather*}
\frac{(n+m-1)(n+m-2)}{2}+\frac{n(n-1)}{2}+\frac{m(m-1)}{2}+(n-1)(m-1)=\\
\frac{(n^2+2nm-2n+m^2-2m-n-m+2)+(n^2-n)+(m^2-m)+2(nm-n-m+1)}{2}=\\
n^2+2nm-3n+m^2-3m+2=0\mod 2
\end{gather*}
as wanted.


\end{proof}

Recall that we define the suspension of a graded vector space $V$ as the graded vector space $\Sigma V$ having degree components $(\Sigma V)^i=V^{i-1}$.

\begin{theorem}
There is an isomorphism of operads $\End_{\Sigma V}\cong \mathfrak{s}^{-1}\End_V$.
\end{theorem}
\begin{proof}
For each $n$, we clearly have an isomorphism 

$$\End_{\Sigma V}=\Hom((\Sigma V)^{\otimes n},\Sigma V)\cong\Hom(V^{\otimes n},V)\otimes \Sigma^{n-1}sig_n= \mathfrak{s}^{-1}\End_V$$

given by the map $\sigma^{-1}$ defined before as $\sigma^{-1}(F)=(-1)^{\binom{n}{2}}\Sigma^{-1}\circ F\circ\Sigma^{\otimes n}$. We must show that this map is an isomorphism of operads, in other words, it commutes with insertions and with the symmetric group action.

Let us first check that $\sigma^{-1}$ commutes with insertions. Let $F\in \End_{\Sigma V}(n)$ and $G\in \End_{\Sigma V}(m)$. On the one had we have 

$$\sigma^{-1}(F\circ_i G)=(-1)^{\binom{n+m-1}{2}+|G|(i-1)}\Sigma^{-1}\circ F(\Sigma^{\otimes i-1}\otimes G(\Sigma^{\otimes m})\otimes\Sigma^{\otimes n-i}),$$

and on the other hand

$$\sigma^{-1}(F)\tilde{\circ}_i\sigma^{-1}(G)=(-1)^{(n-1)(m-1)+(n-1)(|G|+m-1)+(i-1)(m-1)}\sigma^{-1}(F)\circ_i\sigma^{-1}(G)=$$
$$(-1)^{\binom{n}{2}+\binom{m}{2}+(n-1)(m-1)+(n-1)(|G|+m-1)+(i-1)(m-1)+(|G|+m-1)(n-i)}\Sigma^{-1}\circ F(\Sigma^{\otimes i-1}\otimes G(\Sigma^{\otimes m})\otimes\Sigma^{\otimes n-i}).$$

By the previous lemma, 

$$\binom{n+m-1}{2}=\binom{n}{2}+\binom{m}{2}+(n-1)(m-1)\mod 2,$$

so we only need to check the equation

$$|G|(i-1)=(n-1)(|G|+m-1)+(i-1)(m-1)+(|G|+m-1)(n-i)\mod 2,$$

and this is straightforward.

Now we are going to show that $\sigma^{-1}$ commutes with the action of th symmetric group. Recall that on $\End_{\Sigma V}$ is the usual action, whilst on $\mathfrak{s}^{-1}\End_V$ the action is twisted by the sign of the permutation. It is enough to show this for transpositions of the form $\tau=(i\ i+1)$ since they generate the symmetric group.

On the one hand, 

$$\sigma^{-1}(F\tau)(v_1\otimes\cdots\otimes v_n)=(-1)^{\sum_{j=1}^n (n-j)v_j}\Sigma^{-1}\circ (F\tau)(\Sigma v_1\otimes\cdots\otimes \Sigma v_n)=$$

$$(-1)^{\sum_{j=1}^n (n-j)v_j+(v_i-1)(v_{i+1}-1)}\Sigma^{-1}\circ F(\Sigma v_1\otimes\cdots\otimes\Sigma v_{i+1}\otimes\Sigma v_i\otimes\cdots\otimes \Sigma v_n).$$

On the other hand

$$(\sigma^{-1}(F)\tau) (v_1\otimes\cdots\otimes v_n)=(-1)^{v_iv_{i+1}-1}\Sigma^{-1}\circ F\circ \Sigma^{\otimes n}(v_1\otimes\cdots\otimes v_{i+1}\otimes v_i\otimes\cdots\otimes v_n)=$$

$$(-1)^{v_iv_{i+1}-1+\sum_{j\neq i,i+1}(n-j)v_j +(n-i-1)v_i+(n-i)v_{i+1}}\Sigma^{-1}\circ f(\Sigma v_1\otimes\cdots\otimes \Sigma v_{i+1}\otimes \Sigma v_i\otimes\cdots\otimes \Sigma v_n).$$

Now we just have to check that the signs are the same. Modulo $2$, the sign of the first map is

$$v_iv_{i+1}+v_i+v_{i+1}-1+\sum_{j=1}^n(n-j)v_j=$$
$$v_iv_{i+1}-1+\sum_{j\neq i,i+1}^n(n-j)v_j+(n-i-1)v_i+(n-i)v_{i+1},$$

which indeed coincide with the sign on the second map.

%\url{https://mathoverflow.net/questions/366792/detailed-proof-of-mathfraks-1-mathrmend-v-cong-mathrmend-sigma-v}
\end{proof}

THE INTERLUDE FROM THE SUMMARY

\begin{remark}

Assuming $M_j\in \End_{\Sigma\mathfrak{s}\OO}$, I proved that I was able to define it so that $\deg(M)=2-j$ (and obviously the arity is $j$). So if I have to apply the Koszul rule here, the degree used is just $2-j$. If I get to define $M_j\in\mathfrak{s}\End_{\Sigma\mathfrak{s}\OO}$, then $M_j$ is actually $M_j\otimes e_J$ where $e_J=e_1\land\dots\land e_j$ has degree $j-1$. So 

$$M_j\otimes e_J(x_1,\dots, x_j)=(-1)^{(j-1)(|x_1|+\cdots+|x_j|)}M_j(x_1,\dots, x_j)\otimes e_J$$
being $|x|$ the total degree (the natural degree on $\Sigma\mathfrak{s}\OO$, recall that $M_j$ wa defined via composition on this odd operad). So passing by the $M_j$ component would yield a sign depending on its internal degree, i.e. $2-j$.

For instance, if I get to define $M_2$ such that $$0=M_2\tilde{\circ}M_2=M_2\tilde{\circ}_2 M_2+M_2\tilde{\circ}_1 M_2$$ in the suspension, evaluating at $(x,y,z)$ gives us on the first summand

$$(M_2\circ_2M_2)(x,y,z)=(M_2(1,M_2(1,1))\otimes (e_1\land e_2\land e_3))(x,y,z)=(-1)^{(|x|+|y|+|z|)(3-1)}M_2(x,M_2(y,z))$$

and on the second summand
$$(M_2\circ_1M_2)(x,y,z)=-(M_2(M_2(1,1),1)\otimes (e_1\land e_2\land e_3))(x,y,z)=-(-1)^{(|x|+|y|+|z|)(3-1)}M_2(x,M_2(y,z))$$

Adding the two equals zero so we get the associativity condition $M_2(x,M_2(y,z))=M_2(M_2(x,y),z)$. Note that here $x$ is beeing permuted with $M_2$ but no extra signs appears, which is equivalent to apply the Koszul rule with the internal degree of $M_2$, which is $0$, and is in fact what we have done in the evaluation.

\end{remark}

\end{document}
