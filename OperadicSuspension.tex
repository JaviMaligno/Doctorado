	\documentclass[twoside]{article}
\usepackage{estilo-ejercicios}
\setcounter{section}{0}
\newtheorem{defin}{Definition}[section]
\newtheorem{lem}[defin]{Lemma}
\newtheorem{propo}[defin]{Proposition}
\newtheorem{thm}[defin]{Theorem}
\newtheorem{eje}[defin]{Example}
\newtheorem{obs}[defin]{Observación}
\renewcommand{\baselinestretch}{1,3}

\usepackage{empheq}
\newcommand*\widefbox[1]{\fbox{\hspace{2em}#1\hspace{2em}}}
%--------------------------------------------------------
\begin{document}

\title{Sign conventions}
\author{Javier Aguilar Martín}
\maketitle

\section{Operadic suspension}

Let $sig_n$ be the sign representation of the symmetric group on $n$ symbols concentrated in degree 0. It comes with a natural action of the symmetriic group $S_n$. Let $\Lambda(n)=\Sigma^{n-1}sig_n$. This space can be realized as the one-dimensional vector space spanned by the exterior power $e_1\land\cdots\land e_n$ of degree $n-1$. Let us define an operad structure on $\Lambda=\{\Lambda(n)\}_{n\geq 0}$ via the following insertion maps

\[
\begin{tikzcd}
\Lambda(n)\otimes\Lambda(m) \arrow[r, "\circ_i"] & \Lambda(n+m-1)\\
(e_1\land\cdots\land e_n)\otimes(e_1\land\cdots\land e_m)\arrow[r, mapsto] & (-1)^{(n-i)(m-1)}e_1\land\cdots\land e_{n+m-1}.
\end{tikzcd}
\]

The only difference from Ward's operad is that his sign in the above map comes from counting the $i$-th insertion backwards from $n$, and here we count in the usual way. 

In a similar way we can define $\Lambda^-(n)=\Sigma^{1-n}sig_n$, with the same composition maps.

\begin{definition}
Let $\mathcal{O}$ be a graded linear operad. The \emph{operadic suspension} $\mathfrak{s}\OO$ of $\mathcal{O}$ is given arity-wise by the Hadamard product of operads $\mathfrak{s}\OO(n)=(\mathcal{O}\otimes\Lambda)(n)=\mathcal{O}(n)\otimes\Lambda(n)$. Similarly, we define the \emph{operadic desuspension} $\mathfrak{s}^{-1}\OO(n)=\mathcal{O}(n)\otimes\Lambda^-(n)$.
\end{definition}

It can be checked that $\mathfrak{s}\mathfrak{s}^{-1}\OO\cong\OO$ using the automorphism $f_n:\OO(n)\to\OO(n)$ defined by $f_n(a)=(-1)^{\frac{n(n+1)}{2}+1}a$.

We may identify the elements of $\mathcal{O}$ with the elements the elements of $\mathfrak{s}\OO$. If we write $\circ_i$ for the operadic insertion on $\OO$ and $\tilde{\circ}_i$ for the operadic insertion on $\mathfrak{s}\OO$, we may find a relation between the two insertion maps in the following way. Let $a\in\OO(n)$ and $b\in\OO(m)$, and let us compute $a\tilde{\circ}_i b$.

\begin{align*}
\mathfrak{s}\OO(n)\otimes\mathfrak{s}\OO(m)&=(\OO(n)\otimes\Lambda(n))\otimes (\OO(m)\otimes\Lambda(m))\cong (\OO(n)\otimes \OO(m))\otimes (\Lambda(n)\otimes \Lambda(m))\\
&\xrightarrow{\circ_i\otimes\circ_i} O(m+n-1)\otimes \Lambda(n+m-1)=\mathfrak{s}\OO(n+m-1).
\end{align*}

The symmetric monoidal structure produces the sign $(-1)^{(n-1)\deg(b)}$ and the operadic structure of $\Lambda$ produces the sign $(-1)^{(n-i)(m-1)}$, so 

$$a\tilde{\circ}_ib=(-1)^{(n-1)\deg(b)+(n-i)(m-1)}a\circ_i b.$$

Now we have mod 2

$$(n-i)(m-1)=(n-1-i-1)(m-1)=(n-1)(m-1)+(i-1)(m-1)$$

so we conclude 

$$a\tilde{\circ}_ib=(-1)^{(n-1)(m-1)+(n-1)\deg(b)+(i-1)(m-1)}a\circ_i b.$$

This is exactly the sign from R-W and differs from Ward's (also Gerstenhaber's) sign in $(-1)^{(n-1)(m-1)}$. We can show these two alternative supensions to be isomorphic as operads. For that purpuse we need to find an automorphism $f$ of $\OO$ such that 

$$f(a\circ_i b)=(-1)^{(n-1)(m-1)}f(a)\circ_i f(b).$$

It can be checked that $f(a)=(-1)^{\frac{n(n+1)}{2}+1}a$ is such an automorphism.

\begin{theorem}[Operads in Algebra, Topology and Physics]
Given a graded vector space $A$, there is an isomorphism of operads $End_{\Sigma A}\cong \mathfrak{s}^{-1}End_A$, where $End_A$ is the endomorphism operad of $A$.
\end{theorem}
In the case of our operadic suspension, the isomorphism is given by $$\sigma^{-1}:End_{\Sigma A}\to\mathfrak{s}^{-1}End_A,$$ where $\sigma^{-1}(F)=(-1)^{\binom{n}{2}}\Sigma^{-1}\circ F\circ \Sigma^{\otimes n}$ for $F\in End_{\Sigma A}(n)$. I CAN BE MORE EXPLICIT ABOUT THIS, MAYBE WRITE IT LATER, AT LEAST THE INSERTION, ASK ABOUT THE PERMUTATION ACTION Note that we are using the identification of elements of $End_A$ with those in $\mathfrak{s}^{-1}End_A$. The notation $\sigma^{-1}$ comes from R-W. In the case of Ward's suspension, the isomorphism is given by $(-1)^{\binom{n}{2}}\sigma^{-1}$ (the same map but without the sign).

In R-W the sign for the insertion maps was obtained by computing $\sigma^{-1}(\sigma(a)\circ_i\sigma(b))$. This can be interpreted as sending $a$ and $b$ from $End_A$ to $End_{\Sigma A}$ via $\sigma$ (but only as graded vector spaces, not as operads) and then applying the isomorphism $\sigma^{-1}$. In the end this is the same as simply sending $a$ and $b$ to their images in $\mathfrak{s}^{-1}End_A$, which is what was done at the beginning to obtain the sign.

WRITE ABOUT THE BRACES

WRITE ABOUT THE $M_j$
\end{document}
