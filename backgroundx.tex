\documentclass[join.tex]{subfiles}

\begin{document}



\section{Background and conventions}\label{Sec1}

In this section we include all necessary background and the conventions we use throughout the article. It is divided in two parts, one corresponding to classical $A_\infty$-algebras and another one for derived $A_\infty$-algebras.

\subsection{$A_\infty$-algebras}

Let us start by recalling some background definitions and results that we will need to study $A_\infty$-algebras, as well as stating some conventions.

%MAYBE OMIT ALL MENTION TO SYMMETRIC GROUPS BECAUSE I'M NOT REALLY GONNA USE THEM
We assume that the reader is familiar with the basic definitions regarding $A_{∞}$-algebras and operads, but we are going to briefly recall some of them in this section to establish notation and assumptions.

Our base category is the category of graded $R$-modules and graded maps, where $R$ is a commutative ring with unit of characteristic distinct of 2. All tensor products are taken over $R$. We denote the $i$-th degree component of $A$ as $A^i$. If $x\in A^i$ we write $\deg(x)=i$. The $R$-modules $\Hom_R(A,B)$ are naturally graded by \[\Hom_R(A,B)^i=\prod_k\Hom_R(A^k,B^{k+i}).\]

We adopt the following Koszul sign convention: for $x\in A$, $y\in B$, $f\in\Hom_R(A,C)$ and $g\in\Hom_R(B,D)$

\[(f\otimes g)(a\otimes b)=(-1)^{\deg(x)\deg(g)}f(x)\otimes g(y).\] %NOT SURE IF SUM OR PROD %I think it's prod because it's all homomorphisms that raise the degree by iFFFLVVVV

\begin{defin}
For a graded $R$-module $A$ we consider the \emph{shift} or \emph{suspension} $SA$ for which $SA^i=A^{i-1}$. %I think it is not wortth defining it on maps because I don't use it
\end{defin}
\begin{defin}
An $A_\infty$-algebra is a graded $R$-module $A$ together with a family of maps $m_n:A^{\otimes n}\to A$ of degree $2-n$ satisfying for all $n\geq 1$ the equation

\begin{equation}\label{ainftyequation}
\sum_{r+s+t=n}(-1)^{rs+t}m_{r+t+1}(1^{\otimes r}\otimes m_s\otimes 1^{\otimes t})=0.
\end{equation}
\end{defin}
%This is equivalent to define it on SA without signs,  but there's not need to point that out here and it also depends on the convention of S^n^{-1}
The above equation will sometimes be referred to as the $A_\infty$-\emph{equation}. 

\begin{defin}\label{inftymorphism}
An $\infty$-morphism of $A_\infty$-algebras $A\to B$ is a family of maps \[f_n:A^{\otimes n}\to B\] of degree $1-n$ satisfying for all $n\geq 1$ the equation
\[\sum_{r+s+t=n} (-1)^{rs+t}f_{r+1+t}(1^{\otimes r} \otimes m^A_s\otimes 1^{\otimes t})=\sum_{i_1+\cdots+i_k=n} (-1)^s m^B_k(f_{i_1}\otimes\cdots\otimes f_{i_k}),\]
where

\[s=\sum_{\alpha<\beta}i_\alpha(1-i_\beta).\]
The composition of $\infty$-morphisms $f:A\to B$ and  $g:B\to C$ is given by 

\[(gf)_n=\sum_r\sum_{i_1+\cdots+i_r=n}(-1)^s g_r(f_{i_1}\otimes\cdots
\otimes f_{i_r}).\]
\end{defin}

\begin{defin}
A \emph{morphism} of $A_\infty$-algebras is a degree 0 map $f:A\to B$ such that
\[f(m^A_j)=m^B_j\circ f^{\otimes j}.\]
\end{defin}
%I THINK I AM GOING TO OMIT ALL MENTIONS TO SYMETRIC GROUPS OR MAYBE JUST SAY THAT THE RESULTS  FOR SYMMETRIC VERSIONS ARE ANALOGOUS

%IN A THESIS CHAPTER I WOULD PUT MOST OF THESE DEFINITION AT THE INTRO CHAPTER

We will discuss these topics in the language of operads, for which we need some more definitions.

\subsection{Operads}

We start defining the underlying object of an operad.

\begin{defin}\label{collections}
A \emph{collection} is a family $\OO=\{\OO(n)\}_{n\geq 0}$ of graded $R$-modules. We call the integer $n$ the \emph{arity}. When there is an action of the symmetric group $\Sigma_n$ on each $\OO(n)$ we say that the collection is an $\mathbb{S}-$module. A map of collections $f:\OO\to\mathcal{P}$ is a family of maps $f_n:\OO(n)\to\mathcal{P}(n)$. A map of collection is a map of $\mathbb{S}-$modules when it preserves the symmetric group action.
\end{defin}

\begin{defin}
The \emph{plethysm} or \emph{composite} $\OO\circ\PP$ of two collections $\OO$ and $\PP$ given by
\[(\OO\circ\PP)(n)=\bigoplus_{N\geq 0}\OO(N)\otimes \left(\bigoplus_{a_1+\cdots+a_k=n} \PP(a_1)\otimes\cdots\otimes \PP(a_k)\right).\]
\end{defin}
There is a definition for $\mathbb{S}$-modules that requires some tools from the representation theory of symmetric groups that we are not going to introduce here. The reader is referred to \cite{lodayvallette} for the details. 

\begin{defin}
The \emph{plethysm} or \emph{composite} $f\circ g$ of maps $f:\OO\to\OO'$ and $g:\PP\to\PP'$ is given by
\[(f\circ g)(x\otimes x_1\otimes\cdots\otimes x_k)=(-1)^{\varepsilon} f(x)\otimes g(x_1)\otimes\cdots\otimes g(x_k),\]
where $\varepsilon$ is the Koszul sign obtained from swaping each $g$ by the correspoding elements. 
\end{defin}

It is known that the category of collections with plethysm is a monoidal category, where the unit is the collection $I(1)=R$ and $I(n)=0$ for $n\neq 1$. 
\begin{defin}
A (non-symmetric) \emph{operad} is a collection $\OO=\{\OO(n)\}$ where there is a distinguished \emph{identiy} element $1\in\OO(1)$ and with \emph{insertion maps} 
\[\circ_i:\OO(n)\otimes \OO(m)\to \OO(m+n-1)\]
for each $1\leq i\leq n$ satisfying natural uunitality and associativity axioms. Insertion maps can be iterated to define \emph{composition maps} \[\gamma(x;x_1,\dots, x_n)=(\cdots(x\circ_1 x_1)\circ_2 x_2\cdots
)\circ_n x_n).\]
If $\OO$ is an $\mathbb{S}-$module and the insertion maps satisfy some additional axioms regarding the symmetric group action, we say that $\OO$ is a symmetric operad. See \cite{lodayvallette} for more details.

A map of operads (resp. symmetric operads) is a map of collections (resp. $\mathbb{S}-$modules) that is compatible with insertions.
\end{defin}

It is also known that an operad $\OO$ is equivalent to a monoid in the monoidal category of collections with plethysm, where the multiplication  map is given precisely by the composition $\gamma:\OO\circ\OO\to\OO$. See \cite[\S 5]{lodayvallette} for more details. 

\begin{defin} An operad $\OO$ is called \emph{reduced} if $\OO(0)=0$.\end{defin}

\begin{defin}
The \emph{endomorphism operad} $\End_A$ of a graded $R$-module $A$ is given by the modules \[\End_A(n)=\hom(A^{\otimes n},A).\] Insertion maps are given by
\[f\circ_i g=f(1^{\otimes i-1}\otimes g\otimes 1^{\otimes n-i})\]
for $f\in\End_A(n)$ and $g\in\End_A(m)$. The identity element is given by the identity map and there is a symmetric group action given by permuting the inputs.

An algebra over an operad $\OO$ is a map of operads $\OO\to\End_A$. By adjunction, this is equivalent to a colllection of maps $\OO(n)\otimes A^{\otimes n}\to A$ for each $\geq 0$.  
\end{defin}
The $\mathcal{A}_\infty$-operad is the non-symmetric operad whose algebras are $A_\infty$-algebras. Therefore, it is generated by elements $\mu_i\in\mathcal{A}_\infty(i)$ satisfying the operadic version of the $A_\infty$-equation. More details about this operad can be found in \cite[Chapter 9]{lodayvallette}

All the above definitions generalize with no substantial changes to the bigraded case as well. In the next section we will cover the background specifically needed for bigraded modules.

\subsection{For the study of derived $A_\infty$-algebras}\label{background}

Now  we move to the background and conventions that we need in order to study derived $A_\infty$-algebras. We collect some preliminary definitions. Most of them come from \cite[\S 2]{whitehouse} but we adapt them here to our conventions.
%Let us start by  fixing some notation and conventions. Fix a commutative ring with unit $R$ of characteristic distinct of $2$. All tensor products taken over $R$. %COPY SECTION 2.2 OF DAINFTY AND THEIR HOMOTOPIES AS BACKGROUND, REFERENCES TO WHITEHOUSE. INCLUDING DEFINITION OF TWISTED COMPLEX

%REDEFINE THINGS ACCORDING TO CONVENTIONS

Let $\CC$ be a category and let $A$, $B$ be arbitrary
objects in $\CC$. We denote by $\Hom_\CC(A,B)$ the set of morphisms from $A$ to $B$ in $\CC$. If $(\CC,⊗, 1)$ is
symmetric monoidal closed, then we denote its internal hom-object by $[A,B] ∈ \CC$.



\subsubsection{Filtered Modules and complexes}
%INCLUSION IS REVERSED
We collect some definitions about filtered modules and filtered complexes.

\begin{defin}
A \emph{filtered $R$-module} $(A, F)$ is given by a family of $R$-modules $\{F_pA\}_{p∈\Z}$ indexed by
the integers such that $F_{p}A ⊆ F_{p-1}A$ for all $p ∈ \Z$ and $A = ∪F_pA$. A morphism of filtered modules is a morphism $f : A → B$ of $R$-modules which is compatible with filtrations: 
\[f(F_pA) ⊂ F_pB \text{ for all }p ∈ \Z.\]
\end{defin}

\begin{defin}\label{filteredcomplex}
A \emph{filtered complex} $(K, d, F)$ is a complex $(K, d) ∈ \mathrm{C}_R$ together with a filtration $F$ of each $R$-module $K^n$ such that $d(F_pK^n) ⊂ F_pK^{n+1}$ for all $p, n ∈ \Z$. Its morphisms are given by
morphisms of complexes $f : K → L$ compatible with filtrations: \[f(F_pK) ⊂ F_pL\text{ for all }p ∈ \Z.\]
\end{defin}

We denote by $\fmod$ and $\fc$ the categories of filtered modules and filtered complexes of $R$-modules, respectively.

%$-1$ CHANGED TO 1 (IT MUST BE LIKE THIS BECAUSE I THE MAPS PRESERVING FILTRATION WILL BE LEVEL 0 AND I WANT THE WHOLE AINFTY OPERAD TO BE MAPPED TO THAT, SO R MUST BE IN LEVEL 0 AS WELL)
\begin{defin}\label{filteredtensor}
The tensor product of two filtered $R$-modules $(A, F)$ and $(B, F)$ is a filtered $R$-module,
with
 \[F_p(A ⊗ B) :=\sum_{i+j=p}\Ima(F_iA ⊗ F_jB → A ⊗ B).\]
This makes the category of filtered $R$-modules into a symmetric monoidal category, where the unit is given by $R$ with the trivial filtration $0 = F_{1}R ⊂ F_0R = R$.
\end{defin}


\begin{defin}\label{filterend}
Let $K$ and $L$ be filtered complexes. We define $\underline{\Hom}(K,L)$ to be the filtered complex whose underlying chain complex is $\Hom_{\mathrm{C}_R}(K,L)$ and the filtration $F$ given by 
\[F_p\underline{\Hom}(K,L)=\{f:K\to L\mid f(F_qK)\subset F_{q+p}L\text{ for all }q ∈ \Z.\}\]
In particular, $\Hom_{\fmod}(K,L)=F_0\underline{\Hom}(K,L)$.
\end{defin}

\subsubsection{Bigraded modules, vertical bicomplexes, twisted complexes and sign conventions}


We collect some definitions of basic categories that we need to use and stablish some conventions.
%TRY TO INCLUDE ONLY THE NECESSARY

\begin{defin}
We consider $(\Z,\Z)$-bigraded
$R$-modules $A = \{A^j_i\}$, where elements of $A^j_i$ are said to have bidegree $(i, j)$. We sometimes refer to $i$
as the \emph{horizontal} degree and $j$ the \emph{vertical degree}. The \emph{total degree} of an element $x ∈ A^j_i$ is $i+j$.
\end{defin}
\begin{defin}
A \emph{morphism of bidegree $(p, q)$} maps $A^j_i$ to $A^{j+q}_{i+p}$. The tensor product of two bigraded $R$-modules $A$
and $B$ is the bigraded $R$-module $A ⊗ B$ given by
\[(A ⊗ B)^j_i \coloneqq\bigoplus_{p,q}A^q_p ⊗ B^{j−q}_{i−p} .\]
\end{defin}
We denote by $\bgmod$ the category whose objects are bigraded $R$-modules and whose morphisms
are morphisms of bigraded $R$-modules of bidegree $(0, 0)$. It is symmetric monoidal with the above
tensor product.

We introduce the following scalar product notation for bidegrees: for $x$, $y$ of bidegree $(x_1, x_2)$, $(y_1, y_2)$
respectively, we let $\langle x, y\rangle = x_1y_1 + x_2y_2$.

The symmetry isomorphism
\[τ_{A⊗B} : A ⊗ B → B ⊗ A\]
is given by
\[x ⊗ y \mapsto (−1)^{\langle x,y\rangle}y ⊗ x.\]
We follow the Koszul sign rule: if $f : A → B$ and $g : C → D$ are bigraded morphisms, then the
morphism $f ⊗ g : A ⊗ C → B ⊗ D$ is defined by
\[(f ⊗ g)(x ⊗ z) \coloneqq (−1)^{\langle g,x\rangle}f(x) ⊗ g(z).\]

\begin{defin}
A \emph{vertical bicomplex} is a bigraded $R$-module $A$ equipped with a vertical differential $d^A : A → A$ of bidegree $(0, 1)$. A morphism of vertical bicomplexes is a morphism of bigraded modules
of bidegree $(0, 0)$ commuting with the vertical differential.
\end{defin}

We denote by $\vbc$ the category of vertical bicomplexes. The tensor product of two vertical bicomplexes $A$ and $B$ is given by endowing the tensor product of underlying bigraded modules with
vertical differential \[d^{A⊗B} := d^A ⊗ 1 + 1 ⊗ d^B : (A ⊗ B)^v_u → (A ⊗ B)^{v+1}_u .\] This makes $\vbc$ into a
symmetric monoidal category.

The symmetric monoidal categories $(\mathrm{C}_R,⊗,R)$, $(\bgmod,⊗,R)$ and $(\vbc,⊗,R)$ are related by embeddings $\mathrm{C}_R\to\vbc$ and $\bgmod \to\vbc$ which are monoidal and full.



\begin{defin}\label{delta1}
Let $A,B$ be bigraded modules. We define $[A,B]^∗_∗$
to be the bigraded module of morphisms of bigraded modules $A → B$. Furthermore, if $A,B$ are vertical bicomplexes, and $f ∈
[A,B]^v_u$, we define
\[δ(f) := d_Bf − (−1)^vfd_A.\]
\end{defin}

\begin{lem}
If $A$, $B$ are vertical bicomplexes, then $([A,B]^∗_∗
, δ)$ is a vertical bicomplex.
\end{lem}
\begin{proof}
Direct computation shows $\delta^2=0$.
\end{proof}
%END OF PAGE 5, I DEFINE THE SHIFT LATER, SO MAYBE IT IS NOT NECESSARY, BUT THINK ABOUT SHIFT OF MAPS (IS IT NECESSARY  TO ADD THAT SIGN?) AND COMPARE WITH THE GRADED CASE

%I CHANGED HORIZONTAL DEGREE SIGNS
\begin{defin}\label{twistedcomplex} A \emph{twisted complex} $(A, d_m)$ is a bigraded $R$-module $A = \{A^j_i \}$ together with a family
of morphisms $\{d_m : A → A\}_{m≥0}$ of bidegree $(m,1−m )$ such that for all $m ≥ 0$,
%CHECK CONVENTION IN CASE IT IS EASIER TO TAKE $(-1)^j$ 

%FROM DAINFTY ONE GETS J, BUT MI1 SATISFIES I, SO IN ANY CASE I WOULD HAVE TO DO SOME CHANGE OF VARIABLES
\[\sum_{i+j=m}(−1)^id_id_j = 0.\]

\end{defin}

\begin{defin}\label{twistedmorphisms}
A morphism of twisted complexes $f : (A, d^A_m) → (B, d^B_m)$ is given by a family of morphisms of $R$-modules $\{f_m : A → B\}_{m≥0}$ of bidegree $(m,−m)$ such that for all $m ≥ 0$,
\[\sum_{i+j=m}d^B_if_j =\sum_{i+j=m}(−1)^if_id^A_j.\]
The composition of morphisms is given by $(g \circ f)_m :=\sum_{i+j=m} g_if_j$.

A morphism $f = \{f_m\}_{m≥0}$ is
said to be \emph{strict} if $f_i = 0$ for all $i > 0$. The \emph{identity} morphism $1_A : A → A$ is the strict morphism
given by $(1_A)_0(x) = x.$ A morphism $f = \{f_i\}$ is an isomorphism if and only if $f_0$ is an isomorphism of
bigraded $R$-modules. Indeed, an inverse of $f$ is obtained from an inverse of $f_0$ by solving a triangular system.
\end{defin}
Denote by $\tc$ the category of twisted complexes. The following construction endows $\tc$ with a symmetric monoidal structure. See \cite[Lemma 3.3]{whitehouse} for a proof.
\begin{lem}\label{tensortwisted}
The category $(\tc,⊗,R)$ is symmetric monoidal, where the monoidal structure is given
by the bifunctor
\[⊗ : \tc × \tc → \tc\]
which on objects is given by $((A, d^A_m), (B, d^B_m)) → (A ⊗ B, d^A_m ⊗ 1 + 1 ⊗ d^B_m)$ and on morphisms is
given by $(f, g) → f ⊗ g$, where $(f ⊗ g)_m :=\sum_{i+j=m} f_i ⊗ g_j$. In particular, by the Koszul sign rule we
have that \[(f_i ⊗g_j)(x⊗z) = (−1)^{\langle g_j ,x\rangle}f_i(x)⊗g_j(z).\] The symmetry isomorphism is given by the strict
morphism of twisted complexes
\[τ_{A⊗B} : A ⊗ B → B ⊗ A\]
defined by
\[x ⊗ y\mapsto (−1)^{\langle x,y\rangle}y ⊗ x.\]
\end{lem}

The internal hom on bigraded modules can be extended to twisted complexes via the following lemma whose proof is in \cite[Lemma 3.4]{whitehouse}.
\begin{lem}\label{di} Let $A,B$ be twisted complexes. For $f ∈ [A,B]^v_u$, setting
\[(d_if) := (−1)^{i(u+v)}d^B_if − (−1)^vfd^A_i,\]
for $i ≥ 0$, endows $[A,B]^∗_∗$ with the structure of a twisted complex.
\end{lem}

%UNDERLINED CATEGORIES?
\subsubsection{Totalization}\label{total}
%DEFINITION WITH FILTRATION, DIFFERENTIAL, MONOIDALITY
Here we recall the definition of the totalization functor from \cite{whitehouse} and some of the structure that it comes with.

%CHOOSE NOTATION, T OR TOT

\begin{defin}
The \emph{totalization} $\Tot(A)$ of a bigraded $R$-module $A = \{A^j_i \}$ the graded $R$-module is given by
\[\Tot(A)^n \coloneqq
\bigoplus_{i<0}A^{n-i}_i ⊕\prod_{i\geq 0}A^{n-i}_i .\]
The \emph{column filtration} of $\Tot(A)$ is the filtration given by \[F_p\Tot(A)^n \coloneqq\prod_{i\geq p} A^{n-i}_i .\]
\end{defin}

Given a twisted complex $(A, d_m)$, define a map $d : \Tot(A) → \Tot(A)$ of degree $1$ by letting
\[d(x)_j \coloneqq \sum_{m≥0}(−1)^{mn}d_m(x_{j-m}),\]
for $x = (x_i)_{i∈\Z} ∈ \Tot(A)^n$,
where $x_i ∈ A^{n-i}_i$ denotes the $i$-th component of $x$, and $d(x)_j$ denotes the $j$-th component of $d(x)$. Note
that, for a given $j ∈ \Z$ there is a sufficiently large $m ≥ 0$ such that $x_{j-m′} = 0$ for all $m′ ≥ m$. Hence
$d(x)_j$ is given by a finite sum. Also, for negative $j$ sufficiently large, one has $x_{j-m} = 0$ for all $m ≥ 0$, which
implies $d(x)_j = 0$.

Given a morphism $f : (A, d_m) → (B, d_m)$ of twisted complexes, let $\Tot(f) : \Tot(A) → \Tot(B)$ be
the map of degree 0 defined by
\[(\Tot(f)(x))_j \coloneqq \sum_{m≥0}(−1)^{mn}f_m(x_{j-m}),\]
 for $x = (x_i)_{i∈\Z} ∈ \Tot(A)^n$.
 
\begin{thm}
The assignments $(A, d_m) \mapsto (\Tot(A), d, F)$, where $F$ is the column filtration of $\Tot(A)$,
and $f \mapsto \Tot(f)$ define a functor $\Tot : \tc \to \fc$ which is an isomorphism of categories when restricted to its image.
\end{thm}
\begin{proof}
See \cite[Theorem 3.8]{whitehouse}.
\end{proof}
For a filtered complex of the form $(\Tot(A),d,F)$, where $A = \{A^j_i \}$ is a bigraded $R$-module, we can recover the twisted complex structure on  $A$ as follows. For all $m ≥ 0$, let
$d_m : A → A$ be the morphism of bidegree $(m,1-m)$ defined by 
\[d_m(x) = (−1)^{nm}d(x)_{i+m},\] 
where $x ∈ A^{n-i}_i$ and $d(x)_k$ denotes the $k$-th component of $d(x)$, which lies in $A^{n+1-k}_k$.

%Let us denote the image of $\Tot$ by $\sfc$ and define the inverse $\Tot^{-1}:\sfc\to\tc$ as follows. Let $(\Tot(A),d,F)\in\sfc$, where $A = \{A^j_i \}$ is a bigraded $R$-module. For all $m ≥ 0$, let
%$d_m : A → A$ be the morphism of bidegree $(m,1-m)$ defined by 
%\[d_m(a) = (−1)^{nm}d(a)_{i−m},\] 
%CHECK
%where $a ∈ A^{n-i}_i$ and $d(a)_k$ denotes the $k$-th component of $d(a)$, which lies in $A^{n+1-k}_k$. Since $d$ is compatible
%with the filtration $F$, we have $d_i = 0$ for $i < 0$ CHECK. Then $(A, d_m)$ is a twisted complex and its filtered
%total complex is $(\Tot(A), d, F)$. Lastly, let $f : (\Tot(A), d, F) → (\Tot(B), d, F)$ be a morphism of split
%filtered complexes. For all $m ≥ 0$, let $f_m : A → B$ be the morphism of bidegree $(m,−m)$ defined by
%\[f_m(a) = (−1)^{nm}f(a)_{i−m},\] SAME AS D where $a ∈ A^{n-i}_i$ and $f(a)_k$ denotes the $k$-th component of $f(a)$, which lies
%in $B^{n-k}_k$. Since $f$ is compatible with the filtration $F$, we have that $f_i = 0$ for $i < 0$  SAME AS D. Then the family
%$\{f_m\}_{m≥0}$ is a morphism of twisted complexes whose total morphism is $f$.

We will consider the following bounded categories since the totalization functor has better properties when restricted to them. 
%I THINK I WON'T NEED TO SPECIFY THE SPLIT CATEGORIES
\begin{defin}
We let $\tc^b$, $\vbc^b$, $\bgmod^b$ be the full subcategories of \emph{horizontally bounded on the right} graded twisted
complexes, vertical bicomplexes and bigraded modules respectively. This means that if $A=\{A^j_i\}$ is an object of any of this categories, then there exists $i$ such that $A^j_{i'}=0$ for $i'>i$.

We let $\fmod^b$, $\fc^b$ be the full subcategories of bounded filtered modules, respectively complexes, i.e.
the full subcategories of objects $(K, F)$ such that there exists some $p$ with the property that $F_{p'}K^n = 0$ for all $p> p'$. We refer to all of these as the bounded subcategories of $\tc$, $\vbc$, $\bgmod$, $\fmod$ and $\fc$   respectively.
\end{defin}

\begin{propo}\label{monoidal}
The functors $\Tot : \bgmod → \fmod$ and $\Tot : \tc → \fc$ are lax symmetric
monoidal, with structure maps
\[\epsilon : R → \Tot(R)\text{ and }\mu=μ_{A,B} : \Tot(A) ⊗ \Tot(B) → \Tot(A ⊗ B),\]
given by $\epsilon = 1_R$ and for $x = (x_i)_i ∈ \Tot(A)^{n_1}$ and  $y=(y_j)_j ∈ \Tot(B)^{n_2}$,
\begin{equation}\label{mu1}
μ(x ⊗ y)_k \coloneqq
\sum_{k_1+k_2=k}(−1)^{k_1n_2}x_{k_1} ⊗ y_{k_2} .
\end{equation}

When restricted to the bounded case, $\Tot : \bgmod^b
 → \fmod^b$ and $\Tot : \tc^b → \fc^b$ are
strong symmetric monoidal functors.
\end{propo}
\begin{proof}
See \cite[Proposition 3.11]{whitehouse}.
\end{proof}

\begin{remark}\label{heuristic}
There is a certain heuristic to obtain the sign appearing in the definition of $\mu$ in \Cref{monoidal}. In the bounded case, we can write \[\Tot(A)=\bigoplus_i A_i^{n-i}\]
and direct sum conmutes with tensor products. Therefore, we have
\[\Tot(A)\otimes\Tot(B)=(\bigoplus A_i^{n-i})\otimes \Tot(B)\cong \bigoplus_i  (A_i^{n-i}\otimes \Tot(B)).\]

In the isomorphism we can interpret that each $A_i^{n-i}$ passes by $\Tot(B)$. Since $\Tot(B)$ is total graded, we can think of this degree as being the horizontal degree, while having 0 vertical degree. Thus, using the Koszul sign rule we would get precisely the sign from \Cref{monoidal}. This explanation is just an intuition, and opens the door for other possible sign choices: what if we decide to distribute $\Tot(A)$ over $\bigoplus_i B_i^{n-i}$ instead, or if we consider the total degree as the vertical degree? These alternatives lead to valid definitions of $\mu$, and we will explore the consequences of some of them in \Cref{othermu}.
\end{remark}

\begin{lem}\label{mui}
In the conditions of \Cref{monoidal} for the bounded case, the inverse
\[\mu^{-1}:\Tot(A_{(1)}\otimes\cdots\otimes A_{(m)})\to \Tot(A_{(1)})\otimes\cdots\otimes \Tot(A_{(m)})\]
is given on pure tensors (for notational convenience) as
\begin{equation}\label{mu}
\mu^{-1}(x_{(1)}\otimes\cdots\otimes x_{(m)})=(-1)^{\sum_{j=2}^m n_j\sum_{i=1}^{j-1}k_i}x_{(1)}\otimes\cdots\otimes x_{(m)},
\end{equation}
where $x_{(l)}\in (A_{(m)})_{k_l}^{n_l-k_l}$.
\end{lem}
\begin{proof}
For the case $m=2$,
\[\mu^{-1}:\Tot(A\otimes B)\to \Tot(A)\otimes \Tot(B)\]
is computed explicitly as follows.
Let  $c\in\Tot(A\otimes B)^n$. By definition, we have
\[\Tot(A\otimes B)^n=\bigoplus_k (A\otimes B)^{n-k}_k=\bigoplus_k\bigoplus_{n_1+n_2=n}\bigoplus_{k_1+k_2=k}A_{k_1}^{n_1-k_1}\otimes B_{k_2}^{n_2-k_2}\]
And thus $c=(c_k)_k$ may be written as a finite sum $c=\sum_k c_k$, where 
\[c_k=\sum_{n_1+n_2=n}\sum_{k_1+k_2=k}x_{k_1}^{n_1-k_1}\otimes y_{k_2}^{n_2-k_2}.\]
Here we introduced superscripts to indicate the vertical degree, which unlike in the definition of $\mu$ (\Cref{mu1}), it is not solely determined by the horizontal degree, since the total degree also varies. Distributibity allows us to rewrite $c$ as
\[c=\sum_k \sum_{n_1+n_2=n}\sum_{k_1+k_2=k}x_{k_1}\otimes y_{k_2}=\sum_{n_1+n_2=n}\sum_{k_1}\sum_{k_2}(x_{k_1}\otimes y_{k_2})=\sum_{n_1+n_2=n}\left(\sum_{k_1}x_{k_1}\right)\otimes\left(\sum_{k_2}y_{k_2}\right).\]
Therefore, $\mu^{-1}$ can be defined as
\[\mu^{-1}(c)=\sum_{n_1+n_2=n}\left(\sum_{k_1}(-1)^{k_1n_2}x_{k_1}\right)\otimes\left(\sum_{k_2}y_{k_2}\right).\]

The general case follows inductively.
\end{proof}

\subsubsection{Enriched categories and enriched totalization}
\subsubsection*{Monoidal categories over  a base}
%NOT SURE IF I'M GOING TO NEED SO MUCH GENERAL BACKGROUND (CERTAINLY DO NOT INCLUDE PROOFS) I WOULD LIKE TO INCLUDE THE PROOF OF THE INVERSE INDUCING THE INVERSE TRANSFORMATION


%I MAY NOT  NEED MUCH (BUT I DO MENTION ENRICHED CATEGORIES, I COULD TRY TO ONLY SHOW RESULTS FOR PARTICULAR CATEGORIES) 

%WHAT I REALLY NEED IS THEE DEFINITION OF ENRICHED MU AND I CAN DEFINE IT EXPLICITELY IN THE LEMMA THAT IT IS USED (OR WITH AN EXTRA LEMMA IN WHICH I ALSO SHOW THE INVERSE) USING LEMMA 4.35

We collect some notions and results about enriched categories from \cite{riehl} and \cite[\S 4.2]{whitehouse} that we will need as a categorical setting for our results on derived $A_\infty$-algebras.

\begin{defin}
Let $(\VV ,⊗, 1)$ be a symmetric monoidal category and let $(\CC,⊗, 1)$ be a monoidal category. We say that $\CC$ is a monoidal category over $\VV$ if we have an external tensor product $∗ :\VV × \CC → \CC$ such that we have natural isomorphisms:
\begin{enumerate}[$\bullet$]
\item  $1 ∗ X \cong X$ for all $X ∈ \CC$,
\item $(C ⊗ D) ∗ X \cong C ∗ (D ∗ X)$ for all $C,D ∈ \VV$ and $X ∈ \CC$,
\item $C ∗ (X ⊗ Y ) \cong (C ∗ X) ⊗ Y \cong X ⊗ (C ∗ Y )$ for all $C ∈ \VV$ and $X, Y ∈ \CC$.
\end{enumerate}
\end{defin}
\begin{remark}\label{underline}
We will also assume that there is a bifunctor $\uC(−,−) : \CC^{op} × \CC → \VV$ such that we have natural
bijections
\[\Hom_\CC(C ∗ X, Y ) \cong \Hom_\VV (C,\uC(X, Y )).\]
Under this assumption we get a $\VV$-enriched category $\uC$ with the same objects as $\CC$ and with hom-objects given by $\uC (−,−)$. The unit
morphism $u_A : 1 → \uC (A,A)$ corresponds to the identity map in $\CC$ under the adjunction and the
composition morphism is given by the adjoint of the composite
\[(\uC (B,C) ⊗ \uC (A,B)) ∗ A
\cong \uC (B,C) ∗ (\uC (A,B) ∗ A)
\xrightarrow{id∗ev_{AB}}
\uC (B,C) ∗ B
\xrightarrow{ev_{BC}} C,\]
where $ev_{AB}$ is the adjoint of the identity $\uC (A,B) → \uC (A,B)$. Furthermore, $\uC$ is a monoidal $\VV$-enriched category, namely we have an
enriched functor
\[\underline{⊗} : \uC × \uC → \uC\]
where $\uC × \uC$ is the enriched category with objects $\mathrm{Ob}(\CC ) × \mathrm{Ob}(\CC )$ and hom-objects
\[\uC × \uC ((X, Y ), (W,Z)) \coloneqq \uC (X,W) ⊗ \uC (Y,Z).\]
In particular we get maps in $\VV$
\[\uC (X,W) ⊗ \uC (Y,Z) → \uC (X ⊗ Y,W ⊗ Z),\]
given by the adjoint of the composite
\[(\uC (X,W) ⊗ \uC (Y,Z)) ∗ (X ⊗ Y )\cong (\uC (X,W) ∗ X) ⊗ (\uC (Y,Z) ∗ Y )
\xrightarrow{ev_{XW}⊗ev_{Y Z}} W ⊗ Z\]
\end{remark}

\begin{defin}
Let $\CC$ and $\DD$ be monoidal categories over $\VV$. A \emph{lax functor over $\VV$} consists of a functor $F : \CC → \DD$ together with a natural transformation \[ν_F : − ∗_\DD F(−) ⇒ F(− ∗_\CC −)\]
which is associative and unital with respect to the monoidal structures over $\VV$ of $\CC$ and $\DD$. (See \cite[Proposition 10.1.5]{riehl} for explicit diagrams stating the coherence axioms.) If $ν_F$ is a natural isomorphism
we say $F$ is a \emph{functor over $\VV$}.
Let $F,G : \CC → \DD$ be lax functors over $\VV$. A \emph{natural transformation over $\VV$} is a natural transformation
$μ : F ⇒ G$ such that for any $C ∈ \VV$ and for any $X ∈ \CC$ we have
\[ν_G \circ (1 ∗_\DD μ_X) = μ_C∗_\CC X \circ ν_F .\]
A \emph{(lax) monoidal functor over $\VV$} is a triple $(F, \epsilon, μ)$, where $F : \CC → \DD$ is a lax functor over $\VV$,
$\epsilon : 1_\DD → F(1_\CC)$ is a morphism in $\DD$ and
\[μ : F(−) ⊗ F(−) ⇒ F(− ⊗ −)\]
is a natural transformation over $\VV$ satisfying the standard unit and associativity conditions. If $ν_F$
and $μ$ are natural isomorphisms then we say that $F$ is \emph{monoidal over $\VV$}. 
\end{defin}

\begin{propo}\label{enrichedtrans}
Let $F,G : \CC → \DD$ be lax functors over $\VV$. Then $F$ and $G$ extend to $\VV$-enriched
functors
\[\underline{F},\underline{G} : \uC → \uD\]
where $\uC$ and $\uD$ denote the $\VV$-enriched categories corresponding to $\CC$ and $\DD$ as described in \Cref{underline}. Moreover, any natural transformation $μ : F ⇒ G$ over $\VV$ also extends to a $\VV$-enriched natural
transformation
\[\underline{μ} : \underline{F} ⇒ \underline{G}.\]
In particular, if $F$ is (lax) monoidal over $\VV$, then $F$ is (lax) monoidal in the enriched sense.
\end{propo}
\begin{proof}
See \cite[Proposition 4.11]{whitehouse}.
\end{proof}
\begin{lem}
Let $F,G:\CC\to\DD$ lax functors over $\VV$ and let $\mu : F\Rightarrow G$ a natural transformation over $\VV$. For every $X\in\CC$ and $Y\in\DD$ there is a map \[\uD(GX,Y)\to\uD(FX,Y)\] that is an isomorphism if $\mu$ is an isomorphism.
\end{lem}
\begin{proof}
By \Cref{enrichedtrans} there is a $\VV$-enriched natural transformation 
\[\underline{\mu}:\underline{F}\to\underline{G}\]
whose component \[\underline{\mu}_X:1\to\uD(FX,GX)\] is defined to be the adjoint of $\mu_X:FX\to GX$. The map $\uD(GX,Y)\to\uD(FX,Y)$ is defined as the composite

\begin{equation}\label{enrichedmap}
\uD(GX,Y)\cong\uD(GX,Y)\otimes 1\xrightarrow{1\otimes\umu_X}\uD(GX,Y)\otimes\uD(FX,GX)\xrightarrow{c}\uD(FX,Y)
\end{equation}
where $c$ is the composition map in the enriched setting. 

When $\mu$ is an isomorphism we may analogously define the following map

\[\uD(FX,Y)\cong\uD(FX,Y)\otimes 1\xrightarrow{1\otimes\umui_X}\uD(FX,Y)\otimes\uD(GX,FX)\xrightarrow{c}\uD(GX,Y).\]

We show that this map is the inverse of the map in \Cref{enrichedmap}.

\[
\adjustbox{scale=0.76,center}{%
\begin{tikzcd}[column sep = 0pt, row sep = 20pt]
{\uD(GX,Y) } \arrow[r, "\cong"] \arrow[rd, "(5)", phantom, bend left = 7]                 & {\uD(GX,Y)\otimes 1} \arrow[r, "1\otimes\umu_X"] \arrow[d, "1\otimes\alpha_X"] \arrow[rd, "(4)", phantom] & {\uD(GX,Y)\otimes\uD(FX,GX)} \arrow[r, "c"] \arrow[d, "\cong"]                                                                     & {\uD(FX,Y)} \arrow[ddd, "\cong"]                                               \\
                                                                                     & {\uD(GX,Y)\otimes\uD(GX,GX)} \arrow[lu, "c"] \arrow[lu]                                                  & {\uD(GX,Y)\otimes\uD(FX,GX)\otimes 1} \arrow[ld, "1\otimes 1\otimes \umui_X"] \arrow[rdd, "c\otimes 1"] \arrow[ru, "(1)"', phantom] &                                                                                \\
                                                                                     & {\uD(GX,Y)\otimes\uD(FX,GX)\otimes \uD(GX,FX)} \arrow[u, "1\otimes c"] \arrow[ld, "c\otimes 1"]          &                                                                                                                                    &                                                                                \\
{\uD(FX,Y)\otimes\uD(GX,FX)} \arrow[uuu, "c"] \arrow[ruu, "(3)"', phantom, bend left] & {}                                                                                                       &                                                                                                                                    & {\uD(FX,Y)\otimes 1} \arrow[lll, "1\otimes\umui_X"] \arrow[llu, "(2)"', phantom]
\end{tikzcd}
}
\]

%
%\[
%\adjustbox{scale=1,center}{%
%\begin{tikzcd}[column sep = 1pt]
%{\uD(GX,Y) } \arrow[r, "\cong"] \arrow[rd, "(5)", phantom, bend left = 7]                 & {\uD(GX,Y)\otimes 1} \arrow[r, "1\otimes\umu_X"] \arrow[d, "1\otimes\alpha_X"] \arrow[rd, "(4)", phantom] & {\uD(GX,Y)\otimes\uD(FX,GX)} \arrow[r, "c"] \arrow[d, "\cong"]                                                                     & {\uD(FX,Y)} \arrow[ddd, "\cong"]                                               \\
%                                                                                     & {\uD(GX,Y)\otimes\uD(GX,GX)} \arrow[lu, "c"] \arrow[lu]                                                  & {\uD(GX,Y)\otimes\uD(FX,GX)\otimes 1} \arrow[ld, "1\otimes 1\otimes \umui_X"] \arrow[rdd, "c\otimes 1"] \arrow[ru, "(1)"', phantom] &                                                                                \\
%                                                                                     & {\uD(GX,Y)\otimes\uD(FX,GX)\otimes \uD(GX,FX)} \arrow[u, "1\otimes c"] \arrow[ld, "c\otimes 1"]          &                                                                                                                                    &                                                                                \\
%{\uD(FX,Y)\otimes\uD(GX,FX)} \arrow[uuu, "c"] \arrow[ruu, "(3)"', phantom, bend left] & {}                                                                                                       &                                                                                                                                    & {\uD(FX,Y)\otimes 1} \arrow[lll, "1\otimes\umui_X"] \arrow[llu, "(2)"', phantom]
%\end{tikzcd}
%}
%\]

In the diagram, $\alpha_X$ is adjoint to $1_{GX}:GX\to GX$. Diagrams (1) and (2) clearly commute. Diagram (3) commutes by associativity of $c$. Diagram (4) commutes because $\umui_X$ and $\umu_X$ are adjoint to mutual inverses, so their composition results in the adjoint of the identity. Finally, diagram (5) commutes because we are composing with an identity map. 
\end{proof}

\begin{lem}\label{4.15}
The category $\fc$ is monoidal over $\vbc$. By restriction, $\fmod$ is monoidal over $\bgmod$.
\end{lem}
\begin{proof}
See \cite[Lemma 4.15]{whitehouse} for the proof and more details.
\end{proof}

\subsubsection*{Enriched categories and totalization}

We define here some useful enriched categories and results from \cite[\S 4.3 and 4.4]{whitehouse}. Some of them had to be modified to adjust them to our conventions. 
\begin{defin}\label{weirdenrichment}
Let $A,B,C$ be bigraded modules. We denote by $\underline{\mathpzc{bgMod}_R}(A,B)$ the bigraded module given by
\[\underline{\mathpzc{bgMod}_R}(A,B)^v_u :=\prod_{j≥0}[A,B]^{v−j}_{u+j}\]
where $[A,B]$ is the inner hom-object of bigraded modules. More precisely, $g ∈ \underline{\mathpzc{bgMod}_R}(A,B)^v_u$ is given
by $g := (g_0, g_1, g_2, \dots )$, where $g_j : A → B$ is a map of bigraded modules of bidegree $(u + j, v − j)$.

Moreover, we define a composition morphism
\[c : \underline{\mathpzc{bgMod}_R}(B,C) ⊗ \underline{\mathpzc{bgMod}_R}(A,B) → \underline{\mathpzc{bgMod}_R}(A,C)\]
by
\[c(f, g)_m :=\sum_{i+j=m}(−1)^{i|g|}f_ig_j .\]
\end{defin}

\begin{defin}\label{delta2}
Let $(A, d^A_i), (B, d^B_i)$ be twisted complexes, $f ∈ \underline{\mathpzc{bgMod}_R}(A,B)^v_u$ and consider $d^A :=(d^A_i)_i ∈ \underline{\mathpzc{bgMod}_R}(A,A)^1_0$
and $d^B := (d^B_i)_i ∈ \underline{\mathpzc{bgMod}_R}(B,B)^1_0$. We define
\[δ(f) := c(d^B, f) − (−1)^{\langle f,d^A\rangle}c(f, d^A) ∈ \underline{\mathpzc{bgMod}_R}(A,B)^{v+1}_u\]
where $\langle f, d^A\rangle$ is the scalar product for the bidegrees and $c$ is the composition morphism described in \Cref{weirdenrichment} More precisely,
\[(δ(f))_m :=\sum_{i+j=m}(−1)^{i|f|}d^B_if_j − (−1)^{v+i}f_id^A_j.\]
\end{defin}

The following lemma justifies the above definition. For a proof see \cite[Lemma 4.18]{whitehouse}.

\begin{lem}
The following equations hold
\begin{align*}
&c(d^A, d^A) = 0,\\
&δ^2 = 0,\\
&δ(c(f, g)) = c(δ(f), g) + (−1)^v c(f, δ(g)),
\end{align*}
where the bidegree of $f$ is $(u, v)$. Furthermore, $f ∈ \ubgMod(A,B)$ is a map of twisted complexes if and
only if $δ(f) = 0$. In particular, $f$ is a morphism in $\tc$ if and only if the bidegree of $f$ is $(0, 0)$ and
$δ(f) = 0$. Moreover, for $f$, $g$ morphisms in $\tc$, we have that $c(f, g) = f\circ g$, where the latter denotes
composition in $\tc$.
\end{lem}

\begin{defin}
For $A,B$ twisted complexes, we define $\underline{t\mathcal{C}_R}(A,B)$ to be the vertical bicomplex
$\underline{t\mathcal{C}_R}(A,B) := (\underline{\mathpzc{bgMod}_R}(A,B), δ)$.
\end{defin}

\begin{defin}\label{ubgMod}
We denote by $\ubgMod$ the $\bgmod$-enriched category of bigraded modules given
by the following data.

\begin{enumerate}[(1)]
\item The objects of $\ubgMod$ are bigraded modules.
\item For $A,B$ bigraded modules the hom-object is the bigraded module $\ubgMod(A,B)$.
\item The composition morphism $c : \ubgMod(B,C) ⊗ \ubgMod(A,B) → \ubgMod(A,C)$ is given by \Cref{weirdenrichment}.
\item The unit morphism $R → \ubgMod(A,A)$ is given by the morphism of bigraded modules that
sends $1 ∈ R$ to $1_A : A → A$, the strict morphism given by the identity of $A$.
\end{enumerate}
\end{defin}

\begin{defin}\label{utC}
The $\vbc$-enriched category of twisted complexes $\utC$ is the enriched category given by the following data.
\begin{enumerate}[(1)]
\item The objects of $\utC$ are twisted complexes.
\item For $A,B$ twisted complexes the hom-object is the vertical bicomplex $\utC(A,B)$.
\item The composition morphism $c : \utC(B,C)⊗\utC(A,B) → \utC(A,C)$ is given by \Cref{weirdenrichment}.
\item The unit morphism $R → \utC(A,A)$ is given by the morphism of vertical bicomplexes sending
$1 ∈ R$ to $1_A : A → A$, the strict morphism of twisted complexes given by the identity of $A$.
\end{enumerate}
\end{defin}




The next tensor corresponds to $\underline{\otimes}$ in the categorical setting of \Cref{underline}.


\begin{lem}\label{tensorenriched}
The monoidal structure of $\utC$ is given by the following map of vertical bicomplexes.
\[\underline{⊗}: \utC(A,B) ⊗ \utC(A′,B′) → \utC(A ⊗ A′,B ⊗ B′)\]
\[(f, g) → (f\underline{⊗}g)_m :=\sum_{i+j=m}(−1)^{ij}f_i ⊗ g_j\]
The monoidal structure of $\ubgMod$ is given by the restriction of this map.
\end{lem}
\begin{proof}
See \cite[Lemma 4.27]{whitehouse}
\end{proof}



\begin{defin}\label{ufMod}
The $\bgmod$-enriched category of filtered modules $\ufMod$ is the enriched category given by the following data.
%I HAVE TO KEEP J+U SO THAT THE DEGREES MATCH IN ENRICH TOT (ALTERNATIVELY KEEP V-U BUT I DON'T LIKE THAT)
\begin{enumerate}[(1)]
\item The objects of $\ufMod$ are filtered modules.
\item For filtered modules $(K, F)$ and $(L, F)$, the bigraded module $\ufMod(K,L)$ is given by
\[\ufMod(K,L)^v_u :=\{f : K → L\mid f(F_qK^m) ⊂ F_{q+u}L^{m+u+v}, ∀m, q ∈ \Z\}.\]
\item The composition morphism is given by $c(f, g) = (−1)^{u|g|}fg$, where $f$ has bidegree $(u, v)$.
\item The unit morphism is given by the map $R → \ufMod(K,K)$ given by $1 → 1_K$.
\end{enumerate}
\end{defin}


\begin{defin}\label{fmoddifferential}
Let $(K, d^K, F)$ and $(L, d^L, F)$ be filtered complexes. We define $\ufC(K,L)$ to be the
vertical bicomplex whose underlying bigraded module is $\ufMod(K,L)$ with vertical differential
\[δ(f) := c(d^L, f) − (−1)^{\langle f,d^K\rangle}c(f, d^K) = d^Lf − (−1)^{v+u}fd^K = d^Lf − (−1)^{|f|}fd^K\]
for $f ∈ \ufMod(K,L)^v_u$, where $c$ is the composition map from \Cref{ufMod}.
\end{defin}


\begin{defin}\label{ufC}
The $\vbc$-enriched category of filtered complexes $\ufC$ is the enriched category given
by the following data.
\begin{enumerate}[(1)]
\item The objects of $\ufC$ are filtered complexes.
\item For $K,L$ filtered complexes the hom-object is the vertical bicomplex $\ufC(K,L)$.
\item The composition morphism is given as in $\ufMod$ in \Cref{ufMod}. 
\item The unit morphism is given by the map $R → \ufC(K,K)$ given by $1 → 1_K$.
We denote by $\usfC$ the full subcategory of $\ufC$ whose objects are split filtered complexes.

\end{enumerate}
\end{defin}

The enriched monoidal structure is given by the following lemma.
\begin{defin}\label{tensorenriched2}
The monoidal structure of $\ufC$ is given by the following map of vertical bicomplexes.
\[\underline{⊗}: \ufC(K,L) ⊗ \ufC(K′,L′) → \ufC(K ⊗ K′,L ⊗ L′),\]
\[(f, g) → f\underline{⊗}g := (−1)^{u|g|}f ⊗ g\]
where $f$ has bidegree $(u, v)$.
\end{defin}
\begin{proof}
See \cite[Lemma 4.36]{whitehouse}.
\end{proof}


\begin{lem}\label{adjunction}
Let $A$ be a vertical bicomplex that is horizontally bounded on the right and let $K$ and $L$ be filtered complexes. There is a natural bijection
\[\Hom_{\fc}(\Tot(A)\otimes K,L)\cong \Hom_{\vbc}(A,\ufC(K,L))\]
given by
\[f\mapsto \tilde{f}: a\mapsto (k\mapsto f(a\otimes k)).\]
\end{lem}
\begin{proof}
The proof is included in the proof of \cite[Lemma 4.35]{whitehouse}.
\end{proof}
We now define an enriched version of the totalization functor. %CHECK TOT, OR JUST OMIT IT BECAUSE I DON'T NEED IT
\begin{defin}\label{enrichedtot}
Let $A,B$ be bigraded modules and $f ∈ \ubgMod (A,B)^v_u$ we define

\[\Tot(f) ∈ \ufMod(\Tot(A),\Tot(B))^v_u\]
to be given on any $x ∈ \Tot(A)^n$ by
\[(\Tot(f)(x)))_{j+u} :=
\sum_{m≥0}(−1)^{(m+u)n}f_m(x_{j-m}) ∈ B^{n-j+v}_{j+u} ⊂ \Tot(B)^{n+u+v}.\]
Let $K = \Tot(A)$, $L = \Tot(B)$ and $g ∈ \ufMod(K,L)^v_u$ we define
\[f := \Tot^{−1}(g) ∈ \ubgMod(A,B)^v_u\]
to be $f := (f_0, f_1,\dots)$ where $f_i$ is given on each $A^{m+j}_j$ by the composite
\begin{align*}
f_i : A^{m-j}_j \hookrightarrow\prod_{k\geq j}A^{m-k}_k = F_j(\Tot(A)^m)\xrightarrow{g}&F_{j+u}(\Tot(B)^{m+u+v})\\
&=\prod_{l\geq j+u}B^{m+u+v-l}_l\xrightarrow{×(−1)^{(i+u)m}} B^{m-j+v−i}_{j+u+i} ,
\end{align*}
where the last map is a projection and multiplication with the indicated sign.

%I HAVE TO KEEP J+U SO THAT THE DEGREES MATCH  (ALTERNATIVELY KEEP V-U BUT I DON'T LIKE THAT)
\end{defin} 

\begin{thm}\label{4.39}
Let $A$, $B$ be twisted complexes. The assignments $\mathfrak{Tot}(A) := \Tot(A)$ and
\begin{align*}
\mathfrak{Tot}_{A,B} : \utC(A,B)& → \ufC(\Tot(A),\Tot(B))\\
f &→ \Tot(f)
\end{align*}
define a $\vbc$-enriched functor $\mathfrak{Tot} : \utC → \ufC$ which restricts to an isomorphism onto its image. Furthermore, this functor restricts to a $\bgmod$-enriched functor \[\mathfrak{Tot} : \ubgMod → \ufMod\]
 which also restricts to an isomorphism onto its image.
\end{thm}
\begin{proof}
See \cite[Theorem 4.39]{whitehouse}.
\end{proof}

\begin{propo}\label{4.40}
The enriched functors
\[\mathfrak{Tot} : \ubgMod  → \ufMod ,\hspace{1cm} \mathfrak{Tot} : \utC → \ufC\]
are lax symmetric monoidal in the enriched sense and when restricted to the bounded case they are strong symmetric monoidal in the enriched sense.
\end{propo}
\begin{proof}
See \cite[Proposition 4.40]{whitehouse}.
\end{proof}

We now define an enriched endomorphism operad. %I THINK I ONLY NEED THIS DEFINITION AND NOT THE PREVIOUS OR THE NEXT LEMMA
\begin{defin}
Let $\underline{\mathscr{C}}$ be a monoidal $\mathscr{V}$-enriched category and $A$ an object of $\uC$. We define $\uEnd_A$
to be the collection in $\mathscr{V}$ given by
\[\uEnd_A(n) \coloneqq \uC (A^{⊗n},A) \text{ for }n ≥ 1.\]
\end{defin}
%AT SOME POINT DO SOME MENTION TO THE DIFFERENT CASES THAT I AM GOING TO USE BEECAUSE THE NOTATION IS GOING TO BE THE SSAME
\begin{lemma}
For $A\in\uC$, the collection $\uEnd_A$ defines an operad in $\VV$. 
\end{lemma}
\begin{proof}
See \cite[Proposition 4.43]{whitehouse}.
\end{proof}

\begin{propo}\label{morphism}
Let $\CC$ and $\DD$ be monoidal categories over $\VV$ . Let
$F : \CC → \DD$ be a lax monoidal functor over $\VV$ . Then for any $X ∈ \CC$ there is an operad morphism
\[\uEnd_X→\uEnd_{F(X)}.\]

\end{propo}
\begin{proof}
The proof is in \cite[Proposition 4.46]{whitehouse}. 
\end{proof}

\begin{lem}\label{inverse}
Let $A$ be a twisted complex. Consider $\uEnd_A(n)=\utC(A^{\otimes n},A)$ and $\uEnd_{\Tot(A)}(n)=\ufC(\Tot(A)^{\otimes n},\Tot(A))$. There is a morphism of operads
\[\uEnd_A →\uEnd_{\Tot(A)},\]
which is an isomorphism of operads if $A$ is bounded. The same holds true if $A$ is just a bigraded module. In that case, we use the enriched operads $\uEnd_A(n)=\ubgMod(A^{\otimes n},A)$ and $\uEnd_{\Tot(A)}(n)=\ufMod(\Tot(A)^{\otimes n},\Tot(A))$.
\end{lem}
\begin{proof}
The proof of in the case of a $A$ beig a twisted complex can be found in \cite[Lemma 4.54]{whitehouse}. For the bigraded module case, we are going to do it analogously. First, by \Cref{4.39} we know that $\mathfrak{Tot}:\ubgMod\to\ufMod$ is $\bgmod$-enriched. In fact, by \Cref{4.40} it is lax monoidal in the enriched sense. In addition, both $\bgmod$ and $\fmod$ are monoidal over $\bgmod$. In the case of $\bgmod$ it is in the obvious way and for $\fmod$ is given by \Cref{4.15}. With all of this we may apply \Cref{morphism} to $\mathfrak{Tot}:\ubgMod\to\ufMod$ to obtain the desired map
\[\uEnd_A →\uEnd_{\Tot(A)}.\]
 The fact that it is an isomorphism in the bounded is analogous to the twisted complex case. 
\end{proof}

We are going to construct the inverse in the bounded case explicitly followig \Cref{enrichedmap} (the construction for the direct map is analogue but here we just need the inverse). We do it for a twisted complex $A$, but it is done similarly for a bigraded module.

\begin{lem}\label{composition}
In the conditions of \Cref{inverse} for the bounded case, the inverse is given by the map
\begin{align*}
\uEnd_{\Tot(A)}&\to\uEnd_A\\
f & \mapsto \Tot^{-1}(f\circ \mu^{-1}).
\end{align*}
\end{lem}
\begin{proof}
The inverse is given by the composite
\[\uEnd_{\Tot(A)}(n)=\ufC(\Tot(A)^{\otimes n},\Tot(A))\to \ufC(\Tot(A^{\otimes n}),\Tot(A))\to\utC(A^{\otimes n},A)=\uEnd_A(n) \]

The second map is given by $\mathfrak{Tot}^{-1}$ defined in \Cref{enrichedtot}. To describe the first map, let $R$ be concentrated in bidegree $(0,0)$ with trivial vertical differential. Then the first map is given by the following composite
\begin{align*}
\ufC(\Tot(A)^{\otimes n},\Tot(A))\cong R\otimes\ufC(\Tot(A)^{\otimes n},\Tot(A))\\
\xrightarrow{\underline{\mu}^{-1}\otimes 1}\ufC(\Tot(A^{\otimes n}),\Tot(A)^{\otimes n})\otimes\ufC(\Tot(A)^{\otimes n},\Tot(A))\\
\xrightarrow{c}\ufC(\Tot(A^{\otimes n}),\Tot(A)), 
\end{align*}
where $c$ is the composition in $\ufC$, which is defined in \Cref{ufMod}. The map $\underline{\mu}^{-1}$ is the adjoint of $\mu^{-1}$ under the bijection from \Cref{adjunction}. Explicitly,
\begin{align*}
\underline{\mu}^{-1}:R &\to \ufC(\Tot(A^{\otimes n}),\Tot(A)^{\otimes n})\\
1 &\mapsto (a\mapsto \mu^{-1}(a)).
\end{align*}
Putting all together we get the map 
\begin{align*}
\uEnd_{\Tot(A)}&\to\uEnd_A\\
f & \mapsto \Tot^{-1}(c(f, \mu^{-1})).
\end{align*}
Since the total degree of $\mu^{-1}$ is 0, composition reduces to $c(f,\mu^{-1})=f\circ \mu^{-1}$ and we get the desired map.
\end{proof}





%ONLY MENTION 4.47 WHEN YOU REACH THE PROOF OF THE LONG ISO


\end{document}