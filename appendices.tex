\documentclass[join.tex]{subfiles}

\begin{appendices}
\appendix
\gdef\thesection{\Alph{section}}
\section{Some proofs and details}\label{AppA}

%MAYBE SOME OF THESE CAN BE INCLUDED IN THE MAIN BODY OF TH ARTICLE


\begin{lem}\label{binom}
For any integers $n$ and $m$, the following equality holds mod 2.

\[\binom{n+m-1}{2}+\binom{n}{2}+\binom{m}{2}=(n-1)(m-1).\]
\end{lem}
\begin{proof}
Let us compute 

$$\binom{n+m-1}{2}+\binom{n}{2}+\binom{m}{2}+(n-1)(m-1)\mod 2.$$

By definition, this equals

\begin{gather*}
\frac{(n+m-1)(n+m-2)}{2}+\frac{n(n-1)}{2}+\frac{m(m-1)}{2}+(n-1)(m-1)=\\
\frac{(n^2+2nm-2n+m^2-2m-n-m+2)+(n^2-n)+(m^2-m)+2(nm-n-m+1)}{2}=\\
n^2+2nm-3n+m^2-3m+2=n^2+m+m^2+m=0\mod 2
\end{gather*}
as wanted, because $n^2=n\mod 2$.


\end{proof}




Recall that we define the \emph{suspension} or \emph{shift} of a graded module $A$ as the graded module $S A$ having degree components $(S A)^i=A^{i-1}$.

\begin{theorem}\label{proofthm}
There is an isomorphism of (symmetric) operads $\End_{S A}\cong \mathfrak{s}^{-1}\End_A$.
\end{theorem}
\begin{proof}
For each $n$, we clearly have an isomorphism of graded modules

$$\End_{S A}(n)=\Hom_R((S A)^{\otimes n},S A)\cong\Hom_R(A^{\otimes n},A)\otimes S^{1-n}sig_n= \mathfrak{s}^{-1}\End_A(n)$$

given by the map $\sigma^{-1}$ defined before as \[\sigma^{-1}(F)=(-1)^{\binom{n}{2}}S^{-1}\circ F\circ S^{\otimes n},\] where $\circ$ denotes the composition of maps. We must show that this map is an isomorphism of operads, in other words, it commutes with insertions and with the symmetric group action.

Let us first check that $\sigma^{-1}$ commutes with insertions. For that, let $F\in \End_{S A}(n)$ and $G\in \End_{S A}(m)$. On the one had we have 

\[\sigma^{-1}(F\circ_i G)=(-1)^{\binom{n+m-1}{2}+\deg(G)(i-1)}S^{-1}\circ F(S^{\otimes i-1}\otimes G(S^{\otimes m})\otimes S^{\otimes n-i}),\]


and on the other hand

\[\sigma^{-1}(F)\tilde{\circ}_i\sigma^{-1}(G)=(-1)^{(n-1)(m-1)+(n-1)(\deg(G)+m-1)+(i-1)(m-1)}\sigma^{-1}(F)\circ_i\sigma^{-1}(G)=\]
\[(-1)^{\varepsilon}S^{-1}\circ F(S^{\otimes i-1}\otimes G(S^{\otimes m})\otimes S^{\otimes n-i}),\]
where
\[\varepsilon=\binom{n}{2}+\binom{m}{2}+(n-1)(m-1)+(n-1)(\deg(G)+m-1)+(i-1)(m-1)+(\deg(G)+m-1)(n-i)\]

By \Cref{binom}, 

\[\binom{n+m-1}{2}=\binom{n}{2}+\binom{m}{2}+(n-1)(m-1)\mod 2,\]

so we only need to check the equation

\[\deg(G)(i-1)=(n-1)(\deg(G)+m-1)+(i-1)(m-1)+(\deg(G)+m-1)(n-i)\mod 2.\]

This can be done by direct computation.

Now we are going to show that $\sigma^{-1}$ commutes with the action of the symmetric group. Recall that on $\End_{S A}$ we have the usual permutation action, whilst on $\mathfrak{s}^{-1}\End_A$ the action is twisted by the sign of the permutation. It is enough to show this for transpositions of the form $\tau=(i\ i+1)$ since they generate the symmetric group.

Let us write $(-1)^v$ for $(-1)^{\deg(v)}$. On the one hand, 

\[\sigma^{-1}(F\tau)(v_1\otimes\cdots\otimes v_n)=(-1)^{\sum_{j=1}^n (n-j)v_j}S^{-1}\circ (F\tau)(S v_1\otimes\cdots\otimes S v_n)=\]

\begin{equation}\label{firstmap}
(-1)^{\sum_{j=1}^n (n-j)v_j+(v_i-1)(v_{i+1}-1)}S^{-1}\circ F(S v_1\otimes\cdots\otimes S v_{i+1}\otimes S v_i\otimes\cdots\otimes S v_n).
\end{equation}

The sign $(-1)^{\sum_{j=1}^n (n-j)v_j}$ comes from swapping the shift maps $S$ past the $v_j$'s, and the sign $(-1)^{(v_i-1)(v_{i+1}-1)}$ comes from permuting $v_i$ and $v_{i+1}$. On the other hand, performing similar sign computations we have

\[(\sigma^{-1}(F)\tau) (v_1\otimes\cdots\otimes v_n)=(-1)^{v_iv_{i+1}-1}S^{-1}\circ F\circ S^{\otimes n}(v_1\otimes\cdots\otimes v_{i+1}\otimes v_i\otimes\cdots\otimes v_n)=\]

\begin{equation}\label{secondmap}
(-1)^{v_iv_{i+1}-1+\sum_{j\neq i,i+1}(n-j)v_j +(n-i-1)v_i+(n-i)v_{i+1}}S^{-1}\circ f(S v_1\otimes\cdots\otimes S v_{i+1}\otimes S v_i\otimes\cdots\otimes S v_n).
\end{equation}

Now we just have to check that the signs are the same. Modulo $2$, the sign on equation (\ref{firstmap}) is 

\[v_iv_{i+1}+v_i+v_{i+1}-1+\sum_{j=1}^n(n-j)v_j=\]
\[v_iv_{i+1}-1+\sum_{j\neq i,i+1}^n(n-j)v_j+(n-i-1)v_i+(n-i)v_{i+1},\]

which indeed coincides with the sign on equation (\ref{secondmap}).

%\url{https://mathoverflow.net/questions/366792/detailed-proof-of-mathfraks-1-mathrmend-v-cong-mathrmend-sigma-v}
\end{proof}

\begin{remark}
If in the proof above we replace $S$ with $S^{-1}$, we have that the map

\[\sigma^{-1}(F)=(-1)^{\binom{n}{2}}S^{-1}\circ F\circ S^{\otimes n}\]
 transforms into $(-1)^{\binom{n}{2}}S\circ F\circ (S^{-1})^{\otimes n}=S\circ F\circ (S^{\otimes n})^{-1}$. This is the map $\overline{\sigma}(F)$ from page 9 of \cite{RW}, and following the same proof we have done above but with this change of $S$ into $S^{-1}$ we get the isomorphism of operads

\[
\overline{\sigma}:\End_A\cong\s\End_{SA}.
\]
\end{remark}

%\subsection{Monoidality of operadic suspension}
%SOME BRIEF INTRO TO WHAT I'M DOING
%
%
%When studying the functorial properties of $\s$ it is natural to look for a natural transformation $\mu=\mu_{\OO,\PP}:\s\OO\circ\s\PP\to \s(\OO\circ\PP)$ PLETTHYSM such that the composition 
%\[\s\OO\circ\s\OO\xrightarrow{\mu_{\OO,\OO}} \s(\OO\circ\OO)\xrightarrow{\s\gamma} \s\OO\]
%is the composition $\tilde{\gamma}$ on $\s\OO$. Recall that $\tilde{\gamma}=(-1)^{\eta}\gamma$. Since the functor $\s$ does not add any signs to the above composition REFFERENCE TO THE DEFINITION OF THE FUNCTOR, the  sign $(-1)^{\eta}$ REFERENCE must come entirely from the natural a transformation $\mu$, and thus the definition of $\mu$ should extend this fact. WRITE IT EXPLICITLY FOR A SUMMAND OF THE PLETHYSM
%
%Recall the definition of the plethysm $\circ$
%
%The most natural way to define a map $\s\OO\circ\s\PP\to\s(\OO\circ\PP)$ is by reordering the factors using the symmetry morphism and apply the operadic composition on $\Lambda$, just like we did to obtain the sign $\eta$ REFERENCE TO THE PROOF. Therefore, following that process we obtain the map
%\[\mu:\s\OO\circ\s\PP\to\s(\OO\circ\PP)\]
%which acts on elements of each direct summannd as
%\[\mu(c\otimes e^k\otimes c_1\otimes e^{a_1}\otimes\cdots\otimes c_k\otimes e^{a_k})=(-1)^{\sum_{j<l}a_jq_l+\sum_{j=1}^k (a_j+q_j-1)(k-j)}c\otimes c_1\otimes\cdots\otimes c_k \otimes e^{a_1+\cdots+a_k},\]
%where $q_j$ is the degree of $c_j$ in $\PP$. Since the sign in the above equation is precisely $\eta$ in the case where $k_0=\cdots=k_n=0$, we will simply call it $\eta$. 
%
%Since we know that $\mu_{\OO,\OO}$ induces the operadic composition on $\s\OO$ REFERENCE TO THE EQUATION ABOVE WITH THE COMPOSITION, we know that $\mu$ sends monoids on CATEGORY OF COLLECTIONSS WITH PLETHYSIM  to SAME THING. It is possible to show that $\mu$ is indeed natural and in addition it satisfies the unitality axiom of monoidal functors RECALL UNIT IN THIS CATEGORY (I THINK IT WAS 0 EXCEPT K IN ARITY 1) I SHOULD ACTUALLY CHECK PROPERLY THAT THE AXIOM IS SATISFIED AS WELL AS NATURALIITY AND WRITE IT DOWN SOMEWHERE (NATURALITY IS IN MY NOTEBOOK)
%
%FOR UNITALITY I THINK 1$\to$ S1 IS A MORPHISM BECAUSE COMPOSITION IS AALWAYS 0
%
%CHECK VALUE OOF ETA IN THE UNITALITY AXIOMS (IN THE FIRST AXIOM ETA IS ZERRO BECAUSE IT IS THE CASE $n=1$ AND ALL K'S=0 AND THE  SECOND AXIOM IT IS THE CASE WELL ALL OF THEM ARE ARITY 1 AND DEGREE 0 SO ALSO VANISH)
%\url{https://ncatlab.org/nlab/show/monoidal+functor}
%


\section{Koszul sign on operadic suspension}
The purpose of this section is to clear up the procedure to apply the Koszul sign rule in situations in which operadic suspension is involved.

Let $\End_A$ be the endomorphism operad of some $R$-module $A$ and consider the operadic suspensión $\s\End_A$. We are going to make a few comments on the application of the Koszul rule when applying maps from $\s\End_A(n)$ to elements of $A^{\otimes n}$. Let $f\otimes e^n\in\s\End_A(n)$  of degree $\deg(f)+n-1$. %(do not be confuse with the notation that we used in Section \ref{functorial}, here $\s f$ is an element of an operad). 
For $a\in A^{\otimes n}$ we have \[(f\otimes e^n)(a)=(-1)^{\deg(a)(n-1)}f(a)\otimes e^n\]

because $\deg(e^n)=n-1$. Note that $f\otimes e^n=g\otimes e^n$ if and only if $f=g$. In addition, it is not possible that $f\otimes e^n=g\otimes e^m$ for $n\neq m$. %the 0  map is a different map oon arity n or m
The reader may notice that $f(a)\otimes  e^n\notin A$, but it can be identified with an element of $S^{n-1}A$. This is a reminiscense of the isomorphism $\s^{-1}\End_A\cong \End_{SA}$. %A map of degree d on SA^n->SA corresponds to a map of degree d-n+1 on A^n->A 
 

If we take the tensor product of $f\otimes e^n\in\s\End_A(n)$ and $g\otimes e^m\in\s\End_A(m)$ and apply it to $a\otimes b\in A^{\otimes n}\otimes A^{\otimes m}$, we have

\begin{align*}
((f\otimes e^n)\otimes ( g\otimes e^m))(a\otimes b)=&(-1)^{\deg(a)(\deg(g)+m-1)}(f\otimes e^n)(a)\otimes( g\otimes e^m)(b)\\
=&(-1)^{\deg(a)(\deg(g)+m-1)+\deg(a)(n-1)+\deg(b)(m-1)}(f(a)\otimes e^n)\otimes(f(b)\otimes e^m).
\end{align*}

The last remark that we want to make is the case of a map of the form 
\[f(1^{\otimes k-1}\otimes g\otimes 1^{\otimes n-k})\otimes e^{m+n-1}\in\s\End_A(n+m-1),\] 
such us those produced by operadic the insertion $\s f\tilde{\circ}_{k} \s g$. In this case, this map applied to $a_{k-1}\otimes b\otimes a_{n-k}\in A^{\otimes k-1}\otimes A^{\otimes m}\otimes A
^{\otimes n-k}$ results in

\begin{gather*}
(f(1^{\otimes k-1}\otimes g\otimes 1^{\otimes n-k})\otimes e^{m+n-1})(a_{k-1}\otimes b\otimes a_{n-k})=\\
(-1)^{(m+n)(\deg(a_{k-1})+\deg(b)+\deg(a_{n-k}))}(f(1^{\otimes k-1}\otimes g\otimes 1^{\otimes n-k}(a_{k-1}\otimes b\otimes a_{n-k}))\otimes e^{m+n-1}=\\
(-1)^{(m+n)(\deg(a_{k-1})+\deg(b)+\deg(a_{n-k}))+\deg(a_{k-1})\deg(g)}f(a_{k-1}\otimes g(b)\otimes a_{n-k})\otimes e^{m+n-1}.
\end{gather*}
To go from the first line to the second we switch $e^{m+n-1}$ of degree $m+n-2$  with $a_{k-1}\otimes b\otimes a_{n-k}$. To go from the second line to the third we apply the usual sign rule for graded maps.

The purpose of this last remark is not only review the Koszul sign rule but also remind that the insertion $\s f\tilde{\circ}_{k} \s g$ is of the above form, so that the $e^{m+n-1}$ component is always at the end and does not play a role in the application of the sign rule with the composed maps. In other words, it does not affect their individual degrees, just the degree of the overall composition. %I do this because I made some mistakes with  respect to this

\section{Sign of the braces}\label{rw}



Let us use an analogous strategy to \cite{RW} used to find the signs of the Lie bracket $[f,g]$ on $\End_A$, but here we are going to use it to find the sign of the braces.

Let $A$ be a graded module. Let $S(A)$ be the graded module with $S(A)^v=A^{v+1}$, and so the suspension or \emph{shift} map $S:A\to S(A)$ given by the identity map has internal degree $-1$.

 Let $f\in \End_A(N)^i=\Hom_R(A^{\otimes N},A)^i$. Recall that $\sigma$ is the inverse of the map from Theorem \ref{markl}, so that $\sigma(f)$ is defined as the map making the following diagram commutative
\[
\begin{tikzcd}
S(A)^{\otimes N}\arrow[r, "\sigma(f)"]\arrow[d, "(S^{-1})^{\otimes N}"'] & S(A)\\
A^{\otimes N}\arrow[r,"f"] & A\arrow[u, "S"']
\end{tikzcd}
\]

Explicitly, $\sigma(f)=S\circ f\circ (S^{-1})^{\otimes N}\in \End_A(N)^{i+N-1}$. 

\begin{remark}
In \cite{RW} there is a sign $(-1)^{N+i-1}$ in front of $f$ but it seems to be irrelevant for our purposes. Another fact to remark on is that the suspension of graded modules used here (and in \cite{RW}) is the opposite that we have used to define the operadic suspension in the sense that they define $S(A)^v=A^{v-1}$. This does not change the signs or the procedure, but in the statement of theorem \ref{markl} operadic desuspension should be changed to operadic suspension. %My suspensions is better because it gives the total degree %If I modify the theorem to End_{sO}=sEnd_{SsO} I have to change the phrase
\end{remark}


Notice that, by the Koszul sign rule \[(S^{-1})^{\otimes N}\circ S^{\otimes N}=(-1)^{\sum_{j=1}^{N-1} j}1=(-1)^{\frac{N(N-1)}{2}}1=(-1)^{\binom{N}{2}}1,\] so $(S^{-1})^{\otimes N}= (-1)^{\binom{N}{2}}(S^{\otimes N})^{-1}$. For this reason, given $F\in \End_{S(A)}(m)^j$, we have
\[
\sigma^{-1}(F)=(-1)^{\binom{m}{2}}S^{-1}\circ F\circ S^{\otimes m}\in \End_A(m)^{j-m+1}.
\]

For $g_j\in \End_A(a_j)^{q_j}$, let us write $f[g_1,\dots, g_n]$ for the map \[\sum_{k_0+\cdots+k_n=N-n}f(1^{\otimes k_0}\otimes g_1\otimes 1^{\otimes k_1}\otimes\cdots\otimes g_n\otimes 1^{\otimes k_n})\in \End_A(N-n+\sum a_j)^{i+\sum q_j}.\]

We define \[b_n(f;g_1,\dots, g_n)=\sigma^{-1}(\sigma(f)[\sigma(g_1),\dots, \sigma(g_n)])\in \End_A(N-n+\sum a_j)^{i+\sum q_j}.\]
With this the definition we can prove the following.
\begin{lem}
 We have
\[b_n(f;g_1,\dots,g_n)=\sum_{N-n=k_0+\cdots+k_n} (-1)^\eta
f(1^{\otimes k_0}\otimes g_1\otimes \cdots\otimes g_n\otimes1^{\otimes k_n}),\]
where 
\[\eta=\sum_{0\leq j<l\leq n}k_jq_l+\sum_{1\leq j<l\leq n}a_jq_l+\sum_{j=1}^n (a_j+q_j-1)(n-j)+\sum_{1\leq j\leq l\leq n} (a_j+q_j-1)k_l.\]
\end{lem} 


 


\begin{proof}
Let us compute $\eta$ using the definition of $b_n$.
\begin{align*}
&\sigma^{-1}(\sigma(f)[\sigma(g_1),\dots, \sigma(g_n)])=\\ &=(-1)^{\binom{N-n+\sum a_j}{2}}S^{-1}\circ (\sigma(f)(1^{\otimes k_0}\otimes \sigma(g_1)\otimes 1^{\otimes k_1}\otimes\cdots\otimes \sigma(g_n)\otimes 1^{\otimes k_n}))\circ S^{\otimes N-n+\sum a_j}\\
&=(-1)^{\binom{N-n+\sum a_j}{2}}S^{-1}\circ S\circ f\circ (S^{-1})^{\otimes N}\circ \\ &(1^{\otimes k_0}\otimes (S\circ g_1\circ (S^{-1})^{\otimes a_1})\otimes 1^{\otimes k_1}\otimes\cdots\otimes (S\circ g_n\circ (S^{-1})^{\otimes a_n})\otimes 1^{\otimes k_n}))\circ  S^{\otimes N-n+\sum a_j}\\
&=(-1)^{\binom{N-n+\sum a_j}{2}}f\circ ((S^{-1})^{k_0}\otimes  S^{-1}\otimes\cdots \otimes  S^{-1}\otimes  (S^{-1})^{k_n})\\ &\circ(1^{\otimes k_0}\otimes (S\circ g_1\circ (S^{-1})^{\otimes a_1})\otimes\cdots\otimes (S\circ g_n\circ (S^{-1})^{\otimes a_n})\otimes 1^{\otimes k_n}))\circ S^{\otimes N-n+\sum a_j}.
\end{align*}




Now we move each $1^{\otimes k_{j-1}}\otimes S\circ g_j\circ (S^{-1})^{a_j}$ to apply $(S^{-1})^{k_{j-1}}\otimes S^{-1}$ to it. Doing this to all of them produces a sign

\[
(-1)^{(a_1+q_1-1)(n-1+\sum k_l)+(a_2+q_2-1)(n-2+\sum_2^n k_l)+\cdots+(a_n+q_n-1)k_n}=(-1)^{\sum_{j=1}^n (a_j+q_j-1)(n-j+\sum_j^n k_l)},
\]
 and we call the exponent
 
 $$\varepsilon=\sum_{j=1}^n (a_j+q_j-1)(n-j+\sum_j^n k_l).$$ So now we have, decomposing $S^{\otimes N-n+\sum a_j}$,
 
 \[
 (-1)^{\binom{N-n+\sum a_j}{2}+\varepsilon}f\circ((S^{-1})^{k_0}\otimes  g_1\circ (S^{-1})^{\otimes a_1}\otimes\cdots \otimes  g_n\circ (S^{-1})^{\otimes a_n}\otimes  (S^{-1})^{k_n})\circ (S^{\otimes k_0}\otimes S^{\otimes a_1}\otimes\cdots\otimes S^{\otimes a_n}\otimes S^{\otimes k_n}).
 \]
 
 Now we turn the tensor of inverses into inverses of tensors by introducing the appropriate signs. More precisely we introduce the sign
 \begin{equation}\label{delta}
 (-1)^{\delta}=(-1)^{\binom{k_0}{2}+\sum(\binom{a_j}{2}+\binom{k_j}{2})}
  \end{equation}
 
  
So we now have
\[
 (-1)^{\binom{N-n+\sum a_j}{2}+\varepsilon+\delta}f\circ((S^{k_0})^{-1}\otimes  g_1\circ (S^{\otimes a_1})^{-1}\otimes\cdots \otimes  g_n\circ (S^{\otimes a_n})^{-1}\otimes  (S^{k_n})^{-1})\circ (S^{\otimes k_0}\otimes S^{\otimes a_1}\otimes\cdots\otimes S^{\otimes a_n}\otimes S^{\otimes k_n})
 \]
 And the next step is moving each component of the last tensor product in front of its inverse. This will produce the sign $(-1)^\gamma$, where
 
 \begin{gather*}\gamma=-k_0\sum_1^n(k_j+a_j+q_j)-a_1(\sum_1^n k_j+\sum_2^n (a_j+q_j))-\cdots -a_nk_n\equiv\\ \sum_{j=0}^nk_j\sum_{l=j+1}^n(k_l+a_l+q_l)+\sum_{j=1}^na_j(\sum_{l=j}^nk_l+\sum_{l=j+1}^n(a_l+q_l)).
 \end{gather*}
 

 
 So in the end we have
 \[
 b_n(f;g_1,\dots,g_n)=\sum_{k_0+\cdots+k_n=N-n} (-1)^{\binom{N-n+\sum a_j}{2}+\varepsilon+\delta+\gamma}f(1^{\otimes k_0}\otimes g_1\otimes 1^{\otimes k_1}\otimes\cdots\otimes g_n\otimes 1^{\otimes k_n}).
 \]
This means that 
 \[\eta=\binom{N-n+\sum a_j}{2}+\varepsilon+\delta+\gamma.\]
  Next, we are going to simplify this sign to get rid of the binomial coefficients.
 
 \begin{remark}
If the number top of a binomial coefficient is less than 2, then the coefficient is 0. In the case of arities or $k_j$ this is because $(S^{-1})^{\otimes 1}=(S^{\otimes 1})^{-1}$ (and if the tensor is taken 0 times then it is the identity and the equality also holds, so there are no signs).
\end{remark}


We are now going to simplify the sign to obtain the desired result.

Notice that $N-n+\sum a_j=\sum k_i +\sum a_j$. In general, consider a finite sum $\sum b_i$. We can simplify $\mod 2$ the binomial coefficients

$$\binom{\sum b_i}{2}+\sum\binom{b_i}{2}$$

in the followin way. Note that all terms will appear squared once in the big binomial coefficient and once in the sum, as so will do the terms themselves, so they will cancel. This will leave the double products which cancel out the 2 in the denominator. More precisely, we have the following equality $\mod 2$:

\[\binom{\sum b_i}{2}+\sum\binom{b_i}{2}=\sum_{i<j}b_ib_j.\]
So the result of applying this to $\binom{N-n+\sum a_j}{2}$ and adding $\delta$ (recall $\delta$ from \ref{delta}) in our sign $\eta$ is

\begin{equation}\label{simply}
\sum_{0\leq i<l\leq n}k_ik_l+\sum_{1\leq j<l\leq n}a_ja_l+\sum_{i,j}k_ia_j.
\end{equation}

Recall $\gamma$ in the sign:

\begin{equation*}\label{gamma}
\gamma= \sum_{j=0}^nk_j\sum_{l=j+1}^n(k_l+a_l+q_l)+\sum_{j=1}^na_j(\sum_{l=j}^nk_l+\sum_{l=j+1}^n(a_l+q_l)).
\end{equation*}

As we see, all the sums in the previous simplification appear in $\gamma$ so we can cancel them. Let us rewrite $\gamma$ in a way that this becomes more clear:

$$\sum_{0\leq j<l\leq n}k_jk_l+\sum_{0\leq j<l\leq n}k_ja_l+\sum_{0\leq j<l\leq n}k_jq_l+\sum_{1\leq j\leq l\leq n}a_jk_l+\sum_{1\leq j<l\leq n}a_ja_l+\sum_{1\leq j<l\leq n}a_jq_l.$$

So after adding the expression \ref{simply} modulo 2 we have only the terms that include the internal degrees, i.e.
\begin{equation}\label{sofar}
\sum_{0\leq j<l\leq n}k_jq_l+\sum_{1\leq j<l\leq n}a_jq_l.
\end{equation}
Let us move now to the $\varepsilon$ term in the sign to rewrite it. 
$$\varepsilon=\sum_{j=1}^n (a_j+q_j-1)(n-j+\sum_j^n k_l)=\sum_{j=1}^n (a_j+q_j-1)(n-j)+\sum_{1\leq j\leq l\leq n} (a_j+q_j-1)k_l$$

We may add this to what we had in \ref{sofar} in such a way that the brace sign becomes

\begin{equation}\label{sigma}
\eta=\sum_{0\leq j<l\leq n}k_jq_l+\sum_{1\leq j<l\leq n}a_jq_l+\sum_{j=1}^n (a_j+q_j-1)(n-j)+\sum_{1\leq j\leq l\leq n} (a_j+q_j-1)k_l.
\end{equation}
as announced at the end of Section \ref{sectionbraces}.
\end{proof}
%
%\section{On the degree of $M_j$ and Koszul rule}\label{Ab1}
%
%Here we discuss the necessity of using the total degree, which becomes natural in the shift of the operadic supension $\Sigma\s\OO$. 
%
%
%Let $\mathcal{O}=\prod_n\OO(n)$ be an operad in a graded category with an $A_\infty$-multiplication $m=m_1+m_2+\cdots$. We denote by $\OO(n)_p$ the degree $p$ component of $\OO(n)$ and define the \emph{total degree} of an element $f\in \OO(n)_p$ as $||f||=n+p=a(f)+\deg(f)$, where $a(f)=n$ is the \emph{(operadic) arity} of $f$ and to $\deg(f)=p$ is the \emph{internal degree} of $f$. 
%
%
%
%The classical way to define an $A_\infty$-algebra structure on $\OO$ from $m$ is defining
%
%$$M_n(x_1,\dots, x_n)=b_n(m;x_1,\dots, x_n)=\sum_{j\geq n}b_n(m_j;x_1,\dots, x_n)$$
%
%for $n>1$ and 
%
%$$M_1(x)=[m,x]=b_1(m;x)-(-1)^{||x||-1}b_1(x;m)=\sum_j b_1(m_j;x)-(-1)^{||x||-1}\sum_jb_1(x;m_j).$$ 
%
%
%This construction can be iterated to an $A_\infty$ structure on $\End_\OO$ with an analogue definition of maps $\overline{M}_i$ 
%However, to distinguish the braces on $\End_\OO$ from those on $\OO$, the notation $B_n$ is used instead of $b_n$. Namely, if $n>1$,  
%$$\overline{M}_n(f_1,\dots, f_n)=B_n(M;f_1,\dots, f_n)= B_n(M;f_1,\dots, f_n)$$
%
%and
%
%$$\overline{M}_1(x)=[M,f]=B_1(M;f)-(-1)^{||f||-1}B_1(f;M).$$ 
%
%\subsection{Degree and arity considerations}
%
%We have to make sure that $a(M_j)=j$ and $\deg(M_j)=2-j$, considering the operadic arity and the internal degree as those measured in $End_\OO$ provided that $\OO$ has the total degree. The first equality is clear. To show the second we compute $||M_j(x_1,\dots, x_j)||$ since the internal degree of $M_j$ depends on the grading of $\OO$, on which we have defined a grading in terms of the total degree. To compute this quantity, let us define $M_j^l=b_j(m_l;x_1,\dots, x_j)$, which is a summand of $M_j(x_1,\dots, x_j)$. Now we have 
%
%$$a(M_j^l)=l-j+\sum_i a(x_i)$$
%
%and
%
%$$\deg(M_j^l)=\deg(m_l)+\sum_i\deg(x_i)=2-l+\sum_i \deg(x_i).$$ 
%
%These are the operadic arity and internal degree in $\OO$, so $$||M_j^l||=2-j+\sum_i(a(x_i)+\deg(x_i))=2-j+\sum_i||x_i||.$$ 
%
%This is independent of $l$, and therefore we see that $\deg(M_j)=2-j$, and the same argument is valid for $\overline{M}_j$.
%
%Therefore, it is natural to define $M_j\in\End_{\Sigma\s\OO}$. The suspension $\s\OO$ provide us with the signs we need and the additional shift produces the degree that we need. It can be checked that with other possible ``total'' degree conventions such us $a(x)+\deg(x)-1$, $a(x)-\deg(x)+1$, $a(x)-\deg(x)$ or $a(x)-\deg(x)+2$ (coming respectively from $\s\OO$, $\s^{-1}\OO$, $\Sigma^{-1}\s^{-1}\OO$ and $\Sigma\s^{-1}\OO$), the maps $M_j$ don't have the required degree.
%
%\begin{remark}\label{remark3}
%
%
%Assuming $M_j\in \End_{\Sigma\mathfrak{s}\OO}$, it has been proved that it is possible to define it so that $\deg(M)=2-j$ (and obviously the arity is $j$). So if I have to apply the Koszul rule here, the degree used is just $2-j$. If we get to define $M_j\in\mathfrak{s}\End_{\Sigma\mathfrak{s}\OO}$, then $M_j$ is actually $M_j\otimes e_J$ where $e_J=e_1\land\dots\land e_j$ has degree $j-1$. So 
%
%$$M_j\otimes e_J(x_1,\dots, x_j)=(-1)^{(j-1)(||x_1||+\cdots+||x_j||)}M_j(x_1,\dots, x_j)\otimes e_J$$
%being $||x||$ the total degree (the natural degree on $\Sigma\mathfrak{s}\OO$, recall that $M_j$ wa defined via composition on this odd operad). So passing by the $M_j$ component would yield a sign depending on its internal degree, i.e. $2-j$.
%
%For instance, if in the associative case we define $M_2$ such that $$0=M_2\tilde{\circ}M_2=M_2\tilde{\circ}_2 M_2+M_2\tilde{\circ}_1 M_2$$ in the suspension, evaluating at $(x,y,z)$ gives us on the first summand
%
%$$(M_2\tilde{\circ}_2M_2)(x,y,z)=(M_2(1,M_2(1,1))\otimes (e_1\land e_2\land e_3))(x,y,z)=(-1)^{(||x||+||y||+||z||)(3-1)}M_2(x,M_2(y,z))$$
%
%and on the second summand
%$$(M_2\tilde{\circ}_1M_2)(x,y,z)=-(M_2(M_2(1,1),1)\otimes (e_1\land e_2\land e_3))(x,y,z)=-(-1)^{(||x||+||y||+||z||)(3-1)}M_2(x,M_2(y,z))$$
%
%Adding the two of them equals zero so we get the associativity condition $M_2(x,M_2(y,z))=M_2(M_2(x,y),z)$. Note that here $x$ is beeing permuted with $M_2$ but no extra signs appears, which is equivalent to apply the Koszul rule with the internal degree of $M_2$ in $\End_{\Sigma\s\OO}$, which is $2-2=0$, and is in fact what we have done in the evaluation.
%
%\end{remark}

\section{Twisted complex on an operad}\label{twistedoperad}
In this section we provide a description of the twisted complex structure on an operad $\OO$ with a derived $A_\infty$-multiplication. More precisely, we show by hand that the maps found in \Cref{mi1} define a twisted complex structure on $S\s\OO$.

\begin{lem}\label{twistedmaps}
Let $\OO$ be an operad with a derived $A_\infty$-multiplication $m\in\s\OO$. Then $S\s\OO$ becomes a twisted complex with structure maps
\[M_{i1}(x)= \sum_l (Sb_1(m_{il};S^{-1}x)-(-1)^{\langle x,m_{il}\rangle}Sb_1(S^{-1}x;m_{il})),\]
where $x\in (S\s\OO)^{n-k}_k$ and $\langle x,m_{il}\rangle=ik+(1-i)(n-1-k)$.
\end{lem}
\begin{proof}


Througout the proof we omit the shift maps. Let us first check the twisted complex equation up to signs, to give a conceptual proof before introducing the signs. Up to sign, the maps  $\{M_{i1}\}_{i\geq 0}$ must satisfy the equation

\[\sum_{i+j=u} M_{i1}\circ M_{j1}=0,\]
for all $u$, where $\circ$ is composition of maps. %I may or may not omit the sum to avoid writing too much, as it just means that the composition on every degree must vanish.

Therefore, up to signs we have to compute compute 

\begin{align*}
&\sum_{i+j=u}M_{i1}(M_{j1}(x))=\sum_{i+j=u}M_{i1}\left(\sum_l b_1(m_{jl};x)+b_1(x;m_{jl})\right)=\\
&\sum_{i+j=u}\sum_{l,k}\left(b_1(m_{ik}; b_1(m_{jl};x))+b_1(m_{ik};b_1(x;m_{jl}))+b_1(b_1(m_{jl};x);m_{ik})+b_1(b_1(x;m_{jl});m_{ik})\right).
\end{align*}

%AT FIRST SIGHT IT DOESN'T LOOK POSSIBLE TO CANCEL THE LAST BRACE BECAUSE IT IS THE ONLY ONE WITH X AT THE BEGINNING, BUT THAT SHOULD HAVE BEEN THE SAME FOR THE CLASSICAL CASE, SO I SHOULD REVIEW THAT ONE
%
%ON THE CLASSICAL CASE IT WAS MUCH EASIER BECAUSE AFTER BRACE RELATION B(X;M,M) APPEARS TWICE WITH OPPOSITE SIGN, SO IT CANCELS. HERE IT IS NOT SO OBVIOUS BECAUSE THE SIGN IS NOT JUST $-1$ SO MAYBE IT CANCELS WITH OTHER SUMMANDS (RECALL THAT I AM OMITTING ONE SUM)

Applying the brace relation we obtain

\begin{align*}
\sum_{i+j=u}\sum_{l,k}(b_1(m_{ik}; b_1(m_{jl};x))+b_1(m_{ik};b_1(x;m_{jl}))+\\
 b_2(m_{jl};x,m_{ik})+b_1(m_{jl};b_1(x;m_{ik}))+b_2(m_{jl};m_{ik},x)+\\
b_2(x;m_{jl},m_{ik})+b_1(x;b_1(m_{jl};m_{ik}))+b_2(x;m_{ik},m_{jl})).
\end{align*}

In the sum, all terms of the form $b_1(x;b_1(m_{jl};m_{ik}))$ that can be seen in the last line should add up to vanish provided that $m$ is a $dA_\infty$-multiplication (meaning that up to sign $b_1(m;m)=0$) %A sign of the form $(-1)^i$ %(or maybe $(-1)^j$, depending on the convention) 
 Since $i$ and $j$ are interchangable (i.e. for each pair $(i,j)$ there is the pair $(j,i)$), the terms $b_2(x;m_{jl},m_{ik})+b_2(x;m_{ik},m_{jl})$ in the last line should cancel as well (for this, we should have the pair $(j,i)$ with the opposite sign). Here it is also relevant that the sum runs through all possible values of $k$ and $l$, so that the pair $(j,i)$ appears with $l$ and $k$ interchanged as well. So far the entire last line vanishes up to sign.

Then $b_1(m_{ik};b_1(x;m_{jl}))$ on the first line should cancel with $b_1(m_{jl};b_1(x;m_{ik}))$ on the second line (but from a different summand, the one where $i$ and $j$ are interchanged). Finnaly, the remaining terms $b_1(m_{ik}; b_1(m_{jl};x))+b_2(m_{jl};x,m_{ik})+b_2(m_{jl};m_{ik},x)$ add up to $b_1(b_1(m;m);x)$ up to sign. That would cancel everything.



Let us now add signs. We now compute for all $u$
\[\sum_{i+j=u} (-1)^iM_{i1}\circ M_{j1}.\]
%recalling that for the usual sign convention of twisted complex from a $dA_\infty$-algebra we need to define $d_i=(-1)^im_{i1}$, so that the sign in the equation is $(-1)^j$ instead of $(-1)^i$. This being said, let us compute 
For $x\in\s\OO$, by definition, we have
\begin{align*}
\sum_{i+j=u}(-1)^iM_{i1}(M_{j1}(x))=\sum_{i+j=u}(-1)^iM_{i1}\left(\sum_l b_1(m_{jl};x)-(-1)^{\langle x|m_{jl}\rangle}b_1(x;m_{jl})\right)=\\
\sum_{i+j=u}(-1)^i\sum_{l,k}\left(b_1(m_{ik}; b_1(m_{jl};x))-(-1)^{\langle x|m_{jl}\rangle}b_1(m_{ik};b_1(x;m_{jl}))+\right.\\
\left. -(-1)^{\langle b_1(m_{jl};x)|m_{ik}\rangle}b_1(b_1(m_{jl};x);m_{ik})+(-1)^{\langle b_1(m_{jl};x)|m_{ik}\rangle+\langle x|m_{jl}\rangle}b_1(b_1(x;m_{jl});m_{ik})\right).
\end{align*}
Observe that $\langle b_1(m_{jl};x)|m_{ik}\rangle=\langle m_{ij}|m_{ik}\rangle+\langle x|m_{ik}\rangle$. %in the usual bigraded sign convention (also in the total graded convention). I am not going to explicitly compute these signs yet to see what properties we need from them.

Applying the brace relation we obtain

\begin{align}
\sum_{i+j=u}\sum_{l,k}((-1)^ib_1(m_{ik}; b_1(m_{jl};x))-(-1)^{i+\langle x|m_{jl}\rangle}b_1(m_{ik};b_1(x;m_{jl}))+\nonumber\\
 -(-1)^{i+\langle b_1(m_{jl};x)|m_{ik}\rangle}(b_2(m_{jl};x,m_{ik})+(-1)^{\langle x|m_{ik}\rangle}b_2(m_{jl};m_{ik},x))\nonumber\\
 -(-1)^{i+\langle b_1(m_{jl};x)|m_{ik}\rangle}b_1(m_{jl};b_1(x;m_{ik}))\label{twistedequation}\\
+(-1)^{i+\langle b_1(m_{jl};x)|m_{ik}\rangle+\langle x|m_{jl}\rangle}(b_2(x;m_{jl},m_{ik})+(-1)^{\langle m_{ik}|m_{jl}\rangle}b_2(x;m_{ik},m_{jl}))\nonumber\\
+(-1)^{i+\langle b_1(m_{jl};x)|m_{ik}\rangle+\langle x|m_{jl}\rangle}b_1(x;b_1(m_{jl};m_{ik}))).\nonumber
\end{align}

Recall that $m$ being a $dA_\infty$-multiplication means that $\sum_{i+j=u}\sum_{k,l}(-1)^ib_1(m_{jl};m_{ik})=0$. %Notice that the summand corresponding to each value of $i+j$ must vanish because it corresponds to a given horizontal degree. 
Let us check now the cancellations with the signs. First, let us check that the terms 
\[(-1)^{i+\langle b_1(m_{jl};x)|m_{ik}\rangle+\langle x|m_{jl}\rangle}b_1(x;b_1(m_{jl};m_{ik})))\]
can be added up to vanish. For that, we compute the sign \[\langle b_1(m_{jl};x)|m_{ik}\rangle+\langle x|m_{jl}\rangle=\langle m_{jl}|m_{ik}\rangle+\langle x|m_{ik}\rangle+\langle x|m_{jl}\rangle.\]
Recall that the braces are defined on the operadic suspension, so that the bidegree of $m_{ik}$ is $(i,1-i)$. Therefore, writing the bidegree of $x$ as $(k,n-k)$, so that the total degree is $|x|=n$, the above equals 
\[ji+(1-i)(1-j)+ki+(n-k)(1-i)+kj+(n-k)(1-j)\equiv 1+i+j + (i+j)k+(i+j)(n-k)\mod 2=\]
\[1+(i+j)(1+n)=1+u(1+|x|).\]
Since this sign is constant for all terms $b_1(m_{ik};m_{ij})$ that share the same horizontal degree $i+j=u$, we can rewrite
\[(-1)^{i+\langle b_1(m_{jl};x)|m_{ik}\rangle+\langle x|m_{jl}\rangle}b_1(x;b_1(m_{jl};m_{ik})))=-(-1)^{u(1+|x|)}b_1(x;(-1)^ib_1(m_{ik};m_{jl})).\]
Hence, 
%Multiplying 0 by something is 0, some sum vanishes and you multiply it by a constant sign, it still vanishes
\[\sum_{i+j=u}\sum_{k,l}-(-1)^{u(1+|x|)}b_1(x;(-1)^ib_1(m_{ik};m_{jl}))=0.\]
Therefore, the expression (\ref{twistedequation}) after the brace relation reduces to
\begin{align}
\sum_{i+j=u}\sum_{l,k}((-1)^ib_1(m_{ik}; b_1(m_{jl};x))-(-1)^{i+\langle x|m_{jl}\rangle}b_1(m_{ik};b_1(x;m_{jl}))+\nonumber\\
 -(-1)^{i+\langle b_1(m_{jl};x)|m_{ik}\rangle}(b_2(m_{jl};x,m_{ik})+(-1)^{\langle x|m_{ik}\rangle}b_2(m_{jl};m_{ik},x))\label{twistedequation2}\\
 -(-1)^{i+\langle b_1(m_{jl};x)|m_{ik}\rangle}b_1(m_{jl};b_1(x;m_{ik}))\nonumber\\
+(-1)^{i+\langle b_1(m_{jl};x)|m_{ik}\rangle+\langle x|m_{jl}\rangle}(b_2(x;m_{jl},m_{ik})+(-1)^{\langle m_{ik}|m_{jl}\rangle}b_2(x;m_{ik},m_{jl})).\nonumber
\end{align}
Let us focus on the last line. For each pair $(i,j)$ we should have cancellation with the pair $(j,i)$, which adds the same elements, but with different signs. We also need to consider the pairs $(k,l)$ and $(l,k)$ to get a cancellation. Let us compare the signs. For the pair $((i,j),(k,l))$ we have precisely the last line of the above equation
\[(-1)^{i+\langle b_1(m_{jl};x)|m_{ik}\rangle+\langle x|m_{jl}\rangle}(b_2(x;m_{jl},m_{ik})+(-1)^{\langle m_{ik}|m_{jl}\rangle}b_2(x;m_{ik},m_{jl}))\]

For the pair $((j,i),(l,k))$ we have
\[(-1)^{j+\langle b_1(m_{ik};x)|m_{jl}\rangle+\langle x|m_{ik}\rangle}(b_2(x;m_{ik},m_{jl})+(-1)^{\langle m_{jl}|m_{ik}\rangle}b_2(x;m_{jl},m_{ik})).\]
 Comparing the sign of $b_2(x;m_{jl},m_{ik})$ we find that for $((i,j),(k,l))$ we have

\[-(-1)^{i+(i+j)(1+|x|)}b_2(x;m_{jl},m_{ik})=-(-1)^{j+u|x|}b_2(x;m_{jl},m_{ik})\]
and for the pair $((j,i),(l,k))$ we have
\[(-1)^{j+u|x|}b_2(x;m_{jl},m_{ik}).\]
As we see, we get opposite signs and thus cancellation. For $b_2(x;m_{ik},m_{jl})$ it is completely analogous. Thus, we have reduced expression (\ref{twistedequation2}) to
\begin{align}
\sum_{i+j=u}\sum_{l,k}((-1)^ib_1(m_{ik}; b_1(m_{jl};x))-(-1)^{i+\langle x|m_{jl}\rangle}b_1(m_{ik};b_1(x;m_{jl}))+\nonumber\\
 -(-1)^{i+\langle b_1(m_{jl};x)|m_{ik}\rangle}(b_2(m_{jl};x,m_{ik})+(-1)^{\langle x|m_{ik}\rangle}b_2(m_{jl};m_{ik},x))\label{twistedequation3}\\
 -(-1)^{i+\langle b_1(m_{jl};x)|m_{ik}\rangle}b_1(m_{jl};b_1(x;m_{ik})).\nonumber
\end{align}
In a similar fashion to the previous calculation, we are going to cancel $b_1(m_{ik};b_1(x;m_{jl}))$ in the first line with $b_1(m_{jl};b_1(x;m_{ik})$ in the last line by considering switched pairs. For the pair $((i,j),(k,l))$, the term in the first line is 
\[-(-1)^{i+\langle x|m_{jl}\rangle}b_1(m_{ik};b_1(x;m_{jl}))\]
and for the pair $((j,i),(l,k))$ the term in the last line is
\[-(-1)^{j+\langle b_1(m_{ik};x)|m_{jl}\rangle}b_1(m_{ik};b_1(x;m_{jl}))=(-1)^{1+j+\langle m_{ik}|m_{jl}\rangle+\langle x|m_{jl}\rangle}b_1(m_{ik};b_1(x;m_{jl}))=\]
\[(-1)^{i+\langle x|m_{jl}\rangle}b_1(m_{ik};b_1(x;m_{jl})),\]
which has precisely the opposite sign to the other pair, and thus cancels. This reduces expression (\ref{twistedequation3}) to 
\begin{align}
\sum_{i+j=u}\sum_{l,k}((-1)^ib_1(m_{ik}; b_1(m_{jl};x))&\label{twistedequation4}\\
 -(-1)^{i+\langle b_1(m_{jl};x)|m_{ik}\rangle}(b_2(m_{jl};x,m_{ik})&+(-1)^{i+\langle m_{jl}|m_{ik}\rangle}b_2(m_{jl};m_{ik},x)).\nonumber
\end{align}
We want this terms to add up to something of the form $b_1(b_1(m;m);x)$. Notice that for this we need to switch some pairs. For simplity, we switch the pair of the first term and rewrite the sum as
\begin{align}
\sum_{i+j=u}\sum_{l,k}((-1)^jb_1(m_{jl}; b_1(m_{ik};x))&\\
 -(-1)^{i+\langle b_1(m_{jl};x)|m_{ik}\rangle}b_2(m_{jl};x,m_{ik})&+(-1)^{i+\langle m_{jl}| m_{ik}\rangle}b_2(m_{jl};m_{ik},x)).\nonumber
\end{align}
Simplifying the signs we get
\begin{align*}
\sum_{i+j=u}\sum_{l,k}((-1)^jb_1(m_{jl}; b_1(m_{ik};x))
 +(-1)^{j+\langle x|m_{ik}\rangle}b_2(m_{jl};x,m_{ik})+(-1)^{j}b_2(m_{jl};m_{ik},x)).
\end{align*}
By the brace relation and \Cref{sharp} this equals
\[\sum_{i+j=u}\sum_{l,k}(-1)^jb_1(b_1(m_{jl};m_{ik});x)=0.\]
%For each position of insertion of x we have a 0 map applied to x, so the above sum is indeed equal to 0
\end{proof}

The reader can see that the twisted complex structure given by the above Lemma is the same as the one given by \Cref{mi1}.

%Recall that if $\OO$ is an operad with a derived $A_\infty$-multiplication, there is an $A_\infty$-algebra structure on $B=\Tot(S\s\OO)$ given by maps $M_j:B^{\otimes j}\to B$ induced by braces. We show next that the twisted complex structure induced from the map $M_1$ via \Cref{alternative} is the same as the one we have given above.
%
%\begin{corollary}
%Let $\OO$ be a bigraded operad with a derived $A_\infty$-multiplication and let \[\widetilde{M}_{i1}:S\s\OO\to S\s\OO\] be the arity 1 derived $A_\infty$-algebra maps induced by \Cref{alternative} from \[M_1:\Tot(S\s\OO)\to \Tot(S\s\OO).\]
%Then $\widetilde{M}_{i1}=M_{i1}$, where $M_{i1}$ is defined in \Cref{twistedmaps}.
%\end{corollary}
%\begin{proof}
%Notice that the proof of \Cref{alternative} is essentially the same as the proof \Cref{whitehouse}. This means that the proof of this result is an arity 1 restriction of the proof of \Cref{derivedmaps}. Then we apply \Cref{totsign} to the case $j=1$. Recall that for $x\in (S\s\OO)^{n-k}_{k}$
%\[M_1(x)=b_1^T(m;S^{-1}x)-(-1)^{n-1}b_1^T(S^{-1}x;m).\]
% In this case, there is no $\mu$ involved. Therefore, introducing here also the final extra sign $(-1)^i$ from the proof of \Cref{derivedmaps}, we get from \Cref{totsign} that
%\[\widetilde{M}_{i1}(x)=(-1)^i\sum_l((-1)^{in+i(n-1)} Sb_1(m_{il};S^{-1}x)-(-1)^{n-1+in+k}Sb_1(S^{-1}x;m_{il})),\] where $b_1$ is the brace on $\s\OO$. Simplifying signs we obtain
%\[\widetilde{M}_{i1}(x)=\sum_l Sb_1(m_{il};S^{-1}x)-(-1)^{\langle  m_{il},x\rangle}Sb_1(m_{il};S^{-1}x))=M_{i1}(x),\]
%
%where $\langle  m_{il},x\rangle=ik+(1-i)(n-1-k)$, as claimed.
%\end{proof}
%\appendix
%\renewcommand{\appendixname}{Appendix:}




%\section{Some proofs and details}




\end{appendices}