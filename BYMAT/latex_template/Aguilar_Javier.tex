\TITLE{Induced $A_\infty$-structures} % The title of your talk. Use capital letters only for the first word and names.

\AUTHOR{Javier Aguilar} % The name of the speaker. Format: '[first name] [last name]', e.g. 'John Doe'.

\EMAIL{ja683@kent.ac.uk} % The email of the speaker.

\AFFIL{University of Kent} % Your affiliation, use an abbreviated version if possible.

\ABSTRACT{%
An $A_\infty$-algebra $A$ is a module over a ring equipped with a family of ``multiplications'' $m_n:A^{\otimes n}\to A$ satisfying the relation

$$\sum_{r+s+t=n}(-1)^{rs+t}m_{r+1+t}(1^{\otimes r}\otimes m_s\otimes 1^{\otimes s})=0$$
for all possible values of $n$. When $m_i=0$ for $i\neq 2$, this relation tells us that $A$ is an ordinary associative algebra with multiplication $m_2$.%$m_2$ is an associative product, so associative algebras are a particular case of $A_\infty$-algebras.
%$A$ is an associative algebra with multiplication $m_2$. Similarly, a chain complex is an $A_\infty$-algebra with $m_i=0$ for $i\neq 1$ and a differential graded algebra is an $A_\infty$-algebra with $m_i=0$ for $i\neq 1,2$.

%While the above relation looks quite random, it in fact has strong geometric roots. For a pointed space $X$, its loop space $\Omega X$ \emph{almost} has a multiplication $\Omega X\times \Omega X\to \Omega X$ given by the concatenation of loops, but this map is only associative \emph{up to homotopy}. This situation is similar to what happens to $m_2$ in an $A_\infty$-algebra. More specifically, the singular chain complex of a loop space is an $A_\infty$-algebra.

%For an $A_\infty$-algebra $A$, we consider its Hochschild complex given by the modules $C^n(A,A)=\hom(A^{\otimes n},A)$ and induce an $A_\infty$-algebra structure on it from the one in $A$. To do this, we study these structures from the perspective of algebraic operads, i.e. the study of algebraic structures given by $n$-ary operations.
We will present how this structure naturally arises in algebra and topology and discuss some examples. %We will do this from the perspective of algebraic operads, i.e. the study of algebraic structures given by $n$-ary operations.
}
