\TITLE{Derived $A_\infty$-algebras} % The title of your talk. Use capital letters only for the first word and names.

\AUTHOR{Javier Aguilar Martín} % The name of the speaker. Format: '[first name] [last name]', e.g. 'John Doe'.

\EMAIL{ja683@kent.ac.uk} % The email of the speaker.

\AFFIL{University of Kent} % Your affiliation, use an abbreviated version if possible.

\ABSTRACT{%
The notion of $A_\infty$-algebras has been fruitfully studied in homotopy theory as a homotopical generalization of associative algebras. Its geometrical roots have made them relevant in different areas of mathematics such as algebraic topology, algebraic geometry and representation theory.

 However, there are limitations to the power of $A_\infty$-algebras when working with rings that are not fields. For this reason, a more sophisticated notion of \emph{derived }$A_\inty$-algebras was defined. 
 
 We introduce here the idea of derived $A_\infty$-algebras as a generalization of the classical $A_\infty$-algebras and explain the connection between them.
}
