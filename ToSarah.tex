	\documentclass[twoside]{article}
\usepackage{estilo-ejercicios}
\setcounter{section}{0}
\newtheorem{defin}{Definition}[section]
\newtheorem{lem}[defin]{Lemma}
\newtheorem{propo}[defin]{Proposition}
\newtheorem{thm}[defin]{Theorem}
\newtheorem{eje}[defin]{Example}
\newtheorem{obs}[defin]{Observación}
\renewcommand{\baselinestretch}{1,3}

\usepackage{empheq}
\newcommand*\widefbox[1]{\fbox{\hspace{2em}#1\hspace{2em}}}
%--------------------------------------------------------
\begin{document}

\title{Operadic Suspension}
\author{Javier Aguilar Martín}
\maketitle

\section{Operadic suspension}



Let $sig_n$ be the sign representation of the symmetric group on $n$ symbols concentrated in degree 0. It comes with a natural action of the symmetric group $S_n$. Let $\Lambda(n)=S^{n-1}sig_n$, where $S$ is the shift of graded vector spaces, so that $\Lambda(n)$ is the sign representation of the symmetric group concentrated in degree $n-1$. This space can be realized as the one-dimensional vector space spanned by the exterior power $e_1\land\cdots\land e_n$ of degree $n-1$. Let us define an operad structure on $\Lambda=\{\Lambda(n)\}_{n\geq 0}$ via the following insertion maps

\[
\begin{tikzcd}
\Lambda(n)\otimes\Lambda(m) \arrow[r, "\circ_i"] & \Lambda(n+m-1)\\
(e_1\land\cdots\land e_n)\otimes(e_1\land\cdots\land e_m)\arrow[r, mapsto] & (-1)^{(n-i)(m-1)}e_1\land\cdots\land e_{n+m-1}.
\end{tikzcd}
\]



In a similar way we can define $\Lambda^-(n)=S^{1-n}sig_n$, with the same insertion maps.

\begin{definition}
Let $\mathcal{O}$ be a graded linear operad. The \emph{operadic suspension} $\mathfrak{s}\OO$ of $\mathcal{O}$ is given arity-wise by the Hadamard product of operads $\mathfrak{s}\OO(n)=(\mathcal{O}\otimes\Lambda)(n)=\mathcal{O}(n)\otimes\Lambda(n)$. Similarly, we define the \emph{operadic desuspension} $\mathfrak{s}^{-1}\OO(n)=\mathcal{O}(n)\otimes\Lambda^-(n)$.
\end{definition}


We may identify the elements of $\mathcal{O}$ with the elements the elements of $\mathfrak{s}\OO$. For $a\in\OO(n)$ of degree $\deg(a)$, its ``natural'' degree in $\s\OO$ is $|a|=\deg(a)+n-1$. To distinguish both degrees we call $\deg(a)$ the \emph{internal degree} of $a$. If we write $\circ_i$ for the operadic insertion on $\OO$ and $\tilde{\circ}_i$ for the operadic insertion on $\mathfrak{s}\OO$, we may find a relation between the two insertion maps in the following way. Let $a\in\OO(n)$ and $b\in\OO(m)$, and let us compute $a\tilde{\circ}_i b$.

\begin{align*}
\mathfrak{s}\OO(n)\otimes\mathfrak{s}\OO(m)&=(\OO(n)\otimes\Lambda(n))\otimes (\OO(m)\otimes\Lambda(m))\cong (\OO(n)\otimes \OO(m))\otimes (\Lambda(n)\otimes \Lambda(m))\\
&\xrightarrow{\circ_i\otimes\circ_i} \OO(m+n-1)\otimes \Lambda(n+m-1)=\mathfrak{s}\OO(n+m-1).
\end{align*}

The symmetric monoidal structure produces the sign $(-1)^{(n-1)\deg(b)}$ and the operadic structure of $\Lambda$ produces the sign $(-1)^{(n-i)(m-1)}$, so 

$$a\tilde{\circ}_ib=(-1)^{(n-1)\deg(b)+(n-i)(m-1)}a\circ_i b.$$

Now we have mod 2

$$(n-i)(m-1)=(n-1-i-1)(m-1)=(n-1)(m-1)+(i-1)(m-1)$$

so we conclude 

$$a\tilde{\circ}_ib=(-1)^{(n-1)(m-1)+(n-1)\deg(b)+(i-1)(m-1)}a\circ_i b.$$

This is exactly the sign in R-W from which the sign in the equation defining $A_\infty$-algebras is derived. 



\begin{theorem}[Operads in Algebra, Topology and Physics]
Given a graded vector space $V$, there is an isomorphism of operads $\End_{S V}\cong \mathfrak{s}^{-1}\End_V$, where $\End_V$ is the endomorphism operad of $V$.
\end{theorem}
The proof in the original reference is not very explicit (see \ref{proofthm} for a detailed proof), but in the case of our operadic suspension, the isomorphism is given by $$\sigma^{-1}:\End_{S A}\to\mathfrak{s}^{-1}\End_A,$$ where $\sigma^{-1}(F)=(-1)^{\binom{n}{2}}S^{-1}\circ F\circ S^{\otimes n}$ for $F\in \End_{S A}(n)$. 
Note that we are using the identification of elements of $\End_A$ with those in $\mathfrak{s}^{-1}\End_A$. The notation $\sigma^{-1}$ comes from R-W. 

In R-W the sign for the insertion maps was obtained by computing $\sigma^{-1}(\sigma(a)\circ_i\sigma(b))$. This can be interpreted as sending $a$ and $b$ from $\End_A$ to $\End_{S A}$ via $\sigma$ (but only as graded vector spaces, not as operads) and then applying the isomorphism $\sigma^{-1}$. In the end this is the same as simply sending $a$ and $b$ to their images in $\mathfrak{s}^{-1}\End_A$, which is what has been done here.
\subsection{Braces}
First recall the definition of a brace algebra.

\begin{defi}
A brace algebra on a graded vector space $V$ consists of a family of maps $b_n:V^{\otimes n+1}\to V$ satisfying the \emph{brace relation}


\begin{align*}
b_m(b_n(x;x_1,\dots, x_n);y_1,\dots,y_m)=&\\
\sum_{i_1,\dots, i_n, j_1\dots, j_n}(-1)^{\varepsilon}&b_l(x; y_1,\dots, y_{i_1},b_{j_1}(x_1;y_{i_1+1},\dots, y_{i_1+j_1}),\dots, b_{j_n}(x_n;y_{i_n+1},\dots, y_{i_n+j_n}),\dots,y_m)
\end{align*}
where $l=n+\sum_{p=1}^n i_p$ and $\varepsilon=\sum_{p=1}^n|x_p|\sum_{q=i}^{i_p}|y_q|,$ i.e. the sign is picked up by the $x_i$'s passing by the $y_i$'s in the shuffle.

\end{defi}


We can define braces on $\s\OO$ via operadic composition. More precisely, we define the maps 
$$b_n:\mathfrak{s}\OO(N)\otimes\mathfrak{s}\OO(a_1)\otimes\cdots\otimes\mathfrak{s}\OO(a_n)\to\mathfrak{s}\OO(N-\sum a_i)$$
using the operadic composition $\gamma$ on $\mathfrak{s}\OO$ as

$$b_n(f;g_1,\dots,g_n)=\sum\gamma(f;1,\dots,1,g_1,1,\dots,1,g_n,1,\dots,1),$$

where the sum runs over all possible ordering preserving insertions. 

 This way we will get the same sign as mimicking R-W's strategy (as done in \ref{rw}), with the advantage that now the brace relation follows immediatly from the associativity axiom of operadic composition. To obtain the signs that make this composition differ from the composition in $\OO$, we must first look at the operadic composition on $\Lambda$. In the case of the endomorphism operad, we are interested in compositions of the form $$f(1^{\otimes k_0}\otimes g_1\otimes 1^{\otimes k_1}\otimes\cdots\otimes g_n\otimes 1^{\otimes k_n})$$ where $N-n=k_0+\cdots+k_n$, $f$ has arity $N$ and each $g_i$ has arity $a_i$ and internal degree $q_i$. Therefore, let us consider the operadic composition

\[
\begin{tikzcd}
\Lambda(N)\otimes\Lambda(1)^{k_0}\otimes\Lambda(a_1)\otimes\Lambda(1)^{\otimes k_1}\otimes\cdots\otimes\Lambda(a_n)\otimes\Lambda(1)^{k_n}\arrow[r] & \Lambda(N-n+\sum_{i=1}^na_i)
\end{tikzcd}
\]

The composition can be described in terms of insertions in the obvious way, namely

$$\gamma(f;h_1,\dots, h_N)=(\cdots(f\circ_1 h_1)\circ_{1+a(h_1)}h_2\cdots)\circ_{1+\sum a(h_p)}h_N,$$

where $a(h_p)$ is the arity of $h_i$ (in this case $h_p$ is either $1$ or some $g_i$). So we just have to find out the sign iterating the same argument as in the $i$-th insertion. In this case, each $\Lambda(a_i)$ produces a sign given by the exponent $$(a_i-1)(N-k_0+\cdots-k_{i-1}-i).$$ 

For this, recall that the degree of $\Lambda(a_i)$ is $a_i-1$ and that the generator of this space is inserted in the position $1+\sum_{j=0}^{i-1}k_j+\sum_{j=1}^{i-1}a_j$ of a wedge of $N+\sum_{j=1}^{i-1}a_j-i+1$ generators. Therefore, performing this insertion as described in the previous section yields the aforementioned sign. Now, since $N-n=k_0+\cdots+k_n$, we have that
$$(a_i-1)(N-k_0+\cdots+k_{i-1}-i)=(a_i-1)(n-i+\sum_{l=i}^nk_l).$$

Now we can compute the sign produced by a brace. For this, notice that the isomorphism $(\OO(1)\otimes \Lambda(1))^{\otimes k}\cong \OO(1)^{\otimes k}\otimes \Lambda(1)^{\otimes k}$ does not produce any signs because of degree reasons. Therefore, therefore the sign coming from the isomorphism

$$\OO(N)\otimes\Lambda(N)\otimes (\OO(1)\otimes \Lambda(1))^{\otimes k_0}\bigotimes_{i=1}^n(\OO(a_i)\otimes\Lambda(a_i)\otimes(\OO(1)\otimes\Lambda(1))^{\otimes k_i}$$
$$\cong \OO(N)\otimes(\bigotimes_{i=1}^n \OO(a_i))\otimes \Lambda(N)\otimes(\bigotimes_{i=1}^n \Lambda(a_i))$$
is determined by the exponent

$$(N-1)\sum_{i=1}^nq_i+\sum_{i=1}^n (a_i-1)\sum_{l>i}q_l.$$

This equals
$$(\sum_{j=0}^nk_j +n-1)\sum_{i=1}^nq_i+\sum_{i=1}^n (a_i-1)\sum_{l>i}q_l.$$

After doing the composition 
$$\OO(N)\otimes(\bigotimes_{i=1}^n \OO(a_i))\otimes \Lambda(N)\otimes(\bigotimes_{i=1}^n \Lambda(a_i))\longrightarrow \OO(N-n+\sum_{i=1}^na_i)\otimes \Lambda(N-n+\sum_{i=1}^na_i)$$

we can add the sign coming from the suspension, so all in all the sign for the braces is

$$\sum_{i=1}^n(a_i-1)(n-i+\sum_{l=i}^nk_l)+(\sum_{j=0}^nk_j +n-1)\sum_{i=1}^nq_i+\sum_{i=1}^n (a_i-1)\sum_{l>i}q_l.$$

It is not hard to check that this can be rewritten as 
$$\sum_{0\leq j<l\leq n}k_jq_l+\sum_{1\leq j<l\leq n}a_jq_l+\sum_{j=1}^n (a_j+q_j-1)(n-j)+\sum_{1\leq j\leq l\leq n} (a_j+q_j-1)k_l.$$

This is the sign that is obtained following R-W's strategy in \ref{rw}.

%WATCH OUT. GIVEN THE INSERTION MAP IN THE NON-GRADED CASE THE OPERADIC COMPOSITION IS $\gamma(f;g_1,\dots, g_p)=(\cdots(f\circ_p g_p)\cdots)\circ_1 g_1$ AND IN THE GRADED CASE IT COMES WITH THE KOSZUL SIGN. OTHERWISE IT CAN BE COMPUTED AS 
%$\gamma(f;g_1,\dots, g_p)=(\cdots(f\circ_1 g_1)\circ_{1+a(g_1)}g_2\cdots)\circ_{1+\sum a(g_i)}g_p$ I HAVE TO CHECK IF WHAT I'VE DONE IS THIS OR IF I'VE MIXED UP THINGS

%\subsection{Advantages of this approach}
%First of all, we get an easier way to obtain the signs and the brace relation follows easily. In addition, this explanation fits better in the context of operads and feels more natural.
%
%Furthermore, since we have an isomorphism of operads $\End_{\Sigma A}\cong \mathfrak{s}^{-1}\End_A$, we can translate if needed, results from an operad to its desuspension, which has the same signs in composition as the suspension, but with opposite grading. We can also use this isomorphism if we define maps on $\s\End_{\Sigma A}$, since this is then isomorphic to $\End_A$, which is the naïve Hochschild complex.
%
%%\section{$A_\infty$-structure on $\End_{\Sigma\mathfrak{s}\OO}$}
%%
%%Let $\Sigma\s\OO$ be the shift as a graded vector space of $\s\OO$. For an element $x\in\Sigma\s\OO$ let us write $||x||$ for its total degree (the natural degree in this case, arity plus internal degree) and $|x|=||x||-1$ for its \emph{reduced degree} (which is the natural degree in $\s\OO$). We had defined the maps $M_j:(\Sigma\s\OO)^{\otimes j}\to\Sigma\s\OO$ by 
%%
%%$$M_j(x_1,\dots,x_j)=b_j(m;x_1,\dots, x_j)$$
%%
%%for $j>1$ and
%%
%%$$M_1(x)=b_1(m;x)-(-1)^{|x|}b_1(x;m).$$
%%
%%We know that $M_j$ must be defined on $\Sigma\s\OO$ to be of degree $2-j$ because it must take the total degree, i.e. $M_j\in\End_{\Sigma\s\OO}$ (see \ref{Ab1}). 
%%
%%By Getzler we know that these maps define an $A_\infty$-structure on $\End_{\Sigma\s\OO}$ in the sense of $M\circ M=0$ for the operadic composition on $\End_{\Sigma\s\OO}$ (without signs). If we use the operad isomorphism $\sigma^{-1}:\End_{\Sigma\s\OO}\cong\s^{-1}\End_{\s\OO}$ we can obtain the relation $\sigma^{-1}(M)\tilde{\circ}\sigma^{-1}(M)=0$ (now with the signs we normally use). But the problem is that if we want to define an $A_\infty$-structure on $\overline{M}_j$ on $\s^{-1}\End_{\s\OO}$, we face the problem of degree. Again, $\overline{M}_j$ must take the total degree. But desuspending substracts the arity instead of adding it (and a shift up or down doesn't fix this). In addition, this solution is not totally satisfying since $\sigma^{-1}(M)$ is defined in terms of maps from other operad. 
%%
%%
%%So the alternative is redefining the maps $M_j$ to obtain some maps $M_j'$ that satisfy $M'\tilde{\circ}M'=0$, so that $M_j'$ can be seen as elements of $\s\End_{\Sigma\OO}$ and the new map $\overline{M}_j$ can be defined on the shift of this operad.
%%
%%The strategy is similar to the sign twist in the dg-case, where the associative product was defined as $xy=(-1)^{|x|}b_2(m;x,y)$. In particular, $M_2'(x,y)=(-1)^{|x|}b_2(m;x,y)$.
%%
%%\begin{remark}
%%Another possibility is sending $\sigma^{-1}(M_j)\in\s^{-1}\End_{\s\OO}$ to $\s\End_{\s\OO}$. The signs are the same and this identification consists of adding some exterior products, so it doesn't really modify the map $\sigma^{-1}(M_j)$ or the operadic composition. The problem is that it gives the opposite degree: if $\sigma^{-1}(M_j)$ has degree $2-j$ in $\s^{-1}\End_{\s\OO}$ then it has degree $j$ when seen as an element of $\s\End_{\s\OO}$.
%%\end{remark}
%%
%%\subsection{Redefining the maps}
%%I am going to use the notation $M_j$ for what I've called $M_j'$ before since we are going to be interested only in these new maps. $M_1$ remains unmodified and $M_2$ has already been defined as $M_2(x,y)=(-1)^{|x|}b_2(m;x,y)$. We want to define $M_j$ for $j\geq 3$ such that for each decomposition $n=r+s+t$ we have
%%
%%$$\sum_n (-1)^{rs+t}M_{r+1+t}(1^{\otimes r}\otimes M_s\otimes 1^{\otimes t})=0.$$
%%
%%For $n=1,2$ we already know that this relation is satisfied since only $M_1$ and $M_2$ are involved. We are going to look at the case $n=3$ to define $M_3$. Since we are going to modify $b_3(m;x,y,z)$ by a sign depending on the elements involved, we need to rewrite the above relation after applying it to elements. 
%%
%%\begin{remark}
%%Let $\mathcal{P}=\s\End_{\Sigma\s\OO}$.  If $f\in\mathcal{P}(n)$, then $f=f'\otimes (e_1\land\cdots\land e_n)$, so $$f(x_1,\dots,x_n)=(-1)^{(n-1)\sum_i ||x_i||}f'(x_1,\dots,x_n)\otimes(e_1\land\cdots\land e_n).$$
%%
%%Fortunately, this sign is not going to be relevant in our equations since it's the same for any two maps of the same arity and we will be able to cancel it. More precisely, for each fixed $n$,
%%
%%$$0=\sum_{r+s+t=n}(-1)^{rs+t}M_{r+1+t}(1^{\otimes r}\otimes M_s\otimes 1^{t})(x_1,\dots, x_n)=$$
%%$$(-1)^{(n-1)\sum_i||x_i||}(-1)^{(2-s)\sum_{i=1}^r||x_i||}(-1)^{rs+t}M_{r+1+s}(x_1,\dots, x_r, M_s(x_{r+1},\dots, x_{r+s}), x_{r+s+1},\dots, x_n)$$
%%
%%so we can cancel the factor $(-1)^{(n-1)\sum_i||x_i||}$. Note that the Koszul rule applied here takes the total degree, since that is the degree on $\Sigma\s\OO$, where the maps are defined (more about this in \ref{remark3}). In particular, the Leibniz rule takes the form of
%%
%%$$M_1(M_2(x,y))=M_2(x, M_1(x))+(-1)^{||x||}M_2(x,M_1(y)).$$
%%
%%The total degree in the sign is consistent with the oddity that we originally found. In particular, we already know that this relation holds, which is the $A_\infty$-equation for $n=2$. For $n=1$ it is just $M_1(M_1(x))=0$, that we also know (more about this in \ref{remark3}). We have to be careful because the reduced degree is also going to appear in the operadic relations such as the brace relation.
%%\end{remark}
%%\subsection{Definition of $M_3$}
%%We are going to define $M_3(x,y,z)=(-1)^{\varepsilon(x,y,z)}b_3(m;x,y,z)$ and find necessary conditions that $\varepsilon(x,y,z)$ must satisfy. To do that we look at the $A_\infty$-equation for $n=3$. Before proceding, let us impose some previous conditions on $\varepsilon(x,y,z)$. It should depend only on the total or reduced degree of $x$, $y$ and $z$. In particular, it should not distinguish between $b_1(m;x)$ and $b_1(x;m)$, so we may define $\varepsilon(M_1(x),y,z)$ and so on. We denote 
%%\begin{gather*}
%%\varepsilon_1\coloneqq\varepsilon(M_1(x),y,z),\\
%%\varepsilon_2\coloneqq\varepsilon(x,M_1(y),z),\\
%%\varepsilon_3\coloneqq\varepsilon(x,y,M_1(z)).\\
%%\varepsilon\coloneqq\varepsilon(x,y,z)
%%\end{gather*}
%%And now let us look at the $A_\infty$-equation for $n=3$, which is
%%
%%\begin{align*}
%%M_3(M_1(x),y,z)+(-1)^{||x||}M_3(x,M_1(y),z)+(-1)^{||x||+||y||}M_3(x,y,M_1(z))\\
%%-M_2(M_2(x,y),z)+M_2(x,M_2(y,z))+M_1(M_3(x,y,z))=0.
%%\end{align*}
%%
%%First we apply the definitions of $M_1$ and $M_2$.
%%\begin{align*}
%%M_3(b_1(m;x),y,z)+(-1)^{||x||}M_3(x,b_1(m;y),z)+(-1)^{||x||+||y||}M_3(x,y,b_1(m;z))\\
%%-(-1)^{|x|}M_3(b_1(x;m),y,z)-(-1)^{|y|+||x||}M_3(x,b_1(y;m),z)-(-1)^{|z|+||x||+||y||}M_3(x,y,b_1(z;m))\\
%%-(-1)^{|y|+1}b_2(m;b_2(m;x,y),z)+(-1)^{|x|+|y|}b_2(m;x,b_2(m;y,z))\\+b_1(m;M_3(x,y,z))-(-1)^{|x|+|y|+|z|+1}b_1(M_3(x,y,z);m)=0.
%%\end{align*}
%%And now we apply the definition of $M_3(x,y,z)$.
%%
%%\begin{align*}
%%(-1)^{\varepsilon_1}b_3(m;b_1(m;x),y,z)+(-1)^{||x||+\varepsilon_2}b_3(m;x,b_1(m;y),z)+(-1)^{||x||+||y||+\varepsilon_3}b_3(m;x,y,b_1(m;z))\\
%%-(-1)^{|x|+\varepsilon_1}b_3(m;b_1(x;m),y,z)-(-1)^{|y|+||x||+\varepsilon_2}b_3(m;x,b_1(y;m),z)-(-1)^{|z|+||x||+||y||+\varepsilon_3}b_3(m;x,y,b_1(z;m))\\
%%+(-1)^{|y|}b_2(b_2(m;x,y),z)+(-1)^{|x|+|y|}b_2(m;x,b_2(m;y,z))\\+(-1)^{\varepsilon}b_1(m;b_3(m;x,y,z))+(-1)^{|x|+|y|+|z|+\varepsilon}b_1(b_3(m;x,y,z);m)=0.
%%\end{align*}
%%
%%Following Getzler's proof of $M\circ M=0$, we next apply the brace relation to the last term $(-1)^{|x|+|y|+|z|+\varepsilon}b_1(b_3(m;x,y,z);m)$. We will after that impose the cancellation of the second line of the equation above to obtain some conditions on the signs.
%%
%%\begin{align*}
%%(-1)^{|x|+|y|+|z|+\varepsilon}b_1(b_3(m;x,y,z);m)=
%%&(-1)^{|x|+|y|+|z|+\varepsilon}b_4(m;x,y,z,m)+(-1)^{|x|+|y|+|z|+\varepsilon}b_3(m;x,y,b_1(z;m))\\
%%&+(-1)^{|x|+|y|+\varepsilon}b_4(m;x,y,m,z)+(-1)^{|x|+|y|+\varepsilon}b_3(m;x,b_1(y;m),z)\\
%%&+(-1)^{|x|+\varepsilon}b_4(m;x,m,y,z)+(-1)^{|x|+\varepsilon}b_3(m;b_1(x;m),y,z)\\
%%&+(-1)^{\varepsilon}b_4(m;m,x,y,z).
%%\end{align*}
%%The conditions modulo 2 that we get from the cancellation condition are the following:
%%
%%\begin{gather}
%%|x|+\varepsilon_1=|x|+\varepsilon\Rightarrow\varepsilon_1=\varepsilon\\
%%|y|+||x||+\varepsilon_2=|x|+|y|+\varepsilon\Rightarrow \varepsilon_2=\varepsilon-1\\
%%|z|+||x||+||y||+\varepsilon_3=|x|+|y|+|z|+\varepsilon\Rightarrow\varepsilon_3=\varepsilon
%%\end{gather}
%%From condition (1) and (3) we deduce $\varepsilon$ does not depend on the first and third argument, and from condition (2) we deduce $\varepsilon(x,M_1(y),z)=\varepsilon(x,y,z)+1$. Therefore the natural way to define $\varepsilon$ is by $\varepsilon(x,y,z)=|y|$, because $|M_1(y)|=|y|+1$ (defining it as $||y||$ would also do the job, but sticking to $|y|$ will be more convenient).
%%
%%Thus, specifying $\varepsilon(x,y,z)=|y|$ in the $A_\infty$-equation together with the brace relation and some simplification of signs gives us
%%
%%\begin{align*}
%%(-1)^{|y|}b_3(m;b_1(m;x),y,z)+(-1)^{|x|+|y|}b_3(m;x,b_1(m;y),z)+(-1)^{|x|}b_3(m;x,y,b_1(m;z))\\
%%+(-1)^{|y|}b_2(m;b_2(m;x,y),z)+(-1)^{|x|+|y|}b_2(m;x,b_2(m;y,z))+(-1)^{|y|}b_1(m;b_3(x,y,z))\\
%%+(-1)^{|x|+|z|}b_4(m;x,y,z,m)+(-1)^{|x|}b_4(m;x,y,m,z)\\
%%+(-1)^{|x|+|y|}b_4(m;x,m,y,z)+(-1)^{|y|}b_4(m;m,x,y,z)=0
%%\end{align*}
%%
%%It can be checked using the brace relation that the above expression equals $(-1)^{|y|}b_3(b_1(m;m);x,y,z)$, so it is indeed 0 since we are assuming that $b_1(m;m)=0$. 
%%
%%This shows that $M_3(x,y,z)=(-1)^{|y|}b_3(m;x,y,z)$ is a good definition. The next step would be trying to generalize this to higher maps. So far, the pattern that can be observed is
%%
%%$$M_j(x_1,\dots,x_j)=(-1)^{|x_{j-1}|}b_j(m;x_1,\dots, x_j),$$
%%
%%but we will have to test it. If it fails, then I would try to use the $n=4$ case of the $A_\infty$-equation to deduce the conditions for the definition of $M_4$.
%
%\appendix
%\renewcommand{\appendixname}{Appendix:}
\begin{appendices}
\appendix
\gdef\thesection{Appendix \Alph{section}}
\section{Some proofs and details}



\begin{lemma}\label{binom}
For any integers $n$ and $m$, the following equiality  holds mod 2:

$$\binom{n+m-1}{2}+\binom{n}{2}+\binom{m}{2}=(n-1)(m-1).$$
\end{lemma}
\begin{proof}
Let us compute 

$$\binom{n+m-1}{2}+\binom{n}{2}+\binom{m}{2}+(n-1)(m-1)\mod 2.$$

By definition, this equals

\begin{gather*}
\frac{(n+m-1)(n+m-2)}{2}+\frac{n(n-1)}{2}+\frac{m(m-1)}{2}+(n-1)(m-1)=\\
\frac{(n^2+2nm-2n+m^2-2m-n-m+2)+(n^2-n)+(m^2-m)+2(nm-n-m+1)}{2}=\\
n^2+2nm-3n+m^2-3m+2=0\mod 2
\end{gather*}
as wanted.


\end{proof}

\begin{lemma}
$\mathfrak{s}^{-1}\mathfrak{s}\OO\cong\OO\cong\mathfrak{s}\mathfrak{s}^{-1}\OO$.
\end{lemma}
\begin{proof}
We are only showing the first isomorphism since the other one is analogous. We only need to look at the isomorphism
\begin{align*}
(\mathcal{O}(n)\otimes\Sigma^{n-1}sig_n\otimes \Sigma^{1-n}sig_n)\otimes (\mathcal{O}(m)\otimes\Sigma^{m-1}sig_m\otimes \Sigma^{1-m}sig_m)\cong\\ (\mathcal{O}(m)\otimes \mathcal{O}(m))\otimes (\Sigma^{n-1}sig_n\otimes \Sigma^{m-1}sig_m)\otimes (\Sigma^{1-n}sig_n\otimes \Sigma^{1-m}sig_m).
\end{align*}
After insertions, the only sign that do not cancel is $(-1)^{(n-1)(m-1)}$. So we need to find an automorphism $f$ of $\OO$ such that, for $a\in\OO(n)$ and $b\in\OO(m)$,

$$f(a\circ_i b)=(-1)^{(n-1)(m-1)}f(a)\circ_i f(b).$$

By the previous lemma it can be checked that $f(a)=(-1)^{\binom{n}{2}}a$ is such an automorphism.
%It can be checked that $f(a)=(-1)^{\frac{n(n+1)}{2}+1}a$ is such an automorphism.
\end{proof}



Recall that we define the \emph{suspension} or \emph{shift} of a graded vector space $V$ as the graded vector space $S V$ having degree components $(S V)^i=V^{i-1}$.

\begin{theorem}\label{proofthm}
There is an isomorphism of operads $\End_{S V}\cong \mathfrak{s}^{-1}\End_V$.
\end{theorem}
\begin{proof}
For each $n$, we clearly have an isomorphism 

$$\End_{S V}=\Hom((S V)^{\otimes n},S V)\cong\Hom(V^{\otimes n},V)\otimes S^{1-n}sig_n= \mathfrak{s}^{-1}\End_V$$

given by the map $\sigma^{-1}$ defined before as $\sigma^{-1}(F)=(-1)^{\binom{n}{2}}S^{-1}\circ F\circ S^{\otimes n}$. We must show that this map is an isomorphism of operads, in other words, it commutes with insertions and with the symmetric group action.

Let us first check that $\sigma^{-1}$ commutes with insertions. Let $F\in \End_{S V}(n)$ and $G\in \End_{S V}(m)$. On the one had we have 

$$\sigma^{-1}(F\circ_i G)=(-1)^{\binom{n+m-1}{2}+|G|(i-1)}S^{-1}\circ F(S^{\otimes i-1}\otimes G(S^{\otimes m})\otimes S^{\otimes n-i}),$$

and on the other hand

$$\sigma^{-1}(F)\tilde{\circ}_i\sigma^{-1}(G)=(-1)^{(n-1)(m-1)+(n-1)(|G|+m-1)+(i-1)(m-1)}\sigma^{-1}(F)\circ_i\sigma^{-1}(G)=$$
$$(-1)^{\binom{n}{2}+\binom{m}{2}+(n-1)(m-1)+(n-1)(|G|+m-1)+(i-1)(m-1)+(|G|+m-1)(n-i)}\Sigma^{-1}\circ F(S^{\otimes i-1}\otimes G(S^{\otimes m})\otimes S^{\otimes n-i}).$$

By lemma \ref{binom}, 

$$\binom{n+m-1}{2}=\binom{n}{2}+\binom{m}{2}+(n-1)(m-1)\mod 2,$$

so we only need to check the equation

$$|G|(i-1)=(n-1)(|G|+m-1)+(i-1)(m-1)+(|G|+m-1)(n-i)\mod 2,$$

and this is straightforward.

Now we are going to show that $\sigma^{-1}$ commutes with the action of th symmetric group. Recall that on $\End_{S V}$ is the usual action, whilst on $\mathfrak{s}^{-1}\End_V$ the action is twisted by the sign of the permutation. It is enough to show this for transpositions of the form $\tau=(i\ i+1)$ since they generate the symmetric group.

On the one hand, 

$$\sigma^{-1}(F\tau)(v_1\otimes\cdots\otimes v_n)=(-1)^{\sum_{j=1}^n (n-j)v_j}S^{-1}\circ (F\tau)(S v_1\otimes\cdots\otimes S v_n)=$$

$$(-1)^{\sum_{j=1}^n (n-j)v_j+(v_i-1)(v_{i+1}-1)}S^{-1}\circ F(S v_1\otimes\cdots\otimes S v_{i+1}\otimes S v_i\otimes\cdots\otimes S v_n).$$

On the other hand

$$(\sigma^{-1}(F)\tau) (v_1\otimes\cdots\otimes v_n)=(-1)^{v_iv_{i+1}-1}S^{-1}\circ F\circ S^{\otimes n}(v_1\otimes\cdots\otimes v_{i+1}\otimes v_i\otimes\cdots\otimes v_n)=$$

$$(-1)^{v_iv_{i+1}-1+\sum_{j\neq i,i+1}(n-j)v_j +(n-i-1)v_i+(n-i)v_{i+1}}S^{-1}\circ f(S v_1\otimes\cdots\otimes S v_{i+1}\otimes S v_i\otimes\cdots\otimes S v_n).$$

Now we just have to check that the signs are the same. Modulo $2$, the sign of the first map is

$$v_iv_{i+1}+v_i+v_{i+1}-1+\sum_{j=1}^n(n-j)v_j=$$
$$v_iv_{i+1}-1+\sum_{j\neq i,i+1}^n(n-j)v_j+(n-i-1)v_i+(n-i)v_{i+1},$$

which indeed coincide with the sign on the second map.

%\url{https://mathoverflow.net/questions/366792/detailed-proof-of-mathfraks-1-mathrmend-v-cong-mathrmend-sigma-v}
\end{proof}

\section{Sign of the braces}\label{rw}

REWRITE THIS PROPERLY

Let us use the same strategy as R-W for the signs of the bracket $[f,g]$. Let $V$ be a graded vector space and $f\in C^{N,i}(V,V)=\hom(C^{\otimes N},V)^i$. Let $S(V)$ be the graded vector space with $S(V)^v=V^{v+1}$, and so the suspension map $S:V\to S(V)$ given by the identity map has internal degree $-1$. Define $\sigma(f)$ as the map making the following diagram commutative
\[
\begin{tikzcd}
S(V)^{\otimes N}\arrow[r, "\sigma(f)"]\arrow[d, "(S^{-1})^{\otimes N}"'] & S(V)\\
V^{\otimes N}\arrow[r,"f"] & V\arrow[u, "S"']
\end{tikzcd}
\]

Explicitly, $\sigma(f)=S\circ f\circ (S^{-1})^{\otimes N}\in C^{N,i+N-1}(V,V)$. 

In R-W there is a sign $(-1)^{N+i-1}$ in front of $f$ but it seems to be irrelevant for this purpose. 

Notice that, by the Koszul sign rule $(S^{-1})^{\otimes N}\circ S^{\otimes N}=(-1)^{\sum_{j=1}^{N-1} j}Id=(-1)^{\frac{N(N-1)}{2}}Id=(-1)^{\binom{N}{2}}Id$, so $(S^{-1})^{\otimes N}= (-1)^{\binom{N}{2}}(S^{\otimes N})^{-1}$. For this reason, given $F\in C^{m,j}(S(V),S(V))$, we have
\[
\sigma^{-1}(F)=(-1)^{\binom{m}{2}}S^{-1}\circ F\circ S^{\otimes m}\in C^{m,j-m+1}(V,V).
\]

For $g_j\in C^{a_j,q_j}(V,V)$, let $$f[g_1,\dots, g_n]=\sum_{k_0+\cdots+k_n=N-n}f(1^{\otimes k_0}\otimes g_1\otimes 1^{\otimes k_1}\otimes\cdots\otimes g_n\otimes 1^{\otimes k_n})\in C^{N-n+\sum a_j, i+\sum q_j}(V,V).$$

We define $f\{g_1,\dots, g_n\}=\sigma^{-1}(\sigma(f)[\sigma(g_1),\dots, \sigma(g_n)])\in C^{N-n+\sum a_j, i+N-1+\sum (q_j+a_j-1)}(V,V).$

This is the same as $f[g_1,\dots, g_n]$ up to sign, so let us compute the sign.
\[
\sigma^{-1}(\sigma(f)[\sigma(g_1),\dots, \sigma(g_n)])=(-1)^{\binom{N-n+\sum a_j}{2}}S^{-1}\circ (\sigma(f)(1^{\otimes k_0}\otimes \sigma(g_1)\otimes 1^{\otimes k_1}\otimes\cdots\otimes \sigma(g_n)\otimes 1^{\otimes k_n}))\circ S^{\otimes N-n+\sum a_j}
\]
\[
=(-1)^{\binom{N-n+\sum a_j}{2}}S^{-1}\circ S\circ f\circ (S^{-1})^{\otimes N}\circ (1^{\otimes k_0}\otimes (S\circ g_1\circ (S^{-1})^{\otimes a_1})\otimes 1^{\otimes k_1}\otimes\cdots\otimes (S\circ g_n\circ (S^{-1})^{\otimes a_n})\otimes 1^{\otimes k_n}))\circ S^{\otimes N-n+\sum a_j}
\]
\begin{align*}
&=(-1)^{\binom{N-n+\sum a_j}{2}}f\circ ((S^{-1})^{k_0}\otimes  S^{-1}\otimes\cdots \otimes  S^{-1}\otimes  (S^{-1})^{k_n})\\ &\circ(1^{\otimes k_0}\otimes (S\circ g_1\circ (S^{-1})^{\otimes a_1})\otimes\cdots\otimes (S\circ g_n\circ (S^{-1})^{\otimes a_n})\otimes 1^{\otimes k_n}))\circ S^{\otimes N-n+\sum a_j}
\end{align*}




Now we move each $1^{\otimes k_{j-1}}\otimes S\circ g_j\circ (S^{-1})^{a_j}$ to apply $(S^{-1})^{k_{j-1}}\otimes S^{-1}$ to it. Doing this to all of them produces a sign

\[
(-1)^{(a_1+q_1-1)(n-1+\sum k_l)+(a_2+q_2-1)(n-2+\sum_2^n k_l)+\cdots+(a_n+q_n-1)k_n}=(-1)^{\sum_{j=1}^n (a_j+q_j-1)(n-j+\sum_j^n k_l)},
\]
 which we call $(-1)^{\varepsilon}$ in the meantime. So now we have, decomposing $S^{\otimes N-n+\sum a_j}$
 
 \[
 (-1)^{\binom{N-n+\sum a_j}{2}+\varepsilon}f\circ((S^{-1})^{k_0}\otimes  g_1\circ (S^{-1})^{\otimes a_1}\otimes\cdots \otimes  g_n\circ (S^{-1})^{\otimes a_n}\otimes  (S^{-1})^{k_n})\circ (S^{\otimes k_0}\otimes S^{\otimes a_1}\otimes\cdots\otimes S^{\otimes a_n}\otimes S^{\otimes k_n}).
 \]
 
 Now we turn the tensor of inverses into inverses of tensors by introducing the appropriate signs. More precisely we introduce the sign
 \[
 (-1)^{\delta}=(-1)^{\binom{k_0}{2}+\sum(\binom{a_j}{2}+\binom{k_j}{2})}
 \]
 
  
So we now have
\[
 (-1)^{\binom{N-n+\sum a_j}{2}+\varepsilon+\delta}f\circ((S^{k_0})^{-1}\otimes  g_1\circ (S^{\otimes a_1})^{-1}\otimes\cdots \otimes  g_n\circ (S^{\otimes a_n})^{-1}\otimes  (S^{k_n})^{-1})\circ (S^{\otimes k_0}\otimes S^{\otimes a_1}\otimes\cdots\otimes S^{\otimes a_n}\otimes S^{\otimes k_n})
 \]
 And the next step is moving each component of the last tensor product in front of its inverse. This will produce the sign $(-1)^\gamma$, where
 
 \begin{gather*}\gamma=-k_0\sum_1^n(k_j+a_j+q_j)-a_1(\sum_1^n k_j+\sum_2^n (a_j+q_j))-\cdots -a_nk_n\equiv\\ \sum_{j=0}^nk_j\sum_{l=j+1}^n(k_l+a_l+q_l)+\sum_{j=1}^na_j(\sum_{l=j}^nk_l+\sum_{l=j+1}^n(a_l+q_l)).
 \end{gather*}
 

 
 So in the end we have
 \[
 f\{g_1,\dots,g_n\}=\sum_{k_0+\cdots+k_n=N-n} (-1)^{\binom{N-n+\sum a_j}{2}+\varepsilon+\delta+\gamma}f(1^{\otimes k_0}\otimes g_1\otimes 1^{\otimes k_1}\otimes\cdots\otimes g_n\otimes 1^{\otimes k_n}).
 \]
 This sign is going to be called $(-1)^{\sigma}$. Next, we are going to simplify this sign to get rid of the binomial coefficients.
 
 \begin{remark}
If anything that goes at the top of a binomial coefficient is less that 2, then the coefficient is 0. In the case of arities or $k_j$ this is because $(S^{-1})^{\otimes 1}=(S^{\otimes 1})^{-1}$ (and if the tensor is taken 0 times then it is the identity and the equality also holds, so there are no signs).
\end{remark}
\subsection{Simplifying sign}


Notice that $N-n+\sum a_j=\sum k_i +\sum a_j$. In general, consider a finite sum $\sum b_i$. We can simplify $\mod 2$ the binomial coefficients

$$\binom{\sum b_i}{2}+\sum\binom{b_i}{2}$$

in the followin way. Note that all terms will appear squared once in the big binomial coefficient and once in the sum, as so will do the terms themselves, so they will cancel. This will leave the double products which cancel out the 2 in the denominator, so the result of applying this to $\binom{N-n+\sum a_j}{2}$ in our sign $\sigma$ is

$$\sum_{0\leq i<l\leq n}k_ik_l+\sum_{1\leq j<l\leq n}a_ja_l+\sum_{i,j}k_ia_j.$$

Recall $\gamma$ in the sign

$$\gamma= \sum_{j=0}^nk_j\sum_{l=j+1}^n(k_l+a_l+q_l)+\sum_{j=1}^na_j(\sum_{l=j}^nk_l+\sum_{l=j+1}^n(a_l+q_l)).$$

As we see, all the sums in the previous simplification appear in $\gamma$ so we can cancel them. Let us rewrite $\gamma$ in a way that this becomes more clear:

$$\sum_{0\leq j<l\leq n}k_jk_l+\sum_{0\leq j<l\leq n}k_ja_l+\sum_{0\leq j<l\leq n}k_jq_l+\sum_{1\leq j\leq l\leq n}a_jk_l+\sum_{1\leq j<l\leq n}a_ja_l+\sum_{1\leq j<l\leq n}a_jq_l.$$

So after simplifying we have only the terms that include the internal degrees, i.e.
$$\sum_{0\leq j<l\leq n}k_jq_l+\sum_{1\leq j<l\leq n}a_jq_l.$$

Let us move now to the $\varepsilon$ term in the sign. 
$$\varepsilon=\sum_{j=1}^n (a_j+q_j-1)(n-j+\sum_j^n k_l)=\sum_{j=1}^n (a_j+q_j-1)(n-j)+\sum_{1\leq j\leq l\leq n} (a_j+q_j-1)k_l$$

We may add this to what we have so far in such a way that the brace sign becomes

$$\sigma=\sum_{0\leq j<l\leq n}k_jq_l+\sum_{1\leq j<l\leq n}a_jq_l+\sum_{j=1}^n (a_j+q_j-1)(n-j)+\sum_{1\leq j\leq l\leq n} (a_j+q_j-1)k_l.$$

%
%\section{On the degree of $M_j$ and Koszul rule}\label{Ab1}
%
%Here we discuss the necessity of using the total degree, which becomes natural in the shift of the operadic supension $\Sigma\s\OO$. 
%
%
%Let $\mathcal{O}=\prod_n\OO(n)$ be an operad in a graded category with an $A_\infty$-multiplication $m=m_1+m_2+\cdots$. We denote by $\OO(n)_p$ the degree $p$ component of $\OO(n)$ and define the \emph{total degree} of an element $f\in \OO(n)_p$ as $||f||=n+p=a(f)+\deg(f)$, where $a(f)=n$ is the \emph{(operadic) arity} of $f$ and to $\deg(f)=p$ is the \emph{internal degree} of $f$. 
%
%
%
%The classical way to define an $A_\infty$-algebra structure on $\OO$ from $m$ is defining
%
%$$M_n(x_1,\dots, x_n)=b_n(m;x_1,\dots, x_n)=\sum_{j\geq n}b_n(m_j;x_1,\dots, x_n)$$
%
%for $n>1$ and 
%
%$$M_1(x)=[m,x]=b_1(m;x)-(-1)^{||x||-1}b_1(x;m)=\sum_j b_1(m_j;x)-(-1)^{||x||-1}\sum_jb_1(x;m_j).$$ 
%
%
%This construction can be iterated to an $A_\infty$ structure on $\End_\OO$ with an analogue definition of maps $\overline{M}_i$ 
%However, to distinguish the braces on $\End_\OO$ from those on $\OO$, the notation $B_n$ is used instead of $b_n$. Namely, if $n>1$,  
%$$\overline{M}_n(f_1,\dots, f_n)=B_n(M;f_1,\dots, f_n)= B_n(M;f_1,\dots, f_n)$$
%
%and
%
%$$\overline{M}_1(x)=[M,f]=B_1(M;f)-(-1)^{||f||-1}B_1(f;M).$$ 
%
%\subsection{Degree and arity considerations}
%
%We have to make sure that $a(M_j)=j$ and $\deg(M_j)=2-j$, considering the operadic arity and the internal degree as those measured in $End_\OO$ provided that $\OO$ has the total degree. The first equality is clear. To show the second we compute $||M_j(x_1,\dots, x_j)||$ since the internal degree of $M_j$ depends on the grading of $\OO$, on which we have defined a grading in terms of the total degree. To compute this quantity, let us define $M_j^l=b_j(m_l;x_1,\dots, x_j)$, which is a summand of $M_j(x_1,\dots, x_j)$. Now we have 
%
%$$a(M_j^l)=l-j+\sum_i a(x_i)$$
%
%and
%
%$$\deg(M_j^l)=\deg(m_l)+\sum_i\deg(x_i)=2-l+\sum_i \deg(x_i).$$ 
%
%These are the operadic arity and internal degree in $\OO$, so $$||M_j^l||=2-j+\sum_i(a(x_i)+\deg(x_i))=2-j+\sum_i||x_i||.$$ 
%
%This is independent of $l$, and therefore we see that $\deg(M_j)=2-j$, and the same argument is valid for $\overline{M}_j$.
%
%Therefore, it is natural to define $M_j\in\End_{\Sigma\s\OO}$. The suspension $\s\OO$ provide us with the signs we need and the additional shift produces the degree that we need. It can be checked that with other possible ``total'' degree conventions such us $a(x)+\deg(x)-1$, $a(x)-\deg(x)+1$, $a(x)-\deg(x)$ or $a(x)-\deg(x)+2$ (coming respectively from $\s\OO$, $\s^{-1}\OO$, $\Sigma^{-1}\s^{-1}\OO$ and $\Sigma\s^{-1}\OO$), the maps $M_j$ don't have the required degree.
%
%\begin{remark}\label{remark3}
%
%
%Assuming $M_j\in \End_{\Sigma\mathfrak{s}\OO}$, it has been proved that it is possible to define it so that $\deg(M)=2-j$ (and obviously the arity is $j$). So if I have to apply the Koszul rule here, the degree used is just $2-j$. If we get to define $M_j\in\mathfrak{s}\End_{\Sigma\mathfrak{s}\OO}$, then $M_j$ is actually $M_j\otimes e_J$ where $e_J=e_1\land\dots\land e_j$ has degree $j-1$. So 
%
%$$M_j\otimes e_J(x_1,\dots, x_j)=(-1)^{(j-1)(||x_1||+\cdots+||x_j||)}M_j(x_1,\dots, x_j)\otimes e_J$$
%being $||x||$ the total degree (the natural degree on $\Sigma\mathfrak{s}\OO$, recall that $M_j$ wa defined via composition on this odd operad). So passing by the $M_j$ component would yield a sign depending on its internal degree, i.e. $2-j$.
%
%For instance, if in the associative case we define $M_2$ such that $$0=M_2\tilde{\circ}M_2=M_2\tilde{\circ}_2 M_2+M_2\tilde{\circ}_1 M_2$$ in the suspension, evaluating at $(x,y,z)$ gives us on the first summand
%
%$$(M_2\tilde{\circ}_2M_2)(x,y,z)=(M_2(1,M_2(1,1))\otimes (e_1\land e_2\land e_3))(x,y,z)=(-1)^{(||x||+||y||+||z||)(3-1)}M_2(x,M_2(y,z))$$
%
%and on the second summand
%$$(M_2\tilde{\circ}_1M_2)(x,y,z)=-(M_2(M_2(1,1),1)\otimes (e_1\land e_2\land e_3))(x,y,z)=-(-1)^{(||x||+||y||+||z||)(3-1)}M_2(x,M_2(y,z))$$
%
%Adding the two of them equals zero so we get the associativity condition $M_2(x,M_2(y,z))=M_2(M_2(x,y),z)$. Note that here $x$ is beeing permuted with $M_2$ but no extra signs appears, which is equivalent to apply the Koszul rule with the internal degree of $M_2$ in $\End_{\Sigma\s\OO}$, which is $2-2=0$, and is in fact what we have done in the evaluation.
%
%\end{remark}
\end{appendices}

\end{document}
