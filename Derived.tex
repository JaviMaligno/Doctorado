	\documentclass[twoside]{article}
\usepackage{estilo-ejercicios}
\setcounter{section}{0}
\newtheorem{defin}{Definition}[section]
\newtheorem{lem}[defin]{Lemma}
\newtheorem{propo}[defin]{Proposition}
\newtheorem{thm}[defin]{Theorem}
\newtheorem{eje}[defin]{Example}
\renewcommand{\baselinestretch}{1,3}

\usepackage{empheq}
\newcommand*\widefbox[1]{\fbox{\hspace{2em}#1\hspace{2em}}}
%--------------------------------------------------------
\begin{document}



Let $A=\bigoplus_{i,k} A^i_k$ be a bigraded module and let us say we have a linear map $\circ :A\otimes A\to A$ of bidegree $(0,0)$. This operation satisfies certain associativity conditions of operads with respect to the braiding morphism $\gamma(a\otimes b)=(-1)^{ij+kl}b\otimes a$ for $a\in A^i_k$ and $b\in A^j_l$ (we say that $a$ has bidegree $(k,i)$ and $b$ has bidegree $(l,j)$, where the first component is the horizontal degree and the second component is the vertical degree). 

A morphism $f:A\to B$ that preserves this structure is a linear map $f:A\to B$ of bidegree $(q,p)$ such that $f(a\circ b)=(-1)^{pi+qk}f(a)\circ f(b)$.



If I modify the operation by $a\bar{\circ}b=(-1)^{l(l+j)}a\circ b$, I get the same associativity relation with respect to the braiding $\gamma'(a\otimes b)=(-1)^{(i+k)(j+l)}b\otimes a$. I would like to define $\bar{f}=\pm f$ so that $\bar{f}(a\bar{\circ}b)=(-1)^{(p+q)(i+k)}\bar{f}(a)\bar{\circ}\bar{f}(b)$, in other words $\bar{f}$ preserves $\bar{\circ}$ with respect to the braiding $\gamma'$. The sign factor of $\bar{f}$ should depend on the bidegree components of $f$ and its input.


Attemp

Let us say $f(a)=(-1)^{\varepsilon(a)}f(a)$. Now we have

$$\bar{f}(a\bar{\circ} b)=(-1)^{k(j+l)}\bar{f}(a\circ b)=(-1)^{k(j+l)+\varepsilon(a\circ b)}f(a\circ b).$$

Using that $f$ preserves $\circ$ with respect to $\gamma$ we have

$$\bar{f}(a\bar{\circ} b)=(-1)^{k(j+l)+\varepsilon(a\circ b)+pi+qk}f(a)\circ f(b).$$

Since $f$ has bidegree $(q,p)$, $f(a)$ has bidegree $(k+q,i+p)$ and $f(b)$ has bidegree $(l+q,j+p)$ so now we have

$$\bar{f}(a\bar{\circ} b)=(-1)^{k(l+j)+\varepsilon(a\circ b)+pi+qk+(l+q)(i+p+k+q)}f(a)\bar{\circ} f(b).$$

Finally, we have 

$$\bar{f}(a\bar{\circ} b)=(-1)^{k(l+j)+\varepsilon(a\circ b)+pi+qk+(l+q)(i+p+k+q)+\varepsilon(a)+\varepsilon(b)}\bar{f}(a)\bar{\circ} \bar{f}(b).$$

And therefore that huge exponent should have the same parity as $(p+q)(i+k)=pi+pk+qi+qk$. However, cancelling some equal terms this implies

$$\varepsilon(a\circ b)+\varepsilon(a)+\varepsilon(b)+kj+l(i+p+q)+q(p+k+q)\equiv pk.$$

Any reasonable choice of $\varepsilon(a)$ seems to imply $\varepsilon(a\circ b)=\varepsilon(a)+\varepsilon(b)$ since the bidegree of $a\circ b$ is $(k+l,i+j)$, so all the $\varepsilon$'s would vanish and the resulting equation modulo 2 would not be true in general.

Am I making any mistake or it is not possible to define such $\bar{f}$?


%\title{Brace structure on operads}
%\author{Javier Aguilar Martín}
%\maketitle
%
%\section{Introduction}
%I am trying to generalize the operadic suspension to a bigraded version that contextualizes derived $A_\infty$-algebras.
%
%
%\section{Bigraded operadic suspension}
%I WILL HAVE TO CHANGE SOME NOTATION TO DISTINGUISH THIS FROM THE GRADED ONE. HOWEVER IF THE OPERAD IS CONCENTRATED IN HORIZONTAL DEGREE 0 THE EFFECT ON INSERTIONS IS THE SAME AS THE GRADED ONE, SO MAYBE THE ABUSE OF NOTATION IS JUSTIFIED
%
%MAKE COMMAND FOR VERTICAL AND HORIZONTAL DEGREE
%
%The operadic suspension defined here is a bigraded generalization of the operadic suspension I defined previously.
%%Everything should be valid for R-modules (char not 2, as in fields). The sign representation would have to be a free R-module of rank 1
%
%Fix a commutative ring with unit $R$ of characteristic distinct of $2$. All tensor products taken over $R$. %for a commutative (at least with 1\neq 0) ring the rank is well defined, in general it is not
%
%Let $sig_n$ be the sign representation of the symmetric group on $n$ symbols concentrated in bidegree $(0,0)$. This is a free $R$-module of rank one that comes with a natural action of the symmetric group $S_n$ that multiplies each element by the sign of each given permutation. Let $\Lambda(n)=S^{1,n-1}sig_n$, where $S^{i,j}$ is a shift of bigraded modules, so that $\Lambda(n)$ is the sign representation of the symmetric group concentrated in bidegree $(1,n-1)$. This module can be realized as the free $R$-module of rank one spanned by the exterior power $e_1\land\cdots\land e_n$ of bidegree $(1,n-1)$.
%
%
%
%The operad structure on the bigraded $\Lambda=\{\Lambda(n)\}_{n\geq 0}$ is the same as in the graded case, namely
%
%\[
%\begin{tikzcd}
%\Lambda(n)\otimes\Lambda(m) \arrow[r, "\circ_i"] & \Lambda(n+m-1)\\
%(e_1\land\cdots\land e_n)\otimes(e_1\land\cdots\land e_m)\arrow[r, mapsto] & (-1)^{(n-i)(m-1)}e_1\land\cdots\land e_{n+m-1}.
%\end{tikzcd}
%\]
%
%
%
%In a similar way we can define $\Lambda^-(n)=S^{-1,1-n}sig_n$, with the same insertion maps.
%%The sign might arise naturally from the permutation action. If I have the wedge of n wedge the wedge of m-1 (because the final result must be n+m-1 in total), I would permute the last m-1 until the reach the i-th position via transpositions, each transpotision produces a minus sign. Or simply considering the lat m as a single element of degree m-1 being permuted in the wedge
%\begin{definition}
%Let $\mathcal{O}$ be a bigraded linear operad, i.e. an operad on the category of bigraded $R$-modules. The \emph{bigraded operadic suspension} $\mathfrak{s}\OO$ of $\mathcal{O}$ is given arity-wise by the Hadamard product of operads, in other words, $\mathfrak{s}\OO(n)=(\mathcal{O}\otimes\Lambda)(n)=\mathcal{O}(n)\otimes\Lambda(n)$ with diagonal composition and symmetric group action. Similarly, we define the \emph{operadic desuspension} $\mathfrak{s}^{-1}\OO(n)=\mathcal{O}(n)\otimes\Lambda^-(n)$.
%\end{definition}
%
%
%We may identify the elements of $\mathcal{O}$ with the elements the elements of $\mathfrak{s}\OO$. For $a\in\OO(n)$ of bidegree $(h\deg(a),v\deg(a))$, its ``natural'' bidegree in $\s\OO$ is \[|a|=(h\deg(a)+1,v\deg(a)+n-1).\] To distinguish both degrees we call $(h\deg(a),v\deg(a))$ the \emph{internal bidegree} of $a$, since this is the degree that $a$ inherits from the grading of $\OO$. If we write $\circ_i$ for the operadic insertion on $\OO$ and $\tilde{\circ}_i$ for the operadic insertion on $\mathfrak{s}\OO$, we may find a relation between the two insertion maps in the following way. Let $a\in\OO(n)$ and $b\in\OO(m)$, and let us compute $a\tilde{\circ}_i b$.
%
%\begin{align*}
%\mathfrak{s}\OO(n)\otimes\mathfrak{s}\OO(m)&=(\OO(n)\otimes\Lambda(n))\otimes (\OO(m)\otimes\Lambda(m))\cong (\OO(n)\otimes \OO(m))\otimes (\Lambda(n)\otimes \Lambda(m))\\
%&\xrightarrow{\circ_i\otimes\circ_i} \OO(m+n-1)\otimes \Lambda(n+m-1)=\mathfrak{s}\OO(n+m-1).
%\end{align*}
%
%The symmetric monoidal structure produces the sign $(-1)^{(n-1)v\deg(b)+h\deg(b)}$ in the isomorphism $\Lambda(n)\otimes \OO(m)\cong\OO(m)\otimes\Lambda(n)$, and the operadic structure of $\Lambda$ produces the sign $(-1)^{(n-i)(m-1)}$, so 
%
%\begin{equation}\label{sign}
%a\tilde{\circ}_ib=(-1)^{(n-1)v\deg(b)+h\deg(b)+(n-i)(m-1)}a\circ_i b.
%\end{equation}
%
%If $h\deg(b)=0$ we recover the sign of the classical operadic suspension. Note that the bracket induced in the bigraded case by this insertion is not exactly the same as in \cite{RW}, but that is expected since they did not define the derived structure as $[m,m]=0$. 
%
%IS THERE AN ANALOGUE TO THIS THEOREM?
%\begin{thm}\label{markl}(\cite[Chapter 3, Lemma 3.16]{operads})
%Given a graded $R$-module $A$, there is an isomorphism of operads $\End_{S A}\cong \mathfrak{s}^{-1}\End_A$, where $\End_A$ is the endomorphism operad of $A$.
%\end{thm}
% 
%
%\section{Derived $A_\infty$-algebras}
%Using the notation in \cite{RW}, a \emph{derived $A_\infty$-algebra} on a $(\N,\Z)$-bigraded $R$-module $A$ consist of a family of $R$-linear maps 
%\[m_{ij}:A^{\otimes j}\to A\]
%of bidegree $(i,2-(i+j))$ for each $j\geq 1$, $i\geq 0$, satisfying the equation
%\[\underset{j=r+1+t}{\sum_{u=i+p, v=j+q-1}}(-1)^{rq+t+pj}m_{ij}(1^{\otimes r}\otimes m_{pq}\otimes 1^{\otimes t})=0\]
%for all $u\geq 0$ and $v\geq 1$. Let us see that the bigraded operadic suspension defined above recovers this sign. We have from (\ref{sign})
%
%\[m_{ij}\tilde{\circ}_{r+1}m_{pq}=(-1)^{(j-r-1)(q-1)+(j-1)(p+q)+p}m_{ij}\circ m_{pq}.\]
%
%
%Using that $t=j-r-1$ we may multiply and simplify the exponent modulo $2$ to obtain
%\[m_{ij}\tilde{\circ}_{r+1}m_{pq}=(-1)^{rq+t+pj}m_{ij}\circ m_{pq}\]
%as we wanted. This means that a derived $A_\infty$-algebra can now be understood as an element $m=\sum_{i,j}m_{ij}\in\s\OO$ such that $m\tilde{\circ}m=0$.
%
%
%\subsection{Braces}\label{sectionbraces}
%First recall the definition of a brace algebra.
%
%\begin{defi}\label{braces}
%A brace algebra on a graded module $A$ consists of a family of maps \[b_n:A^{\otimes 1+n}\to A\] called \emph{braces}, that we evaluate on $(x,x_1,\dots, x_n)$ as $b_n(x;x_1,\dots, x_n)$, satisfying the \emph{brace relation}
%
%
%\begin{align*}
%b_m(b_n(x;x_1,\dots, x_n);y_1,\dots,y_m)=&\\
%\sum_{i_1,\dots, i_n, j_1\dots, j_n}(-1)^{\varepsilon}&b_l(x; y_1,\dots, y_{i_1},b_{j_1}(x_1;y_{i_1+1},\dots, y_{i_1+j_1}),\dots, b_{j_n}(x_n;y_{i_n+1},\dots, y_{i_n+j_n}),\dots,y_m)
%\end{align*}
%where $l=n+\sum_{p=1}^n i_p$ and $\varepsilon=\sum_{p=1}^n|x_p|\sum_{q=i}^{i_p}|y_q|,$ i.e. the sign is picked up by the $x_i$'s passing by the $y_i$'s in the shuffle.
%
%
%
%\end{defi}
%
%\begin{remark}
%Some authors might use the notation $b_{1+n}$ instead of $b_n$, but the first element is usually going to have a different role than the others. A shorter notation for $b_n(x;x_1,\dots,x_n)$ found in the literature is $x\{x_1,\dots, x_n\}$. Also note that we have used the notation $|x_p|$ for the degree of $x_p$ in $V$. The reason is that our main example for $V$ is going to be the operadic suspension of an operad.
%\end{remark}
%
%We can define braces on $\s\OO$ via operadic composition. More precisely, we define the maps 
%$$b_n:\mathfrak{s}\OO(N)\otimes\mathfrak{s}\OO(a_1)\otimes\cdots\otimes\mathfrak{s}\OO(a_n)\to\mathfrak{s}\OO(N-\sum a_i)$$
%using the operadic composition $\gamma$ on $\mathfrak{s}\OO$ as
%
%\[b_n(f;g_1,\dots,g_n)=\sum\gamma(f;1,\dots,1,g_1,1,\dots,1,g_n,1,\dots,1),\]
%
%where the sum runs over all possible ordering preserving insertions. The brace $b_n(f;g_1,\dots,g_n)$ vanishes whenever $n>N$ and $b_0(f)=f$.
%
%INTERLUDE: USING BRACES TO DEFINE A DERIVED AINFTY ALGEBRA ON THE OPERAD
%
%Let $m_{il}$ a component of the derived $A_\infty$-multiplication $m$ and let us compute the arity and bidegree of $b_j(m_{il};x_1,\dots, x_j)$. As usual the arity is $l-j+\sum a(x_k)$. The horizontal degree is just $i+\sum h\deg(x_k)$. The vertical degree $2-i-l+\sum v\deg(x_k)$. According to our construction, the horizontal degree is independent of the arity, so will need to take that into account. In our construction is also natural to add arity and vertical degree, which adds up to $2-(i+j)+\sum (a(x_k)+v\deg(x_k))$. So if we consider the shift $S^{-1,1}\s\OO$, we may define
%
%\[M_{ij}(x_1,\dots, x_j)=\sum_l b_j(m_{il};x_1,\dots, x_j)\]
%\[M_{01}(x)= \sum_l (b_1(m_{0l};x)-(-1)^{hdeg(x)+1+vdeg(x)+a(x)-1}b_1(x;m_{0l}))\]
%to be a derived structure on that shift. CONSIDER CHANGING THE BIGRADED OPERADIC SUSPENSION SO THAT THE SHIFT IS 1,1 In the particular case that the maps are concentrated in horizontal degree 0 ($A_\infty$-algebra) we DON'T FUCKING recover the $A_\infty$-structure BECAUSE THE BRACKET GIVES THE OPPOSITE SIGN   Adding the 3 parameters we get $2-j+\sum all$ so under some grading it might even be an $A_\infty$ algebra, but I doubt this graded could be obtained operadically since the horizontal degree is independent of the arity, unlike the vertical degree. I CAN SIMPLY OMMIT THE $+1$ BY DEFINITION AND FUCK OFF, BUT THAT MIGHT BE INCONSISTENT WITH FURTHER DEVELOPMENTS
%
%PROVISIONAL END OF INTERLUDE
%
%  To obtain the signs that make this composition differ from the composition in $\OO$, we must first look at the operadic composition on $\Lambda$. In the case of the endomorphism operad, we are interested in compositions of the form \[f(1^{\otimes k_0}\otimes g_1\otimes 1^{\otimes k_1}\otimes\cdots\otimes g_n\otimes 1^{\otimes k_n})\] where $N-n=k_0+\cdots+k_n$, $f$ has arity $N$ and each $g_i$ has arity $a_i$ and internal bidegree $(hq_i,vq_i)$. Therefore, let us consider the operadic composition
%
%\[
%\begin{tikzcd}
%\Lambda(N)\otimes\Lambda(1)^{k_0}\otimes\Lambda(a_1)\otimes\Lambda(1)^{\otimes k_1}\otimes\cdots\otimes\Lambda(a_n)\otimes\Lambda(1)^{k_n}\arrow[r] & \Lambda(N-n+\sum_{i=1}^na_i)
%\end{tikzcd}
%\]
%
%The operadic composition can be described in terms of insertions in the obvious way, namely I THINK THIS PART IS THE SAME AS IN THE GRADED CASE BECAUSE THE INSERTIONS ARE THE SAME
%
%$$\gamma(f;h_1,\dots, h_N)=(\cdots(f\circ_1 h_1)\circ_{1+a(h_1)}h_2\cdots)\circ_{1+\sum a(h_p)}h_N,$$
%
%where $a(h_p)$ is the arity of $h_p$ (in this case $h_p$ is either $1$ or some $g_i$). So we just have to find out the sign iterating the same argument as in the $i$-th insertion. In this case, each $\Lambda(a_i)$ produces a sign given by the exponent $$(a_i-1)(N-k_0+\cdots-k_{i-1}-i).$$ 
%
%For this, recall that the degree of $\Lambda(a_i)$ is $a_i-1$ and that the generator of this space is inserted in the position $1+\sum_{j=0}^{i-1}k_j+\sum_{j=1}^{i-1}a_j$ of a wedge of $N+\sum_{j=1}^{i-1}a_j-i+1$ generators. Therefore, performing this insertion as described in the previous section yields the aforementioned sign. Now, since $N-n=k_0+\cdots+k_n$, we have that
%\[(a_i-1)(N-k_0+\cdots+k_{i-1}-i)=(a_i-1)(n-i+\sum_{l=i}^nk_l).\]
%
%Now we can compute the sign factor of a brace. I WILL HAVE TO ADD HORIZONTAL DEGREES HERE For this, notice that the isomorphism $(\OO(1)\otimes \Lambda(1))^{\otimes k}\cong \OO(1)^{\otimes k}\otimes \Lambda(1)^{\otimes k}$ SO NOW THIS WILL ADD SIGNS IN HORIZONTAL DEGREE IN GENERAL, BUT IF I'M USING THE UNIT I THINK IT SHOULD HAVE BIDEGREE (0,0) does not produce any signs because of degree reasons. Therefore, therefore the sign coming from the isomorphism
%
%\[\OO(N)\otimes\Lambda(N)\otimes (\OO(1)\otimes \Lambda(1))^{\otimes k_0}\bigotimes_{i=1}^n(\OO(a_i)\otimes\Lambda(a_i)\otimes(\OO(1)\otimes\Lambda(1))^{\otimes k_i}\]
%\[\cong \OO(N)\otimes(\bigotimes_{i=1}^n \OO(a_i))\otimes \Lambda(N)\otimes(\bigotimes_{i=1}^n \Lambda(a_i))\]
%is determined by the exponent
%
%\[(N-1)\sum_{i=1}^nq_i+\sum_{i=1}^n (a_i-1)\sum_{l>i}q_l.\]
%
%This equals
%\[(\sum_{j=0}^nk_j +n-1)\sum_{i=1}^nq_i+\sum_{i=1}^n (a_i-1)\sum_{l>i}q_l.\]
%
%After doing the composition 
%\[\OO(N)\otimes(\bigotimes_{i=1}^n \OO(a_i))\otimes \Lambda(N)\otimes(\bigotimes_{i=1}^n \Lambda(a_i))\longrightarrow \OO(N-n+\sum_{i=1}^na_i)\otimes \Lambda(N-n+\sum_{i=1}^na_i)\]
%
%we can add the sign coming from the suspension, so all in all the sign for the braces is
%
%\[\sum_{i=1}^n(a_i-1)(n-i+\sum_{l=i}^nk_l)+(\sum_{j=0}^nk_j +n-1)\sum_{i=1}^nq_i+\sum_{i=1}^n (a_i-1)\sum_{l>i}q_l.\]
%
%It can be checked that this can be rewritten $\mod 2$ as 
%\[\eta\coloneqq\sum_{0\leq j<l\leq n}k_jq_l+\sum_{1\leq j<l\leq n}a_jq_l+\sum_{j=1}^n (a_j+q_j-1)(n-j)+\sum_{1\leq j\leq l\leq n} (a_j+q_j-1)k_l.\]
%
%This is the sign that is obtained following \cite{RW} in \ref{rw} (expression \ref{sigma}). In particular, $b_1(f;g)$ is the operation $f\circ g$ defined in \cite{RW} that induces a Lie bracket. More precisely,
%\[
%[f,g]=b_1(f;g)-(-1)^{|f||g|}b_1(g;f)
%\]
%is the same bracket that was defined in \cite{RW}. However, we may use $f\tilde{\circ}g$ to make clear that we are using the operadic composition in $\s\OO$. Note that
%
%\[
%b_1(f;g)=\sum_i f\tilde{\circ}_i g,
%\]
%so the notation $f\tilde{\circ} g$ is suggestive, and $f\circ g$ might be left to a general case of any operad that is not necessarily an operadic suspension or to the specific case of the endomorphism operad with no additional signs.
%
%
%%The hardest part is finding \sum q_j(n-j) in the first expression. There is (n-1)\sum q_j+sum_{j>i}q_j. So q_l apperas (n-1) times first and then l-1 times (because the nequality is strict). Therefore, q_l appears n-1+l-1=n-l mod 2 times.
%
%
%
%%\subsection{Advantages of this approach}
%%First of all, we get an easier way to obtain the signs and the brace relation follows easily. In addition, this explanation fits better in the context of operads and feels more natural.
%%
%%Furthermore, since we have an isomorphism of operads $\End_{\Sigma A}\cong \mathfrak{s}^{-1}\End_A$, we can translate if needed, results from an operad to its desuspension, which has the same signs in composition as the suspension, but with opposite grading. We can also use this isomorphism if we define maps on $\s\End_{\Sigma A}$, since this is then isomorphic to $\End_A$, which is the naïve Hochschild complex.
%%
%%%\section{$A_\infty$-structure on $\End_{\Sigma\mathfrak{s}\OO}$}
%%%
%%%Let $\Sigma\s\OO$ be the shift as a graded vector space of $\s\OO$. For an element $x\in\Sigma\s\OO$ let us write $||x||$ for its total degree (the natural degree in this case, arity plus internal degree) and $|x|=||x||-1$ for its \emph{reduced degree} (which is the natural degree in $\s\OO$). We had defined the maps $M_j:(\Sigma\s\OO)^{\otimes j}\to\Sigma\s\OO$ by 
%%%
%%%$$M_j(x_1,\dots,x_j)=b_j(m;x_1,\dots, x_j)$$
%%%
%%%for $j>1$ and
%%%
%%%$$M_1(x)=b_1(m;x)-(-1)^{|x|}b_1(x;m).$$
%%%
%%%We know that $M_j$ must be defined on $\Sigma\s\OO$ to be of degree $2-j$ because it must take the total degree, i.e. $M_j\in\End_{\Sigma\s\OO}$ (see \ref{Ab1}). 
%%%
%%%By Getzler we know that these maps define an $A_\infty$-structure on $\End_{\Sigma\s\OO}$ in the sense of $M\circ M=0$ for the operadic composition on $\End_{\Sigma\s\OO}$ (without signs). If we use the operad isomorphism $\sigma^{-1}:\End_{\Sigma\s\OO}\cong\s^{-1}\End_{\s\OO}$ we can obtain the relation $\sigma^{-1}(M)\tilde{\circ}\sigma^{-1}(M)=0$ (now with the signs we normally use). But the problem is that if we want to define an $A_\infty$-structure on $\overline{M}_j$ on $\s^{-1}\End_{\s\OO}$, we face the problem of degree. Again, $\overline{M}_j$ must take the total degree. But desuspending substracts the arity instead of adding it (and a shift up or down doesn't fix this). In addition, this solution is not totally satisfying since $\sigma^{-1}(M)$ is defined in terms of maps from other operad. 
%%%
%%%
%%%So the alternative is redefining the maps $M_j$ to obtain some maps $M_j'$ that satisfy $M'\tilde{\circ}M'=0$, so that $M_j'$ can be seen as elements of $\s\End_{\Sigma\OO}$ and the new map $\overline{M}_j$ can be defined on the shift of this operad.
%%%
%%%The strategy is similar to the sign twist in the dg-case, where the associative product was defined as $xy=(-1)^{|x|}b_2(m;x,y)$. In particular, $M_2'(x,y)=(-1)^{|x|}b_2(m;x,y)$.
%%%
%%%\begin{remark}
%%%Another possibility is sending $\sigma^{-1}(M_j)\in\s^{-1}\End_{\s\OO}$ to $\s\End_{\s\OO}$. The signs are the same and this identification consists of adding some exterior products, so it doesn't really modify the map $\sigma^{-1}(M_j)$ or the operadic composition. The problem is that it gives the opposite degree: if $\sigma^{-1}(M_j)$ has degree $2-j$ in $\s^{-1}\End_{\s\OO}$ then it has degree $j$ when seen as an element of $\s\End_{\s\OO}$.
%%%\end{remark}
%%%
%%%\subsection{Redefining the maps}
%%%I am going to use the notation $M_j$ for what I've called $M_j'$ before since we are going to be interested only in these new maps. $M_1$ remains unmodified and $M_2$ has already been defined as $M_2(x,y)=(-1)^{|x|}b_2(m;x,y)$. We want to define $M_j$ for $j\geq 3$ such that for each decomposition $n=r+s+t$ we have
%%%
%%%$$\sum_n (-1)^{rs+t}M_{r+1+t}(1^{\otimes r}\otimes M_s\otimes 1^{\otimes t})=0.$$
%%%
%%%For $n=1,2$ we already know that this relation is satisfied since only $M_1$ and $M_2$ are involved. We are going to look at the case $n=3$ to define $M_3$. Since we are going to modify $b_3(m;x,y,z)$ by a sign depending on the elements involved, we need to rewrite the above relation after applying it to elements. 
%%%
%%%\begin{remark}
%%%Let $\mathcal{P}=\s\End_{\Sigma\s\OO}$.  If $f\in\mathcal{P}(n)$, then $f=f'\otimes (e_1\land\cdots\land e_n)$, so $$f(x_1,\dots,x_n)=(-1)^{(n-1)\sum_i ||x_i||}f'(x_1,\dots,x_n)\otimes(e_1\land\cdots\land e_n).$$
%%%
%%%Fortunately, this sign is not going to be relevant in our equations since it's the same for any two maps of the same arity and we will be able to cancel it. More precisely, for each fixed $n$,
%%%
%%%$$0=\sum_{r+s+t=n}(-1)^{rs+t}M_{r+1+t}(1^{\otimes r}\otimes M_s\otimes 1^{t})(x_1,\dots, x_n)=$$
%%%$$(-1)^{(n-1)\sum_i||x_i||}(-1)^{(2-s)\sum_{i=1}^r||x_i||}(-1)^{rs+t}M_{r+1+s}(x_1,\dots, x_r, M_s(x_{r+1},\dots, x_{r+s}), x_{r+s+1},\dots, x_n)$$
%%%
%%%so we can cancel the factor $(-1)^{(n-1)\sum_i||x_i||}$. Note that the Koszul rule applied here takes the total degree, since that is the degree on $\Sigma\s\OO$, where the maps are defined (more about this in \ref{remark3}). In particular, the Leibniz rule takes the form of
%%%
%%%$$M_1(M_2(x,y))=M_2(x, M_1(x))+(-1)^{||x||}M_2(x,M_1(y)).$$
%%%
%%%The total degree in the sign is consistent with the oddity that we originally found. In particular, we already know that this relation holds, which is the $A_\infty$-equation for $n=2$. For $n=1$ it is just $M_1(M_1(x))=0$, that we also know (more about this in \ref{remark3}). We have to be careful because the reduced degree is also going to appear in the operadic relations such as the brace relation.
%%%\end{remark}
%%%\subsection{Definition of $M_3$}
%%%We are going to define $M_3(x,y,z)=(-1)^{\varepsilon(x,y,z)}b_3(m;x,y,z)$ and find necessary conditions that $\varepsilon(x,y,z)$ must satisfy. To do that we look at the $A_\infty$-equation for $n=3$. Before proceding, let us impose some previous conditions on $\varepsilon(x,y,z)$. It should depend only on the total or reduced degree of $x$, $y$ and $z$. In particular, it should not distinguish between $b_1(m;x)$ and $b_1(x;m)$, so we may define $\varepsilon(M_1(x),y,z)$ and so on. We denote 
%%%\begin{gather*}
%%%\varepsilon_1\coloneqq\varepsilon(M_1(x),y,z),\\
%%%\varepsilon_2\coloneqq\varepsilon(x,M_1(y),z),\\
%%%\varepsilon_3\coloneqq\varepsilon(x,y,M_1(z)).\\
%%%\varepsilon\coloneqq\varepsilon(x,y,z)
%%%\end{gather*}
%%%And now let us look at the $A_\infty$-equation for $n=3$, which is
%%%
%%%\begin{align*}
%%%M_3(M_1(x),y,z)+(-1)^{||x||}M_3(x,M_1(y),z)+(-1)^{||x||+||y||}M_3(x,y,M_1(z))\\
%%%-M_2(M_2(x,y),z)+M_2(x,M_2(y,z))+M_1(M_3(x,y,z))=0.
%%%\end{align*}
%%%
%%%First we apply the definitions of $M_1$ and $M_2$.
%%%\begin{align*}
%%%M_3(b_1(m;x),y,z)+(-1)^{||x||}M_3(x,b_1(m;y),z)+(-1)^{||x||+||y||}M_3(x,y,b_1(m;z))\\
%%%-(-1)^{|x|}M_3(b_1(x;m),y,z)-(-1)^{|y|+||x||}M_3(x,b_1(y;m),z)-(-1)^{|z|+||x||+||y||}M_3(x,y,b_1(z;m))\\
%%%-(-1)^{|y|+1}b_2(m;b_2(m;x,y),z)+(-1)^{|x|+|y|}b_2(m;x,b_2(m;y,z))\\+b_1(m;M_3(x,y,z))-(-1)^{|x|+|y|+|z|+1}b_1(M_3(x,y,z);m)=0.
%%%\end{align*}
%%%And now we apply the definition of $M_3(x,y,z)$.
%%%
%%%\begin{align*}
%%%(-1)^{\varepsilon_1}b_3(m;b_1(m;x),y,z)+(-1)^{||x||+\varepsilon_2}b_3(m;x,b_1(m;y),z)+(-1)^{||x||+||y||+\varepsilon_3}b_3(m;x,y,b_1(m;z))\\
%%%-(-1)^{|x|+\varepsilon_1}b_3(m;b_1(x;m),y,z)-(-1)^{|y|+||x||+\varepsilon_2}b_3(m;x,b_1(y;m),z)-(-1)^{|z|+||x||+||y||+\varepsilon_3}b_3(m;x,y,b_1(z;m))\\
%%%+(-1)^{|y|}b_2(b_2(m;x,y),z)+(-1)^{|x|+|y|}b_2(m;x,b_2(m;y,z))\\+(-1)^{\varepsilon}b_1(m;b_3(m;x,y,z))+(-1)^{|x|+|y|+|z|+\varepsilon}b_1(b_3(m;x,y,z);m)=0.
%%%\end{align*}
%%%
%%%Following Getzler's proof of $M\circ M=0$, we next apply the brace relation to the last term $(-1)^{|x|+|y|+|z|+\varepsilon}b_1(b_3(m;x,y,z);m)$. We will after that impose the cancellation of the second line of the equation above to obtain some conditions on the signs.
%%%
%%%\begin{align*}
%%%(-1)^{|x|+|y|+|z|+\varepsilon}b_1(b_3(m;x,y,z);m)=
%%%&(-1)^{|x|+|y|+|z|+\varepsilon}b_4(m;x,y,z,m)+(-1)^{|x|+|y|+|z|+\varepsilon}b_3(m;x,y,b_1(z;m))\\
%%%&+(-1)^{|x|+|y|+\varepsilon}b_4(m;x,y,m,z)+(-1)^{|x|+|y|+\varepsilon}b_3(m;x,b_1(y;m),z)\\
%%%&+(-1)^{|x|+\varepsilon}b_4(m;x,m,y,z)+(-1)^{|x|+\varepsilon}b_3(m;b_1(x;m),y,z)\\
%%%&+(-1)^{\varepsilon}b_4(m;m,x,y,z).
%%%\end{align*}
%%%The conditions modulo 2 that we get from the cancellation condition are the following:
%%%
%%%\begin{gather}
%%%|x|+\varepsilon_1=|x|+\varepsilon\Rightarrow\varepsilon_1=\varepsilon\\
%%%|y|+||x||+\varepsilon_2=|x|+|y|+\varepsilon\Rightarrow \varepsilon_2=\varepsilon-1\\
%%%|z|+||x||+||y||+\varepsilon_3=|x|+|y|+|z|+\varepsilon\Rightarrow\varepsilon_3=\varepsilon
%%%\end{gather}
%%%From condition (1) and (3) we deduce $\varepsilon$ does not depend on the first and third argument, and from condition (2) we deduce $\varepsilon(x,M_1(y),z)=\varepsilon(x,y,z)+1$. Therefore the natural way to define $\varepsilon$ is by $\varepsilon(x,y,z)=|y|$, because $|M_1(y)|=|y|+1$ (defining it as $||y||$ would also do the job, but sticking to $|y|$ will be more convenient).
%%%
%%%Thus, specifying $\varepsilon(x,y,z)=|y|$ in the $A_\infty$-equation together with the brace relation and some simplification of signs gives us
%%%
%%%\begin{align*}
%%%(-1)^{|y|}b_3(m;b_1(m;x),y,z)+(-1)^{|x|+|y|}b_3(m;x,b_1(m;y),z)+(-1)^{|x|}b_3(m;x,y,b_1(m;z))\\
%%%+(-1)^{|y|}b_2(m;b_2(m;x,y),z)+(-1)^{|x|+|y|}b_2(m;x,b_2(m;y,z))+(-1)^{|y|}b_1(m;b_3(x,y,z))\\
%%%+(-1)^{|x|+|z|}b_4(m;x,y,z,m)+(-1)^{|x|}b_4(m;x,y,m,z)\\
%%%+(-1)^{|x|+|y|}b_4(m;x,m,y,z)+(-1)^{|y|}b_4(m;m,x,y,z)=0
%%%\end{align*}
%%%
%%%It can be checked using the brace relation that the above expression equals $(-1)^{|y|}b_3(b_1(m;m);x,y,z)$, so it is indeed 0 since we are assuming that $b_1(m;m)=0$. 
%%%
%%%This shows that $M_3(x,y,z)=(-1)^{|y|}b_3(m;x,y,z)$ is a good definition. The next step would be trying to generalize this to higher maps. So far, the pattern that can be observed is
%%%
%%%$$M_j(x_1,\dots,x_j)=(-1)^{|x_{j-1}|}b_j(m;x_1,\dots, x_j),$$
%%%
%%%but we will have to test it. If it fails, then I would try to use the $n=4$ case of the $A_\infty$-equation to deduce the conditions for the definition of $M_4$.
%%
%%\appendix
%%\renewcommand{\appendixname}{Appendix:}
%\begin{appendices}
%\appendix
%\gdef\thesection{Appendix \Alph{section}}
%\section{Some proofs and details}
%
%
%
%\begin{lemma}\label{binom}
%For any integers $n$ and $m$, the following equiality  holds mod 2:
%
%$$\binom{n+m-1}{2}+\binom{n}{2}+\binom{m}{2}=(n-1)(m-1).$$
%\end{lemma}
%\begin{proof}
%Let us compute 
%
%$$\binom{n+m-1}{2}+\binom{n}{2}+\binom{m}{2}+(n-1)(m-1)\mod 2.$$
%
%By definition, this equals
%
%\begin{gather*}
%\frac{(n+m-1)(n+m-2)}{2}+\frac{n(n-1)}{2}+\frac{m(m-1)}{2}+(n-1)(m-1)=\\
%\frac{(n^2+2nm-2n+m^2-2m-n-m+2)+(n^2-n)+(m^2-m)+2(nm-n-m+1)}{2}=\\
%n^2+2nm-3n+m^2-3m+2=0\mod 2
%\end{gather*}
%as wanted.
%
%
%\end{proof}
%
%I WILL HAVE TO CHECK THIS AGAIN
%
%\begin{lemma}
%There are isomorphisms of operads $\mathfrak{s}^{-1}\mathfrak{s}\OO\cong\OO\cong\mathfrak{s}\mathfrak{s}^{-1}\OO$.
%\end{lemma}
%\begin{proof}
%We are only showing the first isomorphism since the other one is analogous. We only need to look at the isomorphism
%\begin{align*}
%(\mathcal{O}(n)\otimes\Sigma^{n-1}sig_n\otimes \Sigma^{1-n}sig_n)\otimes (\mathcal{O}(m)\otimes\Sigma^{m-1}sig_m\otimes \Sigma^{1-m}sig_m)\cong\\ (\mathcal{O}(m)\otimes \mathcal{O}(m))\otimes (\Sigma^{n-1}sig_n\otimes \Sigma^{m-1}sig_m)\otimes (\Sigma^{1-n}sig_n\otimes \Sigma^{1-m}sig_m).
%\end{align*}
%After insertions, the only sign that do not cancel is $(-1)^{(n-1)(m-1)}$. So we need to find an automorphism $f$ of $\OO$ such that, for $a\in\OO(n)$ and $b\in\OO(m)$,
%
%$$f(a\circ_i b)=(-1)^{(n-1)(m-1)}f(a)\circ_i f(b).$$
%
%By the previous lemma it can be checked that $f(a)=(-1)^{\binom{n}{2}}a$ is such an automorphism.
%%It can be checked that $f(a)=(-1)^{\frac{n(n+1)}{2}+1}a$ is such an automorphism.
%\end{proof}
%
%
%
%Recall that we define the \emph{suspension} or \emph{shift} of a graded module $A$ is the graded module $S A$ having degree components $(S A)^i=A^{i-1}$.
%
%\begin{theorem}\label{proofthm}
%There is an isomorphism of operads $\End_{S A}\cong \mathfrak{s}^{-1}\End_A$.
%\end{theorem}
%\begin{proof}
%For each $n$, we clearly have an isomorphism of graded modules
%
%$$\End_{S A}(n)=\Hom((S A)^{\otimes n},S A)\cong\Hom(A^{\otimes n},A)\otimes S^{1-n}sig_n= \mathfrak{s}^{-1}\End_A(n)$$
%
%given by the map $\sigma^{-1}$ defined before as $\sigma^{-1}(F)=(-1)^{\binom{n}{2}}S^{-1}\circ F\circ S^{\otimes n}$, where $\circ$ denotes the composition of maps. We must show that this map is an isomorphism of operads, in other words, it commutes with insertions and with the symmetric group action.
%
%Let us first check that $\sigma^{-1}$ commutes with insertions. Let $F\in \End_{S A}(n)$ and $G\in \End_{S A}(m)$. On the one had we have 
%
%$$\sigma^{-1}(F\circ_i G)=(-1)^{\binom{n+m-1}{2}+\deg(G)(i-1)}S^{-1}\circ F(S^{\otimes i-1}\otimes G(S^{\otimes m})\otimes S^{\otimes n-i}),$$
%
%and on the other hand
%
%$$\sigma^{-1}(F)\tilde{\circ}_i\sigma^{-1}(G)=(-1)^{(n-1)(m-1)+(n-1)(\deg(G)+m-1)+(i-1)(m-1)}\sigma^{-1}(F)\circ_i\sigma^{-1}(G)=$$
%$$(-1)^{\binom{n}{2}+\binom{m}{2}+(n-1)(m-1)+(n-1)(\deg(G)+m-1)+(i-1)(m-1)+(\deg(G)+m-1)(n-i)}\Sigma^{-1}\circ F(S^{\otimes i-1}\otimes G(S^{\otimes m})\otimes S^{\otimes n-i}).$$
%
%By lemma \ref{binom}, 
%
%$$\binom{n+m-1}{2}=\binom{n}{2}+\binom{m}{2}+(n-1)(m-1)\mod 2,$$
%
%so we only need to check the equation
%
%$$\deg(G)(i-1)=(n-1)(\deg(G)+m-1)+(i-1)(m-1)+(\deg(G)+m-1)(n-i)\mod 2,$$
%
%and for this it is enough to develop the right hand side.
%
%Now we are going to show that $\sigma^{-1}$ commutes with the action of th symmetric group. Recall that on $\End_{S A}$ we have the usual action, whilst on $\mathfrak{s}^{-1}\End_A$ the action is twisted by the sign of the permutation. It is enough to show this for transpositions of the form $\tau=(i\ i+1)$ since they generate the symmetric group.
%
%Let us write $(-1)^v$ for $(-1)^{\deg(v)}$. On the one hand, 
%
%$$\sigma^{-1}(F\tau)(v_1\otimes\cdots\otimes v_n)=(-1)^{\sum_{j=1}^n (n-j)v_j}S^{-1}\circ (F\tau)(S v_1\otimes\cdots\otimes S v_n)=$$
%
%$$(-1)^{\sum_{j=1}^n (n-j)v_j+(v_i-1)(v_{i+1}-1)}S^{-1}\circ F(S v_1\otimes\cdots\otimes S v_{i+1}\otimes S v_i\otimes\cdots\otimes S v_n).$$
%
%On the other hand
%
%$$(\sigma^{-1}(F)\tau) (v_1\otimes\cdots\otimes v_n)=(-1)^{v_iv_{i+1}-1}S^{-1}\circ F\circ S^{\otimes n}(v_1\otimes\cdots\otimes v_{i+1}\otimes v_i\otimes\cdots\otimes v_n)=$$
%
%$$(-1)^{v_iv_{i+1}-1+\sum_{j\neq i,i+1}(n-j)v_j +(n-i-1)v_i+(n-i)v_{i+1}}S^{-1}\circ f(S v_1\otimes\cdots\otimes S v_{i+1}\otimes S v_i\otimes\cdots\otimes S v_n).$$
%
%Now we just have to check that the signs are the same. Modulo $2$, the sign of the first map is
%
%$$v_iv_{i+1}+v_i+v_{i+1}-1+\sum_{j=1}^n(n-j)v_j=$$
%$$v_iv_{i+1}-1+\sum_{j\neq i,i+1}^n(n-j)v_j+(n-i-1)v_i+(n-i)v_{i+1},$$
%
%which indeed coincides with the sign on the second map.
%
%%\url{https://mathoverflow.net/questions/366792/detailed-proof-of-mathfraks-1-mathrmend-v-cong-mathrmend-sigma-v}
%\end{proof}
%
%\begin{remark}
%If in the proof above we replace $S$ with $S^{-1}$, we have that the map
%
%\[\sigma^{-1}(F)=(-1)^{\binom{n}{2}}S^{-1}\circ F\circ S^{\otimes n}\]
% transforms into $(-1)^{\binom{n}{2}}S\circ F\circ (S^{-1})^{\otimes n}=S\circ F\circ (S^{\otimes n})^{-1}$. This is the map $\overline{\sigma}(F)$ from page 9 of \cite{RW}, and following the same proof we have done above but with this change of $S$ into $S^{-1}$ we get the isomorphism of operads
%
%\[
%\overline{\sigma}:\End_A\cong\s\End_{SA}.
%\]
%\end{remark}
%
%\section{Sign of the braces}\label{rw}
%I THINK THERE IS NO EQUIVALENT FOR THIS SINCE THIS ONE WAS ALREADY USED TO THE DERIVED CASE BUT NEEDED TO INTRODUCE SOME EXTRA SHIFT AFTERWARDS
%
%
%Let us use the same strategy as \cite{RW} used to find the signs of the bracket $[f,g]$, but here we are going to use it to find the sign of the braces. Let $A$ be a graded module and $f\in C^{N,i}(A,A)=\hom(A^{\otimes N},A)^i$. Let $S(A)$ be the graded module with $S(A)^v=A^{v+1}$, and so the suspension or \emph{shift} map $S:A\to S(A)$ given by the identity map has internal degree $-1$. Define $\sigma(f)$ as the map making the following diagram commutative
%\[
%\begin{tikzcd}
%S(A)^{\otimes N}\arrow[r, "\sigma(f)"]\arrow[d, "(S^{-1})^{\otimes N}"'] & S(A)\\
%A^{\otimes N}\arrow[r,"f"] & A\arrow[u, "S"']
%\end{tikzcd}
%\]
%
%Explicitly, $\sigma(f)=S\circ f\circ (S^{-1})^{\otimes N}\in C^{N,i+N-1}(A,A)$. 
%
%\begin{remark}
%In \cite{RW} there is a sign $(-1)^{N+i-1}$ in front of $f$ but it seems to be irrelevant for our purposes. Another fact to remark is that the suspension of graded modules used here (and in \cite{RW}) is the opposite that we have used to define the operadic suspension. This does not change the signs or the procedure, but in the statement of theorem \ref{markl} operadic desuspension should be changed to operadic suspension. %My suspensions is better because it gives the total degree %If I modify the theorem to End_{sO}=sEnd_{SsO} I have to change the phrase
%\end{remark}
%
%Notice that, by the Koszul sign rule $(S^{-1})^{\otimes N}\circ S^{\otimes N}=(-1)^{\sum_{j=1}^{N-1} j}Id=(-1)^{\frac{N(N-1)}{2}}Id=(-1)^{\binom{N}{2}}Id$, so $(S^{-1})^{\otimes N}= (-1)^{\binom{N}{2}}(S^{\otimes N})^{-1}$. For this reason, given $F\in C^{m,j}(S(A),S(A))$, we have
%\[
%\sigma^{-1}(F)=(-1)^{\binom{m}{2}}S^{-1}\circ F\circ S^{\otimes m}\in C^{m,j-m+1}(A,A).
%\]
%
%For $g_j\in C^{a_j,q_j}(A,A)$, let us write $f[g_1,\dots, g_n]$ for the map \[\sum_{k_0+\cdots+k_n=N-n}f(1^{\otimes k_0}\otimes g_1\otimes 1^{\otimes k_1}\otimes\cdots\otimes g_n\otimes 1^{\otimes k_n})\in C^{N-n+\sum a_j, i+\sum q_j}(A,A).\]
%
%We define \[b_n(f;g_1,\dots, g_n)=\sigma^{-1}(\sigma(f)[\sigma(g_1),\dots, \sigma(g_n)])\in C^{N-n+\sum a_j, i+\sum q_j}(A,A),\] so that
%%recall the degree of \sigma^{-1}(F)
%\[b_n(f;g_1,\dots, g_n)=(-1)^{\eta}f[g_1,\dots, g_n].\]
%
%We will see that this $b_n(f;g_1,\dots, g_n)$ is the same as in Definition \ref{braces}. The purpose of this Appendix is to find $\eta$, so let us compute it.
%\begin{align*}
%&\sigma^{-1}(\sigma(f)[\sigma(g_1),\dots, \sigma(g_n)])=\\ &=(-1)^{\binom{N-n+\sum a_j}{2}}S^{-1}\circ (\sigma(f)(1^{\otimes k_0}\otimes \sigma(g_1)\otimes 1^{\otimes k_1}\otimes\cdots\otimes \sigma(g_n)\otimes 1^{\otimes k_n}))\circ S^{\otimes N-n+\sum a_j}\\
%&=(-1)^{\binom{N-n+\sum a_j}{2}}S^{-1}\circ S\circ f\circ (S^{-1})^{\otimes N}\circ \\ &(1^{\otimes k_0}\otimes (S\circ g_1\circ (S^{-1})^{\otimes a_1})\otimes 1^{\otimes k_1}\otimes\cdots\otimes (S\circ g_n\circ (S^{-1})^{\otimes a_n})\otimes 1^{\otimes k_n}))\circ  S^{\otimes N-n+\sum a_j}\\
%&=(-1)^{\binom{N-n+\sum a_j}{2}}f\circ ((S^{-1})^{k_0}\otimes  S^{-1}\otimes\cdots \otimes  S^{-1}\otimes  (S^{-1})^{k_n})\\ &\circ(1^{\otimes k_0}\otimes (S\circ g_1\circ (S^{-1})^{\otimes a_1})\otimes\cdots\otimes (S\circ g_n\circ (S^{-1})^{\otimes a_n})\otimes 1^{\otimes k_n}))\circ S^{\otimes N-n+\sum a_j}.
%\end{align*}
%
%
%
%
%Now we move each $1^{\otimes k_{j-1}}\otimes S\circ g_j\circ (S^{-1})^{a_j}$ to apply $(S^{-1})^{k_{j-1}}\otimes S^{-1}$ to it. Doing this to all of them produces a sign
%
%\[
%(-1)^{(a_1+q_1-1)(n-1+\sum k_l)+(a_2+q_2-1)(n-2+\sum_2^n k_l)+\cdots+(a_n+q_n-1)k_n}=(-1)^{\sum_{j=1}^n (a_j+q_j-1)(n-j+\sum_j^n k_l)},
%\]
% and we call the exponent
% 
% $$\varepsilon=\sum_{j=1}^n (a_j+q_j-1)(n-j+\sum_j^n k_l).$$ So now we have, decomposing $S^{\otimes N-n+\sum a_j}$,
% 
% \[
% (-1)^{\binom{N-n+\sum a_j}{2}+\varepsilon}f\circ((S^{-1})^{k_0}\otimes  g_1\circ (S^{-1})^{\otimes a_1}\otimes\cdots \otimes  g_n\circ (S^{-1})^{\otimes a_n}\otimes  (S^{-1})^{k_n})\circ (S^{\otimes k_0}\otimes S^{\otimes a_1}\otimes\cdots\otimes S^{\otimes a_n}\otimes S^{\otimes k_n}).
% \]
% 
% Now we turn the tensor of inverses into inverses of tensors by introducing the appropriate signs. More precisely we introduce the sign
% \begin{equation}\label{delta}
% (-1)^{\delta}=(-1)^{\binom{k_0}{2}+\sum(\binom{a_j}{2}+\binom{k_j}{2})}
%  \end{equation}
% 
%  
%So we now have
%\[
% (-1)^{\binom{N-n+\sum a_j}{2}+\varepsilon+\delta}f\circ((S^{k_0})^{-1}\otimes  g_1\circ (S^{\otimes a_1})^{-1}\otimes\cdots \otimes  g_n\circ (S^{\otimes a_n})^{-1}\otimes  (S^{k_n})^{-1})\circ (S^{\otimes k_0}\otimes S^{\otimes a_1}\otimes\cdots\otimes S^{\otimes a_n}\otimes S^{\otimes k_n})
% \]
% And the next step is moving each component of the last tensor product in front of its inverse. This will produce the sign $(-1)^\gamma$, where
% 
% \begin{gather*}\gamma=-k_0\sum_1^n(k_j+a_j+q_j)-a_1(\sum_1^n k_j+\sum_2^n (a_j+q_j))-\cdots -a_nk_n\equiv\\ \sum_{j=0}^nk_j\sum_{l=j+1}^n(k_l+a_l+q_l)+\sum_{j=1}^na_j(\sum_{l=j}^nk_l+\sum_{l=j+1}^n(a_l+q_l)).
% \end{gather*}
% 
%
% 
% So in the end we have
% \[
% b_n(f;g_1,\dots,g_n)=\sum_{k_0+\cdots+k_n=N-n} (-1)^{\binom{N-n+\sum a_j}{2}+\varepsilon+\delta+\gamma}f(1^{\otimes k_0}\otimes g_1\otimes 1^{\otimes k_1}\otimes\cdots\otimes g_n\otimes 1^{\otimes k_n}).
% \]
%This means that 
% $$\eta=\binom{N-n+\sum a_j}{2}+\varepsilon+\delta+\gamma.$$
%  Next, we are going to simplify this sign to get rid of the binomial coefficients.
% 
% \begin{remark}
%If the number top of a binomial coefficient is less than 2, then the coefficient is 0. In the case of arities or $k_j$ this is because $(S^{-1})^{\otimes 1}=(S^{\otimes 1})^{-1}$ (and if the tensor is taken 0 times then it is the identity and the equality also holds, so there are no signs).
%\end{remark}
%
%
%\subsection{Simplifying sign}
%
%
%Notice that $N-n+\sum a_j=\sum k_i +\sum a_j$. In general, consider a finite sum $\sum b_i$. We can simplify $\mod 2$ the binomial coefficients
%
%$$\binom{\sum b_i}{2}+\sum\binom{b_i}{2}$$
%
%in the followin way. Note that all terms will appear squared once in the big binomial coefficient and once in the sum, as so will do the terms themselves, so they will cancel. This will leave the double products which cancel out the 2 in the denominator. More precisely, we have the following equality $\mod 2$:
%
%$$\binom{\sum b_i}{2}+\sum\binom{b_i}{2}=\sum_{i<j}b_ib_j.$$
%So the result of applying this to $\binom{N-n+\sum a_j}{2}+\delta$ (recall $\delta$ from \ref{delta}) in our sign $\eta$ is
%
%\begin{equation}\label{simply}
%\sum_{0\leq i<l\leq n}k_ik_l+\sum_{1\leq j<l\leq n}a_ja_l+\sum_{i,j}k_ia_j.
%\end{equation}
%
%Recall $\gamma$ in the sign:
%
%\begin{equation*}\label{gamma}
%\gamma= \sum_{j=0}^nk_j\sum_{l=j+1}^n(k_l+a_l+q_l)+\sum_{j=1}^na_j(\sum_{l=j}^nk_l+\sum_{l=j+1}^n(a_l+q_l)).
%\end{equation*}
%
%As we see, all the sums in the previous simplification appear in $\gamma$ so we can cancel them. Let us rewrite $\gamma$ in a way that this becomes more clear:
%
%$$\sum_{0\leq j<l\leq n}k_jk_l+\sum_{0\leq j<l\leq n}k_ja_l+\sum_{0\leq j<l\leq n}k_jq_l+\sum_{1\leq j\leq l\leq n}a_jk_l+\sum_{1\leq j<l\leq n}a_ja_l+\sum_{1\leq j<l\leq n}a_jq_l.$$
%
%So after adding the expression \ref{simply} modulo 2 we have only the terms that include the internal degrees, i.e.
%\begin{equation}\label{sofar}
%\sum_{0\leq j<l\leq n}k_jq_l+\sum_{1\leq j<l\leq n}a_jq_l.
%\end{equation}
%Let us move now to the $\varepsilon$ term in the sign to rewrite it. 
%$$\varepsilon=\sum_{j=1}^n (a_j+q_j-1)(n-j+\sum_j^n k_l)=\sum_{j=1}^n (a_j+q_j-1)(n-j)+\sum_{1\leq j\leq l\leq n} (a_j+q_j-1)k_l$$
%
%We may add this to what we had in \ref{sofar} in such a way that the brace sign becomes
%
%\begin{equation}\label{sigma}
%\eta=\sum_{0\leq j<l\leq n}k_jq_l+\sum_{1\leq j<l\leq n}a_jq_l+\sum_{j=1}^n (a_j+q_j-1)(n-j)+\sum_{1\leq j\leq l\leq n} (a_j+q_j-1)k_l.
%\end{equation}
%as announced at the end of Section \ref{sectionbraces}.
%%
%%\section{On the degree of $M_j$ and Koszul rule}\label{Ab1}
%%
%%Here we discuss the necessity of using the total degree, which becomes natural in the shift of the operadic supension $\Sigma\s\OO$. 
%%
%%
%%Let $\mathcal{O}=\prod_n\OO(n)$ be an operad in a graded category with an $A_\infty$-multiplication $m=m_1+m_2+\cdots$. We denote by $\OO(n)_p$ the degree $p$ component of $\OO(n)$ and define the \emph{total degree} of an element $f\in \OO(n)_p$ as $||f||=n+p=a(f)+\deg(f)$, where $a(f)=n$ is the \emph{(operadic) arity} of $f$ and to $\deg(f)=p$ is the \emph{internal degree} of $f$. 
%%
%%
%%
%%The classical way to define an $A_\infty$-algebra structure on $\OO$ from $m$ is defining
%%
%%$$M_n(x_1,\dots, x_n)=b_n(m;x_1,\dots, x_n)=\sum_{j\geq n}b_n(m_j;x_1,\dots, x_n)$$
%%
%%for $n>1$ and 
%%
%%$$M_1(x)=[m,x]=b_1(m;x)-(-1)^{||x||-1}b_1(x;m)=\sum_j b_1(m_j;x)-(-1)^{||x||-1}\sum_jb_1(x;m_j).$$ 
%%
%%
%%This construction can be iterated to an $A_\infty$ structure on $\End_\OO$ with an analogue definition of maps $\overline{M}_i$ 
%%However, to distinguish the braces on $\End_\OO$ from those on $\OO$, the notation $B_n$ is used instead of $b_n$. Namely, if $n>1$,  
%%$$\overline{M}_n(f_1,\dots, f_n)=B_n(M;f_1,\dots, f_n)= B_n(M;f_1,\dots, f_n)$$
%%
%%and
%%
%%$$\overline{M}_1(x)=[M,f]=B_1(M;f)-(-1)^{||f||-1}B_1(f;M).$$ 
%%
%%\subsection{Degree and arity considerations}
%%
%%We have to make sure that $a(M_j)=j$ and $\deg(M_j)=2-j$, considering the operadic arity and the internal degree as those measured in $End_\OO$ provided that $\OO$ has the total degree. The first equality is clear. To show the second we compute $||M_j(x_1,\dots, x_j)||$ since the internal degree of $M_j$ depends on the grading of $\OO$, on which we have defined a grading in terms of the total degree. To compute this quantity, let us define $M_j^l=b_j(m_l;x_1,\dots, x_j)$, which is a summand of $M_j(x_1,\dots, x_j)$. Now we have 
%%
%%$$a(M_j^l)=l-j+\sum_i a(x_i)$$
%%
%%and
%%
%%$$\deg(M_j^l)=\deg(m_l)+\sum_i\deg(x_i)=2-l+\sum_i \deg(x_i).$$ 
%%
%%These are the operadic arity and internal degree in $\OO$, so $$||M_j^l||=2-j+\sum_i(a(x_i)+\deg(x_i))=2-j+\sum_i||x_i||.$$ 
%%
%%This is independent of $l$, and therefore we see that $\deg(M_j)=2-j$, and the same argument is valid for $\overline{M}_j$.
%%
%%Therefore, it is natural to define $M_j\in\End_{\Sigma\s\OO}$. The suspension $\s\OO$ provide us with the signs we need and the additional shift produces the degree that we need. It can be checked that with other possible ``total'' degree conventions such us $a(x)+\deg(x)-1$, $a(x)-\deg(x)+1$, $a(x)-\deg(x)$ or $a(x)-\deg(x)+2$ (coming respectively from $\s\OO$, $\s^{-1}\OO$, $\Sigma^{-1}\s^{-1}\OO$ and $\Sigma\s^{-1}\OO$), the maps $M_j$ don't have the required degree.
%%
%%\begin{remark}\label{remark3}
%%
%%
%%Assuming $M_j\in \End_{\Sigma\mathfrak{s}\OO}$, it has been proved that it is possible to define it so that $\deg(M)=2-j$ (and obviously the arity is $j$). So if I have to apply the Koszul rule here, the degree used is just $2-j$. If we get to define $M_j\in\mathfrak{s}\End_{\Sigma\mathfrak{s}\OO}$, then $M_j$ is actually $M_j\otimes e_J$ where $e_J=e_1\land\dots\land e_j$ has degree $j-1$. So 
%%
%%$$M_j\otimes e_J(x_1,\dots, x_j)=(-1)^{(j-1)(||x_1||+\cdots+||x_j||)}M_j(x_1,\dots, x_j)\otimes e_J$$
%%being $||x||$ the total degree (the natural degree on $\Sigma\mathfrak{s}\OO$, recall that $M_j$ wa defined via composition on this odd operad). So passing by the $M_j$ component would yield a sign depending on its internal degree, i.e. $2-j$.
%%
%%For instance, if in the associative case we define $M_2$ such that $$0=M_2\tilde{\circ}M_2=M_2\tilde{\circ}_2 M_2+M_2\tilde{\circ}_1 M_2$$ in the suspension, evaluating at $(x,y,z)$ gives us on the first summand
%%
%%$$(M_2\tilde{\circ}_2M_2)(x,y,z)=(M_2(1,M_2(1,1))\otimes (e_1\land e_2\land e_3))(x,y,z)=(-1)^{(||x||+||y||+||z||)(3-1)}M_2(x,M_2(y,z))$$
%%
%%and on the second summand
%%$$(M_2\tilde{\circ}_1M_2)(x,y,z)=-(M_2(M_2(1,1),1)\otimes (e_1\land e_2\land e_3))(x,y,z)=-(-1)^{(||x||+||y||+||z||)(3-1)}M_2(x,M_2(y,z))$$
%%
%%Adding the two of them equals zero so we get the associativity condition $M_2(x,M_2(y,z))=M_2(M_2(x,y),z)$. Note that here $x$ is beeing permuted with $M_2$ but no extra signs appears, which is equivalent to apply the Koszul rule with the internal degree of $M_2$ in $\End_{\Sigma\s\OO}$, which is $2-2=0$, and is in fact what we have done in the evaluation.
%%
%%\end{remark}
%\end{appendices}
%%\phantomsection
%\bibliographystyle{ieeetr}
%\bibliography{newbibliography}
\end{document}
