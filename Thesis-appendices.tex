\documentclass[Thesis.tex]{subfiles}
\doublespacing
\sloppy
\begin{document}


\chapter*{Appendix}
\addcontentsline{toc}{chapter}{Appendix}
%CONSIDER THIS \url{https://tex.stackexchange.com/questions/37277/book-class-appendix}
%\begin{appendices}
\appendix
\setcounter{equation}{0}
\renewcommand\theequation{\thesection\arabic{equation}}
%\appendixpage
\gdef\thesection{\Alph{section}}
\section{Some proofs and details}\label{AppA}
In this appendix we prove some results that rely on sign calculations and combinatorics.


\begin{lem}\label{binom}
For any integers $n$ and $m$, the following equality holds mod 2.

\[\binom{n+m-1}{2}+\binom{n}{2}+\binom{m}{2}=(n-1)(m-1).\]
\end{lem}
\begin{proof}
Let us compute 

\[\binom{n+m-1}{2}+\binom{n}{2}+\binom{m}{2}+(n-1)(m-1)\mod 2.\]

By definition, this equals

\[\frac{(n+m-1)(n+m-2)}{2}+\frac{n(n-1)}{2}+\frac{m(m-1)}{2}+(n-1)(m-1)\]

Let us expand the above expression into the following.

\begin{align*}
\frac{n^2+2nm-2n+m^2-2m-n-m+2}{2}&+\\
\frac{n^2-n+m^2-m+2(nm-n-m+1)}{2}&=\\
n^2+2nm-3n+m^2-3m+2&=\\
n^2+m+m^2+m&=\\
0\mod 2
\end{align*}
as desired, because $n^2=n\mod 2$.


\end{proof}




Recall that we define the \emph{suspension} or \emph{shift} of a graded module $A$ as the graded module $S A$ having degree components $(S A)^i=A^{i-1}$.

\begin{thm}\label{proofthm}
There is an isomorphism of (symmetric) operads $\End_{S A}\cong \mathfrak{s}^{-1}\End_A$.
\end{thm}
\begin{proof}
For each $n$, we clearly have an isomorphism of graded modules

\begin{align*}
\End_{S A}(n)&=\Hom_R((S A)^{\otimes n},S A)\\
&\cong\Hom_R(A^{\otimes n},A)\otimes S^{1-n}sig_n\\
&= \mathfrak{s}^{-1}\End_A(n)
\end{align*}

given by the map $\sigma^{-1}$ defined before as \[\sigma^{-1}(F)=(-1)^{\binom{n}{2}}S^{-1}\circ F\circ S^{\otimes n},\] where $\circ$ denotes the composition of maps. We must show that this map is an isomorphism of operads, in other words, it commutes with insertions and with the symmetric group action.

Let us first check that $\sigma^{-1}$ commutes with insertions. For that, let $F\in \End_{S A}(n)$ and $G\in \End_{S A}(m)$. On the one had we have 

\[\sigma^{-1}(F\circ_i G)=(-1)^{\binom{n+m-1}{2}+\deg(G)(i-1)}S^{-1}\circ F(S^{\otimes i-1}\otimes G(S^{\otimes m})\otimes S^{\otimes n-i}),\]


and on the other hand

\begin{align*}
\sigma^{-1}(F)\tilde{\circ}_i\sigma^{-1}(G)=&\\
(-1)^{(n-1)(m-1)+(n-1)(\deg(G)+m-1)+(i-1)(m-1)}\sigma^{-1}(F)\circ_i\sigma^{-1}(G)=&\\
(-1)^{\varepsilon}S^{-1}\circ F(S^{\otimes i-1}\otimes G(S^{\otimes m})\otimes S^{\otimes n-i}),&
\end{align*}
where
\begin{align*}
\varepsilon=&\binom{n}{2}+\binom{m}{2}+(n-1)(m-1)+(n-1)(\deg(G)+m-1)\\
&+(i-1)(m-1)+(\deg(G)+m-1)(n-i).
\end{align*}

By \Cref{binom}, 

\[\binom{n+m-1}{2}=\binom{n}{2}+\binom{m}{2}+(n-1)(m-1)\mod 2,\]

so we only need to check that $\deg(G)(i-1)\mod 2$ equals

\[(n-1)(\deg(G)+m-1)+(i-1)(m-1)+(\deg(G)+m-1)(n-i).\]

This can be done by direct computation.

Now we are going to show that $\sigma^{-1}$ commutes with the action of the symmetric group. Recall that on $\End_{S A}$ we have the usual permutation action, whilst on $\mathfrak{s}^{-1}\End_A$ the action is twisted by the sign of the permutation. It is enough to show this for transpositions of the form $\tau=(i\ i+1)$ since they generate the symmetric group.

Let us write $(-1)^v$ for $(-1)^{\deg(v)}$. On the one hand, 

\[\sigma^{-1}(F\tau)(v_1\otimes\cdots\otimes v_n)=(-1)^{\sum_{j=1}^n (n-j)v_j}S^{-1}\circ (F\tau)(S v_1\otimes\cdots\otimes S v_n)\]
Applying $\tau$ we obtain

\begin{equation}\label{firstmap}
(-1)^{\sum_{j=1}^n (n-j)v_j+(v_i-1)(v_{i+1}-1)}S^{-1}\circ F(S v_1\otimes\cdots\otimes S v_{i+1}\otimes S v_i\otimes\cdots\otimes S v_n).
\end{equation}

The sign $(-1)^{\sum_{j=1}^n (n-j)v_j}$ comes from swapping the shift maps $S$ past the $v_j$'s, and the sign $(-1)^{(v_i-1)(v_{i+1}-1)}$ comes from permuting $v_i$ and $v_{i+1}$. On the other hand, performing similar sign computations we have

\begin{align}\label{secondmap}
&(\sigma^{-1}(F)\tau) (v_1\otimes\cdots\otimes v_n)\\
&=(-1)^{v_iv_{i+1}-1}S^{-1}\circ F\circ S^{\otimes n}(v_1\otimes\cdots\otimes v_{i+1}\otimes v_i\otimes\cdots\otimes v_n)\nonumber\\
&=(-1)^{\delta}S^{-1}\circ f(S v_1\otimes\cdots\otimes S v_{i+1}\otimes S v_i\otimes\cdots\otimes S v_n)\nonumber
\end{align}
where $\delta = v_iv_{i+1}-1+\sum_{j\neq i,i+1}(n-j)v_j +(n-i-1)v_i+(n-i)v_{i+1}$.

Now we just have to check that the signs are the same. Modulo $2$, the sign on \Cref{firstmap} is 

\begin{align*}
&v_iv_{i+1}+v_i+v_{i+1}-1+\sum_{j=1}^n(n-j)v_j=\\
&v_iv_{i+1}-1+\sum_{j\neq i,i+1}^n(n-j)v_j+(n-i-1)v_i+(n-i)v_{i+1},
\end{align*}

which indeed coincides with the sign on \Cref{secondmap}.
\end{proof}

\begin{remark}
If in the proof above we replace $S$ with $S^{-1}$, we have that the map

\[\sigma^{-1}(F)=(-1)^{\binom{n}{2}}S^{-1}\circ F\circ S^{\otimes n}\]
 transforms into $(-1)^{\binom{n}{2}}S\circ F\circ (S^{-1})^{\otimes n}=S\circ F\circ (S^{\otimes n})^{-1}$. This is the map $\overline{\sigma}(F)$ from page 9 of \cite{RW}, and following the same proof we have done above but with this change of $S$ into $S^{-1}$ we get the isomorphism of operads

\[
\overline{\sigma}:\End_A\cong\s\End_{SA}.
\]
\end{remark}




\section{Koszul sign on operadic suspension}\label{koszulsigns}
The purpose of this section is to clear up the procedure to apply the Koszul sign rule in situations in which operadic suspension is involved.

Let $\End_A$ be the endomorphism operad of some $R$-module $A$ and consider the operadic suspension $\s\End_A$. We are going to make a few comments on the application of the Koszul rule when applying maps from $\s\End_A(n)$ to elements of $A^{\otimes n}$. Let $f\otimes e^n\in\s\End_A(n)$ be of degree $\deg(f)+n-1$. %(do not be confuse with the notation that we used in Section \ref{functorial}, here $\s f$ is an element of an operad). 
For $a\in A^{\otimes n}$ we have \[(f\otimes e^n)(a)=(-1)^{\deg(a)(n-1)}f(a)\otimes e^n\]

because $\deg(e^n)=n-1$. Note that $f\otimes e^n=g\otimes e^n$ if and only if $f=g$. In addition, it is not possible that $f\otimes e^n=g\otimes e^m$ for $n\neq m$. %the 0  map is a different map oon arity n or m
The reader may notice that $f(a)\otimes  e^n\notin A$, but it can be identified with an element of $S^{n-1}A$. This is a reminiscence of the isomorphism $\s^{-1}\End_A\cong \End_{SA}$. %A map of degree d on SA^n->SA corresponds to a map of degree d-n+1 on A^n->A 
 

If we take the tensor product of two maps $f\otimes e^n\in\s\End_A(n)$ and $g\otimes e^m\in\s\End_A(m)$ and apply it to $a\otimes b\in A^{\otimes n}\otimes A^{\otimes m}$, we have

\begin{align*}
((f\otimes e^n)\otimes &( g\otimes e^m))(a\otimes b)\\
&=(-1)^{\deg(a)(\deg(g)+m-1)}(f\otimes e^n)(a)\otimes( g\otimes e^m)(b)\\
&=(-1)^{\varepsilon}(f(a)\otimes e^n)\otimes(f(b)\otimes e^m),
\end{align*}
where $\varepsilon = \deg(a)(\deg(g)+m-1)+\deg(a)(n-1)+\deg(b)(m-1)$. 
The last remark that we want to make is the case of a map of the form 
\[f(1^{\otimes k-1}\otimes g\otimes 1^{\otimes n-k})\otimes e^{m+n-1}\in\s\End_A(n+m-1),\] 
such as those produced by the operadic insertion $\s f\tilde{\circ}_{k} \s g$. In this case, this map applied to $a_{k-1}\otimes b\otimes a_{n-k}\in A^{\otimes k-1}\otimes A^{\otimes m}\otimes A
^{\otimes n-k}$ resulting in 

\begin{align*}
(f(1^{\otimes k-1}\otimes g\otimes 1^{\otimes n-k})\otimes e^{m+n-1})(a_{k-1}\otimes b\otimes a_{n-k})=&\\
(-1)^{\nu}f(1^{\otimes k-1}\otimes g\otimes 1^{\otimes n-k}(a_{k-1}\otimes b\otimes a_{n-k}))\otimes e^{m+n-1}=&\\
(-1)^{\nu+\deg(a_{k-1})\deg(g)}f(a_{k-1}\otimes g(b)\otimes a_{n-k})\otimes e^{m+n-1}&.
\end{align*}
where $\nu=(m+n)(\deg(a_{k-1})+\deg(b)+\deg(a_{n-k}))$.
To go from the first line to the second, we switch $e^{m+n-1}$ of degree $m+n-2$  with $a_{k-1}\otimes b\otimes a_{n-k}$. To go from the second line to the third we apply the usual sign rule for graded maps.

The purpose of this last remark is not only review the Koszul sign rule but also remind that the insertion $\s f\tilde{\circ}_{k} \s g$ is of the above form, so that the $e^{m+n-1}$ component is always at the end and does not play a role in the application of the sign rule with the composed maps. In other words, it does not affect the individual degrees of the maps, just the degree of the overall composition. %I do this because I made some mistakes with  respect to this

\section{Sign of the braces}\label{rw}



In order to find the sign of the braces on $\s\End_A$, let us use an analogous strategy to the one used in \cite[Appendix]{RW} to find the signs of the Lie bracket $[f,g]$ on $\End_A$.

Let $A$ be a graded module. Let $SA$ be the graded module with $SA^v=A^{v+1}$, and so the \emph{suspension} or \emph{shift} map $S:A\to SA$ given by the identity map has degree $-1$.

 Let $f\in \End_A(N)^i=\Hom_R(A^{\otimes N},A)^i$. Recall that $\sigma$ is the inverse of the map from Theorem \ref{markl}, so that $\sigma(f)$ is defined as the map making the following diagram commute.
\[
\begin{tikzcd}
SA^{\otimes N}\arrow[r, "\sigma(f)"]\arrow[d, "(S^{-1})^{\otimes N}"'] & SA\\
A^{\otimes N}\arrow[r,"f"] & A\arrow[u, "S"']
\end{tikzcd}
\]

Explicitly, $\sigma(f)=S\circ f\circ (S^{-1})^{\otimes N}\in \End_A(N)^{i+N-1}$. 

\begin{remark}
In \cite{RW} there is a sign $(-1)^{N+i-1}$ in front of $f$, but it seems to be irrelevant for our purposes. Another fact to remark on is that the suspension of graded modules used here and in \cite{RW} is the opposite that we have used to define the operadic suspension in the sense that in \Cref{lambda} we used $SA^v=A^{v-1}$. This does not change the signs or the procedure, but in the statement of \Cref{markl}, operadic desuspension should be changed to operadic suspension. %My suspension is better because it gives the total degree
\end{remark}


Notice that by the Koszul sign rule 
\begin{align*}
(S^{-1})^{\otimes N}\circ S^{\otimes N}&=(-1)^{\sum_{j=1}^{N-1} j}1\\
&=(-1)^{\frac{N(N-1)}{2}}1\\
&=(-1)^{\binom{N}{2}}1,
\end{align*}
so $(S^{-1})^{\otimes N}= (-1)^{\binom{N}{2}}(S^{\otimes N})^{-1}$. For this reason, given $F\in \End_{S(A)}(m)^j$, we have
\[
\sigma^{-1}(F)=(-1)^{\binom{m}{2}}S^{-1}\circ F\circ S^{\otimes m}\in \End_A(m)^{j-m+1}.
\]

For $g_j\in \End_A(a_j)^{q_j}$, let us write $f[g_1,\dots, g_n]$ for the map 
\[\sum_{\mathclap{k_0+\cdots+k_n=N-n}}f(1^{\otimes k_0}\otimes g_1\otimes 1^{\otimes k_1}\otimes\cdots\otimes g_n\otimes 1^{\otimes k_n})\in \End_A(N-n+\sum a_j)^{i+\sum q_j}\]

We define $b_n(f;g_1,\dots, g_n)\in \End_A(N-n+\sum a_j)^{i+\sum q_j}$ as
\[b_n(f;g_1,\dots, g_n)=\sigma^{-1}(\sigma(f)[\sigma(g_1),\dots, \sigma(g_n)]).\]
With this the definition we can prove the following.
\begin{lem}\label{rwversion}
 We have
\[b_n(f;g_1,\dots,g_n)=\sum_{N-n=k_0+\cdots+k_n} (-1)^\eta
f(1^{\otimes k_0}\otimes g_1\otimes \cdots\otimes g_n\otimes1^{\otimes k_n}),\]
where 
\[\eta=\sum_{0\leq j<l\leq n}k_jq_l+\sum_{1\leq j<l\leq n}a_jq_l+\sum_{j=1}^n (a_j+q_j-1)(n-j)+\sum_{1\leq j\leq l\leq n} (a_j+q_j-1)k_l.\]
\end{lem} 


 


\begin{proof}
Let us compute $\eta$ using the definition of $b_n$.
\begin{align*}
&\sigma^{-1}(\sigma(f)[\sigma(g_1),\dots, \sigma(g_n)])\\ &=(-1)^{\binom{N-n+\sum a_j}{2}}S^{-1}\circ\\ 
&(\sigma(f)(1^{\otimes k_0}\otimes \sigma(g_1)\otimes 1^{\otimes k_1}\otimes\cdots\otimes \sigma(g_n)\otimes 1^{\otimes k_n}))\circ S^{\otimes N-n+\sum a_j}\\
&=(-1)^{\binom{N-n+\sum a_j}{2}}S^{-1}\circ S\circ f\circ (S^{-1})^{\otimes N}\circ \\ &\left(1^{\otimes k_0}\bigotimes ((S\circ g_i\circ (S^{-1})^{\otimes a_i})\otimes 1^{\otimes k_i})\right)\circ  S^{\otimes N-n+\sum a_j}\\
&=(-1)^{\binom{N-n+\sum a_j}{2}}f\circ ((S^{-1})^{k_0}\otimes  S^{-1}\otimes\cdots \otimes  S^{-1}\otimes  (S^{-1})^{k_n})\circ\\ 
&\left(1^{\otimes k_0}\bigotimes ((S\circ g_i\circ (S^{-1})^{\otimes a_i})\otimes 1^{\otimes k_i})\right)\circ  S^{\otimes N-n+\sum a_j}
\end{align*}
%
%&=(-1)^{\binom{N-n+\sum a_j}{2}}S^{-1}\circ S\circ f\circ (S^{-1})^{\otimes N}\circ \\ &(1^{\otimes k_0}\otimes (S\circ g_1\circ (S^{-1})^{\otimes a_1})\otimes\cdots\otimes (S\circ g_n\circ (S^{-1})^{\otimes a_n})\otimes 1^{\otimes k_n}))\circ  S^{\otimes N-n+\sum a_j}\\


Now we move each $1^{\otimes k_{j-1}}\otimes S\circ g_j\circ (S^{-1})^{a_j}$ to apply $(S^{-1})^{k_{j-1}}\otimes S^{-1}$ to it. Doing this for all $j=1,\dots, n$ produces a sign
\begin{align*}
(-1)^{(a_1+q_1-1)(n-1+\sum k_l)+(a_2+q_2-1)(n-2+\sum_2^n k_l)+\cdots+(a_n+q_n-1)k_n}\\
=(-1)^{\sum_{j=1}^n (a_j+q_j-1)(n-j+\sum_j^n k_l)},
\end{align*}
 and we denote the exponent by
 
 \[\varepsilon=\sum_{j=1}^n (a_j+q_j-1)\left(n-j+\sum_j^n k_l\right).\] So now we have that, decomposing $S^{\otimes N-n+\sum a_j}$, the last map up to multiplication by $(-1)^{\binom{N-n+\sum a_j}{2}+\varepsilon}$ is
 
 \begin{gather*}
 (-1)^{\binom{N-n+\sum a_j}{2}+\varepsilon}f\circ((S^{-1})^{k_0}\otimes  g_1\circ (S^{-1})^{\otimes a_1}\otimes\cdots \otimes  g_n\circ\\ (S^{-1})^{\otimes a_n}\otimes  (S^{-1})^{k_n})\circ (S^{\otimes k_0}\otimes S^{\otimes a_1}\otimes\cdots\otimes S^{\otimes a_n}\otimes S^{\otimes k_n}).
 \end{gather*}
 
 Now we turn the tensor of inverses into inverses of tensors by introducing the appropriate signs. More precisely, we introduce the sign
 \begin{equation}\label{delta}
 (-1)^{\delta}=(-1)^{\binom{k_0}{2}+\sum_j\left(\binom{a_j}{2}+\binom{k_j}{2}\right)}.
  \end{equation}
 
  
Therefore we have up to multiplication by $(-1)^{\binom{N-n+\sum a_j}{2}+\varepsilon+\delta}$ the map
\begin{gather*}
 f\circ((S^{k_0})^{-1}\otimes  g_1\circ (S^{\otimes a_1})^{-1}\otimes\cdots \otimes  g_n\circ (S^{\otimes a_n})^{-1}\otimes  (S^{k_n})^{-1})\circ\\ (S^{\otimes k_0}\otimes S^{\otimes a_1}\otimes\cdots\otimes S^{\otimes a_n}\otimes S^{\otimes k_n}).
 \end{gather*}
 The next step is moving each component of the last tensor product in front of its inverse. This will produce the sign $(-1)^\gamma$, where
 
 \begin{align}\label{gammasign}
 \gamma&=-k_0\sum_1^n(k_j+a_j+q_j)-a_1\left(\sum_1^n k_j+\sum_2^n (a_j+q_j)\right)-\cdots -a_nk_n\\
 &= \sum_{j=0}^nk_j\sum_{l=j+1}^n(k_l+a_l+q_l)+\sum_{j=1}^na_j\left(\sum_{l=j}^nk_l+\sum_{l=j+1}^n(a_l+q_l)\right)\mod 2\nonumber.
 \end{align}
 

 
 So in the end we have
 \[
 b_n(f;g_1,\dots,g_n)=\sum(-1)^{\binom{N-n+\sum a_j}{2}+\varepsilon+\delta+\gamma}f(1^{\otimes k_0}\otimes g_1\otimes\cdots\otimes g_n\otimes 1^{\otimes k_n}).
 \]
This means that 
 \[\eta=\binom{N-n+\sum a_j}{2}+\varepsilon+\delta+\gamma.\]
  Next, we are going to simplify this sign to get rid of the binomial coefficients.
 
 \begin{remark}
If the top number of a binomial coefficient is less than 2, then the coefficient is 0. In the case of arities or $k_j$ this is because $(S^{-1})^{\otimes 1}=(S^{\otimes 1})^{-1}$, and if the tensor is taken 0 times then it is the identity and the equality also holds, so there are no signs.
\end{remark}


We are now going to simplify the sign to obtain the desired result. Notice that $N-n+\sum_j a_j=\sum_i k_i +\sum_j a_j$. In general, consider a finite sum $\sum_i b_i$. We can simplify the binomial coefficients mod 2

\[\binom{\sum_i b_i}{2}+\sum_i\binom{b_i}{2}\]

in the following way. Note that all the $b_i$'s will appear squared once in the big binomial coefficient and once in the sum, as so will do the terms themselves, so they will cancel. This will leave the double products which cancel out the 2 in the denominator. More precisely, we have the following equality mod 2.

\[\binom{\sum b_i}{2}+\sum\binom{b_i}{2}=\sum_{i<j}b_ib_j\mod 2.\]
The result of applying this to $\binom{N-n+\sum a_j}{2}$ and adding $\delta$ from \Cref{delta} in our sign $\eta$ is

\begin{equation}\label{simply}
\sum_{0\leq i<l\leq n}k_ik_l+\sum_{1\leq j<l\leq n}a_ja_l+\sum_{i,j}k_ia_j.
\end{equation}

Recall $\gamma$ in the sign from \Cref{gammasign}.

\begin{equation*}\label{gamma}
\gamma= \sum_{j=0}^nk_j\sum_{l=j+1}^n(k_l+a_l+q_l)+\sum_{j=1}^na_j\left(\sum_{l=j}^nk_l+\sum_{l=j+1}^n(a_l+q_l)\right).
\end{equation*}

As we see, all the sums in the previous simplification appear in $\gamma$ so we can cancel them. Let us rewrite $\gamma$ in a way that this becomes more clear.

\begin{align*}
\gamma =& \sum_{0\leq j<l\leq n}k_jk_l+\sum_{0\leq j<l\leq n}k_ja_l+\sum_{0\leq j<l\leq n}k_jq_l+\sum_{1\leq j\leq l\leq n}a_jk_l\\
&+\sum_{1\leq j<l\leq n}a_ja_l+\sum_{1\leq j<l\leq n}a_jq_l.
\end{align*}

So after adding the expression (\ref{simply}) modulo 2 we have only the terms that include the internal degrees, i.e.
\begin{equation}\label{sofar}
\sum_{0\leq j<l\leq n}k_jq_l+\sum_{1\leq j<l\leq n}a_jq_l.
\end{equation}
Let us move now to the $\varepsilon$ term in the sign to rewrite it. 
\begin{align*}
\varepsilon&=\sum_{j=1}^n (a_j+q_j-1)(n-j+\sum_j^n k_l)\\
&=\sum_{j=1}^n (a_j+q_j-1)(n-j)+\sum_{1\leq j\leq l\leq n} (a_j+q_j-1)k_l
\end{align*}

We may add this to what we had in (\ref{sofar}) in such a way that the brace sign becomes

\begin{equation}\label{eta}
\eta=\sum_{0\leq j<l\leq n}k_jq_l+\sum_{1\leq j<l\leq n}a_jq_l+\sum_{j=1}^n (a_j+q_j-1)(n-j)+\sum_{1\leq j\leq l\leq n} (a_j+q_j-1)k_l.
\end{equation}
as announced at the end of Section \ref{sectionbraces}.
\end{proof}


\section{Twisted complex on an operad}\label{twistedoperad}
In this section we provide a description of the twisted complex structure on an operad $\OO$ with a derived $A_\infty$-multiplication. More precisely, we show by hand that the maps found in \Cref{mi1} define a twisted complex structure on $S\s\OO$.

\begin{lem}\label{twistedmaps}
Let $\OO$ be an operad and  $m\in\s\OO$ a derived $A_\infty$-multiplication. Then $S\s\OO$ becomes a twisted complex with structure maps
\[M_{i1}(x)= \sum_l (Sb_1(m_{il};S^{-1}x)-(-1)^{\langle x,m_{il}\rangle}Sb_1(S^{-1}x;m_{il})),\]
where $x\in (S\s\OO)^{n-k}_k$ and $\langle x,m_{il}\rangle=ik+(1-i)(n-1-k)$.
\end{lem}
\begin{proof}


Throughout the proof we omit the shift maps. Let us first check the twisted complex equation up to signs, to give a conceptual proof before introducing the signs. Up to sign, the maps  $\{M_{i1}\}_{i\geq 0}$ must satisfy the equation

\[\sum_{i+j=u} M_{i1}\circ M_{j1}=0,\]
for all $u$, where $\circ$ is composition of maps. %I may or may not omit the sum to avoid writing too much, as it just means that the composition on every degree must vanish.

Therefore, up to signs we have to compute 

\begin{align*}
\sum_{i+j=u}M_{i1}(M_{j1}(x))=&\sum_{i+j=u}M_{i1}\left(\sum_l b_1(m_{jl};x)+b_1(x;m_{jl})\right)\\
=&\sum_{i+j=u}\sum_{l,k}\left(b_1(m_{ik}; b_1(m_{jl};x))+b_1(m_{ik};b_1(x;m_{jl}))\right.\\
&\left.+b_1(b_1(m_{jl};x);m_{ik})+b_1(b_1(x;m_{jl});m_{ik})\right).
\end{align*}

Applying the brace relation we obtain

\begin{align*}
\sum_{i+j=u}\sum_{l,k}(b_1(m_{ik}; b_1(m_{jl};x))+b_1(m_{ik};b_1(x;m_{jl}))+\\
 b_2(m_{jl};x,m_{ik})+b_1(m_{jl};b_1(x;m_{ik}))+b_2(m_{jl};m_{ik},x)+\\
b_2(x;m_{jl},m_{ik})+b_1(x;b_1(m_{jl};m_{ik}))+b_2(x;m_{ik},m_{jl})).
\end{align*}

In the sum, all terms of the form $b_1(x;b_1(m_{jl};m_{ik}))$ that can be seen in the last line should add up to vanish provided that $m$ is a $dA_\infty$-multiplication, meaning that up to sign $b_1(m;m)=0$. %A sign of the form $(-1)^i$ %(or maybe $(-1)^j$, depending on the convention) 
 Since $i$ and $j$ are interchangeable, i.e. for each pair $(i,j)$ there is the pair $(j,i)$, the terms $b_2(x;m_{jl},m_{ik})+b_2(x;m_{ik},m_{jl})$ in the last line should cancel as well. For this, we should have the pair $(j,i)$ with the opposite sign. Here it is also relevant that the sum runs through all possible values of $k$ and $l$, so that the pair $(j,i)$ appears with $l$ and $k$ interchanged as well. So far the entire last line vanishes up to sign.

Then $b_1(m_{ik};b_1(x;m_{jl}))$ on the first line should cancel with $b_1(m_{jl};b_1(x;m_{ik}))$ on the second line, but from a different summand: the one where $i$ and $j$ are interchanged. Finally, the remaining terms $b_1(m_{ik}; b_1(m_{jl};x))+b_2(m_{jl};x,m_{ik})+b_2(m_{jl};m_{ik},x)$ add up to $b_1(b_1(m;m);x)$ up to sign. That would cancel everything.

Let us now introduce the signs. We now compute for all $u$ the sum%the sum $\displaystyle{\sum_{i+j=u} (-1)^iM_{i1}\circ M_{j1}}$.
\[\sum_{i+j=u} (-1)^iM_{i1}\circ M_{j1}.\]
%recalling that for the usual sign convention of twisted complex from a $dA_\infty$-algebra we need to define $d_i=(-1)^im_{i1}$, so that the sign in the equation is $(-1)^j$ instead of $(-1)^i$. 
For $x\in\s\OO$, by definition, we have
\begin{align*}
\sum_{i+j=u}(-1)^iM_{i1}(M_{j1}(x))=&\\
\sum_{i+j=u}(-1)^iM_{i1}\left(\sum_l b_1(m_{jl};x)-(-1)^{\langle x,m_{jl}\rangle}b_1(x;m_{jl})\right)=&\\
\sum_{i+j=u}(-1)^i\sum_{l,k}\left(b_1(m_{ik}; b_1(m_{jl};x))-(-1)^{\langle x,m_{jl}\rangle}b_1(m_{ik};b_1(x;m_{jl}))+\right.&\\
-(-1)^{\langle b_1(m_{jl};x),m_{ik}\rangle} b_1(b_1(m_{jl};x);m_{ik})&\\
\left.+(-1)^{\langle b_1(m_{jl};x),m_{ik}\rangle+\langle x,m_{jl}\rangle}b_1(b_1(x;m_{jl});m_{ik})\right).&
\end{align*}
Observe that $\langle b_1(m_{jl};x),m_{ik}\rangle=\langle m_{ij},m_{ik}\rangle+\langle x,m_{ik}\rangle$.

Applying the brace relation we obtain

\begin{align}\label{twistedequation}
\sum_{i+j=u}\sum_{l,k}((-1)^ib_1(m_{ik}; b_1(m_{jl};x))-(-1)^{i+\langle x,m_{jl}\rangle}b_1(m_{ik};b_1(x;m_{jl}))+\nonumber\\
 -(-1)^{i+\langle b_1(m_{jl};x),m_{ik}\rangle}(b_2(m_{jl};x,m_{ik})+(-1)^{\langle x,m_{ik}\rangle}b_2(m_{jl};m_{ik},x))\nonumber\\
 -(-1)^{i+\langle b_1(m_{jl};x),m_{ik}\rangle}b_1(m_{jl};b_1(x;m_{ik}))\nonumber\\
+(-1)^{i+\langle b_1(m_{jl};x),m_{ik}\rangle+\langle x,m_{jl}\rangle}(b_2(x;m_{jl},m_{ik})+(-1)^{\langle m_{ik},m_{jl}\rangle}b_2(x;m_{ik},m_{jl}))\nonumber\\
+(-1)^{i+\langle b_1(m_{jl};x),m_{ik}\rangle+\langle x,m_{jl}\rangle}b_1(x;b_1(m_{jl};m_{ik}))).
\end{align}

Recall from \Cref{sharp} that $m$ being a $dA_\infty$-multiplication means that \[\sum_{i+j=u}\sum_{k,l}(-1)^ib_1(m_{jl};m_{ik})=0.\] %Notice that the summand corresponding to each value of $i+j$ must vanish because it corresponds to a given horizontal degree. 
Let us check now the cancellations with the signs. First, let us check that the terms 
\[(-1)^{i+\langle b_1(m_{jl};x),m_{ik}\rangle+\langle x,m_{jl}\rangle}b_1(x;b_1(m_{jl};m_{ik})))\]
can be added up to vanish. For that, we compute the sign \[\langle b_1(m_{jl};x),m_{ik}\rangle+\langle x,m_{jl}\rangle=\langle m_{jl},m_{ik}\rangle+\langle x,m_{ik}\rangle+\langle x,m_{jl}\rangle.\]
Recall that the braces are defined on the operadic suspension, so that the bidegree of $m_{ik}$ is $(i,1-i)$. Therefore, writing the bidegree of $x$ as $(k,n-k)$, so that the total degree is $|x|=n$, the above equals 

\begin{align*}
&ji+(1-i)(1-j)+ki+(n-k)(1-i)+kj+(n-k)(1-j)\\
&= 1+i+j + (i+j)k+(i+j)(n-k)\mod 2\\
&=1+(i+j)(1+n)=1+u(1+|x|).
\end{align*}
Since this sign is constant for all terms $b_1(m_{ik};m_{ij})$ that share the same horizontal degree $i+j=u$, we can rewrite
\[(-1)^{i+\langle b_1(m_{jl};x),m_{ik}\rangle+\langle x,m_{jl}\rangle}b_1(x;b_1(m_{jl};m_{ik})))\]
as \[-(-1)^{u(1+|x|)}b_1(x;(-1)^ib_1(m_{ik};m_{jl})).\]
Hence, 
%Multiplying 0 by something is 0, some sum vanishes and you multiply it by a constant sign, it still vanishes
\[\sum_{i+j=u}\sum_{k,l}-(-1)^{u(1+|x|)}b_1(x;(-1)^ib_1(m_{ik};m_{jl}))=0.\]
Therefore, after applying the brace relation, expression (\ref{twistedequation}) reduces to

\begin{align}\label{twistedequation2}
\sum_{i+j=u}\sum_{l,k}((-1)^ib_1(m_{ik}; b_1(m_{jl};x))-(-1)^{i+\langle x,m_{jl}\rangle}b_1(m_{ik};b_1(x;m_{jl}))+\nonumber\\
 -(-1)^{i+\langle b_1(m_{jl};x),m_{ik}\rangle}(b_2(m_{jl};x,m_{ik})+(-1)^{\langle x,m_{ik}\rangle}b_2(m_{jl};m_{ik},x))\nonumber\\
 -(-1)^{i+\langle b_1(m_{jl};x),m_{ik}\rangle}b_1(m_{jl};b_1(x;m_{ik}))\nonumber\\
+(-1)^{i+\langle b_1(m_{jl};x),m_{ik}\rangle+\langle x,m_{jl}\rangle}(b_2(x;m_{jl},m_{ik})+(-1)^{\langle m_{ik},m_{jl}\rangle}b_2(x;m_{ik},m_{jl})).
\end{align}

Let us focus on the last line. For each pair $(i,j)$ we should have cancellation with the pair $(j,i)$, which adds the same elements, but with different signs. We also need to consider the pairs $(k,l)$ and $(l,k)$ to get a cancellation. Let us compare the signs. For the pair $((i,j),(k,l))$ we have precisely the last line of the above equation

\[(-1)^{i+\langle b_1(m_{jl};x),m_{ik}\rangle+\langle x,m_{jl}\rangle}(b_2(x;m_{jl},m_{ik})+(-1)^{\langle m_{ik},m_{jl}\rangle}b_2(x;m_{ik},m_{jl}))\]

For the pair $((j,i),(l,k))$ we have

\[(-1)^{j+\langle b_1(m_{ik};x),m_{jl}\rangle+\langle x,m_{ik}\rangle}(b_2(x;m_{ik},m_{jl})+(-1)^{\langle m_{jl},m_{ik}\rangle}b_2(x;m_{jl},m_{ik})).\]

 Comparing the sign of $b_2(x;m_{jl},m_{ik})$ we find that for $((i,j),(k,l))$ we have

\[-(-1)^{i+(i+j)(1+|x|)}b_2(x;m_{jl},m_{ik})=-(-1)^{j+u|x|}b_2(x;m_{jl},m_{ik})\]

and for the pair $((j,i),(l,k))$ we have

\[(-1)^{j+u|x|}b_2(x;m_{jl},m_{ik}).\]

As we see, we get opposite signs and thus cancellation. For $b_2(x;m_{ik},m_{jl})$ it is completely analogous. Thus, we have reduced expression (\ref{twistedequation2}) to

\begin{align}\label{twistedequation3}
\sum_{i+j=u}\sum_{l,k}((-1)^ib_1(m_{ik}; b_1(m_{jl};x))-(-1)^{i+\langle x,m_{jl}\rangle}b_1(m_{ik};b_1(x;m_{jl}))+\nonumber\\
 -(-1)^{i+\langle b_1(m_{jl};x),m_{ik}\rangle}(b_2(m_{jl};x,m_{ik})+(-1)^{\langle x,m_{ik}\rangle}b_2(m_{jl};m_{ik},x))\nonumber\\
 -(-1)^{i+\langle b_1(m_{jl};x),m_{ik}\rangle}b_1(m_{jl};b_1(x;m_{ik})).
\end{align}

In a similar fashion to the previous calculation, we are going to cancel $b_1(m_{ik};b_1(x;m_{jl}))$ in the first line with $b_1(m_{jl};b_1(x;m_{ik}))$ in the last line by considering switched pairs. For the pair $((i,j),(k,l))$, the term in the first line is 

\[-(-1)^{i+\langle x,m_{jl}\rangle}b_1(m_{ik};b_1(x;m_{jl}))\]

and for the pair $((j,i),(l,k))$ the term in the last line is

\begin{align*}
-(-1)^{j+\langle b_1(m_{ik};x),m_{jl}\rangle}b_1(m_{ik};b_1(x;m_{jl}))=&\\
(-1)^{1+j+\langle m_{ik},m_{jl}\rangle+\langle x,m_{jl}\rangle}b_1(m_{ik};b_1(x;m_{jl}))=&\\
(-1)^{i+\langle x,m_{jl}\rangle}b_1(m_{ik};b_1(x;m_{jl}))&,
\end{align*}

which has precisely the opposite sign to the other pair, and thus cancels. This reduces expression (\ref{twistedequation3}) to 

\begin{align}\label{twistedequation4}
\sum_{i+j=u}\sum_{l,k}((-1)^ib_1(m_{ik}; b_1(m_{jl};x))&\nonumber\\
 -(-1)^{i+\langle b_1(m_{jl};x),m_{ik}\rangle}(b_2(m_{jl};x,m_{ik})&+(-1)^{i+\langle m_{jl},m_{ik}\rangle}b_2(m_{jl};m_{ik},x)).
\end{align}

We want these terms to add up to something of the form $b_1(b_1(m;m);x)$. Notice that for this we need to switch some pairs. For simplicity, we switch the pair of the first term and rewrite the sum as

\begin{align*}
\sum_{i+j=u}\sum_{l,k}((-1)^jb_1(m_{jl}; b_1(m_{ik};x))&\\
 -(-1)^{i+\langle b_1(m_{jl};x),m_{ik}\rangle}b_2(m_{jl};x,m_{ik})&+(-1)^{i+\langle m_{jl}, m_{ik}\rangle}b_2(m_{jl};m_{ik},x)).
\end{align*}

Simplifying the signs we get

\begin{gather*}
\sum_{i+j=u}\sum_{l,k}((-1)^jb_1(m_{jl}; b_1(m_{ik};x))+(-1)^{j+\langle x,m_{ik}\rangle}b_2(m_{jl};x,m_{ik})\\+(-1)^{j}b_2(m_{jl};m_{ik},x)).
\end{gather*}

By the brace relation and \Cref{sharp} this equals
\[\sum_{i+j=u}\sum_{l,k}(-1)^jb_1(b_1(m_{jl};m_{ik});x)=0.\]
%For each position of insertion of x we have a 0 map applied to x, so the above sum is indeed equal to 0
\end{proof}

The reader can see that the twisted complex structure given by the above Lemma is the same as the one given by \Cref{mi1}.

%\end{appendices}
\end{document}