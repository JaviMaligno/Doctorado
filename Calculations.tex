	\documentclass[twoside]{article}
\usepackage{estilo-ejercicios}
\setcounter{section}{0}
\renewcommand{\baselinestretch}{1,3}

\usepackage{empheq}
\newcommand*\widefbox[1]{\fbox{\hspace{2em}#1\hspace{2em}}}
%--------------------------------------------------------
\begin{document}



Let $A=\bigoplus_{i,k} A^i_k$ be a bigraded module and let us say we have a linear map $\circ :A\otimes A\to A$ of bidegree $(0,0)$. This operation satisfies certain associativity conditions of operads with respect to the braiding morphism $\gamma(a\otimes b)=(-1)^{kl+ij}b\otimes a$ for $a\in A^i_k$ and $b\in A^j_l$ (we say that $a$ has bidegree $(k,i)$ and $b$ has bidegree $(l,j)$, where the first component is the horizontal degree and the second component is the vertical degree). 

A morphism $f:A\to B$ that preserves this structure is a linear map $f:A\to B$ of bidegree $(q,p)$ such that $f(a\circ b)=(-1)^{qk+pi}f(a)\circ f(b)$.



If I modify the operation by $a\bar{\circ}b=(-1)^{k(l+j)}a\circ b$, I get the same associativity relation with respect to the braiding $\gamma'(a\otimes b)=(-1)^{(k+i)(l+j)}b\otimes a$. I would like to define $\bar{f}=\pm f$ so that $\bar{f}(a\bar{\circ}b)=(-1)^{(q+p)(k+i)}\bar{f}(a)\bar{\circ}\bar{f}(b)$, in other words $\bar{f}$ preserves $\bar{\circ}$ with respect to the braiding $\gamma'$. This $\bar{f}$ would be the image of $f$ under the totalization functor. The sign factor of $\bar{f}$ should depend on the bidegree components of $f$ and its input.


Attemp

Let us say $\bar{f}(a)=(-1)^{\varepsilon(a)}f(a)$. Now we have

$$\bar{f}(a\bar{\circ} b)=(-1)^{k(j+l)}\bar{f}(a\circ b)=(-1)^{k(j+l)+\varepsilon(a\circ b)}f(a\circ b).$$

Using that $f$ preserves $\circ$ with respect to $\gamma$ we have

$$\bar{f}(a\bar{\circ} b)=(-1)^{k(j+l)+\varepsilon(a\circ b)+pi+qk}f(a)\circ f(b).$$

Since $f$ has bidegree $(q,p)$, $f(a)$ has bidegree $(k+q,i+p)$ and $f(b)$ has bidegree $(l+q,j+p)$ so now we have

$$\bar{f}(a\bar{\circ} b)=(-1)^{k(l+j)+\varepsilon(a\circ b)+pi+qk+(k+q)(j+p+l+q)}f(a)\bar{\circ} f(b).$$

Finally, we have 

$$\bar{f}(a\bar{\circ} b)=(-1)^{k(l+j)+\varepsilon(a\circ b)+pi+qk+(k+q)(j+p+l+q)+\varepsilon(a)+\varepsilon(b)}\bar{f}(a)\bar{\circ} \bar{f}(b).$$

And therefore that huge exponent should have the same parity as $(p+q)(i+k)=pi+pk+qi+qk$. However, cancelling some equal terms this implies

$$\varepsilon(a\circ b)+\varepsilon(a)+\varepsilon(b)+kq+q(j+p+l+q)\equiv qi.$$

Any reasonable choice of $\varepsilon(a)$ seems to imply $\varepsilon(a\circ b)=\varepsilon(a)+\varepsilon(b)$ since the bidegree of $a\circ b$ is $(k+l,i+j)$, so all the $\varepsilon$'s would vanish and the resulting equation modulo 2 would not be true in general.

Am I making any mistake or it is not possible to define such $\bar{f}$?

MAPS OF OPERADS MUST BE OF BIDEGRE 0 BECAUSE THE LEFT-HAND SIDE MUST HAVE THE SAME DEGREE ASS THE RIGH-HAND SIDE, SO SINCE F APPEARS MULTIPLE TIMES ON THE RIGHT HAND SIDE AND ONLY ONE ON THE LEFT, P=Q=0 AND IN THIS CASE THE RELATION IS SATISFIED SO TOTALIZATION DOES INDEED TAKE OPERADS TO OPERADS EVEN IF IT I NOT NATURAL WITH RESPECT TO EVERY BIGRADED MAP

A CONSEQUENCE OF THIS IF WE START WITH ALL THE BIGRADED MAPS IS THAT THE EMBEDDING GIVEN BY TOTALIZATION IS NOT FULL, BECAUSE ON GRADED OPERADS A MAP OF DEGREE 0 COULD COME FROM A BIDEGREE (M,-M) MAP (LIKE A TWISTED COMPLEX)

\newpage

\section{Twisted complex on operad}
I have the following candidates to induce derived $A_\infty$-structure on $S\s\OO$ after shifting (similar to what I did for $A_\infty$-algebras). 
\[M_{ij}(x_1,\dots, x_j)=\sum_l b_j(m_{il};x_1,\dots, x_j)\]
\[M_{i1}(x)= \sum_l (b_1(m_{il};x)-(-1)^{\langle x,m_{il}\rangle}b_1(x;m_{il}))\]

To apply the theorem by Sarah I need the bigraded module $S\s\OO$ to be a twisted complex. Since all the arities involved are 1 and the shift is only vertical, while the signs appearing in the twisted complex equation are related only to the horizontal degree, this is identical to a twisted complex structure on $\s\OO$. The twisted complex structure should be given by the maps $\{M_{i1}\}_{i\geq 0}$. I am going to try to show that this the case. First I will try without signs, to see that at least it is conceptually possible, and  then I would work out the signs.

Up to sign, the maps  $\{M_{i1}\}_{i\geq 0}$ must satisfy the equation

\[\sum_{i+j=m} M_{i1}\circ M_{ji}=0,\]
where $\circ$ is composition of maps. I may or may not omit the sum to avoid writing too much, as it just means that the composition on every degree must vanish.

Therefre, up to signs I compute 

\begin{align*}
\sum_{i+j=m}M_{i1}(M_{j1}(x))=\sum_{i+j=m}M_{i1}\left(\sum_l b_1(m_{jl};x)+b_1(x;m_{jl})\right)=\\
\sum_{i+j=m}\sum_{l,k}\left(b_1(m_{ik}; b_1(m_{jl};x))+b_1(m_{ik};b_1(x;m_{jl}))+b_1(b_1(m_{jl};x);m_{ik})+b_1(b_1(x;m_{jl});m_{ik})\right)
\end{align*}

%AT FIRST SIGHT IT DOESN'T LOOK POSSIBLE TO CANCEL THE LAST BRACE BECAUSE IT IS THE ONLY ONE WITH X AT THE BEGINNING, BUT THAT SHOULD HAVE BEEN THE SAME FOR THE CLASSICAL CASE, SO I SHOULD REVIEW THAT ONE
%
%ON THE CLASSICAL CASE IT WAS MUCH EASIER BECAUSE AFTER BRACE RELATION B(X;M,M) APPEARS TWICE WITH OPPOSITE SIGN, SO IT CANCELS. HERE IT IS NOT SO OBVIOUS BECAUSE THE SIGN IS NOT JUST $-1$ SO MAYBE IT CANCELS WITH OTHER SUMMANDS (RECALL THAT I AM OMITTING ONE SUM)

Applying the brace relation we obtain

\begin{align*}
\sum_{i+j=m}\sum_{l,k}(b_1(m_{ik}; b_1(m_{jl};x))+b_1(m_{ik};b_1(x;m_{jl}))+\\
 b_2(m_{jl};x,m_{ik})+b_1(m_{jl};b_1(x;m_{ik}))+b_2(m_{jl};m_{ik},x)+\\
b_2(x;m_{jl},m_{ik})+b_1(x;b_1(m_{jl};m_{ik}))+b_2(x;m_{ik},m_{jl}))
\end{align*}

In the sum all terms of the form $b_1(x;b_1(m_{jl};m_{ik}))$ that can be seen in the last line should add up to vanish provided that $m$ is a $dA_\infty$-multiplication (meaning that up to sign $b_1(m;m)=0$). A sign of the form $(-1)^i$ (or maybe $(-1)^j$) should be in in front of each of these terms. Since $i$ and $j$ are interchangable (i.e. for each pair $(i,j)$ there is the pair $(j,i)$), the terms $b_2(x;m_{jl},m_{ik})+b_2(x;m_{ik},m_{jl}))$ in the last line should cancel as well (for this, the other appearance should come with opposite sign). Here it is relevant as well, that the sum covers all the values of $k$ and $l$, so that the pair $(j,i)$ comes with $l$ and $k$ interchanged as well. 

Then $b_1(m_{ik};b_1(x;m_{jl}))$ in te first line should cancel with $b_1(m_{jl};b_1(x;m_{ik}))$ on the second line (but from a different summand, the one where $i$ and $j$ are interchanged). Finnaly, the reamining terms $b_1(m_{ik}; b_1(m_{jl};x))+b_2(m_{jl};x,m_{ik})+b_2(m_{jl};m_{ik},x)$ seem to be adding up to $b_1(b_1(m;m);x)$, but again some signs are going to be needed. That would cancel everything, so at least up to sign this makes sense.


\subsection{Signed version}

Now let us add signs. We now compute 
\[\sum_{i+j=m} (-1)^iM_{i1}\circ M_{ji}\]
recalling that for the usual sign convention of twisted complex from a $dA_\infty$-algebra we need to define $d_i=(-1)^im_{i1}$, so that the sign in tthe equation is $(-1)^j$ instead of $(-1)^i$. This being said, let us compute 
\begin{align*}
\sum_{i+j=m}(-1)^iM_{i1}(M_{j1}(x))=\sum_{i+j=m}(-1)^iM_{i1}\left(\sum_l b_1(m_{jl};x)-(-1)^{\langle x|m_{jl}\rangle}b_1(x;m_{jl})\right)=\\
\sum_{i+j=m}(-1)^i\sum_{l,k}\left(b_1(m_{ik}; b_1(m_{jl};x))-(-1)^{\langle x|m_{jl}\rangle}b_1(m_{ik};b_1(x;m_{jl}))+\right.\\
\left. -(-1)^{\langle b_1(m_{jl};x)|m_{ik}\rangle}b_1(b_1(m_{jl};x);m_{ik})+(-1)^{\langle b_1(m_{jl};x)|m_{ik}\rangle+\langle x|m_{jl}\rangle}b_1(b_1(x;m_{jl});m_{ik})\right)
\end{align*}
Observe that $\langle b_1(m_{jl};x)|m_{ik}\rangle=\langle m_{ij}|m_{ik}\rangle+\langle x|m_{ik}\rangle$ in the usual bigraded sign convention (also in the total graded convention). I am not going to explicitly compute these signs yet to see what properties we need from them.

Applying the brace relation we obtain

\begin{align*}
\sum_{i+j=m}\sum_{l,k}((-1)^ib_1(m_{ik}; b_1(m_{jl};x))-(-1)^{i+\langle x|m_{jl}\rangle}b_1(m_{ik};b_1(x;m_{jl}))+\\
 -(-1)^{i+\langle b_1(m_{jl};x)|m_{ik}\rangle}(b_2(m_{jl};x,m_{ik})+(-1)^{\langle x|m_{ik}\rangle}b_2(m_{jl};m_{ik},x))\\
 -(-1)^{i+\langle b_1(m_{jl};x)|m_{ik}\rangle}b_1(m_{jl};b_1(x;m_{ik}))\\
+(-1)^{i+\langle b_1(m_{jl};x)|m_{ik}\rangle+\langle x|m_{jl}\rangle}(b_2(x;m_{jl},m_{ik})+(-1)^{\langle m_{ik}|m_{jl}\rangle}b_2(x;m_{ik},m_{jl}))\\
+(-1)^{i+\langle b_1(m_{jl};x)|m_{ik}\rangle+\langle x|m_{jl}\rangle}b_1(x;b_1(m_{jl};m_{ik})))
\end{align*}

Recall that $m$ being a $dA_\infty$-multiplication means that $\sum_{i+j=m}\sum_{k,l}(-1)^ib_1(m_{ik};m_{ij})=0$. Notice that the summand corresponding to each value of $i+j$ must vanish because it corresponds to a given horizontal degree. Let us check now the cancellations with the signs. First, let us check that the terms 
\[(-1)^{i+\langle b_1(m_{jl};x)|m_{ik}\rangle+\langle x|m_{jl}\rangle}b_1(x;b_1(m_{jl};m_{ik})))\]
can be added up to vanish. For that, we compute the sign \[\langle b_1(m_{jl};x)|m_{ik}\rangle+\langle x|m_{jl}\rangle=\langle m_{jl}|m_{ik}\rangle+\langle x|m_{ik}\rangle+\langle x|m_{jl}\rangle\]
Recall that the braces are defined on the operadic suspension, so that the bidegree of $m_{ik}$ is $(i,1-i)$. Therefore, writing the bidegree of $x$ as $(x_h,x_v)$, the above equals 
\[ji+(1-i)(1-j)+x_hi+x_v(1-i)+x_hj+x_v(1-j)\equiv 1+i+j + (i+j)x_h+(i+j)x_v\mod 2=\]
\[(i+j)(1+x_h+x_v)=1+m(1+x)\]
Since this sign is constant for all terms $b_1(m_{ik};m_{ij})$ that share the same horizontal degree $i+j=m$, we can rewrite
\[(-1)^{i+\langle b_1(m_{jl};x)|m_{ik}\rangle+\langle x|m_{jl}\rangle}b_1(x;b_1(m_{jl};m_{ik})))=-(-1)^{m(1+x)}b_1(x;(-1)^ib_1(m_{ik};m_{jl})),\]
which vanishes when we consider the whole sum
%Multiplying 0 by something is 0, some sum vanishe and you multiply it by a constant sign, it still vanishes
\[\sum_{i+j=m}\sum_{k,l}-(-1)^{m(1+x)}b_1(x;(-1)^ib_1(m_{ik};m_{jl}))=0.\]
Therefore, the equation after the brace relation reduces to
\begin{align*}
\sum_{i+j=m}\sum_{l,k}((-1)^ib_1(m_{ik}; b_1(m_{jl};x))-(-1)^{i+\langle x|m_{jl}\rangle}b_1(m_{ik};b_1(x;m_{jl}))+\\
 -(-1)^{i+\langle b_1(m_{jl};x)|m_{ik}\rangle}(b_2(m_{jl};x,m_{ik})+(-1)^{\langle x|m_{ik}\rangle}b_2(m_{jl};m_{ik},x))\\
 -(-1)^{i+\langle b_1(m_{jl};x)|m_{ik}\rangle}b_1(m_{jl};b_1(x;m_{ik}))\\
+(-1)^{i+\langle b_1(m_{jl};x)|m_{ik}\rangle+\langle x|m_{jl}\rangle}(b_2(x;m_{jl},m_{ik})+(-1)^{\langle m_{ik}|m_{jl}\rangle}b_2(x;m_{ik},m_{jl}))
\end{align*}
Let us focus on the last line. For each pair $(i,j)$ we should have cancellation with the pair $(j,i)$, which adds the same elements, but with different signs. We also need to consider the pairs $(k,l)$ and $(l,k)$ to get a cancellation. Let us compare the signs. For the pair $((i,j),(k,l))$ we have precisely the last line of the above equation
\[(-1)^{i+\langle b_1(m_{jl};x)|m_{ik}\rangle+\langle x|m_{jl}\rangle}(b_2(x;m_{jl},m_{ik})+(-1)^{\langle m_{ik}|m_{jl}\rangle}b_2(x;m_{ik},m_{jl}))\]

For the pair $((j,i),(l,k))$ we have
\[(-1)^{j+\langle b_1(m_{ik};x)|m_{jl}\rangle+\langle x|m_{ik}\rangle}(b_2(x;m_{ik},m_{jl})+(-1)^{\langle m_{jl}|m_{ik}\rangle}b_2(x;m_{jl},m_{ik}))\]
 Comparing the sign of $b_2(x;m_{jl},m_{ik})$ we find that for $((i,j),(k,l))$ we have

\[-(-1)^{i+(i+j)(1+x)}b_2(x;m_{jl},m_{ik})=-(-1)^{j+(i+j)x}b_2(x;m_{jl},m_{ik})\]
and for the pair $((j,i),(l,k))$ we have
\[(-1)^{j+(i+j)x}b_2(x;m_{jl},m_{ik})\]
As we see, we get opposite signs and thus cancellation. For $b_2(x;m_{ik},m_{jl})$ it is completely analogous. Thus, we have reduced our main equation to 
\begin{align*}
\sum_{i+j=m}\sum_{l,k}((-1)^ib_1(m_{ik}; b_1(m_{jl};x))-(-1)^{i+\langle x|m_{jl}\rangle}b_1(m_{ik};b_1(x;m_{jl}))+\\
 -(-1)^{i+\langle b_1(m_{jl};x)|m_{ik}\rangle}(b_2(m_{jl};x,m_{ik})+(-1)^{\langle x|m_{ik}\rangle}b_2(m_{jl};m_{ik},x))\\
 -(-1)^{i+\langle b_1(m_{jl};x)|m_{ik}\rangle}b_1(m_{jl};b_1(x;m_{ik}))
\end{align*}
In a similar fashion to the previous calculation, we are going to try to cancel $b_1(m_{ik};b_1(x;m_{jl}))$ in the first line with $b_1(m_{jl};b_1(x;m_{ik})$ in the last line by considering switched pairs. For the pair $((i,j),(k,l))$, the term in the first line is 
\[-(-1)^{i+\langle x|m_{jl}\rangle}b_1(m_{ik};b_1(x;m_{jl}))\]
and for the pair $((j,i),(l,k))$ the term in the last line is
\[-(-1)^{j+\langle b_1(m_{ik};x)|m_{jl}\rangle}b_1(m_{ik};b_1(x;m_{jl}))=(-1)^{1+j+\langle m_{ik}|m_{jl}\rangle+\langle x|m_{jl}\rangle}b_1(m_{ik};b_1(x;m_{jl}))=\]
\[(-1)^{i+\langle x|m_{jl}\rangle}b_1(m_{ik};b_1(x;m_{jl}))\]
which has precisely the opposite sign to the other pair, and thus cancels. This reduces the main equation to just 
\begin{align*}
\sum_{i+j=m}\sum_{l,k}((-1)^ib_1(m_{ik}; b_1(m_{jl};x))
 -(-1)^{i+\langle b_1(m_{jl};x)|m_{ik}\rangle}(b_2(m_{jl};x,m_{ik})+(-1)^{\langle x|m_{ik}\rangle}b_2(m_{jl};m_{ik},x))
\end{align*}
We want this therms to add up to something of the form $b_1(b_1(m;m);x)$. Notice that for that we need to switch some pairs. For simplity, we switch the pair of the first term and rewrite the sum after simplifying some signs as
\begin{align*}
\sum_{i+j=m}\sum_{l,k}((-1)^jb_1(m_{jl}; b_1(m_{ik};x))
 -(-1)^{i+\langle b_1(m_{jl};x)|m_{ik}\rangle}b_2(m_{jl};x,m_{ik})-(-1)^{i+\langle m_{jl}| m_{ik}\rangle}b_2(m_{jl};m_{ik},x))
\end{align*}
Simplifying further the signs we get
\begin{align*}
\sum_{i+j=m}\sum_{l,k}((-1)^jb_1(m_{jl}; b_1(m_{ik};x))
 +(-1)^{j+\langle x|m_{ik}\rangle}b_2(m_{jl};x,m_{ik})+(-1)^{j}b_2(m_{jl};m_{ik},x)).
\end{align*}
By the brace relation this equals
\[\sum_{i+j=m}\sum_{l,k}(-1)^jb_1(b_1(m_{j,l};m_{ik});x)=0\]
%For each position of insertion of x we have a 0 map applied to x, so the above sum is indeed equal to 0

\end{document}
