	\documentclass[twoside]{article}
\usepackage{estilo-ejercicios}
\setcounter{section}{0}
\newtheorem{defin}{Definition}[section]
\newtheorem{lem}[defin]{Lemma}
\newtheorem{propo}[defin]{Proposition}
\newtheorem{thm}[defin]{Theorem}
\newtheorem{eje}[defin]{Example}
\renewcommand{\baselinestretch}{1,3}

\usepackage{empheq}
\newcommand*\widefbox[1]{\fbox{\hspace{2em}#1\hspace{2em}}}
%--------------------------------------------------------
\begin{document}



Let $A=\bigoplus_{i,k} A^i_k$ be a bigraded module and let us say we have a linear map $\circ :A\otimes A\to A$ of bidegree $(0,0)$. This operation satisfies certain associativity conditions of operads with respect to the braiding morphism $\gamma(a\otimes b)=(-1)^{ij+kl}b\otimes a$ for $a\in A^i_k$ and $b\in A^j_l$ (we say that $a$ has bidegree $(k,i)$ and $b$ has bidegree $(l,j)$, where the first component is the horizontal degree and the second component is the vertical degree). 

A morphism $f:A\to B$ that preserves this structure is a linear map $f:A\to B$ of bidegree $(q,p)$ such that $f(a\circ b)=(-1)^{pi+qk}f(a)\circ f(b)$.



If I modify the operation by $a\bar{\circ}b=(-1)^{l(l+j)}a\circ b$, I get the same associativity relation with respect to the braiding $\gamma'(a\otimes b)=(-1)^{(i+k)(j+l)}b\otimes a$. I would like to define $\bar{f}=\pm f$ so that $\bar{f}(a\bar{\circ}b)=(-1)^{(p+q)(i+k)}\bar{f}(a)\bar{\circ}\bar{f}(b)$, in other words $\bar{f}$ preserves $\bar{\circ}$ with respect to the braiding $\gamma'$. This $\bar{f}$ would be the image of $f$ under the totalization functor. The sign factor of $\bar{f}$ should depend on the bidegree components of $f$ and its input.


Attemp

Let us say $f(a)=(-1)^{\varepsilon(a)}f(a)$. Now we have

$$\bar{f}(a\bar{\circ} b)=(-1)^{k(j+l)}\bar{f}(a\circ b)=(-1)^{k(j+l)+\varepsilon(a\circ b)}f(a\circ b).$$

Using that $f$ preserves $\circ$ with respect to $\gamma$ we have

$$\bar{f}(a\bar{\circ} b)=(-1)^{k(j+l)+\varepsilon(a\circ b)+pi+qk}f(a)\circ f(b).$$

Since $f$ has bidegree $(q,p)$, $f(a)$ has bidegree $(k+q,i+p)$ and $f(b)$ has bidegree $(l+q,j+p)$ so now we have

$$\bar{f}(a\bar{\circ} b)=(-1)^{k(l+j)+\varepsilon(a\circ b)+pi+qk+(l+q)(i+p+k+q)}f(a)\bar{\circ} f(b).$$

Finally, we have 

$$\bar{f}(a\bar{\circ} b)=(-1)^{k(l+j)+\varepsilon(a\circ b)+pi+qk+(l+q)(i+p+k+q)+\varepsilon(a)+\varepsilon(b)}\bar{f}(a)\bar{\circ} \bar{f}(b).$$

And therefore that huge exponent should have the same parity as $(p+q)(i+k)=pi+pk+qi+qk$. However, cancelling some equal terms this implies

$$\varepsilon(a\circ b)+\varepsilon(a)+\varepsilon(b)+kj+l(i+p+q)+q(p+k+q)\equiv pk.$$

Any reasonable choice of $\varepsilon(a)$ seems to imply $\varepsilon(a\circ b)=\varepsilon(a)+\varepsilon(b)$ since the bidegree of $a\circ b$ is $(k+l,i+j)$, so all the $\varepsilon$'s would vanish and the resulting equation modulo 2 would not be true in general.

Am I making any mistake or it is not possible to define such $\bar{f}$?



\end{document}
