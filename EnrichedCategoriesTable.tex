	\documentclass[twoside]{article}
\usepackage{estilo-ejercicios}
\setcounter{section}{0}
\newtheorem{defin}{Definition}[section]
\newtheorem{lem}[defin]{Lemma}
\newtheorem{propo}[defin]{Proposition}
\newtheorem{thm}[defin]{Theorem}
\newtheorem{eje}[defin]{Example}
\newtheorem{obs}[defin]{Observación}
\DeclareMathAlphabet{\mathpzc}{OT1}{pzc}{m}{it}
\renewcommand{\baselinestretch}{1,3}

\usepackage{empheq}
\newcommand*\widefbox[1]{\fbox{\hspace{2em}#1\hspace{2em}}}
%--------------------------------------------------------
\begin{document}

\title{Table of enriched categories}
\author{Javier Aguilar Martín}
\maketitle

\section{Summary table}
Chain means cochain. I may change the distribution to put all underlines together or each category folled by its underlined (enriched) version. Maybe full subcategories (including bounded categories) should go on a different table where I specify the supercategory. Also the alternatively enriched (mathpzc and mathcal font) categories could go on a different table where I specify where they are enriched over. There are several $\delta$ defined depending  on the enrichment used, possibly specify which is one in case it is confusing.

I NEED TO ADD SOME MORE, CHECK LAST DEFINITIONS
\begin{tabular}{|c|c|c|c}
\hline
\textbf{Category} & \textbf{Objects} & \textbf{Morphisms} & Monoidal structure?\\
\hline
$\mathrm{C}_R$ & Chain complexes & Maps complexes &\\
\hline
$\mathrm{fC}_R$ & Filtered complexes & Maps of filtered complexes &\\
\hline
$\mathrm{bgMod}_R$ & Bigraded modules & Maps of bidegree $(0,0)$&\\
\hline
$\mathrm{vbC}_R$ & Vertical bicomplexes & Maps of bigraded modules & \\
                 &                      & that commute with differential & \\ 
\hline
$\underline{\mathrm{C}_R}$ & Chain complexes & Maps of chain complexes &\\
                           &                 & with differential $\delta$&\\
\hline
$\underline{\mathrm{bgMod}_R}$ & Bigraded modules & Maps of any bidegree & \\
& & &\\
\hline
$\underline{\mathrm{vbC}_R}$ & Vertical bicomplexes & Maps of any bidegree &\\
                             &                      & with differential $\delta$&\\
\hline
$\mathrm{tC}_R$& Twisted complexes & Maps of twisted complexes & \\
\hline
$\underline{\mathrm{tC}_R}$ & Twisted complexes & Maps of twisted complexes & \\
& & with differentials $d_i$&\\
\hline
$\mathrm{sfC}_R$ & Split filtered complexes & Maps of filtered complexes & \\
\hline
$\underline{\mathpzc{bgMod}_R}$ & Bigraded modules & Weird enrichment (bgMod)&\\
& & &\\
\hline
$\underline{t\mathcal{C}_R}$& Twisted complexes & Weird enrichment &\\
& & with differential (vbC) & \\
\end{tabular}

Bonus: $\mathrm{bgMod}^\infty_R$ is the full subcategory of $\mathrm{tC}_R$ whose objects have $d_m=0$ for all $m$.

Maybe leave the split one as a bonus as well. 

Definitionn 3.10 for bounded subcategories. 

Diagram before section 3.3 can be useful

Endomorphism operad?

If I find some notation confusing because the same is used for different things I may use other notations at some point.
\section{Definitions}

Create a custom ennviroment to use the numbering of Sarah's paper for better identification \url{https://tex.stackexchange.com/questions/53978/custom-theorem-numbering}
\begin{defin}[Filtered $R$-module and morphisms]
A \emph{filtered $R$-module} $(A, F)$ is given by a family of $R$-modules $\{F_pA\}_{p∈\Z}$ indexed by the integers such that $F_{p−1}A ⊆ F_pA$ for all $p ∈ \Z$ and $A = ∪F_pA$. A morphism of filtered modules is a
morphism $f : A → B$ of $R$-modules which is compatible with filtrations: \[f(F_pA) ⊂ F_pB\text{ for all }p ∈ \Z.\]
\end{defin}

\begin{defin}[Filtered complexes and morphism]
A \emph{filtered complex} $(K, d, F)$ is a complex $(K, d) ∈ \mathrm{C}_R$ together with a filtration $F$ of each $R$-module $K^n$ such that $d(F_pK^n) ⊂ F_pK^{n+1}$ for all $p, n ∈ \Z$. Its morphisms are given by
morphisms of complexes $f : K → L$ compatible with filtrations: \[f(F_pK) ⊂ F_pL\text{ for all }p ∈ \Z.\]
\end{defin}

Tensor product?

\begin{defin}[Vertical bicomplexes and morphisms]
A \emph{vertical bicomplex} is a bigraded $R$-module $A$ equipped with a vertical differential $d_A : A → A$ of bidegree $(0, 1)$. A morphism of vertical bicomplexes is a morphism of bigraded modules
of bidegree $(0, 0)$ commuting with the vertical differential.
\end{defin}
Tensor product?

\begin{defin}[Enrichment over themselves]
Let $A,B$ be bigraded modules. We define $[A,B]^∗_∗$
to be the bigraded module of morphisms of bigraded modules $A → B$. Furthermore, if $A,B$ are vertical bicomplexes, and $f ∈
[A,B]^v_u$, we define
\[δ(f) := d_Bf − (−1)^vfd_A.\]
\end{defin}

\begin{defin}[Twisted complexes] A \emph{twisted complex} $(A, d_m)$ is a bigraded $R$-module $A = \{A^j_i \}$ together with a family
of morphisms $\{d_m : A → A\}_{m≥0}$ of bidegree $(−m,−m + 1)$ such that for all $m ≥ 0$,
\[\sum_{i+j=m}(−1)^id_id_j = 0.\]
\end{defin}

\begin{defin}[Morphisms of twisted complexes]
A morphism of twisted complexes $f : (A, d^A_m) → (B, d^B_m)$ is given by a family of morphisms of $R$-modules $\{f_m : A → B\}_{m≥0}$ of bidegree $(−m,−m)$ such that for all $m ≥ 0$,
\[\sum_{i+j=m}d^B_if_j =\sum_{i+j=m}(−1)^if_id^A_j.\]
The composition of morphisms is given by $(g \circ f)_m :=\sum_{i+j=m} g_if_j$.
\end{defin}

Tensor product?

This is actually a lemma but I don't care
\begin{defin}[Enrichment over twisted complexes] Let $A,B$ be twisted complexes. For $f ∈ [A,B]^v_u$, setting
\[(d_if) := (−1)^{i(u+v)}d^B_if − (−1)^vfd^A_i,\]
for $i ≥ 0$, endows $[A,B]^∗_∗$ with the structure of a twisted complex.
\end{defin}

I will probably use direct sum totalization, whih has a specific notation and name in the paper, but I think the results for this totalization applies to the direct sum version (totalization with compact support).

\begin{defin}[Totalization and column filtration]
The total graded $R$-module $\mathrm{Tot}(A)$ of a bigraded $R$-module $A =\{A^j_i \}$ is given by
\[\mathrm{Tot}(A)^n :=\prod_{i≤0}A^{n+i}_i ⊕\bigoplus_{i>0}A^{n+i}_i .\]
The \emph{column filtration} of $\mathrm{Tot}(A)$ is the filtration given by \[F_p\mathrm{Tot}(A)^n :=\prod_{i≤p} A^{n+i}_i\text{ for all }p, n ∈ \Z.\]
\end{defin}

This one comes much later but I prefer it to be here. It is possible to omit the filtration (but would have to check what consequences it has).
\begin{defin}
The \emph{totalization with compact support} of a vertical bicomplex $A$ is the filtered complex given by
\[\mathrm{Tot}_c(A)^n :=\sum_{i∈\Z}A^{n+i}_i\]
with the column filtration and with differential as for the totalization functor.

Given a morphism of vertical bicomplexes $f : A → B$ we get a morphism of filtered complexes $\mathrm{Tot}_c(f) : Tot_c(A) → Tot_c(B)$
constructed analogously to $\mathrm{Tot}(f)$.
\end{defin}

\begin{defin}[Split (filtered) complexes]
A filtered complex $(K, d, F)$ is said to be \emph{split} if $K = \mathrm{Tot}(A)$ is the total graded module of a bigraded $R$-module $A = \{A^j_i \}$ and $F$ is the column filtration of $\mathrm{Tot}(A)$.
\end{defin}

\begin{defin}[Weird enrichment]
Let $A,B,C$ be bigraded modules. We denote by $\underline{\mathpzc{bgMod}_R}(A,B)$ the bigraded module given by
\[\underline{\mathpzc{bgMod}_R}(A,B)^v_u :=\prod_{j≥0}[A,B]^{v−j}_{u−j}\]
where $[A,B]$ is the inner hom-object of bigraded modules. More precisely, $g ∈ \underline{\mathpzc{bgMod}_R}(A,B)^v_u$ is given
by $g := (g_0, g_1, g_2, \dots )$, where $g_j : A → B$ is a map of bigraded modules of bidegree $(u − j, v − j)$.

Moreover, we define a composition morphism
\[c : \underline{\mathpzc{bgMod}_R}(B,C) ⊗ \underline{\mathpzc{bgMod}_R}(A,B) → \underline{\mathpzc{bgMod}_R}(A,C)\]
by
\[c(f, g)_m :=\sum_{i+j=m}(−1)^{i|g|}f_ig_j .\]
\end{defin}
Try to find a conceptual explanation of the sign in the above composition map.

\begin{defin}
Let $(A, d^A_i), (B, d^B_i)$ be twisted complexes, $f ∈ \underline{\mathpzc{bgMod}_R}(A,B)^v_u$ and consider $d^A :=(d^A_i)_i ∈ \underline{\mathpzc{bgMod}_R}(A,A)^1_0$
and $d^B := (d^B_i)_i ∈ \underline{\mathpzc{bgMod}_R}(B,B)^1_0$. We define
\[δ(f) := c(d^B, f) − (−1)^{\langle f,d^A\rangle}c(f, d^A) ∈ \underline{\mathpzc{bgMod}_R}(A,B)^{v+1}_u\]
where $\langle f, d^A\rangle$ is the scalar product for the bidegrees and $c$ is the composition morphism described in Definition 3.32. More precisely,
\[(δ(f))_m :=\sum_{i+j=m}(−1)^{i|f|}d^B_if_j − (−1)^{v+i}f_id^A_j.\]
\end{defin}
Lemma 4.18? (in results obvsly)
\begin{defin}[Twisted complex weird enrichment]
For $A,B$ twisted complexes, we define $\underline{t\mathcal{C}_R}(A,B)$ to be the vertical bicomplex
$\underline{t\mathcal{C}_R}(A,B) := (\underline{\mathpzc{bgMod}_R}(A,B), δ)$.
\end{defin}

Definition 4.23
\begin{defin}
The $\vbc$-enriched category of twisted complexes $\utC$ is the enriched category given by the following data.
\begin{enumerate}[(1)]
\item The objects of $\utC$ are twisted complexes.
\item For $A,B$ twisted complexes the hom-object is the vertical bicomplex $\utC(A,B)$.
\item The composition morphism $c : \utC(B,C)⊗\utC(A,B) → \utC(A,C)$ is given by Definition 3.32.
\item The unit morphism $R → \utC(A,A)$ is given by the morphism of vertical bicomplexes sending
$1 ∈ R$ to $1_A : A → A$, the strict morphism of twisted complexes given by the identity of $A$.
\end{enumerate}
\end{defin}

Definition 4.24
\begin{defin}
We denote by $\ubgMod$ the $\bgmod$-enriched category of bigraded modules given
by the following data.

\begin{enumerate}[(1)]
\item The objects of $\ubgMod$ are bigraded modules.
\item For $A,B$ bigraded modules the hom-object is the bigraded module $\ubgMod(A,B)$.
\item The composition morphism $c : \ubgMod(B,C) ⊗ \ubgMod(A,B) → \ubgMod(A,C)$ is given by
Definition 3.32.
\item The unit morphism $R → \ubgMod(A,A)$ is given by the morphism of bigraded modules that
sends $1 ∈ R$ to $1_A : A → A$, the strict morphism given by the identity of $A$.
\end{enumerate}
\end{defin}

Tensor product? (Lemma 4.27) Find a concetual explanation for that tensor product or at least see why it is necessary to satisfy the axioms of monoidal category.

Definition 4.28

\begin{defin}
The $\bgmod$-enriched category of filtered modules $\ufMod$ is the enriched category
given by the following data.
\begin{enumerate}[(1)]
\item The objects of $\ufMod$ are filtered modules.
\item For filtered modules $(K, F)$ and $(L, F)$, the bigraded module $\ufMod(K,L)$ is given by
\[\ufMod(K,L)^v_u :=\{f : K → L\mid f(F_qK^m) ⊂ F_{q+u}L^{m+v−u}, ∀m, q ∈ \Z\}.\]
\item The composition morphism is given by $c(f, g) = (−1)^{u|g|}fg$, where $f$ has bidegree $(u, v)$.
\item The unit morphism is given by the map $R → \ufMod(K,K)$ given by $1 → 1_K$.
\end{enumerate}
\end{defin}

Definition 4.31
\begin{defin}
Let $(K, d^K, F)$ and $(L, d^L, F)$ be filtered complexes. We define $\ufC(K,L)$ to be the
vertical bicomplex whose underlying bigraded module is $\ufMod(K,L)$ with vertical differential
\[δ(f) := c(d^L, f) − (−1)^{\langle f,d^K\rangle}c(f, d^K) = d^Lf − (−1)^{v+u}fd^K = d^Lf − (−1)^{|f|}fd^K\]
for $f ∈ \ufMod(K,L)^v_u$. Composition from $\ufMod$.
\end{defin}

Definition 4.33
\begin{defin}
The $\vbc$-enriched category of filtered complexes $\ufC$ is the enriched category given
by the following data.
\begin{enumerate}[(1)]
\item The objects of fCR are filtered complexes.
\item For $K,L$ filtered complexes the hom-object is the vertical bicomplex $\ufC(K,L)$.
\item The composition morphism is given as in $\ufMod$ in Definition 4.28.
\item The unit morphism is given by the map $R → \ufC(K,K)$ given by $1 → 1_K$.
We denote by $\usfC$ the full subcategory of $\ufC$ whose objects are split filtered complexes.

\end{enumerate}
\end{defin}

Tensor producct? (Lemma 4.36) Again see how the signs do their job

Definition 4.38, here Tot is probably the one that comes with column filtration, so recall that (It wouldn't hurt to filter my modules with the column filtration)

Definition 4.42 for the weird End (Example 4.44 as well)

Definitionn 4.49 for Ainfty morphism of twisted complexes. Later, before 4.50 they say Ainfty algebra in twisted complex, I understand this means the Ainfty multiplications are morphisms in the category, but exactly which one (like, some enrichment, the same as in def 4.49?)

NEXT SECTION 4.6, IT STARTS WITH THE ASSERTION OF AN ISOMORPHISM THAT I HAVE TO FIND OR CHECK
\section{Results}
If the results need previous definitions or results that I haven't included, indicate it and add them (even if they're above it might be useful to refer to the definitions)


Theorem 3.8 with the previous definitions of Tot and d on morphisms

Theorem 3.11, remark that forgetting about the filtration the result is still true

Lemma 4.37, maybe the filtratioon can be omitted as in Theorem 3.11. This one needs Proposition 3.11 but also some theory that I have skipped for now

Theorem 4.39, checking this proof might shed some light onto why we need filtrations to extend Tot to any bidegree

Proposition 4.46 which uses the categorical machinery, I would like to know how the enriched functor things works for instance if we are working with twisted complexes (so that the weird End is as in Example 4.44)

Propositioon 4.47. A very important one that I have checed before, but may check again in more detail after understanding monoidal structures. Should clear what $\mathrm{End}_A$ is in this theorem. Since it is in $\vbc$ it should come with some of the enrichment (the one giving it a vertical diferential), that should be more clear

Theorem 4.50? May be useful

NEXT SECTION 4.6, IT STARTS WITH THE ASSERTION OF AN ISOMORPHISM THAT I HAVE TO FIND OR CHECK
\end{document}
