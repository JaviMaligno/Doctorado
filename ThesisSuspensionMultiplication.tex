\documentclass[Thesis.tex]{subfiles}
%\usepackage{estilo-ejercicios}
\doublespacing
\sloppy



%--------------------------------------------------------
\begin{document}



%\section*{Abstract}
\chapter{$A_\infty$-algebras on operads}\label{Sec2}

In this chapter we aim to encode $A_\infty$-algebras in a simple operadic way. To do that we use operadic suspension, following an approach similar to the one introduced by Ward \cite{ward}. We explore some properties of this construction and the connection to other ways of encoding $A_\infty$-algebras that are found in the literature. We then describe a brace algebra structure on operadic suspension and construct $A_\infty$-algebras on a certain family of operads. We finally use these structures to prove \Cref{theorem}, which was originally claimed by Gerstenhaber and Voronov \cite{GV}.

\section{Operadic suspension}

In this section we define an operadic suspension, which is a slight modification of the one found in \cite{ward}. This construction will help us define $A_\infty$-multiplications in a simple way. The motivation to introduce operadic suspension is that signs in $A_\infty$-algebras and related Lie structures are know to arise from a sequence of shifts. In order to discuss derived structures later, we need to pin this down more generally and rigorously. We are going to work only with non-symmetric operads, although most of what we do is also valid in the symmetric case.

\subsection{Operadic suspension and $A_\infty$-algebras}
Let $\Lambda(n)=S^{n-1}R$, where $S$ is the shift of graded modules, so that $\Lambda(n)$ is the ring $R$ concentrated in degree $n-1$. We view this module as the free $R$-module of rank one spanned by the exterior power $e^n=e_1\land\cdots\land e_n$ of degree $n-1$, where $e_i$ is the $i$-th element of the canonical basis of $R^n$. By convention, $\Lambda(0)$ is one-dimensional concentrated in degree $-1$ and generated by $e^0$.


Let us define an operad structure on $\Lambda=\{\Lambda(n)\}_{n\geq 0}$ via the following insertion maps

\[
\begin{tikzcd}
\Lambda(n)\otimes\Lambda(m) \arrow[r, "\circ_i"] & \Lambda(n+m-1)\\
(e_1\land\cdots\land e_n)\otimes(e_1\land\cdots\land e_m)\arrow[r, mapsto] & (-1)^{(n-i)(m-1)}e_1\land\cdots\land e_{n+m-1}.
\end{tikzcd}
\]

We are inserting the second factor onto the first one, so the sign can be explained  by moving the power $e^m$ of degree $m-1$ to the $i$-th position of $e^n$ passing by $e_{n}$ through $e_{i+1}$. More compactly, \[e^n\circ_i e^m=(-1)^{(n-i)(m-1)}e^{n+m-1}.\] The unit of this operad is $e^1\in\Lambda(1)$. It can be checked by direct computation that $\Lambda$ satisfies the axioms of an operad of graded modules.

In a similar way we can define $\Lambda^-(n)=S^{1-n}R$, with the same insertion maps.

\begin{defin}
Let $\mathcal{O}$ be an operad. The \emph{operadic suspension} $\mathfrak{s}\OO$ of $\mathcal{O}$ is given arity-wise by the Hadamard product of the operads $\OO$ and $\Lambda$, in other words, $\mathfrak{s}\OO(n)=(\mathcal{O}\otimes\Lambda)(n)=\mathcal{O}(n)\otimes\Lambda(n)$ with diagonal composition. Similarly, we define the \emph{operadic desuspension} arity-wise as $\mathfrak{s}^{-1}\OO(n)=\mathcal{O}(n)\otimes\Lambda^-(n)$.
\end{defin}


Even though the elements of $\s\OO$ are tensor products of the form $x\otimes e^n$, we may identify the elements of $\mathcal{O}$ with the elements the elements of $\mathfrak{s}\OO$ and simply write $x$ as an abuse of notation. 

\begin{defin}
For $x\in\OO(n)$ of degree $\deg(x)$, its \emph{natural degree} $|x|$ is the degree of $x\otimes e^n$ as an element of $\s\OO$, namely, $|x|=\deg(x)+n-1$. To distinguish both degrees we call $\deg(x)$ the \emph{internal degree} of $x$, since this is the degree that $x$ inherits from the grading of $\OO$. 
\end{defin}

If we write $\circ_i$ for the operadic insertion on $\OO$ and $\tilde{\circ}_i$ for the operadic insertion on $\mathfrak{s}\OO$, we may find a relation between the two insertion maps in the following way. 

\begin{lem}\label{tilde}
For $x\in\OO(n)$ and $y\in\OO(m)$ we have
\[(x\otimes e^n)\tilde{\circ}_i(y\otimes e^m)=(-1)^{(n-1)(m-1)+(n-1)\deg(y)+(i-1)(m-1)}(x\circ_i y)\otimes e^{n+m-1},\]
or written more compactly,
\[x\tilde{\circ}_iy=(-1)^{(n-1)(m-1)+(n-1)\deg(y)+(i-1)(m-1)}x\circ_i y.\]
\end{lem}
\begin{proof}
Let $x\in\OO(n)$ and $y\in\OO(m)$, and let us compute $(x\otimes e^n)\tilde{\circ}_i (y\otimes e^m)$, which we will usually write as $x\tilde{\circ}y$ as an abuse of notation.

\begin{align*}
\mathfrak{s}\OO(n)\otimes\mathfrak{s}\OO(m)&=(\OO(n)\otimes\Lambda(n))\otimes (\OO(m)\otimes\Lambda(m))\\
&\cong (\OO(n)\otimes \OO(m))\otimes (\Lambda(n)\otimes \Lambda(m))\\
&\xrightarrow{\circ_i\otimes\circ_i} \OO(m+n-1)\otimes \Lambda(n+m-1)=\mathfrak{s}\OO(n+m-1).
\end{align*}

The symmetric monoidal structure produces the sign $(-1)^{(n-1)\deg(y)}$ in the isomorphism $\Lambda(n)\otimes \OO(m)\cong\OO(m)\otimes\Lambda(n)$, and the operadic structure of $\Lambda$ produces the sign $(-1)^{(n-i)(m-1)}$, so 

\[(x\otimes e^n)\tilde{\circ}_i(y\otimes e^m)=(-1)^{(n-1)\deg(y)+(n-i)(m-1)}(x\circ_i y)\otimes e^{n+m-1}.\]

More compactly, this can be written as
\begin{equation}\label{tildecircle}
x\tilde{\circ}_iy=(-1)^{(n-1)\deg(y)+(n-i)(m-1)}x\circ_i y.
\end{equation}

Now we can rewrite the exponent using that we have mod 2

\[(n-i)(m-1)=(n-1-i-1)(m-1)=(n-1)(m-1)+(i-1)(m-1)\]

so we conclude 

\[x\tilde{\circ}_iy=(-1)^{(n-1)(m-1)+(n-1)\deg(y)+(i-1)(m-1)}x\circ_i y.\]
\end{proof}
\pagebreak
\begin{remark}
The sign from \Cref{tilde} is exactly the sign in \cite{RW} from which the sign in the equation defining $A_\infty$-algebras, i.e. \Cref{ainftyequation} is derived. This means that if $m_s\in\OO(s)$ has degree $2-s$ and $m_{r+1+t}\in \OO(r+1+t)$ has degree $1-r-t$, we get (abusing notation) that

\[m_{r+1+t}\tilde{\circ}_{r+1}m_s=(-1)^{rs+t}m_{r+1+t}\circ_{r+1}m_s.\]
\end{remark}

Next, we are going to use the above fact to obtain an way to describe $A_\infty$-algebras in simplified operadic terms. We are also going to compare this description with a classical approach that is more general but requires heavier operadic machinery. 


\begin{defin}\label{ainftymult}
An operad $\OO$ has an \emph{$A_\infty$-multiplication} if there is a map $\mathcal{A}_\infty\to\OO$ from the operad of $A_\infty$-algebras.
\end{defin}

 Therefore, we have the following. 

\begin{lem}\label{twisting}
An $A_\infty$-multiplication on an operad $\OO$ is equivalent to an element $m\in\s\OO$ of degree 1 concentrated in positive arity such that $m\tilde{\circ}m=0$, where $x\tilde{\circ} y=\sum_i x\tilde{\circ}_i y$. 
\end{lem}
\begin{proof}
By definition, an $A_\infty$-multiplication on $\OO$ corresponds to a map of operads \[f:\mathcal{A}_\infty\to\OO.\] Such a map is determined by the images of the generators $\mu_i\in\mathcal{A}_\infty(i)$ of degree $2-i$. Whence, $f$ it is determined by $m_i=f(\mu_i)\in\OO(i)$. 

Let $m=m_1+m_2+\cdots$. Since $\deg(m_i)=\deg(\mu_i)=2-i,$
we have that the image of $m_i$ in $\s\OO$ is of degree $2-i+i-1=1$. Therefore, $m\in\s\OO$ is homogeneous of degree 1. Now, let us check that $m\tilde{\circ}m=0$. Note that by equation (\ref{tildecircle}) we have the operation $\tilde{\circ}$ defined as
\[x\tilde{\circ}y=\sum_{i=1}^n(-1)^{(n-1)(m-1)+(n-1)\deg(y)+(i-1)(m-1)}x\circ_i y\]
for $x\in\OO(n)$ and $y\in\OO(m)$. Therefore, applying this definition to $m_{r+1+t}$ and $m_s$ we obtain that
\begin{equation}\label{tildequation}
m_{r+1+t}\tilde{\circ }_{r+1}m_s=(-1)^{rs+t}m_{r+1+t}\circ_{r+1} m_s,
\end{equation}
which is the sign appearing in the definition of an $A_\infty$-algebra (equation (\ref{ainftyequation})). Since the elements $\mu_i$ satisfy the $A_\infty$-equation and $f$ is a map of operads, so do the elements $m_i=f(\mu_i)$. Therefore, we have
\[0=\underset{r,t\geq 0,\ s\geq 1}{\sum_{r+s+t}}(-1)^{rs+t}m_{r+1+t}\circ_{r+1} m_s=\underset{r,t\geq 0,\ s\geq 1}{\sum_{r+s+t}}m_{r+1+t}\tilde{\circ}_{r+1}m_s=m\tilde{\circ}m.\] 
%In the above sum, $r,t\geq 0$ and $s\geq 1$.
Conversely, if $m\in\s\OO$ of degree 1 such that $m\tilde{\circ}m=0$, let $m_i$ be the component of $m$ lying in arity $i$. We have $m=m_1+m_2+\cdots$. By the usual identification, $m_i$ has degree $1-i+1=2-i$ in $\OO$. Now we can use equation (\ref{tildequation}) to conclude that $m\tilde{\circ}m=0$ implies 
\[\underset{r,t\geq 0,\ s\geq 1}{\sum_{r+s+t}}(-1)^{rs+t}m_{r+1+t}\circ_{r+1} m_s=0.\]

This shows that the elements $m_i$ determine a map $f:\mathcal{A}_\infty\to\OO$ defined on generators by $f(\mu_i)=m_i$, as desired. 
\end{proof}

\begin{remark}\label{dg}
If one works with dg-operads, then the definition of $\calA_\infty$ as a quasi-free operad should be used, see \Cref{internal}. In that case, similarly to \Cref{twisting}, the equation that an $A_\infty$-multiplication on $\OO$ satisfies is $\partial(m)+m\tilde{\circ} m = 0$, where $\partial$ is the differential on $\OO$ and $m$ is concentrated on arity at least 2. A similar analysis can be carried out from here, but we will stick to operads of graded modules for the most part.
\end{remark}

We can connect \Cref{ainftymult} with the existing literature. Recall that the Koszul dual cooperad $\mathcal{A}s^{¡}$ of the associative operad $\mathcal{A}s$ is $k\mu_n$ in arity $n$, where $\mu_n$ has degree $n-1$ for $n\geq 1$. Thus, for a graded module $A$, we have the following operad isomorphisms, where the notation $(\geq 1)$ means that we are taking the reduced sub-operad with trivial arity 0 component.


\[\Hom(\mathcal{A}s^{¡},\End_A)\cong \End_{S^{-1}A}(\geq 1)\cong\s\End_A(\geq 1).\]
 
The first operad is the convolution operad (see \cite[\S 6.4.1]{lodayvallette}), for which \[\Hom(\mathcal{A}s^{¡},\End_A)(n)=\Hom_R(\mathcal{A}s^{¡}(n),\End_A(n)).\] Explicitly, for $f\in\End_A(n)$ and $g\in\End_A(m)$, the convolution product is given by

\[f\star g=\sum_{i=1}^n(-1)^{(n-1)(m-1)+(n-1)\deg(b)+(i-1)(m-1)}f\circ_i g=\sum_{i=1}^nf\tilde{\circ}_i g=f\tilde{\circ}g.\]

It is known that $A_\infty$-structures on $A$ are determined by elements $\varphi\in\Hom(\mathcal{A}s^{¡},\End_A)$ of degree 1 such that $\varphi\star \varphi=0$ \cite[Proposition 10.1.3]{lodayvallette}. Since the convolution product coincides with the operation $\tilde{\circ}$, such an element $\varphi$ is sent via the above isomorphism to an element $m\in\s\End_A(\geq 1)$ of degree 1 satisfying $m\tilde{\circ}m=0$. Therefore, we see that this classical interpretation of $A_\infty$-algebras is equivalent to the one that \Cref{twisting} provides in the case of the operad $\End_A$. See \cite[Proposition 10.1.11]{lodayvallette} for more details about convolution operads and the more classical operadic interpretion of $A_\infty$-agebras,  taking into account that in the dg-setting the definition has to be modified slightly, see \Cref{dg}. There is also a difference in sign conventions that arises from the choice of the isomorphism $\End_{SA}\cong\s^{-1}\End_A$, see \Cref{markl}.

What is more, replacing $\End_A$ by any operad $\OO$ and doing similar calculations to \cite[Proposition 10.1.11]{lodayvallette}, we retrieve the notion of $A_\infty$-multiplication on $\OO$ given by \Cref{ainftymult}.

\begin{remark}
Above we needed to specify that only positive arity was considered. This is the case in many situations in literature, but for our purposes, we cannot assume that operads have trivial component in arity 0 in general, and this is what forces us to specify that $A_\infty$-multiplications are concentrated in positive arity.
\end{remark}

When we obtain the signs for the full operadic composition on operadic suspension we will be able to also give an interpretation of $\infty$-morphisms in terms of operadic suspension. But before that, let us expose the relation between operadic suspension and the usual suspension or shift of graded modules.

\begin{thm}\label{markl}(\cite[Chapter 3, Lemma 3.16]{operads})
Given a graded $R$-module $A$, there is an isomorphism of operads $\sigma^{-1}:\End_{S A}\cong \mathfrak{s}^{-1}\End_A$, where $\End_A$ is the endomorphism operad of $A$.
\end{thm}
The original statement is about vector spaces, but it is still true when $R$ is not a field. The proof in the original reference is not very explicit, see  \Cref{proofthm} for a detailed proof. But in the case of the operadic suspension defined above, the isomorphism is given by \[\sigma^{-1}:\End_{S A}\to\mathfrak{s}^{-1}\End_A,\] where $\sigma^{-1}(F)=(-1)^{\binom{n}{2}}S^{-1}\circ F\circ S^{\otimes n}$ for $F\in \End_{S A}(n)$. The symbol $\circ$ here is just composition of maps.
Note that we are using the identification of elements of $\End_A$ with those in $\mathfrak{s}^{-1}\End_A$. The notation $\sigma^{-1}$ comes from \cite{RW}, where a twisted version of this map is the inverse of a map $\sigma$. Here, we define $\sigma:\End_A(n)\to\End_{SA}(n)$ as the map of graded modules given by
\begin{equation}\label{sigma}
\sigma(f)= S\circ f \circ (S^{-1})^{\otimes n}.
\end{equation}

In \cite{RW} the sign for the insertion maps was obtained by computing $\sigma^{-1}(\sigma(a)\circ_i\sigma(b))$. This can be interpreted as sending $x$ and $x$ from $\End_A$ to $\End_{S A}$ via $\sigma$ (which is a map of graded modules, not of operads), and then applying the isomorphism induced by $\sigma^{-1}$. In the end this is the same as simply sending $x$ and $y$ to their images in $\mathfrak{s}^{-1}\End_A$, which is what \Cref{markl} does.

Even though $\sigma$ is only a map of graded modules, it can be shown in a completely analogous way to Theorem \ref{markl} that $\overline{\sigma}=(-1)^{\binom{n}{2}}\sigma$ induces an isomorphism of operads
\begin{equation}\label{barsigma}
\overline{\sigma}:\End_{A}\cong\mathfrak{s}\End_{SA}.
\end{equation}
This isomorphism can also  be proved in a more direct way using the isomorphism \[\s\s^{-1}\OO\cong\OO\] from \Cref{suspiso}, namely, since $\End_{SA}\cong \s^{-1}\End_A$, we have \[\s\End_{SA}\cong \s\s^{-1}\End_A\cong \End_A.\]
In this case the isomorphism map that we obtain goes in the opposite direction to $\overline{\sigma}$, and it is precisely its inverse.


\begin{lem}\label{suspiso}
There are isomorphisms of operads $\mathfrak{s}^{-1}\mathfrak{s}\OO\cong\OO\cong\mathfrak{s}\mathfrak{s}^{-1}\OO$.
\end{lem}
\begin{proof}
We are only showing the first isomorphism since the other one is analogous. Note that as graded $R$-modules, \[\s^{-1}\s\OO(n)= \OO(n)\otimes S^{1-n}R\otimes S^{n-1}R\cong\OO(n),\] 
and any automorphism of $\OO(n)$ determines such an isomorphism. Therefore, we are going to find an automorphism $f$ of $\OO(n)$ such that the above isomorphism induces a map of operads, i.e $f$ induces a map that preserves insertions. Observe that the insertion in $\s^{-1}\s\OO$ differs from that of $\OO$ in just a sign. The insertion on $\s^{-1}\s\OO$ is defined as the composition of the isomorphism
\begin{align*}
(\mathcal{O}(n)\otimes S^{n-1}sig_n\otimes S^{1-n}sig_n)\otimes (\mathcal{O}(m)\otimes S^{m-1}sig_m\otimes S^{1-m}sig_m)\cong\\ (\mathcal{O}(m)\otimes \mathcal{O}(m))\otimes (S^{n-1}sig_n\otimes S^{m-1}sig_m)\otimes (S^{1-n}sig_n\otimes S^{1-m}sig_m)
\end{align*}
and the tensor product of the insertions correspoding to each operad. After all these maps, the only sign left is $(-1)^{(n-1)(m-1)}$. So we need to find an automorphism $f$ of $\OO$ such that, for $x\in\OO(n)$ and $y\in\OO(m)$,

\[f(x\circ_i y)=(-1)^{(n-1)(m-1)}f(x)\circ_i f(y).\]

By \Cref{binom}, $f(x)=(-1)^{\binom{n}{2}}x$ is such an automorphism.
\end{proof}


\subsection{Functorial properties of operadic suspension}\label{functorial}


Here we study operadic suspension at the level of the underlying collections as an endofunctor. Recall from \Cref{collections} that a collection is a family $\OO=\{\OO(n)\}_{n\geq 0}$ of graded $R$-modules.

We define the suspension of a collection $\OO$ as $\mathfrak{s}\OO(n)=\OO(n)\otimes R[n-1]$, where $R[n-1]$ is the ground ring concentrated in degree $n-1$. We first show that $\s$ is a functor both on collections and on operads. %The ring $k[n-1]$ can of course be equipped with the sign action of the symmetric group, so we may have a diagonal action on the tensor product. 
Given a morphism of collections $f:\OO\to\mathcal{P}$, there is an obvious induced morphism

\begin{equation}\label{sf}
\s f:\s\OO\to\s\mathcal{P},\ \s f(x\otimes e^n)=f(x)\otimes e^n.
\end{equation}
Since morphisms of collections preserve arity, this map is well defined because $e^n$ is the same for $x$ and $f(x)$. Note that if $f$ is homogeneous, the degree of $\s f$ is the same as that of $f$.

\begin{lem}
The assigment $\OO\mapsto \s\OO$ and $f\mapsto \s f$ is a functor on both the category $\mathrm{Col}$ of collections and the category $\mathrm{Op}$ of operads.
\end{lem}

\begin{proof}
The assigment preserves composition of maps. Indeed, given any $g:\mathcal{P}\to\CC$, by definition $\s(g\circ f)(x\otimes e^n)=g(f(x))\otimes e^n$, and also \[(\s g\circ \s f)(x\otimes e^n)=\s g (f(x)\otimes e^n)=g(f(x))\otimes e^n.\] This means that $\s$ defines an endofunctor on the category $\mathrm{Col}$ of collections.


We know that when $\mathcal{O}$ is an operad, $\mathfrak{s}\OO$ is again  an operad. What is more, if $f$ is a map of operads, then the map $\s f$ is again a map of operads, since for $a\in\OO(n)$ and $b\in\OO(m)$ we have

\begin{align*}
\s f(x\tilde{\circ}_i y)&=\s f ((x\otimes e^n)\tilde{\circ}_i (y\otimes e^m))\\
&=(-1)^{(n-1)\deg(y)+(n-i)(m-1)}\s f((x\circ_i y) \otimes e^{n+m-1})\\
&=(-1)^{(n-1)\deg(y)+(n-i)(m-1)}f(x\circ_i y)\otimes e^{n+m-1}\\
&=(-1)^{(n-1)\deg(y)+(n-i)(m-1)+\deg(f)\deg(x)}(f(x)\circ_i f(y))\otimes e^{n+m-1}
\end{align*}
By the definition of $\tilde{\circ}$ this equals
\begin{align*}
&(-1)^{(n-1)\deg(y)+(n-1)(\deg(y)+\deg(f))+\deg(f)\deg(x)}(f(x)\otimes e^n)\tilde{\circ}_i (f(y)\otimes e^m)\\
&=(-1)^{\deg(f)(\deg(x)+n-1)}\s f(x)\tilde{\circ}_i\s f(y).
\end{align*}
Note that $\deg(x)+n-1$ is the degree of $a\otimes e^n$ and as we said before $\deg(\s f)=\deg(f)$, so the above relation is consistent with the Koszul sign rule. In any case, recall that a morphism of operads is necessarily of degree 0, but the above calcultion hints at some monoidality properties of $\s$ that we will study afterwards. Clearly $\s f$ preserves the unit, so $\s f$ is a morphism of operads. 
\end{proof}



The fact that $\s$ is a functor allows to describe algebras over operad using operadic suspension. For instance, an $A_\infty$-algebra is a map of operads $\OO\to\mathcal{P}$ where $\OO$ is an operad with $A_\infty$-multiplication. Since $\s$ is a functor, this map corresponds to a map $\s\OO\to\s\mathcal{P}$. Since in addition the map $\s\OO\to\s\mathcal{P}$ is fully determined by the original map $\OO\to\mathcal{P}$, this correspondence is bijective, and algebras over $\OO$ are equivalent to algebras over $\s\OO$. In fact, using \Cref{suspiso}, it is not hard to show the following.

\begin{propo}
The functor $\s$ is an equivalence of categories both at the level of collections and at the level of operads. \qed %it is not an isomorphism because at the level of collections, not every collection is EQUAL to some suspension
\end{propo}
In particular, for $A_\infty$-algebras it is more convenient to work with $\s\OO$ since the formulation of an $A_\infty$-multiplication on this operad is much simpler but we do not lose any information.
\pagebreak
\subsection{Monoidal properties of operadic suspension}\label{monoidalsusp}
Now we are going to explore the monoidal properties of operadic suspension. Since operads are precisely monoids on the category $\mathrm{Col}$ of collections, we have the following.
\begin{propo}\label{monoidality} %\mbox{}
The endofunctor $\s:\mathrm{Col}\to\mathrm{Col}$ sends monoids to monoids and morphisms of monoids to morphisms of monoids, in other words, it induces a well-defined endofunctor on the category of monoids $\mathrm{Mon}(\mathrm{Col})$. \qed%is  lax monoidal with respect to the composition of $\mathbb{S}$-modules. 
\end{propo}


In fact, we can show a stronger result.

\begin{propo}
The functor $\s:\col\to \col$ defines a lax monoidal functor. When restricted to the subcategory of reduced operads, it is strong monoidal.
\end{propo}
\begin{proof}
Firstly, we need to define the structure maps of a lax monoidal functor. Namely, we define the unit morphism $\varepsilon:I\to\s I$ to be the map $\varepsilon(n):I(n)\to I(n)\otimes S^{n-1}R$ to be the identity for $n\neq 1$ and the isomorphism $R\cong R\otimes R$ for $n=1$. We also need to define a natural transformation $\mu:\s\OO\circ\s\PP\to\s(\OO\circ\PP)$. To define it, observe that for $\PP=\OO$ we would want the map

\[\s\OO\circ\s\OO\xrightarrow{\mu}\s(\OO\circ\OO)\xrightarrow{\s\gamma}\s\OO\]
 to coincide with the operadic composition $\tilde{\gamma}$ on $\s\OO$, where $\gamma$ is the composition on $\OO$. 
 
We know that $\s\gamma$ does not add any signs. Therefore, if $\tilde{\gamma}=(-1)^\eta\gamma$, with $\eta$ explicitly computed in \Cref{bracesign}, the sign must come entirely from the map $\s\OO\circ\s\OO\to\s(\OO\circ\OO)$. Thus, we define the map \[\mu:\s\OO\circ\s\PP\to\s(\OO\circ\PP)\] as the map given by
 \[x\otimes e^N\otimes x_1\otimes e^{a_1}\otimes\cdots\otimes x_N\otimes e^{a_N}\mapsto (-1)^\eta x\otimes x_1\otimes\cdots\otimes x_N \otimes e^n,\]
 where $a_1+\cdots+a_N=n$ and 
 \[\eta=\sum_{j<l}a_j\deg(b_l)+\sum_{j=1}^N (a_j+\deg(b_j)-1)(N-j),\]
 which is the case $k_0=\cdots=k_n=0$ in \Cref{bracesign}. Note that $(-1)^\eta$ only depends on degrees and arities, so the map is well defined. Another way to obtain this map is using the associativity isomorphisms and operadic composition on $\Lambda$ to obtain a map $\s\OO\circ\s\PP\to\s(\OO\circ\PP)$.
 
We now show that $\mu$ is natural, or in other words, for $f:\OO\to\OO'$ and $g:\PP\to\PP'$, we show that the following diagram commutes.
\[\begin{tikzcd}
\mathfrak{s}\mathcal{O}\circ\mathfrak{s}\mathcal{P} \arrow[r, "\mathfrak{s}f\circ\mathfrak{s}g"] \arrow[d, "\mu"'] & \mathfrak{s}\mathcal{O}'\circ\mathfrak{s}\mathcal{P}' \arrow[d, "\mu"] \\
\mathfrak{s}(\mathcal{O}\circ\mathcal{P}) \arrow[r, "\mathfrak{s}(f\circ g)"]                                      & \mathfrak{s}(\mathcal{O}'\circ\mathcal{P}')                           
\end{tikzcd}\]
 Let $c=x\otimes e^N\otimes x_1\otimes e^{a_1}\otimes\cdots\otimes x_N\otimes e^{a_N}\in \mathfrak{s}\mathcal{O}\circ\mathfrak{s}\mathcal{P}$ and let us compute $\s(f\circ g)(\mu(c))$. One has
 
\begin{align*}
\s(f\circ g)(\mu(c))&=\s(f\circ g)((-1)^{\varepsilon} x\otimes x_1\otimes\cdots\otimes x_N \otimes e^n)\\
&=(-1)^{\varepsilon+\delta}f(x)\otimes g(x_1)\otimes\cdots\otimes g(x_N) \otimes e^n\
\end{align*}
where
\[\varepsilon=\sum_{j<l}a_j\deg(x_l)+\sum_{j=1}^N (\deg(x_j)+a_j-1)(N-j).\]
and
\[\delta = N\deg(g)\deg(x)+\deg(g)\sum_{j=1^N}\deg(x_j)(N-j)\]


Now, let us compute $\mu((\s f\circ \s  g)(c))$. We have

%\[\mu((\s f\circ \s  g)(c))=\mu((-1)^{k\deg(g)(\deg(a)+k-1)+\deg(g)\sum_{j=1}^k(\deg(b_j)+i_j-1)(k-j)}\s f(a\otimes e^k)\otimes \s g(b_1\otimes e^{i_1})\otimes\cdots\otimes \s g(b_k\otimes e^{i_k}))=\]
\[\mu((\s f\circ \s  g)(c))=(-1)^\sigma f(x)\otimes g(x_1)\otimes\cdots\otimes g(x_N) \otimes e^n,\]
where 
\begin{align*}
\sigma=& \deg(g)\sum_{j=1}^N(\deg(x_j)+a_j-1)(N-j)+N\deg(g)(\deg(x)+N-1)\\
&+\sum_{j<l}a_j(\deg(x_j)+\deg(g))\\
&+\sum_{j=1}^N(a_j+\deg(x_j)+\deg(g)-1)(N-j).
\end{align*}
 
 Now we compare the two signs by computing $\varepsilon+\delta+\sigma\mod 2$. After some cancellations of common terms and using that $N(N-1)=0\mod 2$ we get
 
 \begin{align*}
 &\deg(g)\sum_{j=1}^N(a_j-1)(N-j)+\sum_{j<l}a_j\deg(g)+\sum_{j=1}^N\deg(g)(N-j)\\
 &=\deg(g)\sum_{j=1}^Na_j(N-j)+\deg(g)\sum_{j<l}a_j\\
 &=\deg(g)\left(\sum_{j=1}^N a_j(N-j)+\sum_{j=1}^N a_j(N-j)\right)\\
 &=0\mod 2.
 \end{align*}

 This shows naturality of $\mu$. %Next, we have to show that $\mu$ and $\varepsilon$ satisfy the axioms of a lax monoidal functor. 
 Unitality follows directly from the definitions by direct computation. In the case of associativity, oberve that by the definition of $\mu$, the associativity axiom for $\mu$ is equivalent to the associativity of the operadic composition $\tilde{\gamma}$, which we know to be true. This shows that $\s$ is a lax monoidal functor.
 
In the case where the operads have trivial arity 0 component, we may define an inverse to the operadic composition on $\Lambda$ from \Cref{Sec2}. Namely, for $n>0$, we may define

\[\Lambda(n)\to \bigoplus_{N\geq 0} \Lambda(N)\otimes\left(\bigoplus_{a_1+\cdots+a_N=n}\Lambda(a_1)\otimes\cdots\otimes\Lambda(a_N)\right)\]
as the map
\[e^n\mapsto\sum_{a_1+\cdots+a_N=n}(-1)^{\delta}e^N\otimes  e^{a_1}\otimes\cdots\otimes e^{a_N},\]
where $\delta$ is the same sign that appears in the operadic composition on $\Lambda$ (see \Cref{bracesign}) and where $a_1,\dots,a_k>0$. Since there are only finitely many ways of decomponsing $n$ into $N$ positive integers, the sum is finite and thus the map is well defined. In fact, this map defines a cooperad structure on the reduced sub-operad of $\Lambda$ with trivial arity 0 component. This map induces the morphism $\mu^{-1}:\s(\OO\circ\PP)\to \s\OO\circ\s\PP$ that we are looking for.

The unit morphism $\varepsilon$ is always an isomorphism, so this shows $\s$ is strong monoidal in the reduced case.

\end{proof}

\begin{remark}
If we decide to work with symmetric operads, we just need to introduce the sign action of the symmetric group on $\Lambda(n)$, turning it into the sign representation of the symmetric group. The action on tensor products is diagonal, and the results we have obtained follow similarly replacing $\col$ by the category of $\mathbb{S}$-modules.
\end{remark}

\section{Brace algebras}\label{sectionbraces}
Brace algebras appear naturally in the context of operads when we fix the first argument of operadic composition \cite{GV}. This simple idea gives rises to a very rich structure that is the building block of the derived $A_\infty$-structures that we are going to construct.

In this section we define a brace algebra structure for an arbitrary operad using operadic suspension. Using operadic suspension will have as a result  a generalization of the Lie bracket defined in \cite{RW}. First recall the definition of a brace algebra.
\pagebreak
\begin{defin}\label{braces}
A \emph{brace algebra} on a graded module $A$ consists of a family of maps \[b_n:A^{\otimes 1+n}\to A\] called \emph{braces}, that we evaluate on $(x,x_1,\dots, x_n)$ as $b_n(x;x_1,\dots, x_n)$. They must satisfy the \emph{brace relation}


\begin{align*}
b_m(b_n(x;x_1,\dots, x_n);y_1,\dots,y_m)=&\\
\underset{j_1\dots, j_n}{\sum_{i_1,\dots, i_n}}(-1)^{\varepsilon}b_l(x; y_1,\dots,b_{j_1}(x_1;y_{i_1+1},&\dots),\dots, b_{j_n}(x_n;y_{i_n+1},\dots),\dots,y_m)
\end{align*}
where $l=n+\sum_{p=1}^n i_p$ and $\varepsilon=\sum_{p=1}^n\deg(x_p)\sum_{q=1}^{i_p}\deg(y_q),$ i.e. the sign is picked up by the $x_i$'s passing by the $y_i$'s in the shuffle.
\end{defin}

\begin{remark}
Some authors might use the notation $b_{1+n}$ instead of $b_n$, but the first element is usually going to have a different role from the others, so we found $b_n$ more intuitive. A shorter notation for $b_n(x;x_1,\dots,x_n)$ found in the literature (\cite{GV}, \cite{getzler}) is $x\{x_1,\dots, x_n\}$. 
\end{remark}

We will also see a bigraded version of this kind of map in \Cref{sectionbraces}.
\subsection{Brace algebra structure on an operad}


Given an operad $\OO$ with composition map $\gamma:\OO\circ\OO\to\OO$ we can define a brace algebra on the underlying module of $\OO$ by setting
\[b_n:\OO(N)\otimes\OO(a_1)\otimes\cdots\otimes\OO(a_n)\to\OO\left(N-n+\sum a_i\right)\]

\[b_n(x;x_1,\dots, x_n)=\sum\gamma(x;1,\dots,1,x_1,1,\dots,1,x_n,1,\dots,1),\]
where the sum runs over all possible order-preserving insertions. The brace $b_n(x;x_1,\dots,x_n)$ vanishes whenever $n>N$ and $b_0(x)=x$. The brace relation follows from the associativity axiom of operads.


This construction can  be used to define braces on $\s\OO$. More precisely, we define maps 
\[b_n:\mathfrak{s}\OO(N)\otimes\mathfrak{s}\OO(a_1)\otimes\cdots\otimes\mathfrak{s}\OO(a_n)\to\mathfrak{s}\OO\left(N-n+\sum a_i\right)\]
using the operadic composition $\tilde{\gamma}$ on $\mathfrak{s}\OO$ as

\[b_n(x;x_1,\dots,x_n)=\sum\tilde{\gamma}(x;1,\dots,1,x_1,1,\dots,1,x_n,1,\dots,1).\]

%\begin{remark} For Constanze and I. I am thinking of using tilde notation $\tilde{b}_n$ and $\tilde{\gamma}$ for the maps defined on operadic suspension, but I am not sure if this is going to be too cumbersome or unnecesssary. Here I am just using it for $\tilde{\gamma}$ because that operation does not appear too often.
%\end{remark}
We have the following relation between the brace maps $b_n$ defined on $\s\OO$ and the operadic composition $\gamma$ on $\OO$. 
\begin{propo}\label{bracesign}
For $x\in \s\OO(N)$ and $x_i\in\s\OO(a_i)$ of internal degree $q_i$ ($1\leq i\leq n$), we have
\[b_n(x;x_1,\dots,x_n)=\sum_{N-n=k_0+\cdots+k_n} (-1)^\eta \gamma
(x\otimes 1^{\otimes k_0}\otimes x_1\otimes \cdots\otimes x_n\otimes1^{\otimes k_n}),\]
where 
\[\eta=\sum_{0\leq j<l\leq n}k_jq_l+\sum_{1\leq j<l\leq n}a_jq_l+\sum_{j=1}^n (a_j+q_j-1)(n-j)+\sum_{1\leq j\leq l\leq n} (a_j+q_j-1)k_l.\]
\end{propo}


\begin{proof}
To obtain the signs that make $\tilde{\gamma}$ differ from $\gamma$, we must first look at the operadic composition on $\Lambda$. 
We are interested in compositions of the form \[\tilde{\gamma}(x\otimes 1^{\otimes k_0}\otimes x_1\otimes 1^{\otimes k_1}\otimes\cdots\otimes x_n\otimes 1^{\otimes k_n})\] where $N-n=k_0+\cdots+k_n$, $x$ has arity $N$ and each $x_i$ has arity $a_i$ and internal degree $q_i$. Therefore, let us consider the corresponding operadic composition 

\[
\Lambda(N)\otimes\Lambda(1)^{k_0}\otimes\Lambda(a_1)\otimes\cdots\otimes\Lambda(a_n)\otimes\Lambda(1)^{k_n}\longrightarrow \Lambda\left(N-n+\sum_{i=1}^na_i\right).
\]

The operadic composition can be described in terms of insertions in the obvious way, namely, if $f\in\s\OO(N)$ and $h_1,\dots, h_N\in\s\OO$, then we have

\[\tilde{\gamma}(x;y_1,\dots, y_N)=(\cdots(x\tilde{\circ}_1 y_1)\tilde{\circ}_{1+a(y_1)}y_2\cdots)\tilde{\circ}_{1+\sum a(y_p)}y_N,\]

where $a(y_p)$ is the arity of $y_p$ (in this case $y_p$ is either $1$ or some $x_i$). So we just have to find out the sign iterating the same argument as in the $i$-th insertion. In this case, each $\Lambda(a_i)$ produces a sign given by the exponent $$(a_i-1)(N-k_0+\cdots-k_{i-1}-i).$$ 

For this, recall that the degree of $\Lambda(a_i)$ is $a_i-1$ and that the generator of this space is inserted in the position $1+\sum_{j=0}^{i-1}k_j+\sum_{j=1}^{i-1}a_j$ of a wedge of $N+\sum_{j=1}^{i-1}a_j-i+1$ generators. Therefore, performing this insertion as described in the previous section yields the aforementioned sign. Now, since $N-n=k_0+\cdots+k_n$, we have that
\[(a_i-1)(N-k_0+\cdots+k_{i-1}-i)=(a_i-1)\left(n-i+\sum_{l=i}^nk_l\right).\]

Let us introduce for the rest of the proof the notation $a_0 = N$ for the sake of compactness of the formulas. Now we can compute the sign factor of a brace. For this, notice that the isomorphism \[(\OO(1)\otimes \Lambda(1))^{\otimes k}\cong \OO(1)^{\otimes k}\otimes \Lambda(1)^{\otimes k}\] does not produce any signs because of degree reasons. Therefore, therefore the sign coming from the isomorphism

%\[\OO(N)\otimes\Lambda(N)\otimes (\OO(1)\otimes \Lambda(1))^{\otimes k_0}\otimes \bigotimes_{i=1}^n(\OO(a_i)\otimes\Lambda(a_i)\otimes(\OO(1)\otimes\Lambda(1))^{\otimes k_i}\]
%\[\cong \OO(N)\otimes\OO(1)^{\otimes k_0}\otimes(\bigotimes_{i=1}^n \OO(a_i)\otimes \OO(1)^{\otimes k_i})\otimes \Lambda(N)\otimes\Lambda(1)^{\otimes k_0}\otimes(\bigotimes_{i=1}^n \Lambda(a_i)\otimes \Lambda(1)^{\otimes k_i})\]
\begin{align*}
&\bigotimes_{i=0}^n\left(\OO(a_i)\otimes\Lambda(a_i)\otimes(\OO(1)\otimes\Lambda(1)\right)^{\otimes k_i}\\
&\cong \left(\bigotimes_{i=0}^n \OO(a_i)\otimes \OO(1)^{\otimes k_i}\right)\otimes \left(\bigotimes_{i=0}^n \Lambda(a_i)\otimes \Lambda(1)^{\otimes k_i}\right)
\end{align*}
is determined by the exponent

\[(N-1)\sum_{i=1}^nq_i+\sum_{i=1}^n (a_i-1)\sum_{l>i}q_l.\]

This equals
\[\left(\sum_{j=0}^nk_j +n-1\right)\sum_{i=1}^nq_i+\sum_{i=1}^n (a_i-1)\sum_{l>i}q_l.\]

After doing the operadic composition 
\[\bigotimes_{i=0}^n \OO(a_i)\otimes \bigotimes_{i=0}^n \Lambda(a_i)\to \OO\left(-n+\sum_{i=0}^na_i\right)\otimes \Lambda\left(-n+\sum_{i=0}^na_i\right)\]

we can add the sign coming from the suspension, so all in all the sign $(-1)^\eta$ we were looking for is given by

\[\eta=\sum_{i=1}^n(a_i-1)(n-i+\sum_{l=i}^nk_l)+(\sum_{j=0}^nk_j +n-1)\sum_{i=1}^nq_i+\sum_{i=1}^n (a_i-1)\sum_{l>i}q_l.\]

It can be checked that this can be rewritten modulo $2$ as 
\[\eta=\sum_{0\leq j<l\leq n}k_jq_l+\sum_{1\leq j<l\leq n}a_jq_l+\sum_{j=1}^n (a_j+q_j-1)(n-j)+\sum_{1\leq j\leq l\leq n} (a_j+q_j-1)k_l\]
as we stated.
\end{proof}

 Notice that for $\OO=\End_A$, the brace on operadic suspension is precisely
 
 \[b_n(f;g_1,\dots,g_n)=\sum (-1)^\eta f(1,\dots,1,g_1,1,\dots,1,g_n,1,\dots,1).\]
Using the brace structure on $\s\End_A$, the sign $\eta$ gives us in particular the the same sign of the Lie bracket defined in \cite{RW}. More precisely, we have the following.

\begin{corollary} The brace $b_1(f;g)$ is the operation $f\circ g$ defined in \cite{RW} that induces a Lie bracket on the Hochschild complex of an $A_\infty$-algebra via
\[
[f,g]=b_1(f;g)-(-1)^{|f||g|}b_1(g;f).
\]
\end{corollary} 
However, we may use $f\tilde{\circ}g$ to make clear that we are using the operadic composition in $\s\OO$. Note that

\[
b_1(f;g)=\sum_i f\tilde{\circ}_i g=f\tilde{\circ}g,
\]
so the notation $f\tilde{\circ} g$ is suggestive for operadic suspension. The notation $f\circ g$ will still be used whenever the insertion maps are denoted by $\circ_i$.

In \cite{RW}, the sign is computed using a strategy that we generalize in \Cref{rw}, see \Cref{eta}. The approach we have followed here has the advantage that the brace relation follows immediatly from the associativity axiom of operadic composition. This approach also works for any operad since the difference between $\gamma$ and $\tilde{\gamma}$ is going to be the same sign. 

\subsection{Reinterpretation of $\infty$-morphisms}\label{reinterpretation}
As we mentioned before, we can show an alternative description of $\infty$-morphisms of $A_\infty$-algebras and their composition in terms of suspension of collections (recall \Cref{inftymorphism} for the definition of these morphisms).

Defining the suspension $\mathfrak{s}$ at the level of collections as we did in \Cref{functorial} allows us to talk about $\infty$-morphisms of $A_\infty$-algebras in this setting, since they live in collections of the form\[\End^A_B=\{\Hom_R(A^{\otimes n},B)\}_{n\geq 1}.\] More precisely, there is a left module structure on $\End^A_B$ over the operad $\End_B$
\[\End_B\circ \End^A_B\to \End^A_B\] given by compostion of maps 

\[f\otimes g_1\otimes\cdots\otimes g_n\mapsto f(g_1\otimes\cdots\otimes g_n)\]
for $f\in\End_B(n)$ and $g_i\in \End^A_B$, and also an infinitesimal right module structure over the operad  $\End_A$ 
\[\End^A_B \circ_{(1)} \End_A\to \End^A_B\]
given by insertion of maps

\[f\otimes 1^{\otimes r}\otimes g\otimes 1^{\otimes n-r-1}\mapsto f(1^{\otimes r}\otimes g\otimes 1^{\otimes n-r-1})\] for $f\in \End^A_B(n)$ and $g\in \End_A$.  In addition, we have a composition $\End^B_C\circ \End^A_B\to\End^A_C$ analogous to the left module described above. They induce maps on the respective operadic suspensions which differ from the original ones by some signs that can be calculated in an analogous way to what we do on \Cref{bracesign}. These induced maps will give us the characterization of $\infty$-morphisms in \Cref{infinitymorphisms}.

For these collections we also have $\mathfrak{s}^{-1}\End^A_B\cong \End^{SA}_{SB}$ in analogy with \Cref{markl}, and the proof is similar but shorter since we do not need to worry about insertions. 


\begin{lem}\label{infinitymorphisms}
An $\infty$-morphism of $A_\infty$-algebras $A\to B$ with respective structure maps $m^A$ and $m^B$ is equivalent to an element $f\in\s\End^A_B$ of degree 0 concentrated in positive arity such that \[\rho(f\circ_{(1)}m^A)=\lambda(m^B\circ f),\] 

where \[\lambda:\mathfrak{s}\End_B\circ \mathfrak{s}\End^A_B\to \mathfrak{s}\End^A_B\] is induced by the left module structure on $\End^A_B$ and \[\rho:\mathfrak{s}\End_B\circ_{(1)}\mathfrak{s}\End^A_B\to \mathfrak{s}\End^A_B\] is induced by the right infinitesimal module structure on $\End^A_B$. 

In addition, the composition of $\infty$-morphisms is given by the natural composition \[\s\End^B_C\circ \s\End^A_B\to \s\End^A_C.\]
\end{lem}
\begin{proof}
From the definitions of the operations in the equation

\begin{equation}\label{operadicmorphism}
\rho(f\circ_{(1)}m^A)=\lambda(m^B\circ f),
\end{equation} 

we know that this equation coincides with the one defining $\infty$-morphisms of $A_\infty$-algebras (\Cref{inftymorphism}) up to sign. The signs that appear in the above equation are obtained in a similar way to that on $\tilde{\gamma}$ (see the proof of \Cref{bracesign}). Thus, it is enough to plug in $\eta$ (the sign from \Cref{bracesign}) the corresponding degrees and arities to obtain the desired result. The composition of $\infty$-morphisms follows similarly.
\end{proof}
Notice the similarity between this definition and the definitions given in \cite[Section 10.2.4]{lodayvallette} taking into account the minor modifications to accommodate the dg-case.

In the case that $f:A\to A$ is an $\infty$-endomorphism, \Cref{operadicmorphism} can be written in terms of operadic composition as $f\tilde{\circ}m=\tilde{\gamma}(m\circ f)$. 


\section{$A_\infty$-algebra structures on operads}\label{sect2}


Let $\OO$ be an operad of graded $R$-modules and $\s\OO$ its operadic suspension. Let us consider the underlying graded module of the operad $\s\OO$, which we  call $\s\OO$ again by abuse of notation, i.e. \[\s\OO=\prod_n \s\OO(n)\] with grading given by its \emph{natural degree}, i.e. $|x|=\deg(x)+n-1$ for $x\in \s\OO(n)$, where $\deg(x)$ is its internal degree (the degree as an element of $\OO(n))$. 

For any operad $\OO$, recall the operation $\circ$ defined as

\[
x\circ y=\sum_{i=1}^n x\circ_i y\in\OO(n+m-1)
\]
for $x\in\OO(n)$ and $y\in \OO(m)$. We write $x\tilde{\circ}y$ for the corresponding operation on $\s\OO$, namely

\[
x\tilde{\circ} y=\sum_{i=1}^n x\tilde{\circ}_i y=b_1(x;y)\in\s\OO(n+m-1)
\]

where
\[x\tilde{\circ}_iy=(-1)^{(n-1)\deg(y)+(n-i)(m-1)}x\circ_i y.\]

\pagebreak
\begin{defin}\label{ainftymultiplication}
Let $m\in\s\OO$ be of natural degree 1 and concentrated in positive arity such that $m\tilde{\circ}m=0$, or equivalently $m=m_1+m_2+\cdots$ is a formal sum of maps $m_j\in\OO(j)^{2-j}$ satisfying the usual $A_\infty$-equation for all $n\geq 1$
\begin{equation}\label{Ainftyeq}
\sum_{r+s+t=n}(-1)^{rs+t}m_{r+1+t}\circ_{r+1}m_s=0.
\end{equation} 
Such $m$ is an \emph{$A_\infty$-multiplication} on $\OO$. As we saw in \Cref{twisting}, its existence is equivalent to a map of operads $\mathcal{A}_\infty\to \OO$ from the operad $\mathcal{A}_\infty$ of $A_\infty$-algebras to $\OO$. We may call each $m_j$ the $j$-th \emph{component} of $m$.
\end{defin}

\begin{remark}\label{multiplicationalgebra}
An $A_\infty$-multiplication on the operad $\End_A$ is equivalent to an $A_\infty$-algebra structure on $A$.
\end{remark}

Following \cite{GV} and \cite{getzler}, if we have an $A_\infty$ multiplication $m\in\OO$, one would define an $A_\infty$-algebra structure on $\s\OO$ using the maps 

\begin{align*}
M'_1(x)\coloneqq [m,x]=m\tilde{\circ} x-(-1)^{|x|}x\tilde{\circ}m, & &  \\
M'_j(x_1,\dots, x_j)\coloneqq b_j(m;x_1,\dots, x_j),& &j>1.
\end{align*}
The prime notation here is used to indicate that these are not the definitive maps that we are going to take. Getzler shows in \cite{getzler} that $M'=M'_1+M'_2+\cdots$ satisfies the relation $M'\circ M'=0$ using that $m\circ m=0$, and the proof is independent of the operad in which $m$ is defined, so it is still valid if $m\tilde{\circ}m=0$. But we have two problems here. The equation $M'\circ M'=0$ does depend on how the circle operation is defined, more precisely, this circle operation in \cite{getzler} is the natural circle operation on the endomorphism operad, which does not have any additional signs, so $M'$ is not an $A_\infty$-structure under our convention. The other problem has to do with the degrees. We need $M'_j$ to be homogeneous of degree $2-j$ as a map $\s\OO^{\otimes j}\to \s\OO$, but we find that $M'_j$ is homogeneous of degree 1 instead as the following lemma shows.
\begin{lem}\label{lemmadegree}
For $x\in\s\OO$ we have that  the degree of $b_j(x;-)$ as a map of graded modules \[b_j(x;-):\s\OO^{\otimes j}\to\s\OO\] is precisely $|x|$.
\end{lem}
\begin{proof}
Let $a(x)$ denote the arity of $x$, i.e. $a(x)=n$ whenever $x\in\s\OO(n)$. Also, let $\deg(x)$ be its internal degree in $\OO$. The natural degree of $b_j(x;x_1,\dots,x_j)$ for $a(x)\geq j$ is computed as follows. By definition, we have that the natural degree of $b_j(x;x_1,\dots,x_j)$ as an element of $\s\OO$ is

\[|b_j(x;x_1,\dots,x_j)|=a(b_j(x;x_1,\dots,x_j))+\deg(b_j(x;x_1,\dots,x_j))-1.\]

We have 

\[a(b_j(x;x_1,\dots,x_j))=a(x)-j+\sum_i a(x_i)\]

and 

\[\deg(b_j(x;x_1,\dots,x_j)=\deg(x)+\sum_i\deg(x_i).\]

Combining these two we obtain

\begin{align*}
a(b_j(x;x_1,\dots,x_j))+\deg(b_j(x;x_1,\dots,x_j))-1=\\
a(x)-j+\sum_i a(x_i)+\deg(x)+\sum_i\deg(x_i)-1=\\
a(x)+\deg(x)-1+\sum_i a(x_i)+\sum_i\deg(x_i)-j=\\
a(x)+\deg(x)-1+\sum_i (a(x_i)+\deg(x_i)-1)=\\
|x|+\sum_i|x_i|.
\end{align*}
This means that the degree of the map $b_j(x;-)$ as a map $\s\OO^{\otimes j}\to \s\OO$ equals $|x|$.

\end{proof} %A first alternative after finding this result is considering $M'_j$ to be an element of $\s\End_{\s\OO}$ instead of just $\End_{\s\OO}$. This solves the problem of the degree, but not the one of the sign convention. 

\begin{corollary}
The maps 
\begin{equation*}
M_j':\s\OO^{\otimes j}\to \s\OO,\, (x_1,\dots, x_j)\mapsto b_j(m;x_1,\dots, x_j)
\end{equation*}
for $j>1$ and the map
\begin{equation*}
M_1':\s\OO\to \s\OO,\, x\mapsto b_1(m;x)-(-1)^{|x|}b_1(m;x)
\end{equation*}
are homogeneous of degree 1. 
\end{corollary}
\begin{proof}
For $j>1$ it is a direct consequence of \Cref{lemmadegree}. For $j=1$ we have the summand $b_1(m;x)$ whose degree follows as well from \Cref{lemmadegree}. The degree of the other summand, $b_1(x;m)$, can be computed in a similar way as in the proof \Cref{lemmadegree}, giving that $|b_1(x;m)|=1+|x|$. This concludes the proof.
\end{proof}

The problem we have encountered with the degrees can be resolved using shift maps as the following proposition shows. Recall that the \emph{shift} of a graded module $A$ is given by $SA^i=A^{i-1}$ and that we have maps $A\to SA$ of degree 1 given by the identity. 

\begin{propo}\label{ainftystructure}
If $\OO$ is an operad with an $A_\infty$-multiplication $m\in\OO$, then there is an $A_\infty$-algebra structure on the shifted module $S\s\OO$. 
\end{propo}
\begin{proof}
Note in the proof of \Cref{lemmadegree} that a way to turn $M'_j$ into a map of degree $2-j$ is introducing a grading on $\s\OO$ given by arity plus internal degree (without subtracting 1). This is equivalent to defining an $A_\infty$-algebra structure $M$ on $S\s\OO$ shifting the map $M'=M'_1+M'_2+\cdots$, where $S$ is the shift of graded modules. Therefore, we define $M_j$ to be the map making the following diagram commute.

\[
\begin{tikzcd}
(S\s\OO)^{\otimes j}\arrow[r,"M_j"]\arrow[d, "(S^{\otimes j})^{-1}"'] & S\s\OO\\
\s\OO^{\otimes j}\arrow[r, "M'_j"] & \s\OO\arrow[u,"S"']
\end{tikzcd}
\]

In other words, $M_j=\overline{\sigma}(M'_j)$, where $\overline{\sigma}(F)=S\circ F\circ (S^{\otimes n})^{-1}$ for $F\in\End_{\s\OO}(n)$ is the map inducing an isomorphism $\End_{\s\OO}\cong \s\End_{S\s\OO}$ (\Cref{barsigma}). Since $\overline{\sigma}$ is an operad morphism, for $M=M_1+M_2+\cdots$, we have

\[
M\tilde{\circ}M=\overline{\sigma}(M')\tilde{\circ}\overline{\sigma}(M')=\overline{\sigma}(M'\circ M')=0.
\]
%MAYBE DEFINE $\overline{\sigma}_n$ FOR EACH ARITY SO THAT THE ABOVE IS NOT AN ABUSE OF NOTATION. OTHERWISE SAY IT IS AN ABUSE OF NOTATION

So now we have that $M\in\s\End_{S\s\OO}$ is an element of natural degree 1 such that $M\tilde\circ M=0$. Therefore, in light of \Cref{multiplicationalgebra}, $M$ is the desired $A_\infty$-algebra structure on $S\s\OO$. 
\end{proof}
Notice that $M$ is defined as an structure map on $S\s\OO$. This kind of shifted operad is called \emph{odd operad} in \cite{ward}. This means that $S\s\OO$ is not an operad anymore, since the associativity relation for graded operads involves signs that depend on the degrees, which are now shifted. 

\subsection{Iterating the process}\label{sect3}

We have defined $A_\infty$-structure maps $M_j\in\s\End_{S\s\OO}$. Now we can use the brace structure of the operad $\s\End_{S\s\OO}$ to define get $A_\infty$-algebra structure given by maps
\begin{equation}\label{barmaps}
\overline{M}_j:(S\s\End_{S\s\OO})^{\otimes j}\to S\s\End_{S\s\OO}
\end{equation}
by applying $\overline{\sigma}$ to maps
\[\overline{M}'_j:(\s\End_{S\s\OO})^{\otimes j}\to \s\End_{S\s\OO}\]
defined as
\begin{align*}
&\overline{M}'_j(f_1,\dots,f_j)=\overline{B}_j(M;f_1,\dots, f_j) & j>1,\\
&\overline{M}'_1(f)=\overline{B}_1(M;f)-(-1)^{|f|}\overline{B}_1(f;M),
\end{align*}
where $\overline{B}_j$ denotes the brace map on $\s\End_{S\s\OO}$.

We define the Hochschild complex as done by Ward in \cite{ward}.

\begin{defin}
The \emph{Hochschild cochains} of a graded module $A$ is defined to be the graded module $S\s\End_A$. If  $(A,d)$ is a chain complex, then $S\s\End_A$ is endowed with a differential \[\partial(f)=[d,f]=d\circ f -(-1)^{|f|}f\circ d\] where $|f|$ is the natural degree of $f$ and $\circ$ is the plethysm operation given by insertions.
\end{defin}
In particular, $S\s\End_{S\s\OO}$ is the module of Hochschild cochains of $S\s\OO$. If $\OO$ has an $A_\infty$-multiplication, then the differential of the Hochschild complex is $\overline{M}_1$ from \Cref{barmaps}.
\begin{remark}
The functor $S\s$ is called the ``oddification'' of an operad in the literature \cite{ward}. %Ward but the whole thesis 
The reader might find odd to define the Hochschild complex in this way instead of just $\End_A$. The reason is that the operadic suspension provides the necessary signs and the extra shift gives us the appropriate degrees. In addition, this definition allows the extra structure to arise naturally instead of having to define the signs by hand. For instance, if we have an associative multiplication $m_2\in\End_A(2)=\Hom(A^{\otimes 2},A)$, the element $m_2$ would not satisfy the equation $m_2\circ m_2=0$ and thus cannot be used to induce a multiplication on $\End_A$ as we did above.
\end{remark}

 A natural question to ask is what relation there is between the $A_\infty$-algebra structure on $S\s\OO$ and the one on $S\s\End_{S\s\OO}$. In \cite{GV} it is claimed that given an operad $\OO$ with an $A_\infty$-multiplication, the map

%I'M WRITING THIS BRACE WITH BAR BECAUSE WITHOUT BAR BECAUSE I WILL HAVE TO USE $B$ FOR THE BRACE IN THE ENDORMORPHISM OPERAD (NON OPERADIC-SUSPENDED). A POSSIBILITY TO BE CONSISTENT IS USING THE LETTER B FOR NON-SUSPENDED OPERADS AND BAR B FOR SUSPENDED OPERADS, INTRODUCING THE BAR WHEN IT IS THE ENDOMORPHISM OF ANOTHER OPERAD
\begin{align*}
&\OO \to \End_\OO\\
&x\mapsto \sum_{n\geq 0}b_n(x;-)
\end{align*}
is a morphism of $A_\infty$-algebras. In the associative case, this result leads to the definition of homotopy $G$-algebras, which connects with the classical Deligne conjecture. We are going to adapt the statement of this claim to our context and prove it. This way we will obtain an $A_\infty$ version of homotopy $G$-algebras and consequently an $A_\infty$ version of the Deligne conjecture. Let $\Phi'$ the map defined as above but on $\s\OO$, i.e.
\begin{align*}
\Phi'\colon&\s\OO \to \End_{\s\OO}\\
&x\mapsto \sum_{n\geq 0}b_n(x;-).
\end{align*}
Let $\Phi:S\s\OO\to S\s\End_{S\s\OO}$ the map making the following diagram commute
\begin{equation}\label{Phi}
\begin{tikzcd}
S\s\OO\arrow[rr, "\Phi"]\arrow[d] & & S\s\End_{S\s\OO}\\
\s\OO\arrow[r, "\Phi'"]& \End_{\s\OO}\arrow[r, "\cong"]& \s\End_{S\s\OO}\arrow[u]
\end{tikzcd}
\end{equation}
where the isomorphism $\End_{\s\OO}\cong\s\End_{S\s\OO}$ is given in \Cref{barsigma}. Note that the degree of the map $\Phi$ is zero.

\begin{remark}
Notice that we have only used the operadic structure on $\s\OO$ to define an $A_\infty$-algebra structure on $S\s\OO$, so the constructions and results in these sections are valid if we replace $\s\OO$ by any graded module $A$ such that $SA$ is an $A_\infty$-algebra. 
\end{remark}
\pagebreak
\begin{thm}\label{theorem}
The map $\Phi$ defined in diagram (\ref{Phi}) above is a morphism of $A_\infty$-algebras, i.e. for all $j\geq 1$ the equation

\[\Phi(M_j)=\overline{M}_j(\Phi^{\otimes j})\]
holds, where the $M_j$ is the $j$-th component of the $A_\infty$-algebra structure on $S\s\OO$ and $\overline{M}_j$ is the $j$-th componnent of the $A_\infty$-algebra structure on $S\s\End_{S\s\OO}$. 
\end{thm}
\begin{proof}
Let us have a look at the following diagram

\begin{equation}\label{proofdiagram}
\begin{tikzcd}[column sep = 1.2em]
(S\s\OO)^{\otimes j}\arrow[dr,red] \arrow[ddd, bend right=10,"M_j"']\arrow[rrrr,bend left=10, "\Phi^{\otimes j}"]& & & & (S\s\End_{S\s\OO})^{\otimes j}\arrow[ddd, bend left=10, "\overline{M}_j"]\\
&\s\OO^{\otimes j}\arrow[r,blue, "(\Phi')^{\otimes j}"]\arrow[d, blue, "M'_j"] & (\End_{\s\OO})^{\otimes j}\arrow[r, blue,"\overline{\sigma}^{\otimes j}"] \arrow[d, dashed, "\mathcal{M}_j",blue]& (\s\End_{S\s\OO})^{\otimes j}\arrow[ur,red]\arrow[d, "\overline{M}'_j",blue]& \\
&\s\OO\arrow[r, blue, "\Phi'"]& \End_{\s\OO} \arrow[r, blue, "\overline{\sigma}"] & \s\End_{S\s\OO}\arrow[dr,red]& \\
S\s\OO\arrow[rrrr, bend right=10, "\Phi"']\arrow[ur,red]& & & & S\s\End_{S\s\OO}
\end{tikzcd}
\end{equation}

where the diagonal red arrows are shifts of graded $R$-modules. We need to show that the diagram defined by the external black arrows commutes. But these arrows are defined so that they commute whith the red and blue arrows, so it is enough to show that the inner blue diagram commutes, since the outer squares commute by definition. The blue diagram can be split into two different squares using the dashed arrow $\mathcal{M}_j$ that we are going to define next, so it will be enough to show that the two squares commute. The commutativity of the  left square will be more involved as we will have to distinguish between different kinds of insertions.

 The map 
\[\mathcal{M}_j:(\End_{\s\OO})^{\otimes j}\to\End_{\s\OO}\]
is defined by 
\begin{align*}
&\mathcal{M}_j(f_1, \dots, f_j)=B_j(M';f_1,\dots, f_j) &\text{ for }j>1,\\
&\mathcal{M}_1(f)=B_1(M';f)-(-1)^{|f|}B_1(f;M'),
\end{align*}
 where $B_j$ is the natural brace structure map on the operad $\End_{\s\OO}$, i.e. for $f\in\End_{\s\OO}(n)$, 
\[B_j(f;f_1,\dots, f_j)=\sum_{k_0+\cdots+k_j=n-j} f(1^{\otimes k_0}\otimes f_1\otimes 1^{\otimes k_1}\otimes\cdots\otimes f_j\otimes 1^{\otimes k_j}).\]
 The $1$'s in the brace structure are identity maps. In the above definition, $|f|$ denotes the degree of $f$ as an element of $\End_{\s\OO}$, which is the same as the degree $\overline{\sigma}(f)\in \s\End_{S\s\OO}$ because $\overline{\sigma}$ is an isomorphism, as mentioned in \Cref{barsigma}.  %the degree as a map sO^n\to sO, which is computed by evaluating and computing arity +degree-1
 
 The inner square  of diagram (\ref{proofdiagram}) is divided into two halves, so we divide the proof into two as well, showing the commutativity of each half independently.
 \subsection*{\centering{Commutativity of the right blue square} }
 Let us show now that the right square commutes. Recall that $\overline{\sigma}$ is an isomorphism of operads and $M=\overline{\sigma}(M')$. Then we have for $j>1$
 
 \begin{align*}
 \overline{M}'_j(\overline{\sigma}(f_1),\dots,\overline{\sigma}(f_j))&=\overline{B}_j(M;\overline{\sigma}(f_1),\dots,\overline{\sigma}(f_j))\\
 &=\overline{B}_j(\overline{\sigma}(M');\overline{\sigma}(f_1),\dots,\overline{\sigma}(f_j)).
 \end{align*}
 Now, since the brace structure is defined as an operadic composition, it commutes with $\overline{\sigma}$, so
 
 \begin{align*}
 \overline{B}_j(\overline{\sigma}(M');\overline{\sigma}(f_1),\dots,\overline{\sigma}(f_j))&=\overline{\sigma}(B_j(M';f_1,\dots, f_j))\\
&=\overline{\sigma}(\mathcal{M}_j(f_1,\dots, f_j))
  \end{align*}
 and therefore the right blue square commutes for $j>1$. For $j=1$ the result follows analogously taking into account that the degree of $f$ in $\End_{\s\OO}$ is the same as the degree of $\overline{\sigma}(f)$ in $\s\End_{S\s\OO}$.\\
 


The proof that the left blue square commutes consists of several lenghty calculations so we are going to devote the next section to that. However, it is worth noting that the commutativity of the left square does not depend on the particular operad $\s\OO$, so it is still valid if $m$ satisfies $m\circ m=0$ for any circle operation defined in terms of insertions. This is essentialy the original statement in \cite{GV}.
\subsection*{\centering{Commutativity of the left blue square}}
We are going to show here that the left blue square in diagram (\ref{proofdiagram}) commutes, i.e. that 

\begin{equation}\label{commutative}
\Phi'(M'_j)=\mathcal{M}_j((\Phi')^{\otimes j})
\end{equation}

for all $j\geq 1$. First we prove the case $j>1$. Let $x_1,\dots, x_j\in \s\OO^{\otimes j}$. We have on the one hand



\begin{align*}
\Phi'(M'_j(x_1,\dots, x_j))& = \Phi'(b_j(m;x_1,\dots, x_j))=\sum_{n\geq 0} b_n(b_j(m;x_1,\dots, x_j);-)\\
&=\sum_n\sum_l\sum b_l(m; -, b_{i_1}(x_1;-),\cdots,b_{i_j}(x_j;-),-)
\end{align*}
where $l=n-(i_1+\cdots+i_j)+j$. The sum with no subindex runs over all the possible order-preserving insertions. Note that $l\geq j$. Evaluating the above map on elements would yield Koszul signs coming from the brace relation. Also recall from \Cref{lemmadegree} that $|b_j(x;-)|=|x|$. Now, fix some value of $l\geq j$ and let us compute the $M'_l$ component of

\begin{align*}
\mathcal{M}_j(\Phi'(x_1),\dots, \Phi'(x_j))=B_j(M';\Phi'(x_1),\dots, \Phi'(x_j))
\end{align*}

that is, $B_j(M'_l;\Phi'(x_1),\dots, \Phi'(x_j))$. By definition, this equals

\begin{align*}
\sum M'_l(-,\Phi'(x_1),\cdots, \Phi'(x_j),-)=&\\
\sum_{i_1,\dots, i_j}\sum M'_l(-,b_{i_1}(x_1;-),\cdots,b_{i_j}(x_j;-),-)=\\
\sum_{i_1,\dots, i_j}\sum b_l(m;-,b_{i_1}(x_1;-),\cdots,b_{i_j}(x_j;-),-).
\end{align*}

We are using hyphens instead of $1$'s to make the equality of both sides of the equation (\ref{commutative}) more apparent, and to make clear that when evaluating on elements those are the places where the elements go. %In this case, evaluating yields the same signs as in the other side of the equation. 

For each tuple $(i_1,\dots, i_j)$ we can choose $n$ such that \[n-(i_1+\cdots+i_j)+j=l,\] so the above sum equals

\[\underset{n-(i_1+\cdots+i_j)+j=l}{\sum_{n,i_1,\dots, i_j}}\sum b_l(m;-,b_{i_1}(x_1;-),\cdots,b_{i_j}(x_j;-),-).\]

So each $M'_l$ component for $l\geq j$ produces precisely the terms $b_l(m;\dots)$ appearing in $\Phi'(M'_j)$. Conversely, for every $n\geq 0$ there exists some tuple $(i_1,\dots, i_j)$ and some $l\geq j$ such that $n$ is the that $n-(i_1+\cdots+i_j)+j=l$, so we do get all the summands from the left hand side of the equation (\ref{commutative}), and thus we have the equality $\Phi'(M'_j)=\mathcal{M}_j((\Phi')^{\otimes j})$ for all $j>1$.

It is worth treating the case $n=0$ separately since in that case we have the summand \[b_0(b_j(m;x_1,\dots, x_j))\] 
in $\Phi'(b_j(m;x_1,\dots, x_j))$, where we cannot apply the brace relation. This summand is equal to 
\begin{align*}
B_j(M'_j;b_0(x_1),\dots, b_0(x_j))&=M'_j(b_0(x_1),\dots, b_0(x_j))\\
&=b_j(m;b_0(x_1),\dots, b_0(x_j)), 
\end{align*}
since by definition $b_0(x)=x$.% We obtained this map from $\overline{M}_1(\Phi(x))$. To see that the two maps are actually equal, apply them to $1\in k$ to output $b_1(m;x)$ in both cases. %Notice that the terms that $b_1(M_i;x)$ produces for $i>1$ appear using the brace relation in $b_k(b_1(m;x);-)$ when $k>0$, more precisely, in the summand $b_k(b_1(m_i;x);-)$. 

Now we are going to show the case $j=1$, that is

\begin{equation}\label{case1}
\Phi'(M'_1(x))=\mathcal{M}_1(\Phi'(x)).
\end{equation} This going to be divided into two parts, since $M'_1$ has two clearly distinct summands, one of them consisting of braces of the form $b_l(m;\cdots)$ (insertions in $m$) and another one consisting of braces of the form $b_l(x;\cdots)$ (insertions in $x$). We will therefore show that both types of braces cancel on each side of \Cref{case1}.

\subsubsection*{Insertions in $m$}

Let us first focus on the insertions in $m$ that appear in equation (\ref{case1}). Recall that 

\begin{equation}\label{phim}
\Phi'(M'_1(x))=\Phi'([m,x])=\Phi'(b_1(m;x))-(-1)^{|x|}\Phi'(b_1(x;m))
\end{equation}

so we focus on the first summand

\begin{align*}
\Phi'(b_1(m;x))&=\sum_n b_n(b_1(m;x);-)\\
&=\sum_n \underset{n\geq i}{\sum_i} \sum b_{n-i+1}(m;-, b_i(x;-),-)\\
&=\underset{n-i+1> 0}{\sum_{n,i}}\sum b_{n-i+1}(m;-, b_i(x;-),-)
\end{align*}

where the sum with no indices runs over all the positions in which $b_i(x;-)$ can be inserted (from $1$ to $n-i+1$ in this case). 


On the other hand, since $|\Phi'(x)|=|x|$, the right hand side of equation (\ref{case1}) becomes

\begin{equation}\label{mphi}
\mathcal{M}_1(\Phi'(x))=B_1(M';\Phi'(x))-(-1)^{|x|}B_1(\Phi'(x);M').
\end{equation}

Again, we are focusing now on the first summand, but with the exception of the part of $M_1$ that corresponds to $b_1(\Phi(x);m)$. From here the argument is a particular case of the proof for $j>1$, so the terms of the form $b_l(m;\cdots)$ are the same on both sides of the equation (\ref{case1}). 




\subsubsection*{Insertions in $x$}

And now, let us study the insertions in $x$ that appear in equation (\ref{case1}). For that we will check that insertions in $x$ from the left hand side and right hand side cancel. Let us look first at the left hand side. From $\Phi'(M'_1(x))$ in equation (\ref{phim}) we had 

\[-(-1)^{|x|}\Phi'(b_1(x;m))=-(-1)^{|x|}\sum_n b_n(b_1(x;m);-).\]

The factor $(-1)^{|x|}$ is going to appear everywhere, so we may cancel it. Then we just have

\[\Phi'(b_1(x;m))=\sum_n b_n(b_1(x;m);-).\]
We are going to evaluate each term of the sum, so let $z_1,\dots, z_n\in \s\OO$. We have by the brace relation that

\begin{align}\label{insertionx1}
b_n(b_1(x;m);z_1,\dots, z_n)&=\\
 \sum_{l+j=n+1}\sum_{i=1}^{n-j+1}&(-1)^{\varepsilon} b_l(x;z_1,\dots,b_j(m;z_{i},\dots, z_{i+j}),\dots, z_n)\nonumber\\
 &+\sum_{i=1}^{n+1}(-1)^{\varepsilon}b_{n+1}(x;z_1,\dots, z_{i-1},m,z_i,\dots, z_n),\nonumber
\end{align}

where $\varepsilon$ is the usual Koszul sign with respect to the grading in $\s\OO$. We have to check that the insertions in $x$ that appear in $\mathcal{M}_1(\Phi'(x))$ (right hand side of the \cref{case1}) are exactly those in the equation (\ref{insertionx1}) above (left hand side of equation \cref{case1}).

Therefore let us look at the right hand side of equation (\ref{case1}). Here we will study the cancellations from each of the two summands that naturally appear. From equation (\ref{mphi}), i.e. \[\mathcal{M}_1(\Phi'(x))=B_1(M';\Phi'(x))-(-1)^{|x|}B_1(\Phi'(x);M')\]  we have 
\[-(-1)^{|x|}b_1(\Phi'(x);m)=-(-1)^{|x|}\sum_n b_1(b_n(x;-);m)\] 
coming from the first summand since $B_1(M'_1;\Phi'(x))=M'_1(\Phi'(x))$. We are now only interested in insertions in $x$. Again, cancelling $-(-1)^{|x|}$ we get
\[b_1(\Phi'(x);m)=\sum_n b_1(b_n(x;-);m).\] 
Each term of the sum can be evaluated on $(z_1,\dots, z_n)$ to produce

\begin{align}\label{insertionx2}
b_1(b_n(x;z_1, \dots, z_n);m)&=\\
\sum_{i=1}^n (-1)^{\varepsilon+|z_i|}&b_n(x;z_1,\dots, b_1(z_i;m),\dots, z_n)\\
+\sum_{i=1}^{n+1} (-1)^{\varepsilon}&b_{n+1}(x;z_1,\dots, z_{i-1},m,z_{i},\dots, z_n)\nonumber
\end{align}

Note that we have to apply the Koszul sign rule twice: once at evaluation, and once more to apply the brace relation. Now, from the second summand of $\mathcal{M}_1(\Phi'(x))$ in the right hand side of \cref{mphi}, after cancelling $(-1)^{|x|}$ we obtain 


\begin{align*}
B_1(\Phi'(x);M')=&\sum_l B_1(b_l(x;-);M')\\
=&-(-1)^{|x|}\sum_l\sum b_l(x;-,M',-) \\
=& \sum_l\sum_{j> 1}\sum b_l(x;-,b_j(m;-),-)\\
&+\sum_l\sum b_l(x;-,b_1(-;m),-).
\end{align*}
We are going to evaluate on $(z_1,\dots, z_n)$ to make this map more explicit. This evaluation gives us the following
 
 \begin{align}\label{lhs2}
 \sum_{l+j=n+1}\sum_{i=1}^{n-j+1}(-1)^{\varepsilon} b_l(x;z_1,\dots,b_j(m;z_{i},\dots, z_{i+j}),\dots, z_n)\\\nonumber -\sum_{i=1}^{n} (-1)^{\varepsilon+|z_i|}b_n(x;z_1,\dots,b_1(z_{i};m),\dots, z_n)
 \end{align}
 
 The minus sign comes from the fact that $b_1(z_i;m)$ comes from $M'_1(z_i)$, so we apply the signs in the definition of $M'_1(z_i)$. We therefore have that the right hand side of \cref{mphi} is the result of adding equations (\ref{insertionx2}) and (\ref{lhs2}). After this addition we can see that the first sum of \cref{insertionx2} cancels the second sum of \cref{lhs2}. 

 We also have that the second sum in \cref{insertionx2} is the same as the second sum in \cref{insertionx1}, so we are left with only the first sum of \cref{lhs2}. This is the same as the first sum in \cref{insertionx1}, so we have already checked that the equation $\Phi'(M'_1)=\mathcal{M}_1(\Phi')$ holds. 
  
 In the case $n=0$, we have to note that $B_1(b_0(x);m)$ vanishes because of arity reasons: $b_0(x)$ is a map of arity 0, so we cannot insert any inputs. And this finishes the proof.
 \end{proof}
 
 \subsection{Explicit $A_\infty$-algebra structure and Deligne Conjecture}\label{sect4}


We have given an implicit definition of the components of the $A_\infty$-algebra structure on $S\s\OO$, namely, \[M_j=\overline{\sigma}(M'_j)=(-1)^{\binom{j}{2}}S\circ M'_j\circ(S^{-1})^{\otimes j},\]
but it is useful to have an explicit expression that determines how it is evaluated on elements of $S\s\OO$. We will also need these expressions to state the $A_\infty$-version of the Deligne conjecture in a precise way. Recall that the classical Deligne conjecture \cite{GV} states that the Hochschild complex of an associative algebra has a structure of homotopy $G$-algebra. Here, we will define $J$-algebras as the $A_\infty$-generalization of homotopy $G$-algebras. We will do this in terms of the explicit expressions we give for the maps $M_j$. This explicit formulas will also clear up the connection with the work of Gerstenhaber and Voronov. We also hope that these formulas can be useful to perform calculations in other mathematical contexts where $A_\infty$-algebras are used.


\begin{lem}\label{explicit}
For $x,x_1,\dots,x_n\in\s\OO$, we have the following expressions.

\begin{align*}
&M_n(Sx_1,\dots, Sx_n)=(-1)^{\sum_{i=1}^n(n-i)|x_i|}Sb_n(m;x_1,\dots, x_n) & & n>1,\\
&M_1(Sx)=Sb_1(m;x)-(-1)^{|x|}Sb_1(x;m).
\end{align*}

Here $|x|$ is the degree of $x$ as an element of $\s\OO$, i.e. its natural degree. 
\end{lem}
\begin{proof}
The deduction of these explicit formulas is done as follows. Let $n>1$ and $x_1,\dots, x_n\in \s\OO$. Then

\begin{align}\label{aboveexpression}
M_n(Sx_1,\dots, Sx_n)&=SM'_n((S^{\otimes n})^{-1})(Sx_1,\dots, Sx_n)\nonumber\\
&=(-1)^{\binom{n}{2}}SM'_n((S^{-1})^{\otimes n})(Sx_1,\dots, Sx_n)\nonumber\\
&=(-1)^{\binom{n}{2}+\sum_{i=1}^n(n-i)(|x_i|+1)}SM'_n(x_1,\dots,x_n)
\end{align}

Now, note that $\binom{n}{2}$ is even exactly when $n\equiv 0,1\mod 4$. In these cases, an even amount of $|x_i|$'s have an odd coefficient in the sum (when $n\equiv 0\mod 4$ these are the $|x_i|$ with even index, and when $n\equiv 1\mod 4$, the $|x_i|$ with odd index). This means that 1 is added on the exponent an even number of times, so the sign is not changed by the binomial coefficient nor by adding 1 on each term. Similarly, when $\binom{n}{2}$ is odd, i.e. when $n\equiv 2,3\mod 4$, there is an odd number of $|x_i|$ with odd coefficient, so the addition of 1 an odd number of times cancels the binomial coefficient. This means that \Cref{aboveexpression} can be simplified to

\[M_n(Sx_1,\dots, Sx_n)=(-1)^{\sum_{i=1}^n(n-i)|x_i|}SM'_n(x_1,\dots,x_n),\]
which by definition equals
\[(-1)^{\sum_{i=1}^n(n-i)|x_i|}Sb_n(m;x_1,\dots,x_n).\]

The case $n=1$ is analogous, one just has to note that 

\[
M'_1(x)=b_1(m;x)-(-1)^{|x|}b_1(x;m)
\]
and that $\overline{\sigma}$ is linear. 
\end{proof}

It is possible to show that the maps defined explicitly as we have just done satisfy the $A_\infty$-equation without relying on the fact that $\overline{\sigma}$ is a map of operads, but it is a lengthy and tedious calculation.

\begin{remark}
In the case $n=2$, omitting the shift symbols by abuse of notation, we obtain 

\[M_2(x,y)=(-1)^{|x|}b_2(m;x,y).\]
Let $M^{GV}_2$ be the product defined in \cite{GV} as \[M^{GV}_2(x,y)=(-1)^{|x|+1}b_2(m;x,y).\] We see that $M_2=-M^{GV}_2$. Since the authors of \cite{GV} work in the associative case $m=m_2$, this minus sign does not affect the $A_\infty$-relation (which in this case reduces to the associativity and differential relations). This difference in sign can be explained by the difference between $(S^{\otimes n})^{-1}$ and $(S^{-1})^{\otimes n}$, since any of these shift maps can be used to define a map $(S\s\OO)^{\otimes n}\to \s\OO^{\otimes n}$. 
\end{remark}

Now that we have the explicit formulas for the $A_\infty$-structure on $S\s\OO$ we can state and prove an $A_\infty$-version of the Deligne conjecture. Let us first re-adapt the definition of homotopy $G$-algebra from \cite[Definition 2]{GV} to our conventions.

\begin{defin}\label{homotopygalgebras}
A \emph{homotopy $G$-algebra} is differential graded algebra $V$ with a differential $M_1$ and a product $M_2$ such that the shift $S^{-1}V$ is a brace algebra with brace maps $b_n$. The product $M_2$ must satisfy the following compatibility identities. Let $x,x_1,x_2,y_1,\dots, y_n\in S^{-1}V$. We demand 
\begin{align*}
Sb_n(&S^{-1}M_2(Sx_1,Sx_2);y_1,\dots, y_n) = \\
&\sum_{k=0}^n (-1)^{(|x_2|+1)\sum_{i=1}^k|y_i|}M_2(b_k(x_1;y_1,\dots, y_k),b_{n-k}(x_2;y_{k+1},\dots, y_n))
\end{align*}
%I'm getting a +nk on the sign, but let's pretend it's not there for now and later see how the general formulas behave with respect to this
and
\begin{align*}
S&b_n(S^{-1}M_1(Sx);y_1,\dots, y_n)-M_1(Sb_n(x;y_1,\dots,y_n))\\
-&(-1)^{|x|+1}\sum_{p=1}^n(-1)^{\sum_{i=1}^p|y_i|}Sb_n(x;y_1,\dots,M_1(Sy_p),\dots, y_n)\\
=&-(-1)^{(|x|+1)|y_1|}M_2(Sy_1,Sb_{n-1}(x;y_2,\dots, y_n))\\
 &+(-1)^{|x|+1}\sum_{p=1}^{n-1}(-1)^{n-1+\sum_{i=1}^p|y_i|}Sb_{n-1}(x;y_1,\dots,M_2(Sy_p,Sy_{p+1}),\dots y_n)\\
 &-(-1)^{|x|+\sum_{i=1}^{n-1}|y_i|}M_2(Sb_{n-1}(x;y_1,\dots, y_{n-1}),Sy_n)
\end{align*}
\end{defin}

Notice that our signs are slightly different to those in \cite{GV} as a consequence of our conventions. Our signs will be a particular case of those in \Cref{Jalgebras}, which are set so that \Cref{ainftydeligne} holds in consistent way with operadic suspension and all the shifts that tha authors of \cite{GV} do not consider.

We now introduce $J$-algebras as an $A_\infty$-generalization of homotopy $G$-algebras. This will allow us to generalize the Deligne conjecture to the $A_\infty$-setting. %We introduce now the notation $||x||=|x|+1$ in order to make the equations more compact.

\begin{defin}\label{Jalgebras}
A \emph{$J$-algebra} $V$ is an $A_\infty$-algebra with structure maps $\{M_j\}_{j\geq 1}$ such that the shift $S^{-1}V$ is a brace algebra. Furthermore, the braces and the $A_\infty$-structure satisfy the following compatibility relations. Let $x, x_1,\dots, x_j, y_1,\dots, y_n\in S^{-1}V$. For $n\geq 0$ we demand 
\begin{align*}
(-1)^{\sum_{i=1}^n(n-i)|y_i|}Sb_n(S^{-1}&M_1(Sx);y_1,\dots, y_n)=\\
&\underset{\mathclap{1\leq i_1\leq n-k+1}}{\sum_{\mathclap{l+k-1=n}}}(-1)^{\varepsilon}M_l(Sy_1,\dots, Sb_{k}(x;y_{i_1},\dots),\dots, Sy_n)\\
-(-1)^{|x|}\underset{\mathclap{1\leq i_1\leq n-k+1}}{\sum_{\mathclap{l+k-1=n}}}&(-1)^{\eta} Sb_k(x;y_1,\dots, S^{-1}M_l(Sy_{i_1},\dots,), \dots, y_n)
\end{align*}
where
%\[
%\varepsilon = \sum_{v=1}^{i_1-1}(|y_v|+1)(|x|-k+1)+\sum_{v=1}^{k}(|y_{v+i_1-1}|+1)(k-v)+(l-1)|x|+(i_1-1)k.
%\]

%\begin{align*}
%\varepsilon = &\sum_{v=1}^{i_1-1}||y_v||(||x||-k)+\sum_{v=1}^{k}(||y_{i_1+v-1}||-1)(k-v)\\
%&+(l-1)(||x||-1)+(i_1-1)(k-1).
%\end{align*}
%
%\begin{align*}
%\varepsilon = &\sum_{v=1}^{i_1-1}(|y_v|+1)(|x|-k+1)+\sum_{v=1}^{k}|y_{i_1+v-1}|(k-v)\\
%&+(l-1)|x|+(i_1-1)(k-1).
%\end{align*}
%CALCULATIONS
%\begin{align*}
%\varepsilon = &\sum_{v=1}^{i_1-1}(|y_v|+1)|x|+\sum_{v=1}^{i_1-1}|y_v|(k-1)+\sum_{v=1}^{k}|y_{i_1+v-1}|(k-v)+(l-1)|x|.
%\end{align*}
%\begin{align*}
%\varepsilon = &\sum_{v=1}^{i_1-1}|y_v||x|+\sum_{v=1}^{i_1-1}|x|+\sum_{v=1}^{i_1-1}|y_v|(k-1)+\sum_{v=1}^{k}|y_{i_1+v-1}|(k-v)+(l-1)|x|.
%\end{align*}
\begin{align*}
\varepsilon = &\sum_{v=1}^{i_1-1}|y_v|(|x|-k+1)+\sum_{v=1}^{k}|y_{i_1+v-1}|(k-v)+(l-i_1)|x|.
\end{align*}
and
\begin{align*}
\eta=& \sum_{v=1}^{i_1-1}(k-v)|y_v|+l\sum_{v=1}^{i_1-1}|y_v|+\sum_{v=i_1}^{i_1+l-1}(k-i_1)|y_v|+\sum_{v=i_1}^{n-l}(k-v)|y_{v+l}|
\end{align*}

%\[1+2+\cdots \]
%\[
%\eta=l(k-1)+\sum_{v=1}^{i_1-1}(n+1-v)(|y_v|+1)+\sum_{v=i_1}^{i_1+l-1}(k-i_1)(|y_v|+1)+\sum_{v=i_1+l}^n(n-v)|(|y_v|+1).
%\]

For $j>1$ we demand
\begin{align*}
&(-1)^{\sum_{i=1}^n(n-i)|y_i|}Sb_n(S^{-1}M_j(Sx_1,\dots, Sx_j);y_1,\dots, y_n)=\\
&\sum(-1)^{\varepsilon}M_l(Sy_1,\dots, Sb_{k_1}(x_1;y_{i_1},\dots),\dots, Sb_{k_j}(x_j;y_{i_j},\dots),\dots, Sy_n).
\end{align*}
The unindexed sum runs over all possible choices of non-negative integers that satisfy $l+k_1+\cdots+k_j-j=n$ and over all possible ordering-preserving insertions. The right hand side sign is given by

%\begin{align*}
%\varepsilon =& \underset{1\leq v\leq k_t}{\sum_{\mathclap{1\leq t\leq j}}} (||y_{i_t+v-1}||-1)(k_t-v)+\sum_{\mathclap{1\leq v< l\leq j}}k_v(||x_l||-1)\\
%&+\underset{\mathclap{i_t\leq v< i_{t+1}}}{\sum_{\mathclap{0\leq t\leq l\leq j}}}||y_v||(||x_l||-k_l)+\sum_{\mathclap{0\leq v<l\leq j}}(i_{v+1}-i_v-k_v)(||x_l||-k_l)\\&+\sum_{\mathclap{1\leq v\leq l\leq j}} (||x_v||-1)(i_{l+1}-i_l-k_l)
%\end{align*}
\begin{align*}
\varepsilon =& \underset{1\leq v\leq k_t}{\sum_{\mathclap{1\leq t\leq j}}} |y_{i_t+v-1}|(k_t-v)+\sum_{\mathclap{1\leq v< l\leq j}}k_v|x_l|+\sum_{\mathclap{1\leq v\leq l\leq j}} |x_v|(i_{l+1}-i_l-k_l)\\
&+\underset{\mathclap{i_t\leq v< i_{t+1}}}{\sum_{\mathclap{0\leq t< l\leq j}}}(|y_v|+1)(|x_l|-k_l+1)+\sum_{\mathclap{0\leq v<l\leq j}}(i_{v+1}-i_v-k_v)(|x_l|-k_l+1)\
\end{align*}
In the sums we are setting $i_0=0$ and $i_{j+1}=n+1$. %not sure if it is worth using double bars
%\underset{\mathclap{i_{t-1}+k_{t-1}\leq v< i_t}}{\sum_{\mathclap{1\leq t\leq l\leq j}}}||y_v||(||x_l||-k_l)
%\underset{\mathclap{i_t\leq v< i_t+k_t}}{\sum_{\mathclap{1\leq t< l\leq j}}}||y_v||(||x_l||-k_l)


%\begin{align*}
%\varepsilon = \sum_{\mathclap{1\leq i< l\leq j}}k_v|x_l|+\underset{\mathclap{i_{t-1}+k_{t-1}\leq v< i_t}}{\sum_{\mathclap{1\leq t\leq l\leq j}}}(|y_v|+1)(|x_l|-k_l+1)+\underset{\mathclap{0\leq v< k_t}}{\sum_{\mathclap{1\leq t\leq j}}} (|y_{v+i_t}|+1)(k_v-v+1)\\
%\underset{\mathclap{i_t\leq v< i_t+k_t}}{\sum_{\mathclap{1\leq t< l\leq j}}}(|y_v|+1)(|x_l|-k_l+1)+\sum_{\mathclap{0\leq v<l\leq j}}(i_{v+1}-i_v-k_v)(|x_l|-k_l+1)\\+\sum_{\mathclap{1\leq v\leq l\leq j}} |x_v|(i_{l+1}-i_l-k_l)
%\end{align*}

%\[
%\varepsilon = \sum_{1\leq i< l\leq j}k_v|x_l|+\underset{\mathclap{i_{t-1}+k_{t-1}\leq v< i_t}}{\sum_{1\leq t\leq l\leq j}}(|y_v|+1)(|x_l|-k_l+1)+\underset{\mathclap{i_t\leq v< i_t+k_t}}{\sum_{1\leq t\leq j}} (|y_v|+1)(k_v-v)
%\]
%\[
%\varepsilon = \sum_{\mathclap{1\leq i< l\leq j}}k_v|x_l|+\underset{\mathclap{i_{t-1}+k_{t-1}\leq v< i_t}}{\sum_{\mathclap{1\leq t\leq l\leq j}}}(|y_v|+1)(|x_l|-k_l+1)+\underset{\mathclap{i_t\leq v< i_t+k_t}}{\sum_{\mathclap{1\leq t\leq j}}} (|y_v|+1)(k_v-v+i_t+1)
%\]
%\[
%\varepsilon = \sum_{\mathclap{1\leq i< l\leq j}}k_v|x_l|+\underset{\mathclap{1\leq v< i_t}}{\sum_{1\leq t\leq l\leq j}}(|y_v|+1)(|x_l|-k_l+1)+\underset{\mathclap{0\leq v< k_t}}{\sum_{\mathclap{1\leq t\leq j}}} (|y_{v+i_t}|+1)(k_v-v+1)
%\]


\end{defin}


\begin{corollary}[The $A_\infty$-Deligne Conjecture]\label{ainftydeligne}
If $A$ is an $A_\infty$-algebra, then its Hochschild complex $S\s\End_A$ is a $J$-algebra.
\end{corollary}
\begin{proof}
 Clearly, $\s\End_A$ is a brace algebra as it is an operad. Since $A$ is an $A_\infty$-algebra, the structure map $m=m_1+m_2+\cdots$ determines an $A_\infty$-multiplication $m\in\s\End_A$. It follows by \Cref{ainftystructure} that $S\s\End_A$ is an $A_\infty$-algebra. Therefore, we need to show the compatibility relations. The result follows by direct computation from \Cref{theorem}, expanding the definitions and taking into account the sign rules described in \Cref{koszulsigns}. Let us do this in detail. 
 
Recall that \Cref{theorem} states that $\Phi\circ M_j = \overline{M}_j\circ \Phi^{\otimes j}$. We start with the case $j>1$. Let $Sx_1,\dots, Sx_j\in S\s\End_A$. Recall \Cref{Phi} for the definition of $\Phi$. On the left hand side we have by definition
\begin{align*}
\Phi(M_j(Sx_1,\dots, Sx_j)) = S\overline{\sigma}\Phi'(S^{-1}M_j(Sx_1,\dots, Sx_j)).
\end{align*}
Notice that this map belongs to $S\s\End_{S\s\OO}$, where $\OO=\End_A$, so let us consider just its arity $n$ component. We are going to omit the external shift and consider the equation on $\s\End_{S\s\OO}$ since this extra shift will cancel.
\begin{align*}
\overline{\sigma}b_n(S^{-1}M_j(Sx_1,\dots, Sx_j);-)=(-1)^{\binom{n}{2}}Sb_n(S^{-1}M_j(Sx_1,\dots, Sx_j);S^{-n}).
\end{align*}
Now evaluate this on $Sy_1,\dots, Sy_n\in S\s\End_A$. Using the same calculation as in the proof of \Cref{explicit} we get
\begin{align}
(-1)^{\sum_{i=1}^n(n-i)|y_i|}Sb_n(S^{-1}M_j(Sx_1,\dots, Sx_j);y_1,\dots, y_n).
\end{align}
This is already the left hand side in \Cref{Jalgebras}. Let us now have a look at the right hand side of \Cref{Phi}. We evaluate again on $Sx_1,\dots, Sx_j$ to obtain
\begin{align*}
\overline{M}_j(\Phi^{\otimes j})(Sx_1,\dots, Sx_j) & = \overline{M}_j(\Phi(Sx_1),\dots, \Phi(Sx_j))\\
&=\sum_{k_1,\dots, k_j}\overline{M}_j(S\overline{\sigma}\Phi'(x_1),\dots, S\overline{\sigma}\Phi'(x_j)).
\end{align*}
Expanding, this equals
\begin{equation}\label{intermediate}
\sum_{k_1,\dots, k_j}\overline{M}_j(S(-1)^{\binom{k_1}{2}}Sb_{k_1}(x_1;S^{-k_1}),\dots,S(-1)^{\binom{k_j}{2}}Sb_{k_j}(x_j;S^{-k_j}) ).
\end{equation}
We now apply the definition of $\overline{M}_j$. Notice that by the isomorphism $\overline{\sigma}$ and \Cref{lemmadegree} we have
\[
|Sb_k(x;S^{-k})|=|\overline{\sigma}(b_k(x;-))|=|b_k(x;-)|=|x|.
\]
Therefore, from \Cref{intermediate} we proceed again similarly as in \Cref{explicit} to get
\[
\sum_{k_1,\dots, k_j}(-1)^{\sum_{v=1}^j\left[\binom{k_v}{2}+(j-v)|x_v|\right]}\overline{B}_j(M;Sb_{k_1}(x_1;S^{-k_1}),\dots,Sb_{k_j}(x_j;S^{-k_j})).
\]
Here we have omitted the extra shift just like on the left hand side. Now we need to use \Cref{bracesign} to turn the above brace into composition of maps. Taking only the arity $n$ component yields
\[
\sum(-1)^{\sum_{v=1}^j\binom{k_v}{2}+\xi}M_l(-,Sb_{k_1}(x_1;S^{-k_1}),\dots,Sb_{k_j}(x_j;S^{-k_j}),-).
\]
where
\begin{align*}
\xi = \underset{1\leq v<l\leq j}{\sum}k_v|x_l|+\sum_{\mathclap{0\leq v<l\leq j}}space_v(|x_l|-k_l+1)+ \sum_{\mathclap{1\leq v\leq l\leq j}} |x_v|space_l.
\end{align*}
The variable $space_v$ represents the space between the $v$-th and the $(v+1)$-th brace. More precisely, if the $v$-th brace is inserted in the position $i_v$, $space_v=i_{v+1}-i_v-k_v$. The unindexed sum runs all possible ordering-preserving insertions and over all possible choices of integers that satisfy $l+k_1+\cdots+ k_j-j=n$. We have also used the fact that $k_v^2\equiv k_v\mod 2$ to simplify the sign. Finally, we evaluate on $Sy_1,\dots, Sy_n$. Here we need to take into account the sign rules explained in \Cref{koszulsigns}. In particular, this means that we use the internal degree of $Sb_k(x;S^{-k})$ which is $|x|-k+1$. This evaluation and some simplification produces the desired sign $(-1)^{\varepsilon}$.

Let us now treat the $j=1$ case, where we have $\Phi\circ M_1=\overline{M}_1\circ\Phi$. The left hand side is analogous to the general case, so we have

\begin{equation}
(-1)^{\sum_{i=1}^n(n-i)|y_i|}Sb_n(S^{-1}M_1(Sx);y_1,\dots, y_n).
\end{equation}
On the right hand side we have
\begin{align*}
\overline{M}_1(\Phi(Sx))&=\sum_{k}\overline{M}_1(S\overline{\sigma}\Phi'(x))\\
&=\sum_{k}\overline{M}_1(S(-1)^{\binom{k}{2}}Sb_{k}(x;S^{-k})).
\end{align*}
Recalling that $|Sb_{k}(x;S^{-k})|=|x|$ and cancelling again the extra shift we may expand the above expression to obtain
\begin{equation}\label{twoterms}
\sum_{k}(-1)^{\binom{k}{2}}\left(\overline{B}_1(M;Sb_{k}(x;S^{-k}))-(-1)^{|x|}\overline{B}_1(Sb_{k}(x;S^{-k});M)\right)
\end{equation}

The first term is analogous to the general case, yielding 
\begin{equation}
\underset{1\leq i_1\leq n-k+1}{\sum_{l+k-1=n}}(-1)^{\varepsilon}M_l(Sy_1,\dots, Sb_{k}(x;y_{i_1},\dots),\dots, Sy_n)
\end{equation}
upon evaluation and cancelling $(-1)^{\binom{k}{2}}$, where 

\[
\varepsilon = \sum_{v=1}^{i_1-1}(|y_v|+1)(|x|-k+1)+\sum_{v=1}^{k}|y_{i_1+v-1}|(k-v)+(l-1)|x|+(i_1-1)(k-1).
\]
Let us now focus on the second term of \Cref{twoterms} and let us omit the sign $(-1)^{\binom{k}{2}+|x|}$ for now. On arity $n$ we have
\begin{align*}
\overline{B}_1(Sb_{k}(x;&S^{-k});M)=\\ &\underset{\mathclap{1\leq i_1\leq n-k+1}}{\sum_{l+k-1=n}}(-1)^{l(k-1)+k-i_1} Sb_k(x;S^{-(i_1-1)}, S^{-1}M_l, S^{-(k - i_1)}).
\end{align*}
The sign is computed using \Cref{bracesign} and the Koszul sign rule for the shifts. Notice that here we need to use the internal degree of $M_l\in\s\End_A$, that is, $2-l$. Finally, evaluating at $Sy_1,\dots, Sy_n$ and combining the resulting signs with the factor $(-1)^{\binom{k}{2}+|x|}$ produces the result.


%\[
%\sum_{v=1}^{i_1-1}(k-v+l)||y_v||+\sum_{v=i_1}^{i_1+l-1}(k-i_1)||y_v||+\sum_{v=i_1}^{n-l}(k-v)||y_{v+l}||
%\]
%\[
%\sum_{v=1}^{i_1-1}(k-v)(|y_v|+1)+\sum_{v=1}^{i_1-1}l||y_v||+\sum_{v=i_1}^{i_1+l-1}(k-i_1)||y_v||+\sum_{v=i_1}^{n-l}(k-v)(|y_{v+l}|+1)
%\]
%%n-ele = k-one, cancellation of binom{k}{2}
%\[
%\sum_{v=1}^{i_1-1}(k-v)|y_v|+\sum_{v=1}^{i_1-1}l||y_v||+\sum_{v=i_1}^{i_1+l-1}(k-i_1)||y_v||+\sum_{v=i_1}^{n-l}(k-v)|y_{v+l}|
%\]
%\[
%\sum_{v=1}^{i_1-1}(k-v)|y_v|+\sum_{v=1}^{i_1-1}|y_v|+l(i_1-1)+\sum_{v=i_1}^{i_1+l-1}(k-i_1)|y_v|+l(k-i_1)+\sum_{v=i_1}^{n-l}(k-v)|y_{v+l}|
%\]
%%cancellation of l(k-1) and I'm going to pretend k-i_1 does not exist, originally I wasn't getting it maybe it's cancelling with something as well
%\[
%\sum_{v=1}^{i_1-1}(k-v)|y_v|+\sum_{v=1}^{i_1-1}l|y_v|+\sum_{v=i_1}^{i_1+l-1}(k-i_1)|y_v|+\sum_{v=i_1}^{n-l}(k-v)|y_{v+l}|
%\]


\end{proof}


\begin{remark}
In \Cref{ainftydeligne}, when $A$ is just an associative algebra, we recover the homotopy $G$-algebra structure on its Hochschild complex according to \Cref{homotopygalgebras}, as described in \cite{GV}.
\end{remark}


%I MIGHT NEED TO DEDUCE THE EXPRESSIONS OF PHI(M1)=M1(PHI) AND SAME WITH M2 TO OBTAIN SOME STRUCTURE ON COHOMOLOGY AS IN GV

%THE LEFT BLUE SQUARES IMPLIES G-V EQUATIONS EXCEPT WITH BRACE INSTEAD OF THE PRODUCT (SO UP TO THAT PRECISE SIGN), WITH THE PRODUCTS OTHER SIGNS APPEAR RELATED TO THE BRACES WHICH ARE NOT EXACTLY PHI
%\appendix
%\renewcommand{\appendixname}{Appendix:}
%\begin{appendices}
%\appendix
%\gdef\thesection{Appendix \Alph{section}}




%\end{appendices}
%\phantomsection

%\bibliographystyle{ieeetr}
%\bibliography{newbibliography}
\end{document}
