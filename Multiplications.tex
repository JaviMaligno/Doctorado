	\documentclass[twoside]{article}
\usepackage{estilo-ejercicios}
\setcounter{section}{0}

\renewcommand{\baselinestretch}{1,3}

\usepackage{empheq}
\newcommand*\widefbox[1]{\fbox{\hspace{2em}#1\hspace{2em}}}
%--------------------------------------------------------
\begin{document}

\title{$A_\infty$-structures on operads}
\author{Javier Aguilar Martín}
\maketitle

\section{Introduction}

In a previous text we defined operadic suspension and used it to define the notion of $A_\infty$-multiplication on an operad. We then defined a brace structure on the operadic suspension of an operad that encapsulated the definition of $A_\infty$-multiplication. We also established the relation between this brace structure and the canonical brace structure on an operad given by operadic composition.

Once one has an $A_\infty$-multiplication on an operad, it is natural to ask whether this multiplication induces an $A_\infty$-algebra structure on the operad, i.e if $\OO$ is an operad with an $A_\infty$-multiplication $m\in\OO$, are there linear maps $M_j:\OO^{\otimes j}\to \OO$ satisfying the $A_\infty$-algebra axioms? In Section \ref{sect2} we use the aforementioned brace structure to define such linear maps on a shifted version of the operadic suspension. We then iterate this process in Section \ref{sect3} to define an $A_\infty$-structure on the Hochschild complex of an operad with $A_\infty$-multiplication. This iteration process was inspired by the work of Getzler in \cite{getzler}.

Finally, we prove our main result, Theorem \ref{theorem}, which relates the $A_\infty$-structure on an operad with the one induced on its Hochschild complex, more precisely, there is a morphism of $A_\infty$-algebras $\Phi:S\s\OO\to S\s\End_{S\s\OO}$. This result was hinted at by Gerstenhaber and Voronov in \cite{GV}, but here we introduce more context and prove the theorem.

Throughout this text we use the same notation that we used in the previous text, but we recall some of the definitions along the exposition.

%Continuing with the operadic suspension and the brace structure obtained from it, we then define an $A_\infty$-algebra structure on $\s\OO$.


\section{$A_\infty$-algebra structures on an operad}\label{sect2}


Let $\OO$ be an operad of graded $R$-modules and $\s\OO$ its operadic suspension. Let us consider the underlying graded module of the operad $\s\OO$, which we  call $\s\OO$ again by abuse of notation, i.e. \[\s\OO=\bigoplus_n \s\OO(n)\] with grading given by its \emph{natural degree}, i.e. if $x\in \s\OO(n)$ recall that its natural degree is \[|x|=n+\deg(x)-1,\] where $\deg(x)$ is its internal degree (the degree as an element of $\OO(n))$. 

For any operad $\OO$, recall the operation $\circ$ defined as

\[
a\circ b=\sum_{i=1}^n a\circ_i b\in\OO(n+m-1)
\]
for $a\in\OO(n)$ and $b\in \OO(m)$. We write $a\tilde{\circ}b$ for the corresponding operation on $\s\OO$, namely

\[
a\tilde{\circ} b=\sum_{i=1}^n a\tilde{\circ}_i b=b_1(a;b)\in\s\OO(n+m-1).
\]



\begin{defin}
Let $m\in\s\OO$ be of natural degree 1 such that $m\tilde{\circ}m=0$, or equivalently $m=m_1+m_2+\cdots$ is a formal sum of maps $m_j\in\OO(j)^{2-j}$ satisfying the usual $A_\infty$-equation for all $n$
\begin{equation}\label{Ainftyeq}
\sum_{r+s+t=n}(-1)^{rs+t}m_{r+1+t}\circ_{r+1}m_s=0.
\end{equation} 
Such $m$ is said to be an \emph{$A_\infty$-multiplication} on $\OO$ and as we saw in the previous text its existence is equivalent to a map of operads $\mathcal{A}_\infty\to \OO$ from the operad $\mathcal{A}_\infty$ of $A_\infty$-algebras to $\OO$. We may call each $m_j$ the $j$-th \emph{component} of $m$.
\end{defin}

\begin{remark}\label{multiplicationalgebra}
An $A_\infty$-multiplication on the operad $\End_A$ is equivalent to an $A_\infty$-algebra structure on $A$.
\end{remark}

Following \cite{GV} and \cite{getzler}, if we have an $A_\infty$ multiplication $m\in\OO$, one would define an $A_\infty$-algebra structure on $\s\OO$ using the maps 

\begin{align*}
M'_1(x)\coloneqq [m,x]=m\tilde{\circ} x-(-1)^{|x|}x\tilde{\circ}m, & &  \\
M'_j(x_1,\dots, x_j)\coloneqq b_j(m;x_1,\dots, x_j),& &j>1.
\end{align*}
The prime notation here is used to indicate that these are not the definitive maps that we are going to take. Getzler shows in \cite{getzler} that $M'=M'_1+M'_2+\cdots$ satisfies the relation $M'\circ M'=0$ using that $m\circ m=0$, and the proof is independent of the operad in which $m$ is defined, so it is still valid if $m\tilde{\circ}m=0$. But we have two problems here. The equation $M'\circ M'=0$ does depend on how the circle operation is defined, more precisely, this circle operation in \cite{getzler} is the natural circle on the endomorphism operad, which does not have any additional signs, so $M'$ is not an $A_\infty$-structure under our convention. The other problem has to do with the degrees. We need $M'_j$ to be homogeneous of degree $2-j$ as a map $\s\OO^{\otimes j}\to \s\OO$, but we find that $M'_j$ is homogeneous of degree 1 instead as the following lemma shows.
\begin{lemma}\label{lemmadegree}
For $x\in\s\OO$ we have that  the degree of $b_j(x;-)$ as a map of graded modules \[b_j(x;-):\s\OO^{\otimes j}\to\s\OO\] is precisely $|x|$.
\end{lemma}
\begin{proof}
Let $a(x)$ denote the arity of $x$, i.e. $a(x)=n$ whenever $x\in\s\OO(n)$. Also, let $\deg(x)$ be its internal degree in $\OO$. The natural degree of $b_j(x;x_1,\dots,x_j)$ for $a(x)\geq j$ is computed as follows. By definition, we have that the natural degree of $b_j(x;x_1,\dots,x_j)$ as an element of $\s\OO$ is

\[|b_j(x;x_1,\dots,x_j)|=a(b_j(x;x_1,\dots,x_j))+\deg(b_j(x;x_1,\dots,x_j))-1.\]

We have 

\[a(b_j(x;x_1,\dots,x_j))=a(x)-j+\sum_i a(x_i)\]

and 

\[\deg(b_j(x;x_1,\dots,x_j)=\deg(x)+\sum_i\deg(x_i),\]

so 
\begin{align*}
a(b_j(x;x_1,\dots,x_j))+\deg(b_j(x;x_1,\dots,x_j))-1=\\
a(x)-j+\sum_i a(x_i)+\deg(x)+\sum_i\deg(x_i)-1=\\
a(x)+\deg(x)-1+\sum_i a(x_i)+\sum_i\deg(x_i)-j=\\
a(x)+\deg(x)-1+\sum_i (a(x_i)+\deg(x_i)-1)=\\
|x|+\sum_i|x_i|.
\end{align*}
This means that the degree of the map $b_j(x;-)$ as a map $\s\OO^{\otimes j}\to \s\OO$ equals $|x|$.

\end{proof} %A first alternative after finding this result is considering $M'_j$ to be an element of $\s\End_{\s\OO}$ instead of just $\End_{\s\OO}$. This solves the problem of the degree, but not the one of the sign convention. 

\begin{corollary}
The maps 
\begin{align*}
M_j':\s\OO^{\otimes j}&\to \s\OO\\
(x_1,\dots, x_j)&\mapsto b_j(m;x_1,\dots, x_j)
\end{align*}
for $j>1$ and the map
\begin{align*}
M_1':\s\OO&\to \s\OO\\
x&\mapsto b_1(m;x)-(-1)^{|x|}b_1(m;x)
\end{align*}
are homogeneous of degree 1. 
\end{corollary}
\begin{proof}
For $j>1$ it is a direct consequence of \Cref{lemmadegree}. For $j=1$ we have the summand $b_1(m;x)$ whose degree follows as well from \Cref{lemmadegree}. The degree of other summand, $b_1(x;m)$, can be computed in a similar way as in the proof \Cref{lemmadegree}, giving that $|b_1(x;m)|=1+|x|$. This concludes the proof.
\end{proof}

The problem we have encountered with the degrees can be resolved using shift maps as the following proposition shows. Recall that the \emph{shift} of a graded module $A$ is given by $SA^i=A^{i-1}$ and that we have maps $A\to SA$ of degree 1 given by the identity. 

\begin{propo}
If $\OO$ is an operad with an $A_\infty$-multiplication $m\in\OO$, then there is an $A_\infty$-algebra structure on the shifted module $S\s\OO$. 
\end{propo}
\begin{proof}
Note in the proof of \Cref{lemmadegree} that a way to turn $M'_j$ into a map of degree $2-j$ is introducing a grading on $\s\OO$ given by arity plus internal degree (without substracting one). This is equivalent to defining an $A_\infty$-algebra structure $M$ on $S\s\OO$ shifting the map $M'=M'_1+M'_2+\cdots$, where $S$ is the shift of graded modules. Therefore, we define $M_j$ to be the map making the following diagram commute.

\[
\begin{tikzcd}
(S\s\OO)^{\otimes j}\arrow[r,"M_j"]\arrow[d, "(S^{\otimes j})^{-1}"'] & S\s\OO\\
\s\OO^{\otimes j}\arrow[r, "M'_j"] & \s\OO\arrow[u,"S"']
\end{tikzcd}
\]

In other words, $M_j=\overline{\sigma}(M'_j)$, where $\overline{\sigma}(F)=S\circ F\circ (S^{\otimes n})^{-1}$ for $F\in\End_{\s\OO}(n)$ is the map inducing an isomorphism $\End_{\s\OO}\cong \s\End_{S\s\OO}$. Since $\overline{\sigma}$ is an operad morphism, for $M=M_1+M_2+\cdots$, we have

\[
M\tilde{\circ}M=\overline{\sigma}(M')\tilde{\circ}\overline{\sigma}(M')=\overline{\sigma}(M'\circ M')=0.
\]
%MAYBE DEFINE $\overline{\sigma}_n$ FOR EACH ARITY SO THAT THE ABOVE IS NOT AN ABUSE OF NOTATION. OTHERWISE SAY IT IS AN ABUSE OF NOTATION

So now we have that $M\in\s\End_{S\s\OO}$ is an element of natural degree 1 and such that $M\tilde\circ M=0$. Therefore, in light of \Cref{multiplicationalgebra}, $M$ is the desired $A_\infty$-algebra structure on $S\s\OO$. 
\end{proof}
Notice that $M$ is defined as an structure map on $S\s\OO$. This kind of shifted operad is called \emph{odd operad} in \cite{ward}. This means that $S\s\OO$ is not an operad anymore, since the associativity relation for graded operads involves signs that depend on the degrees, which are now shifted. 

\section{Iterating the process}\label{sect3}

Now we can apply the same construction to the operad $\s\End_{S\s\OO}$ and we get $A_\infty$-algebra structure given by maps
\[\overline{M}_j:(S\s\End_{S\s\OO})^{\otimes j}\to S\s\End_{S\s\OO}\]
obtained using $\overline{\sigma}$ from maps
\[\overline{M}'_j:(\s\End_{S\s\OO})^{\otimes j}\to \s\End_{S\s\OO}\]
defined as
\begin{align*}
&\overline{M}'_j(f_1,\dots,f_j)=\overline{B}_j(M;f_1,\dots, f_j) & j>1,\\
&\overline{M}'_1(f)=\overline{B}_1(M;f)-(-1)^{|f|}\overline{B}_1(f;M),
\end{align*}
where $\overline{B}_j$ denotes the brace map on $\s\End_{S\s\OO}$.

We define the Hochschild complex as done by Ward in \cite{ward}.
\begin{defin}
The Hochschild cochains of a graded module $A$ to be the graded module $S\s\End_A$. In particular, $S\s\End_{S\s\OO}$ is the module of Hochschild cochains of $S\s\OO$.
\end{defin}
\begin{remark}
The functor $S\s$ is called the ``oddification'' of an operad in the literature. %Ward but the whole thesis 
The reader might find odd to define the Hochschild complex in this way instead of just $\End_A$. The reason is that the operadic suspension provides the necessary signs and the extra shift gives us the appropriate degrees. In addition, this definition allows the extra structure to arise naturally instead of having to define the signs by hand. For instance, if we have an associative multiplication $m_2\in\End_A(2)=\Hom(A^{\otimes 2},A)$, the element $m_2$ would not satisfy the equation $m_2\circ m_2=0$ and thus cannot be used to induce a multiplication on $\End_A$ as we did above.
\end{remark}

 A natural question to ask is what relation there is between the $A_\infty$-algebra structure on $S\s\OO$ and the one on $S\s\End_{S\s\OO}$. In \cite{GV} it is claimed that given an operad $\OO$ with an $A_\infty$-multiplication, the map

%I'M WRITING THIS BRACE WITH BAR BECAUSE WITHOUT BAR BECAUSE I WILL HAVE TO USE $B$ FOR THE BRACE IN THE ENDORMORPHISM OPERAD (NON OPERADIC-SUSPENDED). A POSSIBILITY TO BE CONSISTENT IS USING THE LETTER B FOR NON-SUSPENDED OPERADS AND BAR B FOR SUSPENDED OPERADS, INTRODUCING THE BAR WHEN IT IS THE ENDOMORPHISM OF ANOTHER OPERAD
\begin{align*}
&\OO \to \End_\OO\\
&x\mapsto \sum_{n\geq 0}b_n(x;-)
\end{align*}
is a morphism of $A_\infty$-algebras. We are going to adapt the statement of this claim to our context and prove it. Let $\Phi'$ the map defined as above but on $\s\OO$, i.e.
\begin{align*}
\Phi'\colon&\s\OO \to \End_{\s\OO}\\
&x\mapsto \sum_{n\geq 0}b_n(x;-).
\end{align*}
Let $\Phi:S\s\OO\to S\s\End_{S\s\OO}$ the map making the following diagram commute
\[
\begin{tikzcd}
S\s\OO\arrow[rr, "\Phi"]\arrow[d] & & S\s\End_{S\s\OO}\\
\s\OO\arrow[r, "\Phi'"]& \End_{\s\OO}\arrow[r, "\cong"]& \s\End_{S\s\OO}\arrow[u]
\end{tikzcd}
\]
where the isomorphism $\End_{\s\OO}\cong\s\End_{S\s\OO}$ is given by $\overline{\sigma}$. Note that the degree of the map $\Phi$ is zero.

\begin{remark}
Notice that we have only used the operadic structure on $\s\OO$ to define an $A_\infty$-algebra structure on $S\s\OO$, so the constructions and results in these sections are valid if we replace $\s\OO$ by any graded module $A$ such that $SA$ is an $A_\infty$-algebra. 
\end{remark}

\begin{thm}\label{theorem}
The map $\Phi$ defined above is a morphism of $A_\infty$-algebras, i.e. for all $j\geq 1$ the equation

\[\Phi(M_j)=\overline{M}_j(\Phi^{\otimes j})\]
holds, where the $M_j$ is the $j$-th component of the $A_\infty$-algebra structure on $S\s\OO$ and $\overline{M}_j$ is the $j$-th componnent of the $A_\infty$-algebra structure on $S\s\End_{S\s\OO}$. 
\end{thm}
\begin{proof}
Let us have a look at the following diagram

%\[
%\begin{tikzcd}
%(S\s\OO)^{\otimes j}\arrow[r,red] \arrow[d, bend right=15,"M_j"']\arrow[rrrr,bend left=15, "\Phi^{\otimes j}"]&\s\OO^{\otimes j}\arrow[r,blue, "(\Phi')^{\otimes j}"]\arrow[d, blue, "M'_j"] & (\End_{\s\OO})^{\otimes j}\arrow[r, blue,"\overline{\sigma}^{\otimes j}"] \arrow[d, dashed, "\mathcal{M}_j",blue]& (\s\End_{S\s\OO})^{\otimes j}\arrow[r,red]\arrow[d, "\overline{M}'_j",blue]& (S\s\End_{S\s\OO})^{\otimes j}\arrow[d, bend left=15, "\overline{M}_j"] \\
%S\s\OO\arrow[rrrr, bend right=15, "\Phi"']\arrow[r,red]&\s\OO\arrow[r, blue, "\Phi'"]& \End_\s\OO \arrow[r, blue, "\overline{\sigma}"] & \s\End_{S\s\OO}\arrow[r,red]& S\s\End_{S\s\OO}
%\end{tikzcd}
%\]


\[
\begin{tikzcd}
(S\s\OO)^{\otimes j}\arrow[dr,red] \arrow[ddd, bend right=10,"M_j"']\arrow[rrrr,bend left=10, "\Phi^{\otimes j}"]& & & & (S\s\End_{S\s\OO})^{\otimes j}\arrow[ddd, bend left=10, "\overline{M}_j"]\\
&\s\OO^{\otimes j}\arrow[r,blue, "(\Phi')^{\otimes j}"]\arrow[d, blue, "M'_j"] & (\End_{\s\OO})^{\otimes j}\arrow[r, blue,"\overline{\sigma}^{\otimes j}"] \arrow[d, dashed, "\mathcal{M}_j",blue]& (\s\End_{S\s\OO})^{\otimes j}\arrow[ur,red]\arrow[d, "\overline{M}'_j",blue]& \\
&\s\OO\arrow[r, blue, "\Phi'"]& \End_{\s\OO} \arrow[r, blue, "\overline{\sigma}"] & \s\End_{S\s\OO}\arrow[dr,red]& \\
S\s\OO\arrow[rrrr, bend right=10, "\Phi"']\arrow[ur,red]& & & & S\s\End_{S\s\OO}
\end{tikzcd}
\]
where the diagonal red arrows are shifts of graded $R$-modules. We need to show that the diagram defined by the external black arrows commute. But these arrows are defined so that they commute whith the red and blue arrows, so it is enough to show that the inner blue diagram commutes. The blue diagram can be split into two different squares using the dashed arrow $\mathcal{M}_j$ that we are going to define next, so it will be enough to show that the two squares commute. The commutativity of the  left square will be more involved as we will have to distinguish between different kinds of insertions.

 The map 
\[\mathcal{M}_j:(\End_{\s\OO})^{\otimes j}\to\End_{\s\OO}\]
is defined by 
\begin{align*}
&\mathcal{M}_j(f_1, \dots, f_j)=B_j(M';f_1,\dots, f_j) &\text{ for }j>1,\\
&\mathcal{M}_1(f)=B_1(M';f)-(-1)^{|f|}B_1(f;M'),
\end{align*}
 where $B_j$ is the natural brace structure map on the operad $\End_{\s\OO}$, i.e. 
\[B_j(f;f_1,\dots, f_j)=\sum_{k_0+\cdots+k_j=a(f)-j} f(1^{\otimes k_0}\otimes f_1\otimes 1^{\otimes k_1}\otimes\cdots\otimes f_j\otimes 1^{\otimes k_j}).\]
 The $1$'s in the brace structure are identity maps. In the above definition, $|f|$ denotes the degree of $f$ as an element of $\End_{\s\OO}$, which is the same as the degree $\overline{\sigma}(f)\in \s\End_{S\s\OO}$ because $\overline{\sigma}$ is an isomorphism.  %the degree as a map sO^n\to sO, which is computed by evaluating and computing arity +degree-1
 \subsection*{Commutativity of the right blue square}
 Let us show now that the right square commutes. Recall that $\overline{\sigma}$ is an isomorphism of operads and $M=\overline{\sigma}(M')$. Then we have for $j>1$
 
 \[\overline{M}'_j(\overline{\sigma}(f_1),\dots,\overline{\sigma}(f_j))=\overline{B}_j(M;\overline{\sigma}(f_1),\dots,\overline{\sigma}(f_j))=\overline{B}_j(\overline{\sigma}(M');\overline{\sigma}(f_1),\dots,\overline{\sigma}(f_j)).\]
 Now, since the brace structure is defined as an operadic composition, it commutes with $\overline{\sigma}$, so
 
 \[\overline{B}_j(\overline{\sigma}(M');\overline{\sigma}(f_1),\dots,\overline{\sigma}(f_j))=\overline{\sigma}(B_j(M';f_1,\dots, f_j))=\overline{\sigma}(\mathcal{M}_j(f_1,\dots, f_j)),\]
 and therefore the right blue square commutes for $j>1$. For $j=1$ the result follows analogously taking into account that the degree of $f$ in $\End_{\s\OO}$ is the same as the degree of $\overline{\sigma}(f)$ in $\s\End_{S\s\OO}$.\\
 
 

%BEING A MORPHISM OF AINFTY FORCES THE DEGREE OF THE MAP TO BE ZERO BECAUSE ONE SIDE HAS DEGREE 2-J+DEG AND THE OTHER SIDE HAS 2-J+J(DEG), SO DEG=0

%$\Phi^j$ SHOULD BE DEFINED USING $(S^{-1})^j$ INSTEAD OF THE INVERSE OF THE TENSOR, BUT THE EXTRA SIGN CANCELS BECAUSE THIS MAP IS USED TWO TIMES. IN ADDITION I NEED THE INVERSE OF THE TENSOR TO DEFINE THE AINFTY MAPS

%PART OF THE PROOF CAN BE SEEN AS SHOWING EXACTLY WHAT THEY SAID (IT DOESN'T DEPEND ON THE PARTICULAR OPERAD AS LONG AS THE BRACE STRUCTURE IS THE NATURAL ONE ON THAT OPERAD), SO TELL IT MAYBE DURING THE PROOF. THE PROOF IS QUITE LONG, IT PROBABLY DESERVES ITS OWN SECTION, OR MAYBE INDICATING THAT THE CALCULATIONS FOR THE FIRST HALF OF DIAGRAM ARE IN ANOTHER SECTION
%\vspace{0.5cm}

The proof that the left blue square commutes consists of several lenghty calculations so we are going to devote the next section to that. However, it is worth noting that the commutativity of the left square does not depend on the particular operad $\s\OO$, so it is still valid if $m$ satisfies $m\circ m=0$ for any circle operation defined in terms of insertions. This is essentialy the original statement in \cite{GV}.
\subsection*{Commutativity of the left blue square}
We are going to show here that the left blue square in the diagram of the proof of Theorem \ref{theorem} commutes, i.e. that 

\begin{equation}\label{commutative}
\Phi'(M'_j)=\mathcal{M}_j((\Phi')^{\otimes j})
\end{equation}

for all $j\geq 1$. First we prove the case $j>1$. Let $x_1,\dots, x_j\in \s\OO^{\otimes j}$. We have on the one hand



\begin{align*}
\Phi'(M'_j(x_1,\dots, x_j))=&\Phi'(b_j(m;x_1,\dots, x_j))=\sum_{n\geq 0} b_n(b_j(m;x_1,\dots, x_j);-)=\\
&\sum_n\sum_l\sum b_l(m; -, b_{i_1}(x_1;-),\cdots,b_{i_j}(x_j;-),-)
\end{align*}
where $l=n-(i_1+\cdots+i_j)+j$. The sum with no subindex runs over all the possible order-preserving insertions. Note that $l\geq j$. Evaluating the above map on elements would yield Koszul signs coming from the brace relation. Also recall that $|b_j(x;-)|=|x|$. Now, fix some value of $l\geq j$ and let us compute

\begin{align*}
\mathcal{M}_j(\Phi'(x_1),\dots, \Phi'(x_j))=B_j(M';\Phi'(x_1),\dots, \Phi'(x_j))
\end{align*}

but focus on the $M'_l$ component, i.e. on $B_j(M'_l;\Phi'(x_1),\dots, \Phi'(x_j))$. By definition, this equals

\begin{align*}
\sum M'_l(-,\Phi'(x_1),\cdots, \Phi'(x_j),-)=&\sum_{i_1,\dots, i_j}\sum M'_l(-,b_{i_1}(x_1;-),\cdots,b_{i_j}(x_j;-),-)=\\
&\sum_{i_1,\dots, i_j}\sum b_l(m;-,b_{i_1}(x_1;-),\cdots,b_{i_j}(x_j;-),-)
\end{align*}

We are using hyphens instead of $1$'s to make the equality of both sides of the equation (\ref{commutative}) more apparent, and to make clear that when evaluating on elements those are the places where the elements go. %In this case, evaluating yields the same signs as in the other side of the equation. 

For each tuple $(i_1,\dots, i_j)$ we can choose $n$ such that $n-(i_1+\cdots+i_j)+j=l$, so the above sum equals

\[\underset{n-(i_1+\cdots+i_j)+j=l}{\sum_{n,i_1,\dots, i_j}}\sum b_l(m;-,b_{i_1}(x_1;-),\cdots,b_{i_j}(x_j;-),-).\]

So each $M'_l$ component for $l\geq j$ produces precisely the terms $b_l(m;\dots)$ appearing in $\Phi'(M'_j)$. Conversely, for every $n\geq 0$ there exists some tuple $(i_1,\dots, i_j)$ and some $l\geq j$ such that $n$ is the that $n-(i_1+\cdots+i_j)+j=l$, so we do get all the summands from the left hand side of the equation (\ref{commutative}), and thus we have the equality $\Phi'(M'_j)=\mathcal{M}_j((\Phi')^{\otimes j})$ for all $j>1$.

It is worth treating the case $n=0$ separately since in that case we have the summand \[b_0(b_j(m;x_1,\dots, x_j))\] 
in $\Phi'(b_j(m;x_1,\dots, x_j))$, where we cannot apply the brace relation. This summand is equal to \[B_j(M'_j;b_0(x_1),\dots, b_0(x_j))=M'_j(b_0(x_1),\dots, b_0(x_j))=b_j(m;b_0(x_1),\dots, b_0(x_j)),\] since by definition $b_0(x)=x$.% We obtained this map from $\overline{M}_1(\Phi(x))$. To see that the two maps are actually equal, apply them to $1\in k$ to output $b_1(m;x)$ in both cases. %Notice that the terms that $b_1(M_i;x)$ produces for $i>1$ appear using the brace relation in $b_k(b_1(m;x);-)$ when $k>0$, more precisely, in the summand $b_k(b_1(m_i;x);-)$. 

Now we are going to show that 

\begin{equation}\label{case1}
\Phi'(M'_1(x))=\mathcal{M}_1(\Phi'(x)).
\end{equation} This going to be divided into two parts, since $M'_1$ has to clearly distinct summands.

\subsubsection*{Insertions in $m$}

Let us first focus on the insertions in $m$ that appear in equation (\ref{case1}). Recall that 

\begin{equation}\label{phim}
\Phi'(M'_1(x))=\Phi'([m,x])=\Phi'(b_1(m;x))-(-1)^{|x|}\Phi'(b_1(x;m))
\end{equation}

so we focus on the first summand. 

\begin{align*}
\Phi'(b_1(m;x))=&\sum_n b_n(b_1(m;x);-)=\sum_n \underset{n\geq i}{\sum_i} \sum b_{n-i+1}(m;-, b_i(x;-),-)=\\
&\underset{n-i+1> 0}{\sum_{n,i}}\sum b_{n-i+1}(m;-, b_i(x;-),-)
\end{align*}

where the sum with no indices runs over all the positions in which $b_i(x;-)$ can be inserted (from $1$ to $n-i+1$ in this case). 


On the other hand, since $|\Phi'(x)|=|x|$, the right hand side of equation (\ref{case1}) becomes

\begin{equation}\label{mphi}
\mathcal{M}_1(\Phi'(x))=B_1(M';\Phi'(x))-(-1)^{|x|}B_1(\Phi'(x);M').
\end{equation}

Again, we are focusing now on the first summand, but with the exception of the part of $M_1$ that corresponds to $b_1(\Phi(x);m)$. From here the argument is a particular case of the proof for $j>1$, so the terms of the form $b_l(m;\cdots)$ are the same on both sides of the equation (\ref{case1}). 

%. We now look at the index $l=n-i+1> 0$ that determines the arity of the map $b_{n-i+1}(m;\dots)$. We show that for each value of $l$, $B_1(M_{n-i+1};\Phi(x))$ produces exactly the terms involving $b_{k-i+1}(m;\dots)$. 
%
%\begin{align*}
%B_1(M_{k-i+1};\Phi(x))=&\sum M_{k-i+1}(-,\Phi(x),-)=\sum_j\sum M_{k-i+1}(-,b_j(x;-),-)=\\
%&\sum_j\sum b_{k-i+1}(m;-,b_j(x;-),-)
%\end{align*}
%Again, the sum without limits runs over the possible insertions. Note that we have fixed the value $l=k+i-1$, but not $k$ or $i$, so for each $j$ we can choose $i$ and and a value $k'$ of $k$ for which $k'-j+1=k-i+1$. This can be done thanks to the fact that $k$ runs over all natural numbers (including 0). So we may simply rewrite the above sum as
%
%$$\underset{k-i+1=l}{\sum_{k,i}}\sum b_{k-i+1}(m;-,b_i(x;-),-)$$
%
%which is precisely what we had before for each fixed value of $k-i+1$. %Since there's no $M_0$, we have to treat the case $l=0$ separately. In that case
%


\subsubsection*{Insertions in $x$}

And now, let us study the insertions in $x$ that appear in equation (\ref{case1}). Let us look first at the left hand side. From $\Phi'(M'_1(x))$ in equation (\ref{phim}) we had 

\[-(-1)^{|x|}\Phi'(b_1(x;m))=-(-1)^{|x|}\sum_n b_n(b_1(x;m);-).\]

The factor $(-1)^{|x|}$ is going to appear everywhere, so we may cancel it. Then we just have

\[\Phi'(b_1(x;m))=\sum_n b_n(b_1(x;m);-).\]
We are going to evaluate each term of the sum, so let $z_1,\dots, z_n\in \s\OO$. We have by the brace relation that

\begin{align}\label{insertionx1}
b_n(b_1(x;m);z_1,\dots, z_n)&=\\
 \sum_{l+j=n+1}\sum_{i=1}^{n-j+1}(-1)^{\varepsilon} b_l(x;z_1,\dots,b_j(m;z_{i},\dots, z_{i+j}),\dots, z_n)&+\sum_{i=1}^{n+1}(-1)^{\varepsilon}b_{n+1}(x;z_1,\dots, z_{i-1},m,z_i,\dots, z_n),\nonumber
\end{align}

where $\varepsilon$ is the usual Koszul sign with respect to the grading in $\s\OO$. We have to check that the insertions in $x$ that appear in $\mathcal{M}_1(\Phi'(x))$ (the right hand side of the equation (\ref{case1})) are exactly those in the equation (\ref{insertionx1}) above (that corresponds to the left hand side of equation (\ref{case1})).

Now, let us look at the right hand siide of equation (\ref{case1}). From $\mathcal{M}_1(\Phi'(x))=B_1(M';\Phi'(x))-(-1)^{|x|}B_1(\Phi'(x);M')$ (equation (\ref{mphi})) we have 
\[-(-1)^{|x|}b_1(\Phi'(x);m)=-(-1)^{|x|}\sum_n b_1(b_n(x;-);m)\] 
coming from the first summand since $B_1(M'_1;\Phi'(x))=M'_1(\Phi'(x))$. We are now only interested in insertions in $x$. Again, cancelling $-(-1)^{|x|}$ we get
\[b_1(\Phi'(x);m)=\sum_n b_1(b_n(x;-);m).\] 
Each term of the sum can be evaluated on $(z_1,\dots, z_n)$ to produce

\begin{align}\label{insertionx2}
b_1(b_n(x;z_1, \dots, z_n);m)&=\\
\sum_{i=1}^n (-1)^{\varepsilon+|z_i|}b_n(x;z_1,\dots, b_1(z_i;m),\dots, z_n)&+\sum_{i=1}^{n+1} (-1)^{\varepsilon}b_{n+1}(x;z_1,\dots, z_{i-1},m,z_{i},\dots, z_n)\nonumber
\end{align}

Note that we have to apply the Koszul sign rule twice: once at evaluation, and once more to apply the brace relation. The second sum in equation (\ref{insertionx2}) is the same as the second sum in equation (\ref{insertionx1}), so from now on we only need to use the first sum of the equation (\ref{insertionx2}). Now, from the second summand of $\mathcal{M}_1(\Phi'(x))$ in the right hand side of equation (\ref{mphi}) we obtain

\begin{align*}
-(-1)^{|x|}B_1(\Phi'(x);M')=&-(-1)^{|x|}\sum_l B_1(b_l(x;-);M')=-(-1)^{|x|}\sum_l\sum b_l(x;-,M',-) \\
=&-(-1)^{|x|}\left(\sum_{j> 1} \sum_l\sum b_l(x;-,b_j(m;-),-)+\sum_l\sum b_l(x;-,b_1(-;m),-)\right).
\end{align*}
 Once again we cancel the factor $(-1)^{|x|}$ and we are going to evaluate on $(z_1,\dots, z_n)$ to make this map more explicit. This evaluation gives us the following
 
 \begin{align*}
 \sum_{l+j=n+1}\sum_{i=1}^{n-j+1}(-1)^{\varepsilon} b_l(x;z_1,\dots,b_j(m;z_{i},\dots, z_{i+j}),\dots, z_n)&-\sum_{i=1}^{n} (-1)^{\varepsilon+|z_i|}b_n(x;z_1,\dots,b_1(z_{i};m),\dots, z_n)
 \end{align*}
The minus sign comes from the fact that $b_1(z_i;m)$ comes from $M'_1(z_i)$, so we apply the signs in the definition of $M'_1(z_i)$. We can see that the second sum above is the same as the first sum in equation (\ref{insertionx2}), so we have now cancelled both sums of that equation (recall that the second sum of equation (\ref{insertionx2})) was already cancelled before).
 
 So we are left with only the first sum of the last expression which is the same as the first sum in equation (\ref{insertionx1}), so we have already checked that the equation $\Phi'(M'_1)=\mathcal{M}_1(\Phi')$ holds. 
  
 In the case $n=0$, we have to note that $B_1(b_0(x);m)$ vanishes because of arity reasons: $b_0(x)$ is a map of arity 0, so we cannot insert any inputs. And this finishes the proof.
 \end{proof}
%\appendix
%\renewcommand{\appendixname}{Appendix:}
\begin{appendices}
\appendix
\gdef\thesection{Appendix \Alph{section}}
\section{Explicit $A_\infty$-algebra structure}\label{sect4}
% PROBABLY ALSO INCLUDE THE COMPUTATION OF THE AINFTY EQUATION

We have given an implicit definition of the components of the $A_\infty$-algebra structure on $S\s\OO$, namely, \[M_j=\overline{\sigma}(M'_j)=(-1)^{\binom{j}{2}}S\circ M'_j\circ(S^{-1})^{\otimes j},\]
but it is useful to have an explicit expression that determines how it is evaluated on elements of $S\s\OO$.  This explicit formulas will make more clear the connection with the work of Gerstenhaber and Voronov.  We also hope that these explicit expression can be useful to perform calculations in other mathematical contexts where $A_\infty$-algebras are used.

\begin{lemma}
For $x,x_1,\dots,x_n\in\s\OO$, we have the following expressions.

\begin{align*}
&M_n(Sx_1,\dots, Sx_n)=(-1)^{\sum_{i=1}^n(n-i)|x_i|}Sb_n(m;x_1,\dots, x_n) & & n>1\\
&M_1(Sx)=Sb_1(m;x)-(-1)^{|x|}Sb_1(x;m).
\end{align*}

Here $|x|$ is the degree of $x$ as an element of $\s\OO$. 
\end{lemma}
\begin{proof}
The deduction of these explicit formulas is done as follows. Let $n>1$ and $x_1,\dots, x_n\in \s\OO$. Then

\begin{align*}
M_n(Sx_1,\dots, Sx_n)=SM'_n((S^{\otimes n})^{-1})(Sx_1,\dots, Sx_n)\\
(-1)^{\binom{n}{2}}SM'_n((S^{-1})^{\otimes n})(Sx_1,\dots, Sx_n)=\\
(-1)^{\binom{n}{2}+\sum_{i=1}^n(n-i)(|x_i|+1)}SM'_n(S^{-1}Sx_1,\dots, S^{-1}Sx_n)=\\
(-1)^{\binom{n}{2}+\sum_{i=1}^n(n-i)(|x_i|+1)}SM'_n(x_1,\dots,x_n)
\end{align*}

Now, note that $\binom{n}{2}$ is even exatly when $n\equiv 0,1\mod 4$. In these cases an even number of $|x_i|$ have an odd coefficient in the sum (when $n\equiv 0\mod 4$ these are the $|x_i|$ with even index, and when $n\equiv 1\mod 4$, the $|x_i|$ with odd index). This means that 1 is added on the exponent an even number of times, so the sign is not changed by the binomial coefficient nor by adding 1 on each term. Similarly, when $\binom{n}{2}$ is odd, i.e. when $n\equiv 2,3\mod 4$, there is an odd number of $|x_i|$ with odd coefficient, so the addition of 1 an odd number of times cancels the binomial coefficient. This means that the above expression equals

\[(-1)^{\sum_{i=1}^n(n-i)|x_i|}SM'_n(x_1,\dots,x_n),\]
which by definition equals
\[(-1)^{\sum_{i=1}^n(n-i)|x_i|}Sb_n(m;x_1,\dots,x_n).\]

The case $n=1$ is analogous, one just has to note that 

\[
M'_1(x)=b_1(m;x)-(-1)^{|x|}b_1(x;m)
\]
and that $\overline{\sigma}$ is linear. 
\end{proof}

It is possible to show that the maps defined explicitly as we have just done satisfy the $A_\infty$-equation without relying on the fact that $\overline{\sigma}$ is a map of operads, but it is a lengthy and tedious calculation.

\begin{remark}
In the case $n=2$, omitting the shift symbols by abuse of notation, we obtain 

\[M_2(x,y)=(-1)^{|x|}b_2(m;x,y).\]
Let $M^{GV}_2$ be the product defined in \cite{GV} as \[M^{GV}_2(x,y)=(-1)^{|x|+1}b_2(m;x,y).\] We see that $M_2=-M^{GV}_2$. Since the authors of \cite{GV} work in the associative case $m=m_2$, this minus sign does not affect the $A_\infty$-relation (which in this case reduces to the associativity and differential relations). This difference in sign can be explained by the difference between $(S^{\otimes n})^{-1}$ and $(S^{-1})^{\otimes n}$, since any of these maps can be used to define a map $(S\s\OO)^{\otimes n}\to \s\OO^{\otimes n}$. 
\end{remark}
%
%I MIGHT NEED TO DEDUCE THE EXPRESSIONS OF PHI(M1)=M1(PHI) AND SAME WITH M2 TO OBTAIN SOME STRUCTURE ON COHOMOLOGY AS IN GV

%THE LEFT BLUE SQUARES IMPLIES G-V EQUATIONS EXCEPT WITH BRACE INSTEAD OF THE PRODUCT (SO UP TO THAT PRECISE SIGN), WITH THE PRODUCTS OTHER SIGNS APPEAR RELATED TO THE BRACES WHICH ARE NOT EXACTLY PHI
\end{appendices}
%\phantomsection
\bibliographystyle{ieeetr}
\bibliography{newbibliography}
\end{document}
