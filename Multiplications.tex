	\documentclass[twoside]{article}
\usepackage{estilo-ejercicios}
\setcounter{section}{0}
\newtheorem{defin}{Definition}[section]
\newtheorem{lem}[defin]{Lemma}
\newtheorem{propo}[defin]{Proposition}
\newtheorem{thm}[defin]{Theorem}
\newtheorem{eje}[defin]{Example}
\renewcommand{\baselinestretch}{1,3}

\usepackage{empheq}
\newcommand*\widefbox[1]{\fbox{\hspace{2em}#1\hspace{2em}}}
%--------------------------------------------------------
\begin{document}

\title{$A_\infty$-structures on operads}
\author{Javier Aguilar Martín}
\maketitle

\section{Introduction}
Continuing with the operadic suspension and the brace structure obtained from it, we then define an $A_\infty$-algebra structure on $\s\OO$.


\section{Defining the $A_\infty$-maps}


Let us consider the underlying graded vector space of the operad $\s\OO$ with grading given by its natural degree, i.e. if $x\in \s\OO(n)$ we definee its natural degree $|x|=n+\deg(x)-1$ where $\deg(x)$ is its internal degree (the degree as an element of $\OO(n))$. 

For any operad $\OO$, recal the operation $\circ$ defined as

\[
a\circ b=\sum_{i=1}^n a\circ_i b
\]
for $a\in\OO(n)$ and $b\in \OO(m)$. We write $a\tilde{\circ}b$ for the corresponding operation on $\s\OO$, namely

\[
a\tilde{\circ} b=\sum_{i=1}^n a\tilde{\circ}_i b.
\]


Let $m\in\s\OO$ be of natural degree 1 such that $m\tilde{\circ}m=0$, so that $m=m_1+m_2+\cdots$ is a formal sum of maps $m_j\in\s\OO(j)$ of internal degree $2-j$ satisfying the usual $A_\infty$-equation for all $n$
\begin{equation}\label{Ainftyeq}
\sum_{r+s+t=n}(-1)^{rs+t}m_{r+1+t}(1^{\otimes r}\otimes m_s\otimes 1^{\otimes t})=0
\end{equation} 


Following \cite{GV} and \cite{getzler}, one would define an $A_\infty$-algebra structure on $\s\OO$ using the maps 

\begin{align*}
M'_1(x)\coloneqq [m,x]=m\tilde{\circ} x-(-1)^{|x|}x\tilde{\circ}m, & &  \\
M'_j(x_1,\dots, x_j)\coloneqq b_j(m;x_1,\dots, x_j),& &j>1.
\end{align*}
The prime notation here is used to indicate that these are not the definitive maps that we are going to take. Getzler shows in \cite{getzler} that $M'=M'_1+M'_2+\cdots$ satisfies the relation $M'\circ M'=0$ using that $m\circ m=0$, but the proof is independent of the operad in which $m$ is defined, so it is still valid if $m\tilde{\circ}m=0$. But we have two problems here. The equation $M'\circ M'=0$ does depend on how the circle operation is defined, more precisely, this circle operation in \cite{getzler} is the one without signs on the endomorphism operad, so $M'$ is not an $A_\infty$-structure under our convention. The other problem has to do with the degrees. We need $M'_j$ to be homogeneous of degree $2-j$ as a map $\s\OO^{\otimes j}\to \s\OO$, but we find that $M'_j$ is homogeneous but of degree 1. 

Indeed, the degree of $b_j(m_n;x_1,\dots,x_j)$ for $n>j$ is computed as follows. Let $a(x)$ denote the arity of $x$, i.e. $a(x)=n$ whenever $x\in\s\OO(n)$. By definition, we have that the degree of $b_j(m_n;x_1,\dots,x_j)$ as an element of $\s\OO$ is

\[a(b_j(m_n;x_1,\dots,x_j))+\deg(b_j(m_n;x_1,\dots,x_j))-1.\]

We have 

\[a(b_j(m_n;x_1,\dots,x_j)=n-j+\sum_i a(x_i)\]

and 

\[\deg(b_j(m_n;x_1,\dots,x_j)=2-n+\sum_i\deg(x_i),\]

so 
\begin{align*}
a(b_j(m_n;x_1,\dots,x_j))+\deg(b_j(m_n;x_1,\dots,x_j))-1=\\
n-j+\sum_i a(x_i)+2-n+\sum_i\deg(x_i)-1=\\
2-j+\sum_i a(x_i)+\sum_i\deg(x_i)-1=\\
1+\sum_i|x_i|.
\end{align*}
This means that the degree of the map $b_j(m_n;-)$ does not depend on $n$ and equals $1$, so $|M'_j|=1$. %A first alternative after finding this result is considering $M'_j$ to be an element of $\s\End_{\s\OO}$ instead of just $\End_{\s\OO}$. This solves the problem of the degree, but not the one of the sign convention. 

Note in the above computation that a way to turn $M'_j$ into a map of degree $2-j$ is introducing a grading on $\s\OO$ given by arity plus internal degree (without substracting one). This is equivalent to defining an $A_\infty$ structure $M$ on $S\s\OO$ shifting the map $M'$, where $S$ is the shift of graded modules. Therefore, we define $M_j$ to be the map making the following diagram commute.

\[
\begin{tikzcd}
(S\s\OO)^{\otimes j}\arrow[r,"M_j"]\arrow[d, "(S^{\otimes j})^{-1}"'] & S\s\OO\\
\s\OO^{\otimes j}\arrow[r, "M'_j"] & \s\OO\arrow[u,"S"']
\end{tikzcd}
\]

In other words, $M_j=\overline{\sigma}(M'_j)$, where $\overline{\sigma}(F)=S\circ F\circ (S^{\otimes n})^{-1}$ for $F\in\End_{\s\OO}(n)$ is the map inducing an isomorphism $\End_{\s\OO}\cong \s\End_{S\s\OO}$. Since $\overline{\sigma}$ is an operad morphism, we have

\[
M\tilde{\circ}M=\overline{\sigma}(M)\tilde{\circ}\overline{\sigma}(M)=\overline{\sigma}(M'\circ M')=0.
\]
%MAYBE DEFINE $\overline{\sigma}_n$ FOR EACH ARITY SO THAT THE ABOVE IS NOT AN ABUSE OF NOTATION. OTHERWISE SAY IT IS AN ABUSE OF NOTATION

So now we have that $M\in\s\End_{S\s\OO}$ is a formal sum $M=M_1+M_2+\cdots$ such that $M_j$ has arity $j$ and internal degree $2-j$; and such that $M\tilde\circ M=0$. Therefore $M$ is the desired $A_\infty$-structure. Nevertheless, recall that $M$ is defined as an structure map on $S\s\OO$. This kind of shifted operad is called \emph{odd operad} in \cite{ward}. This means that $S\s\OO$ is not an operad anymore, since the associativity relation for graded operads involves signs that depend on the degrees, which are now shifted.

\section{Iterating the process}

Now we can apply the same construction to the operad $\s\End_{S\s\OO}$ and we get $A_\infty$-maps 
\[\overline{M}_j:(S\s\End_{S\s\OO})^{\otimes j}\to S\s\End_{S\s\OO}\]
obtained using $\overline{\sigma}$ from maps
\[\overline{M}'_j:(\s\End_{S\s\OO})^{\otimes j}\to \s\End_{S\s\OO}\]
defined as
\[\overline{M}'_j(f_1,\dots,f_j)=\overline{B}_j(M;f_1,\dots, f_j),\]
where $\overline{B}_j$ denotes the brace map on $\s\End_{S\s\OO}$. A natural question to ask is what relation is there between the $A_\infty$-structure on $S\s\OO$ and the one on $S\s\End_{S\s\OO}$ if any. In \cite{GV} it is claimed that given an operad $\OO$ with an $A_\infty$-multiplication, the map

\begin{align*}
&\OO \to \End_\OO\\
&x\mapsto \sum_{n\geq 0}b_n(x;-)
\end{align*}
is a morphism of $A_\infty$-algebras. We are going to adapt the statement of this claim to our context and prove it. Let PART OF THE PROOF CAN BE SEEN AS SHOWING EXACTLY WHAT THEY SAID, SO TELL IT MAYBE DURING THE PROOF

\begin{thm}

\end{thm}



I'M WRITING THIS BRACE WITH BAR BECAUSE WITHOUT BAR BECAUSE I WILL HAVE TO USE $B$ FOR THE BRACE IN THE ENDORMORPHISM OPERAD (NON OPERADIC-SUSPENDED). A POSSIBILITY TO BE CONSISTENT IS USING THE LETTER B FOR NON-SUSPENDED OPERADS AND BAR B FOR SUSPENDED OPERADS, INTRODUCING THE BAR WHEN IT IS THE ENDOMORPHISM OF ANOTHER OPERAD
%\appendix
%\renewcommand{\appendixname}{Appendix:}
\begin{appendices}
\appendix
\gdef\thesection{Appendix \Alph{section}}
\section{Explicit $A_\infty$-maps}
DERIVATION OF THE SIGN THAT COMES IN FRON THE MAPS $M_J$ AND TELL THAT THIS COINCIDES WITH G-V IN THE ASSOCIATIVE CASE. PROBABLY ALSO INCLUDE THE COMPUTATION OF THE AINFTY EQUATION


%
%\end{remark}
\end{appendices}
%\phantomsection
\bibliographystyle{ieeetr}
\bibliography{newbibliography}
\end{document}
