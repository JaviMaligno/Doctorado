	\documentclass[twoside]{article}
\usepackage{estilo-ejercicios}
\setcounter{section}{0}
\newtheorem{defin}{Definition}[section]
\newtheorem{lem}[defin]{Lemma}
\newtheorem{propo}[defin]{Proposition}
\newtheorem{thm}[defin]{Theorem}
\newtheorem{eje}[defin]{Example}
\renewcommand{\baselinestretch}{1,3}

\usepackage{empheq}
\newcommand*\widefbox[1]{\fbox{\hspace{2em}#1\hspace{2em}}}
%--------------------------------------------------------
\begin{document}

\title{$A_\infty$-structures on operads}
\author{Javier Aguilar Martín}
\maketitle

\section{Introduction}
Continuing with the operadic suspension and the brace structure obtained from it, we then define an $A_\infty$-algebra structure on $\s\OO$.


\section{Defining the $A_\infty$-maps}


Let us consider the underlying graded vector space of the operad $\s\OO$ with grading given by its natural degree, i.e. if $x\in \s\OO(n)$ we definee its natural degree $|x|=n+\deg(x)-1$ where $\deg(x)$ is its internal degree (the degree as an element of $\OO(n))$. 

For any operad $\OO$, recal the operation $\circ$ defined as

\[
a\circ b=\sum_{i=1}^n a\circ_i b
\]
for $a\in\OO(n)$ and $b\in \OO(m)$. We write $a\tilde{\circ}b$ for the corresponding operation on $\s\OO$, namely

\[
a\tilde{\circ} b=\sum_{i=1}^n a\tilde{\circ}_i b.
\]


Let $m\in\s\OO$ be of natural degree 1 such that $m\tilde{\circ}m=0$, so that $m=m_1+m_2+\cdots$ is a formal sum of maps $m_j\in\s\OO(j)$ of internal degree $2-j$ satisfying the usual $A_\infty$-equation for all $n$
\begin{equation}\label{Ainftyeq}
\sum_{r+s+t=n}(-1)^{rs+t}m_{r+1+t}(1^{\otimes r}\otimes m_s\otimes 1^{\otimes t})=0
\end{equation} 


Following \cite{GV} and \cite{getzler}, one would define an $A_\infty$-algebra structure on $\s\OO$ using the maps 

\begin{align*}
M'_1(x)\coloneqq [m,x]=m\tilde{\circ} x-(-1)^{|x|}x\tilde{\circ}m, & &  \\
M'_j(x_1,\dots, x_j)\coloneqq b_j(m;x_1,\dots, x_j),& &j>1.
\end{align*}
The prime notation here is used to indicate that these are not the definitive maps that we are going to take. Getzler shows in \cite{getzler} that $M'=M'_1+M'_2+\cdots$ satisfies the relation $M'\circ M'=0$ using that $m\circ m=0$, but the proof is independent of the operad in which $m$ is defined, so it is still valid if $m\tilde{\circ}m=0$. But we have two problems here. The equation $M'\circ M'=0$ does depend on how the circle operation is defined, more precisely, this circle operation in \cite{getzler} is the one without signs on the endomorphism operad, so $M'$ is not an $A_\infty$-structure under our convention. The other problem has to do with the degrees. We need $M'_j$ to be homogeneous of degree $2-j$ as a map $\s\OO^{\otimes j}\to \s\OO$, but we find that $M'_j$ is homogeneous but of degree 1. 

Indeed, the degree of $b_j(m_n;x_1,\dots,x_j)$ for $n>j$ is computed as follows. Let $a(x)$ denote the arity of $x$, i.e. $a(x)=n$ whenever $x\in\s\OO(n)$. By definition, we have that the degree of $b_j(m_n;x_1,\dots,x_j)$ as an element of $\s\OO$ is

\[a(b_j(m_n;x_1,\dots,x_j))+\deg(b_j(m_n;x_1,\dots,x_j))-1.\]

We have 

\[a(b_j(m_n;x_1,\dots,x_j)=n-j+\sum_i a(x_i)\]

and 

\[\deg(b_j(m_n;x_1,\dots,x_j)=2-n+\sum_i\deg(x_i),\]

so 
\begin{align*}
a(b_j(m_n;x_1,\dots,x_j))+\deg(b_j(m_n;x_1,\dots,x_j))-1=\\
n-j+\sum_i a(x_i)+2-n+\sum_i\deg(x_i)-1=\\
2-j+\sum_i a(x_i)+\sum_i\deg(x_i)-1=\\
1+\sum_i|x_i|.
\end{align*}
This means that the degree of the map $b_j(m_n;-)$ does not depend on $n$ and equals $1$, so $|M'_j|=1$. %A first alternative after finding this result is considering $M'_j$ to be an element of $\s\End_{\s\OO}$ instead of just $\End_{\s\OO}$. This solves the problem of the degree, but not the one of the sign convention. 

Note in the above computation that a way to turn $M'_j$ into a map of degree $2-j$ is introducing a grading on $\s\OO$ given by arity plus internal degree (without substracting one). This is equivalent to defining an $A_\infty$ structure $M$ on $S\s\OO$ shifting the map $M'$, where $S$ is the shift of graded modules. Therefore, we define $M_j$ to be the map making the following diagram commute.

\[
\begin{tikzcd}
(S\s\OO)^{\otimes j}\arrow[r,"M_j"]\arrow[d, "(S^{\otimes j})^{-1}"'] & S\s\OO\\
\s\OO^{\otimes j}\arrow[r, "M'_j"] & \s\OO\arrow[u,"S"']
\end{tikzcd}
\]

In other words, $M_j=\overline{\sigma}(M'_j)$, where $\overline{\sigma}(F)=S\circ F\circ (S^{\otimes n})^{-1}$ for $F\in\End_{\s\OO}(n)$ is the map inducing an isomorphism $\End_{\s\OO}\cong \s\End_{S\s\OO}$. Since $\overline{\sigma}$ is an operad morphism, we have

\[
M\tilde{\circ}M=\overline{\sigma}(M)\tilde{\circ}\overline{\sigma}(M)=\overline{\sigma}(M'\circ M')=0.
\]
%MAYBE DEFINE $\overline{\sigma}_n$ FOR EACH ARITY SO THAT THE ABOVE IS NOT AN ABUSE OF NOTATION. OTHERWISE SAY IT IS AN ABUSE OF NOTATION

So now we have that $M\in\s\End_{S\s\OO}$ is a formal sum $M=M_1+M_2+\cdots$ such that $M_j$ has arity $j$ and internal degree $2-j$; and such that $M\tilde\circ M=0$. Therefore $M$ is the desired $A_\infty$-structure. Nevertheless, recall that $M$ is defined as an structure map on $S\s\OO$. This kind of shifted operad is called \emph{odd operad} in \cite{ward}. This means that $S\s\OO$ is not an operad anymore, since the associativity relation for graded operads involves signs that depend on the degrees, which are now shifted.

\section{Iterating the process}

Now we can apply the same construction to the operad $\s\End_{S\s\OO}$ and we get $A_\infty$-maps 
\[\overline{M}_j:(S\s\End_{S\s\OO})^{\otimes j}\to S\s\End_{S\s\OO}\]
obtained using $\overline{\sigma}$ from maps
\[\overline{M}'_j:(\s\End_{S\s\OO})^{\otimes j}\to \s\End_{S\s\OO}\]
defined as
\[\overline{M}'_j(f_1,\dots,f_j)=\overline{B}_j(M;f_1,\dots, f_j),\]
where $\overline{B}_j$ denotes the brace map on $\s\End_{S\s\OO}$. A natural question to ask is what relation is there between the $A_\infty$-structure on $S\s\OO$ and the one on $S\s\End_{S\s\OO}$ if any. In \cite{GV} it is claimed that given an operad $\OO$ with an $A_\infty$-multiplication, the map

I'M WRITING THIS BRACE WITH BAR BECAUSE WITHOUT BAR BECAUSE I WILL HAVE TO USE $B$ FOR THE BRACE IN THE ENDORMORPHISM OPERAD (NON OPERADIC-SUSPENDED). A POSSIBILITY TO BE CONSISTENT IS USING THE LETTER B FOR NON-SUSPENDED OPERADS AND BAR B FOR SUSPENDED OPERADS, INTRODUCING THE BAR WHEN IT IS THE ENDOMORPHISM OF ANOTHER OPERAD
\begin{align*}
&\OO \to \End_\OO\\
&x\mapsto \sum_{n\geq 0}b_n(x;-)
\end{align*}
is a morphism of $A_\infty$-algebras. We are going to adapt the statement of this claim to our context and prove it. Let $\Phi'$ the map defined as above but on $\s\OO$, i.e.
\begin{align*}
\Phi'\colon&\s\OO \to \End_{\s\OO}\\
&x\mapsto \sum_{n\geq 0}b_n(x;-)
\end{align*}
and let $\Phi:S\s\OO\to S\s\End_{S\s\OO}$ the map making the following diagram commute
\[
\begin{tikzcd}
S\s\OO\arrow[rr, "\Phi"]\arrow[d] & & S\s\End_{S\s\OO}\\
\s\OO\arrow[r, "\Phi'"]& \End_{\s\OO}\arrow[r, "\cong"]& \s\End_{S\s\OO}\arrow[u]
\end{tikzcd}
\]
where the isomorphism $\End_{\s\OO}\cong\s\End_{S\s\OO}$ is given by $\overline{\sigma}$. Note that the degree of the map $\Phi$ is zero.

\begin{thm}
The map $\Phi$ defined above is a morphism of $A_\infty$-algebras, i.e. for all $j\geq 1$ the equation

\[\Phi(M_j)=\overline{M}_j(\Phi^{\otimes j})\]
holds, where the $M_j$ is the structure map on $S\s\OO$ and the $\overline{M}_j$ is the structure map on $S\s\End_{S\s\OO}$.
\end{thm}
\begin{proof}
Let us have a look at the following diagram

\[
\begin{tikzcd}
 & & &(S\s\End_{S\s\OO})^{\otimes j}\arrow[ddr, bend left=15, "\overline{M}_j"]&\\
(S\s\OO)^{\otimes j}\arrow[r,red] \arrow[dd, "M_j"']\arrow[urrr,bend left=10, "\Phi^{\otimes j}"]&\s\OO^{\otimes j}\arrow[r,blue, "(\Phi')^{\otimes j}"]\arrow[d, blue, "M'_j"] & (\End_{\s\OO})^{\otimes j}\arrow[r, blue,"\overline{\sigma}^{\otimes j}"] \arrow[d, dashed, "\mathcal{M}",blue]& (\s\End_{S\s\OO})^{\otimes j}\arrow[u,red]\arrow[d, "\overline{M}'_j",blue]& \\
&\s\OO\arrow[dl,red]\arrow[r, blue, "\Phi'"]& \End_\s\OO \arrow[r, blue, "\overline{\sigma}"] & \s\End_{S\s\OO}\arrow[r,red]& S\s\End_{S\s\OO}\\
S\s\OO\arrow[urrrr, bend right=10, "\Phi"']& &
\end{tikzcd}
\]

where the red arrows are shifts of graded $R$-modules. We need to show that the diagram defined by the external black arrows commute. But these arrows are defined so that they commute whith the blue arrows, so it is enough to show that the inner blue diagram commutes. The blue diagram can be split into two different squares using the dashed arrow $\mathcal{M}$ that we are going define now, so it will be enough to show that the two squares commute. The map 
\[\mathcal{M}:(\End_{\s\OO})^{\otimes j}\to\End_{\s\OO}\]
is defined by 
\begin{align*}
&\mathcal{M}(f_1, \dots, f_j)=B_j(M';f_1,\dots, f_j) &\text{ for }j>1,\\
&\mathcal{M}(f)=B_1(M';f)-(-1)^{|f|}B_1(f;M'),
\end{align*}
 where $B_j$ is a natural brace structure map on the operad $\End_{\s\OO}$, i.e. 
\[B(f;f_1,\dots, f_j)=\sum_{k_0+\cdots+k_j=a(f)-j} f(1^{\otimes k_0}\otimes f_1\otimes 1^{\otimes k_1}\otimes\cdots\otimes f_j\otimes 1^{\otimes k_j}).\]
 In the above definition, $|f|$ denotes the degree of $f$ as an element of $\End_{\s\OO}$. %the degree as a map sO^n\to sO, which is computed by evaluating and computing arity +degree-1
 
 Let us show now that the right square commutes. Recall that $\overline{\sigma}$ is an isomorphism of operads and that $M=\overline{\sigma}(M')$. Then we have 
 
 \[\overline{M}'_j(\overline{\sigma}(f_1),\dots,\overline{\sigma}(f_j))=\overline{B}_j(M;\overline{\sigma}(f_1),\dots,\overline{\sigma}(f_j))=\overline{B}_j(\overline{\sigma}(M');\overline{\sigma}(f_1),\dots,\overline{\sigma}(f_j)).\]
 Now, since the brace structure is defined as an operadic composition we have that it commutes with $\overline{\sigma}$, so
 
 \[\overline{B}_j(\overline{\sigma}(M');\overline{\sigma}(f_1),\dots,\overline{\sigma}(f_j))=\overline{\sigma}(B_j(M';f_1,\dots, f_j))=\overline{\sigma}(\mathcal{M}(f_1,\dots, f_j)),\]
 and therefore the right blue square commutes.
 
 The proof that the left blue square commutes consists of several lenghty calculations so we are going to devote the next section to that. However, it is worth noting that the commutativity of the left square does not depend on the particular operad $\s\OO$, so it is actually valid for any operad $\OO$ with an $A_\infty$-multiplication $m$ such that $m\circ m=0$. This is essentialy the original statement in \cite{GV}.
\end{proof}
%BEING A MORPHISM OF AINFTY FORCES THE DEGREE OF THE MAP TO BE ZERO BECAUSE ONE SIDE HAS DEGREE 2-J+DEG AND THE OTHER SIDE HAS 2-J+J(DEG), SO DEG=0

%$\Phi^j$ SHOULD BE DEFINED USING $(S^{-1})^j$ INSTEAD OF THE INVERSE OF THE TENSOR, BUT THE EXTRA SIGN CANCELS BECAUSE THIS MAP IS USED TWO TIMES. IN ADDITION I NEED THE INVERSE OF THE TENSOR TO DEFINE THE AINFTY MAPS

%PART OF THE PROOF CAN BE SEEN AS SHOWING EXACTLY WHAT THEY SAID (IT DOESN'T DEPEND ON THE PARTICULAR OPERAD AS LONG AS THE BRACE STRUCTURE IS THE NATURAL ONE ON THAT OPERAD), SO TELL IT MAYBE DURING THE PROOF. THE PROOF IS QUITE LONG, IT PROBABLY DESERVES ITS OWN SECTION, OR MAYBE INDICATING THAT THE CALCULATIONS FOR THE FIRST HALF OF DIAGRAM ARE IN ANOTHER SECTION
\subsection{Commutativity of the left blue square}



%\appendix
%\renewcommand{\appendixname}{Appendix:}
\begin{appendices}
\appendix
\gdef\thesection{Appendix \Alph{section}}
\section{Explicit $A_\infty$-maps}
DERIVATION OF THE SIGN THAT COMES IN FRON THE MAPS $M_J$ AND TELL THAT THIS COINCIDES WITH G-V IN THE ASSOCIATIVE CASE. PROBABLY ALSO INCLUDE THE COMPUTATION OF THE AINFTY EQUATION


%
%\end{remark}
\end{appendices}
%\phantomsection
\bibliographystyle{ieeetr}
\bibliography{newbibliography}
\end{document}
